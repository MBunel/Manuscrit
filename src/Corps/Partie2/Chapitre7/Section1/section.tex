La \emph{spatialisation} est la phase de notre méthodologie (\autoref{chap:04}) où les \emph{indices de localisation,} décomposés de manière à ne contenir qu'une \emph{relation de localisation atomique} et qu'un \emph{objet de référence nommé} (\autoref{chap:05}), sont interprétés pour construire une \emph{zone de localisation compatible} ---~délimitant la région où \emph{l'indice de localisation} est vrai~---, représentée par un raster flou (\autoref{chap:06}).

\subsection{Considérations générales}

Le recours à une implémentation raster (\autoref{chap:06}) a deux conséquences importantes. 

\tdi{Importance de la zir}
Ensuite il est nécessaire de définir la \ac{zir}

\tdi{Importance du raster de base et de la maille}
Tout d'abord il est nécessaire de définir une résolution de travail, qui influencera sur la précision et la durée du calcul.

Tous les rasters partageant les mêmes \emph{zones initiales de recherche} et résolutions sont parfaitement superposables.

\tdi{Pas de reseampling}%
Il serait possible de travailler à partir de rasters d'étendue ou de résolution variable

\subsection{Les parties du processus de \emph{spatialisation}}

\tdi{Présenter les 3 grandes étapes de la démarche, leur rôle et leur justification}

La méthode de spatialisation que nous proposons est donc une composition de trois grandes étapes (\autoref{fig:methodo_spatialisation}).

\tdi{méthode de spatialisation = rasterisation + calcul métrique +
  sélection floue}

\begin{figure}
  \centering
  \begin{tikzpicture}
  \begin{scope}
    \path[ffa] (0,0) rectangle (2,2);
    \path[ffc] (0,0) rectangle (2,2);
    \node[text width=3cm, align=center, anchor=center] at (1,-.75)
    {\footnotesize \itshape Zone initiale de recherche};
  \end{scope}

  \begin{scope}[yshift=2.5cm]
    \fill[ffa] (1.25,.7) circle [radius=7pt];
    \path[ffc] (1.25,.7) circle [radius=7pt];
    \node[text width=3cm, align=center, anchor=center] at (1,.15)
    {\footnotesize \itshape Objet de référence};
  \end{scope}

  
  \path[draw, ->] (2.5,1) --++ (1,0)  node[pos=.5, above]
  {\footnotesize \itshape rasterisation};
  
  \begin{scope}[xshift=4.25cm]
    \begin{scope}
      \foreach \x in {0,.25,...,1.75}{
        \foreach \y in {0,.25,...,1.75}
        \path[draw, line width=.01mm] (\x,\y) rectangle (\x +.25, \y + .25);
      }
      % Pixels
      \path[ffa] (1,.25) rectangle (1.25, .5);
      \path[ffa] (1.25,.25) rectangle (1.5, .5);
      \path[ffa] (.75,.5) rectangle (1, .75);
      \path[ffa] (1,.5) rectangle (1.25, .75);
      \path[ffa] (1.25,.5) rectangle (1.5, .75);
      \path[ffa] (1.5,.5) rectangle (1.75, .75);
      \path[ffa] (.75,.75) rectangle (1, 1);
      \path[ffa] (1,.75) rectangle (1.25, 1);
      \path[ffa] (1.25,.75) rectangle (1.5,1);
      \path[ffa] (1.5,.75) rectangle (1.75,1);
      \path[ffc] (0,0) rectangle (2,2);
    \end{scope}
     \node[text width=3cm, align=center, anchor=center] at (1,-.75)
     {\footnotesize \itshape Zone de localisation compatible};
  \end{scope}
  
  % \path[draw, ->] (8.5,1) --++ (1,0)  node[pos=.5, above]
  % {\footnotesize \itshape spatialisation};
  
  \begin{scope}[xshift=8.5cm]
    \begin{scope}
      \foreach \x in {0,.25,...,1.75}{
        \foreach \y in {0,.25,...,1.75}
        \path[draw, line width=.01mm] (\x,\y) rectangle (\x +.25, \y +
        .25);
        \node[circle, inner sep=0pt,minimum size=4pt, fill] (c) at (\x+.125,\y+.125) {};
      }
      \path[ffc] (0,0) rectangle (2,2);
    \end{scope}
  \end{scope}
  
  \begin{scope}[xshift=12.75cm]
    \begin{scope}
      \foreach \x in {0,.25,...,1.75}{
        \foreach \y in {0,.25,...,1.75}
        \path[draw, line width=.01mm] (\x,\y) rectangle (\x +.25, \y + .25);
      }
      \path[ffc] (0,0) rectangle (2,2);
    \end{scope}
  \end{scope}
\end{tikzpicture}
  \caption{Méthode de \emph{spatialisation}}
  \label{fig:methodo_spatialisation}
\end{figure}

\tdi{rasterisation, pourquoi + justifications}%
Dans un premier temps il est nécessaire de transformer \emph{l'objet de référence} en raster, de manière ce qu'il puisse être traité. Cette partie est nommée \emph{rasterisation} est, comme nous le verrons dans la partie qui lui est dédiée (\autoref{chap:07-sec2}), réalisable de plusieurs façons.

\tdi{métrique}%
La seconde partie de la \emph{spatialisation} est le calcul de la métrique.

La métrique est une grandeur, généralement quantitative, recoupant la sémantique de la relation de \emph{localisation atomique modélisée.}

Par conséquent des \emph{relations de localisation atomiques} décrivant des situations très différentes peuvent faire appel à la même métrique.
%
Par exemple, \onto[orla]{Distance\-Quantitative} et \onto[orla]{Exterieur} sont des concepts sémantiquement assez éloignés, décrivant respectivement une relation métrique et une relation topologique. Pourtant, ces deux concepts traduisent une forme d'éloignement par rapport à \emph{l'objet de référence,} si ce n'est qu'il est d'une valeur donnée (\ie la distance) pour \onto[orla]{Distance\-Quantitative} alors qu'il est qualitatif pour \onto[orla]{Exterieur}. Les \ac{zlc} de ces deux \emph{relations de localisation atomiques} sont modélisables à partir d'une même \emph{métrique,} la distance, mais différent par la manière dont cette métrique est interprétée et donc transformée en un degré d'appartenance.

% Fuzzyficaiton
Cette transformation est effectuée lors de la \emph{fuzzyfication,} dernière partie de la \emph{spatialisation.} 


%%% Local Variables:
%%% mode: latex
%%% TeX-master: "../../../../main"
%%% End:
