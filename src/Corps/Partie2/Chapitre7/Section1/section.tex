La \emph{spatialisation} est la phase de notre méthodologie
(\autoref{chap:04}) où les \emph{indices de localisation,} décomposés
de manière à ne contenir qu'une \emph{relation de localisation
  atomique} et qu'un \emph{objet de référence nommé}
(\autoref{chap:05}), sont interprétés pour construire une \emph{zone
  de localisation compatible} ---~délimitant la région où
\emph{l'indice de localisation} est vrai~---, représentée par un
raster flou (\autoref{chap:06}).

Comme nous l'expliquions dans le \autoref{chap:04} et conformément au
\emph{principe de modélisation autonome,} la \emph{spatialisation} est
une étape effectuée indépendamment et parallèlement pour tous les
\emph{indices de localisation} résultant de la \emph{phase de
  décomposition.} Le processus que nous détaillons ici est donc conçu
pour traiter des \emph{indices de localisation} décomposés,
c'est-à-dire qui ne sont composés que d'une \emph{relation de
  localisation atomique} et d'un \emph{objet de référence nommé.}

\subsection{Contraintes à la définition d'une \emph{méthode de spatialisation}}

Le recours à une implémentation raster (\autoref{chap:06}) impacte
fortement le processus de construction de la \ac{zlc}

\tdi{Importance de la zir}%
Ensuite il est nécessaire de définir la \ac{zir}

\tdi{Importance du raster de base et de la maille}%
Tout d'abord il est nécessaire de définir une résolution de travail,
qui influencera sur la précision et la durée du calcul.

Tous les rasters partageant les mêmes \emph{zones initiales de
  recherche} et résolutions sont parfaitement superposables.

\tdi{Pas de reseampling}%
Il serait possible de travailler à partir de rasters d'étendue ou de
résolution variable


\tdi{contraintes techniques (raster) ?? }
\tdi{rasterisation obj ref}


\subsection{Définition de la méthode de \emph{spatialisation}}

Pour définir une \emph{méthode de spatialisation,} le mieux est encore
de s’interroger sur la nature des résultats que nous souhaitons
obtenir, les \emph{zones de localisations compatibles.} Ces dernières
doivent délimiter la région correctement décrite par \emph{l'indice de
  localisation} spatialisé, c'est-à-dire qu'en toute position de la
\ac{zlc}, l'assertion : \enquote{\emph{l'indice de localisation} est
  vrai} est correcte. L'utilisation de la \emph{théorie des
  sous-ensembles flous} (\autoref{chap:06}) permet une modélisation
plus fine, considérant des nuances de vérité et permettant de
distinguer les positions où l'assertion précédente est parfaitement
correcte (\ie où le \emph{degré d'appartenance} est de 1), de celles
où elle l'est partiellement (\ie où le \emph{degré d'appartenance} est
entre 0 et 1). Le choix d'une implémentation raster
(\autoref{chap:06}) facilite, quant à lui, la définition de zone de la
\ac{zlc}, en résumant sa construction au calcul d'un \emph{degré
  d'appartenance} pour chaque pixel d'un raster échantillonnant la
\ac{zir}. Ces deux éléments ne définissent cependant que la manière de
représenter la \emph{zone de localisation compatible} et non la façon
de la construire, cette tâche incombant à la \emph{méthode de
  spatialisation.}

Définir une \emph{méthode de spatialisation} revient donc à identifier
une procédure permettant de vérifier, pour chaque pixel d'un raster
donné et à l'aide de la \emph{théorie des sous-ensembles flous,} la
validité de l'assertion \enquote{\emph{l'indice de localisation} est
  vrai}. Or, le \emph{principe de décomposition} nous a amené à
définir des \emph{relations de localisation atomiques,} les plus
différentes possibles et nécessitant par conséquent, des
\emph{méthodes de spatialisation} qui le sont tout autant. Il semble
donc nécessaire de développer une \emph{méthode de spatialisation}
pour chaque \emph{relation de localisation atomique} définie dans
\ac{orla}. C'est par exemple ce que proposent \textcite{Vanegas2011}
ou \textcite{Takemura2012} en proposant des \emph{méthodes de
  spatialisation} dédiées à une \emph{relation de localisation
  donnée.} Nous proposons cependant d'utiliser, à nouveau, une
démarche de décomposition.

Prenons, par exemple deux \emph{relations de localisation atomiques,}
\onto[orla]{In\-te\-ri\-eur} et \onto[orla]{Ex\-te\-ri\-eur}
\footnote{Ces deux \emph{relations de localisation atomiques} sont
  utilisées pour décomposer des \emph{relations} comme
  \protect\onto[orl]{Dans\-La\-Partie\-Sud\-De} ou
  \protect\onto[orl]{Au\-Nord\-De\-Externe}, mais également
  directement dans les \emph{relations}
  \protect\onto[orl]{Dans\-Planimétrique} et
  \protect\onto[orl]{Hors\-De\-Planimétrique}.}. Ces deux concepts
décrivent une \emph{relation} entre le \emph{sujet} et \emph{l'objet
  de référence} opposée, mais de même nature (\ie topologique). Aussi
on peut considérer que ce qui distingue ces deux \emph{relations de
  localisation atomiques} n'est pas la nature de la configuration
spatiale qu'elles décrivent, mais la manière dont elles évaluent la
validité de l'assertion : \enquote{\emph{l'indice de localisation} est
  vrai}. Pour illustrer cette affirmation prenons l'exemple de la
\autoref{fig:Exemple_Metrique_vs_Selecteur}. Un \emph{objet de
  référence} (en rouge) est situé au centre d'un raster de 4 pixels,
représentant une \ac{zir} simplifiée. Imaginons que l'on cherche à
construire, sans prendre en compte \emph{l'imprécision,} les \ac{zlc}
correspondant aux \emph{relations} \onto[orla]{Interieur} et
\onto[orla]{Exterieur}, pour cet \emph{objet de référence.} L'approche
raster, nous permet de construire ces deux \ac{zlc} en sélectionnant
les pixels situés, respectivement, à l'intérieur et à l'extérieur de
\emph{l'objet de référence.} Pour cela, il est tout d'abord nécessaire
d'identifier la relation topologique entre chaque pixel et
\emph{l'objet de référence.} Étant donné que \emph{l'imprécision}
n'est pas considérée ici, nous pouvons décrire ces relations
topologiques à l'aide d'une matrice 4IM \footnote{Nous aurions pu
  utiliser n'importe quelle autre formalisation des relations
  topologiques présentée dans le \autoref{chap:03} (RCC-8, 9IM,
  DE-9IM, \emph{etc.}). Le modèle 4IM nous a simplement semblé le plus
  clair pour illustrer cet exemple.}  calculée en chaque pixel de la
\ac{zir} \footnote{Le pixel traité fait alors office d'objet \(a\) et
  \emph{l'objet de référence} d'objet \(b\) (\autoref{eq:matrice_4IM},
  \autoref{chap:03}).}
(\autoref{fig:Exemple_Metrique_vs_Selecteur}). On obtient alors un
raster contenant (explicitement) une information permettant de
distinguer les pixels situés à l'intérieur et à l'extérieur de
\emph{l'objet de référence.} On peut alors construire :
% 
\begin{itemize}
\item La \ac{zlc} \textcolor{RdBu-9-1}{\textsf{A}}, correspondant à la
  \emph{relation de localisation atomique}
  \onto[orla]{In\-te\-ri\-eur}, en sélectionnant l'ensemble des pixels
  pour lesquels la matrice 4IM est :
  \(\left[\begin{smallmatrix}V&F\\F&F\\\end{smallmatrix}\right]\).
\item La \ac{zlc} \textcolor{RdBu-9-9}{\textsf{B}}, correspondant à la
  \emph{relation} \onto[orla]{Ex\-te\-ri\-eur}, en sélectionnant les
  pixels dont la matrice 4IM est :
  \(\left[\begin{smallmatrix}F&F\\F&V\\\end{smallmatrix}\right]\).
\end{itemize}
%
Ces deux \ac{zlc}, bien qu'ayant une sémantique et une emprise
spatiale inconfusibles, sont \emph{spatialisées} avec une méthode
similaire, dont seule la dernière étape ---~la sélection des pixels
satisfaisant la \emph{relation de localisation atomique}~--- différè.
Ainsi, ce qui distingue ces deux \emph{relations de localisation
  atomiques} n'est pas le critère utilisé pour les modéliser (\ie les
matrices 4IM), mais la manière dont les valeurs obtenues sont filtrées
pour construire les \ac{zlc}. On peut alors identifier deux parties
distinctes dans cette \emph{méthode de spatialisation,} le calcul du
critère de modélisation de la \emph{relation de localisation atomique}
---~on parlera dorénavant de \emph{métrique}~--- et la
\emph{sélection} des pixels dont la valeur de la \emph{métrique}
correspond à la sémantique de la \emph{relation de localisation
  atomique} spatialisée.

\begin{figure}
  \centering
  \begin{tikzpicture}
  % Arrow
  \begin{scope}
    \path[draw, ->, shorten >=5pt, shorten <=5pt] (2,1) --++ (4,0) node[pos=.5, font=\footnotesize, fill=white]
    {\itshape rasterisation};
    
    \path[draw, ->, shorten >=5pt, shorten <=5pt] (7,2) |- (12,3)
    node[pos=.75, font=\footnotesize, fill=white] (labsel1)
    {\itshape sélection};
    % 
    \node[anchor=north west,baseline, text width=2.5cm, font=\tiny] at
    ([xshift=1.5ex, yshift=1ex]labsel1.south west) {des pixels dont la
    matrice 4IM est : \(\left[\begin{smallmatrix}V&V\\V&F\\\end{smallmatrix}\right]\)};

    \path[draw, ->, shorten >=5pt, shorten <=5pt] (7,0) |- (12,-1)
    node[pos=.75, font=\footnotesize, fill=white] (labsel2)
    {\itshape sélection};
    % 
    \node[anchor=north west,baseline, text width=2.5cm, font=\tiny] at
    ([xshift=1.5ex, yshift=1ex]labsel2.south west) {des pixels dont la
    matrice 4IM est : \(\left[\begin{smallmatrix}F&F\\F&V\\\end{smallmatrix}\right]\)};
  \end{scope}
  % Représentation objet de référence rasterisé
  \begin{scope}
    \begin{scope}
      \foreach \x in {0,1}{
        \foreach \y in {0,1}
        \path[draw, line width=.01mm] (\x,\y) rectangle (\x +1, \y + 1);
      }
      % Limite zir
      \path[ffc, black] (0,0) rectangle (2,2);
      % Pixels rasterisation
      \path[ffa] (1,1) rectangle (2,2);
    \end{scope}
    \node[text width=2.5cm, align=center, anchor=north,
    font=\footnotesize, fill=white] at (1,-.25)
    {\itshape Objet de référence rastérisé};
  \end{scope}
  
  % Représentation métrique
  \begin{scope}[xshift=6cm]
    % \path[ffa] (1,.25) rectangle (1.25, .5);
    \begin{scope}
      \foreach \x in {0,1}{
        \foreach \y in {0,1}
        \path[draw, line width=.01mm] (\x,\y) rectangle (\x +1, \y + 1);
      }
      % 
      \node[font=\small, align=center,anchor=center] at (.5, .5)
      {\(\left[\begin{smallmatrix}F&F\\F&V\\\end{smallmatrix}\right]\)};
      \node[font=\small, align=center,anchor=center] at (.5, 1.5)
      {\(\left[\begin{smallmatrix}F&F\\F&V\\\end{smallmatrix}\right]\)};
      \node[font=\small, align=center,anchor=center] at (1.5, .5)
      {\(\left[\begin{smallmatrix}F&F\\F&V\\\end{smallmatrix}\right]\)};
      \node[font=\small, align=center,anchor=center] at (1.5, 1.5)
      {\(\left[\begin{smallmatrix}V&V\\V&F\\\end{smallmatrix}\right]\)};
    \end{scope}
    \path[ffc, black] (0,0) rectangle (2,2);
    \node[text width=4.5cm, align=center, anchor=north,
    font=\footnotesize, fill=white] at (1,-.25)
    {\itshape Métrique};
  \end{scope}

    \begin{scope}[xshift=12cm, yshift=2cm]
    \begin{scope}
      \foreach \x in {0,1}{
        \foreach \y in {0,1}
        \path[draw, line width=.01mm] (\x,\y) rectangle (\x +1, \y + 1);
      }
      \path[ffa] (1,1) rectangle (2,2);
      \path[ffc] (1,1) rectangle (2,2);
    \end{scope}
    \node[text width=2.5cm, align=center, anchor=north, font=\footnotesize] at (1,-.25)
    {\itshape Zone de localisation compatible \normalfont \textcolor{RdBu-9-1}{\textsf{A}}};
  \end{scope}
  
  \begin{scope}[xshift=12cm, yshift=-2cm]
    \begin{scope}
      \foreach \x in {0,1}{
        \foreach \y in {0,1}
        \path[draw, line width=.01mm] (\x,\y) rectangle (\x +1, \y + 1);
      }

      \begin{scope}
        \fill[ffa2] (0,0) rectangle (2,1);
        \fill[ffa2] (0,1) rectangle (1,2);
      \end{scope}
      
      \path[ffc2] (0,0) --++(0,2) --++(1,0) --++(0,-1) --++ (1,0)
      --++(0,-1) --cycle;
    \end{scope}
    \node[text width=2.5cm, align=center, anchor=north, font=\footnotesize] at (1,-.25)
    {\itshape Zone de localisation compatible \normalfont \textcolor{RdBu-9-9}{\textsf{B}}};
  \end{scope}
\end{tikzpicture}
  \caption{Illustration du processus de construction d'une \ac{zlc}
    nette.}
  \label{fig:Exemple_Metrique_vs_Selecteur}
\end{figure}

% Importance de la rasterisation
Cependant ces deux parties ne sont pas suffisantes pour distinguer
toutes les \emph{relations de localisation atomiques.} Par exemple,
les descriptions de position : \enquote{Je suis à 300 mètres de ma
  maison} et \enquote{Je suis à la frontière du bois de Vincennes},
décrivent des situations en apparence très différentes, elles sont
d'ailleurs modélisables avec des \emph{relations de localisation
  atomiques} distinctes, \onto[orla]{Distance\-Quantitive} et
\onto[orla]{A\-La\-Frontiere\-De}. Pourtant si l'on se penche sur leur
signification on peut s’apercevoir qu'elles partagent leur
\emph{métrique,} mais également son interprétation. En effet, ces deux
\emph{relations de localisation atomiques} décrivent un éloignement
spatial entre le \emph{sujet} et \emph{l'objet de référence,} si ce
n'est qu'il est explicite et quantifié dans un cas et sous-entendu
dans l'autre. Elles peuvent donc être modélisée avec la même métrique,
la distance à \emph{l'objet de référence.} On pourrait penser que ces
\emph{relations} se distinguent par leur interprétation de cette même
\emph{métrique,} comme c'est le cas pour les \emph{relations de
  localisation atomiques} \onto[orla]{Exterieur} et
\onto[orla]{Intérieur}, ce n'est pourtant pas le cas. La
\emph{relation} \onto[orla]{Distance\-Quantitive} indique
explicitement que la distance entre le \emph{sujet} et \emph{l'objet
  de référence} est proche d'une valeur donnée (ici \SI{300}{\meter}),
mais c'est également le cas de la \emph{relation}
\onto[orla]{A\-La\-Frontiere\-De}, si ce n'est que cette distance est
implicitement nulle. Bien entendu la notion de \enquote{proximité},
correspondant à \emph{l'imprécision} de la zone modélisée peut être
différente entre ces deux \emph{relations,} mais cela ne change pas
notre propos, \onto[orla]{Distance\-Quantitive} et
\onto[orla]{A\-La\-Frontiere\-De} décrivent toutes deux une situation
où la distance à \emph{l'objet de référence} (\ie la métrique) est
proche d'une valeur donnée. Ces deux \emph{relations} ne sont
différentiables qu'en ajoutant un nouveau paramètre de
\emph{spatialisation,} le point d'origine de la \emph{métrique.} En
effet, si ces deux \emph{relations} décrivent un éloignement par
rapport à \emph{l'objet de référence} elles ne se basent pas sur la
même partie de \emph{l'objet de référence,} là où
\onto[orla]{Distance\-Quantitive} décrit un éloignement par rapport à
l'objet dans son ensemble, \onto[orla]{A\-La\-Frontiere\-De} décrit,
explicitement, un éloignement par rapport à sa frontière, ce qui est
une distinction sémantique forte et qui modifie la manière dont les
positions situées au sein de \emph{l'objet de référence} sont
interprétées. En effet, si dans la première situation les positions
situées au sein de \emph{l'objet de référence} ont une distance à
\emph{l'objet de référence} nulle, dans le second cas cette distance
est strictement positive, car calculée à partir de la frontière de
\emph{l'objet de référence.} Ainsi, il également nécessaire de prendre
en compte le point ou la partie de \emph{l'objet de référence,}
faisant réellement office de référence et donc de point de calcul de
la \emph{métrique.}

%\subsubsection{Les étapes de la \emph{méthode de spatialisation}}

Trois composantes sont donc à prendre en compte pour
\emph{spatialiser} une \emph{relation de localisation atomique,} une
\emph{métrique,} la partie de \emph{l'objet de référence} servant
d'origine à son calcul et une \emph{méthode de sélection} permettant
d'identifier les valeurs de la \emph{métrique} correspondant à la
sémantique de la relation de \emph{localisation atomique spatialisée.}
À ces trois composantes s'ajoutent trois contraintes à respecter, la
nécessité de limiter la \emph{spatialisation} à la \ac{zir}, de
rasteriser les \emph{objets de référence} et de construire une
\ac{zlc} floue (contrairement aux résultats de la
\autoref{fig:Exemple_Metrique_vs_Selecteur}). Nous proposons de
définir une \emph{méthode de spatialisation} en trois parties, la
\emph{rasterisation,} le calcul de la \emph{métrique} et la
\emph{fuzzyfication} (\autoref{fig:methodo_spatialisation}).

\begin{figure}
  \centering
  \begin{tikzpicture}
  \begin{scope}
    \path[ffa] (0,0) rectangle (2,2);
    \path[ffc] (0,0) rectangle (2,2);
    \node[text width=3cm, align=center, anchor=center] at (1,-.75)
    {\footnotesize \itshape Zone initiale de recherche};
  \end{scope}

  \begin{scope}[yshift=2.5cm]
    \fill[ffa] (1.25,.7) circle [radius=7pt];
    \path[ffc] (1.25,.7) circle [radius=7pt];
    \node[text width=3cm, align=center, anchor=center] at (1,.15)
    {\footnotesize \itshape Objet de référence};
  \end{scope}

  
  \path[draw, ->] (2.5,1) --++ (1,0)  node[pos=.5, above]
  {\footnotesize \itshape rasterisation};
  
  \begin{scope}[xshift=4.25cm]
    \begin{scope}
      \foreach \x in {0,.25,...,1.75}{
        \foreach \y in {0,.25,...,1.75}
        \path[draw, line width=.01mm] (\x,\y) rectangle (\x +.25, \y + .25);
      }
      % Pixels
      \path[ffa] (1,.25) rectangle (1.25, .5);
      \path[ffa] (1.25,.25) rectangle (1.5, .5);
      \path[ffa] (.75,.5) rectangle (1, .75);
      \path[ffa] (1,.5) rectangle (1.25, .75);
      \path[ffa] (1.25,.5) rectangle (1.5, .75);
      \path[ffa] (1.5,.5) rectangle (1.75, .75);
      \path[ffa] (.75,.75) rectangle (1, 1);
      \path[ffa] (1,.75) rectangle (1.25, 1);
      \path[ffa] (1.25,.75) rectangle (1.5,1);
      \path[ffa] (1.5,.75) rectangle (1.75,1);
      \path[ffc] (0,0) rectangle (2,2);
    \end{scope}
     \node[text width=3cm, align=center, anchor=center] at (1,-.75)
     {\footnotesize \itshape Zone de localisation compatible};
  \end{scope}
  
  % \path[draw, ->] (8.5,1) --++ (1,0)  node[pos=.5, above]
  % {\footnotesize \itshape spatialisation};
  
  \begin{scope}[xshift=8.5cm]
    \begin{scope}
      \foreach \x in {0,.25,...,1.75}{
        \foreach \y in {0,.25,...,1.75}
        \path[draw, line width=.01mm] (\x,\y) rectangle (\x +.25, \y +
        .25);
        \node[circle, inner sep=0pt,minimum size=4pt, fill] (c) at (\x+.125,\y+.125) {};
      }
      \path[ffc] (0,0) rectangle (2,2);
    \end{scope}
  \end{scope}
  
  \begin{scope}[xshift=12.75cm]
    \begin{scope}
      \foreach \x in {0,.25,...,1.75}{
        \foreach \y in {0,.25,...,1.75}
        \path[draw, line width=.01mm] (\x,\y) rectangle (\x +.25, \y + .25);
      }
      \path[ffc] (0,0) rectangle (2,2);
    \end{scope}
  \end{scope}
\end{tikzpicture}
  \caption{Méthode de \emph{spatialisation} d'une \emph{relation de
      localisation atomique} donnée.}
  \label{fig:methodo_spatialisation}
\end{figure}

La \emph{rasterisation} est la première partie de la
\emph{spatialisation.} Elle consiste à transformer les \emph{objets de
  référence} vectoriels en des rasters à partir desquels nous pourrons
calculer la \emph{métrique} nécessaire à la \emph{spatialisation.} La
rasterisation de \emph{l'objet de référence} nécessite de définir une
résolution et une taille de raster, qui dans notre cas correspond à la
\ac{zir}. Ainsi, la zone de \emph{rasterisation} correspond à la
\emph{zone initiale de recherche} et cette étape fait donc également
office de délimitation de la zone de calcul. De plus il est nécessaire
de ne sélectionner que la partie de \emph{l'objet de référence} qui
est utile pour la \emph{relation de localisation atomique} à
spatialiser (\eg frontière, centroide, \emph{etc.}). Il est plus
simple de réaliser cette opération dans un premier temps et de ne
rasteriser ensuite que la partie sélectionnée. Ainsi l'étape de
\emph{rasterisation} rempli trois rôles :
%
\begin{enumerate*}[label=(\alph*)]
\item la sélection de la partie de \emph{l'objet de référence} à rasteriser,
\item la prise en compte de la \ac{zir}, et
\item la rasterisation de \emph{l'objet de référence.}  
\end{enumerate*}
%
Comme nous le verrons dans la partie qui lui est dédiée
(\autoref{chap:07-sec2}), l'opération de rasterisation peut donner des
résultats très différents en fonction des paramètres sélectionnée. La
seconde partie de la \emph{spatialisation} est le calcul de la
\emph{métrique.} Ce calcul est effectué sur l'aire et à la résolution
fixée lors de l'opération de \emph{rasterisation.} Les
\emph{métriques} utilisables sont nombreuses et fortement liées à la
sémantique de la \emph{relation de localisation atomique} spatialisée,
comme nous le verrons dans la partie dédiée à cette question
(\autoref{chap:07-sec3}). La dernière partie de la \emph{méthodologie
  de spatialisation} est la \emph{fuzzyfication.} C'est durant cette
étape que les pixels sont filtrés en fonction de la valeur de la
métrique, à délimiter la \emph{zone de localisation compatible}
correspondant à la \emph{relation de localisation atomique}
spatialisée. Contrairement à l'exemple présenté
(\autoref{fig:Exemple_Metrique_vs_Selecteur}), cette opération doit
aboutir à une \ac{zlc} floue, ce que nous détaillerons dans la
\autoref{chap:07-sec4}.


%%% Local Variables:
%%% mode: latex
%%% TeX-master: "../../../../main"
%%% End:
