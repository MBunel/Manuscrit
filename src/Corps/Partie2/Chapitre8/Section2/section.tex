\tdi{voi si c,'est pas mieux de le mettre en section}


La théorie des \emph{sous-ensembles flous} ne permet pas, du moins
dans sa formulation initiale, de modéliser l'incertitude d'une
connaissance, seulement son \emph{imprécision.}

\textcite{Zadeh1978}

\begin{figure}
  \centering
  %\begin{tikzpicture}
  \begin{scope}[xshift=-2cm]
    \begin{scope}
      \foreach \x in {0,.25,...,1.75}{ \foreach \y in {0,.25,...,1.75}
        {
          % Calcul et représentation floue distance euclidienne
          \pgfmathsetmacro\dist{sqrt((\x-0.125)^2+(\y-1.375)^2)}
          % Fuzzyfication de la distance
          \pgfmathsetmacro\fuzzy{%
            ifthenelse(\dist < .25,1,%
            ifthenelse(\dist < 2,-0.57*\dist+1.14,0)%
            )
          }
          % Calcul du rayon à partir de la fuzzyfication
          \pgfmathsetmacro\radius{max(\fuzzy, 0.6)*2.5}
          \fill[black] (\x+.125,.125+\y) circle (\radius pt);
        }
      }
      \path[ffc, black] (0,0) rectangle (2,2);
    \end{scope}
  \end{scope} 
  
  \begin{scope}[xshift=1.2cm]
    \begin{scope}
      \foreach \x in {0,.25,...,1.75}{ \foreach \y in {0,.25,...,1.75}
        {
          % Calcul et représentation floue distance euclidienne
          \pgfmathsetmacro\dist{sqrt((\x-0.125)^2+(\y-1.375)^2)}
          % Fuzzyfication de la distance
          \pgfmathsetmacro\fuzzy{%
            ifthenelse(\dist < .25,1,%
            ifthenelse(\dist < 2,-0.57*\dist+1.14,0)%
            )
          }
          % Calcul du rayon à partir de la fuzzyfication
          \pgfmathsetmacro\radius{max(\fuzzy, 0.2)*2.5}
          \fill[RdBu-9-1] (\x+.125,.125+\y) circle (\radius pt);
        }
      }
      \path[ffc] (0,0) rectangle (2,2);
    \end{scope}
  \end{scope}

    \begin{scope}[xshift=6.8cm]
    \begin{scope}
      \foreach \x in {0,.25,...,1.75}{ \foreach \y in {0,.25,...,1.75}
        {
          % Calcul et représentation floue distance euclidienne
          \pgfmathsetmacro\dist{sqrt((\x-0.125)^2+(\y-1.375)^2)}
          % Fuzzyfication de la distance
          \pgfmathsetmacro\fuzzy{%
            ifthenelse(\dist < .25,1,%
            ifthenelse(\dist < 2,-0.57*\dist+1.14,0)%
            )
          }
          % Calcul du rayon à partir de la fuzzyfication
          \pgfmathsetmacro\radius{max(\fuzzy, 0.6)*2.5}
          \fill[RdBu-9-9] (\x+.125,.125+\y) circle (\radius pt);
        }
      }
      \path[ffc2] (0,0) rectangle (2,2);
    \end{scope}
  \end{scope}

  \node[scale=5] at (-3.2,1) {\(\top(\)};
  \node[scale=5] at (.45,0) {\(,\)};
  \node[scale=5] at (4.7,1) {\() = \)};
  
\end{tikzpicture}
  \caption{Influence de l'incertitude d'un \emph{indice de
      localisation} sur la \emph{zone de localisation} résultant de la
    fusion de deux \ac{zlc}.}
  \label{fig:intersection}
\end{figure}


Une fonction d'appartenance normalisée est donc une distribution de
possibilité.

La confiance peut se définir en amont de la spatialisation.

Diminuer la confiance c'est augmenter le degré d'appartenance minimal

Proposer une version améliorée de la fig finale du chapitre 4 avec la
confiance.

Dire que les fonctions présentées reviennent a considérer que la
confiance est absolue

Parler des seuils de la confiance (3) et des valeurs associées ou
contraintes.


\begin{figure}
  \centering  \subfloat[\label{fig:fnc_app_inc}]{\begin{tikzpicture}[scale=.7]
  \def\decalageX{-.2}
  \def\decalageY{-.2}
  % Courbe
  \begin{scope}[transparency group]
    % fond
    \begin{scope}
      \path[ffa]  (1,.4) -- (3.3,.4) -- (4.5, 2) -- (5.7,.4) -- (8,.4)
      -- (8,0) -- (1,0) -- cycle;
      \path[ffa_fade_m] (0,.4) -- (1,.4) -- (1,0) -- (0,0) -- cycle ;
      \path[ffa_fade] (8,.4) -- (9,.4) -- (9,0) -- (8,0) -- cycle ;
    \end{scope}
    % bords
    \begin{scope}
      \path[ffc] (1,.4) -- (3.3,.4) -- (4.5, 2) -- (5.7,.4) -- (8,.4);
      \path[ffc, dotted] (3.3,.4) -- (3,0);
      \path[ffc, dotted] (5.7,.4) -- (6,0);
      \path[ffc_fade_m] (0,.4) -- (1,.4) ;
      \path[ffc_fade] (8,.4) -- (9,.4) ;
    \end{scope}
  \end{scope}
  % Axes X, Y
  \begin{scope}
    % Axe X
    \begin{scope}
      % Axe
      \draw[<->] (0, \decalageX) --++ (9, 0) coordinate (x axis);
      % Graduations
      \foreach \n/\t in {1/{},2/{},3/{},4/{},5/{},6/{},7/{},8/{}}
      {
        \draw[-] (\n, \decalageX - .05) --++ (0, .1);
        \node[below, font=\footnotesize] at (\n, \decalageX - .05) {\t};
      }
      % label
      \node[below left, yshift=-.1cm, font=\small] at (x axis) {\itshape Métrique};
    \end{scope}
    % Axe Y
    \begin{scope}
      % Axe
      \draw[-] (\decalageY ,0) --++ (0, 2) coordinate (y axis);
      % Graduations
      \foreach \n/\t in {0/{0},2/{1}}
      {
        \draw[-] (\decalageY -.05, \n) --++ (.1, 0);
        \node[left, font=\footnotesize] at (\decalageY -.05, \n) {\t};
      }
      % Label
      \node[above] at (y axis) {$\mu$};
    \end{scope}
  \end{scope}
  \begin{scope}
    % Seuil 1
    \draw[ffc,line width=.5] (3.3,\decalageY) -- (3.3,.4);
    \draw[fill, RdBu-9-1] (3.3,\decalageY) circle (2pt);
    \draw[fill, RdBu-9-1] (3.3,.4) circle (2pt);
    % Seuil 2
    \draw[ffc,line width=.5] (4.5,\decalageY) -- (4.5,2);
    \draw[fill, RdBu-9-1] (4.5,\decalageY) circle (2pt);
    \draw[fill, RdBu-9-1] (4.5,2) circle (2pt);
    \node[above] at (4.5,2) {\(v\)};
    % Seuil 3
    \draw[ffc,line width=.5] (5.7,\decalageY) -- (5.7,.4);
    \draw[fill, RdBu-9-1] (5.7,\decalageY) circle (2pt);
    \draw[fill, RdBu-9-1] (5.7,.4) circle (2pt);
    
    \draw[|-|] (3,-.7cm) --++(3,0) node[pos=.5, fill=white, inner
    sep=1pt, font=\small] {$\delta$};

    \draw[<->, shorten >=2pt,shorten <=2pt] (2,.4cm) --++ (0,1.6cm) node[pos=.5, fill=white, inner
    sep=1pt, font=\small] {$i$};
  \end{scope}
\end{tikzpicture}
}\hfill
  \subfloat[\label{fig:fnc_app_inc_2}]{\begin{tikzpicture}[scale=.7]
  \def\decalageX{-.2}
  \def\decalageY{-.2}
  % Courbe
  \begin{scope}[transparency group]
    % fond
    \begin{scope}
      \path[ffa_fade_m]  (0,1.2) -- (2, 1.2)  -- (2,0) -- (0,0) --
      cycle;
      \path[ffa]  (2,1.2) -- (4.8, 1.2) --(6,2)  -- (8,2) -- (8,0) --(2,0) -- cycle;
      \path[ffa_fade]  (8,2) -- (9, 2)  -- (9,0) -- (8,0) -- cycle;
    \end{scope}
    % bords
    \begin{scope}
      \path[ffc_fade_m] (0,1.2) -- (2,1.2) ;
      \path[ffc, dotted] (4.8,1.2) -- (3,0);
      \path[ffc] (2,1.2) -- (4.8,1.2) -- (6, 2) -++ (3,0) ;
      \path[ffc_fade] (8,2) -- (9,2) ;
    \end{scope}
  \end{scope}
  % Axes X, Y
  \begin{scope}
    % Axe X
    \begin{scope}
      % Axe
      \draw[<->] (0, \decalageX) --++ (9, 0) coordinate (x axis);
      % Graduations
      \foreach \n/\t in {1/{},2/{},3/{},4/{},5/{},6/{},7/{},8/{}}
      {
        \draw[-] (\n, \decalageX - .05) --++ (0, .1);
        \node[below, font=\footnotesize] at (\n, \decalageX - .05) {\t};
      }
      % label
      \node[below left, yshift=-.1cm, font=\small] at (x axis) {\itshape Métrique};
    \end{scope}
    % Axe Y
    \begin{scope}
      % Axe
      \draw[-] (\decalageY ,0) --++ (0, 2) coordinate (y axis);
      % Graduations
      \foreach \n/\t in {0/{0},2/{1}}
      {
        \draw[-] (\decalageY -.05, \n) --++ (.1, 0);
        \node[left, font=\footnotesize] at (\decalageY -.05, \n) {\t};
      }
      % Label
      \node[above] at (y axis) {$\mu$};
    \end{scope}
  \end{scope}
  \begin{scope}
    % Seuil 1
    \draw[fill, RdBu-9-1] (3,\decalageY) circle (2pt);
    % Seuil 2
    \draw[ffc,line width=.5] (6,\decalageY) -- (6,2);
    \draw[fill, RdBu-9-1] (6,\decalageY) circle (2pt);
    \draw[fill, RdBu-9-1] (6,2) circle (2pt);
    \node[above] at (6,2) {\(v\)};

    \draw[|-|] (3,-.7cm) --++(3,0) node[pos=.5, fill=white, inner
    sep=1pt, font=\small] {$\delta$};
  \end{scope}
\end{tikzpicture}
}

  \subfloat[\label{fig:fnc_app_inc_3}]{\begin{tikzpicture}[scale=.7]
  \def\decalageX{-.2}
  \def\decalageY{-.2}
  % Courbe
  \begin{scope}[transparency group]
    % fond
    \begin{scope}
      \path[ffa_fade_m]  (0,2) -- (1, 2)  -- (1,0) -- (0,0) -- cycle;
      \path[ffa]  (1,2) -- (3, 2)  -- (3.6,1.6) --(8,1.6)-- (8,0) -- (1,0) -- cycle;
      \path[ffa_fade_m]  (8,1.6) -- (9, 1.6)  -- (9,0) -- (8,0) -- cycle;
    \end{scope}
    % bords
    \begin{scope}
      \path[ffc_fade_m] (0,2) -- (2,2) ;
      \path[ffc, dotted] (3.6,1.6) -- (6,0);
      \path[ffc] (2,2) -- (3,2) -- (3.6,1.6) --(8,1.6) ;
      \path[ffc_fade] (8,1.6) -- (9,1.6) ;
    \end{scope}
  \end{scope}
  % Axes X, Y
  \begin{scope}
    % Axe X
    \begin{scope}
      % Axe
      \draw[<->] (0, \decalageX) --++ (9, 0) coordinate (x axis);
      % Graduations
      \foreach \n/\t in {1/{},2/{},3/{},4/{},5/{},6/{},7/{},8/{}}
      {
        \draw[-] (\n, \decalageX - .05) --++ (0, .1);
        \node[below, font=\footnotesize] at (\n, \decalageX - .05) {\t};
      }
      % label
      \node[below left, yshift=-.1cm, font=\small] at (x axis)
      {\itshape Métrique};
    \end{scope}
    % Axe Y
    \begin{scope}
      % Axe
      \draw[-] (\decalageY ,0) --++ (0, 2) coordinate (y axis);
      % Graduations
      \foreach \n/\t in {0/{0},2/{1}}
      {
        \draw[-] (\decalageY -.05, \n) --++ (.1, 0);
        \node[left, font=\footnotesize] at (\decalageY -.05, \n) {\t};
      }
      % Label
      \node[above] at (y axis) {$\mu$};
    \end{scope}
  \end{scope}
  \begin{scope}
    % Seuil 1
    \draw[ffc,line width=.5] (3,\decalageY) -- (3,2);
    \draw[fill, RdBu-9-1] (3,\decalageY) circle (2pt);
    \draw[fill, RdBu-9-1] (3,2) circle (2pt);
    % Seuil 2
    \draw[fill, RdBu-9-1] (6,\decalageY) circle (2pt);
    \node[above] at (3,2) {\(v\)};

    \draw[|-|] (3,-.7cm) --++(3,0) node[pos=.5, fill=white, inner
    sep=1pt, font=\small] {$\delta$};
  \end{scope}
\end{tikzpicture}
}\hfill
  \subfloat[\label{fig:fnc_app_inc_4}]{\begin{tikzpicture}[scale=.7]
  \def\decalageX{-.2}
  \def\decalageY{-.2}
  % Courbe
  \begin{scope}[transparency group]
    % fond
    \begin{scope}
      \path[ffa2]  (1,0)--(1,2) --(3,2) -- (3.75,1) -- (5.25,1) -- (6,2) --(8,2) --(8,0)-- cycle;
      \path[ffa2_fade_m]  (0,2) -- (1, 2) -- (1,0) --(0,0) -- cycle;
      \path[ffa2_fade]  (8,2) -- (9, 2) -- (9,0) --(8,0) -- cycle;
    \end{scope}
    % bords
    \begin{scope}
      \path[ffc2, dotted] (3,2) -- (4.5,0) ;
      \path[ffc2, dotted] (4.5,0) -- (6,2) ;
      \path[ffc2] (1,2) -- (3,2) -- (3.75,1) -- (5.25,1) -- (6,2) -- (8,2) ;
      \path[ffc2_fade_m] (0,2) -- (1,2) ;
      \path[ffc2_fade] (8,2) -- (9,2) ;
    \end{scope}
  \end{scope}
  % Axes X, Y
  \begin{scope}
    % Axe X
    \begin{scope}
      % Axe
      \draw[<->] (0, \decalageX) --++ (9, 0) coordinate (x axis);
      % Graduations
      \foreach \n/\t in {1/{},2/{},3/{},4/{},5/{},6/{},7/{},8/{}}
      {
        \draw[-] (\n, \decalageX - .05) --++ (0, .1);
        \node[below, font=\footnotesize] at (\n, \decalageX - .05) {\t};
      }
      % label
      \node[below left, yshift=-.1cm, font=\small] at (x axis) {\itshape Métrique};
    \end{scope}
    % Axe Y
    \begin{scope}
      % Axe
      \draw[-] (\decalageY ,0) --++ (0, 2) coordinate (y axis);
      % Graduations
      \foreach \n/\t in {0/{0},2/{1}}
      {
        \draw[-] (\decalageY -.05, \n) --++ (.1, 0);
        \node[left, font=\footnotesize] at (\decalageY -.05, \n) {\t};
      }
      % Label
      \node[above] at (y axis) {$\mu$};
    \end{scope}
  \end{scope}
  \begin{scope}
    % Seuil 1
    \draw[ffc2,line width=.5] (3,\decalageY) -- (3,2);
    \draw[fill, RdBu-9-9] (3,\decalageY) circle (2pt);
    \draw[fill, RdBu-9-9] (3,2) circle (2pt);
    % Seuil 2
    \draw[fill, RdBu-9-9] (4.5,\decalageY) circle (2pt);
    \node[above] at (4.5,0) {\contour{white}{\(v\)}};
    % Seuil 3
    \draw[ffc2,line width=.5] (6,\decalageY) -- (6,2);
    \draw[fill, RdBu-9-9] (6,\decalageY) circle (2pt);
    \draw[fill, RdBu-9-9] (6,2) circle (2pt);
    \draw[|-|] (3,-.7cm) --++(3,0) node[pos=.5, fill=white, inner
    sep=1pt, font=\small] {$\delta$};

    \draw[<->,shorten <=2pt] (4,1cm) --++ (0,1cm) node[pos=.5, right,
    font=\small] {$c$};
  \end{scope}
\end{tikzpicture}
}
  %
  \caption[Représentation des \emph{fonctions d'appartenance} de
  différents \emph{fuzzyfieurs} avec une
  \emph{incertitude}]{Représentation des \emph{fonctions
      d'appartenance} des \emph{fuzzyfieurs}
    \protect\onto[orla]{Eq\-Val} \protect\subref{fig:fnc_app_inc},
    \protect\onto[orla]{Sup\-Val} \protect\subref{fig:fnc_app_inc_2},
    \protect\onto[orla]{Inf\-Val} \protect\subref{fig:fnc_app_inc_3}
    et \protect\onto[orla]{Eq\-Val} avec le \emph{modifieur}
    \protect\onto[orla]{Not} \protect\subref{fig:fnc_app_inc_4}, avec
    un degré de \emph{certitude} (\(i\)) variable (respectivement
    0,8, 0,4, 0,2 et 0,5).}
  \label{fig:qs}
\end{figure}




\begin{figure}
  \centering
  \missingfigure{Comparaison fusion indices avec et sans incertitude}
\end{figure}


%%% Local Variables:
%%% mode: latex
%%% TeX-master: "../../../../main"
%%% End:
