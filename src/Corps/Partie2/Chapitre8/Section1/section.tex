La \emph{fusion} est la dernière phase de notre méthodologie
(\autoref{chap:04}). C'est au cours de cette phase que les \emph{zones
  de localisation compatibles} construites durant la \emph{phase de
  spatialisation} (\autoref{chap:07}) sont \emph{fusionnées} de
manière à obtenir une \emph{zone de localisation probable,} figurant
le résultat de la méthodologie pour \emph{l'ensemble des indices de
  localisation} traités.

\subsection{Présnetation des 3 étapes}


\tdi{Transition Pour faire ces fusions on a besoin d'opérateurs}

La mise en place de ces trois étapes nécessite l'utilisation
d'opérateurs de fusions.


\subsection{Opérateurs}

Comme nous l'avons montré dans l'état de l'art (\autoref{sec:3-2}), il
existe plusieurs opérateurs permettant d'étendre les unions et les
intersections ensemblistes à la théorie des sous-ensembles
flous. Cependant, aucun opérateur n'offre la possibilité de conserver
toutes les caractéristiques de la théorie des ensembles et tous
ne possèdent pas les mêmes propriétés. Le choix des opérateurs
d'intersection et de fusion nécessite donc d'identifier les
propriétés nécessaires.


\tdi{Si ne pas passer les caractéristiques des opérateurs ici.}

\tdi{Nécessité d'utiliser des opérateurs duals}

\tdi{Enlever les opérateurs drastiques car ils sont trop sévères et ne
permettent pas de conserver des positions dont le degré d'appartenance
n'est pas au moins un fois de 1. En plus si j'ai n indices il faut
qu'il y ait n-1 indices dont le dégré est de 1.}

\tdi{Loi du tier-exclu}

\begin{figure}
  \centering
  \begin{tikzpicture}
  \def\decalageX{-.2}
  \def\decalageY{-.2}

  \newcommand{\calcangle}[4]{%
    % #1 nom de la varaible renvoyée
    % #2 coordonnées de l'objet de référence
    % #3 coordonnées du point visé
    % #4 coordonées de la position testée
    \pgfmathanglebetweenlines{#2}{#3}{#2}{#4}%
    \global\let#1\pgfmathresult%
  }

  % Arrow
  \begin{scope}
    \path[draw, -, shorten >=5pt, shorten <=5pt] (-1,0) -- (-5,-2);
    %\path[draw, ->,shorten >=5pt, shorten <=5pt] (8.25,3) -- (10,3);
    %\path[draw, -,shorten >=12pt, shorten <=5pt] (1,-3) |- (3.65,-1);
    %\path[draw, ->,shorten >=5pt, shorten <=5pt] (8.25,-1) -- (10,-1);
  \end{scope}
  % Fuzzyfication 1 (Sud)
  \begin{scope}[xshift=-2cm, local bounding box=fuzz]
    \begin{scope}
      \foreach \x in {0,.25,...,1.75}{ \foreach \y in {0,.25,...,1.75}
        {
          % Calcul de l'angle
          \calcangle{\angle}{\pgfpoint{1cm}{1cm}}{\pgfpoint{1cm}{0cm}}{\pgfpoint{\x cm}{\y cm}}
          % Fuzzyfication de l'angle
          \pgfmathsetmacro\fuzzy{%
            ifthenelse(\angle < 90,(90-\angle)/90,
            ifthenelse(\angle > 270,(\angle-270)/90,0)
            )
          }
          % Calcul du rayon à partir de la fuzzyfication
          \pgfmathsetmacro\radius{\fuzzy*2.5}
          \fill[RdBu-9-1] (\x+.125,.125+\y) circle (\radius pt);
        }
      }
      \path[ffc] (0,0) rectangle (2,2);
    \end{scope}
    \node[text width=3cm, align=center, anchor=north, font=\footnotesize] at (1,-.25)
    {\itshape Zone de localisation compatible \normalfont \textcolor{RdBu-9-1}{\textsf{A}}};
  \end{scope} 
  % Fuzzyfication 2
  \begin{scope}[xshift=2cm,local bounding box=fuzz2]
    \begin{scope}
      \foreach \x in {0,.25,...,1.75}{
        \foreach \y in {0,.25,...,1.75}
        {
          % Calcul de l'angle
          \calcangle{\angle}{\pgfpoint{1cm}{1cm}}{\pgfpoint{1cm}{0cm}}{\pgfpoint{\x cm}{\y cm}}
          % Fuzzyfication de l'angle
          \pgfmathsetmacro\fuzzy{%
            ifthenelse(\angle < 90,(90-\angle)/90,
            ifthenelse(\angle > 270,(\angle-270)/90,0)
            )
          }
          % Calcul du rayon à partir de la fuzzyfication
          \pgfmathsetmacro\radius{(1-\fuzzy)*2.5}
          \fill[RdBu-9-9] (\x+.125,.125+\y) circle (\radius pt);
        }
      }
      \path[ffc2] (0,0) rectangle (2,2);
    \end{scope}
    \node[text width=3cm, align=center, anchor=north, font=\footnotesize] at (1,-.25)
    {\itshape Zone de localisation compatible \normalfont \textcolor{RdBu-9-9}{\textsf{A\up{C}}}};
  \end{scope}
  % Fusion 1
  \begin{scope}[xshift=-6cm, yshift=-4cm,local bounding box=fus1]
    \begin{scope}
      \foreach \x in {0,.25,...,1.75}{
        \foreach \y in {0,.25,...,1.75}
        {
          % Calcul des angles
          \calcangle{\angle}{\pgfpoint{1cm}{1cm}}{\pgfpoint{1cm}{0cm}}{\pgfpoint{\x
              cm}{\y cm}}
          % Fuzzyfication des angles
          \pgfmathsetmacro\fuzzy{%
            ifthenelse(\angle < 90,(90-\angle)/90,
            ifthenelse(\angle > 270,(\angle-270)/90,0)
            )
          }          
          % fusion (zadeh)
          \pgfmathsetmacro\fus{max(\fuzzy, 1-\fuzzy)}          
          % Calcul du rayon à partir de la fuzzyfication
          \pgfmathsetmacro\radius{\fus*2.5}
          \fill[RdBu-9-9] (\x+.125,.125+\y) circle (\radius pt);
        }
      }
      \path[ffc2] (0,0) rectangle (2,2);
    \end{scope}
    \node[text width=3cm, align=center, anchor=north, font=\footnotesize] at (1,-.25)
    {\itshape Zone de localisation compatible \normalfont \textcolor{RdBu-9-9}{\textsf{B}}};
  \end{scope}
  % Fusion 2
  \begin{scope}[xshift=-2cm, yshift=-4cm,local bounding box=fus2]
    \begin{scope}
      \foreach \x in {0,.25,...,1.75}{
        \foreach \y in {0,.25,...,1.75}
        {
          % Calcul des angles
          \calcangle{\angle}{\pgfpoint{1cm}{1cm}}{\pgfpoint{1cm}{0cm}}{\pgfpoint{\x
              cm}{\y cm}}
          % Fuzzyfication des angles
          \pgfmathsetmacro\fuzzy{%
            ifthenelse(\angle < 90,(90-\angle)/90,
            ifthenelse(\angle > 270,(\angle-270)/90,0)
            )
          }
          % fusion (luka)
          \pgfmathsetmacro\fus{min(\fuzzy + (1-\fuzzy), 0)}          
          % Calcul du rayon à partir de la fuzzyfication
          \pgfmathsetmacro\radius{\fus*2.5}
          \fill[RdBu-9-9] (\x+.125,.125+\y) circle (\radius pt);
        }
      }
      \path[ffc2] (0,0) rectangle (2,2);
    \end{scope}
    \node[text width=3cm, align=center, anchor=north, font=\footnotesize] at (1,-.25)
    {\itshape Zone de localisation compatible \normalfont \textcolor{RdBu-9-9}{\textsf{B}}};
  \end{scope}
  % Fusion 3
  \begin{scope}[xshift=2cm, yshift=-4cm,local bounding box=fus3]
    \begin{scope}
      \foreach \x in {0,.25,...,1.75}{
        \foreach \y in {0,.25,...,1.75}
        {
          % Calcul des angles
          \calcangle{\angle}{\pgfpoint{1cm}{1cm}}{\pgfpoint{1cm}{0cm}}{\pgfpoint{\x
              cm}{\y cm}}
          % Fuzzyfication des angles
          \pgfmathsetmacro\fuzzy{%
            ifthenelse(\angle < 90,(90-\angle)/90,
            ifthenelse(\angle > 270,(\angle-270)/90,0)
            )
          }
          % fusion (zadeh)
          \pgfmathsetmacro\fus{(\fuzzy + (1-\fuzzy)) - \fuzzy * (1-\fuzzy)}          
          % Calcul du rayon à partir de la fuzzyfication
          \pgfmathsetmacro\radius{\fus*2.5}
          \fill[RdBu-9-9] (\x+.125,.125+\y) circle (\radius pt);
        }
      }
      \path[ffc2] (0,0) rectangle (2,2);
    \end{scope}
    \node[text width=3cm, align=center, anchor=north, font=\footnotesize] at (1,-.25)
    {\itshape Zone de localisation compatible \normalfont \textcolor{RdBu-9-9}{\textsf{B}}};
  \end{scope}
  % Fusion 4
  \begin{scope}[xshift=6cm, yshift=-4cm,local bounding box=fus4]
    \begin{scope}
      \foreach \x in {0,.25,...,1.75}{
        \foreach \y in {0,.25,...,1.75}
        {
          % Calcul des angles
          \calcangle{\angle}{\pgfpoint{1cm}{1cm}}{\pgfpoint{1cm}{0cm}}{\pgfpoint{\x
              cm}{\y cm}}
          % Fuzzyfication des angles
          \pgfmathsetmacro\fuzzy{%
            ifthenelse(\angle < 90,(90-\angle)/90,
            ifthenelse(\angle > 270,(\angle-270)/90,0)
            )
          }
          % fusion (drast)
          \pgfmathsetmacro\fus{%
            ifthenelse(\fuzzy < 0.001, (1-\fuzzy),
            ifthenelse((1-\fuzzy) < 0.001, \fuzzy, 1)
            )
          }          
          % Calcul du rayon à partir de la fuzzyfication
          \pgfmathsetmacro\radius{\fus*2.5}
          \fill[RdBu-9-9] (\x+.125,.125+\y) circle (\radius pt);
        }
      }
      \path[ffc2] (0,0) rectangle (2,2);
    \end{scope}
    \node[text width=3cm, align=center, anchor=north, font=\footnotesize] at (1,-.25)
    {\itshape Zone de localisation compatible \normalfont \textcolor{RdBu-9-9}{\textsf{B}}};
  \end{scope}
\end{tikzpicture}
  \caption{tier-ex}
\end{figure}

\tdi{Principe de non contradiction}

\begin{figure}
  \centering
  \begin{tikzpicture}
  % Arrow
  \begin{scope}
    \path[draw, -, shorten >=5pt, shorten <=5pt] (-1,0) -- (-5,-2);
    % \path[draw, ->,shorten >=5pt, shorten <=5pt] (8.25,3) -- (10,3);
    % \path[draw, -,shorten >=12pt, shorten <=5pt] (1,-3) |- (3.65,-1);
    % \path[draw, ->,shorten >=5pt, shorten <=5pt] (8.25,-1) -- (10,-1);
  \end{scope}
  % Fuzzyfication 1 (Sud)
  \begin{scope}[xshift=-2cm, local bounding box=fuzz]
    \begin{scope}
      \foreach \x in {0,.25,...,1.75}{ \foreach \y in {0,.25,...,1.75}
        {
          % Calcul et représentation floue distance euclidienne
          \pgfmathsetmacro\dist{sqrt((\x-0.125)^2+(\y-1.375)^2)}
          % Fuzzyfication de la distance
          \pgfmathsetmacro\fuzzy{%
            ifthenelse(\dist < .25,1,%
            ifthenelse(\dist < 2,-0.57*\dist+1.14,0)%
            )
          }
          % Calcul du rayon à partir de la fuzzyfication
          \pgfmathsetmacro\radius{\fuzzy*2.5}
          \fill[RdBu-9-1] (\x+.125,.125+\y) circle (\radius pt);
        }
      }
      \path[ffc] (0,0) rectangle (2,2);
    \end{scope}
    \node[text width=3cm, align=center, anchor=north, font=\footnotesize] at (1,-.25)
    {\itshape Zone de localisation compatible \normalfont \textcolor{RdBu-9-1}{\textsf{A}}};
  \end{scope} 
  % Fuzzyfication 2
  \begin{scope}[xshift=2cm,local bounding box=fuzz2]
    \begin{scope}
      \foreach \x in {0,.25,...,1.75}{
        \foreach \y in {0,.25,...,1.75}
        {
          % Calcul et représentation floue distance euclidienne
          \pgfmathsetmacro\dist{sqrt((\x-0.125)^2+(\y-1.375)^2)}
          % Fuzzyfication de la distance
          \pgfmathsetmacro\fuzzy{%
            ifthenelse(\dist < .25,1,%
            ifthenelse(\dist < 2,-0.57*\dist+1.14,0)%
            )
          }
          % Calcul du rayon à partir de la fuzzyfication
          \pgfmathsetmacro\radius{(1-\fuzzy)*2.5}
          \fill[RdBu-9-9] (\x+.125,.125+\y) circle (\radius pt);
        }
      }
      \path[ffc2] (0,0) rectangle (2,2);
    \end{scope}
    \node[text width=3cm, align=center, anchor=north, font=\footnotesize] at (1,-.25)
    {\itshape Zone de localisation compatible \normalfont \textcolor{RdBu-9-9}{\textsf{A\up{C}}}};
  \end{scope}
  % Fusion 1
  \begin{scope}[xshift=-6cm, yshift=-4cm,local bounding box=fus1]
    \begin{scope}
      \foreach \x in {0,.25,...,1.75}{
        \foreach \y in {0,.25,...,1.75}
        {
          % Calcul et représentation floue distance euclidienne
          \pgfmathsetmacro\dist{sqrt((\x-0.125)^2+(\y-1.375)^2)}
          % Fuzzyfication de la distance
          \pgfmathsetmacro\fuzzy{%
            ifthenelse(\dist < .25,1,%
            ifthenelse(\dist < 2,-0.57*\dist+1.14,0)%
            )
          }
          \pgfmathsetmacro\fus{min(\fuzzy, 1-\fuzzy)}          
          % Calcul du rayon à partir de la fuzzyfication
          \pgfmathsetmacro\radius{\fus*2.5}
          \fill[black] (\x+.125,.125+\y) circle (\radius pt);
        }
      }
      \path[ffc, black] (0,0) rectangle (2,2);
    \end{scope}
    \node[text width=3cm, align=center, anchor=north, font=\footnotesize] at (1,-.25)
    {\itshape Intersection avec la \emph{t-norme} de \bsc{Zadeh}};
  \end{scope}
  % Fusion 2
  \begin{scope}[xshift=-2cm, yshift=-4cm,local bounding box=fus2]
    \begin{scope}
      \foreach \x in {0,.25,...,1.75}{
        \foreach \y in {0,.25,...,1.75}
        {
          % Calcul et représentation floue distance euclidienne
          \pgfmathsetmacro\dist{sqrt((\x-0.125)^2+(\y-1.375)^2)}
          % Fuzzyfication de la distance
          \pgfmathsetmacro\fuzzy{%
            ifthenelse(\dist < .25,1,%
            ifthenelse(\dist < 2,-0.57*\dist+1.14,0)%
            )
          }
          \pgfmathsetmacro\fus{max(\fuzzy + (1-\fuzzy) -1, 0)}          
          % Calcul du rayon à partir de la fuzzyfication
          \pgfmathsetmacro\radius{\fus*2.5}
          \fill[black] (\x+.125,.125+\y) circle (\radius pt);
        }
      }
      \path[ffc, black] (0,0) rectangle (2,2);
    \end{scope}
    \node[text width=3cm, align=center, anchor=north, font=\footnotesize] at (1,-.25)
    {\itshape Intersection avec la \emph{t-norme} de \bsc{Łukasiewicz}};
  \end{scope}
  % Fusion 3
  \begin{scope}[xshift=2cm, yshift=-4cm,local bounding box=fus3]
    \begin{scope}
      \foreach \x in {0,.25,...,1.75}{
        \foreach \y in {0,.25,...,1.75}
        {
          % Calcul et représentation floue distance euclidienne
          \pgfmathsetmacro\dist{sqrt((\x-0.125)^2+(\y-1.375)^2)}
          % Fuzzyfication de la distance
          \pgfmathsetmacro\fuzzy{%
            ifthenelse(\dist < .25,1,%
            ifthenelse(\dist < 2,-0.57*\dist+1.14,0)%
            )
          }
          \pgfmathsetmacro\fus{\fuzzy * (1-\fuzzy)}          
          % Calcul du rayon à partir de la fuzzyfication
          \pgfmathsetmacro\radius{\fus*2.5}
          \fill[black] (\x+.125,.125+\y) circle (\radius pt);
        }
      }
      \path[ffc, black] (0,0) rectangle (2,2);
    \end{scope}
    \node[text width=3cm, align=center, anchor=north, font=\footnotesize] at (1,-.25)
    {\itshape Intersection avec la \emph{t-norme} probabiliste};
  \end{scope}
  % Fusion 4
  \begin{scope}[xshift=6cm, yshift=-4cm,local bounding box=fus4]
    \begin{scope}
      \foreach \x in {0,.25,...,1.75}{
        \foreach \y in {0,.25,...,1.75}
        {
          % Calcul et représentation floue distance euclidienne
          \pgfmathsetmacro\dist{sqrt((\x-0.125)^2+(\y-1.375)^2)}
          % Fuzzyfication de la distance
          \pgfmathsetmacro\fuzzy{%
            ifthenelse(\dist < .25,1,%
            ifthenelse(\dist < 2,-0.57*\dist+1.14,0)%
            )
          }
          % fusion (drast)
          \pgfmathsetmacro\fus{%
            ifthenelse(\fuzzy > 0.999, (1-\fuzzy),
            ifthenelse((1-\fuzzy)  >0.999, \fuzzy, 0)
            )
          }          
          % Calcul du rayon à partir de la fuzzyfication
          \pgfmathsetmacro\radius{\fus*2.5}
          \fill[black] (\x+.125,.125+\y) circle (\radius pt);
        }
      }
      \path[ffc, black] (0,0) rectangle (2,2);
    \end{scope}
    \node[text width=3cm, align=center, anchor=north, font=\footnotesize] at (1,-.25)
    {\itshape Intersection avec la \emph{t-norme} drastique};
  \end{scope}
\end{tikzpicture}
  \caption{non contradiction}
\end{figure}

Concepts non excluant

On a pas eu de problèmes du fait que cette loi ne soit pas validée

On peut trouver des exemples réels de cas ou ce n'est pas vrai

Lister les opérateurs qui valident tout

Caractéristiques non excluantes

Discuter du fait que ce soit mieux que ces principes ne soit pas
validés.

\tdi{Opérateurs archimédiens}

Mettre une figure de l'intersection d'angle avec les opérateurs
archimédiens (ex. sud et sud-est).

Parler l'exemple de l'union, je suis à côté d'un lac et plus vrai si
on est a côté de deux lac -> wtf

Ce critère discalifie lukacevicz et probabiliste

\tdi{Raisonnement en termes d'interprétabilité}

On choisi Zadeh

\begin{figure}
  \centering
  \begin{tikzpicture}
  \def\decalageX{-.2}
  \def\decalageY{-.2}

  \newcommand{\calcangle}[4]{%
    % #1 nom de la varaible renvoyée
    % #2 coordonnées de l'objet de référence
    % #3 coordonnées du point visé
    % #4 coordonées de la position testée
    \pgfmathanglebetweenlines{#2}{#3}{#2}{#4}%
    \global\let#1\pgfmathresult%
  }

  % Arrow
  \begin{scope}
    \path[draw, -, shorten >=5pt, shorten <=5pt] (-1,0) -- (-5,-2);
    %\path[draw, ->,shorten >=5pt, shorten <=5pt] (8.25,3) -- (10,3);
    %\path[draw, -,shorten >=12pt, shorten <=5pt] (1,-3) |- (3.65,-1);
    %\path[draw, ->,shorten >=5pt, shorten <=5pt] (8.25,-1) -- (10,-1);
  \end{scope}
  % Fuzzyfication 1 (Sud)
  \begin{scope}[xshift=-2cm, local bounding box=fuzz]
    \begin{scope}
      \foreach \x in {0,.25,...,1.75}{ \foreach \y in {0,.25,...,1.75}
        {
          % Calcul de l'angle
          \calcangle{\angle}{\pgfpoint{1cm}{1cm}}{\pgfpoint{1cm}{0cm}}{\pgfpoint{\x cm}{\y cm}}
          % Fuzzyfication de l'angle
          \pgfmathsetmacro\fuzzy{%
            ifthenelse(\angle < 90,(90-\angle)/90,
            ifthenelse(\angle > 270,(\angle-270)/90,0)
            )
          }
          % Calcul du rayon à partir de la fuzzyfication
          \pgfmathsetmacro\radius{\fuzzy*2.5}
          \fill[RdBu-9-1] (\x+.125,.125+\y) circle (\radius pt);
        }
      }
      \path[ffc] (0,0) rectangle (2,2);
    \end{scope}
    \node[text width=3cm, align=center, anchor=north, font=\footnotesize] at (1,-.25)
    {\itshape Zone de localisation compatible \normalfont \textcolor{RdBu-9-1}{\textsf{A}}};
  \end{scope} 
  % Fuzzyfication 2
  \begin{scope}[xshift=2cm,local bounding box=fuzz2]
    \begin{scope}
      \foreach \x in {0,.25,...,1.75}{
        \foreach \y in {0,.25,...,1.75}
        {
          % Calcul de l'angle
          \calcangle{\angle}{\pgfpoint{1cm}{1cm}}{\pgfpoint{2cm}{1cm}}{\pgfpoint{\x cm}{\y cm}}
          % Fuzzyfication de l'angle
          \pgfmathsetmacro\fuzzy{%
            ifthenelse(\angle < 90,(90-\angle)/90,
            ifthenelse(\angle > 270,(\angle-270)/90,0)
            )
          }
          % Calcul du rayon à partir de la fuzzyfication
          \pgfmathsetmacro\radius{\fuzzy*2.5}
          \fill[RdBu-9-9] (\x+.125,.125+\y) circle (\radius pt);
        }
      }
      \path[ffc2] (0,0) rectangle (2,2);
    \end{scope}
    \node[text width=3cm, align=center, anchor=north, font=\footnotesize] at (1,-.25)
    {\itshape Zone de localisation compatible \normalfont \textcolor{RdBu-9-9}{\textsf{B}}};
  \end{scope}
  % Fusion 1
  \begin{scope}[xshift=-6cm, yshift=-4cm,local bounding box=fus1]
    \begin{scope}
      \foreach \x in {0,.25,...,1.75}{
        \foreach \y in {0,.25,...,1.75}
        {
          % Calcul des angles
          \calcangle{\angleA}{\pgfpoint{1cm}{1cm}}{\pgfpoint{1cm}{0cm}}{\pgfpoint{\x
              cm}{\y cm}}
          % Fuzzyfication des angles
          \pgfmathsetmacro\fuzzyA{%
            ifthenelse(\angleA < 90,(90-\angleA)/90,
            ifthenelse(\angleA > 270,(\angleA-270)/90,0)
            )
          }
          \calcangle{\angleB}{\pgfpoint{1cm}{1cm}}{\pgfpoint{2cm}{1cm}}{\pgfpoint{\x cm}{\y cm}}
          \pgfmathsetmacro\fuzzyB{%
            ifthenelse(\angleB < 90,(90-\angleB)/90,
            ifthenelse(\angleB > 270,(\angleB-270)/90,0)
            )
          }
          % fusion (zadeh)
          \pgfmathsetmacro\fus{max(\fuzzyA, \fuzzyB)}          
          % Calcul du rayon à partir de la fuzzyfication
          \pgfmathsetmacro\radius{\fus*2.5}
          \fill[RdBu-9-9] (\x+.125,.125+\y) circle (\radius pt);
        }
      }
      \path[ffc2] (0,0) rectangle (2,2);
    \end{scope}
    \node[text width=3cm, align=center, anchor=north, font=\footnotesize] at (1,-.25)
    {\itshape Zone de localisation compatible \normalfont \textcolor{RdBu-9-9}{\textsf{B}}};
  \end{scope}
  % Fusion 2
  \begin{scope}[xshift=-2cm, yshift=-4cm,local bounding box=fus2]
    \begin{scope}
      \foreach \x in {0,.25,...,1.75}{
        \foreach \y in {0,.25,...,1.75}
        {
          % Calcul des angles
          \calcangle{\angleA}{\pgfpoint{1cm}{1cm}}{\pgfpoint{1cm}{0cm}}{\pgfpoint{\x
              cm}{\y cm}}
          % Fuzzyfication des angles
          \pgfmathsetmacro\fuzzyA{%
            ifthenelse(\angleA < 90,(90-\angleA)/90,
            ifthenelse(\angleA > 270,(\angleA-270)/90,0)
            )
          }
          \calcangle{\angleB}{\pgfpoint{1cm}{1cm}}{\pgfpoint{2cm}{1cm}}{\pgfpoint{\x cm}{\y cm}}
          \pgfmathsetmacro\fuzzyB{%
            ifthenelse(\angleB < 90,(90-\angleB)/90,
            ifthenelse(\angleB > 270,(\angleB-270)/90,0)
            )
          }
          % fusion (luka)
          \pgfmathsetmacro\fus{min(\fuzzyA + \fuzzyB, 1)}          
          % Calcul du rayon à partir de la fuzzyfication
          \pgfmathsetmacro\radius{\fus*2.5}
          \fill[RdBu-9-9] (\x+.125,.125+\y) circle (\radius pt);
        }
      }
      \path[ffc2] (0,0) rectangle (2,2);
    \end{scope}
    \node[text width=3cm, align=center, anchor=north, font=\footnotesize] at (1,-.25)
    {\itshape Zone de localisation compatible \normalfont \textcolor{RdBu-9-9}{\textsf{B}}};
  \end{scope}
  % Fusion 3
  \begin{scope}[xshift=2cm, yshift=-4cm,local bounding box=fus3]
    \begin{scope}
      \foreach \x in {0,.25,...,1.75}{
        \foreach \y in {0,.25,...,1.75}
        {
          % Calcul des angles
          \calcangle{\angleA}{\pgfpoint{1cm}{1cm}}{\pgfpoint{1cm}{0cm}}{\pgfpoint{\x
              cm}{\y cm}}
          % Fuzzyfication des angles
          \pgfmathsetmacro\fuzzyA{%
            ifthenelse(\angleA < 90,(90-\angleA)/90,
            ifthenelse(\angleA > 270,(\angleA-270)/90,0)
            )
          }
          \calcangle{\angleB}{\pgfpoint{1cm}{1cm}}{\pgfpoint{2cm}{1cm}}{\pgfpoint{\x cm}{\y cm}}
          \pgfmathsetmacro\fuzzyB{%
            ifthenelse(\angleB < 90,(90-\angleB)/90,
            ifthenelse(\angleB > 270,(\angleB-270)/90,0)
            )
          }
          % fusion (zadeh)
          \pgfmathsetmacro\fus{(\fuzzyA + \fuzzyB) - \fuzzyA * \fuzzyB}          
          % Calcul du rayon à partir de la fuzzyfication
          \pgfmathsetmacro\radius{\fus*2.5}
          \fill[RdBu-9-9] (\x+.125,.125+\y) circle (\radius pt);
        }
      }
      \path[ffc2] (0,0) rectangle (2,2);
    \end{scope}
    \node[text width=3cm, align=center, anchor=north, font=\footnotesize] at (1,-.25)
    {\itshape Zone de localisation compatible \normalfont \textcolor{RdBu-9-9}{\textsf{B}}};
  \end{scope}
  % Fusion 4
  \begin{scope}[xshift=6cm, yshift=-4cm,local bounding box=fus4]
    \begin{scope}
      \foreach \x in {0,.25,...,1.75}{
        \foreach \y in {0,.25,...,1.75}
        {
          % Calcul des angles
          \calcangle{\angleA}{\pgfpoint{1cm}{1cm}}{\pgfpoint{1cm}{0cm}}{\pgfpoint{\x
              cm}{\y cm}}
          % Fuzzyfication des angles
          \pgfmathsetmacro\fuzzyA{%
            ifthenelse(\angleA < 90,(90-\angleA)/90,
            ifthenelse(\angleA > 270,(\angleA-270)/90,0)
            )
          }
          \calcangle{\angleB}{\pgfpoint{1cm}{1cm}}{\pgfpoint{2cm}{1cm}}{\pgfpoint{\x cm}{\y cm}}
          \pgfmathsetmacro\fuzzyB{%
            ifthenelse(\angleB < 90,(90-\angleB)/90,
            ifthenelse(\angleB > 270,(\angleB-270)/90,0)
            )
          }
          % fusion (drast)
          \pgfmathsetmacro\fus{%
            ifthenelse(\fuzzyA < 0.001, \fuzzyB,
            ifthenelse(\fuzzyB < 0.001, \fuzzyA, 1)
            )
          }          
          % Calcul du rayon à partir de la fuzzyfication
          \pgfmathsetmacro\radius{\fus*2.5}
          \fill[RdBu-9-9] (\x+.125,.125+\y) circle (\radius pt);
        }
      }
      \path[ffc2] (0,0) rectangle (2,2);
    \end{scope}
    \node[text width=3cm, align=center, anchor=north, font=\footnotesize] at (1,-.25)
    {\itshape Zone de localisation compatible \normalfont \textcolor{RdBu-9-9}{\textsf{B}}};
  \end{scope}
\end{tikzpicture}
  \caption{sq}
\end{figure}

\begin{figure}
  \centering
  \begin{tikzpicture}
  \def\decalageX{-.2}
  \def\decalageY{-.2}

  \newcommand{\calcangle}[4]{%
    % #1 nom de la varaible renvoyée
    % #2 coordonnées de l'objet de référence
    % #3 coordonnées du point visé
    % #4 coordonées de la position testée
    \pgfmathanglebetweenlines{#2}{#3}{#2}{#4}%
    \global\let#1\pgfmathresult%
  }

  % Arrow
  % \begin{scope}
  %   \path[draw, -, shorten >=12pt, shorten <=5pt] (1,3) |- (3.65,3);
  %   \path[draw, ->,shorten >=5pt, shorten <=5pt] (8.25,3) -- (10,3);
  %   \path[draw, -,shorten >=12pt, shorten <=5pt] (1,-3) |- (3.65,-1);
  %   \path[draw, ->,shorten >=5pt, shorten <=5pt] (8.25,-1) -- (10,-1);
  % \end{scope}
  % Fuzzyfication 1 (Sud)
  \begin{scope}[xshift=-2cm, local bounding box=fuzz]
    \begin{scope}
      \foreach \x in {0,.25,...,1.75}{ \foreach \y in {0,.25,...,1.75}
        {
          % Calcul de l'angle
          \calcangle{\angle}{\pgfpoint{1cm}{1cm}}{\pgfpoint{1cm}{0cm}}{\pgfpoint{\x cm}{\y cm}}
          % Fuzzyfication de l'angle
          \pgfmathsetmacro\fuzzy{%
            ifthenelse(\angle < 90,(90-\angle)/90,
            ifthenelse(\angle > 270,(\angle-270)/90,0)
            )
          }
          % Calcul du rayon à partir de la fuzzyfication
          \pgfmathsetmacro\radius{\fuzzy*2.5}
          \fill[RdBu-9-1] (\x+.125,.125+\y) circle (\radius pt);
        }
      }
      \path[ffc] (0,0) rectangle (2,2);
    \end{scope}
    \node[text width=3cm, align=center, anchor=north, font=\footnotesize] at (1,-.25)
    {\itshape Zone de localisation compatible \normalfont \textcolor{RdBu-9-1}{\textsf{A}}};
  \end{scope} 
  % Fuzzyfication 2
  \begin{scope}[xshift=2cm,local bounding box=fuzz2]
    \begin{scope}
      \foreach \x in {0,.25,...,1.75}{
        \foreach \y in {0,.25,...,1.75}
        {
          % Calcul de l'angle
          \calcangle{\angle}{\pgfpoint{1cm}{1cm}}{\pgfpoint{2cm}{1cm}}{\pgfpoint{\x cm}{\y cm}}
          % Fuzzyfication de l'angle
          \pgfmathsetmacro\fuzzy{%
            ifthenelse(\angle < 90,(90-\angle)/90,
            ifthenelse(\angle > 270,(\angle-270)/90,0)
            )
          }
          % Calcul du rayon à partir de la fuzzyfication
          \pgfmathsetmacro\radius{\fuzzy*2.5}
          \fill[RdBu-9-9] (\x+.125,.125+\y) circle (\radius pt);
        }
      }
      \path[ffc2] (0,0) rectangle (2,2);
    \end{scope}
    \node[text width=3cm, align=center, anchor=north, font=\footnotesize] at (1,-.25)
    {\itshape Zone de localisation compatible \normalfont \textcolor{RdBu-9-9}{\textsf{B}}};
  \end{scope}
  % Fusion 1
  \begin{scope}[xshift=-6cm, yshift=-4cm,local bounding box=fus1]
    \begin{scope}
      \foreach \x in {0,.25,...,1.75}{
        \foreach \y in {0,.25,...,1.75}
        {
          % Calcul des angles
          \calcangle{\angleA}{\pgfpoint{1cm}{1cm}}{\pgfpoint{1cm}{0cm}}{\pgfpoint{\x
              cm}{\y cm}}
          % Fuzzyfication des angles
          \pgfmathsetmacro\fuzzyA{%
            ifthenelse(\angleA < 90,(90-\angleA)/90,
            ifthenelse(\angleA > 270,(\angleA-270)/90,0)
            )
          }
          \calcangle{\angleB}{\pgfpoint{1cm}{1cm}}{\pgfpoint{2cm}{1cm}}{\pgfpoint{\x cm}{\y cm}}
          \pgfmathsetmacro\fuzzyB{%
            ifthenelse(\angleB < 90,(90-\angleB)/90,
            ifthenelse(\angleB > 270,(\angleB-270)/90,0)
            )
          }
          % fusion (zadeh)
          \pgfmathsetmacro\fus{min(\fuzzyA, \fuzzyB)}          
          % Calcul du rayon à partir de la fuzzyfication
          \pgfmathsetmacro\radius{\fus*2.5}
          \fill[RdBu-9-9] (\x+.125,.125+\y) circle (\radius pt);
        }
      }
      \path[ffc2] (0,0) rectangle (2,2);
    \end{scope}
    \node[text width=3cm, align=center, anchor=north, font=\footnotesize] at (1,-.25)
    {\itshape Zone de localisation compatible \normalfont \textcolor{RdBu-9-9}{\textsf{B}}};
  \end{scope}
  % Fusion 2
  \begin{scope}[xshift=-2cm, yshift=-4cm,local bounding box=fus2]
    \begin{scope}
      \foreach \x in {0,.25,...,1.75}{
        \foreach \y in {0,.25,...,1.75}
        {
          % Calcul des angles
          \calcangle{\angleA}{\pgfpoint{1cm}{1cm}}{\pgfpoint{1cm}{0cm}}{\pgfpoint{\x
              cm}{\y cm}}
          % Fuzzyfication des angles
          \pgfmathsetmacro\fuzzyA{%
            ifthenelse(\angleA < 90,(90-\angleA)/90,
            ifthenelse(\angleA > 270,(\angleA-270)/90,0)
            )
          }
          \calcangle{\angleB}{\pgfpoint{1cm}{1cm}}{\pgfpoint{2cm}{1cm}}{\pgfpoint{\x cm}{\y cm}}
          \pgfmathsetmacro\fuzzyB{%
            ifthenelse(\angleB < 90,(90-\angleB)/90,
            ifthenelse(\angleB > 270,(\angleB-270)/90,0)
            )
          }
          % fusion (luka)
          \pgfmathsetmacro\fus{max(\fuzzyA + \fuzzyB - 1, 0)}          
          % Calcul du rayon à partir de la fuzzyfication
          \pgfmathsetmacro\radius{\fus*2.5}
          \fill[RdBu-9-9] (\x+.125,.125+\y) circle (\radius pt);
        }
      }
      \path[ffc2] (0,0) rectangle (2,2);
    \end{scope}
    \node[text width=3cm, align=center, anchor=north, font=\footnotesize] at (1,-.25)
    {\itshape Zone de localisation compatible \normalfont \textcolor{RdBu-9-9}{\textsf{B}}};
  \end{scope}
  % Fusion 3
  \begin{scope}[xshift=2cm, yshift=-4cm,local bounding box=fus3]
    \begin{scope}
      \foreach \x in {0,.25,...,1.75}{
        \foreach \y in {0,.25,...,1.75}
        {
          % Calcul des angles
          \calcangle{\angleA}{\pgfpoint{1cm}{1cm}}{\pgfpoint{1cm}{0cm}}{\pgfpoint{\x
              cm}{\y cm}}
          % Fuzzyfication des angles
          \pgfmathsetmacro\fuzzyA{%
            ifthenelse(\angleA < 90,(90-\angleA)/90,
            ifthenelse(\angleA > 270,(\angleA-270)/90,0)
            )
          }
          \calcangle{\angleB}{\pgfpoint{1cm}{1cm}}{\pgfpoint{2cm}{1cm}}{\pgfpoint{\x cm}{\y cm}}
          \pgfmathsetmacro\fuzzyB{%
            ifthenelse(\angleB < 90,(90-\angleB)/90,
            ifthenelse(\angleB > 270,(\angleB-270)/90,0)
            )
          }
          % fusion (zadeh)
          \pgfmathsetmacro\fus{\fuzzyA * \fuzzyB}          
          % Calcul du rayon à partir de la fuzzyfication
          \pgfmathsetmacro\radius{\fus*2.5}
          \fill[RdBu-9-9] (\x+.125,.125+\y) circle (\radius pt);
        }
      }
      \path[ffc2] (0,0) rectangle (2,2);
    \end{scope}
    \node[text width=3cm, align=center, anchor=north, font=\footnotesize] at (1,-.25)
    {\itshape Zone de localisation compatible \normalfont \textcolor{RdBu-9-9}{\textsf{B}}};
  \end{scope}
  % Fusion 4
  \begin{scope}[xshift=6cm, yshift=-4cm,local bounding box=fus4]
    \begin{scope}
      \foreach \x in {0,.25,...,1.75}{
        \foreach \y in {0,.25,...,1.75}
        {
          % Calcul des angles
          \calcangle{\angleA}{\pgfpoint{1cm}{1cm}}{\pgfpoint{1cm}{0cm}}{\pgfpoint{\x
              cm}{\y cm}}
          % Fuzzyfication des angles
          \pgfmathsetmacro\fuzzyA{%
            ifthenelse(\angleA < 90,(90-\angleA)/90,
            ifthenelse(\angleA > 270,(\angleA-270)/90,0)
            )
          }
          \calcangle{\angleB}{\pgfpoint{1cm}{1cm}}{\pgfpoint{2cm}{1cm}}{\pgfpoint{\x cm}{\y cm}}
          \pgfmathsetmacro\fuzzyB{%
            ifthenelse(\angleB < 90,(90-\angleB)/90,
            ifthenelse(\angleB > 270,(\angleB-270)/90,0)
            )
          }
          % fusion (drast)
          \pgfmathsetmacro\fus{%
            ifthenelse(\fuzzyA > 0.999, \fuzzyB,
            ifthenelse(\fuzzyB > 0.999, \fuzzyA, 0)
            )
          }          
          % Calcul du rayon à partir de la fuzzyfication
          \pgfmathsetmacro\radius{\fus*2.5}
          \fill[RdBu-9-9] (\x+.125,.125+\y) circle (\radius pt);
        }
      }
      \path[ffc2] (0,0) rectangle (2,2);
    \end{scope}
    \node[text width=3cm, align=center, anchor=north, font=\footnotesize] at (1,-.25)
    {\itshape Zone de localisation compatible \normalfont \textcolor{RdBu-9-9}{\textsf{B}}};
  \end{scope}
\end{tikzpicture}
  \caption{sq}
\end{figure}



\subsection{La prise en compte de la confiance}

La confiance peut se définir en amont de la spatialisation.

Diminuer la confiance c'est augmenter le degré d'appartenance minimal

Proposer une version améliorée de la fig finale du chapitre 4 avec la
confiance.

Dire que les fonctions présentées reviennent a considérer que la
confiance est absolue

Parler des seuils de la confiance (3) et des valeurs associées ou
contraintes.


\begin{figure}
  \centering
  \begin{tikzpicture}[scale=.7]
  \def\decalageX{-.2}
  \def\decalageY{-.2}
  % Courbe
  \begin{scope}[transparency group]
    % fond
    \begin{scope}
      \path[ffa]  (1,.4) -- (3.3,.4) -- (4.5, 2) -- (5.7,.4) -- (8,.4)
      -- (8,0) -- (1,0) -- cycle;
      \path[ffa_fade_m] (0,.4) -- (1,.4) -- (1,0) -- (0,0) -- cycle ;
      \path[ffa_fade] (8,.4) -- (9,.4) -- (9,0) -- (8,0) -- cycle ;
    \end{scope}
    % bords
    \begin{scope}
      \path[ffc] (1,.4) -- (3.3,.4) -- (4.5, 2) -- (5.7,.4) -- (8,.4);
      \path[ffc, dotted] (3.3,.4) -- (3,0);
      \path[ffc, dotted] (5.7,.4) -- (6,0);
      \path[ffc_fade_m] (0,.4) -- (1,.4) ;
      \path[ffc_fade] (8,.4) -- (9,.4) ;
    \end{scope}
  \end{scope}
  % Axes X, Y
  \begin{scope}
    % Axe X
    \begin{scope}
      % Axe
      \draw[<->] (0, \decalageX) --++ (9, 0) coordinate (x axis);
      % Graduations
      \foreach \n/\t in {1/{},2/{},3/{},4/{},5/{},6/{},7/{},8/{}}
      {
        \draw[-] (\n, \decalageX - .05) --++ (0, .1);
        \node[below, font=\footnotesize] at (\n, \decalageX - .05) {\t};
      }
      % label
      \node[below left, yshift=-.1cm, font=\small] at (x axis) {\itshape Métrique};
    \end{scope}
    % Axe Y
    \begin{scope}
      % Axe
      \draw[-] (\decalageY ,0) --++ (0, 2) coordinate (y axis);
      % Graduations
      \foreach \n/\t in {0/{0},2/{1}}
      {
        \draw[-] (\decalageY -.05, \n) --++ (.1, 0);
        \node[left, font=\footnotesize] at (\decalageY -.05, \n) {\t};
      }
      % Label
      \node[above] at (y axis) {$\mu$};
    \end{scope}
  \end{scope}
  \begin{scope}
    % Seuil 1
    \draw[ffc,line width=.5] (3.3,\decalageY) -- (3.3,.4);
    \draw[fill, RdBu-9-1] (3.3,\decalageY) circle (2pt);
    \draw[fill, RdBu-9-1] (3.3,.4) circle (2pt);
    % Seuil 2
    \draw[ffc,line width=.5] (4.5,\decalageY) -- (4.5,2);
    \draw[fill, RdBu-9-1] (4.5,\decalageY) circle (2pt);
    \draw[fill, RdBu-9-1] (4.5,2) circle (2pt);
    \node[above] at (4.5,2) {\(v\)};
    % Seuil 3
    \draw[ffc,line width=.5] (5.7,\decalageY) -- (5.7,.4);
    \draw[fill, RdBu-9-1] (5.7,\decalageY) circle (2pt);
    \draw[fill, RdBu-9-1] (5.7,.4) circle (2pt);
    
    \draw[|-|] (3,-.7cm) --++(3,0) node[pos=.5, fill=white, inner
    sep=1pt, font=\small] {$\delta$};

    \draw[<->, shorten >=2pt,shorten <=2pt] (2,.4cm) --++ (0,1.6cm) node[pos=.5, fill=white, inner
    sep=1pt, font=\small] {$i$};
  \end{scope}
\end{tikzpicture}

  \caption{sq}
  \label{fig:qs}
\end{figure}





%%% Local Variables:
%%% mode: latex
%%% TeX-master: "../../../../main"
%%% End:
