Comme nous l'expliquions dans le chapitre précédent (voir
\ref{subsec:1-2-3}, page \pageref{subsec:1-2-3}), notre thèse
s'inscrit dans un projet de recherche dont l'objectif est de
développer des solutions d'aide à la localisation de personnes perdues
en montagne. Notre objectif principal consiste quant à lui à élaborer
une solution permettant de délimiter la zone correspondant à une
description orale de position. Pour ce faire nous avons identifié cinq
objectifs scientifiques, présentés ci-dessous.

On trouvera une synthèse des objectifs scientifiques, ainsi que de
leurs verrous et des contributions attendues dans le
\autoref{tab:synthese_objectifs} page
\pageref{tab:synthese_objectifs}.

\subsection{La spatialisation \emph{d'indices de localisation}}
\label{subsec:2-1-1}

% int
Le premier objectif de cette thèse est de développer une méthode
permettant d'identifier une position à partir de sa
description.

% Présentation de l'objectif
Comme nous l'expliquions précédement, notre thèse s'inscrit dans un
axe de recherche du projet Choucas dont l'objectif est de faciliter
les secouristes à identifier une position décrite.
%
Cette tâche nécessite d'arriver à définir des méthodes permettant de
construire une zone à partir de descriptions, telles que \enquote{je
  suis sous une route.} Cette tâche nécessite donc de travailler à
l'identification de la sémantique des termes utilisés pour décrire une
position, puis d'en proposer une modélisation, à même de pouvoir
systématiser la construction d'une \enquote{zone} à partir d'une
description.

% Pas nécessaire de faire de l'analyse du langage naturel
Comme nous le mentionnons dans le chapitre précédent (\todo{mettre res
  du passage où j'explique cet aspect dans le chapitre 1}), notre
objectif n'est pas de déveloper une solution se sustituant aux
secouristes, mais de leur proposer une solution s'intégrant dans leur
processus métier actuel. Cela implique que nous pouvons toujours
compter sur la présence du secouriste pour effectuer certaines
actions. Parmi celles-ci, l'écoute et la compréhension du discours du
requérant. En effet, il n'est pas nécessaire, voire souhaitable
\footnote{Tout du moins du point de vue des professionels du
  secours.}, d'automatiser entièrement le processus de localisation de
la victime. Un tel objectif nécessiterait par ailleurs de développer
des solutions d'analyse automatisée du langage parlé, en vue d'en
extraire les relations de localisation.

Cette étape n'est pas nécessaire dans notre cas, puisque nous pouvons
confier cette tache de \enquote{sémantisation} de la description orale
d'une position au secouriste. 

Notre travail se limitera donc à identifier les \enquote{types} de
descriptions utilisées dans notre contexte, de les exprimer dans un
vocabulaire controlé, puis d'en proposer une modélisation a même d'en
permettre la spatialisation \todo{Attention terme non défini, ajouter
  une phrase de définition}.

\subsubsection{Contexte scientifique}

La question de l'indentification de positions décrites à fait l'objet
de différents travaux, aux visées applicatives diverses.


% Réflexions autour de la sémantique des relations spatiales

Enfin, dans une thématique plus proche de notre travail,  

\emph{Naive geography} \autocite{Egenhofer1995}

Mateh

Moncla



Vasardini
Bloch
Brézil icc

vandeloise, kratch, augenarge, borillio 

\paragraph{La formalisation des \emph{relations de localisation}}
\autocite{Vandeloise1986}
\autocite{Borillo1998}
\autocite{Bateman2010}
\autocite{Kracht2002}
\autocite{Aurnague1993}
\autocite{Mark1999}
\autocite{Freksa2018}
\autocite{Carlson2005}
\autocite{Lang1991}

% vérifier
\autocite{Hois2008}
\autocite{Frank1992}
\autocite{Moratz2008}
\autocite{Cohn2001}
\autocite{Kordjamshidi2012}
\autocite{Matsakis2010}
\autocite{Frank1997}
\autocite{Talmy2005}
\autocite{Freeman1975}
\autocite{Zelinsky-Wibbelt1993}
\autocite{Carlson2004}
\autocite{Gui-Wu2009}

\paragraph{Applications}

De nombreux domaines de recherche utilisent la formalisation de
relations spatiales pour répondre à des problématiques qui leur sont
propres.

\autocite{Xu2007a}
\autocite{Xu2007}
\autocite{Xu2006}
\autocite{Wolter2018}
\autocite{Dittrich2015}
\autocite{Vanegas2011}
\autocite{Du2016}
\autocite{Hornsby2009}
\autocite{Mathet2000}
\autocite{Hudelot2008a}
\autocite{Bloch1996}
\autocite{Hudelot2008}
\autocite{Levit2007}
\autocite{Evans2008}
\autocite{Takemura2012}
\autocite{Shariff1998}
\autocite{Skubic2004}
\autocite{Hall2015}
\autocite{Leopold2015}
\autocite{Denis1997}


Purves "geographic information revieal''

\tdi{livrable 3.2.1}

\subsubsection{Verrous scientifiques}

La mise en place d'une telle méthode de spatialisation se heurte
néanmoins à deux verrous majeurs, l'un intrinsèquement lié à cet
exercice, l'autre dépendant de notre contexte applicatif.

\paragraph{Nature des relations spatiales}

Les victimes peuvent, certes, utiliser les mêmes mots pour décrire
leur position, mais dans un contexte différent, ou en se rapportant à
des objets qui le sont tout autant. On peut être \enquote{sur}
\emph{un chemin} comme \emph{une montagne.}

\paragraph{Données}

\begin{verbatim}
- Peu de données
- Impossibilité de faire de l'apprentissage
- Pas de recoupement sur les positions
- Description sensiblement différentes
- Utilisation variable des mêmes relations spatiales
- Variabilité sémantique des relations spatiales
\end{verbatim}

le second verrou est lié à la richesse des données dont nous disposons
pour développer et tester nos méthodes de spatialisation.

Les seules données dont nous disposons (et pouvons disposer) sont des
enregistrements d'alertes, traitées par le \ac{pghm} de
Grenoble. Chaque enregistrement, contient \emph{l'ensemble des indices
  de localisation} utilisés par le requérant. Un problème notable est
que chacune de ces alertes est unique. Il nous est donc impossible de
les confronter.
% Peu de données et Faible recoupement
Par exemple, nous n'avons pas à notre disposition, deux
enregistrements différents, décrivant la même position. Il nous est
donc impossible de comparer différentes descriptions qui auraient été
faites de la même position. De même, les \emph{indices de
  localisation} utilisés par les requérants sont si divers, que nous
ne connaissons que peu de cas, où le même \emph{indice de
  localisation} (\eg \enquote{je suis dans une forêt}) est utilisé
pour décrire une position différente \footnote{La majorité de ces cas
  concerne \emph{l'indice de localisation :} \enquote{je suis sur
    \textins{un chemin, une route, un sentier, \emph{etc.}}}}. Notre
démarche de spatialisation doit donc être à même de travailler à
partir de descriptions potentiellement très différentes, sans que nous
ne puissions réellement les confronter à d'autres descriptions.

% Descriptions non exhaustives
Un dernier problème notable est que le processus de localisation de la
victime, basé sur l'interprétation par le secouriste d'une description
de position, n'offre aucune garantie quant à la \emph{complétude} de
cette description. La victime peut, pour de nombreuses raisons,
oublier de donner certains détails de sa position qui pourraient être
pertinents pour les secouristes. Ces derniers peuvent, bien entendu,
poser des questions au requérant ou lui demander des précisions, mais
cela ne garantit pas pour autant qu'un \emph{indice} important, voire
essentiel, ne puisse pas être oublié. Cette \emph{incomplétude} de la
description, que l'on peut aisément supposer de systématique, a pour
conséquence de rendre impossible la supposition de l’existence
d'indices, non donnés explicitement par le.a requérant.e. Par exemple,
ce n'est pas parce que la victime ne précise pas qu'elle est en forêt,
qu'elle n'est effectivement pas en forêt, \ie qu'un \emph{indice} non
connu du secouriste n'est pas nécessairement faux. Par conséquent, il
est hasardeux, d'inférer des nouveaux \emph{indices} à partir d'une
\emph{description de position.}

\subsubsection{Apports envisagés}

Le premier des apports que nous proposons est d'identifier les
\emph{relations de localisation} utilisées dans notre contexte
applicatif. On peut, en effet, supposer que seul un sous-ensemble des
\emph{relations de localisation} présentes dans la langue française
est utilisé, du moins régulièrement, pour décrire une position en
montagne. L'identification de ces \emph{relations de localisation}
nous permettra, d'une part, d'identifier les spécificités de la
description d'une position en langage naturel, dans un contexte
montagnard et, de l'autre, de nous concentrer sur la modélisation de
ces \emph{relations de localisation.}

Cet apport sera complété par un travail d'identification de la
sémantique des relations de localisation utilisées dans notre
contexte. Il est, en effet, possible que les \emph{relations de
  localisations} utilisées dans notre contexte, le soient dans un sens
particulier, plus réduit que le sens général. Ainsi, nous allons
compléter le recensement des \emph{relations de localisation}
utilisées dans notre contexte, par un travail d'identification de leur
sémantique, dans ce même contexte.

Ce même travail de recensement sera entrepris pour les \emph{objets de
  référence.} Comme pour les \emph{relations de localisation,} on peut
supposer que, seule une petite partie des objets pouvant être utilisés
comme points de repère, le sont réellement, dans ce contexte. On
s'attendra, par exemple, à trouver des \emph{indices de localisation}
tels que \enquote{je suis au niveau \emph{du sommet}} ou \enquote{la
  victime est dans \emph{une combe}}, alors que la phrase :
\enquote{je suis proche \emph{d'un kiosque}}, semble peu
probable. L'identification et le recensement de ces objets permettra
---~comme pour les \emph{relations de localisation}~--- d'avoir à la
fois une meilleure connaissance des descriptions de positions dans
notre contexte applicatif, mais également de focaliser notre travail
sur les objets (et les \emph{relations de localisation}) les plus
pertinent(e)s.

Enfin, notre apport principal ---~qui nécessite la bonne réalisation
des points précédent~--- à cet objectif sera de développer une méthode
de spatialisation de ces \emph{indices de localisation} et d'en
proposer une implémentation fonctionnelle. La méthode développée devra
être suffisamment générique pour fonctionner avec les différentes
\emph{relations de localisation} et les différents \emph{objets de
  référence} identifié$\cdot$es, tout en prenant en compte leurs
spécificités, dans le but de proposer une solution de spatialisation
précise et adaptée au contexte, mais suffisamment générique pour être
enrichie ultérieurement voire étendue à des contextes différents.

\subsection{La modélisation de \emph{l'imprécision} des \emph{indices
    de localisation}}
\label{subsec:2-1-2}

La question de la spatialisation nécessiter cependant d'en s'en poser
une seconde, celle de la prise en compte de l'imprécision. En effet,
les positions que nous cherchons à spatialiser possèdent la
particularité d'être exprimées en langage naturel. Orn bien que nous
ne travaillons pas directement à l'interprétation du langage naturel,
\ie que nous ene faisons pas de TAL, nous sommes tout de même
confrontés à certaines de ces limites.

La principale d'entre elles est la question de l'imprécison du langage
naturel. En effet, le langage naturel n'est pas un outil idéal pour
exprimer clairement des propositions, celle-ci sont toujours soumisses
à l'interpréatation et donc imprécises. Il en va nécessairement de
même pour les propositions décrivants dune ou des positions? Ces
dernières s'expriment en langage natuurel et son donc soumisses à
l'iumprécisions.

Dans les faits ,cela implique qu'une même phrase, un même mot,
peutvent -être utilisés pour décrire des localisations dans l'espace
différentes, voir même totalement incompatibles.

Cette nature du langage naturel peut poser plusieurs soucis. D'une
part il est impossible d'utiliser la langue comme un outil de
raisonement précis. C'est d'ailleurs ce constat qui a conduit les
philosophes analytiques, tels que Fredge ou Russel à travailler à la
formalisation de la logique, outil à même d'affranchir la réflexion
des limites du langage naturel.



\subsubsection{Contexte scientifique}

La notion d'imprécision a également été appliquée spécifiquement à des
problématiques de spatialisation, comme celle que nous souhaitons
mettre en place ici.



Travail sur le flou
Russell
Varzi
Zadeh
Dilo ?


\subsubsection{Verrous scientifiques}

\begin{verbatim}
- Comment ajouter la prise en compte de l'imprécision au processus de
spatialisation de manière satisfaisante
- Comment estimer/quantifier la précision des relations spatiales
\end{verbatim}

Dans notre cas, la prise en compte de \emph{l'imprécision} des
\emph{relations de localisation} pose deux problèmes majeurs.

\paragraph{L'intégration de \emph{l'imprécision} dans la \emph{spatialisation}}

Le premier d'entre eux est la question de l'intégration de cette
composante au sein du processus de \emph{spatialisation.}

\paragraph{L'évaluation de \emph{l'imprécision}}

Le second verrou est la question de l'évaluation de
\emph{l'imprécision.} En effet, s'il est acquis que les
\emph{relations de localisation} ---~et plus généralement le langage
naturel~--- sont \emph{imprécis,} il reste qu'ils peuvent l'être à des
degrés divers. Pour l'illustrer, nous pouvons utiliser les
\emph{indices de localisation:} \enquote{je suis \emph{proche de}
  Grenoble} et \enquote{je suis \emph{aux alentours de} Grenoble}. Ces
deux indices ont une signification semblable, tous deux renseignent
sur la proximité du locuteur avec la ville de Grenoble. On aurait
toutefois du mal à les considérer comme parfaitement équivalents, la
relation de localisation \enquote{\emph{aux alentours de}} nous
semblant plus vague, \ie que l'aire que l'on peut définir comme étant
\enquote{\emph{aux alentours de} Grenoble} est plus étendue que celle
qui en est \emph{proche.} On ne peut pas, pour autant, considérer que
la \emph{relation de localisation} \enquote{\emph{proche de}} est
\emph{précise.} Ces deux relations sont \emph{imprécises,} mais a des
degrés divers et si l'on peut les ordonner selon leur \enquote{degré
  \emph{d'imprécision}}, on ne saurait quantifier cet écart et donc
leurs différents degrés \emph{d'imprécision.}

Cette quantification est cependant une étape indispensable à la prise
en compte de \emph{l'imprécision} des \emph{relations de
  localisation.} L'analyse de la sémantique des \emph{relations de
  localisation} devra donc être complétée par une estimation de leur
\emph{imprécision} et leur modélisation devra prendre en considération
ces deux aspects.

\subsubsection{Apports envisagés}

\begin{verbatim}
- Sélection d'une méthode de méthode de prise en compte de
l'imprécision
- Définition d'une méthode de quantification de l'imprécision
\end{verbatim}

\subsection{La modélisation de \emph{la certitude} des secouristes}
\label{subsec:2-1-3}

Si la prise en compte de \emph{l'imprécision} des descriptions de
localisation (obj. \ref{subsec:2-1-1}) permet d'améliorer la qualité
de leur modélisation, ce mécanisme ne permet pas d'en gérer la
plausiblité. En effet, certains \emph{indices de localisation} donnés
par le requérant peuvent être faux, ou, du moins, sujets à caution et
ce indépendament de toute notion de \emph{précision.} La prise en
considération ---~aveugle~--- de ce type d'indices ne peut qu'impacter
négativement la qualité de la \emph{zone de localisation probable}
créée par \emph{l'agrégation} des \emph{indices de localisation.} Les
secouristes sont cependant, grace à leur connaissance du terrain et à
leur expérience, à même d'identifier la plupart de ces erreurs et
approximations, ou, tout du moins, d’émettre un doute sur la véracité
de certains \emph{indices de localisation.} La prise en compte de ces
connaissances, exogènes à l'indice, nous semble très importante, en
plus d'être un excellent moyen pour controler la prise en
considération d'indices de mauvaise qualitée.

Un élément central du projet Choucas étant de développer des solutions
d'assistance et non d'automatisation, nous pouvons prévoir des
interactions avec le secouriste, notamment l'évaluation de la
plausibilité des \emph{indices de localisation} donnés par le
requérant. La prise en compte de la plausibilité des indices de
localisation ajoute donc une nouvelle dimension à la spatialisation
des \emph{indices de localisation.}

\subsubsection{Contexte scientifique}

Travaux sur l'incertitude

Ana-Maria

\subsubsection{Verrous scientifiques}

\begin{verbatim}
- Sélection d'une méthode de méthode de prise en compte de
l'incertitude
- Permettre au secouriste de manipuler cette information de manière
'naturelle' 
\end{verbatim}


\paragraph{Cohabitation avec la méthodologie de spatialisation}

\paragraph{inscription dans le contexte métier}

L'évaluation de la plausibilité des \emph{indices de localisation}
étant déléguée aux secouristes, il est nécessaire que la solution
développée soit a même de représenter l'opinion du secouriste.

% Le secouriste doir pouvoir manipuler assez 'facilement' ces notions

\subsubsection{Apports envisagés}

\begin{verbatim}
- Sélection de la méthode
- Adaptaion de la méthode au contexte métier
\end{verbatim}


\subsection{La fusion des \emph{indices de localisation}}
\label{subsec:2-1-4}

La structure des alertes fait apparaitre un autre problème. En effet,
comme nous l'avon vu lors de présentation du fil rouge (\ref{} page
\pageref{}) \todo{ajouter ref présentation du fil rouge} les
requérants décrivent généralement leur position à l'aide de
différentes phrases, décrivant autant d'aspects de leur position
actuelle. Chacune de ces descriptions, constitue, ce que nous appelons
un \emph{indice de localisation.} Une description de localisation est
donc composée d'au moins un, généralement plus, \emph{indice de
  localisation}. Ce sont ces derniers que nous cherchons à
\emph{spatialiser.} Toutefois, la spatialisation indivoduelle de ces
indices de localisation, ne donne qu'une indication partielle sur la
zone de localisation de la vicitime. Il est donc nécessaire de
combiner la spatilisation des différents indices de localisation, afin
de proposer la meilleure \emph{zone de localisation compatible}
possible.

La méthode développée à cet effet, doit conserver les caractéristiques
que nous voulons voir validées par la méthode de spatilisation
présentée ci-dessus. C'est-à-dire, que la méthode d'aggrégation doit
prendre en compte la nature imprécise des résultats de la méthode de
spatilisation des indices. De plus, elle doit être a-même de prendre
en compte l'évaluation de la plausibilité de l'indice faite par le
secouriste.

\subsubsection{Contexte scientifique}

\begin{verbatim}
- Présentation des théories
- Thèse AM
- Application
- Chaque théorie à ses propres méthodes
\end{verbatim}

\subsubsection{Verrous scientifiques}

Le processus d'agrégation des indices que nous souhaitons développer
correspond à un stade éloigné de notre méthodologie, \ie qu'il est
impacté par les choix effectués en amont. C'est notamment le cas de
ceux portant sur la spatilisation des différents indices.

\paragraph{Compatibilité avec les méthodes de spatialisation}

Comme nous l'expliquions lors de la présentation des objectifs
\ref{subsec:2-1-2} et \ref{subsec:2-1-3}, nous souhaitons que la prise
en compte de \emph{l'imprécision du langage naturel} et de la
\emph{plausibilité des indices} soit intrinsèque à la méthode de
spatialisation des \emph{indices de localisation} (objectif
\ref{subsec:2-1-1}). Il est par conséquent nécessaire que la méthode
d'agrégation des indices soit à même de conserver ces caractérisques.
% 
Or, ces deux composantes sont, par définition, variables. Deux indices
issus d'une même alerte peuvent avoir une plausibilité considérée
comme différente par le secouriste. Ce peut être, par exemple, le cas
si un requérant décrit sa position comme étant : \enquote{proche d'un
  chalet et probablement au sud du sommet}. Dans ce cas le premier
indice de localisation : \enquote{je suis proche d'un chalet}, serait
surement, considéré comme \emph{plus plausible} que le second :
\enquote{je suis probablement au sud sommet}. La méthode d'agrégation
des indices de localisation doit donc être capable de prendre en
considération cette différence entre les deux indices, par exemple les
pondérant. De même, la différence de précision entre deux indices doit
être prise en compte lors du processus d'agrégation.

\paragraph{\texttt{titre ??}}
%\paragraph{Contexte différent}

% La commutativité de l'agrégation
% Différence des indices
% Traitement 'indépendant' de la localisation

Un autre point important est que le processus d'agrégation sera
potentiellement utilisé plusieurs fois lors du traitement d'une
alerte. En effet, les \emph{indices de localisation} sont donnés par
le requérant, au fur et à mesure du processus de localisation, parfois
en répose aux questions des secouristes. L'ensemble des indices de
localisation modélisés est donc amené à évoluer, comme la \emph{zone
  de localisation compatible} qui résulte de leur agrégation. La
méthode d’agrégation doit donc prendre cet aspect en considération et
nous devons veiller à ce que les résultats de ce processus ne soient
pas sensibles

\tdi{revoir}

\paragraph{Sélection des opérateurs de fusion}

\tdi{revoir}

\paragraph{La gestion des conflits}

\tdi{C'est pertinent ? Il n'y a pas vraiment de solution dans ma
  méthodo pour les prendre en compte.}

Un second verrou, fortement lié à la question de la modélisation de la
plausibilité est celui de la gestion des conflits entre indices. Il
n'est, en effet, pas exclu que deux indices d'une même alerte se
contredisent, rendant dès lors impossible la construction de la
\emph{zone de localisation probable.} La prise en compte de la
plausibilité des indices permet, en partie, de contourner ce
problème. Cependant, l'estimation de la plausibilité des indices reste
à la charge du secouriste.

\subsubsection{Apports envisagés}

Notre apport principal pour répondre à cet objectif scientifique sera
de développer une méthode agrégeant les résultats de la
\emph{spatialisation} des différents \emph{indices de localisation}
pour construire une \emph{zone de localisation compatible.}  Cette
méthode sera implémentée à la suite de la méthode de
\emph{spatialisation.} L'implémentation des méthodes de
\emph{spatialisation} et \emph{d'agrégation,} permettra d'appliquer
l'ensemble de notre méthodologie à la résolution de cas réels.

\subsection{L'évaluation de la \emph{zone de localisation probable}}
\label{subsec:2-1-5}

Enfin, notre dernier objectif est de proposer une méthode d'évaluation
de la (ou des) \emph{zone·s de localisation probable} \todo{Veiller à
  ce que le terme soit bien défini auparavant} construites. Cet
objectif nécessite définir des critères d'évaluation pertinants et
adaptés au contexye appliquatif de ce travail.


\subsubsection{Contexte scientifique}

\begin{verbatim}
- Aucune idée de quoi mettre ici
- Tests utilisateurs thèses Cogit, Marion
- Supervisé vs non supervisé
\end{verbatim}

\subsubsection{Verrous scientifiques}

Le principal verrou scientifique à la mise en place d'une méthode
d'évaluation, réside dans le fait qu'il est insatisfaisant de nous
baser uniquement sur une comparaison entre une \enquote{zone estimée},
\ie la \emph{zone de localisation probable} et la position réele de la
victime. En effet, l'absence ---~et inversement la présence~--- de la
position de la victime dans la \emph{zone de localisation probable}
construite, ne signifie pas nécessairement que la modélisation est
erronée. Les \emph{indices de localisation} peuvent être incorrects,
sans pour autant avoir étés identifiés comme tels par le secouriste,
ou le requérant peut se référer à un objet qui n'existe pas dans les
basses de données utilisées. De plus, avec ce critère, l'évaluation ne
peut se faire sans la connaissance de la position réelle de la
victime, ce qui ne permet pas d'envisager la mise en place d'une
\enquote{évaluation dynamique} des résultats, à même de renseigner le
secouriste sur la qualité des résultats lors de la phase de
localisation. L'évalution des résultats ne pouvant être réalisée
qu'après avoir pris connaissance de la position de la victime. Une
autre solution serait de considérer que, moins la \emph{zone de
  localisation probable} est étendue, plus elle est précise et donc de
\enquote{meilleure qualitée}. Cependant cette approche n'est pas plus
satisfaisante.  En effet, la \emph{zone de localisation probable}
correspond à l'ensemble des \emph{positions} validant la descrition
---~transmise sous la forme \emph{d'indices de localisation}~---
donnée par la victime. Or, plusieurs zones peuvent correspondre à
cette descrition, à fortiori si les \emph{indices de localisations}
donnés sont peu nombreux ou discriminants. Utiliser un critère de
\enquote{taille de zone} reviendrait donc à estimer la qualité des
\emph{indices de localisation} et non celle de leur modélisation.

L’évaluation de la qualité de la \emph{zone de localisation probable}
impose donc de développer une méthode à même d'estimer la qualité de
la modélisation des \emph{indices de localisation} donnés par le
requérant et non celle des \emph{indices} en eux-mêmes. 

\subsubsection{Apports envisagés}

Nous souhaitons proposer une méthode d'évaluation de la qualité de la
\emph{zone de localisation probable} qui ne soit pas influencée par la
qualité intrinsèque des \emph{indices de localisation} donnés par le
secouriste. De plus, nous souhaitons que ce processus d'évaluation
puisse ce faire durant la phase de localisation de la victime, pour
apporter aux secouristes une information supplémentaire, facilitant la
résolution de l'alerte.

\begin{landscape}
\begin{table}
  \centering
  \begin{tabular}{p{.2\textheight}>{\small}L{.35\textheight}>{\small}L{.35\textheight}} \toprule
\multicolumn{1}{c}{\bfseries Objectif scientifique} &
\multicolumn{1}{c}{\normalsize\bfseries Verrous} & \multicolumn{1}{c}{\normalsize\bfseries
Apports envisagés} \\ \midrule
% Sémantique des relations spatialesss
  \addlinespace
  \nameref{subsec:2-1-1}
{\par\footnotesize\hspace{.25cm}$\longrightarrow$~Chapitre
\ref{chap:07}} & \begin{minipage}[t]{.35\textheight}
    \begin{itemize}
    \item Variation sémantique des \emph{relations de localisation}
    \item Faible redondance des \emph{indices de localisation}
    \end{itemize}
  \end{minipage} & \begin{minipage}[t]{.35\textheight}
    \begin{itemize}
    \item Recensement :
      \begin{itemize}
      \item des \emph{relations de localisation}
      \item des \emph{objets de référence}
      \end{itemize}
    \item Identification de la sémantique des \emph{relations de
        localisation}
    \item Définition d'une méthode de \emph{spatialisation}
    \end{itemize}
  \end{minipage} \\
  %\addlinespace[.5cm]
  %
  % Prise en compte imprécision
  \nameref{subsec:2-1-2}
{\par\footnotesize\hspace{.25cm}$\longrightarrow$~Chapitre
\ref{chap:07}} & \begin{minipage}[t]{.35\textheight}
    \begin{itemize}
    \item Intégration de \emph{l'imprécision} à la \emph{spatialisation}
    \item Quantification de \emph{l'imprécision} des \emph{relations
        de localisation}
    \end{itemize}
  \end{minipage} & \begin{minipage}[t]{.35\textheight}
    \begin{itemize}
    \item Définition d'une méthodes :
      \begin{itemize}
      \item de prise en compte de \emph{l’imprécision} des
        \emph{relations de localisation}
      \item de quantification de \emph{l'imprécision} des
        \emph{relations de localisation}
      \end{itemize}
    \end{itemize}
  \end{minipage} \\
  %\addlinespace[.5cm]
  %
  % Incertitude
  \nameref{subsec:2-1-3}
{\par\footnotesize\hspace{.25cm}$\longrightarrow$~Chapitre
\ref{chap:08}} & \begin{minipage}[t]{.35\textheight}
    \begin{itemize}
    \item Intégration de \emph{l'incertitude} à la
      \emph{spatialisation}
    \item Quantification de \emph{l'incertitude} des \emph{indices de
        localisation}
    \end{itemize}
  \end{minipage}& \begin{minipage}[t]{.35\textheight}
    \begin{itemize}
    \item Définition de méthodes :
      \begin{itemize}
      \item de prise en compte de \emph{l’incertitude} des
        \emph{indices de localisation}
      \item de quantification de \emph{l'incertitude} des
        \emph{indices de localisation}
      \end{itemize}
    \end{itemize}
  \end{minipage} \\
  %\addlinespace[.5cm]
  %
   \nameref{subsec:2-1-4}
{\par\footnotesize\hspace{.25cm}$\longrightarrow$~Chapitre
\ref{chap:08}} & \begin{minipage}[t]{.35\textheight}
    \begin{itemize}
    \item Compatibilité avec la spatialisation
    \item Gestion des conflits
    \end{itemize}
  \end{minipage}& \begin{minipage}[t]{.35\textheight}
    \begin{itemize}
    \item Définition d'une méthode de fusion
    \end{itemize}
  \end{minipage} \\
  %\addlinespace[.5cm]
  %
  \nameref{subsec:2-1-5}
{\par\footnotesize\hspace{.25cm}$\longrightarrow$~Chapitre
\ref{chap:11}} & \begin{minipage}[t]{.35\textheight} \small
    \begin{itemize}
    \item Risque d'évaluer les \emph{indices}.
    \item Intégration dans le contexte métier.
    \end{itemize}
  \end{minipage}& \begin{minipage}[t]{.35\textheight}
    \begin{itemize}
    \item Développement d'une méthode d'évaluation des \emph{zones de
        localisation}
    \end{itemize}
  \end{minipage}\\
  \bottomrule
\end{tabular}

  \caption{Synthèse des verrous et des apports attendus pour chaque
    objectif scientifique de la thèse}
  \label{tab:synthese_objectifs}
\end{table}
\end{landscape}


%%% Local Variables:
%%% mode: latex
%%% TeX-master: "../../../../main"
%%% End:
