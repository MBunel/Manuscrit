\begin{tikzpicture}
  \def\decalageX{-.2}
  \def\decalageY{-.2}
  % Arrow
  \begin{scope}
    \path[draw, -, shorten >=12pt, shorten <=5pt] (1,2) |- (3.65,3);
    \path[draw, ->,shorten >=5pt, shorten <=5pt] (8.25,3) -- (10,3);
    \path[draw, -,shorten >=12pt, shorten <=5pt] (1,0) |- (3.65,-1);
    \path[draw, ->,shorten >=5pt, shorten <=5pt] (8.25,-1) -- (10,-1);
  \end{scope}
  % Représentation métrique
  \begin{scope}[local bounding box=met]
    % \path[ffa] (1,.25) rectangle (1.25, .5);
    \begin{scope}
      \foreach \x in {0,.25,...,1.75}{
        \foreach \y in {0,.25,...,1.75}
        {
          % Traçage pixels 
          %\path[draw, line width=.01mm] (\x,\y) rectangle (\x +.25, \y +
          %.25);
          % Calcul et représentation distance euclidienne
          \pgfmathsetmacro\radius{sqrt((\x-1.125)^2+(\y-.375)^2)*1.25}
          \fill[black] (\x+.125,.125+\y) circle (\radius pt);
        }
      }
      \path[ffc, black] (0,0) rectangle (2,2);
    \end{scope}
    \node[text width=3cm, align=center, anchor=north, font=\footnotesize,fill=white] at (1,-.25)
    {\itshape Métrique};
  \end{scope}
  % Fuzzyfication 1
  \begin{scope}[xshift=10cm, yshift=2cm,local bounding box=fuzz]
    \begin{scope}
      \foreach \x in {0,.25,...,1.75}{
        \foreach \y in {0,.25,...,1.75}
        {
          % Traçage pixels 
          %\path[draw, line width=.01mm] (\x,\y) rectangle (\x +.25, \y +
          %.25);
          % Calcul et représentation floue distance euclidienne
          \pgfmathsetmacro\dist{sqrt((\x-1.125)^2+(\y-.375)^2)}
          % Fuzzyfication de la distance
          \pgfmathsetmacro\fuzzy{%
            ifthenelse(\dist < .25,1,%
            ifthenelse(\dist < 2,-0.57*\dist+1.14,0)%
            )
          }
          % Calcul du rayon à partir de la fuzzyfication
          \pgfmathsetmacro\radius{\fuzzy*2.5}
          \fill[RdBu-9-1] (\x+.125,.125+\y) circle (\radius pt);
        }
      }
      \path[ffc] (0,0) rectangle (2,2);
    \end{scope}
    \node[text width=3cm, align=center, anchor=north, font=\footnotesize] at (1,-.25)
    {\itshape Zone de localisation compatible \normalfont \textcolor{RdBu-9-1}{\textsf{A}}};
  \end{scope} 
  % Fuzzyfication 2
  \begin{scope}[xshift=10cm, yshift=-2cm,local bounding box=fuzz2]
    \begin{scope}
      \foreach \x in {0,.25,...,1.75}{
        \foreach \y in {0,.25,...,1.75}
        {
          % Traçage pixels 
          %\path[draw, line width=.01mm] (\x,\y) rectangle (\x +.25, \y +
          %.25);
          % Calcul et représentation floue distance euclidienne
          \pgfmathsetmacro\dist{sqrt((\x-1.125)^2+(\y-.375)^2)}
          % Fuzzyfication de la distance
          \pgfmathsetmacro\fuzzy{%
            ifthenelse(\dist < .75,0,%
            ifthenelse(\dist < 1.75,\dist-.75,1)%
            )
          }
          % Calcul du rayon à partir de la fuzzyfication
          \pgfmathsetmacro\radius{\fuzzy*2.5}
          \fill[RdBu-9-9] (\x+.125,.125+\y) circle (\radius pt);
        }
      }
      \path[ffc2] (0,0) rectangle (2,2);
    \end{scope}
    \node[text width=3cm, align=center, anchor=north, font=\footnotesize] at (1,-.25)
    {\itshape Zone de localisation compatible \normalfont \textcolor{RdBu-9-9}{\textsf{B}}};
  \end{scope}
  \begin{scope}[xshift=3.75cm,yshift=2.5cm, scale=.5]
    % Courbe
    \begin{scope}[transparency group]
      % fond
      \begin{scope}
        \path[ffa]  (0,2) -- (1, 2)  -- (8,0) -- (0,0) -- cycle;
      \end{scope}
      % bords
      \begin{scope}
        \path[ffc] (0,2) -- (1,2) -- (8,0);
        \path[ffc_fade] (8,0) -- (9,0) ;
      \end{scope}
    \end{scope}
    % Axes X, Y
    \begin{scope}
      % Axe X
      \begin{scope}
        % Axe
        \draw[->] (0, \decalageX) --++ (9, 0) coordinate (x axis);
        % Graduations
        \foreach \n/\t in {0/,{0},1/{0,25},2/{},3/{},4/{},5/{},6/{},7/{},8/{}}
        {
          \draw[-] (\n, \decalageX - .05) --++ (0, .1);
          \node[below, font=\tiny] at (\n, \decalageX - .05) {\t};
        }
        % label
        \node[below left, font=\tiny] at (x axis) {\itshape Distance
          \normalfont (cm)};
      \end{scope}
      % Axe Y
      \begin{scope}
        % Axe
        \draw[-] (\decalageY ,0) --++ (0, 2) coordinate (y axis);
        % Graduations
        \foreach \n/\t in {0/{0},2/{1}}
        {
          \draw[-] (\decalageY -.05, \n) --++ (.1, 0);
          \node[left, font=\tiny] at (\decalageY -.05, \n) {\t};
        }
        % Label
        \node[above, font=\tiny] at (y axis) {$\mu$};
      \end{scope}
    \end{scope}
  \end{scope}
  \begin{scope}[xshift=3.75cm,yshift=-1.5cm, scale=.5]
    % Courbe
    \begin{scope}[transparency group]
      % fond
      \begin{scope}
        \path[ffa2]  (3,0) -- (6, 2)  -- (8,2) -- (8,0) -- cycle;
        \path[ffa2_fade]  (8,2) -- (9, 2)  -- (9,0) -- (8,0) -- cycle;
      \end{scope}
      % bords
      \begin{scope}
        \path[ffc2] (0,0) -- (3,0) -- (6, 2) -++ (3,0) ;
        \path[ffc2_fade] (8,2) -- (9,2) ;
      \end{scope}
    \end{scope}
    % Axes X, Y
    \begin{scope}
      % Axe X
      \begin{scope}
        % Axe
        \draw[->] (0, \decalageX) --++ (9, 0) coordinate (x axis);
        % Graduations
        \foreach \n/\t in {0/,{0},1/{0.25},2/{},3/{},4/{},5/{},6/{},7/{},8/{}}
        {
          \draw[-] (\n, \decalageX - .05) --++ (0, .1);
          \node[below, font=\tiny] at (\n, \decalageX - .05) {\t};
        }
        % label
        \node[below left, font=\tiny] at (x axis) {\itshape Distance
          \normalfont (cm)};
      \end{scope}
      % Axe Y
      \begin{scope}
        % Axe
        \draw[-] (\decalageY ,0) --++ (0, 2) coordinate (y axis);
        % Graduations
        \foreach \n/\t in {0/{0},2/{1}}
        {
          \draw[-] (\decalageY -.05, \n) --++ (.1, 0);
          \node[left, font=\tiny] at (\decalageY -.05, \n) {\t};
        }
        % Label
        \node[above, font=\tiny] at (y axis) {$\mu$};
      \end{scope}
    \end{scope}
  \end{scope}
\end{tikzpicture}