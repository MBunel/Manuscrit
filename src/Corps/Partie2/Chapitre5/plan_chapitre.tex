% Citation début de chapitre
% \dictum[Aaron Swartz]{\enquote{Ils ont ainsi consacré des heures à des
%     débats excessivement pointilleux pour savoir si une machine à
%     laver était un appareil ménager ou un équipement de nettoyage
%     ménager.}}%


\chaptertoc{}

\addsec{Introduction}
\label{sec:5-int}
\subimport{./}{introduction}

\section{La décomposition des \emph{relations de localisation}}
\label{sec:5-1}
\subimport{Section1/}{section}

\section{Définition du contenu de l'ontologie des relations de
  décomposition}
\label{sec:5-2}
\subimport{Section2/}{section}

\section{L'ontologie des objets de référence}

\tdi{Finir}

Les ontologies \ac{orl} et \ac{orla} sont complétées par une troisième
ontologie : l'ontologie des objets repères \acp{oor}.


\addsec{Conclusion}
\label{sec:5-cnc}
\subimport{./}{conclusion}


%%% Local Variables:
%%% mode: latex
%%% TeX-master: "../../../main"
%%% End:
