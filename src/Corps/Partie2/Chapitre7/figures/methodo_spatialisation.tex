\begin{tikzpicture}
  % Représentation objet de référence
  \begin{scope}[local bounding box=obr]
    \fill[ffa] (1.125,.375) circle [radius=3pt];
    \path[ffc] (1.125,.375) circle [radius=3pt];
    \path[ffc] (0,0) rectangle (2,2);
    \node[text width=3cm, align=center, anchor=north, font=\footnotesize] at (1,-.25)
    {\itshape Objet de référence};
    \node[align=left, anchor=south west, text width=2cm, font=\tiny] (zir) at (1.5,2.25)
    {\itshape Zone initiale de recherche};
    \path[draw, ->,shorten >=3pt] (zir.west) -| (1,2);
  \end{scope}
  % Représentation objet de référence rasterisé
  \begin{scope}[xshift=4.25cm]
    \begin{scope}
      \foreach \x in {0,.25,...,1.75}{
        \foreach \y in {0,.25,...,1.75}
        \path[draw, line width=.01mm] (\x,\y) rectangle (\x +.25, \y + .25);
      }
      % Pixels rasterisation
      \path[ffa] (1,.25) rectangle (1.25, .5);
      % Limite zir
      \path[ffc] (0,0) rectangle (2,2);
    \end{scope}
    \node[text width=3cm, align=center, anchor=north, font=\footnotesize] at (1,-.25)
    {\itshape Objet de référence rastérisé};
  \end{scope}
  % Représentation métrique
  \begin{scope}[xshift=8.5cm,local bounding box=met]
    % \path[ffa] (1,.25) rectangle (1.25, .5);
    \begin{scope}
      \foreach \x in {0,.25,...,1.75}{
        \foreach \y in {0,.25,...,1.75}
        {
          % Traçage pixels 
          \path[draw, line width=.01mm] (\x,\y) rectangle (\x +.25, \y +
          .25);
          % Calcul et représentation distance euclidienne
          \pgfmathsetmacro\radius{sqrt((\x-1.125)^2+(\y-.375)^2)*1.25}
          \fill[black] (\x+.125,.125+\y) circle (\radius pt);
        }
      }
      \path[ffc] (0,0) rectangle (2,2);
    \end{scope}
    \node[text width=3cm, align=center, anchor=north, font=\footnotesize] at (1,-.25)
    {\itshape Métrique};
  \end{scope}
  % Représentation floue
  \begin{scope}[xshift=12.75cm,local bounding box=fuzz]
    \begin{scope}
      \foreach \x in {0,.25,...,1.75}{
        \foreach \y in {0,.25,...,1.75}
        {
          % Traçage pixels 
          \path[draw, line width=.01mm] (\x,\y) rectangle (\x +.25, \y +
          .25);
          % Calcul et représentation floue distance euclidienne
          \pgfmathsetmacro\dist{sqrt((\x-1.125)^2+(\y-.375)^2)}
          % Fuzzyfication de la distance
          \pgfmathsetmacro\fuzzy{%
            ifthenelse(\dist < .25,1,%
            ifthenelse(\dist < 2,-0.57*\dist+1.14,0)%
            )
          }
          % Calcul du rayon à partir de la fuzzyfication
          \pgfmathsetmacro\radius{\fuzzy*2.5}
          \fill[black] (\x+.125,.125+\y) circle (\radius pt);
        }
      }
      \path[ffc] (0,0) rectangle (2,2);
    \end{scope}
    \node[text width=3cm, align=center, anchor=north, font=\footnotesize] at (1,-.25)
    {\itshape Zone de localisation compatible};
  \end{scope}
  % Arrow
  \begin{scope}
    \path[draw, ->] (2.5,1) --++ (1.25,0)  node[pos=.5, above, font=\footnotesize]
    {\itshape rasterisation};
    \path[draw, ->] (6.75,1) --++ (1.25,0)  node[pos=.5, above, font=\footnotesize]
    {\itshape rasterisation};
    \path[draw, ->] (11,1) --++ (1.25,0)  node[pos=.5, above, font=\footnotesize]
    {\itshape fuzzyfication};
  \end{scope}
\end{tikzpicture}