
La construction d'une zone à partir de sa description 



% Expliciter les objectifs scientifiques
Durant le premier chapitre de ce manuscrit nous avons fait emploi de
nombreux termes, dans une acceptation non scientifique, 

Les termes de \emph{description de position} et de \emph{zone,}
utilisés jusqu'ici, l'étaient, volontairement, dans un sens assez
large.


Nous avons été, jusqu'ici, assez évasifs quant à la nature de ce que
nous qualifions d'alertes ou d'indices de localisation.

% Alertes
Le terme \enquote{alerte} que nous avons abondamment employé depuis le
début de ce manuscrit, désigne à la fois …
%
Dans son acceptation formelle, telle que formulé dans
\emph{l’ontologie d'alerte Choucas} \autocite[\ac{oac},][]{Viry2019},
le terme \emph{alerte} désigne l'ensemble des \emph{indices de
  localisation} à la disposition des secours pour identifier la
position d'une victime donnée. Une alerte est également caractérisée
par son contexte, \ie l'événement particulier dont elle traite.
%
C'est en traitant ces alertes que nous construisons la \emph{zone de
localisation} qui y correspond
%
% Indices de localisation
Les \emph{indices de localisation} correspondent, quant à eux, à
l'ensemble des éléments donnés par le requérant pour décrire une
position. Par exemple \enquote{je suis sous une route} ou \enquote{il
  était à La Grave} sont deux \emph{indices de localisation,} bien
qu'ils soient très différents. En effet, le premier exemple décrit la
position actuelle du locuteur (\enquote{je}), alors que le second
décrit la position d'un tiers (\enquote{il}) à un instant révolu. On
pourrait, dès lors, penser que le second exemple n'est pas
représentatif des \emph{indices de localisation} apparaissant dans les
\emph{alertes} traitées. Ce n'est cependant pas le cas, la position
décrite par le locuteur n'est pas nécessairement la sienne, par
exemple lorsque le requérant est un tiers décrivant la position de la
victime. De plus des \emph{indices de localisation} décrivant des
positions passées peuvent également exister, comme le montre le cas du
\emph{fil rouge} (\eg \enquote{La victime est partie de
  Bourg-d'Oisans}, cf. \autoref{subsec:1-1-2-3}). Ces exemples
pourraient donc apparaitre dans une \emph{alerte} réelle. Comme le
laisse supposer sa définition, une \emph{alerte} est composée d'un
\footnote{\label{note:cas_1_indice}Cas que nous n'avons jamais
  rencontré.} ou plusieurs \emph{indices de localisation,} le cas
\emph{fil rouge} en contient, par exemple, une dizaine.
% 
% Relations de localisation
Les \emph{indices de localisation} sont eux-mêmes des objets composés
de plusieurs éléments, un \emph{sujet} (\enquote{je}, \enquote{la
  maison}), qui est l'élément dont la position est décrite ; un
\emph{objet de référence} (\enquote{la maison}, \enquote{moi}), qui
est l'élément a partir duquel est définie la position du \emph{sujet}
et une \emph{relation de localisation} (\enquote{en face},
\enquote{sous}) qui définit la relation entre \emph{sujet} et
\emph{objet de référence.} C'est en interprétant ces trois éléments
que l'on peut construire la \emph{zone de localisation} correspondant
à une description de position, processus que nous nommons
\emph{spatialisation.}

Ainsi, la construction de la \emph{zone de localisation} correspondant
à une description de position ne peut se faire directement, il est
nécessaire de procéder en deux étapes, la première destinée à
\emph{spatialiser} les différents \emph{indices de localisation} et la
seconde combinant les résultats des différentes \emph{spatialisations}
pour aboutir à la \emph{zone de localisation} finale, \ie la
\emph{fusion} des résultats de la spatialisation. On est donc amenées,
pour une même alerte, à définir plusieurs \footnote{À l'exception du
  cas, déjà mentionné, où le requérant ne donne qu'un seul
  \emph{indice de localisation.}  Toutefois, c'est un cas que l'on
  rencontre pas en pratique (cf. \autoref{note:cas_1_indice}).}
\emph{zones de localisation,} une pour chaque \emph{indice de
  localisation,} auxquelles s'ajoute une dernière zone, construite par
la \emph{fusion} des résultats de la \emph{spatialisation.}
L'ontologie \ac{oac} \autocite{Viry2019} fait donc la distinction
entre les \emph{zones de localisation compatibles,} construites par la
spatialisation des \emph{indices de localisation} et la \emph{zone de
  localisation probable,} qui résulte de leur \emph{fusion.}

% Spatialisation
Le processus de \emph{spatialisation} d'un \emph{indice de
  localisation} est illustré par la figure \ref{fig:obj_spa}. Dans cet
exemple, \emph{l'indice de localisation} \enquote{je suis proche d'une
  maison} est \emph{spatialisé} à l'aide d'une zone tampon. On
considère que la zone \enquote{proche d'une maison} correspond à
toutes positions situées à moins d'une distance fixée de \emph{l'objet
  de référence}, la zone résultante est donc un disque. Cette
\emph{spatialisation} est simple, voire simpliste et il ne faut pas
voir cette illustration comme une représentation de la méthode qui
sera effectivement utilisée pour \emph{spatialiser} cet indice, mais
plutôt comme une illustration du processus de \emph{spatialisation.}

\begin{figure}[hb]
  \centering
  \begin{tikzpicture}

\node[text width=4cm, align=center] (0,0) {\enquote{Je suis proche d'une maison}};

\path[draw, ->] (2,0) --++ (2.5,0)  node[pos=.5, above] {\footnotesize \itshape spatialisation};

\begin{scope}[xshift=6.5cm]
\path[ffa] (0,0) circle [radius=30pt];
\path[ffc] (0,0) circle [radius=30pt];

\node[circle, inner sep=0pt,minimum size=4pt, fill] (c) at (0,0) {};
\node[circle, inner sep=0pt,minimum size=8pt] (c2) at (0,0) {};
\node[anchor=west] (m) at  (1.75,1) {\footnotesize \itshape objet de référence};
\path[draw, ->] (m.west) -- (c2);

\node[anchor=west, text width=3cm] (m2) at  (1.75,0) {\footnotesize \itshape zone de localisation compatible};
\path[draw, ->] (m2.west) --++ (-.5,0);
\end{scope}
\end{tikzpicture}
  \caption{Illustration du processus de \emph{spatialisation} d'un
    \emph{indice de localisation.}}
  \label{fig:obj_spa}
\end{figure}

% Fusion
\emph{La spatialisation} est une étape qui doit être répétée pour tous
les \emph{indices de localisation} donnés par le requérant, on dispose
ainsi d'autant de \emph{zones de localisation compatibles} que
d'indices. Chacun d'entre eux donne une information sur la position de
la victime, qui, si l'on exclut le cas où certains indices sont faux,
se situe dans une zone où tous les \emph{indices de localisation} sont
vérifiés, \ie que la \emph{zone de localisation probable} est située à
l'intersection de toutes les \emph{zones de localisation compatibles.}
%
% Exemple
Prenons pour exemple une version simplifiée du \emph{fil rouge}
(cf. \autoref{subsec:1-1-2-3}), composée de seulement deux
\emph{indices:} \enquote{la victime voit une partie de plan d'eau} et
\enquote{elle vient de passer du soleil à l'ombre.}  La \emph{zone de
  localisation probable} correspondant à cette description est, à la
fois une zone à partir de laquelle on peut apercevoir une partie de
plan d'eau (\ie que le premier \emph{indice de localisation} est
validé), mais aussi une zone qui vient de passer du soleil à l'ombre
(\ie le second \emph{indice de localisation} est validé). Par
conséquent la \emph{zone de localisation probable} correspond à la
zone validant les deux \emph{indices de localisation} de
\emph{l'alerte}, \ie l'intersection des deux \emph{zones de
  localisation compatibles} leur correspondant, comme l'illustre la
figure \ref{fig:obj_fus}. Deux \emph{zones de localisation
  compatibles} y sont représentées, lesquelles sont produites par le
processus de \emph{spatialisation} décrit par la figure
\ref{fig:obj_spa}. La méthode de fusion combine ces deux zones pour en
construire une seule, la \emph{zone de localisation probable,}
correspondant à l'intersection des deux \emph{zones de localisation
  compatibles}. Toutefois, comme pour la figure \ref{fig:obj_spa},
cette représentation est grandement simplifiée, notamment car les deux
\emph{zones de localisation compatibles} sont similaires et partagent
le même \emph{objet de référence,} ce qui n'est généralement pas le
cas, comme le montre l'extrait du cas \emph{fil rouge} pris pour
exemple ci-dessus.

\begin{figure}
  \centering
  \begin{tikzpicture}
  % ZLC 1
  \path[ffa] (0,0) circle [radius=30pt]; % Aire
  \path[ffc] (0,0) circle [radius=30pt]; % Contour
  % ZLC 2
  \path[ffa, pattern color=RdBu-9-3] (0,0) ellipse (40pt and 20pt); % Aire 
  \path[ffc, draw=RdBu-9-2] (0,0) ellipse (40pt and 20pt); % Contour
  % Objet de ref
  \node[circle, inner sep=0pt,minimum size=4pt, fill] (c) at (0,0) {}; % visible
  \node[circle, inner sep=0pt,minimum size=8pt] (c2) at (0,0) {}; % fictif
  % Légende
  \node[anchor=west] (m) at  (1.75,1) {\footnotesize \itshape objet de référence};
  \path[draw, ->] (m.west) -- (c2);
  \node[anchor=north west, text width=3cm] (m2) at (1.75,-.4) {\footnotesize
    \itshape zones de localisation compatibles};
  \path[draw, ->] (m2.west) --++ (-.8,0);
  \path[draw, ->] (m2.west) -- (1.2,-.6);
  % Fléche + texte
  \path[draw, ->] (2,0) --++ (2.5,0)  node[pos=.5, above] {\footnotesize \itshape fusion};
  % Fusion
  \begin{scope}[xshift=6.5cm]
    \begin{scope}
      \clip (0,0) ellipse (40pt and 20pt);
      \fill[ffa, pattern color=RdBu-9-8] (0,0) circle [radius=30pt];
      \path[ffc, draw=RdBu-9-9] (0,0) circle [radius=30pt];
    \end{scope}
    \begin{scope}
      \clip (0,0) circle [radius=30pt];
      \path[ffc, draw=RdBu-9-9] (0,0) ellipse (40pt and 20pt);
    \end{scope}
    \node[circle, inner sep=0pt,minimum size=4pt, fill] (c) at (0,0) {};
    \node[anchor=west, text width=3cm] (m2) at  (1.75,0) {\footnotesize \itshape zone de localisation probable};
    \path[draw, ->] (m2.west) --++ (-.5,0);
  \end{scope}
\end{tikzpicture}
  \caption{Illustration du processus de construction de la \emph{zone
      de localisation probable} par la \emph{fusion} des \emph{zones
      de localisation compatibles.}}
  \label{fig:obj_fus}
\end{figure}

La \emph{spatialisation} des \emph{indices de localisation} en des
\emph{zones de localisation compatibles} et leur \emph{fusion} en une
\emph{zone de localisation probable} sont les deux étapes majeures
nécessaires à la transformation d'une description de localisation en
une \emph{zone de localisation.} Par conséquent, ces deux étapes sont
au cœur des deux objectifs principaux de cette thèse, à savoir le
développement de méthodes de \emph{spatialisation} et de \emph{fusion}
qui soient adaptées à notre contexte applicatif. En effet, les deux
exemples présentés font ignorent tous les problèmes que l'on peut
rencontrer dans des cas réels, comme la fausseté des \emph{indices de
  localisation,} la difficulté d'interpréter la sémantique des
\emph{relations de localisation, etc.} Les méthodes de
\emph{spatialisation} et de \emph{fusion} devront donc prendre en
compte ces différents enjeux pour être applicables à notre contexte.


%%% Local Variables:
%%% mode: latex
%%% TeX-master: "../../../../main"
%%% End:
