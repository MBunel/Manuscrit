
\tdi{Faire un parallèle entre le « la signification c'est l'usage » de
  Wittgenstein et le fait que différentes relations spatiales puissent
voir leur signification varier en fonction du contexte. cf. Léna Soler
p.14}

l'aphorisme de \bsc{Wittgenstein} : \enquote{la signification c'est
  l'usage}

\subsection{L'ontologie des \emph{relations de localisation}}

3 classes principales + visibilité

préciser liens avec bateman



\begin{figure}
  \centering
  \begin{tikzpicture}[
  every node/.style={anchor=west,
    font={\tiny\ttfamily\textcolor{red}{orl\#}\sffamily}, 
    % text width=5cm
  },
  abstraite/.style={gray},
  atomique/.style={blue}
  ],
  % 
  \begin{scope}[
    level distance=7.5mm,
    edge from parent path={(\tikzparentnode.south) |- ($(\tikzparentnode.south)!0.5!(\tikzchildnode.north)$) -|
      (\tikzchildnode.north)},
    level 1/.style={sibling distance=55mm, anchor=north,align=center},
    ]
    % \tikzstyle{edge from parent}=[draw,red,thick]
    % Parents
    \node[abstraite](rl){RelationDeLocalisation}
    child{node[level 1, abstraite] (r1){Re\-la\-ti\-on\-Spa\-ti\-ale\-Fonc\-ti\-o\-nel\-le}}
    child{node[level 1, abstraite] (r2){Re\-la\-ti\-on\-Spa\-ti\-ale\-Po\-si\-tion\-Re\-la\-ti\-ve}}
    child{node[level 1, abstraite] (r3){RelationSpatialeProximité}};
  \end{scope}

  \begin{scope}[grow=right,
    level distance=10mm,
    sibling distance=5mm,
    edge from parent path={(\tikzparentnode.east) -| ($(\tikzparentnode.east)!0.5!(\tikzchildnode.west)$) |- (\tikzchildnode.west)},
    ]
    \node[font=\tiny] at (rl.east){}
    child{node{CibleVoitSite}}
    child{node{SiteVoitCible}};
  \end{scope}

  \begin{scope}[grow via three points={one child at (.5,-.15) and
      two children at (.5,-.15) and (.5,-.45)},
    growth parent anchor=south west,
    edge from parent path={([xshift=.25cm]\tikzparentnode.south west) |-
      (\tikzchildnode.west)}]
    \node[font=\tiny] at (r1.west){}
    child{node(tt){Proximal}}
    child{node{SitueSurItineraire\-Ou\-Re\-seau\-Sup\-port}};
  \end{scope}


  \begin{scope}[grow via three points={one child at (.5,-.15) and
      two children at (.5,-.15) and (.5,-.45)},
    growth parent anchor=south west,
    edge from parent path={([xshift=.25cm]\tikzparentnode.south west) |-
      (\tikzchildnode.west)}]
    \node[font=\tiny] at (r2.west){}
    child{node[abstraite]{RelationSpatialeDeContact}
      child{node{AExtremiteDe}}
      child{node{ALaFrontiereDe}}
      child{node{DansPlanimetrique}
        child{node{DansLaPartie\-Bas\-se\-De}}
        child{node{DansLaPartie\-De\-X\-La\-Plus\-Pro\-che\-De\-Y}}
        child{node{DansLaPartieEstDe}}
        child{node{DansLaPartieHauteDe}}
        child{node{DansLaPartieNordDe}}
        child{node{DansLaPartieOuestDe}}
        child{node{DansLaPartieSudDe}}
      }
    }
    child[missing]{}
    child[missing]{}
    child[missing]{}
    child[missing]{}
    child[missing]{}
    child[missing]{}
    child[missing]{}
    child[missing]{}
    child[missing]{}         
    child[missing]{}
    child[missing]{}
    child{node{Proximal}
    }
    child{node[abstraite]{RelationSpatialeDeDirection}
      child{node{ADroiteDe}}
      child{node{AGaucheDe}}
      child{node{AvoirASaDroite}}
      child{node{AvoirASaGauche}}
      child{node{AvoirDerriereSoit}}
      child{node{AvoirDevantSoit}}
      child{node{DansLaDirectionDe}}
      child{node[abstraite]{RelationSpatialeDeCardinalité}
        child{node{AEstDe}
          child{node{AEstDeExterne}}
          child{node{DansLaPartieEstDe}}
        }
        child[missing]{}
        child[missing]{}
        child{node{AOuestDe}
          child{node{AOuestDeExterne}}
          child{node{DansLaPartieOuestDe}}
        }
        child[missing]{}
        child[missing]{}
        child{node{AuNordDe}
          child{node{AuNordDeExterne}}
          child{node{DansLaPartieNordDe}}
        }
        child[missing]{}
        child[missing]{}          
        child{node{AuSudDe}
          child{node{AuSudDeExterne}}
          child{node{DansLaPartieSudDe}}
        }
      }
    }
    child[missing]{}
    child[missing]{}
    child[missing]{}
    child[missing]{}
    child[missing]{}
    child[missing]{}
    child[missing]{}
    child[missing]{}
    child[missing]{}
    child[missing]{}
    child[missing]{}
    child[missing]{}
    child[missing]{}
    child[missing]{}
    child[missing]{}
    child[missing]{}
    child[missing]{}
    child[missing]{}
    child[missing]{}
    child[missing]{}
    child[missing]{}
    child{node[abstraite]{RelationSpatialeDisjointe}
      child{node{AEstDeExterne}}
      child{node{AOuestDeExterne}}
      child{node{AuNordDeExterne}}
      child{node{AuSudDeExterne}}
      child{node{HorsDePlanimetrique}}
    }
    child[missing]{}
    child[missing]{}
    child[missing]{}
    child[missing]{}
    child[missing]{}
    child{node[abstraite]{RelationSpatialeItinéraire}
      child{node{ApresJalonSurItineraire}}
      child{node{AuDessusJalonSurItineraire}}
      child{node{AvantJalonSurItineraire}}
      child{node{SitueSurItineraireOuReseauSupport}}
      child{node{SousJalonSurItineraire}}
    }
    child[missing]{}
    child[missing]{}
    child[missing]{}
    child[missing]{}
    child[missing]{} 
    child[missing]{} 
    child{node[abstraite]{RelationSpatialeTernaire}
      child{node{DansLaPartieDeXLaPlusProcheDeY}}
      child{node{DeAutreCoteDeParRapportA}}
      child{node{DuMemeCoteQueParRapportA}}
      child{node{EntreXetY}}
    }
    child[missing]{}
    child[missing]{}
    child[missing]{}
    child[missing]{}
    child[missing]{} 
    child{node[abstraite]{RelationSpatialeVerticale}
      child{node{ALaMemeAltitudeQue}}
      child{node{AuDessusALAplombDe}}
      child{node{AuDessusAltitude}}
      child{node{AuDessusJalonSurItineraire}}
      child{node{AuDessusProche}}
      child{node{DansLaPartieBasseDe}}
      child{node{DansLaPartieHauteDe}}
      child{node{SousALAplombDe}}
      child{node{SousAltitude}}
      child{node{SousJalonSurItineraire}}
      child{node{SousProcheDe}}
      child{node{SousRecouvertPar}}
    };
  \end{scope}

  \begin{scope}[grow via three points={one child at (.5,-.15) and
      two children at (.5,-.15) and (.5,-.45)},
    growth parent anchor=south west,
    edge from parent path={([xshift=.25cm]\tikzparentnode.south west) |-
      (\tikzchildnode.west)}]
    \node[font=\tiny] at (r3.west){}
    child{node{ADistanceQuantitativeRelative}}
    child{node{ADistanceTemps}
      child{node{ATempsDeMarche}}
    }
    child[missing]{}
    child{node{AuDessusProche}}
    child{node{AuxAlentoursDe}}
    child{node{DistanceQuantitativePlanimetrique}}
    child{node{PresDe}}
    child{node{Proximal}};
  \end{scope}

  % Légende
  \begin{scope}
    
  \end{scope}
  
\end{tikzpicture}	

  \caption{Onto}
  \label{fig:ontho}
\end{figure}

\subsection{L'ontologie des \emph{relations de localisation atomiques}}

Contrairement à \emph{l'ontologie des relations de localisation} dont
elle est une spécialisation, \emph{l'ontologie des relations de
  localisation atomiques} est spécifique à notre travail de
recherche.

Pour la construire nous nous sommes basés sur

- décrire relations atomiques définies

DansLaDirectionDe
ADistanceTemps
Cardinalité (verifier)
même alt
Direction DE
Dans plani
Distance Quanti
Not a la frontiere de
a la frontiere de
altEq
AltSup
Contact
dedans
int
ext
sous altitude
sous proche de 

%%% Local Variables:
%%% mode: latex
%%% TeX-master: "../../../../main"
%%% End:
