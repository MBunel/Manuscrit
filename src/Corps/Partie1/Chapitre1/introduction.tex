L'objectif du premier chapitre de la thèse est de présenter le
contexte professionnel de la thèse. Par professionel je veux parler de
tout ce qui touche à l'activiter de recherche et pas uniquement au
cadre du pghm. L'idée de de cette partie est qu'elle serve de longue
phase de contextualisation avant la présentation des objectifs de la
thèse à propement parler (dans le chapitre 2). je propose de commencer
par un petit retour historique sur les secours en montagne. L'idée est
de montrer que le secours en montagne comme service public ce n'est
pas si vieux que ça et qu'avant la situation actuelle c'était beaucoup
plus compliqué. De la même façon ça peut être intéressant de faire
comprendre au lecteur que certaines méthodes et solutions qui auj sont
naturelles (hélicoptère, treuillage) n'ont pas toujours existé. De la
même façon ce retour historique permet d'expliquer le pourquoi de la
cohabitation entre différentes corps, des spécificités locales (dans
le 74) et surtout d'expliquer comment on en est arrivés à la situation
actuelle (plan ORSEC). Une seconde section sera uniquement consacrée à
la situation actuelle. Elle me servira a bien faire la différence
entre les différents acteurs du secours en montagne et expliquera que
nous on travaille spécifiquement avec le PGHM de l'isère (idée de
partir du large pour aboutir à notre cas particulier. Puis je
présenterai la méthodologie des secours. Peut-être qu'il faut que je
change de terminologie puisque je n'ai pas fait une analyse pointue de
cette méthodologie, à voir… L'idée est d'expliquer comment les
secouristes procèdent pour identifier la postion d'une victime, pas
pour la secourir. Du coup comment faire ? Est-ce qu'il faut déjà
expliquer sur quoi on travaille ? Quoi qu'il en soit cette partie
permettera également de parler des problèmes auxquels ils sont
confrontés et notamment comment ils font pour chercher des victimes
qui n'indiquent pas clairement leur position. Ce sera l'occasion de
parler de GendLoc, à la fois pour l'interface de visualisation qui
leur permet de croiser les infoirmation métier et surtout la partie
evoi de sms pour faciliter la localisation. Puis je peux parler des
limitesn de GendLoc, qui apparaissent dans des cas assez particuliers,
notamment lorsque l'appel est passé par un tiers ou lorsque le
requérant n'a pas de GPS. Ce constat servira d'introduction au projet
CHOUCAS qui est construi pour répondre à ces cas.ame

Comme nous l'expliquions précédement (cf. \nameref{part:int}) cette
thèse est à l'interface de plusieurs enjeux. L'un d'eux est
scientifique, bien évidement, et un second est plus appliquatif,
circonscrit à des problématiques métier, celles du secours en
montagne.

Or, bien que le secours en montagne soit un service public familier,
son histoire, son fonctionnement, ses acteurs et ses problématiques
sont mal connues.

Cette thése, et plus généralement le projet de recherche au sein
duquel elle s'inscrit ont étés construits pour répondre à une
problématique spécifique du secours en montagne actuel et par
))onséquent il nous semble indispendable de commencer la présentation
de notre travail par une mise en contexte axée sur le secours en
montagne et ses problématiques.

Une meilleur compréhension de ces questions nous permettra de
présenter la méthodologie de traitement d'une alerte suivie par les
secouristes des \ac{pghm}, puis d'en montrer les limites.