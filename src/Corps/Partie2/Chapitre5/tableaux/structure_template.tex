\begin{tabular}{L{.3\textwidth}>{\footnotesize}p{.4\textwidth}}
  \toprule \multicolumn{1}{c}{\bfseries Colonne} &
  \multicolumn{1}{c}{\normalsize \bfseries Contenu} \\ \midrule
% Sémantique des relations spatiales
  \addlinespace
  Identifiant de l'extrait & dddd \\
  Identifiant de l'expression & dd \\
  Extrait & Verbatim de la phrase transcrite\\
  Confiance & Permet au transcripteur de saisir sa confiance en
              \emph{l'indice de localisation} (uniquement si la saisie
              est faite par un secouriste)\\
  Timestamp & Début de l'extrait dans le fichier audio source\\
  Locuteur & \\
  Verbe & Verbe utilisé dans \emph{l'indice de localisation}\\
  Modifieur du verbe & \\
  Sujet & Sujet de \emph{l'indice de localisation}\\
  Modifieur du sujet & \\
  \emph{Relation de localisation} & \\
  Modifieur de la relation de localisation & \\
  Objet de référence & Nom ou type de \emph{l'objet de référence}
                       (cette colonne peut être multipliée si la
                       \emph{relation de localisation} est bi ou n-aire)\\
  Modifieur de l'objet de référence & (cette colonne peut être multipliée si la
                                      \emph{relation de localisation} est bi ou n-aire)\\
  Commentaires & Champ permettant au transcripteur de commenter sa
                 saisie ou les indications données par le requérant
                 (erreurs potentielles, fautes de prononciation, \emph{etc.}) \\
  \bottomrule
\end{tabular}
