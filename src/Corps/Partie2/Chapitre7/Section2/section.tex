La première partie de la méthode de \emph{spatialisation} est la
\emph{rasterisation} des \emph{objets de référence.} Cependant, le
processus que nous venons de définir ne peut se résumer à un simple
changement de représentation de la donnée géographique. Le terme de
\enquote{\emph{méthode de rasterisation}}, tel que nous l'avons
introduit, englobe plusieurs questions. Tout d'abord, la sélection de
la partie de \emph{l'objet de référence} qui va être
\emph{rasterisée,} mais également la manière dont on transforme cette
sélection en un raster, c'est-à-dire comment définit si un pixel
appartient ou non à \emph{l'objet de référence.} Toutefois lorsque
nous parlons de définition d'une \emph{méthode de rasterisation} nous
ne traiterons que de ce second point, il n'y a, en effet, pas
d’intérêt à changer la manière dont la sélection des pixels est
effectuée.

\subsection{Considérations générales sur la \emph{rasterisation}}

Un point essentiel de la représentation raster \emph{d'objets
  géographiques} et qu'elle impose, contrairement à une représentation
vectorielle, de définir explicitement une résolution et une zone de
modélisation. L'opération de \emph{rasterisation} nécessite ne peut
donc être effectuée qu'a partir d'un \emph{objet de référence}
vectoriel.

Cependant la zone de rasterisation est, implicitement, définie par le
secouriste, il s'agit de la \ac{zir}.

Seule la résolution reste à fixer.

\tdi{parler des consiérations autour de la rasterisation importance de
  la zir, de la résolution, méthode de sélection des pixels, etc}

% Parler des types d'objets de référence
Une autre contrainte est que les \emph{méthodes de rasterisation}
doivent être à même de traiter différents types \emph{d'objets de
  référence,} comme des \emph{points} utilisés pour représenter des
points d’intérêt ou des lieux nommés, des \emph{polylignes,} utilisées
pour représenter tout type de réseau (\eg voirie, transport d'eau ou
d'énergie) et des \emph{polygones,} utilisés pour représenter tout
objet à l’implantation zonale.

Méthode de sélection des pixels
%
La \autoref{fig:rasterisation_sel_pixels} illustre ces deux
approches. Un même \emph{objet de référence} est \emph{rastérisé,}
d'une part en sélectionnant uniquement les pixels dont le centroïde
est à l'intérieur de \emph{l'objet de référence} (aboutissant au
raster \textcolor{RdBu-9-1}{\textsf{A}}), d'autre part en
sélectionnant tous les pixels touchant \emph{l'objet de référence} (ce
qui donne le raster \textcolor{RdBu-9-9}{\textsf{B}}). Comme on peut
le voir la seconde solution abouti à une zone plus large, la zone
résultant de la \emph{rasterisation} est alors plus importante que
l'emprise réelle de l'objet. Cependant cette seconde approche à
l'avantage de toujours aboutir à un raster non nul. En effet dans
certaines configurations, il est possible qu'un \emph{objet de
  référence,} de surface non nulle, ne contienne aucun centroïde de
pixel, aboutissant à la construction d'un raster vide. Or, cette
situation n'est pas acceptable, car, dans notre méthodologie, cela
aboutirait à ignorer \emph{l'objet de référence,} et donc à ne pas
considérer l'ensemble des informations transmises par le requérant. Ce
phénomène se produit lorsque la maille utilisée pour la
\emph{rasterisation} est trop grande comparativement aux objets que
l'on souhaite \emph{rasteriser,} donc à une mauvaise paramétrisation
de la rasterisation. Ce problème est donc corrigible par
l'augmentation de la résolution du raster cible. Toutefois cette
solution n'est pas toujours possible lors d'une utilisation concrète
d'une \emph{rasterisation.} Prenons pour exemple la
\autoref{fig:rasterisation_cas_limite}, illustrant une situation où
l'on cherche à \emph{rasteriser} un bâtiment quelconque en vue de la
\emph{spatialisation} d'une \emph{relation de localisation atomique.}
Le bâtiment, de taille moyenne (\ie \SI{200}{\metre\squared}) est
situé à cheval sur 4 pixels \footnote{Dont chacun a une superficie de
  \SI{625}{\metre\squared}.} sans contenir de centroïde d'aucun
d'entre eux. Ainsi, si l'on emploie les deux méthodes de
\emph{rasterisation} présentées
(\autoref{fig:rasterisation_sel_pixels}) on obtiendra soit une zone
vide (variante \textcolor{RdBu-9-1}{\textsf{A}}), soit une zone
composée de 4 pixels (variante \textcolor{RdBu-9-9}{\textsf{B}}). Pour
que le raster produit par la première rasterisation soit non vide, il
faudrait diviser la taille des pixels par (environ) 2, ce qui
multiplierait le nombre de pixels par 4. Or, si cela ne pose pas de
problème pour une petite zone, comme celle utilisée pour l'exemple,
cela devient difficile, voir très difficile si l'on souhaite
travailler sur une zone assez étendue (au-delà de la centaine de
\si{\kilo\meter\squared}).

\begin{figure}
  \centering
  \begin{tikzpicture}
  % Arrow
  \begin{scope}
    \path[draw, ->, shorten >=5pt, shorten <=5pt] (1,2) |- (8,3)
    node[pos=.75, font=\footnotesize, fill=white] (labRastA)
    {\itshape rasterisation};
    %
    \node[anchor=north west,baseline, text width=3.5cm, font=\tiny] at
    ([xshift=1.5ex, yshift=1ex]labRastA.south west) {avec sélection
      des pixels dont le centroïde est à l'intérieur de \emph{l'objet
        de référence}};
    
    \path[draw, ->, shorten >=5pt, shorten <=5pt] (1,0) |- (8,-1)
    node[pos=.75, font=\footnotesize, fill=white] (labRastB)
    {\itshape rasterisation};
    %
    \node[anchor=north west,baseline, text width=3.5cm, font=\tiny] at
    ([xshift=1.5ex, yshift=1ex]labRastB.south west) {avec sélection de
    tous les pixels intersectant \emph{l'objet de référence}};
  \end{scope}
  % Représentation objet de référence
  \begin{scope}[local bounding box=obr]
    \fill[ffa, pattern color=black] (1,1) circle [radius=15pt];
    \path[ffc, black] (1,1) circle [radius=15pt];
    \path[ffc, black] (0,0) rectangle (2,2);
    \node[text width=3cm, align=center, anchor=north,
    font=\footnotesize, fill=white] at (1,-.25)
    {\itshape Objet de référence};
  \end{scope}
  
  % Représentation objet de référence rasterisé
  \begin{scope}[xshift=8cm, yshift=2cm]
    \begin{scope}
      \foreach \x in {0,.25,...,1.75}{
        \foreach \y in {0,.25,...,1.75}
        \path[draw, line width=.01mm] (\x,\y) rectangle (\x +.25, \y + .25);
      }
      \path[ffa] (.5,.5) rectangle (1.5, 1.5);
      \path[ffc, black, dashed] (1,1) circle [radius=15pt];
      \path[ffc] (0,0) rectangle (2,2);
    \end{scope}
    \node[text width=3cm, align=center, anchor=north, font=\footnotesize] at (1,-.25)
    {\itshape Objet de référence rastérisé \normalfont \textcolor{RdBu-9-1}{\textsf{A}}};
  \end{scope}
  
  \begin{scope}[xshift=8cm, yshift=-2cm]
    \begin{scope}
      \foreach \x in {0,.25,...,1.75}{
        \foreach \y in {0,.25,...,1.75}
        \path[draw, line width=.01mm] (\x,\y) rectangle (\x +.25, \y + .25);
      }
      % Pixels rasterisation
      %\path[ffc] (1,1) circle [radius=15pt];
      \path[ffa2] (.5,.5) rectangle (1.5, 1.5);
      \path[ffa2] (.25,.75) rectangle (.5, 1.25);
      \path[ffa2] (1.5,.75) rectangle (1.75, 1.25);
      \path[ffa2] (.75,1.5) rectangle (1.25, 1.75);
      \path[ffa2] (.75,.25) rectangle (1.25,.5);
      \path[ffc2, black, dashed] (1,1) circle [radius=15pt];
      \path[ffc2] (0,0) rectangle (2,2);
    \end{scope}
    \node[text width=3cm, align=center, anchor=north, font=\footnotesize] at (1,-.25)
    {\itshape Objet de référence rastérisé \normalfont \textcolor{RdBu-9-9}{\textsf{B}}};
  \end{scope}
\end{tikzpicture}
  \caption{Illustration de l'impact de la méthode de sélection des
    pixels sur le résultat d'une opération de \emph{rasterisation.}}
  \label{fig:rasterisation_sel_pixels}
\end{figure}

Compte-tenu de ces limites, nous avons décidé d'opter pour la seconde
approche, c'est-à-dire une \emph{rasterisation} sélectionnant tous les
pixels touchant \emph{l'objet de référence.} Ce choix nous dispense de
définir la résolution de la \ac{zir} en fonction des \emph{objets de
  référence} utilisés, mais conduit à une fréquente exagération de la
taille des \emph{objets de référence.}

\begin{wrapfigure}{r}{4cm}
  \centering
  \begin{tikzpicture}
  \begin{scope}
    \foreach \x in {0,.5,...,1.5}{
      \foreach \y in {0,.5,...,1.5}{
        \pgfmathsetmacro\xc{\x +.25}
        \pgfmathsetmacro\yc{\y +.25}
        % 
        \path[draw, line width=.01mm] (\x,\y) rectangle (\x +.5, \y +
        .5);
        \fill[black] (\xc,\yc) circle [radius=0.5 pt];
      }
    }
    \path[ffc, black] (0,0) rectangle (2,2);
    \fill[ffa, pattern color=black,rotate around={30:(1,1)}] (.8,.9) --++ (.4,0) --++ (0,.2) --++ (-.4,0) --cycle;
    \path[ffc, black,rotate around={30:(1,1)}] (.8,.9) --++ (.4,0) --++ (0,.2) --++ (-.4,0) --cycle;
  \end{scope}
  \path[<->, draw] (0,2.2) -- (.5,2.2) node[pos=.5, above, font=\tiny] {\SI{25}{\meter}};  
\end{tikzpicture}
  \caption{Illustration d'une situation où la \emph{rasterisation}
    aboutit à un raster vide}
  \label{fig:rasterisation_cas_limite}
\end{wrapfigure}

\subsection{Les méthodes de rasterisation}

La méthode de sélection des pixels ayant été fixée, le seul critère
permettant de distinguer les différentes \emph{méthodes de
  rasterisation} est la sélection de la partie de \emph{l'objet de
  référence} à \emph{rasteriser.} Comme nous l'expliquions
précédemment ce choix est essentiel, car il contribue à définir la
sémantique de la \emph{relation de localisation atomique,} une même
métrique calculée à partir de points différents d'un même \emph{objet
  de référence} pouvant avoir des interprétations très diverses, comme
nous l'avons montré avec l'exemple des \emph{relations de localisation
  atomiques} \onto[orla]{Distance\-Quantitive} et
\onto[orla]{A\-La\-Frontiere\-De} (\autoref{chap:07-sec1}).

On peut imagier de nombreuses \emph{méthodes de rasterisation}
différentes, en fonction de la partie de \emph{l'objet de référence}
sélectionnée, cependant toutes ne sont pas pertinentes ici. Nous avons
identifié trois \emph{méthodes de rasterisation} différentes,
présentées dans la \ref{fig:methode_rasterisation}. Le résultat de la
\emph{rasterisation} est présenté pour chaque type \emph{d'objet de
  référence.}

Une première possibilité est de n'effectuer aucune sélection préalable
et donc de \emph{rasteriser} \emph{l'objet de référence} dans son
ensemble (\autoref{tab:methode_rasterisation}, première ligne). C'est
cette solution, utilisée pour tous les exemples précédents (\ie
figures \ref{fig:methodo_spatialisation},
\ref{fig:rasterisation_sel_pixels} et
\ref{fig:rasterisation_cas_limite}) est utilisée lorsque la
\emph{relation de localisation atomique} ne se référè pas à une partie
spécifique de \emph{l'objet de référence.} C'est par exemple le cas
des \emph{relations} \onto[orla]{Interieur} ou \onto[orla]{Exterieur}
qui s'appliquent à l'objet de référence dans son ensemble. On pourra
remarquer que cette méthode de \emph{rasterisation} s'applique à la
majorité des \emph{relations de localisation atomiques} définies
jusqu'ici. En effet, cette méthode de \emph{rasterisation} est
fortement majoritaire et seules quelques \emph{relations de
  localisation atomiques} spécifiques nécessitent une méthode plus
spécifique.

Les deux autres rasterisers 

Mais il également possible de ne rasteriser qu'une partie de
\emph{l'objet de référence,} comme sa frontière ou son centroïde
(\autoref{tab:methode_rasterisation}).

Le second \emph{rasteriser} défini ne considère que la frontière de
\emph{l'objet de référence} traité.
%
Ce \emph{rasteriser} est notamment employé pour \emph{spatialiser} la
\emph{relation de localisation} \onto[orla]{A\-La\-Frontiere\-De}.

Enfin, un \emph{rasteriser} ne traitant que le centroïde de
\emph{l'objet de référence} a été défini. Contrairement aux autres
\emph{rasterisers} présentés jusqu'ici, celui-ci conduit tout type
d'objet (\ie point, ligne, polygone) a être représentés par un seul
point. Ainsi, tous les objets traités sont représentés pas un point,
lui-même \emph{rasterisé} en un unique pixel.

\begin{figure}
  \centering
  \begin{tabular}{>{\bfseries}R{2cm}C{3cm}C{3cm}C{3cm}}
  \toprule
  & \multicolumn{1}{c}{\bfseries Ponctuel} &
   \multicolumn{1}{c}{\bfseries Linéaire} &
   \multicolumn{1}{c}{\bfseries Région} \\
  \addlinespace
  & \tikz{\fill[black] (.65,.65) circle [radius=1pt];\path[ffc, black] (0,0) rectangle (2,2);}&
  \tikz{\path[ffc,black] (.3,.3) ..controls(.25,1.75) and (1.375,.25) .. (1.7,1.7); \path[ffc,black] (0,0) rectangle (2,2);} &
  \tikz{\fill[ffa, pattern color=black] (1.125,.65) circle [radius=10pt];\path[ffc,black](1.125,.65) circle [radius=10pt];\path[ffc,black](0,0) rectangle (2,2);}\\
  %
  \midrule
  \addlinespace
  Objet
  & \tikz{
    \begin{scope}
      \foreach \x in {0,.25,...,1.75}{
        \foreach \y in {0,.25,...,1.75}
        \path[draw, line width=.01mm] (\x,\y) rectangle (\x +.25, \y + .25);
      }\path[ffa] (.5,.5) rectangle (.75,.75);
      \path[ffc] (0,0) rectangle (2,2);
    \end{scope}
    }
                                           & \tikz{
                                             \begin{scope}
                                               \foreach \x in {0,.25,...,1.75}{
                                                 \foreach \y in {0,.25,...,1.75}
                                                 \path[draw, line width=.01mm] (\x,\y) rectangle (\x +.25, \y + .25);
                                               }
                                               \foreach \i/\j in {.25/.25,.25/.5,.25/.75,.5/.75,.5/1,.75/.75,.75/1,1/.75,1/1,1.25/1,1.5/1,1.5/1.25,1.5/1.5}{
                                                 \path[ffa] (\i,\j) rectangle (\i + .25, \j +.25);
                                               }
                                               \path[ffc] (0,0) rectangle (2,2);
                                             \end{scope}
                                             } &\tikz{
                                                 \begin{scope}
                                                   \foreach \x in {0,.25,...,1.75}{
                                                     \foreach \y in {0,.25,...,1.75}
                                                     \path[draw, line width=.01mm] (\x,\y) rectangle (\x +.25, \y + .25);
                                                   }
                                                   \path[ffa] (.75,.25)
                                                   rectangle (1.5,
                                                   1);\path[ffc] (0,0) rectangle (2,2);
                                                 \end{scope}
                                                 }\\
  \addlinespace
  Frontière&\tikz{
             \begin{scope}
               \foreach \x in {0,.25,...,1.75}{
                 \foreach \y in {0,.25,...,1.75}
                 \path[draw, line width=.01mm] (\x,\y) rectangle (\x +.25, \y + .25);
               }
               \path[ffa] (.5,.5) rectangle (.75,.75);
               \path[ffc] (0,0) rectangle (2,2);
             \end{scope}
             }
                                           & \tikz{
                                             \begin{scope}
                                               \foreach \x in {0,.25,...,1.75}{
                                                 \foreach \y in {0,.25,...,1.75}
                                                 \path[draw, line width=.01mm] (\x,\y) rectangle (\x +.25, \y + .25);
                                               }
                                               \path[ffa] (.25,.25) rectangle (.5, .5);
                                               \path[ffa] (1.5,1.5) rectangle (1.75, 1.75);
                                               \path[ffc] (0,0) rectangle (2,2);
                                             \end{scope}
                                             } &\tikz{
                                                 \begin{scope}
                                                   \foreach \x in {0,.25,...,1.75}{
                                                     \foreach \y in {0,.25,...,1.75}
                                                     \path[draw, line width=.01mm] (\x,\y) rectangle (\x +.25, \y + .25);
                                                   }
                                                   % Pixels rasterisation
                                                   \path[ffa] (.75,.25)
                                                   rectangle (1.5,
                                                   .5);  
                                                   \path[ffa] (.75,.5)
                                                   rectangle (1,.75);
                                                   \path[ffa] (1.25,.5)
                                                   rectangle (1.5,.75);
                                                   \path[ffa] (.75,.75)
                                                     rectangle (1.5,1);
                                                   \path[ffc] (0,0) rectangle (2,2);
                                                 \end{scope}
                                                 }\\
%
  \addlinespace
%
  Centroide&\tikz{
             \begin{scope}
               \foreach \x in {0,.25,...,1.75}{
                 \foreach \y in {0,.25,...,1.75}
                 \path[draw, line width=.01mm] (\x,\y) rectangle (\x +.25, \y + .25);
               }
               \path[ffa] (.5,.5) rectangle (.75,.75);
               \path[ffc] (0,0) rectangle (2,2);
             \end{scope}
             }
                                           & \tikz{
                                             \begin{scope}
                                               \foreach \x in {0,.25,...,1.75}{
                                                 \foreach \y in {0,.25,...,1.75}
                                                 \path[draw, line width=.01mm] (\x,\y) rectangle (\x +.25, \y + .25);
                                               }
                                               \path[ffa] (1,1) rectangle (1.25, 1.25);
                                               \path[ffc] (0,0) rectangle (2,2);
                                             \end{scope}
                                             } &\tikz{
                                                 \begin{scope}
                                                   \foreach \x in {0,.25,...,1.75}{
                                                     \foreach \y in {0,.25,...,1.75}
                                                     \path[draw, line width=.01mm] (\x,\y) rectangle (\x +.25, \y + .25);
                                                   }
                                                   \path[ffa] (1,.5)
                                                   rectangle (1.25, .75);
                                                   \path[ffc] (0,0) rectangle (2,2);
                                                 \end{scope}
                                                 }\\
  \bottomrule    
\end{tabular}
  \caption{illustration du raster résultant des différentes méthodes
    de \emph{rasterisation} définies.}
  \label{fig:methode_rasterisation}
\end{figure}

% Parler rapidement des méthodes définies mais non utilisées (bbox,conxhull)
Nous avons également envisagé d'autres \emph{méthodes de
  rasterisation,} sans parvenir à en identifier un usage pertinent. On
pourrait, par exemple, ignorer les frontières intérieures des régions
ayant une structure annulaire (\eg une forêt avec des
clairières). Mais il est également possible d'employer des méthodes
sensiblement différentes de celles présentées jusqu'ici, comme la
\emph{rasterisation} de l'enveloppe convexe ou de la boite englobante
de \emph{l'objet de référence.} De telles méthodes ne traitent plus de
\emph{l'objet de référence} ou l'une de ces composantes, c'est-à-dire
un sous-ensemble de sa géométrie, mais d'une version modifiée de
celui-ci.

% Conclusion
Ainsi, dans son état actuel nous disposons d'un ensemble de trois
\emph{rasterisers.}

%%% Local Variables:
%%% mode: latex
%%% TeX-master: "../../../../main"
%%% End:
