


\subsection{Verbratim de l'audio}


\begin{dialogue}
  \Sec Allo ?
  \Req Oui…
  \Sec Oui bonjour madame, le secours en montagne à l'appareil. Donc
  j'ai repris les éléments que m'a donné le \ac{codis}. Donc vous
  étiez au sommet du Moucherotte à 17h30…
  \Req Ouais…
  \Sec … et vous êtes redescendue… donc… heu… Vous êtes redescendue
  par le chemin que vous avez pris à la montée ?
  \Req Oui, ben j'ai essayé de le trouver, je l'ai pas trouvé… donc
  j'ai viré un peu à gauche… heu… sur ce… heu… grand… ce grand couloir
  enseigné. Je crois qu'il y a des gens qui repartent du côté… heu…
  qui redescendent par Lens-en-Vercors, mais oui je suis… je suis
  monté de Sain-Nizier du… du haut du tremplin…
  \Sec Vous êtes garée… Vous êtes gardée donc… Heu… À Saint-Nizier ?
  \Req Saint-Nizier ouais… Heu… Le haut du tremplin…
  \Sec D'accord, au tremplin… D'accord… Ça marche…
  \Req Et… Mais là je suis complètement paumée dans… dans les
  cailloux… Jusqu'à maintenant j’étais dans la forêt… un peu de neige…
  Mais là, j'ai failli déjà… me tuer en glissant parce qu’il y a plein
  de petits morceaux de cailloux. Donc en partant du… du sommet du
  Moucherotte, donc j'ai viré un peu à gauche et après j'ai… j'ai
  coupé pour descendre, descendre, descendre, descendre et j'ai pas retrouvé mon
  sentier et je suis certainement descendue beaucoup trop bas et là
  j'ai plus dutout la force de remonter, je suis trop descendue…
  % 
  \Sec \direct{Parle en même temps} Oui, OK, d'accord… ça marche…
  % 
  \Req … et là mon but c'était de descendre, descendre, descendre vers
  les maison qui sont en bas, mais je là je me retrouve dans… J'ai une
  grosse barre rocheuse à droite, j'ai… heu… plein de petits cailloux
  très glissants, c'est très raide et… Voila… Je suis assise là, au
  milieu de nulle part…
  % 
  \Sec \direct{Parle en même temps} Alors vous êtes partie…
  Excusez-moi… Vous êtes partie seule… du… de Saint-Nizier, on est
  d'accord ?
  % 
  \Req Oui…
  % 
  \Sec Vous êtes… vous êtes montée par le GR 91, vous êtes arrivée
  là-haut…
  % 
  \Req \direct{Parle en même temps} Oui…
  % 
  \Sec … et êtes… et vous avez emprunté… en tout cas dans l'idée, vous
  êtes repartie… en direction… heu… du tremplin… On est d'accord ?
  Vous avez emprunté le même chemin…
  % 
  \Req \direct{Parle en même temps} Voila… Oui…
  % 
  \Sec OK…
  % 
  \Req Mais… Mais… Pas par la droite où apparemment y'a deux sortes de
  chemins, pour le rejoindre…
  % 
  \Sec \direct{Parle en même temps} D'accord…
  % 
  \Req … Parce-qu’à un moment donné y'a une… une espèce de cabane
  qu'on voit quand on est au Moucherotte…
  % 
  \Sec \direct{Parle en même temps} Oui…
  % 
  \Req … C'est pas là. Moi je suis allée dans… Plutôt sur la gauche…
  % 
  \Sec \direct{Parle en même temps} Oui, plutôt à gauche… D'accord,
  je vois je que vous voulez dire…
  % 
  \Req … J'ai fais quoi, 200 mètres… à gauche et puis j'ai coupé à la
  perpendiculaire en redescendant…
  % 
  \Sec \direct{Parle en même temps} Et vous avez jamais croisé un
  chemin ?
  % 
  \Req J'étais arrivée par là et j'ai pas retrouvé de chemin… et…
  c'est… c'est… c'est enneigé donc… donc on voit pas bien et j'ai pas
  retrouvé les traces de neige avec lesquelles je m'étais orientée et
  j'ai suivi un gars en montant… et puis… Ben évidement y'avais plus
  personne en descendant…
  % 
  \Sec \direct{Parle en même temps} Alors OK… Vous avez essayé de… de
  crier là pour vois s'il y avait des gens qui étaient encore dans le
  secteur, non loin ?
  % 
  \Req He ben… Non… Mais là y'a absolument personne…
  % 
  \Sec \direct{Parle en même temps} Vous êtes donc plutôt rentrée dans
  la forêt là ?
  % 
  \Req Eh ben… Là je suis un peu à la sortie de la… Je me suis pris…
  Heu… Quand même… Heu… J'ai fait un peu d'escalade là, parce que
  c'est très rocheux…
  % 
  \Sec \direct{Parle en même temps} Oui…
  % 
  \Req … Et… Je suis dans une partie où il y a plein d'arbres… Heu…
  tous secs… Et pas trop d'arbres… Je suis complétement à découvert…
  % 
  \Sec \direct{Parle en même temps} Vous êtes à découvert ? Ok… Vous
  êtes habillée en beige et noir ? C'est ça ? Non en brun et noir,
  c'est ça ?
  % 
  \Req Ouais… Ouais… Et juste à ma droite, je vous dis, y'a une
  grosse, grosse barre rocheuse… Donc je vois pas… Heu… Ouais et
  devant moi j'ai la plaine… Et, et, à ma gauche j'ai quand même des
  sapins, en fait.
  % 
  \Sec Ah OK. Et donc, dans la plaine qu'est-ce que vous voyez là,
  dans la plaine ?
  % 
  \Req Eh ben je vois un peu d'eau… Je vois…
  % 
  \Sec \direct{Parle en même temps} Vous voyez des habitations ? Vous
  voyez une route ? Vous voyez…
  % 
  \Req Un petit peu, mais… Heu… Les habitations c'est très… Éparses…
  C'est des… Là juste devant moi y'a un groupe de… de… 4… 5 maisons ou
  des fermes… Et juste à ma droite…
  % 
  \Sec \direct{Parle en même temps} C'est-à-dire ? Non loin… à combien
  de mètres ?
  % 
  \Req Ah non, mais c'est très, très, très loin ça… Je les vois mais
  c'est pas accessible…
  % 
  \Sec \direct{Parle en même temps} D'accord…
  % 
  \Req … Devant moi y'a des tas de barres rocheuses en fait, je vois
  pas comment je peux arriver jusque-là
  % 
  \Sec \direct{Parle en même temps} Ok…
  % 
  \Req … et en dessous des barres rocheuses ça repart en forêt… En
  fait.
  % 
  \Sec D'accord… Donc… Heu… Ok.
  % 
  \Req Je sais pas si je peux continuer à descendre… Là… C'est
  vachement raide… et glissant.
  % 
  \Sec Par contre vous avez pas un smartphone avec un GPS… Heu… Dedans
  et tout ça ?
  % 
  \Req Non…
  % 
  \Sec D'accord…
  % 
  \direct{Enregistrement coupé}
\end{dialogue}
