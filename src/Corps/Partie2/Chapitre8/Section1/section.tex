\subsection{Présnetation des 3 étapes}


\tdi{Transition Pour faire ces fusions on a besoin d'opérateurs}

\subsection{Opérateurs}

\tdi{Si ne pas passer les caractéristiques des opérateurs ici.}

\tdi{Nécessité d'utiliser des opérateurs duals}

\tdi{Enlever les opérateurs drastiques car ils sont trop sévères et ne
permettent pas de conserver des positions dont le degré d'appartenance
n'est pas au moins un fois de 1. En plus si j'ai n indices il faut
qu'il y ait n-1 indices dont le dégré est de 1.}

\tdi{Loi du tier-exclu}
\tdi{Principe de non contradiction}

Concepts non excluant

On a pas eu de problèmes du fait que cette loi ne soit pas validée

On peut trouver des exemples réels de cas ou ce n'est pas vrai

Lister les opérateurs qui valident tout

Caractéristiques non excluantes

Discuter du fait que ce soit mieux que ces principes ne soit pas
validés.

\tdi{Opérateurs archimédiens}

Mettre une figure de l'intersection d'angle avec les opérateurs
archimédiens (ex. sud et sud-est).

Parler l'exemple de l'union, je suis à côté d'un lac et plus vrai si
on est a côté de deux lac -> wtf

Ce critère discalifie lukacevicz et probabiliste

\tdi{Raisonnement en termes d'interprétabilité}


On choisi Zadeh

\subsection{La prise en compte de la confiance}

La confiance peut se définir en amont de la spatialisation.

Diminuer la confiance c'est augmenter le degré d'appartenance minimal

Proposer une version améliorée de la fig finale du chapitre 4 avec la
confiance.

Dire que les fonctions présentées reviennent a considérer que la
confiance est absolue

Parler des seuils de la confiance (3) et des valeurs associées ou contraintes.