Une fois que la structure globale de l'ontologie des règles de
décomposition a été fixée, il est nécessaire de définir son contenu en
détail (\ie les \emph{relations de localisation,} les \emph{relations
  de décomposition} et les \emph{relations de localisation
  atomiques}). Pour ce faire, on peut se fonder sur la littérature et
plus particulièrement sur des travaux formalisant des \emph{relations
  de localisation} \autocite{Vandeloise1986,Aurnague1997,Borillo1998},
pour identifier des \emph{relations de localisation} et en proposer
une décomposition. Cependant, bien que cette approche soit pertinente,
elle ne nous semble pas être pleinement adaptée à notre cas
d’application. En effet ces travaux traitent des \emph{relations de
  localisation} dans leur acception générale, alors que nous
travaillons sur un contexte particulier, avec un relief et des
\emph{objets de référence spécifiques} (\eg montagne, col,
\emph{etc.}). On peut donc augurer que le contexte influence
l’utilisation des \emph{relations de localisation} et donc impacte
leur sémantique, comme l'illustre l'aphorisme de \bsc{Wittgenstein} :
\enquote{la signification c'est l'usage}. Or nous souhaitons
développer une ontologie orientée métier et donc prenant en compte les
spécificités de localisation en milieu montagneux. Il semble alors
essentiel d'étudier les descriptions de positions dans le contexte du
secours en montagne afin de construire une ontologie la plus
pertinente possible.

Pour ce faire nous avons entrepris de recenser et d'étudier les
\emph{relations de localisation} les plus utilisées lors des
alertes. Les \ac{pghm} disposant d'un enregistreur d'appels, nous
pouvons obtenir des enregistrements de discussions où les requérants
décrivent leur position. En analysant le contenu de ces
enregistrements il est alors possible d'identifier les \emph{relations
  de localisation} les plus utilisées, de savoir si elles le sont dans
un contexte particulier ou avec une signification plus spécifique que
leur sens général.

\subsection{La retranscription et l'analyse des alertes}

L'exploitation des enregistrements audio d'alertes nécessite au
préalable leur retranscription textuelle. Nous avons donc développé
une méthode et un template (ou canevas) de retranscription a cet
effet. Le travail de définition du template et de retranscription a
été effectué collaborativement \autocite{Bunel2019}, les résultats
pouvant être exploités par différents membres du projet Choucas et
notamment pour l'étude du processus d'alerte (Lot 4).

\subsubsection{Les corpus d'alertes}

Les retranscriptions ont été réalisées à partir d'un ensemble de 52
enregistrements d'appels de demande d'assistance auprès des \ac{pghm}
de Chamonix (5 alertes) et de Grenoble (47 alertes). Ces fichiers,
sélectionnés et compilés par le \ac{pghm} de Grenoble, forment deux
corpus d'alertes distincts, dont le premier nous a été transmis en
2017 et le second en 2019. Ces deux corpus sont relativement
similaires, que ce soit par le nombre d'alertes, leur durée ---~qui
varie de quelques secondes à plusieurs dizaines de minutes
(\autoref{fig:dist_temps_alertes})~--- ou leur
contenu. Malheureusement nous n'avons que peu, voir pas,
d'informations exogènes sur ces différentes alertes. Pour certaines
d'entre elles les secouristes nous ont transmis les coordonnées de la
victime ou la date et l'heure de l'intervention, mais dans la plupart
des cas ces informations nous sont inconnues, a moins qu'elles soient
explicitement données dans l'alerte (\eg énoncé oral de coordonnées
GPS, précision de l'heure, \emph{etc.}). De plus, nous ne savons pas
comment les alertes qui nous ont été transmises furent sélectionnées,
bien qu'elles nous aient été décrites comme représentatives des phases
de localisation des victimes.

Une autre particularité importante de ces enregistrements est qu'ils
sont réalisés au niveau de la ligne téléphonique des \ac{pghm} et non
des \ac{codis}, généralement contactés par les requérants
(cf. \autoref{chap:02}). Par conséquent nous n'avons généralement pas
connaissance du début ou de la fin de la communication. De plus, les
fichiers qui nous ont été transmis ne correspondent pas aux
enregistrements initiaux. Ces derniers ont été montés, initialement de
sorte à censurer les informations personnelles (\eg identité de la
victime, description des blessures), mais dans les faits ils ne
continent généralement que la phase de description de position.

\begin{figure}
  \centering
   \begin{tikzpicture}
   \begin{axis}
     \addplot [boxplot] table [y index=0] {./Corps/Partie2/Chapitre5/figures/duration.dat};
     
  %\addplot [thick] gnuplot [raw gnuplot] {plot './Corps/Partie2/Chapitre5/figures/duration.dat' smooth kdensity};
\end{axis}
\end{tikzpicture}

  \caption{Distribution de la durée des alertes}
  \label{fig:dist_temps_alertes}
\end{figure}

\subsubsection{Template de retranscription}

Le recours à un template et non à une retranscription libre des
alertes se justifie par

Notre parti-pris n'a pas été de travailler à partir d'une
retranscription libre des alertes, mais à l'aide d'un tableau
prédéfini, un \emph{template de retranscription}
(\autoref{tab:struct_temp}) dont la structure reprend celle des
\emph{indices de localisation} telle que nous l'avons définie
(cf. \autoref{chap:04}).

Ce choix présente deux avantages. Tout d'abord il permet de faciliter
l'analyse des alertes en imposant, dès la saisie, la définition de
certains critères.
%
De plus, il permet de fixer les 



Pour faciliter les saisies et leur répartition entre transcripteurs,
un fichier est créé pour chaque alerte.

\texttt{XXXXXXXXXXXXXXXXXXx}


Dans sa forme finale, ce template de transcription se présente sous la
forme d'un fichier tabulaire structuré, auto-documenté et exemplifié,
composé de trois onglets principaux
%
---~\begin{enumerate*}[label=(\alph*)]
\item \emph{métadonnées},
\item \emph{interprétation des expressions} et
\item \emph{toponymie}
\end{enumerate*}~---,
%
permettant de transcrire différents aspects des alertes. L'ensemble du
fichier est conçu pour faciliter le processus de retranscription et
limiter au maximum l'influence du transcripteur sur la saisie. Chaque
onglet et champ à saisir est documenté, des instructions de saisie
présentent l'ensemble des tâches à effectuer sont données et un
exemple de saisie complet et complexe est détaillé. De plus, la
structure même du document est conçue pour faciliter la saisie. Les
onglets sont placés dans leur ordre de consultation ou de saisie (\eg
on commence par lire les instructions dans l'onglet
%
\begin{enumerate*}[label=(\arabic*)]
\item \emph{Instructions \& Consignes,} puis on saisit les
  informations générales sur l'alerte dans l'onglet
\item \emph{métadonnées,} avant de saisir l'alerte en détail dans les
  onglets
\item \emph{interprétation des expressions} et
\item \emph{toponymie,} \emph{etc.}
\end{enumerate*}).

L'onglet \emph{métadonnées,} permet de renseigner toutes les
informations d'ordre général sur une alerte, telles que l'identité du
requérant (victime, témoin, \emph{etc.}) ou l’activité pratiquée lors
de l'accident. C'est également ici que sont saisies toutes les
informations exogènes, comme la date et l'heure de l'appel ou la
position réelle de la victime.

L'onglet \emph{interprétation
  des expressions} se compose, quant à lui, de 15 colonnes
(\autoref{tab:struct_temp}), que l'on peut regrouper en trois
catégories :
%
\begin{enumerate*}[label=(\alph*)]
\item \emph{extrait},
\item \emph{contexte} et
\item \emph{expression.}
\end{enumerate*}

La première d'entre elles regroupe les colonnes décrivant l'extrait
audio traité, comme les deux colonnes \enquote{\emph{identifiant}}, la
colonne \enquote{\emph{extrait}} et la colonne
\enquote{\emph{timestamp}}. Les colonnes \enquote{\emph{identifiant}}
permettent de donner une référence unique à chaque élément
transcrit. La présence de deux colonnes permet de faire la distinction
entre les \emph{extraits,} qui correspondent à découpage rythmique de
la conversation (changement de locuteur, nouvelle phrase,
\emph{etc.}), et les \emph{expressions,} contenues dans les
\emph{extraits,} correspondant au découpage en \emph{indices de
  localisation.} Par exemple, si la conversation entre le secouriste
(\bsc{s.}) et le requérant (\bsc{r.}) est :
%
\begin{quote}
  \begin{dialogue}
    \Sec Vous êtes au sommet ?
    \Req Non, je suis bien en dessous.
  \end{dialogue}
\end{quote}
%
On identifiera deux \emph{extraits :}
%
\begin{enumerate*}[label=(\alph*)]
\item \enquote{Vous êtes au sommet ?} et
\item \enquote{Non, je suis bien en dessous},
\end{enumerate*}
%
délimités par le changement de locuteur. Si le premier ne contient
qu'une seule \emph{expression,} le second en contient deux, la
négation de la question, dont la sémantique correspond à
\emph{l'expression} \enquote{je ne suis pas au sommet} et
\emph{l'expression} \enquote{je suis bien en dessous}. Ainsi, cet
exemple est composé de deux \emph{extraits} et de trois
\emph{expressions.} Pour faciliter l'analyse et la vérification des
transcriptions, nous avons défini une colonne
\enquote{\emph{extrait}}, permettant de saisir le verbatim de
l'extrait analysé, et une colonne \enquote{\emph{timestamp}} indiquant
la position de l'extrait dans le fichier audio correspondant.

La seconde catégorie regroupe les colonnes permettant de décrire le
contexte de chaque expression. On y retrouve la colonne
\enquote{\emph{locuteur}}, renseignant sur la personne ayant prononcé
l'extrait étudié (\eg secouriste, requérant) et la colonne
\enquote{\emph{confiance}}, permettant au transcripteur (dans le cas
où il s'agit d'un secouriste) de juger de la plausibilité de
l'information donnée par l'extrait. Si nous distinguons ces deux
colonnes des \emph{métadonnées} de l'extrait (\ie le groupe de
colonnes précédent) c'est car ces dernières demandent une
interprétation de l'extrait, contrairement aux \emph{métadonnées.}

Enfin, la troisième catégorie, qui regroupe la majorité des colonnes,
permet l'interprétation et la saisie de chaque expression. On y
retrouve les colonnes \emph{verbe,} \emph{sujet,} \emph{relation de
  localisation} et \emph{objet de référence,} correspondant aux
éléments du même nom dans la formalisation des \emph{indices de
  localisation} et les \emph{modifieurs} qui y sont associés. Les
colonnes \enquote{\emph{objet de référence}} et
\enquote{\emph{modifieur de l'objet de référence}} ont la
particularité de pourvoir être doublées (voir plus si nécessaire) dans
le cas où la \emph{relation de localisation} le nécessiterait.

% Exemple
Si l'on reprend le second extrait de l'exemple précédent (\ie
\enquote{Non, je suis bien en dessous}), deux extraits sont à
saisir. Le premier est la négation de la question du secouriste (\ie
\enquote{Vous êtes au sommet}), que l'on peut interpréter comme
\emph{l'indice de localisation} \enquote{je ne suis pas au
  sommet}. Ici le verbe de l'expression est \enquote{être}, le sujet
est le requérant, la \emph{relation de localisation} est la
préposition \enquote{à} et l'objet de référence est
---~implicitement~--- \enquote{le sommet} \footnote{Le \enquote{à le}
  est contracté en \enquote{au} dans l'extrait}. La forme négative de
la phrase est transcrite avec un \emph{modifieur du verbe} et les
autres modifieurs ne sont pas renseignés. Pour la seconde expression
(\ie \enquote{je suis bien en dessous}) le \emph{sujet} et sont
modifieur et \emph{l'objet de référence} et sont modifieur sont
inchangés, il s'agit toujours du \emph{requérant} qui se situe par
rapport au \emph{sommet.} Le \emph{verbe} est toujours \enquote{être},
mais son modifieur disparait, l'expression n'étant pas une
négation. Pour finir la \emph{relation de localisation} devient
\enquote{\emph{en dessous}} et son modifieur est
\enquote{\emph{bien}}.

Enfin, le dernier onglet du template de transcription est destiné à la
saisie détaillée des \emph{objets de référence utilisés.} Comme pour
l'onglet des \emph{extraits,} chaque ligne correspond à une
\emph{expression} issue de l'alerte. Par ailleurs toutes les lignes de
ce troisième onglet doivent être saisies, il y a donc une bijection
entre les enregistrements de l'onglet \enquote{expressions} et ceux de
l'onglet \enquote{objets de référence}. Le but de cet onglet est de
permettre la saisie d'informations complémentaires sur les
\emph{objets géographiques} utilisés dans les expressions, comme les
\emph{objets de référence} ou le \emph{sujet,} notamment dans le cas
où l'expression décrit la position d'un autre objet qu'une personne
(\eg \enquote{le chalet à côté de la cascade}). En effet, lorsqu'on
saisit un \emph{objet géographique} (\eg \emph{l'objet de référence})
dans l'onglet \emph{expression} on utilise le terme employé par le
locuteur dans l'expression (\eg \enquote{La Bérarde}, \enquote{un
  sommet}, \emph{etc.}) sans modification. Ainsi les objets définis
sont désignés par leur nom (ou tout du moins le nom que le locuteur
leur donne) et les objets indéfinis par leur type. L'onglet
\emph{toponymie} permet alors de préciser le type de tous les objets
géographiques, en vue de leur analyse ultérieure. Si l'on reprend
l'exemple précédent (\ie \enquote{Non, je suis bien en dessous}) nous
ne sommes pas en mesure d'ajouter des informations sur \emph{l'objet
  de référence} \enquote{sommet}, son nom étant inconnu.

\begin{table}
  \centering
  \begin{tabular}{L{.3\textwidth}>{\footnotesize}p{.4\textwidth}}
  \toprule \multicolumn{1}{c}{\bfseries Colonne} &
  \multicolumn{1}{c}{\normalsize \bfseries Contenu} \\ \midrule
% Sémantique des relations spatiales
  \addlinespace
  Identifiant de l'extrait & dddd \\
  Identifiant de l'expression & dd \\
  Extrait & Verbatim de la phrase transcrite\\
  Confiance & Permet au transcripteur de saisir sa confiance en
              \emph{l'indice de localisation} (uniquement si la saisie
              est faite par un secouriste)\\
  Timestamp & Début de l'extrait dans le fichier audio source\\
  Locuteur & \\
  Verbe & Verbe utilisé dans \emph{l'indice de localisation}\\
  Modifieur du verbe & \\
  Sujet & Sujet de \emph{l'indice de localisation}\\
  Modifieur du sujet & \\
  \emph{Relation de localisation} & \\
  Modifieur de la relation de localisation & \\
  Objet de référence & Nom ou type de \emph{l'objet de référence}
                       (cette colonne peut être multipliée si la
                       \emph{relation de localisation} est bi ou n-aire)\\
  Modifieur de l'objet de référence & (cette colonne peut être multipliée si la
                                      \emph{relation de localisation} est bi ou n-aire)\\
  Commentaires & Champ permettant au transcripteur de commenter sa
                 saisie ou les indications données par le requérant
                 (erreurs potentielles, fautes de prononciation, \emph{etc.}) \\
  \bottomrule
\end{tabular}

  \caption{Structure de l'onglet \enquote{\emph{expressions}} du
    template de retranscription.}
  \label{tab:struct_temp}
\end{table}

\subsubsection{Analyse des retranscriptions}

Ces deux corpus ont été transcrits séparément. Le premier d'entre-eux
a été transcrit au cours de l'année 2018 par un secouriste du
\ac{pghm} de Grenoble. Ce premier essai a mis en évidence de
nombreuses lacunes dans le processus de retranscription. La version
originale du template n'était pas assez documentée et illustrée. Par
conséquent le processus de saisie était assez délicat et les résultats
étaient très dépendants du transcripteur. Nous avons donc clarifié le
template pour faciliter la transcription ---~aboutissant à la version
présentée ci-dessus~--- et avons corrigé collégialement les
transcriptions pour aboutir à une version qui fasse l'objet d'un
consensus.

Compte tenu des problèmes que nous avions rencontrés lors de la saisie
du premier corpus nous avons adopté une démarche de transcription plus
avancée. La trentaine d'alertes à transcrire a été répartie entre
trois \todo{4?} personnes \autocite{Bunel2019}. Après une première
transcription, les fichiers produits ont été amendés et commentés par
un second transcripteur, en vue d'aboutir, en collaboration avec le
transcripteur initial, à une seconde version. Une fois l'ensemble du
second corpus retranscrit nous avons procédé a une harmonisation
collégiale des saisies, de manière à obtenir un ensemble de
retranscriptions le plus cohérent possible.

La retranscription des 52 alertes nous a permis d'identifier 374
expressions différentes. 

Pour chaque expression nous avons cherché à définir un concept
permettant d'exprimer la sémantique de la \emph{relation de
  localisation} utilisée. Pour ce faire nous nous 


% Défintion des concepts

% Hiérarchisation des concepts


\paragraph{configuration vs position}



\subsection{Construction des ontologies}

\subsubsection{Cadre des ontologies}

Il a été choisi de développer deux ontologies complémentaires pour
formaliser les \emph{relations de localisation.} La première d'entre
elles est \emph{l'ontologie des relations de localisation}
\acp{orl}. Son objectif est de constituer un thésaurus des
\emph{relations de localisation} utilisées dans le contexte de la
description d'une position en montagne. Chacune des notions est
définie et une grande partie d'entre elles sont illustrées par des
extraits des retranscriptions. La seconde ontologie définie est celle
des \emph{relations de localisation atomiques} \acp{orla}, dont
l'objectif est de définir les \emph{relations de localisation
  atomiques,} les décompositions dont elles résultent et le processus
de \emph{spatialisation} \footnote{Ce point spécifique sera détaillé
  dans le \autoref{chap:06}.} Ces deux composantes, fortement liées,
auraient pu être combinées dans une même ontologie. Cependant, alors
que la décomposition est fortement liée à notre travail, le thésaurus
des relations de localisation peut être exploité pour d'autres travaux
au sein du projet Choucas ou à l'extérieur. C'est pourquoi nous avons
décidé de séparer les deux ontologies et de ne diffuser que
l'ontologie des relations de localisation.

\begin{table}
  \centering
  \begin{tabular}{>{\small}L{.3\textwidth}>{\small}p{.4\textwidth}}
  \toprule \multicolumn{1}{c}{\ac{orl}} &
  \multicolumn{1}{c}{\ac{orla}} \\ \midrule
  \addlinespace
  Bla & Blo \\
  Bla & Blo \\
  Bla & Blo \\
  Bla & Blo \\
  \bottomrule
\end{tabular}

  \caption{Éléments de comparaison des ontologies \ac{orl} et
    \ac{orla}.}
  \label{tab:orl_vs_orla}
\end{table}

\subsubsection{L'ontologie des \emph{relations de localisation}}

3 classes principales + visibilité

préciser liens avec bateman

La construction de l'ontologie des relations de localisation s'est
faite en deux étapes. Dans un premier temps nous avons défini les
concepts à partir des \emph{relations de localisation} utilisées dans
les alertes, puis nous avons hiérarchisé ces concepts.


L'ontologie des \emph{relations de localisation} défini 62
\emph{classes,} représentant des \emph{relations de localisation} ou
leurs regroupements (\eg Relations spatiales de contact, de direction,
\emph{etc.}). À ces différentes classes s'ajoutent des
\emph{relations,} permettant de définir des liens entre les
différentes classes définies.

% Classes / concepts abstraits
Les classes définies sont de deux types, elles peuvent représenter des
\emph{relations de localisation} (\eg \enquote{AvoirASaDroite},
\enquote{AuxAlentoursDe}) ou des regroupements abstraits de concepts
(\eg \enquote{RelationSpatialeDeDirection},
\enquote{RelationSpatialeDeContact}). Les classes regroupant
différentes \emph{relations de localisation,} qualifiées de classes
abstraites, sont au nombre de 11, par conséquent 51 classes
représentent effectivement des \emph{relations de localisation} et
donc des \emph{concepts spatialisables.}

% Pas de propriétés
\emph{L'ontologie des relations de localisations} n'emploie qu'un seul
type de lien, \emph{SubClassOf,} une relation asymétrique et
transitive indiquant que la classe \enquote{fille} spécifie la classe
\enquote{mère}. Dans le cas de cette ontologie, 78 relations
\emph{SubClassOf} ont été définies. La définition d'autres relations
ne nous a pas semblé importante.

Ces relations permettent de
construire un arbre des relations de localisation
(\autoref{fig:onth}).


\ac{orl} est fortement liée à l'ontologie GUM-Space, proposée par
\textcite{Bateman2010}. Les \emph{relations de localisation} définies
sont, lorsque possible, liées aux concepts équivalent (ou similaires)
de l'ontologie GUM-Space. Plus spécifiquement deux types de liens sont
définis, la relation \emph{seeAlso,} utilisée 24 fois et la relation
\emph{equivalentClass,} utilisée 8 fois. Cette dernière traduit une
équivalence parfaite entre concepts, alors que la relation
\emph{seeAlso} est avant tout informative et traduit une certaine
proximité sémantique entre deux concepts sans qu'ils soient
équivalents ou même similaires. Ainsi, seules 8 des 62 \emph{relations
  de localisation} définies dans \ac{orl} sont identiques à celles
définies par \textcite{Bateman2010}, les autres concepts sont définis
de manière significativement différente.


- Structure orienté utilisateur (concepts fortement liés à
l'acceptation des utilisateurs)

- Hiérarchie croisée

- Règle de différentiation

\begin{figure}
  \centering
  %\begin{tikzpicture}[
  every node/.style={anchor=west,
    font={\tiny\ttfamily\textcolor{red}{orl\#}\sffamily}, 
    % text width=5cm
  },
  abstraite/.style={gray},
  atomique/.style={blue}
  ],
  % 
  \begin{scope}[
    level distance=7.5mm,
    edge from parent path={(\tikzparentnode.south) |- ($(\tikzparentnode.south)!0.5!(\tikzchildnode.north)$) -|
      (\tikzchildnode.north)},
    level 1/.style={sibling distance=55mm, anchor=north,align=center},
    ]
    % \tikzstyle{edge from parent}=[draw,red,thick]
    % Parents
    \node[abstraite](rl){RelationDeLocalisation}
    child{node[level 1, abstraite] (r1){Re\-la\-ti\-on\-Spa\-ti\-ale\-Fonc\-ti\-o\-nel\-le}}
    child{node[level 1, abstraite] (r2){Re\-la\-ti\-on\-Spa\-ti\-ale\-Po\-si\-tion\-Re\-la\-ti\-ve}}
    child{node[level 1, abstraite] (r3){RelationSpatialeProximité}};
  \end{scope}

  \begin{scope}[grow=right,
    level distance=10mm,
    sibling distance=5mm,
    edge from parent path={(\tikzparentnode.east) -| ($(\tikzparentnode.east)!0.5!(\tikzchildnode.west)$) |- (\tikzchildnode.west)},
    ]
    \node[font=\tiny] at (rl.east){}
    child{node{CibleVoitSite}}
    child{node{SiteVoitCible}};
  \end{scope}

  \begin{scope}[grow via three points={one child at (.5,-.15) and
      two children at (.5,-.15) and (.5,-.45)},
    growth parent anchor=south west,
    edge from parent path={([xshift=.25cm]\tikzparentnode.south west) |-
      (\tikzchildnode.west)}]
    \node[font=\tiny] at (r1.west){}
    child{node(tt){Proximal}}
    child{node{SitueSurItineraire\-Ou\-Re\-seau\-Sup\-port}};
  \end{scope}


  \begin{scope}[grow via three points={one child at (.5,-.15) and
      two children at (.5,-.15) and (.5,-.45)},
    growth parent anchor=south west,
    edge from parent path={([xshift=.25cm]\tikzparentnode.south west) |-
      (\tikzchildnode.west)}]
    \node[font=\tiny] at (r2.west){}
    child{node[abstraite]{RelationSpatialeDeContact}
      child{node{AExtremiteDe}}
      child{node{ALaFrontiereDe}}
      child{node{DansPlanimetrique}
        child{node{DansLaPartie\-Bas\-se\-De}}
        child{node{DansLaPartie\-De\-X\-La\-Plus\-Pro\-che\-De\-Y}}
        child{node{DansLaPartieEstDe}}
        child{node{DansLaPartieHauteDe}}
        child{node{DansLaPartieNordDe}}
        child{node{DansLaPartieOuestDe}}
        child{node{DansLaPartieSudDe}}
      }
    }
    child[missing]{}
    child[missing]{}
    child[missing]{}
    child[missing]{}
    child[missing]{}
    child[missing]{}
    child[missing]{}
    child[missing]{}
    child[missing]{}         
    child[missing]{}
    child[missing]{}
    child{node{Proximal}
    }
    child{node[abstraite]{RelationSpatialeDeDirection}
      child{node{ADroiteDe}}
      child{node{AGaucheDe}}
      child{node{AvoirASaDroite}}
      child{node{AvoirASaGauche}}
      child{node{AvoirDerriereSoit}}
      child{node{AvoirDevantSoit}}
      child{node{DansLaDirectionDe}}
      child{node[abstraite]{RelationSpatialeDeCardinalité}
        child{node{AEstDe}
          child{node{AEstDeExterne}}
          child{node{DansLaPartieEstDe}}
        }
        child[missing]{}
        child[missing]{}
        child{node{AOuestDe}
          child{node{AOuestDeExterne}}
          child{node{DansLaPartieOuestDe}}
        }
        child[missing]{}
        child[missing]{}
        child{node{AuNordDe}
          child{node{AuNordDeExterne}}
          child{node{DansLaPartieNordDe}}
        }
        child[missing]{}
        child[missing]{}          
        child{node{AuSudDe}
          child{node{AuSudDeExterne}}
          child{node{DansLaPartieSudDe}}
        }
      }
    }
    child[missing]{}
    child[missing]{}
    child[missing]{}
    child[missing]{}
    child[missing]{}
    child[missing]{}
    child[missing]{}
    child[missing]{}
    child[missing]{}
    child[missing]{}
    child[missing]{}
    child[missing]{}
    child[missing]{}
    child[missing]{}
    child[missing]{}
    child[missing]{}
    child[missing]{}
    child[missing]{}
    child[missing]{}
    child[missing]{}
    child[missing]{}
    child{node[abstraite]{RelationSpatialeDisjointe}
      child{node{AEstDeExterne}}
      child{node{AOuestDeExterne}}
      child{node{AuNordDeExterne}}
      child{node{AuSudDeExterne}}
      child{node{HorsDePlanimetrique}}
    }
    child[missing]{}
    child[missing]{}
    child[missing]{}
    child[missing]{}
    child[missing]{}
    child{node[abstraite]{RelationSpatialeItinéraire}
      child{node{ApresJalonSurItineraire}}
      child{node{AuDessusJalonSurItineraire}}
      child{node{AvantJalonSurItineraire}}
      child{node{SitueSurItineraireOuReseauSupport}}
      child{node{SousJalonSurItineraire}}
    }
    child[missing]{}
    child[missing]{}
    child[missing]{}
    child[missing]{}
    child[missing]{} 
    child[missing]{} 
    child{node[abstraite]{RelationSpatialeTernaire}
      child{node{DansLaPartieDeXLaPlusProcheDeY}}
      child{node{DeAutreCoteDeParRapportA}}
      child{node{DuMemeCoteQueParRapportA}}
      child{node{EntreXetY}}
    }
    child[missing]{}
    child[missing]{}
    child[missing]{}
    child[missing]{}
    child[missing]{} 
    child{node[abstraite]{RelationSpatialeVerticale}
      child{node{ALaMemeAltitudeQue}}
      child{node{AuDessusALAplombDe}}
      child{node{AuDessusAltitude}}
      child{node{AuDessusJalonSurItineraire}}
      child{node{AuDessusProche}}
      child{node{DansLaPartieBasseDe}}
      child{node{DansLaPartieHauteDe}}
      child{node{SousALAplombDe}}
      child{node{SousAltitude}}
      child{node{SousJalonSurItineraire}}
      child{node{SousProcheDe}}
      child{node{SousRecouvertPar}}
    };
  \end{scope}

  \begin{scope}[grow via three points={one child at (.5,-.15) and
      two children at (.5,-.15) and (.5,-.45)},
    growth parent anchor=south west,
    edge from parent path={([xshift=.25cm]\tikzparentnode.south west) |-
      (\tikzchildnode.west)}]
    \node[font=\tiny] at (r3.west){}
    child{node{ADistanceQuantitativeRelative}}
    child{node{ADistanceTemps}
      child{node{ATempsDeMarche}}
    }
    child[missing]{}
    child{node{AuDessusProche}}
    child{node{AuxAlentoursDe}}
    child{node{DistanceQuantitativePlanimetrique}}
    child{node{PresDe}}
    child{node{Proximal}};
  \end{scope}

  % Légende
  \begin{scope}
    
  \end{scope}
  
\end{tikzpicture}	

  \caption{Onto}
  \label{fig:ontho}
\end{figure}

\subsubsection{L'ontologie des \emph{relations de localisation atomiques}}

Contrairement à \emph{l'ontologie des relations de localisation} dont
elle est une spécialisation, \emph{l'ontologie des relations de
  localisation atomiques} est plus proche d'une logique
computationnelle.

Pour la construire nous nous sommes basés sur

- décrire relations atomiques définies

DansLaDirectionDe
ADistanceTemps
Cardinalité (verifier)
même alt
Direction DE
Dans plani
Distance Quanti
Not a la frontiere de
a la frontiere de
altEq
AltSup
Contact
dedans
int
ext
sous altitude
sous proche de 

%%% Local Variables:
%%% mode: latex
%%% TeX-master: "../../../../main"
%%% End:
