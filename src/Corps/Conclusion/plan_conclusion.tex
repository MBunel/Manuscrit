\addsec{Résumé de la thèse}

L'objectif de cette thèse de doctorat est de définir (et de proposer
une implémentation) une méthode permettant d'identifier la (ou les)
position correspondant a une description fondée sur un référentiel
indirect, avec pour finalité d'aider les secouristes en montagne à
localiser des personnes perdues ou blessées.

La première partie (\ref{part:01}) de cette thèse était dédiée à la
présentation du contexte applicatif, organisationnel et scientifique
de cette thèse. Nous y avons présenté le fonctionnement des secours en
montagne français, leur manière de procéder pour localiser des
personnes perdues en montagne et les limites de cette méthode. Nous
avons également présenté de projet de recherche CHOUCAS et les
propositions qu'il porte pour aider les \emph{unités de secours en
  montagne} dans leur tâche de localisation. Enfin, nous avons défini
la problématique de notre thèse et le contexte scientifique dans
lequel elle s'inscrit, le \emph{Geographic Information Retrieval.}

Dans la seconde partie (\ref{part:02}) de ce manuscrit, nous avons
défini une méthode permettant de transformer une description de
position en une zone de localisation compatible et de combiner ces
différentes zones afin d’identifier la position décrite par le
requérant lors d'une alerte. Cette méthode est basée sur un processus
en trois phases : la décomposition des indices de localisation,
présentée dans le \autoref{chap:05} la spatialisation de ces mêmes
indices (\autoref{chap:06}) et la fusion des zones de localisation
(\autoref{chap:07}).

Enfin, la dernière partie (\ref{part:03}) de cette thèse était
consacrée à l’application de notre méthode a des cas réels. À partir
d'enregistrements d'alertes passées, nous avons constitué un ensemble
d'indices de localisation, simulant l'action d'un secouriste durant la
phase d'alerte, et les avons spatialisés et combinés à l'aide des
concepts définis dans l'ontologie des relations de localisation.

\addsec{Résumé des contributions}

La principale contribution de cette thèse est la définition et
l'implémentation d'une méthode de transformation des positions
exprimées dans un référentiel indirect en des positions exprimées dans
un référentiel direct.
%
Le développement de cette méthode nous a nécessité l 

% spatialisation des indices
Cette méthode offre un cadre permettant de modéliser des indices de
localisation 

% Construction ORL et ORLA
Nous avons pris la décision d'adopter une démarche orientée
connaissances, ce qui nous a conduit à définir une ontologie,
\ac{orl}, regroupant l'ensemble des relations de localisations
utilisées dans le notre contexte.

La mise en place de la méthode de spatialisation a nécessité le
développement d'une ontologie 

% modélisation de l'imprécision

%% Modèle de représentation des objets spatiaux imprécis 

% modélisation de l'incertitude

% Fusion des indices

% Prototypage implémentation
Enfin, nous avons proposé une implémentation de cette méthode.

La formalisation d'une méthode de spatialisation des descriptions de
positions est complétée par son implémentation au sein d'un prototype.



% Enseignements

Le développement de RUITOR a quant à lui permi de mettre en évidence
les problèmes que pouvait poser l'implémentation de notre méthode.

\addsec{Perspectives}

Comme l'ont montré les applications de note méthode a des cas réels
(\autoref{chap:09}), de nombreuses pistes d'évolutions sont
envisageables.

On peut identifier trois grandes catégories d'améliorations
potentielles, celles concernant l'affinement des paramètres de la
méthode, celles permettant d'étendre notre méthode, lui permettant de
traiter plus finement des situations plus complexes et celles qui ne
pourront naitre que d'un retour des secouristes.

\subsection*{Calibration de la méthode}

Une première piste d'amélioration générale est de travailler sur la
question de la paramétrisation de la méthode. 


% Seuils de la confiance

Enfin, les seuils proposés pour la modélisation de la confiance ont
été sélectionnés de manière empirique

% La sélection de la maille

Un second problème de paramétrisation est celui du choix de la
résolution du calcul. Comme nous l'avons expliqué dans le
\autoref{chap:06}, la représentation raster des \ac{zlc}, implique de
définir une résolution de maille, qui impacte la précision du
résultat, mais également le temps de calcul et le volume du
résultat. Pour les modélisations présentées dans le \autoref{chap:09},
où nous n'étions pas contrains par les temps de calcul, nous avons
pris la décision de travailler à l'aide de la résolution la plus fine
possible. Toutefois


De nombreux facteurs peuvent par ailleurs influer sur le temps de
calcul, comme le nombre d'indices de localisation dans l'alerte
traitée, la complexité des métriques à calculer et le nombre d'objets
de référence candidats.

%% La sélection des seuils

Mais la principale limite d
%
Une autre limite de la méthode proposée est la paramétrisation des
fuzzyficateurs. Comme nous l'avons montré dans le \autoref{chap:09}
cette paramétrisation doit être effectuée manuellement, en fonction de
considérations empiriques. Or, si l'utilisation de fuzzyficateurs
spécialisés comme \onto[orla]{Sup\-Val\-0} ou \onto[orla]{Eq\-Val\-0}
facilite l'utilisation de certaines relations de localisation (\eg
\onto[orl]{Alt\-Inf} il est toujours nécessaire de fixer la valeur
d'un paramètre \(\delta\). Si cette tâche n'est que peu contraignante
pour l'exercice de modélisation effectué dans le \autoref{chap:09}, il
nous semble difficile de demander aux secouristes de paramétrer les
fuzzyficateurs durant la phase de secours.

Une première solution serait de définir un paramétrage par défaut des
fuzzyficateurs, mais cette solution n'est pas envisageable, un même
fuzzyficateur pouvant être couplé avec plusieurs métriques, qui ne
partagent donc pas leurs unités.
%
On peut raffiner cette solution en rattachant la paramétrisation à la
relation de localisation et non plus au fuzzyficateur. Pour donner un
exemple, avec cette solution toute utilisation de la relation de
localisation \onto[orl]{Sous\-Altitude} utilisera le même paramétrage
implicite du fuzzyficateur \onto[orla]{Inf\-Val\-0}, mais ce
paramétrage pourra être différent lorsque ce fuzzyficateur est appelé
par une autre relation de localisation, comme
\onto[orla]{Dans\-Planimetrique}. Bien qu'elle soit plus
satisfaisante, cette seconde solution ne règle pas tous les problèmes
qui peuvent être posés par la question de la paramétrisation. Peut-on,
par exemple, considérer que la relation de localisation
\onto[orl]{Pres\-De} traduit le même ordre de grandeur lorsqu'elle
s'applique à une maison ou à une ville ? Des relations de localisation
comme \onto[orl]{Aux\-Alentours\-De} intègrent même cette notion dans
leur définition. Il nous semble donc indispensable de prendre
également en compte la taille, voir la saillance, de l'objet de
référence lors de la spatialisation. Les principes de décomposition et
de modélisation autonome (\autoref{chap:04}) permettent, en effet,
d'utiliser un paramétrage spécifique pour chaque objet de référence,
ce qui permettrait d'aboutir à une modélisation beaucoup plus fine.

La définition d'une méthode de paramétrisation des fuzzyficateurs
prenant en compte la nature et les caractéristiques (géométriques ou
non) des objets de référence pose toutefois de nombreux problèmes
scientifiques. Il est en effet nécessaire d'identifier les
caractéristiques des objets influent ces paramètres, d'identifier si
le type d'objet à une influence sur ces paramètres, \emph{etc.}

Une des pistes envisagées est de recourir a des méthodes
d'apprentissage, alimentées par des données recueillies lors
d'entretiens ou de sondages. On pourrait, par exemple, demander a un
panel de randonneurs comment ils décriraient une position donnée. Il a
même été envisagé d'organiser ce type d'études a l'aide d'un
environnement en réalité virtuelle, qui permettrait aux personnes
interrogées de décrire plusieurs positions. Toutes ces solutions ont
leurs propres limites, mais elles offrent des pistes de réflexion
intéressantes quant à la question de la paramétrisation du
fuzzyficateurs.

\subsection*{Extension, extensions}

 %L'évaluation de la qualité des \ac{zlp}

Comme l'étude du \emph{fil rouge} l'a montré (\autoref{chap:09}), la
\emph{zone de localisation probable} produite par notre méthode peut,
en fonction des indices de localisation, avoir des configurations très
différentes. Elle peut être d'un bloc, ou fragmentée. Dans cette
seconde situation,  e


Différentes pistes ont longuement été explorées.

Une approche, fondée sur la théorie des fonctions de croyance
\autocite{Shafer1976} et inspirée des travaux de
\textcite{Olteanu2008},


%% Décomposition des objets de référence

Cette piste, que nous avons déjà explorée \autocite{Bunel2019a},


%% Incomplétude de la base

%% Objets de référence flous

\tdi{Parler de l'avantage de manipuler directement des objets flous,
  par exemple pour la durée de déplacement (bourg d'oisans fil rouge}

Une autre perspective importante de ce travail serait d'étendre notre
méthode de manière a ce qu'elle puisse traiter des objets de référence
imprécis.

Comme l'ont montré certains indices de localisation traités dans le
\autoref{chap:09}, la représentation géométrique de certains objets de
référence, notamment les formes du relief, peu être sommaire. Des
objets très entendus, comme les \emph{combes,} les \emph{pas} ou les
\emph{vallées} peuvent, par exemple, être représentés par des points.


La possibilité de travailler à partir d'objets de référence imprécis
offre également la possibilité d'utiliser les \ac{zlc} produites par
notre méthode comme objets de référence de nouveaux indices de
localisation.


\subsection*{Confrontations}

% Réduire / augmenter la précision

% Pertinance / Non pertinence de relations


% Ajout de relations
Portée sonore

Passage ombre

antenne téléphonique

Une première perspective d'amélioration serait de proposer de
nouvelles relations de localisation adaptées aux besoins métier. Le
\autoref{chap:09} et plus particulièrement la modélisation du
\emph{fil rouge} nous a permis de constater que certains indices très
spécifiques comme \enquote{la victime entend des voitures} ou
\enquote{le téléphone de la victime est connectée à une antenne
  téléphonique donnée}, n'étaient pas correctement modélisés avec les
concepts que nous avions développés.
%
Nous avons donc proposé de compléter l'ontologie des relations de
localisation spécifiquement dédiées a la modélisation de ce type
d'indices
%
La spatialisation de telles relations de localisation est également
assez 


%%% Local Variables:
%%% mode: latex
%%% TeX-master: "../../main"
%%% End:
