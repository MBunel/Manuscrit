L’application de notre méthode à des cas réels nécessite d'aborder
trois questions supplémentaires.

Il est tout d'abord nécessaire de procéder à l'implémentation de la
méthode que nous avons définie.

Puis il est nécessaire de sélectionner les alertes à modéliser

Enfin, il convient de définir les données utilisées lors de la
spatialisation.

\subsection{L'implémentation de la méthode de spatialisation}

L'implémentation de la méthode proposée a été réalisée à l'aide du
langage de programmation python, dans sa version 3.5, de septembre
2015. Le choix de ce langage de programmation a été réalisé, d'une
part en fonction de nos compétences personnelles, mais également car
il s'agit d'un langage de programmation répandu dans le monde de la
géomatique et disposant de nombreux bibliothèques facilitant le
développement.

Plusieurs cadres étaient possibles pour mettre en place

Une première solution est d'employer le framework d'un logiciel SIG,
comme QGis, ArcGis, ou d'autres, ce qui permet d’accéder aux fonctions
internes au logiciel, ce qui, dans notre cas, peut faciliter le
développement des fonctions de spatialisation. Cette solution nous
rend cependant dépendant du logiciel de SIG utilisé, ce qui a deux
défauts majeurs. D'une part il est nécessaire de se familiariser avec
les fonctions spécifiques du logiciel, ce qui ajoute un coup de
développement et d'appropriation du code supplémentaire. De plus cela
ajoute une dépendance forte au code, ce qui est un frein
supplémentaire.

Nous avons donc choisi d'utiliser une solution totalement indépendante
des logiciels de SIG, uniquement fondée sur des bibliothèques de bas
niveau.


Les trois phases de notre méthode traitant d'objets différents (\ie
des \emph{indices de localisation} dans la phase de décomposition et
des \ac{zlc} dans la phase de spatialisation et de fusion), il en
résulte que les bibliothèques qui 

\subsubsection{Implémentation de la phase de décomposition}

La phase de décomposition


\subsubsection{Implémentation des phases de spatialisation \& de fusion}

Les objets géographiques sont manipulés (pour les \emph{objets de
  référence}) et construits (pour les zones de localisation
compatibles et probables) à partir de l'étape de spatialisation.

La mise en place de ces étapes nécessite donc de mettre en place de
nouveaux outils pour manipuler les géométries.

Notre travail faisant une grande utilisation de rasters, la
bibliothèque permettant leur manipulation prend une place centrale
lors de cette étape.

Nous avons construit notre implémentation autour de la bibliothèque
\emph{numpy,} conçue pour le calcul scientifique et plus précisément
la manipulation optimisée de matrices volumineuses
\autocite{vanderWalt2011}. Le type de données défini par \emph{numpy}
\footnote{Le \emph{ndarray,} pour
  \foreigntextquote{english}{N-dimensional array} (Tableau
  multidimensionnel).} fait office de standard du calcul scientifique
dans l'univers du python, il est donc à la base de nombreuses
bibliothèques de calcul scientifique.  La bibliothèque \emph{numpy}
permet d'effectuer de nombreux calculs avancés, toutefois certaines
méthodes spécifiques n'y figurent pas, c'est pourquoi nous avons
également fait appel aux bibliothèques \emph{SciPy} et
\emph{scikit-image} \autocite{vanderWalt2014,Virtanen2020} qui
proposent des outils supplémentaires tout en opérant de à partir des
types de données définis par \emph{numpy.}



Owlready \autocite{Lamy2017}


\subsection{Présentation des alertes}



L'ensemble des retranscriptions et des \emph{templates} de
retranscription qui y sont associés sont présentés dans
l'\autoref{anx:retrans}.


\subsubsection{Statistiques}

Pour justifier le choix des alertes nous avons fait des stats 

\subsection{Données}

L'ensemble des \emph{objets de référence} utilisés pour la
spatialisation des indices de localisation proviennent de la même base
de données, la BDTOPO, produite par l'IGN. Cette dernière
%
Pour nos modélisations nous avons travaillé exclusivement à partir du
millésime de 2018


Une base de données comme la BDTOPO, ne propose pas une représentation
exhaustive et à l'échelle unitaire du territoire qu'elle
cartographie. Tous les objets du territoire n'y figurent pas et les
géométries qui les représentent ne sont que des abstractions
simplifiée de la forme réelle de l'objet.

Le principal problème d'une telle base de donnée est donc que certains
objets du réels, pouvant donc être utilisés comme \emph{objets de
  référence} dans un \emph{indice de localisation,} n'y figurent pas
nécessairement.


La BDALTI est la composante altimétrique du référentiel à grande
échelle de l'IGN.



%%% Local Variables:
%%% mode: latex
%%% TeX-master: "../../../../main"
%%% End:
