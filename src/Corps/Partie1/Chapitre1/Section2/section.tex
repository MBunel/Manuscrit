Une des solutions explorées par les secouristes pour améliorer la
gestion de ces cas critiques a été de participer à la mise en place
d'un projet de recherche destiné à développer des solution innovantes
visant à faciliter la localisation de personnes perdues en montagne.


\subsection{Origine du projet}
\label{subsec:1-2-1}

Le cas du \emph{fil rouge} a mis en évidence certaines limites dans la
méthode de localisaiton des victimes par les \ac{usem}, a savoir que,
dans le cas où la solution \emph{Gend'Loc} n'est pas utilisable, il
peut s'avérer extrèmement difficile de localiser une personne perdue
ou bléssée en montagne. Une réflexion a donc été amorcée au sein du
\ac{pghm} du Grenoble pour trouver solutions permettant d'améliorer la
gestion de ces cas, qui bien que minoritaires, sont
problématiques. Une des voies suivie a été d'organiser des
collaborations avec des laboratoires de recherche pour mettre en place
une réflecion autour de ces questions.

Une de ces collaborations a abouti à création du projet de recherche
Choucas \footnote{\url{http://choucas.ign.fr/}}, dont la note
d'intention est d'appliquer l'expérience des chercheurs de l'IGN dans
la manipulation et le traitement de \emph{données géographiques
  vagues} au secours en montagne. 


\subsection{Problématique du projet}
\label{subsec:1-2-2}

La résolution du cas \emph{fil rouge} s'est heurtée à deux problèmes
principaux, le premier est celui de la disponibilité des données

Le premier point exploré est celui de l'enrichissement des données. En
effet, une partie des descriptions utilisées par les victimes pour
décrire leur position peut se référer à des objets qui n'existent pas
dans les bases de données géographiques utilisées (ou utilisables). De
plus, si certains objets sont présents dans les bases leur
qualification n'est pas forcément suffisamment précise pour que l'on
puisse leur correspondre une description qui en serait donnée par la
victime. Il est, par exemple, possible qu'un chemin soit décrit comme
\enquote{pentu}, \enquote{en lacets} ou encore \enquote{en
  balcon}. Ces qualificatifs ne se retrouvent pas dans les bases de
données géographiques et il n'est donc pas possible de sélectionner
des objets suivant ces qualificatifs. L'enrichissement de cette base
de données permettrait donc d'élargir la richesse des sélecions.

Le second est celui du raisonnement

\tdi{parler du fait qu'on ne veut pas automatiser mais assister le
  secouriste}

\subsection{Objectifs scientifiques du projet}
\label{subsec:1-2-3}

La réponse à la problématique du projet s'est organisée autour de
quatres objectifs scientifiques principaux.

\subsubsection{Structuration des données issues de sources textuelles
  hétérogènes}

Le premier de ces objectifs est l'extraction et la structuration de
donées issues de sources textuelles. Comme nous l'expliquoins
précdement les bases de données géographiques ne contiennent pas
l'ensemble des objets utilisés pour décrire une position en
montagne. Si on y trouve la plupart des sentiers et des sommets, il
est moins fréquent de trouver des représentations de barres rocheuses,
ou de vires, alors qu'il s'agit de points de repères utilies sur le
terrain. L'ajout de ce type d'objet dans la base de données utilisées
par les secours permettrait d'affinier ou de faciliter la localisation
des victimes.

Pour ce faire, il a été proposé d'extraire ce type d'information à
partir de descriptions textuelles d'itinéraires de randonnée. En
effet, la majorité des informations disponible sur les itinéraires de
randonnée est présentée (et diffusée) sous forme de texte. Notamment
par le biais de \emph{topoguides,} de sites collaboratifs
\footnote{Comme le site web \emph{camp2camp.}}, ou encore de
blogs. Les itinéraires y sont généralement présentés sous la forme
d'un texte décrivant principalement les points de bifuraction et les
objets permettant de les repérer, \eg \enquote{prendre à droite à la
  bifurcation située au niveau d'un gros rocher.} Or certains de ces
objets peuvent être absents des bases de données géographiques, que ce
soit à cause de leur taille ou de leur type. Ainsi, certains objets
saillants peuvent être absents des bases de données de référence. Le
développement d'une méthode automatisée d'extraction de ce type
d'objets permettrait donc de compléter les données avec un ensemble
d'objets géographiques utiles dans ce contexte.

\subsubsection{Raisonnement spatial qualitatif flou}

Le second objectif majeur de ce projet de recherche est l’élaboration
de méthodes permettant d'automatiser le processus de localisation à
partir des informations qui en sont données par les victimes, tout en
prennant en compte l'imprécision des descriptions de
positions. L'objectif principal de cette tâche est de proposer une
solution logicielle susceptible d'être utilisée en direct par les
secouristes lors de la phase de localisation. C'est dans cet axe que
s'inscrit notre travail doctoral et nous préciserons donc cet objectif
dans les chapitres suivants.

\subsubsection{Géovisualisation de données multidimensionnelles et
  imparfaites pour la prise de décision}

Un autre point abordé par ce projet de recherche est le développement
d'une interface de géovisualisation permettant aux secouristes
d'exploiter les données produites au sein du projet et d'interagir
avec la solution de spatialisation développé durant notre thèse. Cette
tâche impose de prendre en considération la nature imprécise des
modélisations proposées en amont et d'étudier des solutions de
visualisation adaptées à sa représentation.

\subsubsection{Intégration de sources hétérogènes spatialisables}

Enfin le dernier objectif consite à proposer une structuration de
l'ensemble des données hétérogènes dont disposent les secouristes et
de développer des méthodes et des solutions logicielles permettant de
les interroger.

\missingfigure{infographie structure projet}

%%% Local Variables:
%%% mode: latex
%%% TeX-master: "../../../../main"
%%% End:
