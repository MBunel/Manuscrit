Dans cette seconde partie nous avons présenté la méthode que nous
avons définie pour tranformer une description de position en une zone
de coordonées connues, c'est-à-dire exprimée dans un référentiel
indirect. Nous avons organisé cette méthode selon trois phases,
illustrées par la \autoref{fig:methodo_1}, la décomposition, la
spatialisation et la fusion \autoref{chap:05}. L'objectif de la phase
de décomposition est de transformer \emph{l'ensemble des indices de
  localisation} défini par le secouriste de manière à ce qu'il soit
spatialisable. Cette décomposition s'opère en trois étapes : la
décomposition de l'ensemble d'indices de localisation, la
décomposition des objets de référence non nommés et la décomposition
des relations de localisation. À la suite de ces trois étapes on
dispose d'une liste d'indices de localisation indépendants, qu'il est
possible de spatialiser en parallèle \autoref{chap:05}, de manière a
construire des \emph{zones de localisation} imprécises.

Dans le \autoref{chap:06} nous avons détaillé l'approche retenue pour
représenter les zones de localisation. Nous avons opté pour une
représentation raster de sous-ensembles flous, permettant d'attribuer
à chaque position (échantillonnée par des pixels) de la zone étudiée
un degré, quantifiant l'appartenance de la position à la zone
étudiée.

La méthode de construction de ces zones a été présentée dans le
\autoref{chap:06}. C'est au court de la phase de spatialisation que
les indices de localisation décomposés sont interprétés de manière à
construire les zones de localisation. Nous avons défini un processus
utilisant trois éléments : un \emph{rasteriser,} sélectionnant et
rasterisant la partie de l'objet de référence à utiliser, une
\emph{métrique} qui est une grandeur permettant de quantifier la
sémantique de la relation de localisation modélisée et un
fuzzyficateur, qui interprète la valeur de la métrique de manière à
construire une zone de localisation représentée par un sous-ensemble
flou. Plusieurs de ces trois éléments ont été définis et leur
combinaison permet de définir de nombreuses méthodes de
spatialisation. Nous avons par ailleurs pris
