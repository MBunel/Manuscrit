% Citation début de chapitre
\dictum[Schiller]{So pause ye who would link your fates~\dots}%

\chaptertoc{}

\addsec{Introduction}

Introduction du chapitre

% \cite{Bunel2018}

\section{Secours en montagne}

Le secours en montagne est partagé entre les \glspl{pghm} et les \glspl{pgm},
\gls{crsm}. les \glspl{pghm} et les \glspl{crsm} ne peuvent pas se blairer.

Comme le disait le matcho, Roger Brunet : \enquote{L'espace
  géographique ne se découpe pas plus arbitrairement qu'un poulet à
  table.} Ce qui est, on peut le dire, une véritable \emph{punchline}
géographique.

Malgré des sigles identiques, il convient de ne pas confondre le
\glspl{TCP1} et le \gls{TCP2}.


\section{Le projet Choucas}

\begin{table}
  \centering
  \begin{tabular}{ l !{$\rightarrow$} l} 
    \hline
    Nombre & \num{24415.15625}\\
    Nombre arrondi & \num[round-precision=1]{24415.15625}\\
    Nombre arrondi & \SI[round-precision=1]{24415.15625}{\%}\\
    Nombres & \num{10240x7680} \\
    Angle & \ang{234} \\
    Unité composite & \SI{210}{\km\per\hour} \\
    Plage & \SIrange{10}{25}{\liter} \\
    Liste & \SIlist{0;8;16;32;64}{\mega b} \\

  \end{tabular}
  \caption{test array}
\end{table}


\begin{figure}
  \begin{center}
    \subimport{figures/}{canevas.tikz}
  \end{center}
  \caption{Représentation de la}
  \label{fig:1}
\end{figure}

\begin{table}
  \centering
  \begin{tabular}{ l !{$\rightarrow$} l} 
    \hline
    Nombre & \num{24415.15625}\\
    Nombre arrondi & \num[round-precision=1]{24415.15625}\\
    Nombre arrondi & \SI[round-precision=1]{24415.15625}{\%}\\
    Nombres & \num{10240x7680} \\
    Angle & \ang{234} \\
    Unité composite & \SI{210}{\km\per\hour} \\
    Plage & \SIrange{10}{25}{\liter} \\
    Liste & \SIlist{0;8;16;32;64}{\mega b} \\
  \end{tabular}
  \caption{test array 2}
\end{table}

\begin{carte}
  \begin{center}
    \subimport{figures/}{canevas.tikz}
  \end{center}
  \caption{Représentation de la}
  \label{map:1}
\end{carte}

\addsec{Conclusion}

Conclusion du chapitre

%%% Local Variables:
%%% mode: lualatex
%%% TeX-master: "../main"
%%% End:
