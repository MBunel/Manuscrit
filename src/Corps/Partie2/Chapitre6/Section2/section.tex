Tous les modèles théoriques que nous avons présenté dans l'état de
l'art de cette thèse ne sont pas nécessairement adaptés aux exigences
de modélisation que nous avons formulé.

\tdi{Les modèles discrets (exacts, ensembles épais, autres ?)}

Par exemple, les modèles \enquote{exacts}, comme les modèles
\emph{egg-yolk} \autocite{Cohn1996} ou \emph{min-max}
\autocite{Clementini1996}, imposent une modélisation discrète de
\emph{l'imprécision} spatiale. Or, si ces solutions ont pour avantage
leur grande simplicité, ils ne remplissent pas toutes les conditions
que nous avons identifiées. Tout d'abord, si ces modèles permettent de
modéliser toute forme de \emph{zone de localisation} \footnote{À
  l'exception de zones \emph{nettes,} qui ne sont modélisables qu'avec
  le modèle de \textcite{Bejaoui2009}. Cette caractéristique n'est
  cependant pas limitante dans notre cas, aucune \emph{relations de
    localisation} ne pouvant définir une zone précise.} comme nous le
souhaitons, il leur est impossible de rendre compte de la variation de
\emph{l'imprécision} de la zone frontière. Cette dernière n'étant
définie que par ses bornes supérieures et inférieures. De plus, si les
\emph{zones de localisation} construites par la \emph{spatialisation}
d'une \emph{relation de localisation} sont relativement simples
\footnote{Elles ne sont constituées que de deux polygones marquant les
  limites de la zone frontière (cf. \autoref{chap:03}).} et donc
faciles à interpréter, elles perdent cette propriété après plusieurs
unions ou intersections.
%
Ainsi, si les \emph{zones de localisation compatibles} modélisées avec
un modèle \enquote{exact} pourraient être faciles à interpréter, il
n'en sera pas de même avec la \emph{zone de localisation probable,}
issue de la fusion de toutes ces zones.

Les modèles \enquote{exacts} n'offrent pas la possibilité de modéliser
toutes les caractéristiques que nous avons identifiées, par conséquent
nous ne pouvons utiliser ces différents modèles pour représenter les
\emph{zones de localisation} que nous construisons.

\missingfigure{comparaison décroissance degré d'appartenace en
  fonction de la pente. Voir si possible de le faire avec une arte
  topo et un vrai relief. Illustrer l'hypothèse implicite d'évolution
  linéaire entre les frontières dans les modèles exacts.}

\tdi{Modèles non linéaires et non flous}


Les modèles basés sur la \emph{théorie des sous-ensembles flous}

Tout d'abord, la \emph{théorie des sous-ensembles flous} permet de
donner à chaque position un \emph{degré d'appartenance} allant de 1
(appartenance totale à la \emph{zone de localisation}) à 0
(non-appartenance). Cette approche permet donc de modéliser tout type
de zone \emph{imprécise,} que sa frontière soit très étendue ou non et
que l'appartenance à la \emph{zone de localisation} soit strictement
décroissante ou non.

De plus, les opérations d'union et d'intersection des \emph{zones de
  localisation compatibles} nécessaires à la mise en place de notre
méthodologie peuvent êtres facilement effectuées à l'aide des
opérateurs \emph{t-normes} et \emph{t-conormes} qui renvoient un
\emph{degré d'appartenance.} Ainsi, la sémantique des sous-ensembles
flous est invariante suivant les unions et les intersections. Ce
modèle théorique garanti donc que les résultats seront toujours
interprétables et exprimables dans le même modèle.

Les \emph{sous-ensembles flous} offrent donc un cadre théorique
permettant de répondre à tous les critères de modélisation que nous
avons défini. De plus, ces méthodes ont pour avantage d'avoir été
régulièrement utilisées dans la littérature (cf. \autoref{chap:03}),
ce qui est un atout supplémentaire. Ce modèle apparait comme la
solution la plus pertinente pour modéliser des \emph{zones de
  localisation} et plus généralement des \emph{objets géographiques
  imprécis.} Cependant, de nombreuses implémentations de ce modèle ont
été proposées (cf. \autoref{chap:03}). Il est donc nécessaire de se
pencher sur ces différentes solutions afin d'identifier celle qui est
la plus pertinente pour notre cas d'utilisation.

%%% Local Variables:
%%% mode: latex
%%% TeX-master: "../../../../main"
%%% End:
