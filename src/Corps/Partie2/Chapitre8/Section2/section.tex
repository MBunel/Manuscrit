


\tdi{Dérouler un exemple}

Pour illustrer l'ensemble des \emph{étapes} de la \emph{phase de
  fusion} nous nous fonderons sur l'exemple de la construction de la
\ac{zlp} correspondant à \emph{l'ensemble des indices de localisation}
\enquote{Je suis à l'extérieur, au sud de XXX (qui a été spatialisé
  dans le \autoref{chap:07}) et je suis proche d'un refuge}
(\autoref{fig:exemple_final_fusion}). Pour les besoins de l'exemple,
nous considérerons que le secouriste a attribué une confiance absolue
au premier \emph{indice} \footnote{C'est le choix qui a implicitement
  été fait lors de la \emph{spatialisation} de cet \emph{indice} dans
  le \autoref{chap:05}.} et une confiance moyenne (\ie un coefficient
de \si{0,3}) au second \emph{indice.}


\subsection{La \emph{fusion} des \emph{relations de localisation
atomiques}}

\subsection{La \emph{fusion} des \emph{objets de référence indéfinis}}

\subsection{La \emph{fusion} des indices de \emph{localisation}}


\begin{landscape}
  \begin{figure}[H]
    \centering
    \begin{tikzpicture}
  % grille temp
  % \draw[step=1.0,black,thin] (0,0) grid (20,-12);
  % Arrow
  \begin{scope}
    % \path[draw, -, shorten >=5pt, shorten <=5pt] (-1,0) -- (-5,-2);
    % \path[draw, ->,shorten >=5pt, shorten <=5pt] (8.25,3) -- (10,3);
    % \path[draw, -,shorten >=12pt, shorten <=5pt] (1,-3) |- (3.65,-1);
    % \path[draw, ->,shorten >=5pt, shorten <=5pt] (8.25,-1) -- (10,-1);
  \end{scope}


  %%%%%%%%%%%%
  % Colonne 1
  %%%%%%%%%%%%
  %
  %
  % Zone de localisation compatible 1
  \begin{scope}[local bounding box=zlc1]
    \begin{scope}
      \foreach \x in {0,.25,...,1.75}{ \foreach \y in {0,.25,...,1.75}
        {
          \fill[RdBu-9-1] (\x+.125,.125+\y) circle (2 pt);
        }
      }
      \path[ffc] (0,0) rectangle (2,2);
    \end{scope}
    \node[text width=3cm, align=center, anchor=north, font=\footnotesize] at (1,-.25)
    {\itshape Zone de localisation compatible \normalfont \textcolor{RdBu-9-1}{\textsf{A}}};
  \end{scope}
  %
  % ZLC 2
  \begin{scope}[yshift=-3.5cm,local bounding box=zlc2]
    \begin{scope}
      \foreach \x in {0,.25,...,1.75}{
        \foreach \y in {0,.25,...,1.75}
        {
          \fill[RdBu-9-9] (\x+.125,.125+\y) circle (2 pt);
        }
      }
      \path[ffc2] (0,0) rectangle (2,2);
    \end{scope}
    \node[text width=3cm, align=center, anchor=north, font=\footnotesize] at (1,-.25)
    {\itshape Zone de localisation compatible \normalfont \textcolor{RdBu-9-9}{\textsf{A\up{C}}}};
  \end{scope}
  %
  % ZlC 3
  \begin{scope}[yshift=-8cm,local bounding box=zlc3]
    \begin{scope}
      \foreach \x in {0,.25,...,1.75}{
        \foreach \y in {0,.25,...,1.75}
        {
          \fill[RdBu-9-9] (\x+.125,.125+\y) circle (2 pt);
        }
      }
      \path[ffc2] (0,0) rectangle (2,2);
    \end{scope}
    \node[text width=3cm, align=center, anchor=north, font=\footnotesize] at (1,-.25)
    {\itshape Zone de localisation compatible \normalfont \textcolor{RdBu-9-9}{\textsf{B}}};
  \end{scope}
  %
  % Zlc 4
  \begin{scope}[yshift=-11.5cm,local bounding box=zlc4]
    \begin{scope}
      \foreach \x in {0,.25,...,1.75}{
        \foreach \y in {0,.25,...,1.75}
        {
          \fill[RdBu-9-9] (\x+.125,.125+\y) circle (2 pt);
        }
      }
      \path[ffc2] (0,0) rectangle (2,2);
    \end{scope}
    \node[text width=3cm, align=center, anchor=north, font=\footnotesize] at (1,-.25)
    {\itshape Zone de localisation compatible \normalfont \textcolor{RdBu-9-9}{\textsf{B}}};
  \end{scope}
  %%%%%%%%%%%% 
  % Colonne 2
  %%%%%%%%%%%% 
  \begin{scope}[xshift=6cm, yshift=-1.75cm]
    \begin{scope}
      \foreach \x in {0,.25,...,1.75}{
        \foreach \y in {0,.25,...,1.75}
        {
          \fill[RdBu-9-9] (\x+.125,.125+\y) circle (2 pt);
        }
      }
      \path[ffc2] (0,0) rectangle (2,2);
    \end{scope}
    \node[text width=3cm, align=center, anchor=north, font=\footnotesize] at (1,-.25)
    {\itshape Zone de localisation compatible \normalfont \textcolor{RdBu-9-9}{\textsf{B}}};
  \end{scope}

  \begin{scope}[xshift=6cm, yshift=-8cm]
    \begin{scope}
      \foreach \x in {0,.25,...,1.75}{
        \foreach \y in {0,.25,...,1.75}
        {
          \fill[RdBu-9-9] (\x+.125,.125+\y) circle (2 pt);
        }
      }
      \path[ffc2] (0,0) rectangle (2,2);
    \end{scope}
    \node[text width=3cm, align=center, anchor=north, font=\footnotesize] at (1,-.25)
    {\itshape Zone de localisation compatible \normalfont \textcolor{RdBu-9-9}{\textsf{B}}};
  \end{scope}
  
  \begin{scope}[xshift=6cm, yshift=-11.5cm]
    \begin{scope}
      \foreach \x in {0,.25,...,1.75}{
        \foreach \y in {0,.25,...,1.75}
        {
          \fill[RdBu-9-9] (\x+.125,.125+\y) circle (2 pt);
        }
      }
      \path[ffc2] (0,0) rectangle (2,2);
    \end{scope}
    \node[text width=3cm, align=center, anchor=north, font=\footnotesize] at (1,-.25)
    {\itshape Zone de localisation compatible \normalfont \textcolor{RdBu-9-9}{\textsf{B}}};
  \end{scope}
  %%%%%%%%%%% 
  % Colonne 3
  %%%%%%%%%%%% 
  \begin{scope}[xshift=12cm, yshift=-1.75cm]
    \begin{scope}
      \foreach \x in {0,.25,...,1.75}{
        \foreach \y in {0,.25,...,1.75}
        {
          \fill[RdBu-9-9] (\x+.125,.125+\y) circle (2 pt);
        }
      }
      \path[ffc2] (0,0) rectangle (2,2);
    \end{scope}
    \node[text width=3cm, align=center, anchor=north, font=\footnotesize] at (1,-.25)
    {\itshape Zone de localisation compatible \normalfont \textcolor{RdBu-9-9}{\textsf{B}}};
  \end{scope}
  %
  \begin{scope}[xshift=12cm, yshift=-9.75cm]
    \begin{scope}
      \foreach \x in {0,.25,...,1.75}{
        \foreach \y in {0,.25,...,1.75}
        {
          \fill[RdBu-9-9] (\x+.125,.125+\y) circle (2 pt);
        }
      }
      \path[ffc2] (0,0) rectangle (2,2);
    \end{scope}
    \node[text width=3cm, align=center, anchor=north, font=\footnotesize] at (1,-.25)
    {\itshape Zone de localisation compatible \normalfont \textcolor{RdBu-9-9}{\textsf{B}}};
  \end{scope}
  %%%%%%%%%%% 
  % Colonne 3
  %%%%%%%%%%%% 
  \begin{scope}[xshift=18cm, yshift=-5.75cm]
    \begin{scope}
      \foreach \x in {0,.25,...,1.75}{
        \foreach \y in {0,.25,...,1.75}
        {
          \fill[RdBu-9-9] (\x+.125,.125+\y) circle (2 pt);
        }
      }
      \path[ffc2] (0,0) rectangle (2,2);
    \end{scope}
    \node[text width=3cm, align=center, anchor=north, font=\footnotesize] at (1,-.25)
    {\itshape Zone de localisation compatible \normalfont \textcolor{RdBu-9-9}{\textsf{B}}};
  \end{scope}
  % Annotations
  \begin{scope}
    \begin{scope}
      \begin{scope}
        \path[ffc] (1.75cm,1.75cm) rectangle (2cm,2cm);
      \end{scope}
      \begin{scope}[yshift=-3.5cm]
        \path[ffc] (1.75cm,1.75cm) rectangle (2cm,2cm);
      \end{scope}
      \begin{scope}[xshift=6cm, yshift=-1.75cm]
        \path[ffc] (1.75cm,1.75cm) rectangle (2cm,2cm);
      \end{scope}
      \node[align=center,anchor=center] (zi) at (4,-1) {$\top(a,b)=c$};
      \path[->, draw] (2,2) -- (zi.west);
      \path[->, draw] (2,-1.5) -- (zi.west);
      \path[->, draw] (zi.east) -- (7.75,.25);
    \end{scope}
    %
    \begin{scope}
      \begin{scope}[xshift=6cm, yshift=-8cm]
        \path[ffc] (1.75cm,1.75cm) rectangle (2cm,2cm);
      \end{scope}
      \begin{scope}[xshift=6cm, yshift=-11.5cm]
        \path[ffc] (1.75cm,1.75cm) rectangle (2cm,2cm);
      \end{scope}
      \begin{scope}[xshift=12cm, yshift=-9.75cm]
        \path[ffc] (1.75cm,1.75cm) rectangle (2cm,2cm);
      \end{scope}
      \node[align=center,anchor=center] (zou) at (12,-6) {$\top(a,b)=c$};
      \path[->, draw] (8,-6) -- (zou.west);
      \path[->, draw] (8,-9.5) -- (zou.west);
      \path[->, draw] (zou.east) -- (13.75,-7.75);
    \end{scope}
    %
    \begin{scope}
      \begin{scope}[xshift=12cm, yshift=-1.75cm]
        \path[ffc] (1.75cm,1.75cm) rectangle (2cm,2cm);
      \end{scope}
      \begin{scope}[xshift=18cm, yshift=-5.75cm]
        \path[ffc] (1.75cm,1.75cm) rectangle (2cm,2cm);
      \end{scope}
      \node[align=center,anchor=center] (zu) at (16,-4) {$\top(a,b)=c$};
      \path[->, draw] (14,.25) -- (zu.west);
      \path[->, draw] (14,-7.75) -- (zu.west);
      \path[->, draw] (zu.east) -- (19.75,-3.75);
    \end{scope}
  \end{scope}
\end{tikzpicture}
    \caption{Gros exemple}
    \label{fig:exemple_final_fusion}
  \end{figure}
\end{landscape}


%%% Local Variables:
%%% mode: latex
%%% TeX-master: "../../../../main"
%%% End:
