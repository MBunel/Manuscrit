\section*{Résumé}

Plus de \num{15000} demandes de secours en montagne sont recensées chaque
année, sur le territoire français, en majorité durant les périodes
estivales et hivernales. Lorsque ces demandes concernent des
intervention à l'extérieur des dommaines skiables elles sont réalisées
par des corps spécialisés, comme les pelotons de Gendarmerie en
haute-montagne (PGHM). Ces secouristes spécialisés sont chargés de
l'opération de secours en tant que telle, mais également de sa
préparation, ce qui implique notamment de délimiter la zone
d'intervention et donc d'identifier la position de la victime. Cette
étape peut s'avérer difficile, car elle nécessite que le requérant
décrive le plus précisément possible sa position, malgré la fatigue,
la panique ou sa méconnaissance de la région. Grace à la
multiplication des téléphones dotés de GPS, des solutions de
geo-localisation plus fiables et précises ont été développées, mais
les secouristes doivent encore procéder fréquement à une localisaiton
manuelle, s'appuyant uniquement sur le discours du requêrant, leurs
connaissances et leur expérience.

L'objectif de cette thèse est de proposer une méthode permettant
d'assister les secouristes dans la situation où la position du
requérant ne peut être identifiée que manuellement. Nous proposons de
développer une méthode permettant d'identifier les zones correspondant
à une description orale de position (\eg \enquote{Je suis à côté d'un lac}),
c'est-à-dire permettant de transformer une position exprimée dans un
référentiel indirect (une description orale) en une position exprimée
dans un référentiel direct, c'est-à-dire décrite par des coordonées,
qu'il est alors possible de cartographier. Le développement d'une
telle méthode se heurte à de nombreux verrous scientifiques, comme la
prise en compte de l'imprécision inhérente au langage naturel, des
potentielles erreurs de description ou l’identification de la
sémantique des prépositions utilisées pour décrire une position en
milieu montagneux.

\vspace{.5cm}

\noindent\textbf{Mots clés :} Géoréférencement indirect, Logique
floue, Analyse spatiale, Fusion d'informations, Raisonnement spatial.\par

\section*{Abstract}

More than 15,000 requests for mountain rescue are registered each
year, in France, particularly during summer and winter. When these
requests concern intervention outside of the ski resorts, they are
carried out by specialised rescuers. These rescuers are responsible of
the rescue operation, but also of its preparation, which involves,
among other things, the delimitation of the intervention zone and the
identification of the victim's position. When possible, the location
of the victim can be obtained by using geo-locations
applications. Nevertheless, there are cases where this is not possible
(\eg the victim has not a smartphone, the telephon has no signal,
etc.) and the rescuers have to locate manually the victims, using
their knowledge and experience but also the information given by the
victim. This case can be difficult, as it requires the victim to
describe her/his position as accurately as possible even in condition
of tiredness, panic or lack of knowledge of the region.

The objective of this thesis is to propose a method to help rescuers
in the situation where the position of the victim can only be
identified manually. We propose to develop a method to identify the
corresponding areas of an oral description of a position (\eg "I'm
beside a lake"), i.e. allowing to transform a position expressed in an
indirect referential (an oral description) into a position expressed
in a direct reference referential, \ie described by coordinates,
which can then be mapped. The development of a such a method comes up
against many scientific issues, such as inherent imprecision of
natural language, the inaccuracies of the potential errors in
describing her/his position or identification of the semantics of the
prepositions used to describe a position in mountainous environment.

\vspace{.5cm}

\noindent\textbf{Keywords :} Indirect georeferencing, Fuzzy logic,
Spatial analysis, Spatial reasoning.\par