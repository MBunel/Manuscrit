\tdi{Voir comment citer les 2 naive physics manifestos}

La \emph{spatialisation} des \emph{indices de localisation} nécessite
de développer des méthodes adaptées



\subsection{Éléments généraux sur la localisation par référencement
  indirect et formalisation d'une relation de localisation}

% Mise au point vocabulaire
Comme nous l'expliquions lors de la présentation du concept
\emph{d'élément de localisation} (\autoref{subsec:2-1}), ce dernier
est composé de trois éléments :
%
\begin{enumerate*}[label=(\alph*)]
\item \label{i:site} un \emph{sujet,}
\item \label{i:cible} un \emph{objet de référence} et
\item une \emph{relation de localisation,}
\end{enumerate*}
%
qui, combinés, permettent d'exprimer un \emph{référencement indirect,}
que nous cherchons à transformer en un référencement direct. Ces trois
concepts n'ont cependant pas été présentés dans tous leurs
aspects. L'objectif de cette partie est donc définir chacun de ces
trois termes, mais également de présenter les différents modèles
proposés dans la littérature pour les formaliser.

\subsubsection{Les concepts de \emph{sujet} et \emph{d'objet de
    référence}}

Toutes les formalisations des \emph{référencements spatiaux indirects}
que nous avons rencontré se fondent sur les trois mêmes concepts de
base, le \emph{sujet,} dont la position est décrite par rapport à la
position d'un \emph{objet de référence} à l'aide d'une relation de
localisation. Toutefois, en fonction des auteurs, ces trois concepts
portent des noms bien différents, comme le montre la synthèse des
différents termes utilisés, proposée par
\textcite{RetzSchmidt1988}. Le vocabulaire varie bien évidement en
fonction de la langue de publication, mais également en fonction de la
discipline des auteurs. En linguistique francophone
\autocite{Vandeloise1986,Borillo1998, Aurnague1997, Mathet2000}, par
exemple, les termes de \emph{site} et de \emph{cible} sont employés
pour désigner, respectivement, l'\emph{objet de référence} et le
\emph{sujet.}
%
La littérature anglophone abonde de termes pour désigner ces deux
concepts, comme \emph{primary object} et \emph{reference object}
employés par \textcite{RetzSchmidt1988, Clementini2013}, \emph{figure}
et \emph{ground} utilisés par \textcite{Talmy1983} ou encore
\emph{trajector} et \emph{landmark,} qui sont la utilisée par
\textcite{Vandeloise1984} des termes \emph{site} et \emph{cible.}
%
Mais sous cette grande disparité de vocabulaire se cachent en réalité
les deux mêmes concepts

\subsubsection{\emph{Relations spatiales, relations de localisation}}

Comme pour les termes de \emph{sujet} et \emph{d'objet de référence,}
différents termes sont utilisés comme synonymes de
\enquote{\emph{relation de localisation}}, comme
\enquote{\emph{relation spatiale}}, qui est le terme le plus
fréquemment utilisé. Toutefois, contrairement aux différents synonymes
de \emph{sujet} et \emph{d'objet de référence}, les termes de
\emph{relation spatiale} et de \emph{relation de localisation} ne sont
pas nécessairement considérés comme
équivalents. \textcite{Duchene2019}, par exemple, font la distinction
entre les \emph{relations spatiales,} qui sont énoncées avec une
préposition spatiale (\eg \emph{sur, sous, dans, à côté,} etc.) et les
\emph{relations de localisation} qui décrivent une position sans
nécessairement faire appel à une \emph{préposition spatiale}, comme
dans les phrases : \enquote{Je suis à cinq minutes de marche} ou
\enquote{Je vois un chalet}. Cependant, cette distinction est souvent
occultée, pour \textcite{Vandeloise1986} toute relation décrivant une
position, avec ou sans \emph{préposition spatiale,} est une
\emph{relation spatiale}. Du point de vue de \textcite{Vandeloise1986}
la sémantique prime sur la morphologie.

De plus, dans certains travaux \autocite{Bateman2010}, les relations
décrivant l'organisation spatiale d'un groupe (\eg \enquote{être
  alignés}) sont considérés comme des \emph{relations spatiales,} bien
que ne décrivant pas une position. Si l'on accepte de qualifier ces
relations, décrivant une organisation intra-groupe, comme des
\emph{relations spatiales} alors on ne peut pas considérer que
l'ensemble des \emph{relations spatiales} est inclus dans l'ensemble
des \emph{relations de localisation,} \ie que toute \emph{relation
  spatiale} et une \emph{relation de localisation.}

Nous pourrions donc utiliser exclusivement le terme de \emph{relation
  spatiale,} cependant, le vocabulaire proposé par
\textcite{Duchene2019} permet de distinguer les relations susceptibles
d'être \emph{spatialisées}, \ie les \emph{relations de localisation}
des \emph{relations spatiales,} qui, si l'on intègre les
configurations intra-groupe, comme le propose \textcite{Bateman2010},
ne le sont pas nécessairement. Ainsi, dans la suite de ce travail nous
parlerons exclusivement de \emph{relations de localisation.}

\subsubsection{\emph{Relations de localisation ternaires}}

Contrairement à ce que pourraient laisser penser les différents
exemples \emph{d'indices de localisation} présentés jusqu'ici, ces
derniers peuvent contenir plus d'un \emph{objet de référence}
\autocite{Clementini2013}. Ce cas, relativement peu fréquent, se
présente lorsque la \emph{relation de localisation} utilisée n'est pas
\emph{binaire,} c'est-à-dire qu'elle n'implique pas qu'un \emph{sujet}
et qu'un \emph{objet de référence.} Le cas le plus fréquent est celui
de la \emph{relation de localisation} \enquote{entre} qui implique, au
minimum, deux \emph{objets de référence,} on parle alors de
\emph{relation de localisation ternaire}. Si l'on peut, en effet, dire
que \enquote{La poste est entre le café et la banque}, il est
incorrect d'utiliser cette \emph{relation de localisation} avec un
seul \emph{objet de référence.} La phrase \enquote{Je suis entre
  l'université}, par exemple, est fautive et incompréhensible.

Cependant, touts les \emph{indices de localisation} composés de trois
objets, n'impliquent pas nécessairement une \emph{relation de
  localisation ternaire} \autocite{Duchene2019}. L'un des objets peut,
par exemple, être le support des deux autres, c'est par exemple
lorsqu'une position est décrite sur un itinéraire, comme dans la
phrase \enquote{le parking avant le pont}. De même, une phrases comme
: \enquote{Il est à l'angle de la rue Saint-Jacques et de la rue
  Pierre et Marie Curie}, qui contient trois objets spatiaux :
\emph{le sujet} (\enquote{Il}) et deux objets de référence
(\enquote{rue Saint-Jacques} et \enquote{rue Pierre et Marie Curie}),
ne contient pas \emph{relation de localisation ternaire}. Dans ces
deux cas on peut recombiner l'énoncé en un ensemble \emph{d'indices de
  localisation} utilisant des \emph{relations de localisation
  binaires.} Par exemple, \enquote{l'angle de la rue Saint-Jacques et
  de la rue Pierre et Marie Curie} peut être considéré comme un seul
\emph{objet de référence,} cette description de position peut
s'interpréter comme deux indices de localisation : \enquote{Je suis à
  $\phi$ et $\phi$ est à l'intersection de la rue Saint-Jacques et de
  la rue Pierre et Marie Curie}. Ces exemples n’intègrent donc pas de
\emph{relations de localisation ternaires.} À l'inverse, des phrases
ne contenant qu'un seul \emph{objet de référence} peuvent utiliser des
\emph{relations de localisation ternaires.} On pourra par exemple dire
: \enquote{La chèvre est entre les arbres}, dans ce cas, seul un
\emph{objet de référence,} \enquote{les arbres}, est présent, mais
celui-ci est composite, \textcite{Aurnague1993} parlent alors de
\enquote{\emph{collection}}.


\subsubsection{Contexte d'interprétation et cadre de référence}

\tdi{citer Spatial representation and updating: Evidence from
  neuropsychological investigations}

\tdi{citer The Role of Context in the Interpretation of Natural
  Language Location Descriptions}

Comme les différents exemples utilisés jusqu'ici le laissent supposer,
certaines \emph{relations de localisation} ne peuvent être
interprétées ---~et donc spatialisées~--- sans une connaissance
supplémentaire sur la configuration de la position décrite. C'est par
exemple le cas dans des phrases telles que : \enquote{Elle est devant
  l'université} ou \enquote{la poste est à gauche}. Les
\emph{relations de localisation} \enquote{devant} et \enquote{à
  gauche} ont une signification dépendante du contexte.

Notre gauche
et notre droite sont dépendantes de notre orientation et le
\enquote{devant} de notre sens de déplacement
\autocite{Vandeloise1986}. On pourra facilement se convaincre qu'il
est illusoire d'interpréter ces \emph{relations de localisation} sans
une connaissance exogène, qui ici prend la forme d'un
\emph{référentiel de directions} (ou \emph{cadre de référence,} selon
\cite{Clementini2013}).

Ces \emph{relations de localisation} dont l'interprétation nécessite
un \emph{cadre de référence} sont généralement qualifiées de
\emph{projectives} et \emph{directionnelles}.

L'interprétation de ces \emph{relations de localisation} est
compliquée par le fait que le \emph{cadre de référence} ne soit pas
nécessairement explicité dans \emph{l'indice de localisation,} comme
dans l'exemple précédent, \enquote{Elle est devant l'université}.

Par exemple, dans la phrase \enquote{il est à ma gauche}, le cadre de
\emph{référence} est explicité. L'article définit \enquote{ma} indique
que le locuteur utilise sa ligne de vue comme référence.

\missingfigure{Figure sur les frames of reference}

\subsubsection{Modélisation formelle d'une relation}

\paragraph{Modèles morphologiques}



\paragraph{Modèles sémantiques}

Cette seconde catégorie de modèles formalise les \emph{relations de
  localisations} indépendamment de toute considération linguistique
et morphologique.

\textcite{Aurngague1997} proposent

% 1
\autocite{Bateman2010}

\paragraph{\enquote{ISO-Space}, un modèle pivot}

Le modèle normalisé \enquote{ISO-Space}, proposé par
\textcite{Pustejovsky2017}, a pour objectif d'être un pivot entre les
\emph{modèles morphologiques,} destinés à l'annotation et les
\emph{modèles sémantiques,} construits pour interpréter les
\emph{relations de localisation.}

Ce modèle 


\subsection{Classification des \emph{relations de localisation}}

\subsubsection{Critères de classification}

Les classification des \emph{relations de localisation} proposées dans
la littérature sont généralement organisées suivant un critère donné.

Mais on trouve également des organisations \enquote{hybrides}, fondées
sur plusieurs critères de classement.

Le choix du critère de classement nous semble fortement lié au domaine
d’application de la classification et plus encore à sa discipline de
rattachement. Dans le domaine de la linguistique, les classifications
sont majoritairement organisées suivant un critère sémantique, les
groupes de \emph{relations de localisation} sont constitués en
fonction de leur sens et non en fonction de critères comme leur
formalisation. À l'inverse, les classifications constituées en vue
d'une implémentation, quel qu'en soit le type, comme celles proposées
par \textcite{Hudelot2008a,Bloch2013}, sont généralement fondées sur
des critères tels que la méthode de formalisation.

\subsubsection{Classifications trouvées dans la littérature}




Dans le domaine de la linguistique plusieurs catégorisations des
\emph{relations de localisation} ont été
proposées. \textcite{Pustejovsky2017} distingue quatre classes de
\emph{relations de localisation} :
%
\begin{enumerate*}[label=(\alph*)]
\item les \emph{relations topologiques},
  \item les relations \enquote{directionnelles ou
  orientationnelles},
  \item \emph{les relations métriques} et
  \item les relations liées au mouvement d'un objet.
\end{enumerate*}
%

\textcite{Bateman2010} distinguent quand à eux les \emph{relations de
  distance}, les \emph{relations fonctionnelles} et les
\emph{relations relatives.}

\textcite{Borillo1998} propose quand à elle une classification selon
deux critères sémantiques, distinguant les \emph{relations spatiales
  statiques} des \emph{dynamiques}, et les \emph{relations spatiales
  internes} (ou \emph{topologiques}) des \emph{relations spatiales
  externes} (ou \emph{projectives}). Le premier critère différencie
les relations décrivant un mouvement (\emph{dynamiques}) de celles
désignant une position invariable dans le temps (\emph{statique}). Le
second critère distingue les \emph{relations spatiales} décrivant une
situation où le sujet partage sa position, ou du moins une partie,
avec l'objet de référence (\ie les relations caractérisées
d'internes), des situations où la position du sujet et de l'objet de
référence sont disjointes (\ie les relations caractérisées
d'externes). Pour \textcite{Borillo1998}, l'ensemble des relations de
localisation est réductible à deux catégories sémantiques, les
\emph{relations topologiques} (ou \emph{internes}) et les
\emph{relations projectives} (ou \emph{externes}). La définition des
\emph{relations topologiques} utilisée par \textcite{Borillo1998} est
donc équivalente à celle que l'on retrouve dans les autres
classifications présentées. La catégories des \emph{relations
  projectives} a, quand à elle, une définition beaucoup plus large que
celle qui en est donnée dans d'autres classifications. Pour
\textcite{Borillo1998}, les relations dérivant une orientation (\eg
\enquote{au nord de}), une distance (\eg \enquote{proche de}) ou une
position sur l'axe vertical (\eg \enquote{sous}), qui sont distinguées
dans d'autres classifications, sont ici regroupés en une seule classe.
% 
Cette catégorisation XXXX

D'autres travaux \autocite{Hudelot2008a} utilisent également une
catégorisation en deux classes, similaire à celle de
\textcite{Borillo1998}, mais basés sur des critères
différents. \textcite{Hudelot2008a}, par exemple, distingues les
\emph{relations topologiques} des autres relations, qualifiées de
\emph{métriques,} classe elle-même subdivisée en deux catégories, les
\emph{relations directionnelles} et de \emph{distance.} Bien que
similaires, ces classes ne sont pas équivalentes à celles proposées
par \textcite{Borillo1998}.
%
\textcite{Bloch2013} proposera une extension de la classification
d'\textcite{Hudelot2008a}, en y ajoutant une nouvelle classe au
premier niveau hiérarchique, celle des \emph{relations de localisation
  complexes,} incluant les \emph{relations ternaires,} \emph{n-aires}
et certaines \emph{relations binaires} comme celle énoncée par la
préposition \enquote{parallèle à}.


Comme le font remarquer \textcite{Duchene2019}, l'ensemble des
classifications présentées ici, qu'elles soient organisés suivant une
logique fonctionnelle ou formelle, proposent toujours une classe bien
définie pour les \emph{relations topologiques.} Les autres catégories
sont plus variables, par exemple, la classe des \emph{relations
  projectives} peut inclure les relations métriques, comme dans la
classification de \textcite{Borillo1998}, ou non, comme dans les
classification de \textcite{Bateman2010, Pustejovsky2017}.

\subsection{\texttt{TITRE}}

\subsubsection{La notion d'ontologie}

La notion \emph{d'ontologie}

Les ontologies se distinguent des classifications précédemment
présentées par leur XXXXXX

\tdi{Gruber (1993), comme la spécification d’une conceptualisation partagée d’un domaine.}
\tdi{(Guarino, 1998) distingue trois niveaux d’ontologies}

\tdi{Présentation de la logique de description et des extensions floues}

% OWL
Les ontologies peuvent être représentées avec une grande variété de
langages.

\textcite{Gruber1993}, utilise par exemple le langage KIF, dont la
syntaxe est inspirée du langage de programmation LISP.

La majeure partie des ontologies récentes sont formalisées à l'aide du
langage OWL, qui est lui-même une extension du langage RDFS. Ces deux
langages, portés par le W3C, ont étés développés dans le cadre du web
sémantique.

Le Langage OWL a comme particularité d'avoir trois dialectes
différents, qui correspondent à autant de types de logiques de
description. Le plus simple d'entre-eux est le langage OWL-lite,
correspondant à la logique de description SHIF(D), puis vient le
langage OWL-DL, correspondant à la logique de description SHOIN(D) et
enfin le langage OWL-full, correspondant à la logique de description
SROIQ(D). Le dialect OWL-DL inclut donc le dialecte OWL-Lite et est
lui-même inclut dans le dialecte OWL-full. L’intérêt de l'existance de
ces trois dialectes est d’offrir la possibilité d'adapter la richesse
de la sémantique et la complexité des calculs à la situation
modélisée.

\subsubsection{Les taxonomies et ontologies de relations de localisations
  existantes}

La définition d'ontologies des relations spatiales

\textcite{Duchene2019} ont identifié 4 ontologies qui se destinent à
la modélisation de \emph{relations de localisation.}

Bien que traitant des mêmes relations spatiales, ces ontologies sont
souvent peu génériques et adaptée qu'au domaine d’application pour
lequel elles ont été initialement conçues \autocite{Hudelot2008a}.

% Bateman
La plus récente, et probablement la plus ambitieuse, d'entre-elles est
l'ontologie GUM-Space (aussi connue sous le nom d'OntoSpace), déjà
présentée \autocite{Bateman2010}.
Cette dernière définit une hiérarchie assez complète des différentes
\emph{relations de localisations.} Le premier niveau hiérarchique de
l'ontologie distingue trois classes :
% 
\begin{enumerate*}
\item les relations de distance (\emph{SpatialDistanceModality}),
\item les relations fonctionnelles
  (\emph{FunctionalSpatialModality}) et
\item les relations relatives (\emph{RelativeSpatialModality}).
\end{enumerate*}
% 
Ces trois catégories ne sont cependant pas mutuellement disjointes, un
même concept peut donc hériter de plusieurs de ces classes. C'est par
exemple le cas des \emph{relations de localisation} \emph{Front, Back,
  Left} ou \emph{RightProjectionExternal,} qui décrivent une situation
où le sujet est à l'extérieur et, respectivement, devant, derrière, à
gauche ou à droite de l'objet de référence. \textcite{Bateman2010}
considèrent que ces concepts se rattachent, à la fois, aux
\emph{RelativeSpatialModality,} puisqu'il s'agit de \emph{relations
  projectives,} mais également aux \emph{SpatialDistanceModality,}
puisqu'une notion de proximité est implicitement présente dans ces
concepts. En effet, on ne dira pas de deux objets très éloignés, que
l'un est \enquote{devant} l'autre.

% Ontoi 2
Contrairement GUM-Space, les ontologies des \emph{relations de
  localisation} proposées par
\textcite{Dasiopoulou2005,Miron2007,Hudelot2008a} ont été construites
pour application spécifique. L'ontologie de \textcite{Dasiopoulou2005}
est, par exemple, destinée à l'identification automatisée d'objets
dans des vidéos. Cette dernière définit des relations topologiques et
directionnelles. 

Quant a ONTOAST \autocite{Miron2007}, il s'agit d'une ontologie ne
définissant que des \emph{relations de localisation} qualitatives. Ces
dernières peuvent êtres topologiques (les relations définies sont
équivalentes à celles du modèle RCC8), directionnelles ou bien de
distance.

% Hudelot 
Enfin, l'ontologie FRSO (\emph{Fuzzy Spatial Relation Ontology}),
proposée par \textcite{Hudelot2008a}, a pour objectif de permettre
l’interprétation d'images et plus spécifiquement la segmentation
automatique d'images médicales. Les auteurs justifient le
développement d'une ontologie des \emph{relations de localisation} par
l'inadaptation des ontologies déjà existantes pour l'interprétation
d'images.
%
La particularité principale de l'ontologie FRSO est qu'elle est basée
sur une extension floue de la \emph{logique de description.}
%
Dans cette ontologies sont distinguées 8 relations directionnelles, 3
relations de distance et 8 relations topologiques sont définies.

\subsection{Travaux sur différentes familles de relations}

Nous allons à présent détailler les différents travaux proposés pour
modéliser des familles spécifiques de \emph{relations de
  localisation.}

\subsubsection{Les relations topologiques}

Les relations topologiques 

Ces différentes familles de \emph{relations de localisation} se
distinguent également par leur complexité. Comme l'indiquent
\textcite{Aurnague1997}, les travaux de \textcite{Piaget1948} sur le
développement de la perception spatiale chez l'enfant ont mis en
évidence que les premières notions développées étaient celles
relatives à la topologie, suivies par les concepts relevant de la
perception des distances, finalement suivies par la notion
d'orientation.

Comme l'indique \textcite{Duchene2019}, ces relations sont
\enquote{invariantes par déformation continue de l’espace}.


\tdi{citer navive geography sur la topologie avant les distances}

\paragraph{Modèles computationnels}

Dans le domaine des \ac{sig}, deux grandes \enquote{familles} de
modèles topologiques coexistent.

La première est celle du modèle RCC8, proposé par
\textcite{Randell1992} et de ses dérivés.

Le modèle RCC-8 est un modèle formalisant les \emph{relations
  topologiques} entre régions de deux dimensions. Dans ce modèle, les
régions sont modélisées comme des \emph{ensembles de points}


Huit relations topologiques et méréologiques y sont définies. Ces
dernières sont mutuellement exclusives, par conséquent une et une
seule d'entre-elles est vérifiée, quelque soit la configuration.

Ce modèle permet également de raisonner sur les relations
topologiques, à l'aide d'une \emph{table de transition.}

\tdi{citer A More Expressive 3D Region Connection Calculus pour l'EDA
  RCC}
\tdi{Ajouter ler JEPD et si les extensions le valident}

% RCC5
% Bennett, 1994, Spatial reasoning with propositional logics
\textcite{Bennett1994} propose un premier modèle dérivé du travail de
\textcite{Randell1992}, le modèle RCC-5, réduisant, comme son nom
l'indique, le nombre der \emph{relations topologiques} modélisées à
5. Les relations DC et EC sont regroupées en une nouvelle relation, DR
(\enquote{\emph{discrete}}), indiquant que deux régions ne partagent
pas leurs aires. De même, les relations TPP et NTPP (et leurs
inverses) sont combinées en une seule relation (et son inverse), PP,
pour \enquote{\emph{proper parts}}. En réduisant le nombre de
\emph{relations topologiques} modélisées le modèle RCC-5 perd en
expressivité, cependant l'objectif annoncé pour ce modèle n'est pas
d’enrichir la sémantique du modèle RCC-8 (ce qui est le but de la
majorité des extensions), mais de faciliter le raisonnement automatisé
\textcite{Bennett1994}.

\begin{figure}
  \centering
  \usetikzlibrary{calc}
\begin{tikzpicture}

  \begin{scope}[local bounding box=scope1]
    \path[ffa] (0,0) circle [radius=15pt];
    \path[ffa2] (1.25,0) circle [radius=15pt];
    
    \path[ffc] (0,0) circle [radius=15pt] node{A};
    \path[ffc2] (1.25,0) circle [radius=15pt] node{B};
  \end{scope}

  \begin{scope}[local bounding box=scope2, shift={($(scope1.east)+(1cm,0)$)}]
    \path[ffa] (0,0) circle [radius=15pt];
    \path[ffa2] (30pt,0) circle [radius=15pt];
    
    \path[ffc] (0,0) circle [radius=15pt] node{A};
    \path[ffc2] (30pt,0) circle [radius=15pt] node{B};
  \end{scope}

  \begin{scope}[local bounding box=scope3, shift={($(scope2.east)+(1cm,0)$)}]
    \path[ffa] (0,0) circle [radius=15pt];
    \path[ffa2] (0.6,0) circle [radius=15pt];
    
    \path[ffc] (0,0) circle [radius=15pt] node{A};
    \path[ffc2] (0.6,0) circle [radius=15pt] node{B};
  \end{scope}

  \begin{scope}[local bounding box=scope4, shift={($(scope3.east)+(1cm,0)$)}]
    \path[ffa] (0,0) circle [radius=15pt];
    \path[ffa2] (-5pt,0) circle [radius=10pt];
    
    \path[ffc] (0,0) circle [radius=15pt] node{A};
    \path[ffc2] (-5pt,0) circle [radius=10pt] node{B};
  \end{scope}

  \begin{scope}[local bounding box=scope5, shift={($(scope4.east)+(1cm,0)$)}]
    \path[ffa] (0,0) circle [radius=15pt]; node {A}; % Aire
    \path[ffa2] (0,0) circle [radius=10pt];
    
    \path[ffc] (0,0) circle [radius=15pt] node{A};
    \path[ffc2] (0,0) circle [radius=10pt] node{B};
  \end{scope}

  \begin{scope}[local bounding box=scope6, shift={($(scope5.east)+(1cm,0)$)}]
    \path[ffa2] (0,0) circle [radius=15pt];
    \path[ffa] (-5pt,0) circle [radius=10pt];
    
    \path[ffc2] (0,0) circle [radius=16pt] node{B};
    \path[ffc] (-5pt,0) circle [radius=10pt] node{A};
    
  \end{scope}

  \begin{scope}[local bounding box=scope7, shift={($(scope6.east)+(1cm,0)$)}]
    \path[ffa2] (0,0) circle [radius=15pt];
    \path[ffa] (0,0) circle [radius=10pt];

    \path[ffc2] (0,0) circle [radius=15pt] node{B};
    \path[ffc] (0,0) circle [radius=10pt] node{A};
  \end{scope}

  \begin{scope}[local bounding box=scope8, shift={($(scope7.east)+(1cm,0)$)}]
    \path[ffa2] (0,0) circle [radius=15pt];
    \path[ffa] (0,0) circle [radius=15pt];
    
    \path[ffc] (0,0) circle [radius=15pt] node{A};
    \path[ffc2] (0,0) circle [radius=15pt] node{B};
  \end{scope}
\end{tikzpicture}
  \caption{Les relations topologiques des modèles RCC-8 et RCC-5}
  \label{fig:RCC}
\end{figure}

% RCC23 A.G.  Cohn, B.  Bennet, J.  Dooday, and N.M.  Gotts,
% Qualitative Spatial Representation and Reasoning with the Region
% Connection Calculus, GeoInformatica 1(1), pp 1-44, 1997.
Pour améliorer la modélisation des relations topologiques avec des
régions concaves, \textcite{Cohn1997} proposerons une extension à 23
relations du modèle RCC, définissant le modèle RCC-23. La relation
\emph{externally connected} (EC) est décomposée en 9 nouvelles
relations, permettant, par exemple, de faire la distinction entre un
contact avec inclusion dans l'enveloppe convexe et sans. De même, la
relation \emph{disconnected} (DC) est décomposée en 8 nouvelles
relations, ce qui permet de distinguer une situation où un objet
\enquote{enveloppe} l'autre. Un modèle RCC-62 sera également proposé
pour répondre à ce même problème. Contrairement au modèle RCC-23, le
modèle RCC-62 traite les régions en distinguant leur \emph{intérieur}
de leur \emph{frontière} et de leur \emph{extérieur,} ce qui permet de
définir plus de relations topologiques, au prix d'une complexification
conséquente.

\begin{table}
  \centering
  \subfloat[Exemple de décomposition de la relation topologique EC]{
  \begin{tabular}{rC{5.5cm}C{5.5cm}}
    \toprule
    & OUTSIDE\_OUTSIDE\_EC & P\_INSIDE\_INSIDEi\_EC \\
    \midrule
    EC & \tikz{
         \begin{scope}
           \draw[ffa] (-15pt,5pt) -- (10pt,5pt) -- (10pt,0) arc
           (90:270:20pt) --++ (0,-5pt) --++(-25pt,0) -- cycle;
           \draw[ffc] (-15pt,5pt) -- (10pt,5pt) -- (10pt,0) arc
           (90:270:20pt) --++ (0,-5pt) --++(-25pt,0) -- cycle;
         \end{scope}
         \begin{scope}[xshift=20pt,xscale=-1]
           \draw[ffa2] (-15pt,5pt) -- (10pt,5pt) -- (10pt,0) arc
           (90:270:20pt) --++ (0,-5pt) --++(-25pt,0) -- cycle;
           \draw[ffc2] (-15pt,5pt) -- (10pt,5pt) -- (10pt,0) arc
           (90:270:20pt) --++ (0,-5pt) --++(-25pt,0) -- cycle;
         \end{scope}
         }&
            \tikz{
            \begin{scope}
              \draw[ffa] (-15pt,5pt) -- (10pt,5pt) -- (10pt,0) arc
              (90:270:20pt) --++ (0,-5pt) --++(-25pt,0) -- cycle;
              \draw[ffc] (-15pt,5pt) -- (10pt,5pt) -- (10pt,0) arc
              (90:270:20pt) --++ (0,-5pt) --++(-25pt,0) -- cycle;
            \end{scope}
            \begin{scope}[yshift=-20pt]
              \path[ffa2] (0,0) circle [radius=10pt];
              \path[ffc2] (0,0) circle [radius=10pt] node {B};
            \end{scope}
            } \\
    \bottomrule    
  \end{tabular}
  \label{tab:RCC8_vs_RCC23_1}
}

\subfloat[Exemple de décomposition de la relation topologique DC]{
  \begin{tabular}{rC{5.55cm}C{5.5cm}}
    \toprule
    & INTSIDE\_OUTSIDEi\_DC & P\_OUTSIDE\_OUTSIDE\_EC \\
    \midrule
    DC & \tikz{
         \begin{scope}
           \draw[ffa] (-15pt,5pt) -- (10pt,5pt) -- (10pt,0) arc
           (90:270:20pt) --++ (0,-5pt) --++(-25pt,0) -- cycle;
           \draw[ffc] (-15pt,5pt) -- (10pt,5pt) -- (10pt,0) arc
           (90:270:20pt) --++ (0,-5pt) --++(-25pt,0) -- cycle;
         \end{scope}
         \begin{scope}[xshift=2.5pt,yshift=-20pt]
           \path[ffa2] (0,0) circle [radius=10pt];
           \path[ffc2] (0,0) circle [radius=10pt] node {B};
         \end{scope}
         }&
            \tikz{
            \begin{scope}
              \draw[ffa] (-15pt,5pt) -- (10pt,5pt) -- (10pt,0) arc
              (90:270:20pt) --++ (0,-5pt) --++(-25pt,0) -- cycle;
              \draw[ffc] (-15pt,5pt) -- (10pt,5pt) -- (10pt,0) arc
              (90:270:20pt) --++ (0,-5pt) --++(-25pt,0) -- cycle;
            \end{scope}
            \begin{scope}[xshift=30pt,yshift=-20pt]
              \path[ffa2] (0,0) circle [radius=10pt];
              \path[ffc2] (0,0) circle [radius=10pt] node {B};
            \end{scope}
            } \\
    \bottomrule    
  \end{tabular}
  \label{tab:RCC8_vs_RCC23_2}
}


  \caption{Extrait des nouvelles relations topologiques proposées par
le modèle RCC23, d'après XXXXXXX.}
  \label{tab:RCC8_vs_RCC23}
\end{table}

% Exstention 3D
% Extension temporelle : Wolter
Le modèle RCC sera également étendu d'autres manières, notamment pour
prendre en compte la troisième dimension avec les modèles RCC-3D,
VRCC-3D et VRCC-3D+ ou la temporalité avec les travaux de XXXX
couplant le modèle RCC-8 à l'algèbre temporelle d'Allen. Ces
extensions s'éloignent cependant du cadre de ce travail.

% 4IM et dérivés
La seconde \enquote{famille} de modèle des relations topologiques est
celle du \emph{modèle des 4 intersections}
\autocite[4IM,][]{Egenhofer1989} et de ses dérivés, qui, comme pour le
modèle RCC-8, sont nombreux. Ces deux modèles partagent leurs
fondements. Tous deux s’appuient sur le paradigme des \emph{ensembles
  de points} pour modéliser les objets géométriques et définissent les
\emph{relations topologiques} à partir d'opérations ensemblistes sur
ces ensembles de points. De plus, dans les modèles RCC-8
\autocite{Randell1992} et 4IM, les relations topologiques définies
sont exhaustives et mutuellement disjointes (JEPD). Les deux modèles
ont cependant quelques différences, la première est que le modèle 4IM
et ses dérivés ne permettent pas de raisonner à partir des
\emph{relations topologiques,} il n'est donc pas possible d'inférer
une relation à partir de relations topologiques et d'objets
connus. Cependant, le \emph{modèle des intersections} permet de
modéliser des \emph{relations topologiques} entre des régions, des
lignes et des points, contrairement au modèle RCC-8 qui se limite aux
régions.

Dans la première version du \emph{modèle des intersections,} le modèle
4IM, 8 \emph{relations tolopologiques}, équivalentes à celles du
modèle RCC-8 \autocite{Duchene2019}, sont définies
\textcite{Egenhofer1989,Egenhofer1990,Egenhofer1991a}. Celles-ci sont
définies à partir d'une matrice 4 valeurs booléenes, détaillant les
intersections entre les composantes de deux objets géométriques \(a\)
et \(b\). La première d'entre elles est \emph{l'intérieur} de l'objet,
notée \(aᵒ\) ou \(bᵒ\), en fonction de l'objet concerné. La seconde,
notée \(δa\) ou \(δb\), est la \emph{frontière} de l'objet. Ces deux
ensembles, \emph{intérieur} et \emph{frontière,} sont définis pour
chaque type de géométrie, par exemple la frontière d'une ligne est un
ensemble de deux points, le premier et le dernier de la ligne. Pour un
point les deux notions sont équivalentes. La construction de la
\emph{matrice }d'intersection consiste alors en l'étude des
intersections, deux à deux, de la \emph{frontière} et de l'intérieur
de deux objets géométriques :

\begin{equation}
  \label{eq:matrice_4IM}
  \text{4IM}(a,b) =
  \begin{bmatrix}
    aᵒ ∩ bᵒ ≠ ∅ & aᵒ ∩ δb ≠ ∅ \\
    δa ∩ bᵒ ≠ ∅ & δa ∩ δb ≠ ∅ \\
  \end{bmatrix}
\end{equation}

Si l'intersection entre deux ensembles donnés est vide, alors la
valeur inscrite dans la matrice est \enquote{\(F\)}, dans le cas
contraire, la valeur est \enquote{\(V\)}. Par exemple, une
\emph{relation topologique} \emph{d'inclusion} (nomée \emph{overlap,}
dans le modèle 4IM et correspondant au prédicat PO du modèle RCC-8,
cf. figure \ref{fig:RCC}) correspond à la matrice :
%
\(\left[
  \begin{smallmatrix}
    V&V\\
    V&V\\
  \end{smallmatrix}
\right]\),
%
étant donné que les quatres \emph{intersections} définies dans le
modèle 4IM ne sont pas nulles. La \emph{matrice d'intersections} de ce
modèle permet de définir 16 configurations différentes. Cependant la
moitié d'entre-elles ne sont pas réalisables, comme par exemple la
matrice :
%
\(\left[
  \begin{smallmatrix}
    F&V\\
    V&F\\
  \end{smallmatrix}
\right]\),
%
qui décrit une situation où les deux intérieurs et les deux frontières
ne s'intersectent pas, mais où la frontière de chaque objet intersecte
l'intérieur de chaque objet. Comme pour le modèle RCC-8, les 8
configurations possibles sont toutes nomées (\emph{disjoint, contains,
  inside, equal, meet, covers, coveredBy} et \emph{overlap}), ce qui
rend les \emph{relations topologiques} définies plus intelligible et
facile à manipuler que des matrices de valeurs booléenes.

Dans le modèle 9IM, une extension du modèle 4IM proposée par
\textcite{Egenhofer1991}, la \emph{matrice des intersection} est
étendue par l'ajout d'un nouvel ensemble, \emph{l'extérieur} (\ie tous
les points qui n'appartiennent ni à l'intérieur, ni à la frontière de
l'objet), noté \(aᵉ\) (ou \(bᵉ\) en fonction de l'objet). La matrice
initiale est alors étendue en une matrice de 9 valeurs booléennes :

\begin{equation}
  \label{eq:matrice_9IM}
  \text{9IM}(a,b) =
  \begin{bmatrix}
    aᵒ ∩ bᵒ ≠ ∅ & aᵒ ∩ δb ≠ ∅ & aᵒ ∩ bᵉ ≠ ∅ \\
    δa ∩ bᵒ ≠ ∅ & δa ∩ δb ≠ ∅ & δa ∩ bᵉ ≠ ∅ \\
    aᵉ ∩ bᵒ ≠ ∅ & aᵉ ∩ δb ≠ ∅ & aᵉ ∩ bᵉ ≠ ∅ \\
  \end{bmatrix}
\end{equation}

Ce nouveau modèle ne permet pas de modéliser des configurations
nouvelles entre deux régions, mais il permet de distinguer plus
finement les configurations incluant une ligne, comme le montre la
figure \ref{tab:4IM_vs_9IM}.

\begin{table}
  \centering
  \subfloat[Exemple de décomposition de la relation topologique \emph{overlap}]{
  \begin{tabular}{rC{5.5cm}C{5.5cm}}
    \toprule
    & \scriptsize \(\begin{bmatrix}
      V & V & F \\
      V & V & F \\
      V & F & F \\
    \end{bmatrix}\) & \scriptsize  \(\begin{bmatrix}
      V & V & V \\
      V & V & V \\
      V & V & V \\
    \end{bmatrix}\) \\
    \midrule
    \scriptsize \(\text{overlap} ≡ \begin{bmatrix}
      V & V \\
      V & V \\
    \end{bmatrix}\)
    & \tikz{
      \path[ffa] (0,0) circle [radius=15pt];
      \path[ffc] (0,0) circle [radius=15pt] node {A};
      \path[ffc, draw=RdBu-9-9,o-o,shorten >=-3pt,shorten <=-3pt]
      (0,15pt).. controls (0pt, 35pt) and (30pt, 30pt) .. (10pt,0);
      }
    & \tikz{
      \path[ffa] (0,0) circle [radius=15pt];
      \path[ffc] (0,0) circle [radius=15pt] node {A};
      \path[ffc, draw=RdBu-9-9,o-o,shorten >=-3pt,shorten <=-3pt]
      (0,15pt) arc (90:0:15pt) -- (7.5pt,0);
      } \\
    \bottomrule    
  \end{tabular}
  \label{tab:RCC8_vs_RCC23_1}
}

\subfloat[Exemple de décomposition de la relation topologique \emph{meet}]{
  \begin{tabular}{rC{5.5cm}C{5.5cm}}
    \toprule
    &  \scriptsize \(\begin{bmatrix}
      F & F & V \\
      F & V & F \\
      F & V & F \\
    \end{bmatrix}\) & \scriptsize  \(\begin{bmatrix}
      F & F & V \\
      F & V & V \\
      V & F & V \\
    \end{bmatrix}\) \\
    \midrule
    \scriptsize \(\text{meet} ≡ \begin{bmatrix}
      F & F \\
      F & V \\
    \end{bmatrix}\)
    & \tikz{\path[ffa] (0,0) circle [radius=15pt] node {A};}&
                                                              \tikz{\path[ffa] (0,0) circle [radius=15pt] node {B};} \\
    \bottomrule    
  \end{tabular}
  \label{tab:RCC8_vs_RCC23_2}
}


  \caption{Exemple des raffinements de \emph{relations topologiques}
    permis par le modèle 9IM, d'après \textcite{Egenhofer2011}.}
  \label{tab:4IM_vs_9IM}
\end{table}

Le modèle 9IM sera lui-même étendu par le modèle DE-9IM proposé par
\textcite{Clementini1993}. Modèle qui sera par ailleurs normalisé par
l'OGC et par l'ISO, ce qui a conduit à son adoption massive dans les
\ac{sig} \autocite{Strobl2008}. Comme son sigle le laisse supposer, ce
nouveau modèle conserve la matrice à 9 valeurs du modèle 9IM,
cependant les valeurs qui y sont inscrites ne sont plus limitées aux
booléens. En effet, le \emph{dimensions extend nine-intersection
  model} (DE-9IM) ajoute la possibilité de spécifier la
\emph{dimension} des intersections. La matrice d'intersection de deux
objets \(a\) ou \(b\) est alors :

\begin{equation}
  \label{eq:matrice_DE9IM}
  \text{DE-9IM}(a,b) =
  \begin{bmatrix}
    \text{dim}(aᵒ ∩ bᵒ)&\text{dim}(aᵒ ∩ δb)&\text{dim}(aᵒ ∩ bᵉ)\\
    \text{dim}(δa ∩ bᵒ)&\text{dim}(δa ∩ δb)&\text{dim}(δa ∩ bᵉ)\\
    \text{dim}(aᵉ ∩ bᵒ)&\text{dim}(aᵉ ∩ δb)&\text{dim}(aᵉ ∩ bᵉ)\\
  \end{bmatrix}
\end{equation}

Avec \(\text{dim}(x)\), une fonction renvoyant la dimension de
l'intersection, soit 0 pour un point, 1 pour une ligne et 2 pour une
région. Malgrès cette modification substentielle, le modèle DE-9IM
reste compatible avec le modèle 9IM, les matrices numériques du modèle
DE-9IM étant faciles à convertir en matrices binaires, compatibles
avec le modèle 9IM. Comme les autres extensions présentées, cette
extension du modèle 9IM augmente l'expressivité du \emph{modèle des
  intersections} et offre, par exemple, la possibilité de distinguer
un contact ponctuel, d'un contact en une ligne, \ie de distinguer un
voisinage de Von Neuman, d'un voisinage de Moore
(\autoref{tab:9IM_vs_DE9IM}). Cependant, cette nouvelle proposition se
heurte aux mêmes limites que les modèles RCC-23 et RCC-62, à savoir le
grand nombre de cas possibles. Conscient de se problème
\textcite{Clementini1993} complétent le modèle DE-9IM avec le
\emph{calculus based model} (CMB). Ce second modèle propose de définir
cinq \emph{prédicats topologiques} (\emph{in, cross, overlap, touch,
  et disjoint}), correspondant à des configurations particulières de
\emph{matrices d'intersection}, plus faciles à apprender et a
manipuler.

\begin{table}
  \centering
  \begin{tabular}{R{2.5cm}C{4cm}C{4cm}}
  \toprule
  & \scriptsize \(\begin{bmatrix}
    F & F & 2 \\
    F & \mathbf{0} & 1 \\
    2 & 1 & 2 \\
  \end{bmatrix}\) & \scriptsize  \(\begin{bmatrix}
    F & F & 2 \\
    F & \mathbf{1} & 1 \\
    2 & 1 & 2 \\
  \end{bmatrix}\) \\
  \midrule
  \scriptsize \(\begin{bmatrix}
    F & F & V \\
    F & V & V \\
    V & V & V \\
  \end{bmatrix}\)
  & \tikz{
    \path[ffa,, rotate around={45:(0,0)}] (0,0) rectangle ++(1,1);
    \path[ffc,, rotate around={45:(0,0)}] (0,0) rectangle ++(1,1) node[pos=.5,color=RdBu-9-1] {A};
    \path[ffa2, rotate around={45:(1.41,0)}] (1.41,0) rectangle ++ (1,1);
    \path[ffc2, rotate around={45:(1.41,0)}] (1.41,0) rectangle ++ (1,1) node[pos=.5,color=RdBu-9-9] {B};

    }
      & \tikz{
        \path[ffa] (0,0) rectangle (1,1);
        \path[ffc] (0,0) rectangle (1,1) node[pos=.5,color=RdBu-9-1] {A};
        \path[ffa2] (1,0) rectangle (2,1);
        \path[ffc2] (1,0) rectangle (2,1) node[pos=.5,color=RdBu-9-9]
        {B};
        }
  \\
  \bottomrule    
\end{tabular}



  \caption{Exemple des raffinements de \emph{relations topologiques}
    permis par le modèle DE-9IM.}
  \label{tab:9IM_vs_DE9IM}
\end{table}

Comme pour le modèle RCC-8, des extensions tridimensionnelles du
\emph{modèle des intersection} ont été
proposées. \textcite{DelaLosa2000} propose, par exemple, une extension
du modèle 9IM à la troisième dimension. Si le format et le processus
de construction de la \emph{matrice d'intersection} demeurent
identiques, l'ajout du volume comme primitive topologique conduit à
l'apparition de nouvelles relations topologiques qui étaient
irréalisables, comme par exemple, la relation correspondant à la
matrice: 
%
\(\left[
  \begin{smallmatrix}
    F&V&F\\
    F&F&V\\
    V&V&V\\
  \end{smallmatrix}
\right]\),
%
décrivant une situation où la frontière d'une géométrie intersecte
l'intérieur de la seconde géométrie, sans qu'il y ait intersection de
leurs intérieurs, configuration uniquement possible en 3D,
correspondant, par exemple, à l'intersection d'une anse et d'un
panier.

\subsubsection{Les relations de distance}

Si la notion de distance est \enquote{fondamentale} en géographie
\autocite{Pumain1997}

La notion de \emph{relation de distance} regroupe de nombreux aspects
de  

Les relations de distance permettent de décrire l'éloignement entre le
\emph{sujet} et un \emph{objet de référence.} Comme les relations
directionnelles elles peuvent être exprimées de manière quantitative
(\eg \enquote{Je suis à 50 mètres}) ou qualitatives (\eg \enquote{Je
  suis proche}). Ce

\tdi{Geographic Space and Time are Tightly Coupled}
\tdi{Distances are Asymmetric}
\tdi{Distance Inferences are Local, Not Global}
\tdi{Distances Don’t Add Up Easily}

\tdi{Borillo pas de différence fondamentatle entre distance et temps
  de marche, voir aussi Naive geography}

\enquote{La distance \textelp{} semble \textelp{} numérique par
  essence. Toutefois \textelp{} la notion sous-jacente essentielle est
  en fait un ordre \textelp{}}, \textcite{Aurnague1997}.

Un élément important des relations de distance est qu'elles peuvent
être exprimées de différentes manières, toutes disposant de leurs
propres contraintes et par conséquent spécificités lors de leur
spatialisation. Tout d'abord les distances peuvent être exprimées de
manière quantitative, c'est par exemple le cas dans des phrases telles
que : \enquote{Elle est à 50 mètres}. Mais cette même notion de
distance peut être exprimée de manière qualitative, comme dans la
phrase \enquote{Le meilleur restaurant Japonnais de la ville est juste
  à côté}. Dans les deux cas, la \emph{relation de localisation}
utilisée traduit une notion de distance, mais d'une manière
différente, qui imposera une adaptation de la méthode de
\emph{spatialisation.}

Mais, les \emph{relations de distance} ne se limitent pas
nécessairement à l'expression de distances métriques. Les temps de
déplacement, par exemple, sont également une forme de distance et par
conséquent, il n'y a pas de différence sémantique entre les phrases :
\enquote{Il est proche}, \enquote{Il est à cent mètres} et \enquote{Il
  est à quelques minutes de marche}. Comme les distances métriques,
les temps de déplacement peuvent aussi bien s'exprimer de manière
qualitative, comme dans l'exemple précédent, ou quantitative, comme
dans la phrase : \enquote{C'est à trois minutes de marche}.  Ces
différents types de distance peuvent toutes se retrouver dans les
alertes que nous traitons, ainsi nous présenterons ici aussi bien les
distances métriques (qualitatives et quantitatives) que les temps de
déplacement.

Un second aspect important à prendre en compte est la méthode de
calcul de la distance. Celle-ci doit t-elle être calculée à vol
d'oiseau (\ie distance euclidienne) ou sur le réseau ? On peut
également adopter d'autres approches, comme la distance de Manhattan,
offrant une bonne approximation de la \emph{distance réseau} pour des
maillages orthogonaux. Dans notre contexte d’application différentes
méthodes de calcul de la distance peuvent être employées. La
\emph{distance euclidienne} semble, en effet, plus adaptée au calcul
de distances estimées visuellement, comme dans la phrase : \enquote{Il
  y a un chalet loin devant moi}, alors que la distance réseau est
plus pertinente pour \emph{spatialiser} des \emph{indices de
  localisation} décrivant une durée ou un distance de déplacement,
comme dans la phrase : \enquote{Nous avons marché pendant 2 heures}.

Si le calcul de la \emph{distance euclidienne} ne pose pas de
problèmes spécifiques ce n'est pas le cas pour la \emph{distance
  réseau.}  Divers algorithmes existent, dont les plus connus sont
l'algorithme du plus court chemin de \textcite{Dijkstra1959}, et
l'algorithme A* \autocite{Hart1968}, plus rapide mais ne renvoyant pas
nécessairement le chemin le plus court. Ces deux algorithmes peuvent
être utilisés pour approximer le trajet suivit sur un réseau donné et
entre deux points, comme le proposent \textcite{Berli2018, Bunel2018}
dans leurs travaux de reconstitution de flux maritimes à partir de
matrices origine-destination. Mais cet usage implique de faire
l'hypothèse que le trajet réellement suivit est le chemin le plus
court (ou le moins couteux), ou tout du moins qu'il tend à s'en
rapprocher. Or cette hypothèse est assez contraignante, aussi bien
pour la modélisation des flux maritimes, puisqu'elle peut contredire
certaines réalités météorologiques ou historiques, que du trafic
routier, bien que ces deux algorithmes soient régulièrement utilisés à
cet effet. Une illustration de ce problème peut être trouvée dans les
travaux de \textcite{Pailhous1970}, sur les itinéraires suivis par les
chauffeurs de taxi (avant l'introduction des systèmes d'aide à la
navigation). Dans ses travaux, \textcite{Pailhous1970} met en évidence
que, lors de leurs déplacements, les chauffeurs de taxis cherchent en
priorité à rejoindre le réseau routier principal, composé des grandes
avenues et des voies périphériques, le long desquels ils effectuons la
majorité du trajet. L'entrée dans le réseau routier secondaire, qui
est repoussée au plus tard possible, s'effectue généralement à
proximité de la destination \autocite{Lagesse2016}. Ce type
d'itinéraire est sensiblement différent du plus court chemin,
l'utilisation des algorithmes de \textcite{Dijkstra1959} ou A*
\autocite{Hart1968} est donc à relativiser. Ce même problème se pose
également, mais d'une manière différente, dans notre contexte
applicatif. En effet, des \emph{indices de localisation} tels que :
\enquote{J'ai marché 10 kilomètres depuis mon point de départ} ne
seraient pas nécessairement spatialisés de manière satisfaisante en
utilisant un algorithme renvoyant le plus court chemin, la plupart des
alertes concernant des trajets touristiques qui, par définition, sont
plus proches de la déambulation que du déplacement rationnel.

\paragraph{Distances métriques quantitatives}

Lorsqu'une relation de distance entre le sujet et l'objet de référence
est exprimée de manière quantitative, comme dans la phrase :
\enquote{Je suis à 300 mètres}, on peut s'attendre à ce que l'exercice
de \emph{spatialisation} soit assez simple. Il n'est en effet pas
nécessaire, contrairement aux distances qualitatives, d'estimer la
distance à modéliser, celle-ci étant déjà connue. Cependant, une
valeur de distance quantifiée n'est pas nécessairement précise. En
effet, compte tenu de notre contexte, les distances quantitatives sont
quasi-exclusivement estimées par une personne, généralement le
requérant et non à l'aide d'une mesure, ce qui pourrait être le cas si
nous avions a \emph{spatialiser} des phrases telles que :
\enquote{D'après le GPS, j'ai a roulé 123 km}. Bien entendu, une
distance mesurée peut également être imprécise, par exemple, à cause
d'une mauvaise calibration de l'outil de mesure, mais il s'agit d'un
cas assez différent, car lié à un événement particulier, alors que la
mauvaise estimation d'une distance par un être humain est systémique.

Au problème de la mauvaise estimation des distances s'ajoute un autre
phénomène impactant la qualité des estimations des distances,
\emph{l'attraction des nombres ronds} \autocite{Durand1961}. Ce
phénomène, initialement mis en évidence lors de l'étude de la
distribution des ages, implique une surreprésentation des valeurs se
terminant par 0 et 5, à cause d'un arrondi de la personne
recensée. Par conséquent, on peut s'attendre à ce que les distances
exprimées à \enquote{vue de nez} soient impactées par ce phénomène et
toute autre forme d'arrondi.

Il est donc inutile d'espérer obtenir une bonne estimation de la
distance, la \emph{spatialisation} des \emph{relations de distance}
exprimant une distance métrique qualitative doit donc se faire en
prenant en compte les différentes imprécisions qui peuvent exister.

\paragraph{Distances métriques qualitatives}

\tdi{Robinson, V.: Interactive machine acquisition of a fuzzy spatial
  relation. Comput.  Geosci. 16, 857–872 (1990). Robinson, V.:
  Individual and multipersonal fuzzy spatial relations acquired using
  human-machine interaction. Fuzzy Sets Syst. 113(1), 133–145 (2000)}

\tdi{Spatial Natural Language Generation for Location Description in
  Photo Captions Mark M. Hall 1 , Christopher B. Jones 2( B ) , and
  Philip Smart 2}

\paragraph{Distances-temps et modèles de marche}

\tdi{Bonne ref An analysis of probability of area techniques for
  missing persons in Yosemite National Park}

\tdi{équation fausses, vérifier}

S'il n'y a pas de différences sémantiques fondamentales entre une
distance métriques et une distance-temps, leur processus de
spatialisation est cependant légèrement différent. En effet, toute
spatialisation d'une distance-temps impose de transformer une durée de
parcours en une distance parcourue à l'aide d'un modèle.

La solution la plus simple consiste à définir une vitesse de
déplacement moyenne et de transformer le temps parcouru en une
distance à l'aide d'une règle de proportionnalité. Cette approche est
assez restrictive, mais elle peut être satisfaisante pour des modèles
à petite échelle et simulant des environnements où la difficulté de
déplacement est homogène, comme dans le modèle multi-agent MAGS
\autocite{Moulin2003}, destiné à simuler des comportements humains en
milieu urbain. Dans ce modèle \autocite{Moulin2003}, la vitesse de
déplacement moyenne d'un marcheur a été fixée à
\SI{5}{\kilo\meter\per\hour}. Cette valeur à également été retenue
dans l'outil \enquote{zones de desserte} d'ArcGIS \autocite{ESRI2020},
permettant de calculer des isochrones de déplacement pédestre.

Cette approche n'est cependant pas satisfaisante dans notre contexte
applicatif. En effet, les contraintes de déplacement en milieu urbain
ne sont pas comparables à celles que l'on rencontre en montagne, où il
est fréquent de rencontrer des longues montées et des descentes
escarpées qui influent sur la vitesse de marche. L'estimation des
distances de déplacement en milieu montagneux est donc réalisée avec
des \emph{modèles de marche} plus avancés, adaptant l'estimation de la
vitesse de déplacement en fonction du terrain.

% Naismith
Le plus ancien d'entre eux est le modèle de
\textcite{Naismith1892}. Comme l'indique \textcite{Duchene2019}, ce
modèle ne consiste qu'en une phrase, concluant la présentation d'une
sortie dans un journal d'alpinisme. Cette règle énonce qu'une bonne
estimation du temps de marche consisterait à compter une heure pour 3
miles (\SI{5}{\kilo\meter}) en distance planaire et d'y ajouter une
heure de marche par tranche de \SI{2000}{ft} (\SI{600}{\meter}) de
dénivelé positif \autocite{Naismith1892}. Avec ce modèle, la vitesse
de marche (\(V\)) peut s'exprimer en fonction de la pente (\(S\)), de
la manière suivante:

\begin{equation}
  \label{eq:marche_naismith}
  V = \dfrac{5 × 0,6}{5 + 0,6 × S}
\end{equation}

Avec \(S\) la pente, dont l'expression est:

\begin{equation}
 S = \dfrac{Δh}{Δx} = \tan θ
\end{equation}

Où \(θ\) est l'angle de la pente, \(Δh\) le dénivelé et \(Δx\) la
distance planimétrique. Cette formule peut être généralisée pour
s'adapter à différentes valeurs de vitesses planimétriques (\(Vₓ\)) et
ascensionnelles (\(Vₕ\), en \si{\kilo\meter\per\hour}):

\begin{equation}
  \label{eq:marche_naismith_f}
  V = \dfrac{Vₓ × Vₕ}{Vₓ + Vₕ × S}
\end{equation}

Une première correction de ce modèle sera proposée par
\textcite{Aitken1977}, qui suggère d'utiliser la règle définie par
\autocite{Naismith1892} pour estimer la vitesse de marche sur sentiers
et de réduire la vitesse planimétrique (\(Vₓ\)) à
\SI{4}{\kilo\meter\per\hour} pour estimer une vitesse de marche
hors-sentier.

% Langmuir
Les modèles d'\textcite{Naismith1892,Aitken1977} ne permettent,
cependant, pas d'estimer la vitesse de marche en descente. Cet ajout
sera du fait de \textcite{Langmuir1984}, qui propose d'étendre le
modèle de \textcite{Naismith1892} en :
%
\begin{enumerate*}[label=(\alph*)]
\item \emph{réduisant} le temps de marche estimé de 10 minutes, tous les
300 mètres de dénivelé négatif, si la pente est \emph{faible} (\ie
comprise entre -5 et -12 degrés) et
\item en \emph{augmentant} le temps de marche estimé de 10 minutes,
  tous les 300 mètres de dénivelé négatif, si la pente est
  \emph{forte} (\ie inférieure à -12 degrés).
\end{enumerate*}
%
Dans la quatrième édition de son ouvrage de 1984,
\textcite{Langmuir2013} proposera un second modèle, plus adapté à
l'estimation de la vitesse de marche d'un groupe. La vitesse de marche
est réduite à \SI{4}{\kilo\meter\per\hour} et les montées sont plus
pénalisées, une heure de marche est ajoutée par tranche de
\SI{450}{\meter} de dénivelé positif. Ces deux versions corrigées du
modèle de \autocite{Naismith1892} par \textcite{Langmuir1984,
  Langmuir2013} peuvent s'exprimer de la manière suivante :

\begin{equation}
  \label{eq:marche_langumir}
  \def\arraystretch{1.75}
  V = \left\{
    \begin{array}{ll}
      \dfrac{Vₓ × V_{h1}}{Vₓ + V_{h1} × S} & \text{si}\ θ > -5° \\
      \dfrac{Vₓ × V_{h2}}{Vₓ - V_{h2} × S} & \text{si}\ -12° ≤ θ ≤ -5° \\
      \dfrac{Vₓ × V_{h2}}{Vₓ + V_{h2} × S} & \text{si}\ θ < -12°\\
    \end{array}
  \right.
\end{equation}

Avec \(V_{h1}\) la vitesse ascensionnelle (\SI{600}{\meter\per\hour}
dans \textcite{Langmuir1984} et \SI{450}{\meter\per\hour} dans
\textcite{Langmuir2013}), \(V_{h2}\) ascensionnelle en descente. Avec
ces corrections il devient possible d'évaluer le temps de marche
nécessaire pour effectuer un parcours complet, descente
comprise.
% utilisé par R.cost
Cependant le modèle de \bsc{Langmuir-Naismith} à un défaut majeur, son
comportement pour les faibles pentes
(\autoref{fig:modeles_marche}). En effet les vitesses de marche en
descente estimées sont très importantes et en fort décalage avec la
tendance observée pour les courbes définies pour les montées et les
grandes descentes. Cet effet est particulièrement marqué dans la
formulation originale du modèle, où les vitesses estimées peuvent
approcher les \SI{12}{\kilo\meter\per\hour}, une vitesse de marche
très élevée, même dans des conditions optimales
\autocite{Kerouanton2020}. Les vitesses de marche estimées par la
seconde version du modèle \autocite{Langmuir2013} semblent plus
réalistes, la vitesse maximale étant d'environ
\SI{7}{\kilo\meter\per\hour}. Cependant, la rupture entre l'estimation
pour les fortes descentes et celle pour les faibles descentes
subsiste, puisque si la vitesse estimée pour une pente de 12 degrés
est d'environ \SI{7}{\kilo\meter\per\hour}, celle pour une pente de 13
degrés est d'environ \SI{2,5}{\kilo\meter\per\hour}. Malgré ses
limites, le modèle \bsc{Langmuir-Naismith} propose une caractéristique
intéressante, les vitesses de marches maximales ne sont pas atteintes
sur le plat, mais dans les descentes, suffisamment pentues pour
soulager le randonneur, mais pas assez prononcées pour être pénibles.

% Tobler
Cette caractéristique est partagée avec le modèle de
\textcite{Tobler1993}, où la vitesse de marche est exprimée en
fonction de la pente seule. Bien que présenté sous une forme
analytique, ce modèle se base sur des données empiriques provenant de
\textcite{Imhof1950} \autocite{Tobler1993}.

\begin{equation}
  \label{eq:marche_tobler}
  V = g × 6 × e^{-3,5 × \left| S + 0,05 \right|}
\end{equation}

Quant à \(g\), il s'agit d'un coefficient de
pondération, permettant de faire varier l'estimation de la vitesse en
fonction d'éléments exogènes, comme le mode de
déplacement. \textcite{Tobler1993} en propose 3 valeurs :
%
\begin{enumerate*}[label=(\alph*)]
\item \(g = 1\), le cas standard, pour estimer un temps de marche sur
  sentier,
\item \(g = 0,6\), lorsque la marche est faite hors-sentier et
\item  \(g = 1.25\), pour un déplacement à cheval.
\end{enumerate*}

% Modèle de Ress
\textcite{Rees2004} a proposé d'estimer la vitesse de marche en
fonction de la pente à l'aide d'une fonction quadratique :
\begin{equation}
  V = \dfrac{1}{a + b × S + c × S²}
\end{equation}

Avec \(S\) la pente. Les valeurs des coefficients \(a\), \(b\) et
\(c\), respectivement \SIlist{0,75;0,09;14,6}{\second\per\meter}, ont
été estimés par \textcite{Rees2004} à l'aide

% Modèle Irmisher
\textcite{Irmischer2017,Kerouanton2020} proposent, dans la lignée de
\textcite{Tobler1993}, deux modèles de marche construits à partir de
données empiriques.

% Modèle Jobe White

% Sentiers ou terrain ?

\begin{figure}
  \centering
  \pgfmathdeclarefunction{naismith}{1}{%
  \pgfmathparse{1 / ((1/#1) + tan(x) / 0.6)}%
}

\pgfmathdeclarefunction{langmuir1}{1}{%
  \pgfmathparse{(1 / ((1/#1) + abs(tan(x)) * (0.16 /
    0.3)))}%
}

\pgfmathdeclarefunction{langmuir2}{1}{%
  \pgfmathparse{(1 / ((1/#1) - abs(tan(x)) * (0.16 /
    0.3)))}%
}

\pgfmathdeclarefunction{tobler}{1}{%
  \pgfmathparse{#1 * 6 * exp(-3.5 * abs(tan(x) + 0.05))}%
}

\pgfmathdeclarefunction{ress}{0}{%
  \pgfmathparse{6*exp(-3.5*abs(x*0.05))}%
}

\begin{tikzpicture}
  \begin{axis}[
    width=12cm,
    height=7cm,
    grid=major,
    xlabel=Pente,
    ylabel=Vitesse de marche (\si{\kilo\meter\per\hour}),
    every axis plot post/.append style={
      mark=none,domain=-60:60,samples=500,smooth},
    ymax=12,
    axis x line=bottom,
    axis y line*=left,
    enlargelimits=upper,
    legend style={
      at={(0.5,-0.25)},
      anchor=north,
      legend columns=3,
      font=\small,
      draw=none},
    ]

    %\addplot[restrict x to domain={0:60}, thick] {naismith(5)};
    %\addlegendentry{\textcite{Naismith1892}}
    
    \addplot[restrict x to domain={0:60}, thick] {naismith(4)};
    %\addlegendentry{\textcite{Naismith1892}}

    \addplot[restrict x to domain={-60:-12}, thick] {langmuir1(5)};
    \addplot[restrict x to domain={-12:-5}, thick] {langmuir2(5)};
    \addplot[restrict x to domain={-5:0}, thick] {5};
    \addplot[restrict x to domain={0:60}, thick] {naismith(5)};

    %\draw[black] (axis cs:-12,2) -- (axis cs:-12,12);
    %\draw[black] (axis cs:-5,7) -- (axis cs:-5,5);
    
    \addplot[thick] {tobler(1)};
    %\addlegendentry{\textcite{Tobler1993}, \(g=1\)}
    
    \addplot[thick] {tobler(0.6)};
    %\addlegendentry{\textcite{Tobler1993}, \(g=0,6\)}
    
    %\addplot {ress};

  \end{axis}
\end{tikzpicture}
  \caption{Vitesse de marche estimée en fonction de la pente par les
    différents modèles de marche proposés dans la littérature.}
  \label{fig:modeles_marche}
\end{figure}


\subsubsection{Les relations projectives}

\enquote{\textelp{} qui nécessitent recours à un espace projectif
  \textelp{}} \autocite[p. 18]{Duchene2019}

\paragraph{Les relations XXXXe}

\paragraph{Les relations directionnelles}

\textcite{Frank1992} recense deux catégories de modèles permettant de
modéliser les directions : les modèles à secteur (ou
\emph{cone-based}), dérivés du travail de \textcite{Peuquet1987} et
les modèles à demi-plans. Dans les modèles à secteur la direction est
définie en fonction d'un cône d'angle donné. Dans les modèles à
demi-plan les directions sont définies à l'aide de deux plans,
séparant l'espace en deux direction, par exemple le nord et le
sud. Ces plans peuvent ensuite être combinés pour raffiner les
directions. Par exemple, la combinaison des demi-plans définissant le
nord, le sud, l'est et l'ouest permet de créer quatre nouvelles
directions cardinales, le nord-ouest, le nord-est, le sud-est et le
sud-ouest.

\missingfigure{}

\textcite{Frank1992} propose une extension du modèle a demi-plans
définissant une zone neutre, plus adaptée aux objets d'extension
spatiale non-nulle.

% Cone based



% Demi plan

%
Une autre catégorie de modèle 

% Allen based


% Modélisation floues

Enfin, comme pour les modèles topologiques, des extensions
tridimensionnelles de ces modèles ont été proposées.



\paragraph{Les relations d'orientation relative}

% Définition
Les relations d'orientation relatives sont définies par
\textcite{Duchene2019} comme des relations
\enquote{concern\textins{ant} l'écart angulaire entre deux objets
  ayant une direction privilégiée d'allongement et pouvant à ce titre
  être assimilés à des segments \textelp{}}.

% Exemple

% Travaux


\subsubsection{Les relations de visibilité}

\tdi{citer \url{https://journals.openedition.org/cybergeo/27862}}

\autocite{Llobera2003}

% Cadre théorique
Peut-on réellement parler de géographie sans parler de la vision ?
Sans doute pas, du moins selon \textcite{Brunet1992}, pour qui la vue
est \enquote{le sens premier du géographe}. Mais cette assertion nous
semble réductrice, selon nous le sens de la vision est fondamental
dans notre rapport à l'espace. Il est par conséquent courant de
décrire une position à partir de ce que l'on y voit. Il nous semble
donc indispensable de pouvoir \emph{spatialiser} des descriptions
visuelles.

\paragraph{La visibilité comme relation spatiale}

Comme pour les \enquote{durées de déplacement}, détaillées ci-dessus,
les \enquote{relations de visibilité} ne ne s'expriment pas à l'aide
d'une préposition spatiale, comme \enquote{sur} ou \enquote{dans}, il
s'agit donc d'une \emph{relation de localisation,} mais pas d'une
\emph{relation spatiale.}

Contrairement aux distances, où leur expression peu prendre plusieurs
formes, les \emph{relations de localisation visuelles} sont toujours
exprimé d'une même manière. On décrit ce qu'on a (ou pas) dans notre
champ de vision. Ainsi la spatialisation de ce type \emph{d'indice de
  localisation} nécessite de pouvoir construire le champ de visibilité
d'un objet, quelque soit sa nature.

La question de la construction de zones de visibilités a été
longuement abordée dans le domaine des SIG, notamment en vue
d’applications pour l'étude de paysages.

Mais la notion de visibilité peu également s'appliquer à des éléments
de localisation tels que : \enquote{Je suis à l'ombre}, que l'on
retrouve notamment dans le \emph{fil rouge} (\ref{}). En effet, être,
ou non, à l'ombre revient à avoir, ou non, une relation de visibilité
avec le Soleil. Or, dans un contexte montagnard les ombres sont très
présentes à cause des reliefs, la possibilité d'identifier les zones
situés au Soleil ou à l'ombre est donc utile pour localiser des
personnes \autocite{Houpert2003}.

% Applications

\paragraph{Visibilité planaire}

\tdi{viewshed}

D'autres travaux proposent de traiter les relations de visibilité de
manière multivalente, \ie en distinguant les situations où la
visibilité est partielle de celles ou la visibilité est
totale. \textcite{Ramos2003}, par exemple, définit la notion de
\emph{fenêtre de visibilité,} qui est la géométrie 3D englobant toutes
les lignes reliant la géométrie observée à la géométrie de
l'observateur. Cette modélisation vectorielle permet de construire
l'intersection de la fenêtre de visibilité avec le relief et le
sursol. \textcite{Ramos203} définit 3 types de visibilité en fonction
de l'intersection de la \emph{fenêtre de visibilité.} Si elle est
entièrement visible la visibilité est totale, si elle est totalement
obstrué alors l'objet est invisible et si elle est partielle obstruée
la visibilité est partielle.

\textcite{Lonergan2016} ont, quant a eux, proposé de prendre en compte
les capacités perspectives des observateurs. En effet, dans les
approches précédemment présentée, l'observateur est modélisé par une
géométrie, sans que ces caractéristiques perspectives, comme son champ
de vision, soient prises en compte. \textcite{Lonergan2016}
définissent le concept de \emph{fenêtre de visibilité} comme un
sous-ensemble du \emph{bassin de visibilité,} du quel on aurait retiré
toutes les zones invisibles, compte-tenu des capacités perspectives de
l'observateur. Un être humain, par exemple, ne peut voir que dans une
direction donnée, et pas à 360 degrés autour de lui. La prise en
compte des capacités perspectives permet à \textcite{Lonergan2016} de
définir une typologies des relations de visibilité, distinguant, par
exemple, les moments où l'observateur se focalise sur un point (ce qui
tend à réduire son champ de vision) de ceux où il scanne le paysage
(et ou l'observateur regarde dans toutes les directions autour de
lui).

\paragraph{Études tangentielles ??}

D'autres travaux proposent quant à eux d'étudier les positions
relatives des objets dans le champ de vision de l'observateur.

\textcite{Santos2015} proposent, par exemple, une typologie des
relations de visibilité fondée sur les relations de Allen. Cette
typologie permet de décrire les relations entre deux objets, du point
de vue d'un observateur qui aurait un champ de vision de 180 degrés.

% ROC 20
\textcite{Randell2001} proposent une solution similaire avec le ROC-20
(\emph{Region Occlusion Calculus}), une formalisation des relations
topologiques entre les silhouettes des objets vus par
l'utilisateur. Ce modèle permet également de modéliser des positions
relatives entre objets vus (\eg du point de vue de l'utilisateur
l'objet \emph{a} est à gauche de \emph{b}), mais également des
relations d'occlusion relative (\eg \emph{a} cache un bout de \emph{b}
et inversement).

%%% Local Variables:
%%% mode: latex
%%% TeX-master: "../../../../main"
%%% End:
