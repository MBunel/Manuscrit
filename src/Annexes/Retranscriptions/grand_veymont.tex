Avec \bsc{R.} le requérant, \bsc{S.} le secouriste.

\begin{dialogue*}
  \Sec Oui bonjour vous êtes avec le secours en montagne.
  % 
  \Req Oui bonjour, \textins{anonymisation}. Je suis accompagnateur en
  montagne et… hum… Une victime qui a la maladie de \textins{coupé}…
  Je savais pas… Bref… Là elle est un peu au bout de ses forces, et…
  \direct{Inaudible} hélicoptère, et elle me dit \enquote{oui…}, enfin
  vraiment je le sens pas. J'ai eu du mal à faire, entre le sommet du
  Grand Veymont et là où je suis…
  % 
  \Sec \direct{Parle en même temps} Vous êtes où ?
  % 
  \Req … La descente est très, très lente, et là…
  % 
  \Sec \direct{Parle en même temps} Vous êtes entre le Grand Veymont
  et le Pas de la ville ?
  % 
  \Req Je suis entre le Grand Veymont, sous le Grand Veymont et Pas de
  la Ville, tout à fait. Je suis côté… côté sud donc, hein…
  % 
  \Sec Heu… côté sud du Pas de la Ville ?
  % 
  \Req Excusez-moi, non, côté nord. Au temps pour moi, côté nord,
  pardon. C'est moi qui…
  % 
  \Sec Vous êtes au-delà du Pas de la ville alors ? Vous êtes entre
  Pas de la ville et Pierre blanche ?
  % 
  \Req Tout à fait, je suis au-delà du Pas de la ville, sur la zone…
  la zone à peu près plate, caillouteuse, mais plate. Y'a une petite
  prairie…
  % 
  \Sec Vous êtes à combien… À combien du Pas de la ville ?
  % 
  \Req Heu… J'estime… Heu… en kilomètres… Heu… Peut-être… Heu…
  J'estime à… Je sais pas… 800 mètres, je crois, à peu près… à vol
  d'oiseau…
  % 
  \Sec Ouais, ouais, OK.
  %
  \Req Voilà
  % 
  \Sec D'accord. Vous êtes un groupe de combien ?
  \direct{Enregistrement coupé}
\end{dialogue*}