Dans ce chapitre nous allons présenter la méthode de modélisation des
\emph{zones de localisation compatibles} et \emph{probable} qui sera
utilisée pour \emph{spatialiser} les \emph{indices de localisation.}


\tdi{Modélisation = Modèle théorique + implémentation}

Comme nous l'avons montré dans l'état de l'art (\autoref{chap:03}),
modéliser des \emph{objets spatiaux imprécis} nécessite de
sélectionner un \emph{modèle théorique} et une \emph{implémentation}
de ce modèle. L'ensemble de ces deux éléments forme ce que nous
appelons une \emph{modélisation.} Dans ce chapitre nous allons
présenter notre démarche de sélection d'une modélisation des \emph{objets
spatiaux imprécis.}

\tdi{Cette partie ne concerne que les zlp et les zlc. Les objets de
  référence sont modélisés de façon nette. Mais bloch l'a fait. Le
  flou aux objets est une extension possible.}

Nous ne parlerons ici que de la question de la modalisation des
\emph{zones de localisation} et non de celle des \emph{objets de
  référence} dont nous ne modélisons pas \emph{l'imprécision.}


%%% Local Variables:
%%% mode: latex
%%% TeX-master: "../../../main"
%%% End:
