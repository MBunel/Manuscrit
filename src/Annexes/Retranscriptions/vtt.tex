\subsection{Retranscription de l'alerte}


\begin{dialogue}
\Req \direct{Enregistrement coupé} Euh\footnote{Commentaire blabla}, oui, je m'appelle \textins{anonymisation} et j'ai… Mon pote il est tombé dans
la descente de… de VTT qui desc… qui part des deux Alpes jusqu'à Venosc et il a super mal aux genoux, il peut plus bouger.
\Sec D'accord… Ok, donc vous êtes… euh, je change… euh… Je regarde juste un instant, le temps de prendre un plan des pistes pour esayer de vous localiser. Hum… hop, donc c'est bien VTT ? Côté Venosc, hein ? On est d’accord ?
\Req Oui, c'est une piste rouge, oui.
\Sec Et vous êtes partis… Vous êtes loin du départ ?
\Req Euh… Non, on est pas vraiment loin, non.
\Sec Non ? Vous êtes plutôt dans la première… dans la première partie ?
\Req Oui, on est plutôt dans la première partie, oui.
\Sec Ouais… Par rapport au… au téléphérique, vous le voyez ? Le… les oeufs qui montent de Venosc ? Là ? Vous le voyez ?
\Req Euh… Oui.
\Sec Et… quand vous regardez vers le bas, le téléphérique est à votre droite ? Ou à votre gauche ?
\Req Euh… il est à gauche.
\Sec Il est à votre gauche \direct{Interloqué} ? Quand vous regardez vers le bas ?
\Req Non, non, non, non. Il est à droite. Excusez-moi.
\Sec D'accord… on est d'accord, vous êtes, euh… Vous regardez vers le bas et il y'a la ligne de téléphérique à votre droite.
\Req Oui c'est à droite.
\Sec D'accord. Et vous estimez à… Donc, euh… au niveau de la descente… vous arrivez à estimer, à peut prêt la… le… le niveau ? Vous vous situez ? A combien… combien de mètres ?
\Req Plus vers le début en fait. On a… On a fait deux, trois minutes de descente, pas plus, et y'a… y'a eu la chute.
\Sec D'accord, deux-trois minutes de descente…
\Req Ouais.
\Sec Ouais… Ok. Et donc votre ami qu'est-ce qu'il a ? \direct{Enregistrement coupé}
\end{dialogue}