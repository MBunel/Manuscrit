Dans ce chapitre nous avons détaillé l'objectif principal de cette
thèse, qui consiste à transformer une description de position en une
zone de coordonnées connues. Comme nous l'avons montré dans la
première partie de ce chapitre (\ref{subsec:2-1-1}) cette question se
décompose, en réalité, en deux étapes distinctes. La première consiste
à construire des \emph{zones de localisation} à partir des
\emph{indices de localisation} donnés par le requérant, au cours d'une
opération nommée : \emph{spatialisation.} La seconde étape consiste à
\emph{fusionner} toutes ces zones pour construire la \emph{zone de
  localisation probable,} c'est-à-dire la zone qui vérifie tous les
\emph{indices de localisation.}

Le développement de ces deux méthodes est fortement contraint les par
caractéristiques du langage naturel. Un même mot peut prendre
plusieurs sens, difficiles à distinguer. De plus, les \emph{indices de
  localisation} peuvent être faux, ce qu'il faut également prendre en
compte. Les cinq objectifs scientifiques de la thèse visent à répondre
a ces problèmes. Le traitement de chacun d'entre-eux fera l'objet d'un
chapitre spécifique dans les parties \ref{part:02} et \ref{part:03}.

%%% Local Variables:
%%% mode: latex
%%% TeX-master: "../../../../../../Thèse/Manuscrit/src/main"
%%% End:
