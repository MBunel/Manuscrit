Une fois que la structure globale de l'ontologie des règles de
décomposition a été fixée, il est nécessaire de définir son contenu en
détail (\ie les \emph{relations de localisation,} les \emph{relations
  de décomposition} et les \emph{relations de localisation
  atomiques}). Pour ce faire, on peut se fonder sur les différents
travaux formalisant des \emph{relations de localisation}
\autocite{Vandeloise1986,Aurnague1997,Borillo1998}, pour identifier
des \emph{relations de localisation} et en proposer une
décomposition. Cependant, bien que cette approche soit pertinente,
elle ne nous semble pas être pleinement adaptée à notre cas
d’application. En effet, ces travaux traitent des \emph{relations de
  localisation} dans leur acception générale, alors que nous
travaillons sur un contexte particulier, avec un relief et des
\emph{objets de référence spécifiques} (\eg montagne, col,
\emph{etc.}). Si l'on augure que le contexte influence l’utilisation
des \emph{relations de localisation} et donc impacte leur sémantique,
---~comme l'illustre l'aphorisme de \bsc{Wittgenstein} : \enquote{la
  signification c'est l'usage}~---, il est alors nécessaire de prendre
en compte les spécificités des descriptions de positions en milieu
montagneux, afin de proposer une ontologie la plus pertinente
possible.

Pour ce faire nous avons entrepris de recenser et d'étudier les
\emph{relations de localisation} utilisées lors de cas réels. Les
\ac{pghm} disposant d'un enregistreur d'appels, nous avons pu obtenir
des enregistrements de discussions contenant des descriptions de
positions. En analysant le contenu de ces enregistrements il est alors
possible d'identifier les \emph{relations de localisation} les
fréquentes et si leur(s) cas d'utilisation différent du cas général.

\subsection{La retranscription et l'analyse des alertes}

L'exploitation des enregistrements nécessite leur retranscription
préalable. Pour ce faire, nous avons élaboré un \emph{template} (ou
canevas) de retranscription, qui, en fixant un format, facilite le
travail de saisie et d'analyse des enregistrements. Le travail de
définition du template et de retranscription a été effectué
collaborativement \autocite{Bunel2019}, les résultats pouvant être
exploités par différents membres du projet Choucas.

\subsubsection{Les corpus d'alertes}

Les retranscriptions ont été réalisées à partir d'un ensemble de 52
enregistrements d'appels de demande d'assistance auprès des \ac{pghm}
de Chamonix (5 alertes) et de Grenoble (47 alertes). Ces fichiers,
sélectionnés et compilés par le \ac{pghm} de Grenoble, forment deux
corpus d'alertes distincts, qui nous ont, respectivement, été transmis
en 2017 et en 2019. Ces deux corpus sont relativement similaires, que
ce soit par le nombre d'alertes, leur durée ---~qui varie de quelques
secondes à plusieurs dizaines de minutes
(\autoref{fig:dist_temps_alertes})~--- ou leur contenu. Si nous ne
disposons de relativement peu d'alertes à analyser, ces dernières ont
été sélectionnées par les secouristes de manière à offrir une vision
représentative du processus de localisation des victimes.

Malheureusement, nous n'avons que peu d'informations exogènes sur ces
différentes alertes. Pour certaines d'entre elles les secouristes nous
ont transmis les coordonnées de la victime ou la date et l'heure de
l'intervention, mais dans la plupart des cas ces informations nous
sont inconnues, à moins qu'elles soient explicitement données dans
l'alerte (\eg énoncé oral de coordonnées GPS, précision de l'heure,
\emph{etc.}). Cette absence d'informations exogènes peut, dans de
rares cas, s'opposer à la \emph{spatialisation} \emph{d'indices de
  localisation.} C'est par exemple le cas pour \emph{l'indice de
  localisation,} tiré du \emph{fil rouge} (\autoref{chap:01}) :
\enquote{La victime vient de passer du Soleil à l'ombre}. Or la
\emph{spatialisation} de cet \emph{indice de localisation} ne peut
être faite sans connaissance de l'heure de l'appel. Une autre
particularité importante de ces enregistrements est qu'ils sont
réalisés au niveau de la ligne téléphonique des \ac{pghm} et non des
\ac{codis}, généralement contactés par les requérants
(\autoref{chap:02}). De plus, les fichiers qui nous ont été transmis
ont été montés, de sorte à censurer les informations personnelles ou
médicales, les enregistrements retranscrits ne sont donc qu'un extrait
des alertes et non les appels dans leur ensemble.

\begin{figure}
  \centering
   \begin{tikzpicture}
   \begin{axis}
     \addplot [boxplot] table [y index=0] {./Corps/Partie2/Chapitre5/figures/duration.dat};
     
  %\addplot [thick] gnuplot [raw gnuplot] {plot './Corps/Partie2/Chapitre5/figures/duration.dat' smooth kdensity};
\end{axis}
\end{tikzpicture}

  \caption{Distribution de la durée des alertes}
  \label{fig:dist_temps_alertes}
\end{figure}

\subsubsection{Template de retranscription}

Notre parti-pris n'a pas été de travailler à partir d'une
retranscription libre des alertes, mais à l'aide d'un tableau
prédéfini, un \emph{template de retranscription}
(\autoref{tab:struct_temp}) dont la structure reprend celle des
\emph{indices de localisation} telle que nous l'avons définie
(cf. \autoref{chap:04}). Ce choix présente deux avantages. Tout
d'abord il permet de faciliter l'analyse des alertes en imposant, dès
la saisie, la définition de certains critères. Pour faciliter les
saisies et leur répartition entre transcripteurs, un fichier est créé
pour chaque alerte.

Dans sa forme finale, ce template de transcription se présente sous la
forme d'un fichier tabulaire structuré, auto-documenté et exemplifié,
composé de trois onglets principaux
%
---~\begin{enumerate*}[label=(\alph*)]
\item \emph{métadonnées},
\item \emph{interprétation des expressions} et
\item \emph{toponymie}
\end{enumerate*}~---,
%
permettant de transcrire différents aspects des alertes. L'ensemble du
fichier est conçu pour faciliter le processus de retranscription et
limiter au maximum l'influence du transcripteur sur la saisie. Chaque
onglet et champ à saisir est documenté, des instructions de saisie
présentent l'ensemble des tâches à effectuer sont données et un
exemple de saisie complet et complexe est détaillé. De plus, la
structure même du document est conçue pour faciliter la saisie. Les
onglets sont placés dans leur ordre de consultation ou de saisie (\eg
on commence par lire les instructions dans l'onglet
%
\begin{enumerate*}[label=(\arabic*)]
\item \emph{Instructions \& Consignes,} puis on saisit les
  informations générales sur l'alerte dans l'onglet
\item \emph{métadonnées,} avant de saisir l'alerte en détail dans les
  onglets
\item \emph{interprétation des expressions} et
\item \emph{toponymie,} \emph{etc.}
\end{enumerate*}).

L'onglet \emph{métadonnées,} permet de renseigner toutes les
informations d'ordre général sur une alerte, telles que l'identité du
requérant (victime, témoin, \emph{etc.}) ou l’activité pratiquée lors
de l'accident. C'est également ici que sont saisies toutes les
informations exogènes, comme la date et l'heure de l'appel ou la
position réelle de la victime.

L'onglet \emph{interprétation
  des expressions} se compose, quant à lui, de 15 colonnes
(\autoref{tab:struct_temp}), que l'on peut regrouper en trois
catégories :
%
\begin{enumerate*}[label=(\alph*)]
\item \emph{extrait},
\item \emph{contexte} et
\item \emph{expression.}
\end{enumerate*}

La première d'entre elles regroupe les colonnes décrivant l'extrait
audio traité, comme les deux colonnes \enquote{\emph{identifiant}}, la
colonne \enquote{\emph{extrait}} et la colonne
\enquote{\emph{timestamp}}. Les colonnes \enquote{\emph{identifiant}}
permettent de donner une référence unique à chaque élément
transcrit. La présence de deux colonnes permet de faire la distinction
entre les \emph{extraits,} qui correspondent à découpage rythmique de
la conversation (changement de locuteur, nouvelle phrase,
\emph{etc.}), et les \emph{expressions,} contenues dans les
\emph{extraits,} correspondant au découpage en \emph{indices de
  localisation.} Par exemple, si la conversation entre le secouriste
(\bsc{s.}) et le requérant (\bsc{r.}) est :
%
\begin{quote}
  \begin{dialogue}
    \Sec \enquote{Vous êtes au sommet ?}
    %
    \Req \enquote{Non, je suis bien en dessous.}
  \end{dialogue}
\end{quote}
%
On identifiera deux \emph{extraits :}
%
\begin{enumerate*}[label=(\alph*)]
\item \enquote{Vous êtes au sommet ?} et
\item \enquote{Non, je suis bien en dessous},
\end{enumerate*}
%
délimités par le changement de locuteur. Si le premier ne contient
qu'une seule \emph{expression,} le second en contient deux, la
négation de la question, dont la sémantique correspond à
\emph{l'expression} \enquote{je ne suis pas au sommet} et
\emph{l'expression} \enquote{je suis bien en dessous}. Ainsi, cet
exemple est composé de deux \emph{extraits} et de trois
\emph{expressions.} Pour faciliter l'analyse et la vérification des
transcriptions, nous avons défini une colonne
\enquote{\emph{extrait}}, permettant de saisir le verbatim de
l'extrait analysé, et une colonne \enquote{\emph{timestamp}} indiquant
la position de l'extrait dans le fichier audio correspondant.

La seconde catégorie regroupe les colonnes permettant de décrire le
contexte de chaque expression. On y retrouve la colonne
\enquote{\emph{locuteur}}, renseignant sur la personne ayant prononcé
l'extrait étudié (\eg secouriste, requérant) et la colonne
\enquote{\emph{confiance}}, permettant au transcripteur (dans le cas
où il s'agit d'un secouriste) de juger de la plausibilité de
l'information donnée par l'extrait. Si nous distinguons ces deux
colonnes des \emph{métadonnées} de l'extrait (\ie le groupe de
colonnes précédent) c'est car ces dernières demandent une
interprétation de l'extrait, contrairement aux \emph{métadonnées.}

Enfin, la troisième catégorie, qui regroupe la majorité des colonnes,
permet l'interprétation et la saisie de chaque expression. On y
retrouve les colonnes \emph{verbe,} \emph{sujet,} \emph{relation de
  localisation} et \emph{objet de référence,} correspondant aux
éléments du même nom dans la formalisation des \emph{indices de
  localisation} et les \emph{modifieurs} qui y sont associés. Les
colonnes \enquote{\emph{objet de référence}} et
\enquote{\emph{modifieur de l'objet de référence}} ont la
particularité de pourvoir être doublées (voir plus si nécessaire) dans
le cas où la \emph{relation de localisation} le nécessiterait.

% Exemple
Si l'on reprend le second extrait de l'exemple précédent (\ie
\enquote{Non, je suis bien en dessous}), deux extraits sont à
saisir. Le premier est la négation de la question du secouriste (\ie
\enquote{Vous êtes au sommet}), que l'on peut interpréter comme
\emph{l'indice de localisation} \enquote{je ne suis pas au
  sommet}. Ici le verbe de l'expression est \enquote{être}, le sujet
est le requérant, la \emph{relation de localisation} est la
préposition \enquote{à} et l'objet de référence est
---~implicitement~--- \enquote{le sommet} \footnote{Le \enquote{à le}
  est contracté en \enquote{au} dans l'extrait}. La forme négative de
la phrase est transcrite avec un \emph{modifieur du verbe} et les
autres modifieurs ne sont pas renseignés. Pour la seconde expression
(\ie \enquote{je suis bien en dessous}) le \emph{sujet} et sont
modifieur et \emph{l'objet de référence} et sont modifieur sont
inchangés, il s'agit toujours du \emph{requérant} qui se situe par
rapport au \emph{sommet.} Le \emph{verbe} est toujours \enquote{être},
mais son modifieur disparait, l'expression n'étant pas une
négation. Pour finir la \emph{relation de localisation} devient
\enquote{\emph{en dessous}} et son modifieur est
\enquote{\emph{bien}}.

Enfin, le dernier onglet du template de transcription est destiné à la
saisie détaillée des \emph{objets de référence utilisés.} Comme pour
l'onglet des \emph{extraits,} chaque ligne correspond à une
\emph{expression} issue de l'alerte. Par ailleurs toutes les lignes de
ce troisième onglet doivent être saisies, il y a donc une bijection
entre les enregistrements de l'onglet \enquote{expressions} et ceux de
l'onglet \enquote{objets de référence}. Le but de cet onglet est de
permettre la saisie d'informations complémentaires sur les
\emph{objets géographiques} utilisés dans les expressions, comme les
\emph{objets de référence} ou le \emph{sujet,} notamment dans le cas
où l'expression décrit la position d'un autre objet qu'une personne
(\eg \enquote{le chalet à côté de la cascade}). En effet, lorsqu'on
saisit un \emph{objet géographique} (\eg \emph{l'objet de référence})
dans l'onglet \emph{expression} on utilise le terme employé par le
locuteur dans l'expression (\eg \enquote{La Bérarde}, \enquote{un
  sommet}, \emph{etc.}) sans modification. Ainsi les objets définis
sont désignés par leur nom (ou tout du moins le nom que le locuteur
leur donne) et les objets indéfinis par leur type. L'onglet
\emph{toponymie} permet alors de préciser le type de tous les objets
géographiques, en vue de leur analyse ultérieure. Si l'on reprend
l'exemple précédent (\ie \enquote{Non, je suis bien en dessous}) nous
ne sommes pas en mesure d'ajouter des informations sur \emph{l'objet
  de référence} \enquote{sommet}, son nom étant inconnu.

\begin{table}
  \centering
  \begin{tabular}{L{.3\textwidth}>{\footnotesize}p{.4\textwidth}}
  \toprule \multicolumn{1}{c}{\bfseries Colonne} &
  \multicolumn{1}{c}{\normalsize \bfseries Contenu} \\ \midrule
% Sémantique des relations spatiales
  \addlinespace
  Identifiant de l'extrait & dddd \\
  Identifiant de l'expression & dd \\
  Extrait & Verbatim de la phrase transcrite\\
  Confiance & Permet au transcripteur de saisir sa confiance en
              \emph{l'indice de localisation} (uniquement si la saisie
              est faite par un secouriste)\\
  Timestamp & Début de l'extrait dans le fichier audio source\\
  Locuteur & \\
  Verbe & Verbe utilisé dans \emph{l'indice de localisation}\\
  Modifieur du verbe & \\
  Sujet & Sujet de \emph{l'indice de localisation}\\
  Modifieur du sujet & \\
  \emph{Relation de localisation} & \\
  Modifieur de la relation de localisation & \\
  Objet de référence & Nom ou type de \emph{l'objet de référence}
                       (cette colonne peut être multipliée si la
                       \emph{relation de localisation} est bi ou n-aire)\\
  Modifieur de l'objet de référence & (cette colonne peut être multipliée si la
                                      \emph{relation de localisation} est bi ou n-aire)\\
  Commentaires & Champ permettant au transcripteur de commenter sa
                 saisie ou les indications données par le requérant
                 (erreurs potentielles, fautes de prononciation, \emph{etc.}) \\
  \bottomrule
\end{tabular}

  \caption{Structure de l'onglet \enquote{\emph{expressions}} du
    template de retranscription.}
  \label{tab:struct_temp}
\end{table}

\subsubsection{Analyse des retranscriptions}

Les deux corpus d'alertes ont été transcrits séparément. Le premier
d'entre-eux a été transcrit au cours de l'année 2018 par un secouriste
du \ac{pghm} de Grenoble. Ce premier essai a mis en évidence de
nombreuses lacunes dans le processus de retranscription. La version
originale du template n'était pas assez documentée et ne contenait pas
d'exemples. Par conséquent le processus de saisie était assez délicat
et les résultats étaient très dépendants du transcripteur. Nous avons
donc clarifié le template pour faciliter la transcription
---~aboutissant à la version présentée ci-dessus~--- et avons corrigé
collégialement les transcriptions pour aboutir à une version qui fasse
l'objet d'un consensus. Compte tenu des problèmes que nous avions
rencontrés lors de la saisie du premier corpus nous avons adopté une
démarche de transcription plus avancée. La trentaine d'alertes à
transcrire a été répartie entre trois chercheurs
\autocite{Bunel2019}. Après une première transcription, les fichiers
produits ont été amendés et commentés par un second transcripteur, en
vue d'aboutir, en collaboration avec le transcripteur initial, à une
seconde version. Une fois l'ensemble du second corpus retranscrit nous
avons procédé a une harmonisation collégiale des saisies, de manière à
obtenir un ensemble de retranscriptions le plus cohérent possible. Au
final, la retranscription des 52 alertes nous a permis d'identifier
374 expressions différentes, que nous avons ensuite analysé.

% configuration vs position
L'étude détaillée des \emph{relations de localisation} utilisées dans
les \emph{indices} nous a permis de constater que tous les indices ne
permettent pas de construire \emph{directement} une \emph{zone de
  localisation compatible.} Par exemple, \emph{l'indice de
  localisation :} \enquote{Je marche vers le nord} ne peut pas être
\emph{spatialisé,} à moins de la combiner avec d'autres \emph{indices}
(\eg \enquote{Je marche depuis une heure, vers le nord}) ou de les
approximer avec d'autres relations de localisation (\eg \enquote{Je
  suis en direction du nord}). Ces \emph{indices de localisation,} qui
dénotent une \emph{configuration spatiale,} décrivent généralement une
trajectoire, suivie par le \emph{sujet} et partant de \emph{l'objet de
  référence.} Par conséquent, cette distinction recoupe celle qui est
généralement faite en linguistique entre les \emph{relations de
  localisation statiques,} décrivant une position fixe et les
\emph{relations de localisation dynamiques,} décrivant un mouvement
\autocite{Borillo1998}. Comme l'ontologie que nous développons est
avant tout destinée à la \emph{spatialisation} des \emph{indices de
  localisation,} nous avons pris la décision de ne pas y intégrer les
\emph{relations de localisation} décrivant une \emph{configuration
  spatiale,} la \emph{spatialisation} de trajectoires ne faisant pas
partie de nos objectifs.

\subsection{Construction des ontologies}

%\subsubsection{Cadre des ontologies}

Nous avons choisi de scinder l'ontologie des règles de décomposition
en deux ontologies complémentaires, pour faciliter leur manipulation
et leur diffusion. La première d'entre elles est \emph{l'ontologie des
  relations de localisation} \acp{orl}. Son objectif est de constituer
un thésaurus des \emph{relations de localisation} utilisées dans le
contexte de la description d'une position en montagne. Chacune des
notions est définie et une grande partie d'entre elles sont illustrées
par des extraits des retranscriptions. La seconde ontologie définie
est celle des \emph{relations de localisation atomiques} \acp{orla},
dont l'objectif est de définir les \emph{relations de localisation
  atomiques,} les décompositions dont elles résultent et le processus
de \emph{spatialisation} \footnote{Ce point spécifique sera détaillé
  dans le \autoref{chap:06}.} Ces deux composantes, fortement liées,
auraient pu être combinées dans une même ontologie. Cependant, alors
que la décomposition est fortement liée à notre travail, le thésaurus
des relations de localisation peut être exploité pour d'autres travaux
au sein du projet Choucas ou à l'extérieur. C'est pourquoi nous avons
décidé de séparer les deux ontologies et de ne diffuser que
l'ontologie des \emph{relations de localisation.}

\begin{table}
  \centering
  \begin{tabular}{>{\small}L{.3\textwidth}>{\small}p{.4\textwidth}}
  \toprule \multicolumn{1}{c}{\ac{orl}} &
  \multicolumn{1}{c}{\ac{orla}} \\ \midrule
  \addlinespace
  Bla & Blo \\
  Bla & Blo \\
  Bla & Blo \\
  Bla & Blo \\
  \bottomrule
\end{tabular}

  \caption{Éléments de comparaison des ontologies \ac{orl} et
    \ac{orla}.}
  \label{tab:orl_vs_orla}
\end{table}

\subsubsection{L'ontologie des \emph{relations de localisation}}

La construction de l'ontologie des relations de localisation s'est
faite en deux étapes. Dans un premier temps nous avons défini les
concepts à partir des \emph{relations de localisation} utilisées dans
les alertes, puis nous avons hiérarchisé ces concepts.

% Défintion des concepts
Pour chaque expression nous avons cherché à définir un concept
permettant d'exprimer la sémantique de la \emph{relation de
  localisation} utilisée.
%
Pour ce faire nous avons cherché à identifier, pour chaque expression,
le concept de l'ontologie de \textcite{Bateman2010} le plus proche de
la sémantique de la relation de localisation utilisée par le
requérant.
%
À la suite de cette étape toutes les \emph{relations de localisations}
sont associées à un concept tiré de l'ontologie de
\textcite{Bateman2010}. Cependant, dans la majorité des cas, le
concept issu de GUM-Space ne nous a pas donné entière satisfaction,
soit parce-que le concept retenu ne fonctionnait qu'au pris d'une
interprétation très \enquote{flexible} de la définition de
\textcite{Bateman2010}, soit parce-que la description donnée par le
requérant n'était que grossièrement retranscrite par les concepts de
GUM-Space.

On peut prendre pour exemple la \emph{préposition spatiale}
\enquote{entre} utilisée dans l'extrait suivant :
%
\begin{quote}
  \begin{dialogue}
    \Sec \enquote{Vous êtes entre Grand Veymont et Pas de la Ville ?}
    %
    \Req \enquote{Je suis entre le Grand Veymont, sous le Grand
      Veymont et Pas de la Ville, tout à fait. Je suis côté… heu… côté
      sud.}
  \end{dialogue}
\end{quote}
%
L'ontologie GUM-Space ne propose pas de concepts permettant une
retranscription satisfaisante de la sémantique de cette \emph{relation
  spatiale.} On peut faire appel aux concepts
\onto[gum]{Dis\-tri\-bu\-tion} ou \onto[gum]{Surround\-ing}, mais le
premier s'applique qu'au cas où \emph{l'objet de référence} est une
collection (\eg \enquote{entre les arbres}, \enquote{parmi la foule}),
et le second correspond plus à la \emph{relation de localisation}
\enquote{entouré de}, qui, bien que proche, n'est pas une
approximation satisfaisante de la \emph{relation de localisation}
utilisée par le requérant.

De la même manière, dans l'extrait :
%
\begin{quote}
  \begin{dialogue}
    \Req \enquote{On est à 10 minutes du sommet de la Bastille.}
  \end{dialogue}
\end{quote}
%
Le requérant décrit sa position à l'aide d'une durée de marche à
partir de \emph{l'objet de référence,} \ie à l'aide d'une
\emph{distance-temps quantitative.} Cependant, l'ontologie GUM-Space
ne possède qu'un seul concept permettant d'exprimer une distance
quantitative, \onto[gum]{Quanti\-tative\-Dist\-ance}. Au vu de la
description qui est faite de ce concept dans \textcite{Bateman2010} on
peut légitimement penser que ce concept a essentiellement été prévu
pour conceptualiser des distances métriques, uniquement des distances
métriques. On pourrait se contenter de cette situation, considérant
que ces distances sont équivalentes. Cependant, comme nous
l'expliquions dans l'état de l'art (\autoref{chap:03}) le
distances-temps sont beacoup plus imprécises que les distances
métriques et leur \emph{spatialisation} implique dès lors des méthodes
plus avancées. Par conséquent il nous semble nécessaire de développer
un nouveau concept, \onto{A\-Dist\-ance\-Temps}, plus précis et
adapté.

Au final, 51 \emph{relations de localisation} ont été définies dont
seules 8 sont des concepts originaux de GUM-Space. Les 43 concepts
restants sont soit des précisions de concepts existants (\eg
\onto{A\-Dist\-ance\-Temps}) soit des concepts totalement nouveaux
(\eg \onto{En\-tre\-X\-\&\-Y}).

% Hiérarchisation des concepts
Une fois que l'ensemble des \emph{relations de localisation}
nécessaires a la retranscription des alertes a été défini nous avons
procédé à leur hiérarchisation. Pour ce faire nous avons défini de
nouvelles classes, permettant de regrouper les relations de
localisation par similarité sémantique. Ces nouvelles classes sont
abstraites, \ie qu'elles ne sont pas des \emph{relations de
  localisation,} elles ne servent qu'à leur regroupement.

Pour définir cette hiérarchie nous nous sommes appuyés sur différentes
propositions précédentes et principalement de l'ontologie GUM-Space
\autocite{Bateman2010}.

Au premier niveau de l'ontologie on trouve trois classes principales,
\onto[orl]{Re\-la\-tion\-Spa\-ti\-ale\-De\-Prox\-imi\-té},
\onto[orl]{Re\-la\-tion\-Spa\-ti\-ale\-Fonc\-tio\-nelle} et
\onto[orl]{Re\-la\-tion\-Spa\-ti\-ale\-Po\-si\-ti\-on\-Re\-la\-ti\-ve},
regroupant l'ensemble des \emph{relations de localisation} et deux
\emph{relations de localisation} isolées et complémentaires,
\onto[orl]{Ci\-ble\-Voit\-Si\-te} et \onto[orl]{Si\-te\-Voit\-Ci\-ble}
qui n'entrent dans aucune des ces classes. Les trois classes
principales sont similaires aux trois classes du premier niveau
hiérarchique de l'ontologie GUM-Space. La classe
\onto[orl]{Re\-la\-tion\-Spa\-ti\-ale\-De\-Prox\-imi\-té}, qui
correspond à la classe
\onto[gum]{Spa\-ti\-al\-Dist\-an\-ce\-Mo\-da\-li\-ty} de GUM-Space,
regroupe les \emph{relations de localisation} dont la sémantique
traduit une proximité spatiale. On y retrouve, par exemple, la
relation \onto[orl]{Près\-De} ou la relation
\onto[orl]{A\-Dist\-ance\-Temps}, qui modélise une distance exprimée
en temps de déplacement. La seconde classe,
\onto[orl]{Re\-la\-tion\-Spa\-ti\-ale\-Fonc\-tio\-nelle} est le
pendant de la classe
\onto[gum]{Func\-ti\-on\-al\-Spa\-ti\-al\-Mo\-da\-li\-ty} de
l'ontologie GUM-Space. Elle regroupe les \emph{relations spatiales}
qui décrivent une situation où le sujet et \emph{l'objet de référence}
ont un lien, non seulement spatial, mais aussi fonctionnel. C'est, par
exemple, le cas de la \emph{relation de localisation}
\onto[orl]{Prox\-imal} qui implique une proximité entre le
\emph{sujet} et \emph{l'objet de référence}, mais qui, contrairement à
la relation \onto[orl]{Près\-De}, peut être fonctionnelle et pas
seulement spatiale. C'est également le cas de la \emph{relation de
  localisation}
\onto[orl]{Si\-tué\-Sur\-Iti\-né\-rai\-re\-Ou\-Ré\-seau\-Sup\-port},
la seule autre \emph{relation fonctionnelle} que nous avons défini.
Enfin, la classe
\onto[orl]{Re\-la\-tion\-Spa\-ti\-ale\-Po\-si\-ti\-on\-Re\-la\-ti\-ve},
similaire à la classe
\onto[gum]{Re\-la\-ti\-ve\-Spa\-ti\-al\-Mo\-da\-li\-ty}, regroupe
toutes les \emph{relations} qui décrivent la position du \emph{sujet}
en fonction de celle du \emph{site} mais sans notion de
distance. C'est cette classe qui regroupe le plus de \emph{relations
  de localisation.} On y retrouve, par exemple, les relations
traduisant un contact\footnote{Correspondant à la classe
  \onto[orl]{Re\-la\-tion\-Spa\-ti\-ale\-De\-Con\-tact}.} (\eg
\onto[orl]{Dans\-Pla\-ni\-mé\-tri\-que},
\onto[orl]{A\-La\-Fr\-on\-ti\-ère\-De}), les relations de direction
\footnote{Correspondant à la classe
  \onto[orl]{Re\-la\-tion\-Spa\-ti\-ale\-De\-Di\-rec\-tion}.} (\eg
\onto[orl]{Au\-Nord\-De}, \onto[orl]{Dans\-La\-Dir\-ec\-tion\-De}) ou
les relations verticales \footnote{Correspondant à la classe
  \onto[orl]{Re\-la\-tion\-Spa\-ti\-ale\-De\-Ver\-ti\-ca\-le}.} (\eg
\onto[orl]{So\-us\-Pro\-che\-De}, \onto[orl]{Au\-Des\-sus\-De}). Au
total, l'ontologie des \emph{relations de localisation} contient 11
classes servant à définition de la hiérarchie. Ces classes, qui ne
correspondant à aucune \emph{relation de localisation,} ne sont ni
décomposables, ni spatialisables. C'est pourquoi nous les qualifions
de \emph{classes abstraites} et nous les signalons comme telles dans
l'ontologie, pour éviter qu'elles soient manipulées comme des
\emph{relations de localisation} 

On peut remarquer que certaines des \emph{relations de localisation}
que nous avons présentées pourraient appartenir à plusieurs
\emph{classes abstraites}. C'est par exemple le cas de la
\emph{relation} \onto[orl]{Prox\-imal}, que nous avons décrit comme
appartenant à la \emph{classe abstraite}
\onto[orl]{Re\-la\-tion\-Spa\-ti\-ale\-Fonc\-tio\-nelle}, mais qui
traduit également une notion de proximité spatiale. La \emph{relation}
\onto[orl]{Prox\-imal} pourrait donc également appartenir à la
\emph{classe abstraite}
\onto[orl]{Re\-la\-tion\-Spa\-ti\-ale\-De\-Prox\-imi\-té}. De fait les
regroupements que nous avons définis n'ont pas été conçus pour être
mutuellement disjoints \footnote{C'est également le cas de la
  hiérarchie proposée dans GUM-Space \autocite{Bateman2010}.}, ainsi
une même \emph{relation} peut, comme \onto[orl]{Prox\-imal},
appartenir à plusieurs classes. Ce découpage ne pose cependant pas de
problèmes particuliers, quelle que soit le cas d'utilisation de
l'ontologie. Nous pensons même qu'il peut être utile pour certaines
applications, comme la recherche de \emph{relations de localisation}
par les secouristes utilisant l'interface.

Les concepts définis dans \ac{orl} étant fortement inspirés, voir
repris, de l'ontologie GUM-Space, nous avons complété l'ontologie en
liant, pour chaque concept le concept équivalent (ou similaire) de
l'ontologie GUM-Space. Plus spécifiquement deux types de liens sont
définis, la relation \emph{seeAlso,} utilisée 24 fois et la relation
\emph{equivalentClass,} utilisée 8 fois. Cette dernière traduit une
équivalence parfaite entre concepts, alors que la relation
\emph{seeAlso} est avant tout informative et traduit une certaine
proximité sémantique entre deux concepts sans qu'ils soient
équivalents ou même similaires. Ainsi, seules 8 des 62 \emph{relations
  de localisation} définies dans \ac{orl} sont identiques à celles
définies par \textcite{Bateman2010}, les autres concepts sont définis
de manière significativement différente.

\begin{figure}
  \centering
  \begin{tikzpicture}[
  every node/.style={anchor=west,
    font={\tiny\ttfamily\textcolor{red}{orl\#}\sffamily}, 
    % text width=5cm
  },
  abstraite/.style={gray},
  atomique/.style={blue}
  ],
  % 
  \begin{scope}[
    level distance=7.5mm,
    edge from parent path={(\tikzparentnode.south) |- ($(\tikzparentnode.south)!0.5!(\tikzchildnode.north)$) -|
      (\tikzchildnode.north)},
    level 1/.style={sibling distance=55mm, anchor=north,align=center},
    ]
    % \tikzstyle{edge from parent}=[draw,red,thick]
    % Parents
    \node[abstraite](rl){RelationDeLocalisation}
    child{node[level 1, abstraite] (r1){Re\-la\-ti\-on\-Spa\-ti\-ale\-Fonc\-ti\-o\-nel\-le}}
    child{node[level 1, abstraite] (r2){Re\-la\-ti\-on\-Spa\-ti\-ale\-Po\-si\-tion\-Re\-la\-ti\-ve}}
    child{node[level 1, abstraite] (r3){RelationSpatialeProximité}};
  \end{scope}

  \begin{scope}[grow=right,
    level distance=10mm,
    sibling distance=5mm,
    edge from parent path={(\tikzparentnode.east) -| ($(\tikzparentnode.east)!0.5!(\tikzchildnode.west)$) |- (\tikzchildnode.west)},
    ]
    \node[font=\tiny] at (rl.east){}
    child{node{CibleVoitSite}}
    child{node{SiteVoitCible}};
  \end{scope}

  \begin{scope}[grow via three points={one child at (.5,-.15) and
      two children at (.5,-.15) and (.5,-.45)},
    growth parent anchor=south west,
    edge from parent path={([xshift=.25cm]\tikzparentnode.south west) |-
      (\tikzchildnode.west)}]
    \node[font=\tiny] at (r1.west){}
    child{node(tt){Proximal}}
    child{node{SitueSurItineraire\-Ou\-Re\-seau\-Sup\-port}};
  \end{scope}


  \begin{scope}[grow via three points={one child at (.5,-.15) and
      two children at (.5,-.15) and (.5,-.45)},
    growth parent anchor=south west,
    edge from parent path={([xshift=.25cm]\tikzparentnode.south west) |-
      (\tikzchildnode.west)}]
    \node[font=\tiny] at (r2.west){}
    child{node[abstraite]{RelationSpatialeDeContact}
      child{node{AExtremiteDe}}
      child{node{ALaFrontiereDe}}
      child{node{DansPlanimetrique}
        child{node{DansLaPartie\-Bas\-se\-De}}
        child{node{DansLaPartie\-De\-X\-La\-Plus\-Pro\-che\-De\-Y}}
        child{node{DansLaPartieEstDe}}
        child{node{DansLaPartieHauteDe}}
        child{node{DansLaPartieNordDe}}
        child{node{DansLaPartieOuestDe}}
        child{node{DansLaPartieSudDe}}
      }
    }
    child[missing]{}
    child[missing]{}
    child[missing]{}
    child[missing]{}
    child[missing]{}
    child[missing]{}
    child[missing]{}
    child[missing]{}
    child[missing]{}         
    child[missing]{}
    child[missing]{}
    child{node{Proximal}
    }
    child{node[abstraite]{RelationSpatialeDeDirection}
      child{node{ADroiteDe}}
      child{node{AGaucheDe}}
      child{node{AvoirASaDroite}}
      child{node{AvoirASaGauche}}
      child{node{AvoirDerriereSoit}}
      child{node{AvoirDevantSoit}}
      child{node{DansLaDirectionDe}}
      child{node[abstraite]{RelationSpatialeDeCardinalité}
        child{node{AEstDe}
          child{node{AEstDeExterne}}
          child{node{DansLaPartieEstDe}}
        }
        child[missing]{}
        child[missing]{}
        child{node{AOuestDe}
          child{node{AOuestDeExterne}}
          child{node{DansLaPartieOuestDe}}
        }
        child[missing]{}
        child[missing]{}
        child{node{AuNordDe}
          child{node{AuNordDeExterne}}
          child{node{DansLaPartieNordDe}}
        }
        child[missing]{}
        child[missing]{}          
        child{node{AuSudDe}
          child{node{AuSudDeExterne}}
          child{node{DansLaPartieSudDe}}
        }
      }
    }
    child[missing]{}
    child[missing]{}
    child[missing]{}
    child[missing]{}
    child[missing]{}
    child[missing]{}
    child[missing]{}
    child[missing]{}
    child[missing]{}
    child[missing]{}
    child[missing]{}
    child[missing]{}
    child[missing]{}
    child[missing]{}
    child[missing]{}
    child[missing]{}
    child[missing]{}
    child[missing]{}
    child[missing]{}
    child[missing]{}
    child[missing]{}
    child{node[abstraite]{RelationSpatialeDisjointe}
      child{node{AEstDeExterne}}
      child{node{AOuestDeExterne}}
      child{node{AuNordDeExterne}}
      child{node{AuSudDeExterne}}
      child{node{HorsDePlanimetrique}}
    }
    child[missing]{}
    child[missing]{}
    child[missing]{}
    child[missing]{}
    child[missing]{}
    child{node[abstraite]{RelationSpatialeItinéraire}
      child{node{ApresJalonSurItineraire}}
      child{node{AuDessusJalonSurItineraire}}
      child{node{AvantJalonSurItineraire}}
      child{node{SitueSurItineraireOuReseauSupport}}
      child{node{SousJalonSurItineraire}}
    }
    child[missing]{}
    child[missing]{}
    child[missing]{}
    child[missing]{}
    child[missing]{} 
    child[missing]{} 
    child{node[abstraite]{RelationSpatialeTernaire}
      child{node{DansLaPartieDeXLaPlusProcheDeY}}
      child{node{DeAutreCoteDeParRapportA}}
      child{node{DuMemeCoteQueParRapportA}}
      child{node{EntreXetY}}
    }
    child[missing]{}
    child[missing]{}
    child[missing]{}
    child[missing]{}
    child[missing]{} 
    child{node[abstraite]{RelationSpatialeVerticale}
      child{node{ALaMemeAltitudeQue}}
      child{node{AuDessusALAplombDe}}
      child{node{AuDessusAltitude}}
      child{node{AuDessusJalonSurItineraire}}
      child{node{AuDessusProche}}
      child{node{DansLaPartieBasseDe}}
      child{node{DansLaPartieHauteDe}}
      child{node{SousALAplombDe}}
      child{node{SousAltitude}}
      child{node{SousJalonSurItineraire}}
      child{node{SousProcheDe}}
      child{node{SousRecouvertPar}}
    };
  \end{scope}

  \begin{scope}[grow via three points={one child at (.5,-.15) and
      two children at (.5,-.15) and (.5,-.45)},
    growth parent anchor=south west,
    edge from parent path={([xshift=.25cm]\tikzparentnode.south west) |-
      (\tikzchildnode.west)}]
    \node[font=\tiny] at (r3.west){}
    child{node{ADistanceQuantitativeRelative}}
    child{node{ADistanceTemps}
      child{node{ATempsDeMarche}}
    }
    child[missing]{}
    child{node{AuDessusProche}}
    child{node{AuxAlentoursDe}}
    child{node{DistanceQuantitativePlanimetrique}}
    child{node{PresDe}}
    child{node{Proximal}};
  \end{scope}

  % Légende
  \begin{scope}
    
  \end{scope}
  
\end{tikzpicture}	

  \caption{Onto}
  \label{fig:ontho}
\end{figure}

\subsubsection{L'ontologie des \emph{relations de localisation
    atomiques}}

Le processus de construction de \emph{l'ontologie des relations de
  localisation atomiques} \acp{orla} est assez différent de celui de
\emph{l'ontologie des relations de localisation.} Là où \ac{orl} a été
construite d'un trait, à partir d'un ensemble de cas réels,
\emph{l'ontologie des relations de localisation atomiques} a été
développée progressivement, au fil du traitement de nouvelles
alertes. Dans sa version actuelle, \ac{orla} ne propose pas une
décomposition pour toutes les \emph{relations de localisation}
définies précédemment. Seules les relations de localisation les plus
fréquemment utilisées dans les alertes ont été modélisées.  La
démarche de définition d'\ac{orla} a consisté a définir des
\emph{relations de localisation atomiques} et des décompositions au
fil de l'eau, en fonction des \emph{relations de localisation}
identifiées dans les alertes traitées.

Pour modéliser les \emph{relations topologiques} nous utilisons
quatres relations de localisation atomiques :
\onto[orla]{In\-té\-ri\-eur\-De},
\onto[orla]{À\-La\-Fron\-ti\-ère\-De},
\onto[orla]{Ex\-té\-ri\-eur\-De} et \onto[orla]{Con\-tact}. Deux de
ces \emph{relations atomiques} (\onto[orla]{In\-té\-ri\-eur\-De} et
\onto[orla]{À\-La\-Fron\-ti\-ère\-De}) peuvent être utilisées
directement, elles correspondent alors aux \emph{relations de
  localisation} \onto[orl]{Dans\-Pla\-ni\-mé\-tri\-que} et
\onto[orl]{À\-La\-Fron\-ti\-ère\-De}. Les deux autres \emph{relations
  atomiques} ne sont utilisées que dans des compostions. La
\emph{relation atomique} \onto[orla]{Ex\-té\-ri\-eur\-De} est par
exemple utilisée pour définir les \emph{relations} de cardinalité
\onto[orl]{A\-Est\-De\-Ex\-ter\-ne} ou
\onto[orl]{Au\-Nord\-De\-Ex\-te\-rne}. De part l'importance que
revêtent les relations topologiques (\autoref{chap:03}) dans le
référencement spatial indirect, on retrouve ces \emph{relations de
  localisation atomiques} dans de nombreuses \emph{relations de
  localisation.}

Les \emph{relations de localisation atomiques} formalisant une
distance sont également très utilisées. Seules deux de ces relations
sont définies dans \ac{orla}
:\onto[orla]{Dis\-tan\-ce\-Quan\-ti\-ta\-ti\-ve\-Pla\-ni\-mé\-tri\-que},
équivalente à son homonyme dans \ac{orl} et
\onto[orla]{À\-Dis\-tan\-ce\-Temps}, qui elle n'est utilisé dans
aucune combinaison. La relation
\onto[orla]{Dis\-tan\-ce\-Quan\-ti\-ta\-ti\-ve\-Pla\-ni\-mé\-tri\-que},
est utilisée dans de nombreuses \emph{relations de localisation,}
comme \onto[orl]{Proximal}, \onto[orl]{Aux\-Alen\-tours\-De} ou
\onto[orl]{Près\-De}.

Pour représenter les \emph{relations de localisation} orientaionelles
seule une \emph{relation de localisation atomique} a été définie :
\onto[orla]{Dans\-La\- Direction\-De}. Cette \emph{relation atomique}
sert de base à toutes les relations de cardinalité (\ie
\onto[orl]{Au\-Nord\-De}, \onto[orl]{À\-Ouest\-De}, \emph{etc.}) et
leurs dérivées composés (\eg \onto[orl]{Au\-Sud\-De\-Ex\-ter\-ne}),
mais est également utilisée directement. Elle correspond alors à la
\emph{relation} homonyme définie dans \emph{l'ontologie des relations
  de localisation}.

La notion de verticalité est exprimée à l'aide de trois principales
\emph{relations de localisation atomiques} complémentaires :
\onto[orla]{Al\-ti\-tu\-de\-Équi\-va\-len\-te},
\onto[orla]{Al\-ti\-tu\-de\-Stric\-te\-ment\-Sup\-érieu\-re} et
\onto[orla]{Al\-ti\-tu\-de\-Stric\-te\-ment\-Inf\-érieu\-re}. Chacune
d'entre elles exprime une relation strictement verticale entre le
\emph{sujet} et \emph{l'objet de référence.} Ces trois \emph{relations
  atomiques} sont a la base de toutes les \emph{relations spatiales}
où est présente une notion de verticalité. Elles peuvent être
utilisées telles quelles, pour désigner des positions relatives sur
l'axe vertical ou être combinées avec d'autres \emph{relations
  atomiques.} Dans le cas où ces relations sont utilisées seules,
elles sont équivalentes aux \emph{relations de localisation}
\onto[orl]{À\-La\-Mê\-me\-Al\-ti\-tu\-de\-Que},
\onto[orl]{Au\-Des\-sus\-Al\-ti\-tu\-de} et
\onto[orl]{Sous\-Al\-ti\-tu\-de}, qui n'imposent aucune contrainte sur
la distance entre \emph{l'objet de référence} et \emph{le sujet} (\eg
\enquote{le sommet de la tour Eiffel est moins au que celui de
  l’Everest}). Si on les combine avec la \emph{relation}
\onto[orla]{Dis\-tan\-ce\-Quan\-ti\-ta\-ti\-ve} on peut construire les
relations de localisation \onto[orla]{Sous\-Pro\-che\-De} ou
\onto[orla]{Au\-Des\-sus\-Pro\-che\-De} qui sont très utilisées dans
les alertes retranscrites. Enfin, ces relations peuvent également être
combinées avec des \emph{relations atomiques} topologiques, comme
\onto[orla]{In\-té\-ri\-eur\-De} ce qui permet la définition de
\emph{relations} comme \onto[orl]{Dans\-La\-Par\-tie\-Bas\-se\-de} ou
\onto[orl]{Dans\-La\-Par\-tie\-Hau\-te\-de}.

Ces trois \emph{relations atomiques} ont la particularité d'être
fortement dépendantes de la valeur d'altitude utilisée comme point de
référence. C'est pourquoi nous avons défini, pour chacune de ces trois
\emph{relation de localisation atomiques} des variantes, utilisant une
valeur d'altitude différente (\autoref{tab:atomique_vertical}). Par
exemple, les \emph{relations de localisation atomiques}
\onto[orla]{Al\-ti\-tu\-de\-Stric\-te\-ment\-Sup\-érieu\-re} et
\onto[orla]{Al\-ti\-tu\-de\-Stric\-te\-ment\-Inf\-érieu\-re} se
basent, respectivement, sur les altitudes maximales et minimales de
\emph{l'objet de référence.} Ce parti pris n'est cependant pas
compatible avec certaines \emph{relations de localisation.} Par
exemple les \emph{relations}
\onto[orl]{Dans\-La\-Par\-tie\-Bas\-se\-de} ou
\onto[orl]{Dans\-La\-Par\-tie\-Hau\-te\-de} se référent à l'altitude
moyenne de l'objet et non à son maximum ou son minimum. Ainsi les
trois \emph{relations de localisation atomiques} de verticalité
principales sont complétées par des variantes basées sur l'altitude
moyenne, sur l'altitude du plus proche voisin et sur les altitudes
maximales ou minimales (\autoref{tab:atomique_vertical}), ce qui
permet de disposer d'une expressivité maximale.

\begin{table}
  \centering
  \begin{tabular}{r>{\footnotesize}p{.25\textwidth}>{\footnotesize}p{.25\textwidth}>{\footnotesize}p{.25\textwidth}}
  \toprule
  & \multicolumn{1}{c}{\bfseries Équivalente} & \multicolumn{1}{c}{\bfseries Supérieure} & \multicolumn{1}{c}{\bfseries  Inférieure}\\
  \midrule
  \addlinespace
  \bfseries Moyenne &
                      \multicolumn{1}{p{.25\textwidth}}{\bfseries\onto[orla]{Al\-ti\-tu\-de\-Équi\-va\-len\-te}}
                                              &
                                                \onto[orla]{Al\-ti\-tu\-de\-Sup\-érieu\-re} & \onto[orla]{Al\-ti\-tu\-de\-Inf\-érieu\-re}\\

 \bfseries Maximum &
                     \onto[orla]{Al\-ti\-tu\-de\-Équi\-va\-len\-te\-Maxi\-mum}& \multicolumn{1}{p{.25\textwidth}}{\bfseries\onto[orla]{Al\-ti\-tu\-de\-Stric\-te\-ment\-Sup\-érieu\-re}} & \\
  
  \bfseries Minimum &
                      \onto[orla]{Al\-ti\-tu\-de\-Équi\-va\-len\-te\-Min\-ni\-mum}& & \multicolumn{1}{p{.25\textwidth}}{\bfseries\onto[orla]{Al\-ti\-tu\-de\-Stric\-te\-ment\-Inf\-érieu\-re}}\\
  
  \bfseries Voisin & \onto[orla]{Al\-ti\-tu\-de\-Équi\-va\-len\-te\-Plus\-Pro\-che\-Voi\-sin} & \onto[orla]{Al\-ti\-tu\-de\-Sup\-érieu\-re\-Plus\-Pro\-che\-Voi\-sin} & \onto[orla]{Al\-ti\-tu\-de\-Inf\-érieu\-re\-Plus\-Pro\-che\-Voi\-sin}\\
  \bottomrule
\end{tabular}

  \caption{\emph{Relations de localisation atomiques} de verticalité
    et leurs vairantes.}
  \label{tab:atomique_vertical}
\end{table}

La dernière catégorie des \emph{relations de localisation atomiques}
concerne les \emph{relations} de visibilité. Cette catégorie présente
la particularité de définir des \emph{relations de localisation
  atomiques} qui ne sont utilisables que telles quelles (\ie qu'elles
ne sont composées à aucune autre \emph{relation atomique}). Les deux
relations de visibilité définies dans \ac{orl} (\ie
\onto[orl]{Ci\-ble\-Voit\-Si\-te} et
\onto[orl]{Si\-te\-Voit\-Ci\-ble}) sont donc peu liées aux autres
\emph{relations,} d'une part car elles ne sont pas décomposables, mais
également car les \emph{relations atomiques} qui y correspondent ne
sont utilisées dans aucune autre décomposition.

Chacune des \emph{relations de localisation atomique} définie dans
\ac{orla} est associée à un ensemble de règles supplémentaires,
permettant sa \emph{spatialisation.} Ces différentes règles seront
explicitées lors de la présentation de la phase de spatialisation
(\autoref{chap:07}).


%%% Local Variables:
%%% mode: latex
%%% TeX-master: "../../../../main"
%%% End:
