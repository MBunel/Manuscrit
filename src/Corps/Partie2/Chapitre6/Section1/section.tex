
\tdi{On veut un modèle qui permet de prendre en compte des relations
  de localisation traduisant des limites assez nettes (en dessous de)
  et des relations avec des limites très continues (proches [voir
  concept similaire dans l'ontologie]).}

Tout d'abord il est nécessaire que la modélisation retenue permette de
rendre compte des variations de \emph{précision} inhérentes aux
différentes \emph{relations de localisation.} En effet, là où
certaines des relations définies dans \ac{orl} 

\tdi{Possibilité de construire des zones de localisation issues de
  relations de localisation non linéaire (en dessous de).}

\tdi{Possibilité de fusionner (intersection et union). Nécessité que
l'union et l'intersection soient commutatives.}

\tdi{Nécessité que les résultats soient facilement
  interprétables. Sémantique des résultats et pas trop de zones (par
  exemple beaucoup d'alpha cuts).}

\tdi{Stabilité du modèle ? (par exemple si l'on fait des intersection
  avec le modèle egg-yolk on doit choisir ce qu'est le jaune. C'est
  l'intersection des jaunes ? ou il suffit qu'il y a ait un jaune?
  est-ce qu'il faut introduire des valeurs intermédiaires (gris,
  lorsqu'on intersecte jaune et blanc) -> mais dans ce cas on sort du
  egg yolk, le modèle n'est pas stable alors). Le résultat doit
  pouvoir s'exprimer avec la même modélisation sans perte de
  sémantique et de finesse de modélisation.}

\missingfigure{Figure explicitant le point 5}

%%% Local Variables:
%%% mode: latex
%%% TeX-master: "../../../../main"
%%% End:
