\begin{tikzpicture}
  % grille temp
  % \draw[step=1.0,black,thin] (0,0) grid (20,-12);
  % Arrow

  \newcommand{\calcangle}[4]{%
    % #1 nom de la varaible renvoyée
    % #2 coordonnées de l'objet de référence
    % #3 coordonnées du point visé
    % #4 coordonées de la position testée
    \pgfmathanglebetweenlines{#2}{#3}{#2}{#4}%
    \global\let#1\pgfmathresult%
  }

  
  \begin{scope}
    % \path[draw, -, shorten >=5pt, shorten <=5pt] (-1,0) -- (-5,-2);
    % \path[draw, ->,shorten >=5pt, shorten <=5pt] (8.25,3) -- (10,3);
    % \path[draw, -,shorten >=12pt, shorten <=5pt] (1,-3) |- (3.65,-1);
    % \path[draw, ->,shorten >=5pt, shorten <=5pt] (8.25,-1) -- (10,-1);
  \end{scope}


  %%%%%%%%%%%% 
  % Colonne 1
  %%%%%%%%%%%% 
  % 
  % 
  % Zone de localisation compatible 1
  \begin{scope}[local bounding box=zlc1]
    \begin{scope}
      \foreach \x in {0,.25,...,1.75}{ \foreach \y in {0,.25,...,1.75}
        {
          % Calcul de l'angle
          \calcangle{\angle}{\pgfpoint{0.125cm}{1.375cm}}{\pgfpoint{1.75cm}{0cm}}{\pgfpoint{\x cm}{\y cm}}
          % Fuzzyfication de l'angle
          \pgfmathsetmacro\fuzzy{%
            ifthenelse(\angle < 90,(90-\angle)/90,
            ifthenelse(\angle > 270,(\angle-270)/90,0)
            )
          }
          % Calcul du rayon à partir de la fuzzyfication
          \pgfmathsetmacro\radius{\fuzzy*2.5}
          \fill[RdBu-9-1] (\x+.125,.125+\y) circle (\radius pt);
        }
      }
      \path[ffc] (0,0) rectangle (2,2);
    \end{scope}
    \node[text width=3cm, align=center, anchor=north, font=\footnotesize] at (1,-.25)
    {\itshape Zone de localisation compatible \normalfont \textcolor{RdBu-9-1}{\textsf{A}}};
  \end{scope}
  % 
  % ZLC 2
  \begin{scope}[yshift=-3.5cm,local bounding box=zlc2]
    \begin{scope}
      \foreach \x in {0,.25,...,1.75}{
        \foreach \y in {0,.25,...,1.75}
        {
          % Calcul et représentation floue distance euclidienne
          \pgfmathsetmacro\dist{sqrt((\x-0.125)^2+(\y-1.375)^2)}
          % Fuzzyfication de la distance
          \pgfmathsetmacro\fuzzy{%
            ifthenelse(\dist < .25,1,%
            ifthenelse(\dist < 2,-0.57*\dist+1.14,0)%
            )
          }
          % Calcul du rayon à partir de la fuzzyfication
          \pgfmathsetmacro\radius{(1-\fuzzy)*2.5}
          \fill[RdBu-9-2] (\x+.125,.125+\y) circle (\radius pt);
        }
      }
      \path[ffc, RdBu-9-2] (0,0) rectangle (2,2);
    \end{scope}
    \node[text width=3cm, align=center, anchor=north, font=\footnotesize] at (1,-.25)
    {\itshape Zone de localisation compatible \normalfont \textcolor{RdBu-9-2}{\textsf{B}}};
  \end{scope}
  % 
  % ZlC 3
  \begin{scope}[yshift=-8cm,local bounding box=zlc3]
    \begin{scope}
      \foreach \x in {0,.25,...,1.75}{
        \foreach \y in {0,.25,...,1.75}
        {
          % Calcul et représentation floue distance euclidienne
          \pgfmathsetmacro\dist{sqrt((\x-0.125)^2+(\y-1.375)^2)}
          % Fuzzyfication de la distance
          \pgfmathsetmacro\fuzzy{%
            ifthenelse(\dist < .25,1,%
            ifthenelse(\dist < 2,-0.57*\dist+1.14,0)%
            )
          }
          % Calcul du rayon à partir de la fuzzyfication
          \pgfmathsetmacro\radius{max(\fuzzy,0.3)*2.5}
          \fill[RdBu-9-8] (\x+.125,.125+\y) circle (\radius pt);
        }
      }
      \path[ffc2, RdBu-9-8] (0,0) rectangle (2,2);
    \end{scope}
    \node[text width=3cm, align=center, anchor=north, font=\footnotesize] at (1,-.25)
    {\itshape Zone de localisation compatible \normalfont \textcolor{RdBu-9-8}{\textsf{C}}};
  \end{scope}
  % 
  % Zlc 4
  \begin{scope}[yshift=-11.5cm,local bounding box=zlc4]
    \begin{scope}
      \foreach \x in {0,.25,...,1.75}{
        \foreach \y in {0,.25,...,1.75}
        {
          % Calcul et représentation floue distance euclidienne
          \pgfmathsetmacro\dist{sqrt((\x-2.125)^2+(\y-1.875)^2)}
          % Fuzzyfication de la distance
          \pgfmathsetmacro\fuzzy{%
            ifthenelse(\dist < .25,1,%
            ifthenelse(\dist < 2,-0.57*\dist+1.14,0)%
            )
          }
          % Calcul du rayon à partir de la fuzzyfication
          \pgfmathsetmacro\radius{max(\fuzzy,0.3)*2.5}
          \fill[RdBu-9-9] (\x+.125,.125+\y) circle (\radius pt);
        }
      }
      \path[ffc2] (0,0) rectangle (2,2);
    \end{scope}
    \node[text width=3cm, align=center, anchor=north, font=\footnotesize] at (1,-.25)
    {\itshape Zone de localisation compatible \normalfont \textcolor{RdBu-9-9}{\textsf{D}}};
  \end{scope}
  %%%%%%%%%%%% 
  % Colonne 2
  %%%%%%%%%%%% 
  \begin{scope}[xshift=6cm, yshift=-1.75cm]
    \begin{scope}
      \foreach \x in {0,.25,...,1.75}{
        \foreach \y in {0,.25,...,1.75}
        {
          % Calcul de l'angle
          \calcangle{\angle}{\pgfpoint{0.125cm}{1.375cm}}{\pgfpoint{1.75cm}{0cm}}{\pgfpoint{\x cm}{\y cm}}
          % Fuzzyfication de l'angle
          \pgfmathsetmacro\fuzzya{%
            ifthenelse(\angle < 90,(90-\angle)/90,
            ifthenelse(\angle > 270,(\angle-270)/90,0)
            )
          }
          %
          % Calcul et représentation floue distance euclidienne
          \pgfmathsetmacro\dist{sqrt((\x-0.125)^2+(\y-1.375)^2)}
          % Fuzzyfication de la distance
          \pgfmathsetmacro\fuzzyb{%
            ifthenelse(\dist < .25,1,%
            ifthenelse(\dist < 2,-0.57*\dist+1.14,0)%
            )
          }
          %
          \pgfmathsetmacro\radius{min(\fuzzya,1-\fuzzyb)*2.5}
          \fill[RdBu-9-1] (\x+.125,.125+\y) circle (\radius pt);
        }
      }
      \path[ffc] (0,0) rectangle (2,2);
    \end{scope}
    \node[text width=3cm, align=center, anchor=north, font=\footnotesize] at (1,-.25)
    {\itshape Zone de localisation compatible \normalfont \textcolor{RdBu-9-1}{\textsf{AB}}};
  \end{scope}

  \begin{scope}[xshift=6cm, yshift=-8cm]
    \begin{scope}
      \foreach \x in {0,.25,...,1.75}{
        \foreach \y in {0,.25,...,1.75}
        {
          % Calcul et représentation floue distance euclidienne
          \pgfmathsetmacro\dist{sqrt((\x-0.125)^2+(\y-1.375)^2)}
          % Fuzzyfication de la distance
          \pgfmathsetmacro\fuzzy{%
            ifthenelse(\dist < .25,1,%
            ifthenelse(\dist < 2,-0.57*\dist+1.14,0)%
            )
          }
          % Calcul du rayon à partir de la fuzzyfication
          \pgfmathsetmacro\radius{max(\fuzzy,0.3)*2.5}
          \fill[RdBu-9-8] (\x+.125,.125+\y) circle (\radius pt);
        }
      }
      \path[ffc2, RdBu-9-8] (0,0) rectangle (2,2);
    \end{scope}
    \node[text width=3cm, align=center, anchor=north, font=\footnotesize] at (1,-.25)
    {\itshape Zone de localisation compatible \normalfont \textcolor{RdBu-9-8}{\textsf{C}}};
  \end{scope}
  
  \begin{scope}[xshift=6cm, yshift=-11.5cm]
    \begin{scope}
      \foreach \x in {0,.25,...,1.75}{
        \foreach \y in {0,.25,...,1.75}
        {
          % Calcul et représentation floue distance euclidienne
          \pgfmathsetmacro\dist{sqrt((\x-2.125)^2+(\y-1.875)^2)}
          % Fuzzyfication de la distance
          \pgfmathsetmacro\fuzzy{%
            ifthenelse(\dist < .25,1,%
            ifthenelse(\dist < 2,-0.57*\dist+1.14,0)%
            )
          }
          % Calcul du rayon à partir de la fuzzyfication
          \pgfmathsetmacro\radius{max(\fuzzy,0.3)*2.5}
          \fill[RdBu-9-9](\x+.125,.125+\y) circle (\radius pt);
        }
      }
      \path[ffc2] (0,0) rectangle (2,2);
    \end{scope}
    \node[text width=3cm, align=center, anchor=north, font=\footnotesize] at (1,-.25)
    {\itshape Zone de localisation compatible \normalfont \textcolor{RdBu-9-9}{\textsf{D}}};
  \end{scope}
  %%%%%%%%%%% 
  % Colonne 3
  %%%%%%%%%%%% 
  \begin{scope}[xshift=12cm, yshift=-1.75cm]
    \begin{scope}
      \foreach \x in {0,.25,...,1.75}{
        \foreach \y in {0,.25,...,1.75}
        {
          % Calcul de l'angle
          \calcangle{\angle}{\pgfpoint{0.125cm}{1.375cm}}{\pgfpoint{1.75cm}{0cm}}{\pgfpoint{\x cm}{\y cm}}
          % Fuzzyfication de l'angle
          \pgfmathsetmacro\fuzzya{%
            ifthenelse(\angle < 90,(90-\angle)/90,
            ifthenelse(\angle > 270,(\angle-270)/90,0)
            )
          }
          %
          % Calcul et représentation floue distance euclidienne
          \pgfmathsetmacro\dist{sqrt((\x-0.125)^2+(\y-1.375)^2)}
          % Fuzzyfication de la distance
          \pgfmathsetmacro\fuzzyb{%
            ifthenelse(\dist < .25,1,%
            ifthenelse(\dist < 2,-0.57*\dist+1.14,0)%
            )
          }
          %
          \pgfmathsetmacro\radius{min(\fuzzya,1-\fuzzyb)*2.5}
          \fill[RdBu-9-1] (\x+.125,.125+\y) circle (\radius pt);
        }
      }
      \path[ffc] (0,0) rectangle (2,2);
    \end{scope}
    \node[text width=3cm, align=center, anchor=north, font=\footnotesize] at (1,-.25)
    {\itshape Zone de localisation compatible \normalfont \textcolor{RdBu-9-1}{\textsf{AB}}};
  \end{scope}
  % 
  \begin{scope}[xshift=12cm, yshift=-9.75cm]
    \begin{scope}
      \foreach \x in {0,.25,...,1.75}{
        \foreach \y in {0,.25,...,1.75}
        {
          \pgfmathsetmacro\dista{sqrt((\x-0.125)^2+(\y-1.375)^2)}
          % Fuzzyfication de la distance
          \pgfmathsetmacro\fuzzya{%
            ifthenelse(\dista < .25,1,%
            ifthenelse(\dista < 2,-0.57*\dista+1.14,0)%
            )
          }
          % Calcul et représentation floue distance euclidienne
          \pgfmathsetmacro\distb{sqrt((\x-2.125)^2+(\y-1.875)^2)}
          % Fuzzyfication de la distance
          \pgfmathsetmacro\fuzzyb{%
            ifthenelse(\distb < .25,1,%
            ifthenelse(\distb < 2,-0.57*\distb+1.14,0)%
            )
          }
          \pgfmathsetmacro\radius{max(max(\fuzzya, 0.3), max(\fuzzyb, 0.3))*2.5}
          \fill[RdBu-9-9](\x+.125,.125+\y) circle (\radius pt);
        }
      }
      \path[ffc2] (0,0) rectangle (2,2);
    \end{scope}
    \node[text width=3cm, align=center, anchor=north, font=\footnotesize] at (1,-.25)
    {\itshape Zone de localisation compatible \normalfont \textcolor{RdBu-9-9}{\textsf{CD}}};
  \end{scope}
  %%%%%%%%%%% 
  % Colonne 4
  %%%%%%%%%%%% 
  \begin{scope}[xshift=18cm, yshift=-5.75cm]
    \begin{scope}
      \foreach \x in {0,.25,...,1.75}{
        \foreach \y in {0,.25,...,1.75}
        {          
          % Calcul de l'angle
          \calcangle{\angle}{\pgfpoint{0.125cm}{1.375cm}}{\pgfpoint{1.75cm}{0cm}}{\pgfpoint{\x cm}{\y cm}}
          % Fuzzyfication de l'angle
          \pgfmathsetmacro\fuzzya{%
            ifthenelse(\angle < 90,(90-\angle)/90,
            ifthenelse(\angle > 270,(\angle-270)/90,0)
            )
          }
          %
          % Calcul et représentation floue distance euclidienne
          \pgfmathsetmacro\dist{sqrt((\x-0.125)^2+(\y-1.375)^2)}
          % Fuzzyfication de la distance
          \pgfmathsetmacro\fuzzyb{%
            ifthenelse(\dist < .25,1,%
            ifthenelse(\dist < 2,-0.57*\dist+1.14,0)%
            )
          }

          \pgfmathsetmacro\dista{sqrt((\x-0.125)^2+(\y-1.375)^2)}
          % Fuzzyfication de la distance
          \pgfmathsetmacro\fuzzyc{%
            ifthenelse(\dista < .25,1,%
            ifthenelse(\dista < 2,-0.57*\dista+1.14,0)%
            )
          }
          % Calcul et représentation floue distance euclidienne
          \pgfmathsetmacro\distb{sqrt((\x-2.125)^2+(\y-1.875)^2)}
          % Fuzzyfication de la distance
          \pgfmathsetmacro\fuzzyd{%
            ifthenelse(\distb < .25,1,%
            ifthenelse(\distb < 2,-0.57*\distb+1.14,0)%
            )
          }
          \pgfmathsetmacro\radius{min(\fuzzya, 1-\fuzzyb, max(\fuzzyc, 0.3), max(\fuzzyd, 0.3))*2.5}          
          \fill[RdBu-9-1] (\x+.125,.125+\y) circle (\radius pt);
        }
      }
      \path[ffc2] (0,0) rectangle (2,2);
    \end{scope}
    \node[text width=3cm, align=center, anchor=north, font=\footnotesize] at (1,-.25)
    {\itshape Zone de localisation probable};
  \end{scope}
  % Annotations
  \begin{scope}
    \begin{scope}
      \begin{scope}
        \path[ffc] (1.75cm,1.75cm) rectangle (2cm,2cm);
      \end{scope}
      \begin{scope}[yshift=-3.5cm]
        \path[ffc] (1.75cm,1.75cm) rectangle (2cm,2cm);
      \end{scope}
      \begin{scope}[xshift=6cm, yshift=-1.75cm]
        \path[ffc] (1.75cm,1.75cm) rectangle (2cm,2cm);
      \end{scope}
      \node[align=center,anchor=center] (zi) at (4,-1) {$\top(a,b)=c$};
      \path[->, draw] (2,2) -- (zi.west);
      \path[->, draw] (2,-1.5) -- (zi.west);
      \path[->, draw] (zi.east) -- (7.75,.25);
    \end{scope}
    % 
    \begin{scope}
      \begin{scope}[xshift=6cm, yshift=-8cm]
        \path[ffc] (1.75cm,1.75cm) rectangle (2cm,2cm);
      \end{scope}
      \begin{scope}[xshift=6cm, yshift=-11.5cm]
        \path[ffc] (1.75cm,1.75cm) rectangle (2cm,2cm);
      \end{scope}
      \begin{scope}[xshift=12cm, yshift=-9.75cm]
        \path[ffc] (1.75cm,1.75cm) rectangle (2cm,2cm);
      \end{scope}
      \node[align=center,anchor=center] (zou) at (12,-6) {$\bot(\textcolor{RdBu-9-8}{\textsf{C}},\textcolor{RdBu-9-9}{\textsf{D}})=\textcolor{RdBu-9-9}{\textsf{CD}}$};
      \path[->, draw] (8,-6) -- (zou.west);
      \path[->, draw] (8,-9.5) -- (zou.west);
      \path[->, draw] (zou.east) -- (13.75,-7.75);
    \end{scope}
    % 
    \begin{scope}
      \begin{scope}[xshift=12cm, yshift=-1.75cm]
        \path[ffc] (1.75cm,1.75cm) rectangle (2cm,2cm);
      \end{scope}
      \begin{scope}[xshift=18cm, yshift=-5.75cm]
        \path[ffc] (1.75cm,1.75cm) rectangle (2cm,2cm);
      \end{scope}
      \node[align=center,anchor=center] (zu) at (16,-4) {$\top(a,b)=c$};
      \path[->, draw] (14,.25) -- (zu.west);
      \path[->, draw] (14,-7.75) -- (zu.west);
      \path[->, draw] (zu.east) -- (19.75,-3.75);
    \end{scope}
  \end{scope}
\end{tikzpicture}