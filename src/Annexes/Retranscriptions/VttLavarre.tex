\subsection{Retranscription de l'alerte}
Avec $a_1$ un employé du SAMU
\begin{dialogue}
  \Sec \direct{Enregistrement coupé} Eh ben je veux savoir précisément… ouais…
  où vous êtes, pour, pour… pouvoir venir.
  \Req Alors, précisément on le sait pas…
  \Sec Ouais.
  \Req Mais là on est sur un chemin qui descend vers la passerelle du Drac.
  \Sec Qui descend vers la passerelle du Drac. Ok, ouais ?
  \Req Comme elle est indiquée sur les panneaux de signalétique.
  \Sec Ouais. Eh… heu… Vous êtes à VTT ?
  \Req Oui.
  \Sec Vous êtes à VTT. Vous êtes sur un gros chemin, ou pas ?
  \Req C'est un gros chemin. Il fait… je sais pas… deux mètres de large, quoi.
  \Sec Est-ce que… Ouais, est-ce qu'une voiture pourrait rouler dessus ?
  \Req Ah non, non, non, non, non. En aucun cas.
  \Sec En aucun cas.
  \speak{$a_1$} Le PGHM ?
  \Sec Ouais ?
  \speak{SAMU} C'est le SAMU là. Alors, je vais quitter la conf…
  \Sec Ouais.
  \speak{SAMU} Pour l'instant, si vous pouvez intervenir rapidement ça sera non médicalisé…
  \Sec Ok.
  \speak{SAMU} Parce qu’il y'a déjà le secours en montagne…
  \Sec D'accord.
  \speak{SAMU} … médicalisé en cours…
  \Sec Ok.
  \speak{SAMU} … Et si ça prend du temps, parce qu’il a un peu mal, mais c'est pas une blessure…
  \Sec En gros le bilan c'est quoi ?
  \speak{SAMU} C'est juste un avant-bras… une fracture de l'avant-bras fermée… hyperalgique.
  \Sec Ok, ça roule. Ok
  \speak{SAMU} Merci. Donc je vous laisse hein ? Merci, au revoir.
  \Sec Au revoir
  
  \direct{Tonalité raccrochage}

  \Sec \direct{À mi-voix} Ok, ça roule. \direct{Ton normal} Ok, donc on va reprendre un petit peut, là, pour vous positioner correctement. Donc, vous êtes sur un chemin qui fait à peu près deux mètres de large.
  \Req Oui.
  \Sec Y'a pas possibilité de faire passer une voiture dessus ?
  \Req Non, non, non, non.
  \Sec Non, ok, d'accord, ça roule. Heum… La passerelle qui descend c'est… c'est… heu… c'est pas celle du haut, hein, c'est bien celle du bas, celle qui est un petit peu plus au sud ?
  \Req Heu… la passerelle, heu… là on allait arriver sur la passerelle qui est au sud-est, heu…, du lac.
  \Sec Ouais, ok, d'accord.
  \Req C'est pas la neuve qui est côté Avignonet.
  \Sec C'est pas la neuve qui est côté Avignonet, ok.
  \Req C'est l'autre mais j'y suis jamais allé, donc… je la connait pas, je peux pas vous donner plus de précisions.
  \Sec Heum… A peut pret à combien de distance de cette passerelle, en gros ? Tu sais ou pas ?
  \Req j'ai croisé des gens qui m'ont dit \enquote{à peut pres un kilomètre.}
  \Sec D'accord. Et t'es dans les bois toi ou pas ?
  \Req Heu…
  \Sec C'est de la forêt ou tu est ou pas ?
  \Req Ouais, c'est de la forêt.
  \Sec C'est de la forêt. Et… Est-ce que… \direct{En s'adressant à un autre secouriste} Non c'est pas carrossable il me dit. Ok, d'accord. \direct{S'adresse au requérant} Bon écoute on pense savoir à peut pres où vous êtes. Heu… on voit en arrivant d'hélico… quand… si on arrive d'hélicoptère, est-ce qu'on voit au-dessus où vous êtes ?
  \Req Heu…
  \Sec Est-ce que c'est clairsemé d'arbres ou est-ce que c'est très dense ?
  \Req Ouais… c'est clairsemé mais c'est assez dense quand même.
  \Sec Ah ? C'est dense, ok.
  \Sec Y'a peut-être quelques passages pas très loin qui sont peut-être un peu dégagés ou je peux venir en bas. \direct{Enregistrement coupé}
\end{dialogue}
