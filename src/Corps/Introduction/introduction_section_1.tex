Lorsque George Perec relate sa visite d’une maison usonienne\footnote{
  Néologisme de Frank Lloyd Wright. Qualifie des petites maisons
  individuelles en harmonie avec leur environnement, une part
  importante de son œuvre construit.} du Michigan, il en décrit le
délicat cheminement vers l’intérieur:

\begin{displayquote}[Perec, 1974, pp. 52-53]
  \enquote{%
    On commençait par suivre un sentier\textelp{}. Peu à peu,
    \textelp{} sans qu’à aucun instant on ait été en droit d’affirmer
    avoir perçu quelque chose comme une transition \textelp{}, le
    sentier devenait \textelp{} une allée \textelp{}. Puis
    apparaissait \textelp{} une toiture \textelp{} pratiquement
    indissociable de la végétation \textelp{}. Mais en fait, il était
    déjà trop tard pour savoir si l’on était dehors ou dedans.%
  }
\end{displayquote}

Les jeux avec l’environnement et les matériaux, relevés par
l’écrivain, sont les éléments d’un parcours savamment orchestré par
l’architecte pour fondre la construction dans son environnement,
trompant le visiteur et l’empêchant d’identifier une délimitation
claire entre dehors et dedans ; deux espaces tacitement considérés
comme immiscibles. Ainsi, en lieu et place d’une rupture franche,
c’est une transition progressive qui les sépare, rendant la définition
d’une limite, autre qu’arbitraire, impossible.

Or, la délimitation d’espaces cohérents est centrale dans la réflexion
géographique et l’existence d’objets spatiaux difficilement
délimitables ne va pas sans soulever des questions techniques et
épistémologiques \autocite{Burrough1996b}, auxquelles de nombreux
travaux, comme ceux cherchant à délimiter des espaces à partir de
ressentis \autocite{Arabacioglu2010}, de descriptions de positions
\autocite{Jones2007, Wolter2018,Bunel2019} ou de cartes mentales
\autocite{Dutozia2014}, ont été confrontés. Ces différents travaux ont
pour point commun d’avoir nécessité l’emploi de modèles avancés,
permettant la représentation d’objets spatiaux aux frontières mal
délimitées. Mais le grand nombre de modèles de ce type et l’absence de
consensus pour l’un d’eux rend délicat leur recensement et leur
analyse en vue d’une application originale.

C’est ce travail de recensement et de classification que nous
souhaitons entreprendre ici, en espérant qu’il puisse servir de point
d’entrée à toute personne amenée à manipuler des objets spatiaux
difficilement délimitables.  Cet état de l’art présente avant tout une
catégorisation des différents modèles proposés dans la littérature,
mais également le cadre conceptuel auxquels ils se rattachent et
notamment la notion d’imprécision au centre de ces questions. Notre
objectif n’est cependant pas de proposer une typologie originale des
différents concepts, la question ayant déjà été largement traitée
\autocite{Bouchon-Meunier1995,Fisher2006,Devilliers2019}. Nous
présenterons également les différentes théories mathématiques
permettant la modélisation de l’imprécison, comme l’ont récemment
proposé \autocite{Batton-Hubert2019}, ainsi que les implémentation de
modèles proposées.

Nous commencerons par présenter plus abondamment ces notions et
définir les différents concepts, tels que le vague ou l’imprécision,
nécessaires à leur compréhension. Puis nous énumérerons les
différentes théories mathématiques permettant de modéliser
l’imprécision. Enfin nous présenterons les différentes modélisations
de l’imprécision spatiale proposées dans la littérature.

\section{Les concepts de vague et d’imprécision spatiale}

Pour commencer nous allons définir les notions d’imprécision et de
vague et présenter leurs spécificités lorsqu’elles sont appliquées au
contexte spatial. Nous commencerons par présenter les notions dans
leur ensemble, avant de nous concentrer sur les liens entre
imprécision et géographie. Puis nous définirons l’imprécision
spatiale, concept qui sera utilisé tout au long de cet état de l’art.

L’exercice de définition d’un concept équivoque, tel que le vague,
n’est pas une tâche aisée, car l’on est rapidement confronté à
l’imprécision sémantique du langage naturel. C’est, en partie, ce
constat qui conduisit les philosophes \bsc{Frege} (1848–1925) et
\bsc{Russell} (1872–1970) à travailler sur une formalisation
mathématique de la logique, à même d’affranchir le processus de
réflexion des ambiguïtés du langage naturel \autocite{Williamson1994}.

La réflexion contemporaine sur les notions de précision, de vague et
d’imprécision remonte, selon \textcite{Williamson1994}, aux travaux de
Russell et plus spécifiquement à la publication, en 1923, de son
article \emph{Vagueness} \autocite{Russell1923}. Pour \bsc{Russell},
le vague est l’opposé de la précision. Les deux concepts ont pour
domaine tout système de signes et ne se limitent donc pas au langage
naturel. Ils concernent tout type de représentation (\eg cartes,
photographies, mots) et n’ont donc de sens que pour qualifier une
relation entre deux systèmes de signes, définie comme précise si
bijective, \ie pour un système de signes donné, quel que soit le signe
considéré, il ne partage sa signification qu’avec un et un seul signe
d’un second système de signes. À l’inverse, la relation entre deux
systèmes de signes est vague si une représentation à plus d’un (ou
aucun) équivalent dans le second système, laissant dès lors place à
l’interprétation. Pour illustrer le concept de vague, Russell prend
l’exemple du mot \enquote{rouge} décrivant une teinte tacitement
connue de tous, mais dont on ne peut qu’arbitrairement fixer les
limites. De la même manière, les concepts de lac et île ne sont pas
suffisamment précis pour que leur dénombrement soit trivial
\autocite{Sarjakoski1996}. On pourrait multiplier les exemples à
loisir, car, dans la conception russellienne, aucun domaine n’échappe
au vague; toute représentation l’est\footnote{Y compris le terme
  \enquote{vague}, lui-même \autocite{Russell1923}, on parle alors
  d’imprécision d’ordre supérieur ou \emph{hight-order vagueness}
  \autocite{Williamson1994, Varzi2003}.}, à des degrés divers, et la
précision n’est qu’un idéal, hors d’atteinte.

Dans la littérature, le terme imprécis est régulièrement utilisé comme
synonyme de vague, notamment par \textcite{Zadeh1965} et dans une
grande partie des travaux se rattachant à la logique floue. Par
métonymie, le terme flou\footnote{Ou \emph{fuzzy} dans les
  publications anglophones.} est également employé dans le sens de
vague, imprécis. C’est, par exemple, le cas lorsque
\textcite{Lagacherie1996} parlent de \emph{fuzziness} ou encore quand
\textcite[218]{Brunet1992} définissent le flou comme : \enquote{la
  partie d’un système ou d’un espace dont les contours et les limites
  sont, soit imparfaitement connus ou connaissables, soit instables,
  soit imprécis \textelp{}}. De nombreux autres termes sont
ponctuellement utilisés, rendant la terminologie confuse
(\autoref{tab:1}). Pour rendre notre propos le plus clair possible, nous
n’emploierons le terme \emph{flou} que pour qualifier des
formalisations fondées sur la théorie des sous-ensembles flous de
\bsc{Zadeh.} De plus, pour rester le plus proche possible du
vocabulaire utilisé en géomatique nous préférerons le terme
\emph{imprécis} à celui de \emph{vague}.

\begin{table}
  \centering
  \begin{tabular}{lm{5cm}m{5cm}}
    \firsthline
    &\multicolumn{1}{c}{\textbf{Précis}} & \multicolumn{1}{c}{\textbf{Imprécis}}\\
    \cline{2-3}
    \textbf{Anglophone}&\emph{Crisp, Sharp, Well defined, Fiat boundaries} &\emph{Fuzzy, Indeterminate, Undefined, Uncertain, Ill-Defined, Unclear, Vagueness, Bona fide boundaries}\\
    \textbf{Francophone}&\emph{Net}&\emph{Flou, Incertain, Vague}\\
    \lasthline
  \end{tabular}
  \caption{sqdq}
  \label{tab:1}
\end{table}

La notion d’\emph{imprécision} peut être associée à d’autres concepts,
comme l’\emph{exactitude} chez \textcite{Russell1923}, que l’on
retrouve chez \textcite{Bouchon-Meunier1995,Bouchon-Meunier2007} sous
le nom d’\emph{incertitude}. Ici, l’\emph{incertitude} est entendue
comme le doute que l’on peut avoir sur la validité d’une connaissance
\autocite{Bouchon-Meunier1995}. L’imprécision et l’incertitude sont
foncièrement liées et varient généralement en sens inverse
\autocite{Russell1923}. Ainsi, si la proposition (1) : \enquote{la
  distance de la Terre à la Lune est de \SI{384397}{\km}} est plus
  précise que la proposition (2): \enquote{la distance de la Terre à
    la Lune est d’environ \SI{384000}{\km}}, mais la seconde
  proposition est plus certaine car, son cadre de validité (la plage
  de valeurs de la distance Terre-Lune) est plus large. On peut en
  effet, considérer que la proposition (2) reste vraie pour une
  distance réelle de \SI{384397} ou \SI{384000}{\km} alors que la
moindre variation de l’ordre d’un kilomètre suffit à invalider la
première proposition (1). Dans certains cas, et notamment lorsque ces
notions sont appliquées à des objets spatiaux, il peut être délicat de
distinguer ces deux concepts. Cependant, ils sont fondamentalement
différents : l’\emph{imprécision} est une caractéristique invariable,
alors que l’\emph{incertitude} est contextuelle. Pour reprendre
l’exemple précédent, la proposition (2) est \emph{imprécise} et le
restera quel que soit le contexte, alors que sa certitude dépend des
connaissances de l’observateur. La véracité des propositions (1) et
(2) est, toutes choses égales par ailleurs, invariable, mais la
certitude de cette véracité est contextuelle.

L’\emph{imprécision} et l’\emph{incertitude} sont également associées
à la notion d’\emph{incomplétude}, qui désigne une connaissance
partielle. Ceci est dû au fait qu’un manque de connaissances peut
entraîner des \emph{incertitudes}, mais également des
\emph{imprécisions}
\autocite{Bouchon-Meunier1995,Bouchon-Meunier2007}. Pour
\textcite{Bouchon-Meunier1995} et plus généralement pour la communauté
de chercheurs en intelligence artificielle, la composition de ces
trois concepts définit la notion d’\emph{imperfection}. Dans la suite
de ce document, nous travaillerons à partir de cette typologie, même
si de nombreuses autres typologies de concepts ont cependant été
proposées, que ce soit dans le domaine de l’intelligence artificielle
ou de la géomatique (Fisher et al.,, 2006).

\subsection{Imprécision et géographie}

Les objets et les concepts géographiques n’échappent évidemment pas à
l’imprécision. Ainsi, \textcite{Russell1923} mentionnait déjà
l’existence d’objets dont la délimitation spatiale est imprécise, tel
que le système solaire. De nombreux autres objets spatiaux imprécis
ont été identifiés, comme l’illustre l’exercice de définition du
\emph{Brownfield}, entrepris par \textcite{Alker2000} et relevée par
\textcite{Bennett2001}, qui y voit un bon exemple de la difficulté
d’identifier une délimitation satisfaisante d’espaces naturels. Le
\emph{Brownfield}, tout comme les \emph{forêts}
\autocite{Bennett2001,Dilo2006,Fisher2006}, les \emph{montagnes}
\autocite{Varzi2001,Fisher2006,Chaudhry2008}, les \emph{vallées}
\autocite{Schneider2003} ou même le \emph{Soleil}
\autocite{Simons1999}, appartiennent à cette catégorie d’objets
spatiaux dont on ne peut fixer une limite. Cette énumération pourrait
laisser penser que l’imprécision ne concerne pas les artefacts,
pourtant l’expérience de \bsc{Perec} nuance cette affirmation. Comme
l’indique \textcite{Campari1996}, l’identification des frontières d’un
artefact, n’est pas aisée puisque dépendante du contexte
d’observation. Ainsi, la limite d’une \emph{ville} ou d’un
\emph{village} est tout aussi \emph{vague} que celle d’une zone
\emph{frontalière} \autocite{Varzi2001,Fisher2006}, comme l’illustre
la grande variabilité des définitions du concept de ville.

Tous ces objets géographiques, généralement qualifiés de \emph{vagues}
\autocite{Erwig1997}, \emph{imprécis} \autocite{Winter2000},
\emph{flous} \autocite{Lagacherie1996} ou d’\emph{objets aux
  frontières indéterminées} \autocite{Burrough1996}, s’opposent aux
objets dits \emph{nets} \autocite{Schneider2001} ou \emph{précis.}
\textcite{Smith2000} font usage d’un vocabulaire très différent en
opposant les \emph{fiat boundaries,} \ie les frontières précises, aux
\emph{bona fide boundaries,} caractérisant les objets spatiaux dont la
délimitation est univoque (\nameref{tab:1}). Pour
\textcite{Couclelis1996}, les objets géographiques nets sont plus
l’exception que la norme. Ce constat est corollaire de l’avis d’
\textcite{OddAmbrosetti1987} pour qui \enquote{\textelp{} il est
  problématique et généralement arbitraire de tracer des
  limites\textelp{}}\footnote{\foreignquote{italian}{La realta ci
    mostra quanto sia problematico e spesso arbitrario tracciare dei
    confini \textelp{}} \autocite[200]{OddAmbrosetti1987}, traduction
  de l’auteur.}, limites qui selon \textcite[106]{Brunet2001} sont
\enquote{\textelp{} indécises, fuyant sans cesse devant l’analyse, et
  même, localement indécidables}. \textcite{Dutozia1994} considèrent,
quant à eux, que \enquote{\textelp{} l’espace géographique est par
  essence flou \textelp{}}.

Ainsi, si les concepts présentés jusqu’ici nous semblent actuellement
peu utilisés en géographie, de nombreuses notions et objets entrant
dans le champ d’étude de la discipline y sont fondamentalement
liés. C’est notamment le cas des différents maillages administratifs,
comme les régions \autocite{Brennetot2014}, mais aussi des frontières
\autocite{Brunet1992}, des seuils \autocite{Brunet1992, Levy2013}, des
discontinuités \autocite{Brunet1992, Brunet1997}, des franges
\autocite{Brunet1992}, des confins \autocite{Brunet1997} ou encore des
fronts pionniers \autocite{Brunet1992}, qui, comme tous les concepts
dérivant de la notion de limite sont généralement définis comme
pouvant être graduels ou progressifs \autocite{Brunet1992, Levy2013},
\ie foncièrement imprécis.

La question de la formalisation des objets spatiaux imprécis a
cependant été abordée en géographie, avant même le développement des
systèmes d’information géographiques \autocite{Robinson2003}. Dans les
années 70, ou, à la suite de l’élaboration de la théorie des
sous-ensembles flous \autocite{Zadeh1965}, plusieurs géographes,
rattachés au courant \emph{béhavioriste,} tels que
\textcite{Gale1972,Gale1976}, \textcite{Pipkin1978} ou
\textcite{Leung1979,Leung1987} ont identité les problèmes que
l’existence d’objets géographiques aux limites imprécises pouvait
poser à la géographie. Parmi ces problèmes, \textcite{Gale1976} a
identifié la question de la régionalisation. Problématique qui sera
également abordée par \textcite{Rolland-May1996,Rolland-May1999} lors
de ses travaux sur la définition de \emph{territoires de cohérence.}
Les travaux \emph{béhavioristes} aboutiront à la formalisation du
concept \emph{d’objet géographique imprécis} à l’aide de la théorie
des sous-ensembles flous \autocite{Leung1987}. Ces travaux n’auront,
semble-t-il, pas suffi à inscrire durablement le concept
d’\emph{imprécision} dans le champ de la géographie, puisque,
régulièrement apparaîtront des publications ré-introduisant ce concept
en géographie, c’est notamment le cas de \textcite{Fisher1998},
\textcite{Collins2000, Varzi2001} qui se fonderont indépendamment sur
l’exemple de la définition d’une montagne pour introduire cette
notion.

Parallèlement, \textcite{Rolland-May1984,Rolland-May1987} se fondera
notamment sur les travaux de \bsc{Gale} et \bsc{Leung} pour développer
le \emph{concept d’espace géographique flou.} Cette dénomination
qualifie la formalisation, à l’aide de la théorie des sous-ensembles
flous, de l’espace tel que conceptualisé en géographie. Ce travail,
permettra à \bsc{Rolland-May} de proposer une définition formelle de
notions courant utilisées en géographie, telles que les \emph{franges}
\autocite{Rolland-May1987}, définies comme la limite floue d’un espace
géographique, ou les \emph{discontinuités}, dérites comme une
configuration particulière d’ensemble flou
\autocite{Rolland-May2003}. Les différents travaux de
\bsc{Rolland-May} autour de la question de l’imprécision en géographie
ont offert a différents chercheurs d’aborder différemment des
questions géographiques \autocite{Dutozia2014}, comme, par exemple,
\textcite{deRuffray2004}, qui feront usage des concepts développés par
\bsc{Rolland-May} pour quantifier la cohérence de territoires, ou
\textcite{Didelon2013} qui emploient la logique floue pour exploiter
des cartes mentales.

\subsection{L’imprécision spatiale}

Nous proposons d’utiliser le terme d’imprécision spatiale pour décrire
l’application du concept d’imprécision aux objets spatiaux. Par ce
terme nous entendons qualifier toutes les situations où un objet
spatial, quelle que soit sa nature, voit ses limites difficilement
définissables. Il s’agit donc d’un concept ne portant que sur la
dimension spatiale et non sur les autres aspects. Par exemple, la
difficulté de délimitation spatiale d’un objet géographique tel que la
forêt entre dans le cadre de l’imprécision spatiale ; ce n’est
cependant pas le cas de la difficulté de définition du concept en
lui-même, il s’agit dans ce cas d’imprécision sémantique. Ainsi
l’imprécision spatiale n’est qu’un cas spécifique du concept général
d’imprécision, précédemment présenté, mais son cadre d’application et
les spécificités de la question spatiale justifient la définition d’un
nouveau concept.

Pour illustrer ce concept nous allons nous appuyer l’exemple de la
définition des rives d’un lac artificiel. La Figure 1 est une
orthophotographie de la partie ouest du lac du Chambon (Isère) sur
laquelle a été dessiné la limite de l’eau. On peut cependant se
demander si la limite que nous avons tracée est une délimitation
satisfaisante de l’objet lac. En effet le niveau de l’eau est amené à
bouger au cours du temps. L’orthophotographie permet d’identifier ces
zones, dépourvues de végétation et situées au-delà de la limite
représentée sur la Figure 1. On peut donc tracer une seconde limite,
celle de la zone atteignable par les eaux (Figure 2) et considérer que
c’est ce nouveau tracé qui délimite l’objet lac.

Toutefois, aucune de ces délimitations n’est réellement
satisfaisante. Peut-on considérer qu’une zone pouvant être découverte
appartient autant à l’objet lac qu’une zone qui est toujours
recouverte d’eau ? À l’inverse, peut-on considérer qu’une zone
intermittement située sous l’eau n’appartient pas au lac de la même
manière que la forêt située à plusieurs dizaines de mètres de là ?
Cette difficulté de délimitation est liée, comme nous l’expliquions
précédemment, à l’imprécision de l’objet lac. On ne peut en définir
une limite précise autrement qu’arbitrairement. Nous avons cependant
pu tracer deux limites précises[ Si l’on passe outre l’imprécision
liée à la saisie manuelle], celle de la zone recouverte d’eau (Figure
1) et celle de l’étendue maximale du lac (Figure 2). Ces deux
frontières délimitent une aire de transition, entre le lac et son
extérieur (Figure 3), i.e. la frontière du lac.

De même que pour l’imprécision, le concept d’incertitude ne voit pas
sa définition générale impactée par la prise en compte de la dimension
spatiale. Cependant, la multiplicité des termes utilisés dans la
littérature, les contradictions entre auteurs et les représentations
graphiques utilisées pour présenter les concepts sont sources de
nombreuses confusions entre les concepts d’imprécision et
d’incertitude spatiale.

Comme expliqué précédemment, l’incertitude qualifie le doute que l’on
peut avoir sur une connaissance. On peut donc définir l’incertitude
spatiale comme le doute sur la position d’un objet. Ce concept
peut-être, tout du moins selon Tøssebro et Nygård (2002), décomposée
en deux éléments : l’incertitude positionnelle et l’incertitude
morphologique. L’incertitude positionnelle qualifie un doute sur la
position d’un objet spatial. Tøssebro et Nygård (2008) prennent comme
exemple l’estimation par sonar de la position d’un sous-marin. Ce cas
offre une bonne opportunité pour distinguer imprécision et incertitude
spatiale. L’objet sous-marin, est, de part sa nature d’artefact, net ;
l’identification de ses frontières, à une échelle donnée, ne pose pas
de problèmes. Cependant sa position est mal connue, puisqu’estimée à
l’aide d’un outil peu précis qui ne peut qu’estimer une zone de
présence. La position de l’objet est donc incertaine. L’incertitude
morphologique qualifie, quant à elle, un doute sur la forme de
l’objet, i.e. sur la position de sa frontière. C’est pourquoi on peut
considérer que l’incertitude morphologique correspond à une
incertitude positionnelle portant uniquement sur la frontière de
l’objet. On peut prendre comme exemple une nappe phréatique, dont
l’étendue ne peut être qu’estimée par des relevés terrain. Il est donc
possible de savoir si la nappe est présente ou non en un point de
mesure, mais non d’en définir la frontière, ce qui se traduit par une
incertitude sur la position de la frontière, une incertitude
morphologique telle que définie par Tøssebro et Nygård. La proximité
des concepts d’incertitude morphologique et positionnelle peut
expliquer pourquoi les autres définitions de l’incertitude spatiale,
notamment issues des travaux de Clementini (2008), Lagacherie et
al. (1996), Freksa et Barkowsky[ Freksa et Barkowsky (1996), utilisent
le terme d’incertitude pour qualifier aussi bien ce que nous
définissons comme l’incertitude spatiale, que ce que nous définissons
comme l’imprécision spatiale.] (1996) ou Schneider (1999) fusionnent
ces deux notions. Dutton (1992), quant à lui, inclut ces deux notions
dans sa définition de l’incertitude positionnelle.

On peut illustrer la notion d’incertitude spatiale en réutilisant
l’exemple de la délimitation du lac du Chambon (Figures 1 et
2). Lorsque nous avons tracé la limite du niveau maximal de l’eau
(Figure 2), nous nous sommes appuyés sur l’emplacement de la
végétation. Cependant, dans certains cas, notamment pour la rive sud
du lac, il s’est avéré difficile d’identifier la bonne limite et ce à
cause de la présence de certaines poches de végétation. Ainsi la
frontière sud de l’étendue maximale du lac est incertaine, son tracé
est contestable, mais uniquement à cause de notre manque de
connaissances, il ne s’agit pas d’imprécision. La Figure 4 représente
une zone au sein de laquelle le tracé exact de la frontière est
incertain, i.e. qu’au sein de cette zone, tous les tracés sont
envisageables[ Nous aurions également pu faire les mêmes remarques
pour le tracé de la limite de l’eau (Figure 1). Mais pour éviter les
confusions nous avons choisi de nous limiter à l’exemple de la
frontière extérieure.]. Ainsi, le lac du Chambon, rentre dans la
catégorie des objets spatiaux à la fois imprécis et incertains.

L’incertitude et l’imprécision spatiales peuvent cohabiter, comme dans
le cas du lac du Chambon. La Figure 5 donne un aperçu plus théorique
de la différence qu’il peut y avoir entre ces deux
notions. L’incertitude spatiale est représentée par plusieurs
frontières, illustrant le doute sur la position de la limite de
l’objet. L’imprécision spatiale est, quant à elle, représentée par une
bande, marque d’une frontière progressive, non réductible à une ligne.

De façon similaire à ce qui a été décrit précédemment, l’incertitude
et l’imprécision spatiale sont liées. Par exemple, le fait qu’un objet
géographique soit imprécis complexifie l’identification de sa
frontière, ce qui se traduit par une incertitude morphologique
(Lagacherie et al., 1996). De plus, la précision et la certitude des
attributs d’un objet géographique est fortement liée à la précision et
à la certitude spatiale de ce même objet et inversement (Mark et
Csillag, 1989). Ainsi la définition d’un objet géographique à partir
de données imprécises le sera elle-même, et le recueil d’informations
au sein d’un objet géographique dont la frontière est incertaine ne
pourra qu’être une opération qui l’est tout autant.

Différents auteurs ont listé des facteurs expliquant l’apparition de
l’imprécision et de l’incertitude spatiale, comme Freksa et Barkowsky
(1996) ou Dutton (1992) qui identifient quelques facteurs
explicatifs. D’autres travaux (Hadzilacos, 1996 ; Evans et Waters,
2008) vont plus loin dans le détail en proposant une typologie plus
poussée de ces différents facteurs. Enfin, des typologies d’objets
spatiaux imprécis, comme celle proposée par Liu et al. (2019),
permettent d’identifier d’autres causes, inhérentes au processus de
construction des objets spatiaux.

Une première cause de l’imprécision spatiale, correspondant par
ailleurs à la majorité des exemples précédents, est liée à
l’imprécision de la définition. Par exemple, l’objet géographique
montagne, n’est pas (seulement) imprécis à cause d’une quelconque
difficulté technique limitant la précision des mesures, il l’est car
le concept montagne n’est pas suffisamment clair pour permettre la
délimitation précise d’une portion d’espace. C’est généralement
l’imprécision du concept qui rend l’objet géographique imprécis
(Freksa et Barkowsky, 1996). L’imprécision du concept est à distinguer
des définitions concurrentes, ce qu’Evans et Waters (2008) nomment
definitional disagreement. C’est, par exemple, le cas des frontières
contestées, nécessairement mutuellement exclusives ; qui, même si
définies aussi précisément que possible, ne permettent pas de
construire une frontière unique, sinon en admettant une part
d’incertitude spatiale. Par conséquent, le definitional disagreement,
et plus généralement, l’existence de géométries concurrentes pour un
même individu (Hadzilacos, 1996), sont une source d’incertitude,
inhérente au choix d’une possibilité parmi l’ensemble des possibles.

Cependant, l’imprécision d’une définition peut être souhaitée,
Hadzilacos (1996) parle alors de don't care [boundaries]. C’est, par
exemple, un cas que l’on retrouve fréquemment lors de la description
en langage naturel d’une position. Un exemple de Bateman et
al. (2010), illustre bien cette situation ; si pour décrire sa
position, une personne dit : « Je suis à la poste », on ne peut pas en
conclure qu’elle est située à l’intérieur d’une agence postale. En
effet, si la file d’attente sort du bâtiment, cette description sera
toujours valable. La limite de « la poste » est donc peu précise, mais
dans ce contexte, il n’est pas nécessaire qu’elle le soit
plus. L’information que locuteur cherche à communiquer est sa
proximité et son interaction avec une agence postale, et non sa
présence au sein du bâtiment.

La dimension temporelle peut également être une source
d’imprécision. Hadzilacos (1996) et Liu et al. (2019) mentionnent
respectivement l’existence de time-varying [boundaries] et de dynamic
boundary objects pour qualifier des objets géographiques dont la
frontière varie dans le temps. C’est, par exemple, le cas d’un front
de mer. Pour ce type d’objets, définir une frontière nécessite de
“synthétiser” les différentes évolutions temporelles, ce qui conduit
nécessairement à une frontière imprécise. Dans ce cas l’imprécision
spatiale est un artefact, né de la modélisation atemporelle d’un objet
qui ne l’est pas.

On peut également relever des aspects plus techniques, comme
l’imprécision liée aux instruments de mesure ou au producteur de
données (Follin et al., 2019) et plus généralement au processus de
production de données (Dutton, 1992 ; Evans et Waters, 2008 ; Follin
et al., 2019). D’autres points plus spécifiques peuvent également être
identifiés, comme les limites des modèles de représentation des
données (Dutton, 1992 ; Follin et al., 2019), il est par exemple
impossible de représenter tous les nombres réels informatiquement à
cause de la précision finie des nombres flottants utilisés pour les
figurer.

Enfin, il convient de noter que l’imprécision et l’incertitude peuvent
se transmettre lors de la définition de nouveaux objets à partir de
mesures (Dutton, 1992) ou d’objets géographiques imprécis (Liu et al.,
2019 ; Follin et al., 2019). On peut, dans ce cas, parler
d’imprécision et d’incertitude de second ordre. C’est ce phénomène que
décrivent Liu et al. (2019) lorsqu’ils définissent les
element-clustering objects, des objets spatiaux imprécis construits
par l’agrégation d’autres objets (flous ou nets), et les
object-referenced objects, qui sont construits par subdivision
d’objets spatiaux imprécis.









