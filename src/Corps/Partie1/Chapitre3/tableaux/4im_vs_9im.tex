\subfloat[Exemple de décomposition de la relation topologique \emph{overlap}]{
  \begin{tabular}{rC{5.5cm}C{5.5cm}}
    \toprule
    & \scriptsize \(\begin{bmatrix}
      V & V & F \\
      V & V & F \\
      V & F & F \\
    \end{bmatrix}\) & \scriptsize  \(\begin{bmatrix}
      V & V & V \\
      V & V & V \\
      V & V & V \\
    \end{bmatrix}\) \\
    \midrule
    \scriptsize \(\text{overlap} ≡ \begin{bmatrix}
      V & V \\
      V & V \\
    \end{bmatrix}\)
    & \tikz{
      \path[ffa] (0,0) circle [radius=15pt];
      \path[ffc] (0,0) circle [radius=15pt] node {A};
      \path[ffc, draw=RdBu-9-9] (0,15pt) to [out=90,in=0] (10pt,0) node[pos=0,blue]{} node[pos=1, blue]{};
      }
    & \tikz{
      \path[ffa] (0,0) circle [radius=15pt];
      \path[ffc] (0,0) circle [radius=15pt] node {A};
      } \\
    \bottomrule    
  \end{tabular}
  \label{tab:RCC8_vs_RCC23_1}
}

\subfloat[Exemple de décomposition de la relation topologique \emph{meet}]{
  \begin{tabular}{rC{5.5cm}C{5.5cm}}
    \toprule
    &  \scriptsize \(\begin{bmatrix}
      F & F & V \\
      F & V & F \\
      F & V & F \\
    \end{bmatrix}\) & \scriptsize  \(\begin{bmatrix}
      F & F & V \\
      F & V & V \\
      V & F & V \\
    \end{bmatrix}\) \\
    \midrule
    \scriptsize \(\text{meet} ≡ \begin{bmatrix}
      F & F \\
      F & V \\
    \end{bmatrix}\)
    & \tikz{\path[ffa] (0,0) circle [radius=15pt] node {A};}&
                                                              \tikz{\path[ffa] (0,0) circle [radius=15pt] node {B};} \\
    \bottomrule    
  \end{tabular}
  \label{tab:RCC8_vs_RCC23_2}
}

