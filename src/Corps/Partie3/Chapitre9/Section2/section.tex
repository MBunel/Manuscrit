La première des alertes que nous proposons de traiter est l'alerte
dite du \emph{Grand Veymont,} issue de l'enregistrement d'un appel au
secours effectué par un accompagnateur en montagne, au sujet d'un de
ses clients.
%
Cette alerte est la plus simple est la plus courte de celles que nous
traiterons ici.

\subsection{Présentation de l'alerte}
\label{subsec:9-2-1}

L'alerte du \emph{Grand Veymont} est l'extrait, d'une durée de 1
minute 06, d'un appel effectué par un accompagnateur en montagne au
sujet d'un de ses clients \autoref{anx:retrans-gv-verb}.


Cette alerte est la plus courte de celles que nous présenterons ici,
elle est composée de 12 extraits, pour un total de 16 expressions.



Pas d'objets multiples

les indications données par le requérant sont assez précises et
détaillées, il est donc possible de définir une \emph{zone initiale de
  recherche} de petite taille. Nous avons défini une \ac{zir} de
\SI{25}{\kilo\meter\squared} (\autoref{fig:zir_grand_veyont}).

\begin{figure}
  \centering
  \input{./figures/ZIR_grand_veymont.tex}
  \caption{Zone initiale de recherche pour l'alerte \enquote{Grand Veymont}}
  \label{fig:zir_grand_veyont}
\end{figure}


\subsubsection{Identification des indices de localisation}
\label{subsec:9-2-1-1}


Pour construire la zone de localisation probable de l'alerte il est
nécessaire d'identifier les différents indices de localisation
transmis durant la conversation téléphonique entre le requérant et le
secouriste et de définir la relation de localisation la plus adaptée à
la situation. Comme nous le signalions précédemment, ce travail est
prévu pour être réalisé parle secouriste durant l'alerte et non à
posteriori.

Pour identifier les relations de localisation utilisables pour
modéliser les différents indices de localisation données par le
requérant nous allons nous appuyer sur le verbatim de la discussion
téléphonique entre le requérant et le secouriste, ainsi que son
découpage en \emph{expressions,} réalisé durant la phase de
retranscription (\autoref{chap:05}). Le verbatim complet de l'alerte
est donné dans la \autoref{anx:retrans-gv-verb}. On pourra remarquer
que certains détails présents dans la retranscription en annexe (qui
fait office de référence) ne figurent pas ici, notamment les
hésitations ou les bafouillements. Les extraits que nous faisons
figurer ici ne correspondent donc pas exactement à la discussion
réelle, même si nous avons tout fait pour ne pas altérer le sens des
propos tenus.


Le requérant commence par décrire sa position en donnant plusieurs
éléments de contexte :
%
\begin{dialogue}
  % 
  \Req \lnex{1}{gv:1} J'ai eu du mal à descendre entre le sommet du
  \emph{Grand Veymont} et là où je suis. \lnex{2}{gv:2} La descente
  était très très lente.
\end{dialogue}
%
Ces deux premiers extraits ne peuvent pas être retranscrits à l'aide
d'une des relations spatiales que nous avons définies. En effet,
l'extrait \ref{gv:1} ne décrit pas la position actuelle du requérant,
mais la \emph{configuration} du chemin qu'il a emprunté pour atteindre
cette position. On pourrait supposer que la relation de localisation
\onto[orl]{Sous\-Jalon\-Sur\-Iti\-ne\-rai\-re} est pertinente dans ce
cas. Cependant l'utilisation de cette relation de localisation
nécessite que le site soit un jalon de l’itinéraire actuel du
requérant, or, rien ici ne permet d'affirmer que c'est toujours le
cas. De plus, nous ne savons pas si le requérant est toujours situé
sur un sentier, bien que cette hypothèse paraisse légitime, elle ne
doit pas être systématisée. Ce premier extrait ne nous permet donc pas
d'identifier un indice de localisation.

On peut cependant extrapoler certaines informations de ces deux
extraits. Par exemple, on peut légitimement considérer que le
requérant n'est plus sur le sommet du \emph{Grand Veymont,} ce que
l'on peut alors traduire par la relation de localisation
\onto[orl]{Hors\-De\-Planimétrique}, indiquant que la cible est à
l'extérieur du site. Dans le contexte de l'alerte, le site est le
sommet du \emph{Grand Veymont,} sont étendue est donc particulièrement
réduite. Par conséquent la \ac{zlc} de cet indice sera de grande
taille, ce qui rend cet indice de localisation peu discriminant.

L'extrait \ref{gv:2} ne donne quant à lui pas d'informations sur la
position actuelle du requérant. Il donne cependant une idée de la
vitesse de déplacement du requérant durant la descente et permet de
conclure que la distance parcourue est vraisemblablement inférieure à
celle qu'aurai parcouru le même groupe dans des conditions
normales. Néanmoins, comme notre travail ne porte pas sur l'analyse de
déplacements, nous ne pouvons exploiter cette information.

Les premiers indices donnant une information réellement discriminante
sont ceux que l'on peut construire à partir des extraits \ref{gv:3} et
\ref{gv:4}, plus riches :
%
\begin{dialogue}
  % 
  \Sec \lnex{3}{gv:3} Vous êtes entre \emph{Grand Veymont} et
  \emph{Pas de la Ville} ?

  \Req \lnex{4}{gv:4} Je suis entre le \emph{Grand Veymont,}
  \lnex{5}{gv:5} sous le \emph{Grand Veymont} et \emph{Pas de la
    Ville,} tout à fait, \lnex{6}{gv:6} coté sud.
\end{dialogue}
% 
Les extraits \ref{gv:3} et \ref{gv:4} indiquent que le requérant se
situe entre le sommet du \emph{Grand Veymont} et le \emph{Pas de la
  Ville,} ce que l'on peut modéliser avec la \emph{relation de
  localisation} \onto[orl]{Entre\-X\-et\-Y}.

Dans l'extrait \ref{gv:5} le requérant indique que sa position est
située en dessous du \emph{Grand Veymont.} Bien que ce ne soit pas
explicitement mentionné dans l'extrait, on comprend à l'aide du
contexte et de l'extrait \ref{gv:1} que le requérant se réfère ici au
\emph{sommet} du \emph{Grand Veymont} et non à la montagne dans son
ensemble. On peut donc ajouter un second indice de localisation,
utilisant une relation de localisation retranscrivant cette
description, comme \onto[orl]{Sous\-Altitude}. Nous avons cependant
défini plusieurs relations de localisation dérivées de cette dernière,
comme \onto[orl]{Sous\-Proche\-De}, permettant une spatialisation plus
fine. Cependant, aucune de ces relations ne nous semble adaptée à la
situation. La relation de localisation \onto[orl]{Sous\-Proche\-De},
contraint l'éloignement au site. Or, rien dans les extraits de cette
alerte ne permet d'affirmer que le requérant est proche du site, au
contraire, des extraits comme le \ref{gv:1} ou le \ref{gv:5} laissent
supposer que le requérant s'est déjà considérablement éloigné du
sommet du \emph{Grand Veymont.} De la même manière, la relation de
localisation \onto[orl]{Sous\-A\-L\-Aplomb\-De} ne nous semble pas
adaptée pour représenter la sémantique de cet extrait. En effet, cette
relation de localisation contraint la \ac{zlc} suivant l'axe de plus
grande pente (\autoref{anx:orl_dic}). Cette contrainte nous semble ici
trop restrictive, aucun élément de l'alerte ne permettant d'affirmer
qu'une telle relation existe ici. Quant à la relation de localisation
\onto[orl]{Sous\-Jalon\-Sur\-Itineraire}, cette dernière contraint la
relation de localisation \onto[orl]{Sous\-Altitude}, par la position
des itinéraires, or, comme nous l'avons déjà signalé il n'est pas
nécessairement pertinent de considérer que c'est le cas ici.

Dans l'extrait \ref{gv:6} le requérant complète cette information en
précisant sa position par une relation de localisation
cardinale. Cette description est cependant difficile à interpréter,
d'une part car le contexte ne permet pas de savoir si c'est le sommet
du \emph{Grand Veymont} ou le \emph{Pas de la Ville} qui fait office
de site, ce qui est relevé par le secouriste :
%
\begin{dialogue}
  %
  \Sec \lnex{7}{gv:7} Côté Sud du \emph{Pas de la Ville} ?
%
  \Req \lnex{8}{gv:8} Non, côté Nord.
\end{dialogue}
%
En précisant le site dans sa question (\ref{gv:7}) le secouriste nous
permet d'interpréter l'indice de localisation donné par le requérant
(\ref{gv:8}).
%
Dans ce contexte il est difficile de savoir si la contradiction ente
les extraits \ref{gv:6} et \ref{gv:8} est due à une erreur du
requérant, ou à une utilisation implicite du second site (le
\emph{Grand Veymont}). Toutefois, si l'on observe le cadre
géographique de l'alerte (\autoref{fig:zir_grand_veyont}) on peut
remarquer que seules trois configurations sont possibles :
%
\begin{enumerate*}[label=(\arabic*)]
\item le requérant est au nord du \emph{Grand Veymont} et du \emph{Pas
    de la Ville},
\item le requérant est au nord du \emph{Grand Veymont} et au sud du
  pas de la Ville, et
\item le requérant est au sud du \emph{Pas de la Ville} et du
  \emph{Grand Veymont.}
\end{enumerate*}
%

On peut déduire de ces extraits (\ref{gv:6}, \ref{gv:7}, \ref{gv:8})
que le requérant est situé au
%
On peut donc envisager d'utiliser la relation de localisation
\onto[orl]{Au\-Nord\-De} pour spatialiser cet indice de
localisation. Cependant, cette relation correspond au regroupement de
deux autres relations \onto[orl]{Dans\-La\-Partie\-Nord\-De} et
\onto[orl]{Au\-Nord\-De\-Externe}, dont l'usage peut être ici plus
pertinent.
%
La relation de localisation \onto[orl]{Dans\-La\-Partie\-Nord\-De}
décrit une situation où la cible est située à \emph{l'intérieur} du
site et dans sa partie nord.
%
Rien dans la situation décrite par le requérant n'indique qu'il serait
situé dans le site
%
La relation de localisation \onto[orl]{Au\-Nord\-De\-Externe} décrit
quant à elle une situation où la cible est au nord du site, sans être
ni contact, ni au sein de celui-ci.
%
Dans le contexte de cette alerte, l'utilisation de cette relation
indiquerait que le requérantxxx





\begin{dialogue}
  % 
  \Sec \lnex{9}{gv:9} Vous êtes au-delà du \emph{Pas de la Ville ?}
  \lnex{10}{gv:10} Entre \emph{Pas de la ville} et \emph{Pierre
    Blanche ?}
  % 
  \Req \lnex{11}{gv:11} Oui, je suis au-delà du \emph{Pas de la
    Ville.}
\end{dialogue}
%
Les questions posées par le secouriste dans les extraits \ref{gv:9} et
\ref{gv:10}
%
On ne peut, en effet, pas considérer que les prépositions
\enquote{entre} (\ref{gv:10}) et \enquote{au-delà} (\ref{gv:9}) sont
utilisées comme synonymes.
%
Pourtant, en confirmant qu'il est \enquote{au-delà} du \emph{Pas de la
  Ville} (\ref{gv:11}) le requérant contredit sa précédente
affirmation (\ref{gv:4}), comme quoi il serait situé entre le
\emph{Pas de la Ville} et le \emph{Grand Veymont.}

Compte-tenu du contexte de ces deux extraits dans l'alerte, il semble
probable qu'ils soient une rectification implicite de l'extrait
\ref{gv:4}, l'indice que nous avons précédemment déduit de cet extrait
nous semble donc caduc.

Viennent ensuite les extraits \ref{gv:12} et \ref{gv:13} :
%
\begin{dialogue}
  %
  \Req \lnex{12}{gv:12} Sur une zone à peu près plate et
  caillouteuse. \lnex{13}{gv:13} Sur une petite prairie.
\end{dialogue}
%
Ces deux extraits décrivent la nature du terrain où se situe
actuellement le requérant. Dans ce cas la relation de localisation la
plus adaptée est \onto[orl]{Dans\-Planimetrique}, qui est conçue pour
décrire des situations d'inclusions.


La dernière partie de l'alerte est composée des extraits \ref{gv:14}
et \ref{gv:15}, portant sur l'éloignement de la position du requérant
au \emph{Pas de la ville :}
%
\begin{dialogue}
  \Sec \lnex{14}{gv:14} Vous êtes à combien du Pas de la ville ?
  % 
  \Req \lnex{15}{gv:15} À 800 mètres, je crois, à vol d'oiseau.
\end{dialogue}
%
Comme précédemment, l'interprétation de ces deux extraits ne pose pas
de problèmes particuliers.


Dans cet extrait, le requérant précise clairement la nature de la
distance qu'il décrit, il s'agit d'une distance à vol d'oiseau,
visuellement approximée.

Ce type de configuration est représenté par la relation de
localisation
\onto[orl]{Distance\-Quanti\-ta\-ti\-ve\-Planimetrique}. Comme nous
l'indiquions dans le \autoref{chap:07}, cette relation de localisation
est atomique, elle n'admet donc pas de décomposition.

%
Au terme de cette phase de retranscription et d'identification des
indices de localisation on a donc un ensemble de XX indices de
localisation.
%
\begin{enumerate}
\item ss
\item XX
\item R, \onto[orl]{Distance\-Quantitative}(800m), \emph{Pas de la Ville}
\end{enumerate}


\subsection{Modélisation de l'alerte}
\label{subsec:9-2-2}




\subsubsection{Au nord du Pas de la Ville}

La relation de localisation \onto[orl]{Au\-Nord\-De\-Externe},
utilisée pour retranscrire la sémantique de l'extrait \nex{8} est
une relation de localisation décomposable.
%
Cette relation se décompose en deux relation de localisation
atomiques, \onto[orla]{Au\-Nord\-De} et
\onto[orla]{Hors\-De\-Planimétrique}. La relation de localisation
atomique \onto[orla]{Au\-Nord\-De} \footnote{Que la relation de
  localisation \protect\onto[orla]{Au\-Nord\-De\-Externe} partage avec
  sa relation sœur \protect\onto[orla]{Dans\-La\-Partie\-Nord\-De}.}


\begin{figure}
  \centering
  \input{./figures/EcartNord_PasVille.tex}
  \caption{Métrique \protect\onto[orla]{Ecart\-Angulaire}, calculée
    pour la spatialisation de la relation de localisation
    \protect\onto[orl]{AuNordDe}. La résolution du raster a
    été réduite de 5 à 50 mètres pour la représentation.}
  \label{fig:veyont_EcartNord}
\end{figure}


% \begin{figure}
%   \centering
%   \begin{tikzpicture}[scale=.7]
  \def\decalageX{-.2}
  \def\decalageY{-.2}
  % Courbe
  \begin{scope}[transparency group]
    % fond
    \begin{scope}
      \path[ffa_fade_m] (0,.8) -- (1,.8) -- (1,0) -- (0,0) -- cycle ;
      \path[ffa_fade] (8,.8) -- (9,.8) -- (9,0) -- (8,0) -- cycle;
      \path[ffa]  (1,.8) -- (3.6,.8) -- (4.5, 2) -- (5.4,.8) --(8,.8)
      --(8,0) --(1,0) -- cycle;
    \end{scope}
    % bords
    \begin{scope}
      \path[ffc, dotted] (3.6,.8) -- (3,0);
      \path[ffc, dotted] (5.4,.8) -- (6,0);
      % 
      \path[ffc] (1,.8) -- (3.6,.8) -- (4.5, 2) -- (5.4,.8) -- (8,.8) ;
      \path[ffc_fade_m] (0,.8) -- (1,.8) ;
      \path[ffc_fade] (8,.8) -- (9,.8) ;
    \end{scope}
  \end{scope}
  % Axes X, Y
  \begin{scope}
    % Axe X
    \begin{scope}
      % Axe
      \draw[<->] (0, \decalageX) --++ (9, 0) coordinate (x axis);
      % Graduations
      \foreach \n/\t in {0.5/{},1.5/{},2.5/{400},3.5/{},4.5/{800},5.5/{},6.5/{1200},7.5/{},8.5/{}}
      {
        \draw[-] (\n, \decalageX - .05) --++ (0, .1);
        \node[below, font=\footnotesize] at (\n, \decalageX - .05) {\t};
      }
      % label
      \node[below, yshift=-.1cm, font=\small] at (x axis)
      {\itshape Distance \normalfont (m)};
    \end{scope}
    % Axe Y
    \begin{scope}
      % Axe
      \draw[-] (\decalageY ,0) --++ (0, 2) coordinate (y axis);
      % Graduations
      \foreach \n/\t in {0/{0},2/{1}}
      {
        \draw[-] (\decalageY -.05, \n) --++ (.1, 0);
        \node[left, font=\footnotesize] at (\decalageY -.05, \n) {\t};
      }
      % Label
      \node[above] at (y axis) {$\mu$};
    \end{scope}
  \end{scope}
  \begin{scope}
    % Seuil 1
    \draw[fill, RdBu-9-1] (3,\decalageY) circle (2pt);
    % Seuil 2
    \draw[ffc,line width=.5] (4.5,\decalageY) -- (4.5,2);
    \draw[fill, RdBu-9-1] (4.5,\decalageY) circle (2pt);
    \draw[fill, RdBu-9-1] (4.5,2) circle (2pt);
    % Seuil 3
    \draw[fill, RdBu-9-1] (6,\decalageY) circle (2pt);
  \end{scope}
\end{tikzpicture}

%   \caption{XXXX \enquote{Grand Veymont}}
%   \label{fig:fuzzy_veyont_distance}
% \end{figure}



\subsubsection{Au-delà du Pas de la Ville}

La spatialisation de l'indice de localisation \enquote{je suis au-delà
  du Pas de la Ville}, pose certaines difficultés.

La relation de localisation de cet indice est représentée par le
concept \onto[orl]{Après\-Jalon\-Sut\-Itineraire}
(\autoref{anx:orl_dic})

\subsubsection{Sur une zone plate et caillouteuses}

XXX

\subsubsection{À 800 mètres du Pas de la Ville}

Ce dernier indice de localisation ne présente pas de difficultés
spécifiques pour être mis en place.


Cette relation est spatialisée à l'aide du \emph{rasteriser}
\onto[orla]{Geometrie}, de la \emph{métrique}
\onto[orla]{Dis\-tan\-ce} et du \emph{fuzzyficateur}
\onto[orla]{Eq\-Val}.

Dans le cas présent, \emph{l'objet de référence} mentionné est le
\emph{Pas de la Ville,} représenté par un ponctuel dans la composante
oronymie de la BDTOPO.


La rasterisation de ce



La \autoref{fig:veyont_distance} représente le résultat du calcul de
la métrique \onto[orla]{Distance} ---~calculée à partir du ponctuel
(rasterisé) représentant le \emph{Pas de la Ville}~--- pour l'ensemble
des positions de la \ac{zir}. Si cette métrique ne présente pas de
caractéristiques particulièrement surprenantes, on peut quand même
noter l'approximation qui est ici faite. Le \emph{Pas de la Ville} est
en effet résumé par un point, alors que le toponyme désigne un
passage, une zone de transition, qui serait sans doute mieux
représentée par un polygone, voire une polyligne. De plus, le point
utilisé n'est pas placé au niveau du point central du Pas de la Ville,
mais à plusieurs centaines de mètres de ce dernier.


\begin{figure}
  \centering
  \begin{tikzpicture}
  \tikzset{et/.style={above,font=\footnotesize\vphantom{Ag}}}
  % 
  \node[inner sep=0pt, anchor=south west] (image) at (0,0){\includegraphics{./figures/Distance_PasVille.png}};
  % 
  \begin{scope}
    \node (P2) at ([yshift=-.5cm]image.south east) {};
    \node (P1) at ([yshift=-.5cm]image.south west) {};
    % 
    \node (rect) [anchor=north west, minimum width=1cm,minimum
    height=.25cm] at ([yshift=-.25cm]P1) {}; \path[draw=RdBu-9-1, line
    width=1mm](rect.west) --([xshift=-1ex]rect.south) -- ([xshift=1ex]rect.north)
    -- (rect.east);
    % 
    \node[anchor=west, font=\tiny\vphantom{Ag}, text width = 4cm] at
    ([xshift=1ex]rect.east) {Limite de la \ac{zir}};
    %
    \node[anchor=west, font=\tiny\vphantom{Ag}, text width = 4cm] at
    (P1 |- 0cm,-1.5cm) {Distance au \emph{Pas de la Ville :}};
    %
    \foreach \x [evaluate=\xshift using \x/10, evaluate=\rad using (\x * .0004) + .01] in {0,...,100}
    {
      \draw[fill=black,draw=none, below] ([xshift=\xshift cm, yshift=-1.75cm]P1) circle [radius=\rad cm];
    }
    % 
    \path(P1 |- 0cm,-2.25cm) --++ (10,0)
    node[et,pos=0] {0}
    node[et,pos=.5] {\SI{2500}{\meter}}
    node[et,pos=1] {\SI{5000}{\meter}};

    % Échelle
    \draw[-] (P2 |- -1cm,-1cm) --++ (-1,0) node[et,pos=.5] {\SI{500}{\meter}};
    % Légende détaillée
    \path (P1) -- (P2) node[pos=.5, yshift=-2.5cm] {\tiny Pour la légende détaillée du fond topographique voir \autoref{anx:topo_leg}. Sources: BD TOPO 2018, BD ALTI 2018.}; 
  \end{scope}
\end{tikzpicture}
  \caption{Métrique \protect\onto[orla]{Distance}, calculée pour la
    spatialisation de la relation de localisation atomique
    \protect\onto[orla]{Distance\-Pla\-ni\-mé\-trique}. La résolution
    du raster a été réduite de 5 à 50 mètres pour la représentation.}
  \label{fig:veyont_distance}
\end{figure}

Pour construire la \ac{zlc} spatialisant cet indice de localisation il
est ensuite nécessaire de \emph{fuzzyfier} la métrique
(\autoref{fig:veyont_distance}) à l'aide du fuzzyficateur
\onto[orla]{Eq\-Val}.

\begin{figure}
  \centering
  \begin{tikzpicture}[scale=.7]
  \def\decalageX{-.2}
  \def\decalageY{-.2}
  % Courbe
  \begin{scope}[transparency group]
    % fond
    \begin{scope}
      \path[ffa_fade_m] (0,.8) -- (1,.8) -- (1,0) -- (0,0) -- cycle ;
      \path[ffa_fade] (8,.8) -- (9,.8) -- (9,0) -- (8,0) -- cycle;
      \path[ffa]  (1,.8) -- (3.6,.8) -- (4.5, 2) -- (5.4,.8) --(8,.8)
      --(8,0) --(1,0) -- cycle;
    \end{scope}
    % bords
    \begin{scope}
      \path[ffc, dotted] (3.6,.8) -- (3,0);
      \path[ffc, dotted] (5.4,.8) -- (6,0);
      % 
      \path[ffc] (1,.8) -- (3.6,.8) -- (4.5, 2) -- (5.4,.8) -- (8,.8) ;
      \path[ffc_fade_m] (0,.8) -- (1,.8) ;
      \path[ffc_fade] (8,.8) -- (9,.8) ;
    \end{scope}
  \end{scope}
  % Axes X, Y
  \begin{scope}
    % Axe X
    \begin{scope}
      % Axe
      \draw[<->] (0, \decalageX) --++ (9, 0) coordinate (x axis);
      % Graduations
      \foreach \n/\t in {0.5/{},1.5/{},2.5/{400},3.5/{},4.5/{800},5.5/{},6.5/{1200},7.5/{},8.5/{}}
      {
        \draw[-] (\n, \decalageX - .05) --++ (0, .1);
        \node[below, font=\footnotesize] at (\n, \decalageX - .05) {\t};
      }
      % label
      \node[below, yshift=-.1cm, font=\small] at (x axis)
      {\itshape Distance \normalfont (m)};
    \end{scope}
    % Axe Y
    \begin{scope}
      % Axe
      \draw[-] (\decalageY ,0) --++ (0, 2) coordinate (y axis);
      % Graduations
      \foreach \n/\t in {0/{0},2/{1}}
      {
        \draw[-] (\decalageY -.05, \n) --++ (.1, 0);
        \node[left, font=\footnotesize] at (\decalageY -.05, \n) {\t};
      }
      % Label
      \node[above] at (y axis) {$\mu$};
    \end{scope}
  \end{scope}
  \begin{scope}
    % Seuil 1
    \draw[fill, RdBu-9-1] (3,\decalageY) circle (2pt);
    % Seuil 2
    \draw[ffc,line width=.5] (4.5,\decalageY) -- (4.5,2);
    \draw[fill, RdBu-9-1] (4.5,\decalageY) circle (2pt);
    \draw[fill, RdBu-9-1] (4.5,2) circle (2pt);
    % Seuil 3
    \draw[fill, RdBu-9-1] (6,\decalageY) circle (2pt);
  \end{scope}
\end{tikzpicture}

  \caption{XXXX \enquote{Grand Veymont}}
  \label{fig:fuzzy_veyont_distance}
\end{figure}



\begin{figure}
  \centering
  \input{./figures/Distance_GrandVeymont.tex}
  \caption{XXXX \enquote{Grand Veymont}}
  \label{fig:Distance_GrandVeymont}
\end{figure}




\subsubsection{Fusion des zones de localisation compatibles}

À la suite de la spatialisation des différents indices de localisation
et de la composition des 

\subsection{Critique de la modélisation}
\label{subsec:9-2-3}

\tdi{Impact de la position du toponyme "pas de la ville" sur la
  modélisation}

Un autre problème notable est l'impact que peut avoir la géométrie des
objets de référence sur le résultat de la spatialisation. Par exemple,
l'indice de localisation : \enquote{À \SI{800}{\meter} du Pas de la
  Ville} a été spatialisé en calculant la métrique
\onto[orla]{Distance} à partir du ponctuel plaçant l'oronyme
\enquote{Pas de la Ville} dans la base de données. Or, la
représentation d'un géotype étendu, comme un pas ou une crête, par un
point est une approximation conséquente, impactant fortement la
spatialisation.


%%% Local Variables:
%%% mode: latex
%%% TeX-master: "../../../../main"
%%% End:
