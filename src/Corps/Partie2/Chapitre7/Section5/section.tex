Dans le \autoref{chap:05}, nous avions abordé la question de la
représentation des règles de décomposition des \emph{relations de
  localisation} et avions décidé de les représenter dans une ontologie
\emph{ad hoc,} \ac{orla}. Cette solution présentait le double avantage
de proposer une formalisation des règles qui soit lisible par un être
humain, tout en les centralisant, ce qui, à terme favorise leur
compréhension, leur maintient et leur évolution. Une question
similaire se pose à présent, celle de la représentation des règles de
\emph{spatialisation} des \emph{relations de localisation atomiques.}
Il est à nouveau possible de représenter ces règles dans une
ontologie, nous avons donc décidé de les intégrer à \ac{orla}. Pour
permettre une bonne représentation des règles de
\emph{spatialisation,} il est nécessaire de spécifier quelques
contraintes, comme nous l'avions fait dans la
\autoref{chap:05-sec1}. Tout d'abord, toute \emph{relation de
  localisation atomique} doit, pour être spatialisée, être liée à un
\emph{rasteriser,} une \emph{métrique,} un \emph{fuzzyficateur} et,
optionnellement à un \emph{modifieur.} Comme pour les \emph{relations
  de localisation} nous ne permettons pas l’existence de règles
concurrentes, une \emph{relation de localisation atomique} donnée
ne correspond qu'à une \emph{méthode de spatialisation} et inversement
\footnote{Conformément à la règle d'unicité des \emph{relations de
  localisations atomiques} formulée dans le \autoref{chap:05}.}

Pour représenter les règles de spatialisation dans \emph{l'ontologie
  des relations de localisation atomiques} nous avons défini quatre
nouvelles relations \onto[orla]{has\-Rasteriser},
\onto[orla]{has\-Metrique}, \onto[orla]{has\-Fuzzyfier} et
\onto[orla]{has\-Modifieur}. Chacune de ces nouvelles relations lie
une \emph{relation de localisation atomique} à une classe
représentant, respectivement, une méthode de \emph{rastérisation,} une
méthode de calcul d'une \emph{métrique,} une méthode de
\emph{fuzzyfication} et une méthode appliquant un \emph{modifieur}
(\autoref{fig:structure_spatialisation_ontologie}). Ainsi, les
\emph{relations de localisation atomiques} ne sont pas seulement
définissables comme des \emph{relations de localisation} desquelles ne
proviennent aucune relation
\onto[orla]{has\-Relation\-Spatiale\-Atomique} \footnote{Comme nous
  l'indiquions dans le \autoref{chap:05}.}, mais également comme des
\emph{relations de localisation} auxquelles sont associés un
\emph{rastériser,} une \emph{métrique} et un \emph{fuzzyfier} (\ie à
l'origine d'une relation \onto[orla]{has\-Rasteriser},
\onto[orla]{has\-Metrique} et \onto[orla]{has\-Fuzzyfier}). La
\emph{spatialisation} d'une \emph{relation de localisation atomique}
peut alors être simplement effectuée en calculant, successivement, la
\emph{rastérisation,} la \emph{métrique,} sa \emph{fuzzyfication} et,
le cas échant, le \emph{modifieur,} définis dans \ac{orla} pour cette
\emph{relation.}

% \begin{figure}
%   \centering
%   % Ajout liens modifieurs
%   \begin{tikzpicture}
  \usetikzlibrary{shapes.geometric,fit}

  \begin{scope}[xshift=-3.5cm]
    \node (x1) at (0,0.8) {$rla_1$};
    \node (x2) at (0,0) {$rla_2$};
    \node (x3) at (0,-0.8) {$rla_3$};
  \end{scope}
  
    \begin{scope}
      \node (y1) at (3.5,.8) {$rla_1$};
      \node (y2) at (3.5,0) {$rla_2$};
      \node (y3) at (3.5,-0.8) {$rla_3$};
    \end{scope}
  
    \fill[RdBu-9-9] (x1.east) circle (2pt);
    \fill[RdBu-9-9] (x2.east) circle (2pt);
    \fill[RdBu-9-9] (x3.east) circle (2pt);
    
    \fill[RdBu-9-1] (y1.west) circle (2pt);
    \fill[RdBu-9-1] (y2.west) circle (2pt);
    \fill[RdBu-9-1] (y3.west) circle (2pt);

     \node[fit=(x1.east) (x3.west), ellipse, draw, minimum width=2cm,
    ffc2] (c1) {};
    \node[fit=(y1.east) (y3.west), ellipse, draw, minimum width=2cm,
    ffc] (c2) {};

    % draw the arrows
    \draw[->, shorten >=5pt, shorten <=5pt] (x1.east) -- (y1.west)
    node[pos=0.75, sloped, anchor=center, fill=white] {\tiny\emph{se
        décompose en}};
    
    \draw[->, shorten >=5pt, shorten <=5pt] (x1.east) -- (y2.west)
    node[pos=0.5, sloped, anchor=center, fill=white] {\tiny\emph{se
        décompose en}};
    
    \draw[->, shorten >=5pt, shorten <=5pt] (x2.east) -- (y3.west)
    node[pos=0.5, sloped, anchor=center, fill=white] {\tiny\emph{se
        décompose en}};

    \draw[->, shorten >=5pt, shorten <=5pt] (x3.east) -- (y3.west)
    node[pos=0.5, sloped, anchor=center, fill=white] {\tiny\emph{se
        décompose en}};

    \draw[->, shorten >=5pt, shorten <=5pt] (x2.east) to[out=30, in=-30, looseness=1.5] (x1.east);
    

    \node[text width=3.5cm, anchor=center,align=center] at
    ([yshift=-.5cm]c1.south) {\small Ensemble des \emph{relations de
        localisation}}; \node[text width=4cm,
    anchor=center,align=center] at ([yshift=-.5cm]c2.south) {\small
      Ensemble des \emph{Relations de localisation atomiques}};
\end{tikzpicture}
%   \caption{xxx}
%   \label{fig:structure_spatialisation_ontologie}
% \end{figure}

Par exemple, la \emph{relation de localisation atomique}
\onto[orla]{Sous\-Al\-ti\-tu\-de} ---~notamment utilisée pour définir
la \emph{relation} \onto[orla]{Sous\-Pro\-che\-De}
(\autoref{chap:05})~--- défini une \emph{zone de localisation
  compatible} dont toutes les positions ont une altitude inférieure à
celle de \emph{l'objet de référence}
(\autoref{fig:ex_parties_statialisation_sousalt}). Pour
\emph{spatialiser} cette \emph{relation de localisation atomique} il
faut tout d'abord définir un \emph{rasteriser.} Ici il n'y a pas de
raisons de se focaliser sur une partie spécifique de \emph{l'objet de
  référence,} la position est sous l'objet dans son ensemble et non
une de ces parties. Le \emph{rasteriser} le plus adapté est donc
\onto[orla]{Geometrie}. Le choix de la \emph{métrique} est un peu plus
difficile. On peut, en effet, se questionner sur l'altitude à
considérer pour \emph{l'objet de référence,} doit-on considérer le
point le plus haut, le plus bas, le plus proche ? Ou bien doit-on
utiliser un indicateur de centralité, comme la moyenne ? Tous ces
critères paraissent pertinents, mais l'altitude de la position la plus
proche nous semble être la solution la plus générique, car la moins
sensible aux fortes variations d'altitudes que l'on pourrait retrouver
sur des objets très étendus (\eg vallées, forêts, \emph{etc.}). Ainsi
nous avons opté pour la \emph{métrique}
\onto[orla]{Delta\-Nearest\-Val}. Le choix du \emph{fuzzyfieur} ne
pose, quant à lui, pas de problèmes particuliers. En effet, la
\emph{métrique} \onto[orla]{Delta\-Nearest\-Val} calcule, pour chaque
pixel de la \ac{zir}, la différence entre l'altitude locale et
l'altitude du pixel de \emph{l'objet de référence} le plus
proche. Ainsi, les pixels dont l'altitude est supérieure à celle du
point le plus proche sur \emph{l'objet de référence,} auront une
valeur de \emph{métrique} positive et les pixels dont l'altitude est
inférieure à celle du point le plus proche sur \emph{l'objet de
  référence} ---~que l'on souhaite conserver~--- auront une valeur de
\emph{métrique} négative. On peut donc utiliser le
\emph{fuzzyficateur} \onto[orla]{Inf\-Zero}, spécialement adapté à
cette configuration.

\begin{figure}
  \centering
  \begin{tikzpicture} 
  \node[anchor=east] (x1) at (-3.5,0) {\onto[orla]{Sous\-Altitude}};

  \node[anchor=west] (y1) at (3.5,.8) {\onto[orla]{Geometrie}};
  \node[anchor=west] (y2) at (3.5,0) {\onto[orla]{DeltaNearestVal}};
  \node[anchor=west] (y3) at (3.5,-0.8) {\onto[orla]{Inf\-Val\-0}};
  
  \fill[RdBu-9-9] (x1.east) circle (2pt);
  
  \fill[RdBu-9-1] (y1.west) circle (2pt);
  \fill[RdBu-9-1] (y2.west) circle (2pt);
  \fill[RdBu-9-1] (y3.west) circle (2pt);

  \draw[->, shorten >=5pt, shorten <=5pt] (x1.east) |- (y1.west)
  node[pos=0.75, sloped, anchor=center, fill=white] {\tiny\emph{A pour rasteriser}};
  
  \draw[->, shorten >=5pt, shorten <=5pt] (x1.east) -- (y2.west)
  node[pos=0.5, sloped, anchor=center, fill=white] {\tiny\emph{A pour métrique}};

    \draw[->, shorten >=5pt, shorten <=5pt] (x1.east) |- (y3.west)
  node[pos=0.75, sloped, anchor=center, fill=white] {\tiny\emph{A pour fuzzyficateur}};
\end{tikzpicture}
  \caption{\emph{Rasteriser,} \emph{métrique} et \emph{fuzzyficateur}
    utilisés pour la \emph{spatialisation} de la \emph{relation de
      localisation atomique}
    \protect\onto[orla]{Sous\-Al\-ti\-tu\-de}.}
  \label{fig:ex_parties_statialisation_sousalt}
\end{figure}

On peut définir la \emph{relation de localisation atomique}
\onto[orla]{Dans\-Pla\-ni\-mé\-tri\-que} en reprenant presque tous les
éléments utilisés pour la \emph{spatialisation} de
\onto[orla]{Sous\-Al\-ti\-tu\-de}
(\autoref{fig:ex_parties_statialisation_dansplani}). Tout d'abord,
cette \emph{relation} porte sur \emph{l'objet de référence} dans son
ensemble, il est donc nécessaire d'employer à nouveau le
\emph{rasteziser} \onto[orla]{Geometrie}. Le choix de la
\emph{métrique} nécessite plus de questionnements.
%
Dans l'exemple de la \autoref{fig:Exemple_Metrique_vs_Selecteur}, nous
avions utilisé les matrices 4IM pour modéliser une relation
topologique. Mais cet exemple ignorait la question de
\emph{l'imprécision}. Pour ajouter la prise en compte de
\emph{l'imprécision} il est nécessaire d'utiliser une \emph{métrique}
continue, exprimant la proximité. Nous avons donc choisi d'utiliser la
\emph{métrique} \onto[orla]{Distance}, qui calcule la distance entre
le pixel traité et le point le plus proche de \emph{l'objet de
  référence.} Si cette distance est nulle, alors cela signifie que le
pixel est en contact avec \emph{l'objet de référence.} On peut alors
attribuer un \emph{degré d'appartenance} non nul aux pixels situés à
faible distance de \emph{l'objet de référence.} Ainsi des pixels qui
seraient considérés comme en dehors de \emph{l'objet de référence}
avec une modélisation topologique sont ici pris en compte, mais avec
une pénalisation de leur degré d'appartenance. Étant donné que la
\emph{métrique} \onto[orla]{Distance} ne peut prendre de valeur
négative, deux \emph{fuzzyficateurs,} \onto[orla]{Eq\-Val\-0} et
\onto[orla]{Inf\-Val\-0}, sont ici utilisables. Ces
\emph{fuzzyficateurs} ont cependant une sémantique différente, le
premier signifiant que la \emph{métrique} doit être proche d'une
valeur seuil, là où le second impose que la métrique ait une valeur
inférieure ou égale à ce seuil. Or, dans le cadre, hypothétique, où la
\emph{métrique} \onto[orla]{Distance} prendrait des valeurs négatives
(\ie traduisant une distance à l'intérieur de \emph{l'objet de
  référence}) on voudrait leur attribuer un degré d'appartenance
maximal. Ainsi, bien que cela ne change rien dans les faits, il nous
semble plus correct d'utiliser le \emph{fuzzyficateur}
\onto[orla]{Inf\-0} pour traduire la sémantique de la \emph{relation
  de localisation atomique} \onto[orla]{Dans\-Pla\-ni\-mé\-tri\-que}.

\begin{figure}
  \centering
  \begin{tikzpicture} 
  \node[anchor=east] (x1) at (-3.5,0) {\onto[orla]{DansPlanimétrique}};

  \node[anchor=west] (y1) at (3.5,.8) {\onto[orla]{Geometrie}};
  \node[anchor=west] (y2) at (3.5,0) {\onto[orla]{Distance}};
  \node[anchor=west] (y3) at (3.5,-0.8) {\onto[orla]{Inf\-0}};
  
  \fill[RdBu-9-9] (x1.east) circle (2pt);
  
  \fill[RdBu-9-1] (y1.west) circle (2pt);
  \fill[RdBu-9-1] (y2.west) circle (2pt);
  \fill[RdBu-9-1] (y3.west) circle (2pt);

  \draw[->, shorten >=5pt, shorten <=5pt] (x1.east) |- (y1.west)
  node[pos=0.75, sloped, anchor=center, fill=white] {\tiny\emph{A pour rasteriser}};
  
  \draw[->, shorten >=5pt, shorten <=5pt] (x1.east) -- (y2.west)
  node[pos=0.5, sloped, anchor=center, fill=white] {\tiny\emph{A pour métrique}};

    \draw[->, shorten >=5pt, shorten <=5pt] (x1.east) |- (y3.west)
  node[pos=0.75, sloped, anchor=center, fill=white] {\tiny\emph{A pour fuzzyficateur}};
\end{tikzpicture}
  \caption{\emph{Rasteriser,} \emph{métrique} et \emph{fuzzyficateur}
    utilisés pour la \emph{spatialisation} de la \emph{relation de
      localisation atomique}
    \protect\onto[orla]{Dans\-Pla\-ni\-mé\-tri\-que}.}
  \label{fig:ex_parties_statialisation_dansplani}
\end{figure}

La \emph{relation de localisation atomique}
\onto[orla]{Not\-A\-La\-Fron\-ti\-ere\-De} différè des exemples
précédents par sa formulation négative, qui l'oppose à la
\emph{relation} \onto[orla]{A\-La\-Fron\-ti\-ere\-De}. Ces deux
\emph{relations de localisation atomiques} partagent leur
\emph{rasteriser,} leur métrique et leur \emph{fuzzyficateur,} mais
elles se distinguent par l'utilisation du \emph{modifieur}
\onto[orla]{Not}
(\autoref{fig:ex_parties_statialisation_notalafrontierede}). En effet,
ces deux \emph{relations} traduisent une proximité à la frontière de
\emph{l'objet de référence,} on peut donc utiliser la même
\emph{métrique} que pour la \emph{relation de localisation atomique}
\protect\onto[orla]{Dans\-Pla\-ni\-mé\-tri\-que}, la distance
planaire, représentée par la classe \onto[orla]{Distance}. Cependant
cette distance ne peut être calculée à partir de \emph{l'objet de
  référence} dans son ensemble puisque ces \emph{relations} décrivent
une position exprimée par rapport à la frontière de \emph{l'objet de
  référence,} il est donc nécessaire d'employer le \emph{rasteriser}
\onto[orla]{Frontiere}, déjà présenté. La \emph{métrique} ne
correspond donc plus à la distance à \emph{l'objet de référence,} mais
à la distance à sa frontière. Les positions situées à l'intérieur de
l'objet se voient donc attribuer une distance non nulle. Ces deux
\emph{relations} se distinguent donc par leur \emph{métrique.} Pour la
\emph{relation} \onto[orla]{A\-La\-Fron\-ti\-ere\-De} on souhaite que
le degré d'appartenance soit maximal à la frontière ---~lorsque la
distance est nulle~--- et qu'il diminue avec l'éloignement, ce qui
correspond au \emph{fuzzyficateur} \onto[orla]{Eq\-Val\-0}. Pour la
\emph{relation} \onto[orla]{Not\-A\-La\-Fron\-ti\-ere\-De}, c'est le
comportement opposé qui est attendu. On peut donc utiliser le même
\emph{fuzzyficateur} mais en y adjoignant le \emph{modifieur}
\onto[orla]{Not}, qui a pour effet de transformer la \emph{fonction
  d'appartenance} en son opposée (Figure \ref{fig:fnc_not_eq_val}).

\begin{figure}
  \centering  \begin{tikzpicture} 
  \node[anchor=east] (x1) at (-3.5,0) {\onto[orla]{NotALaFrontiereDe}};

  \node[anchor=west] (y1) at (3.5,1.2) {\onto[orla]{Frontiere}};
  \node[anchor=west] (y2) at (3.5,0.4) {\onto[orla]{Distance}};
  \node[anchor=west] (y3) at (3.5,-0.4) {\onto[orla]{Eq\-Val\-0}};
  \node[anchor=west] (y4) at (3.5,-1.2) {\onto[orla]{Not}};
  
  \fill[RdBu-9-9] (x1.east) circle (2pt);
  
  \fill[RdBu-9-1] (y1.west) circle (2pt);
  \fill[RdBu-9-1] (y2.west) circle (2pt);
  \fill[RdBu-9-1] (y3.west) circle (2pt);
  \fill[RdBu-9-1] (y4.west) circle (2pt);

  \draw[->, shorten >=5pt, shorten <=5pt] (x1.east) |- (y1.west)
  node[pos=0.75, sloped, anchor=center, fill=white] {\tiny\emph{À pour rasteriser}};
  
  \draw[->, shorten >=5pt, shorten <=5pt] (x1.east) |- (y2.west)
  node[pos=0.75, sloped, anchor=center, fill=white] {\tiny\emph{À pour métrique}};

  \draw[->, shorten >=5pt, shorten <=5pt] (x1.east) |- (y3.west)
  node[pos=0.75, sloped, anchor=center, fill=white] {\tiny\emph{À pour
      fuzzyficateur}};
  \draw[->, shorten >=5pt, shorten <=5pt] (x1.east) |- (y4.west)
  node[pos=0.75, sloped, anchor=center, fill=white] {\tiny\emph{À pour Modifieur}};
\end{tikzpicture}
  \caption{\emph{Rasteriser,} \emph{métrique} et \emph{fuzzyficateur}
    utilisés pour la \emph{spatialisation} de la \emph{relation de
      localisation atomique}
    \protect\onto[orla]{Not\-A\-La\-Fron\-ti\-ere\-De}.}
  \label{fig:ex_parties_statialisation_notalafrontierede}
\end{figure}

%%% Local Variables:
%%% mode: latex
%%% TeX-master: "../../../../main"
%%% End:
