Pour formaliser les décompositions des \emph{relations de
  localisations} nous avons opté pour la construction d'une
\emph{ontologie} des \emph{relations de localisation atomiques}
contenant l'ensemble des \emph{relations de localisation} identifiées
comme pertinentes et leur(s) décomposition(s).

\tdi{Ajouter ontologie des objets de référence}

Pour identifier les \emph{relations de localisations} pertinentes pour
notre contexte applicatif nous avons entrepris de recenser celles qui
sont les plus utilisées lors des alertes. Pour ce faire nous avons
retranscrit et analysé différents enregistrements d'alertes passées.

\subsection{Retranscriptions des alertes}

Le travail de retranscription des alertes (et dans une moindre mesure
celui de leur analyse) a été réalisé collaborativement
\autocite{Bunel2019}, les résultats pouvant bénéficier à différents
objectifs du projet Chouacs.

\subsubsection{Le corpus des alertes}

Les retranscriptions que nous avons effectuées pour définir
\emph{l'ontologie des relations de localisation atomiques} ont été
réalisées à partir d'un ensemble de 52 enregistrements audio qui nous
ont été transmis par le \ac{pghm} de Grenoble. Ces fichiers ont été
produits par les \ac{pghm} de Grenoble (5 alertes) et de Chamonix (47
alertes) à l'aide d'un enregistreur numérique. Comme les requérants
contactent généralement les \ac{codis} (cf. \autoref{chap:02}) les
enregistrements ne contient pas nécessairement toute la discussion
entre le requérant et les services de secours, les appels ne sont
enregistrés que lorsque un secouriste du \ac{pghm} intègre la
discussion. De plus les enregistrements qui nous ont été fournis ont
été modifiés, de sorte à censurer les informations soumises au secret
médical.

Ces dernières nous ont été transmises en deux corpus, un premier de 20
alertes en 2017 et un second de 32 alertes en 2019.

dont la durée varie de quelques secondes à plusieurs dizaines de
minutes (\autoref{fig:dist_temps_alertes}).

\begin{figure}
  \centering
   \begin{tikzpicture}
   \begin{axis}
     \addplot [boxplot] table [y index=0] {./Corps/Partie2/Chapitre5/figures/duration.dat};
     
  %\addplot [thick] gnuplot [raw gnuplot] {plot './Corps/Partie2/Chapitre5/figures/duration.dat' smooth kdensity};
\end{axis}
\end{tikzpicture}

  \caption{Distribution de la durée des alertes}
  \label{fig:dist_temps_alertes}
\end{figure}

\subsubsection{Template de retranscription}

Pour exploiter les informations contenues dans les enregistrements il
a été nécessaire de les retranscrire afin de les analyser.
%
Notre parti-pris n'a pas été de travailler à partir d'une
retranscription libre de ces alertes, mais à l'aide d'un tableau
prédéfini, un \emph{template de retranscription}
(\autoref{tab:struct_temp}) dont la structure reprend celle des
\emph{indices de localisation} telle que nous l'avons définie
(cf. \autoref{chap:04}).

\begin{table}
  \centering
  \begin{tabular}{L{.3\textwidth}>{\footnotesize}p{.4\textwidth}}
  \toprule \multicolumn{1}{c}{\bfseries Colonne} &
  \multicolumn{1}{c}{\normalsize \bfseries Contenu} \\ \midrule
% Sémantique des relations spatiales
  \addlinespace
  Identifiant de l'extrait & Permet d'identifier une expression\\
  Identifiant de l'expression & Permet d'identifier une expression au
                                sein d'un extrait en comprenant plusieurs\\
  Extrait & Verbatim de la phrase transcrite\\
  Confiance & Permet au transcripteur de saisir sa confiance en
              \emph{l'indice de localisation} (uniquement si la saisie
              est faite par un secouriste)\\
  Timestamp & Début de l'extrait dans le fichier audio source\\
  Locuteur & Permet d'identifier le locuteur (\eg secouriste,
             requérant, témoin, \emph{etc.})\\
  Verbe & Verbe utilisé dans \emph{l'indice de localisation}\\
  Modifieur du verbe & (\eg marcher \emph{vite})\\
  Sujet & Sujet de \emph{l'indice de localisation}\\
  Modifieur du sujet & (\eg marcher \emph{vite})\\
  \emph{Relation de localisation} & \emph{Relation de localisation}
                                    utilisée dans l'extrait\\
  Modifieur de la relation de localisation & (\eg \emph{très} loin)\\
  \emph{Objet de référence} & Nom ou type de \emph{l'objet de référence}
                              (cette colonne peut être multipliée si la
                              \emph{relation de localisation} est bi ou n-aire)\\
  Modifieur de \emph{l'objet de référence} & (cette colonne peut être multipliée si la
                                             \emph{relation de localisation} est bi ou n-aire)\\
  Commentaires & Champ permettant au transcripteur de commenter sa
                 saisie ou les indications données par le requérant
                 (erreurs potentielles, fautes de prononciation, \emph{etc.}) \\
  \bottomrule
\end{tabular}

  \caption{Structure du template de retranscription}
  \label{tab:struct_temp}
\end{table}

\subsubsection{Analyse des retranscriptions}

\subsection{Construction des ontologies}

\tdi{Faire un parallèle entre le « la signification c'est l'usage » de
  Wittgenstein et le fait que différentes relations spatiales puissent
voir leur signification varier en fonction du contexte. cf. Léna Soler
p.14}


\begin{figure}
  \centering
  \dict{\onto{ADistanceQuantitativeRelativeDe}}{La \emph{cible} est située
  à une certaine distance du \emph{site.} Cette distance est
  quantitative et est exprimée par une métrique dépendant du
  \emph{site} (\eg \enquote{À un pâté de maisons}). Il n’y a pas de
  contrainte sur la dimension de la distance, elle peut être
  \emph{planimétrique,} \emph{altimétrique} ou les
  deux.}{\enquote{\textins{Il} est à une longueur en dessous
    \textins{du dos d'âne}, donc à une petite longueur du haut du
    couloir de la Meije}}%

\dict{\onto{ADistanceTempsDe}}{La \emph{cible} est à une certaine
  distance du \emph{site}. Cette distance est exprimée en temps de
  parcours. Il est par conséquent nécessaire de connaître le mode de
  déplacement ou la vitesse de déplacement.}{}%

\dict{\onto{ADroiteDe}}{La \emph{cible} est à droite du \emph{site.}
  Ce concept nécessite de connaître l’orientation de la
  \emph{cible.}}{\enquote{\textelp{} sur la droite quand on regarde
    vers le haut de la station.}}%

\dict{\onto{ALEstDe}}{La \emph{cible} se situe globalement à l’est du
  \emph{site,} sans préciser si elle est dans la partie est du
  \emph{site} ou disjointe et à l'est du \emph{site.}}{}%

\dict{\onto{ALEstDeExterne}}{La \emph{cible} est disjointe
  ($\Rightarrow$ \onto{Hors\-De\-Pla\-ni\-me\-tri\-que}) et se situe
  globalement à l’est du \emph{site} ($\Rightarrow$
  \onto{A\-Est\-De}).}{}%

\dict{\onto{A\-L\-Extremite\-De}}{Le \emph{site} est de forme allongée
  (il a alors deux extrémités), ou possède une ou plusieurs parties
  saillantes allongées (il peut alors avoir plusieurs extrémités). La
  \emph{cible} est située au niveau ($\Rightarrow$ \onto{Proche\-De})
  l'une de ces extrémités. Elle peut être située à l'intérieur du
  \emph{site,} ou non.}{\enquote{Il est à l'extrémité ouest d'une
    espèce de terrasse en béton.}}%

\dict{\onto{AGaucheDe}}{La \emph{cible} est à gauche du \emph{site.}
  Ce concept nécessite de connaître l’orientation de la
  \emph{cible.}}{\enquote{Juste sur la gauche, il y a une autre dent.
    Je ne sais pas comment elle s’appelle.}}%

\dict{\onto{A\-La\-Frontiere\-De}}{Le \emph{site} a une emprise
  spatiale linéaire ou surfacique. La \emph{cible} est
  \enquote{proche} de ses frontières. La \emph{cible} peut
  indifféremment se situer à l’intérieur ($\Rightarrow$
  \onto{Dans\-Pla\-ni\-mé\-tri\-que}) ou à l'extérieur du \emph{site}
  ($\Rightarrow$ \onto{Hors\-De\-Pla\-ni\-mé\-tri\-que}). Si elle est
  à l’intérieur, elle est alors dans le complémentaire de la zone de
  l’espace qui vérifie la relation \onto{Au\-Milieu\-De}. La distance
  à considérer dépend très probablement de la nature ou de la taille
  du \emph{site.}}{\enquote{Je suis un peu à la sortie de la forêt.}}%

\dict{\onto{ALaMemeAltitudeQue}}{La \emph{cible} est située à la même
  altitude que le \emph{site.} Le \emph{site} peut être une altitude
  absolue ou un objet dont l’altitude sert de
  référence.}{\enquote{C’est à 2300 mètres d’altitude.}}%

\dict{\onto{ALOuestDe}}{La \emph{cible} se situe globalement à l’ouest
  du \emph{site,} sans préciser si elle en est à l'intérieur ou à
  l'extérieur.}{\enquote{\textelp{} et là on est à l'Ouest.}}%

\dict{\onto{ALOuestDeExterne}}{La \emph{cible} se situe globalement à
  l’ouest ($\Rightarrow$ \onto{AOuestDe}) et est disjointe du site
  ($\Rightarrow$ \onto{Hors\-De\-Pla\-ni\-me\-tri\-que}).}{\enquote{On
    est versant ouest, côté Vercors intérieur.}}%

\dict{\onto{ApresJalonSurItineraire}}{La \emph{cible} est située sur
  un itinéraire ($\Rightarrow$
  \onto{Si\-tue\-Sur\-Iti\-ne\-rai\-re\-Ou\-Reseau\-Support}), et
  après le \emph{site,} qui est un jalon de cet itinéraire. Cela
  suppose d’avoir défini un référentiel des directions associé à
  l’itinéraire, généralement lié au sens de progression (l’avant est
  vers la destination, l’arrière vers l’origine).}{\enquote{Après le
    parking de la Villette.}}%

\dict{\onto{ATempsDeMarche}}{La \emph{cible} est à tel temps de marche
  du \emph{site,} en marchant depuis le \emph{site} vers la
  \emph{cible.}}{\enquote{On est à 10 minutes du sommet de la
    Bastille.}}%

\dict{\onto{AuDessusALAplombDe}}{}{}%

\dict{\onto{AuDessusAltitude}}{La \emph{cible} a une altitude
  supérieure au \emph{site.} La distance entre le \emph{site} est la
  \emph{cible} n’est pas contraignante.}{\enquote{Au-dessus de
    Percolin, y’a la cabane du berger de Bellefont et on est 50 mètres
    au-dessus.}}%

\dict{\onto{AuDessusProche}}{La \emph{cible} est proche et a une
  altitude supérieure au site.}{\enquote{Juste au-dessus de Bernin.}}%

\dict{\onto{AuDessusJalonSurItineraire}}{La \emph{cible} est au-dessus
  du \emph{site} et sur un chemin menant au site (implique la relation
  \onto{Situe\-Sur\-Itineraire\-Ou\-Re\-seau\-Support}).}{\enquote{\textelp{}
    Sur le sentier qui mène à la cascade de l’Oursière, haut dessus de
    la cascade de l’Oursière.}}%

\dict{\onto{AuMilieuDe}}{}{}%

\dict{\onto{AuNordDe}}{La \emph{cible} se situe globalement au nord du
  \emph{site,} sans préciser si elle en est à l'intérieur ou à
  l'extérieur.}{}%

\dict{\onto{AuNordDeExterne}}{La \emph{cible} se situe globalement au
  nord et est disjointe du\emph{ site.}}{\enquote{\textelp{} non, côté
    Nord (du Pas de la Ville), pardon.}}%

\dict{\onto{AuSudDe}}{La \emph{cible} se situe globalement au sud du
  site, sans préciser si elle en est à l'intérieur ou à
  l'extérieur.}{}%

\dict{\onto{AuSudDeExterne}}{La \emph{cible} se situe globalement au
  sud et est disjointe du \emph{site.}}{\enquote{Je suis entre le
    grand Veymont et Pas de la Ville, tout à fait, coté sud.}}%

\dict{\onto{AuxAlentoursDe}}{La \emph{cible} est suffisamment proche
  du site \emph{pour} qu’on puisse considérer que le \emph{site} est
  un point de repère qui a un sens. La proximité qui pourrait s’en
  déduire est généralement dépendante d’un \enquote{rayonnement} qui
  pourrait êtrre affecté au \emph{site} (lié à sa renommée, à sa
  taille typiquemnet pour un lieu habité, à sa saillance,
  \emph{etc.}). Il peut y avoir connexion topologique entre la
  \emph{cible} et le \emph{site,} ou non.}{\enquote{Dans le massif de
    la Chartreuse}}%

\dict{\onto{AvantJalonSurItineraire}}{La \emph{cible} est située sur
  un itinéraire (implique la relation
  \onto{Situe\-Sur\-Itineraire\-Ou\-Reseau\-Support}) et avant le
  \emph{site} qui en est un jalon. Cela suppose d’avoir défini un
  référentiel des directions associé à l’itinéraire, généralement lié
  au sens de progression sur l’itinéraire.}{}%

\dict{\onto{AvoirASaDroite}}{La \emph{cible} est munie d’un
  référentiel des directions intrinsèque : elle a une gauche et une
  droite (et un avant, un arrière). Dans ce référentiel des
  directions, et plus précisément sur l’axe gauche-droite de ce
  référentiel, le \emph{site} est situé à droite de la \emph{cible}
  (donc la \emph{cible} a le site à sa \emph{droite,} au sens commun
  de l’expression).}{}%

\dict{\onto{AvoirASaGauche}}{La cible est munie d’un référentiel
  des directions intrinsèque : elle a une gauche et une droite (et un
  avant, un arrière), par exemple parce que c’est un être animé muni
  d’un visage. Dans ce référentiel des directions, et plus précisément
  sur l’axe gauche-droite de ce référentiel, le site est situé à
  gauche de la cible (donc la cible a le site à sa gauche, au sens
  commun de l’expression). Le référentiel des directions doit être
  modélisé explicitement (à préciser : comment)...}{}%

\dict{\onto{AvoirDerriereSoit}}{La cible est munie d’un
  référentiel des directions intrinsèque : elle a un avant et un
  arrière, (et une gauche, une droite), par exemple parce que c’est un
  être animé muni d’un visage. Dans ce référentiel des directions, et
  plus précisément sur l’axe avant-arrière de ce référentiel, le site
  est situé derrière la cible (donc la cible a le site derrière elle,
  au sens commun de l’expression). Le référentiel des directions doit
  être modélisé explicitement (à préciser : comment)...}{}%

\dict{\onto{AvoirDevantSoit}}{La \emph{cible} est munie d’un
  référentiel des directions intrinsèque : elle a un avant et un
  arrière, (et une gauche, une droite), par exemple parce que c’est un
  être animé muni d’un visage. Dans ce référentiel des directions, et
  plus précisément sur l’axe avant-arrière de ce référentiel, le
  \emph{site} est situé devant la \emph{cible} (donc la \emph{cible} a
  le \emph{site} devant elle, au sens commun de
  l’expression).}{\enquote{Devant moi j’ai la plaine, et à gauche j’ai
    quand même des sapins.}}%

\dict{\onto{CibleVoitSite}}{Le \emph{site} est visible depuis la
  \emph{cible}.}{\enquote{Je suis vraiment en montagne, je vois les
    plaines, je vois un grand découvert devant moi.}}%

\dict{\onto{DansLaDirectionDe}}{}{}%

\dict{\onto{DansLaPartieBasseDe}}{La \emph{cible} est située dans
  la partie basse du \emph{site}.}{}%

\dict{\onto{DansLaPartieDeXLaPlusProcheDeY}}{Le \emph{site} 1
  ($X$) est un surfacique (en 2,5D) le \emph{site} 2 ($Y$) est situé
  en dehors du \emph{site} 1 et permet de délimiter une portion du
  \emph{site} 1 qui est \enquote{la partie la plus proche} du
  \emph{site} 2. La \emph{cible} est alors située dans le \emph{site}
  1 (\onto{Dans\-Planimétrique}), dans cette \enquote{partie la
    plus proche} du \emph{site} 2.}{\enquote{Dans la Combe de la
    Glière côté télésiège.}}%

\dict{\onto{DansLaPartieEstDe}}{La \emph{cible} se situe dans le
  \emph{site} et dans sa partie est.}{}%

\dict{\onto{DansLaPartieHauteDe}}{La \emph{cible} est située dans
  la partie haute du \emph{site}. Cela suppose de pouvoir définir une
  \enquote{partie haute} du \emph{site}.}{}%

\dict{\onto{DansLaPartieNordDe}}{La \emph{cible} se situe dans le
  \emph{site} et dans sa partie nord.}{}%

\dict{\onto{DansLaPartieOuestDe}}{La \emph{cible} se situe dans
  le \emph{site} et dans sa partie ouest.}{\enquote{Il est à
    l'extrémité ouest d'une espèce de terrasse en béton.}}%

\dict{\onto{DansLaPartieSudDe}}{La \emph{cible} se situe dans le
  \emph{site} et dans se partie sud.}{}%

\dict{\onto{Dans\-Planimétrique}}{}{\enquote{Non, on est en
    forêt.}}%

\dict{\onto{DeAutreCoteDeParRapportA}}{La \emph{cible} est de l’autre
  côté du \emph{site} 1 par rapport au \emph{site} 2. Autrement dit,
  la \emph{cible} et le \emph{site} 2 sont de part et d’autre du
  \emph{site} 1. C’est une situation opposée à
  \onto{Du\-Meme\-Cote\-Que\-Par\-Rapport\-A}, où la \emph{cible} et
  le \emph{site} 1 sont situés du même côté du \emph{site} 2. Comme
  pour la relation \onto{Du\-Meme\-Cote\-Que\-Par\-Rapport\-A}, cela
  suppose qu’on peut partitionner l’espace en deux zones appelées les
  deux \enquote{côtés} du \emph{site} 1. Selon la forme et la nature
  du \emph{site} 1 (et son contexte spatial) :
  %
  \begin{enumerate*}[label=(\arabic*)]
  \item soit le \emph{site} 1 a intrinsèquement deux côtés (un col,
    une crête, une rivière, une ville située sur une rivière,
    \emph{etc.}),
  \item soit, pour deux \emph{sites} assimilables à des ponctuels et
    sans contexte spatial particulier, c’est le couple (\emph{site} 1,
    \emph{site} 2) qui permet de définir deux côtés au \emph{site} 1 :
    un côté qui contient le \emph{site} 1, et un côté qui ne le
    contient pas.
  \end{enumerate*}
  Typiquement la limite est alors la perpendiculaire au segment
  joignant le \emph{site} 1 au \emph{site} 2, passant par le
  \emph{site} 1.}{\enquote{Les rochers de l’homme, vous descendiez de
    l’autre côté plutôt ?}}%

\dict{\onto{DistanceQuantitativePlanimetrique}}{La \emph{cible} est à
  telle distance du \emph{site}, exprimée dans une unité de longueur
  (et non un temps de marche ou d'accès). La distance considérée est
  planimétrique.}{\enquote{J'estime \textins{que je suis} à 800 m
    \textins{du Pas de la Ville}, je crois, à peu près, à vol
    d'oiseau}}%

\dict{\onto{DuMemeCoteQueParRapportA}}{La \emph{cible} située du même
  côté que le \emph{site} 1, par rapport au \emph{site} 2. Cela
  suppose qu’on peut partitionner l’espace en deux zones appelées les
  deux \enquote{côtés} du \emph{site} 2. Selon la forme et la nature
  du \emph{site} 2 (et son contexte spatial) :
  %
  \begin{enumerate*}[label=(\arabic*)]
  \item soit le \emph{site} 2 a intrinsèquement deux côtés (un col,
    une crète, une rivère, une ville située sur une rivière,
    \emph{etc.}),
  \item soit, pour deux \emph{sites} assimilables à des ponctuels et
    sans contexte spatial particulier, c’est le couple (\emph{site} 1, \emph{site}
    2) qui permet de définir deux côtés au \emph{site} 2 : un côté qui
    contient le \emph{site} 1, et un côté qui ne le contient pas.
  \end{enumerate*}
  Typiquement la limite est alors la perpendiculaire au segment
  joignant le \emph{site} 1 au \emph{site} 2 et passant par le
  \emph{site} 2. Cette relation est l’exact opposée de
  \onto{De\-L\-Autre\-Cote\-De\-Par\-Rap\-port\-A.}}{\enquote{C'est
    côté lac Robert.}}%

\dict{\onto{EntreXetY}}{Relation ternaire, nécessitant deux
  \emph{sites.} La \emph{cible} est située entre les deux
  \emph{sites}. Le cadre de référence est particulièrement important
  ici. Il faudra le modéliser pour pouvoir spatialiser cette
  relation. Par exemple, si les deux \emph{sites} peuvent être
  considérés comme de faible extension spatiale (donc assimilables à
  des ponctuels), le cadre de référence peut inclure un ou plusieurs
  tracés à une dimension qui relient les deux \emph{sites} : tracé
  d’un itinéraire, \emph{etc.} (ou une ligne droite en l’absence
  d’information). Si l’extension spatiale est plus importante, on peut
  définir une zone \enquote{entre les deux} comme dans le modèle 5IM
  (Clementini et Billen 2006). Mais attention aux cas déictiques comme
  \enquote{je vois Z entre X et Y} ou \enquote{nagez entre les deux
    poteaux} de \autocite{Bateman2010}.}{\enquote{Là on est dans le
    secteur entre \textelp{} Cordéac et Pellafol.}}%

\dict{\onto{HorsDePlanimetrique}}{}{}%

\dict{\onto{PresDe}}{}{\enquote{Il était près du sommet et dans le
    brouillard.}}%

\dict{\onto{Proximal}}{La \emph{cible} est dans le \emph{site} ou à sa
  proximité immédiate, qui peut être une proximité
  fonctionnelle.}{\enquote{Elle est à 50m, je suis sur un promontoire
    car le réseau ne passait pas.}}%

\dict{\onto{SiteVoitCible}}{Le \emph{site} voit la \emph{cible}
  (vision active). La position de la \emph{cible} est visible depuis
  une autre position connue (\eg un refuge depuis lequel un témoin
  contacte les secours).}{}%

\dict{\onto{Si\-tue\-Sur\-Iti\-ne\-rai\-re\-Ou\-Re\-seau\-Sup\-port}}{La
  \emph{cible} se situe sur un réseau ou un itinéraire. Le
  \enquote{sur} a ici un sens fonctionnel. Le \emph{site} peut être un
  élément de réseau au sens large : réseau parcourable à pied, réseau
  de pistes de ski, voies d’escalade ou d’alpinisme, réseau
  hydrographique (\eg en kayak ou canyoning), itinéraire de ski de
  randonnée, \emph{etc.} Aucun apriori n’est considéré sur la
  modélisation géométrique du réseau dans les données. Il n’y a pas
  nécessairement de connexion topologique entre la \emph{cible} et le
  \emph{site} : la \emph{cible} peut se trouver de fait légèrement
  éloignée de l’élément de réseau (\eg cas d’une aire de pique-nique
  accesible depuis le réseau de randonnée pédestre).}{\enquote{On est
    sur le GR 54.}}%

\dict{\onto{SousALAplombDe}}{La \emph{cible} est à une altitude
  supérieure à celle du \emph{site} et ce dernier est plus ou moins
  situé sur la ligne de plus grande pente qui passe par la
  \emph{cible,} ou inversement. Typiquement, le \emph{site} et la
  \emph{cible} appartiennent à une même vallée et sont situés à peu
  près au même niveau longitudinalement.}{}%

\dict{\onto{SousAltitude}}{La \emph{cible} a une altitude inférieure à
  celle du \emph{site.} Sa distance au site n’est pas
  contrainte.}{\enquote{Je pense pas qu'ils soient passés par les
    vires en-dessous.}}%

\dict{\onto{SousJalonSurItineraire}}{La \emph{cible} est située sur un
  chemin menant au \emph{site} et sous ce dernier}{\enquote{On est
    sous le déversoir du lac de Belledonne.}}%

\dict{\onto{SousProcheDe}}{La \emph{cible} est proche et a une
  altitude inférieure au \emph{site}.}{\enquote{On est à 100 m ou 150,
    ou 200 m du col, vraiment juste en-dessous.}}%

\dict{\onto{SousRecouvertPar}}{La \emph{cible} est sous le \emph{site}
  et elle est recouverte par ce dernier.}{\enquote{\textelp{} un mec
    \textins{est} coincé sous un arbre.}}%
%%% Local Variables:
%%% mode: latex
%%% TeX-master: "../../main"
%%% End:

  \caption{Onto}
  \label{fig:ontho}
\end{figure}

%%% Local Variables:
%%% mode: latex
%%% TeX-master: "../../../../main"
%%% End:
