Le secours en montagne est aujourd'hui un service public majeur, et
fortement ancré dans inconscient collectif.

La simplicité de suface cache cependant une grande complexité. Il
s'agit d'une institution compliquée.

L'objectif de cette partie est de décrire le secours en montagne et
son fonctionnement

\subsection{Le secours en montagne dans les Alpes françaises}
\label{subsec:1-1-1}


\begin{verbatim}
Expliquer que le secours comme on le voit auj, Professionel, héliporté
et médicalisé est quelque chose qui s'est construit avec le
temps. Pendant longtemps c'était des amateurs et la présence
systématique de médecins est assez récente (Halle)
\end{verbatim}

\subsubsection{Historique}
\label{subsubsec:1-1-1-1}

Nous ne souhaitons pas entreprendre un travail d'historien, très
éloigné de nos compétances. Cette partie n'est pas construitre pour
répondre à une problématique historique, mais plutôt pour dresser un
état des lieux du secours en montagne.

Malheureusement ces événements nous sont apparus peu documentés.

Les événements principaux de l'histoire des secours en montagne
français peuvent etre retrouvés sur la figure
\ref{fig:frise_chronologique} page \pageref{fig:frise_chronologique}.


\paragraph{Le \enquote{proto-secours} en montagne}
\label{par:1-1-1-1-1}

Si l'histoire du secours en montagne est intimement liée à celle de
l'alpinisme et plus généralement du \emph{tourisme montagnard,} on ne
peut l'y réduire. La présence humaine en milieu montagneux, bien
qu'impactée par ces activités, y est bien antérieure, et les motagnes
jouent depuis longemps, aussi bien le rôle de refuges, que de points
de passage, de \enquote{synapses} \autocite[p. 337]{Brunet1992}. De
fait, les massifs montagneux et notamment les Alpes sont parcourus
depuis l'Antiquité et les marchands ou voyageurs en périls sont
secours par les habitants de ces vallées reculées
\autocite{Mezin2016}.

Cependant, cette forme d'assistance est bien différente du secours en
montagne contemporain. D'une part, car les espaces concernés sont
différents. La haute-montagne est, jusqu'au \bsc{xix}\up{e} siècle,
\emph{terra incognita,} et ni les voyageurs, ni les locaux ne s'y
aventurent. Les secours ne se déroulent donc que le long d'itinéraires
traversants et non sur des arêtes effilées ou à flanc de parois
exposées. D'autre part il n'y a aucune forme d'organisation pérenne,
les caravannes de secours se forment lorsque nécessaire, sur la base
du volontariat.

Le champ d'action du secours en montagne va changer avec l’émergence,
au \bsc{xix}\up{e} siècle, du nouveau divertissement
\enquote{\textins{d}es grands bourgeois britaniques}
\autocite{Descamps2018}, l'alpinisme. L'objectif n'est plus de fuir
les massifs les plus denses et les pentes les plus abruptes, mais d'y
converger.  La difficulté des secours augmente avec l'ambition des
ascensions, mais ils sont toujours effectués par des volontaires,
comme des guides locaux\footnote{La première des compagnies de guide
  est crée en 1821 à Chamonix. Il s'agit de la seule compagnie de
  guides créée avant 1850 \autocite{ContributeursWikipedia2020b}.} ou
des alpinistes présents sur place, ultérieurement soutenus par les
militaires des régiments alpins\footnote{Les régiments alpins sont
  crées en 1888 \autocite{Mezin2016}.}.

Toutefois cette organisation ne va pas sans poser quelques
problèmes. Dans un récit de secours publié en 1911,
\bsc{Thomas} \autocite{Thomas1911} se montre critique envers
l'organisation courante des sercours en montagne. Il met en évidence
le fait que, sous couvert de solidarité alpine, les volontaires sont
discrètement dédommagés par la victime ou par sa famille. Cette forme
de rémunération, couplée à l'absence d'institution organisant les
secours, favorise la multiplication des caravanes de secours
concurentes. Ce qui peut rendre les secours plus dangereux et moins
efficaces. L'auteur appelle à rationaliser les secours en favorisant
le recours à des petites caravanes composées d'hommes
préparés et entrainés.

Ce problème sera, en partie, résolu avec le développement de
regroupements de secouristes centralisant l’organisation des
opérations de secours, les \emph{comité de secours.} Le premier
d'entre eux, \emph{les sauveteurs volontaires du Salève,} apparait en
1897 dans les préalpes \autocite{CFDLD}, 14 ans avant l'article de
\bsc{Thomas.} Mais cette initiative sera reprise bien plus tardivement
dans d'autres régions. \textcite{Caille2016} explique cette précocité
par la forte pression touristique subie par la région à la fin du
\bsc{xix}\up{e} siècle.  Le second comité de secours en montagne est
créé à Grenoble en 1910 \autocite{CFDLD,Caille2016}. Quant à la
majorité de ces structures, elles voient le jour au tournant des
années 30. Avec, notamment, la création d'un commité de secours à
Annecy en 1928, à Chamonix en 1929, à Briançon en 1932 ou à Pau en
1936 \autocite{CFDLD, Devies1946}.

Cette organisation des secours, basée sur l'action de
comités locaux et sans réelle coordination entre massifs perdurera
jusqu'à l'après-guerre. Durant cette période l'armée sera la seule
institution publique à participer aux secours en montagne, son action
ira même en se renforçant, notamment par le support direct
(personnels) et indirect (développement de matériel, formation) de
\emph{l'école de haute-montagne} \acp{ehm} crée en 1932 à Chamonix
\autocite{Mezin2016}.

Cependant, en 1946 \textcite{Devies1946} remet en cause l'organisation
des secours en France. De son point de vue les différents comités
locaux sont une solutuion bien insuffisante aux problèmes du secours
en montagne, notamment comparé aux solutions mises en place en Suisse
ou en Autriche. \bsc{Devies} prone une centralisation de
l'organisation des secours, par le biais d'une institution chapeautant
les différents comités locaux et subventionnée par l'état.

\paragraph{Le début d'une organisation nationale}

Les propositions de \bsc{Devies} aboutissent, en 1947, lorsqu'est
créée la commision de secours de la fédération française de montagne
\acp{ffm} \footnote{La \ac{ffm} est officiellement créée deux ans plus
  tôt, en 1945.}  dont il prend la tête. L'objectif de cette nouvelle
institution est de fédérer les différents commités locaux de secours
en montagne pour proposer des réponses plus efficaces aux situations
de crise. Cette organisation permettra la mise en place de secours
d'envergure, comme en 1948, lorsqu'une centaine de secouristes
participent au sauvetage de deux alpinistes coincés dans la face sud,
d'une verticalité extrème, du Pavé dans le massif des Écrins
\autocite{Romanaz2018} ou en 1949 lors du sauvetage de l'Olan
\autocite{Mollaret1993}. Cette commission dirigera également les
secours lors des crashs du \emph{Canadian Pilgrim} sur l'Obiou et du
\emph{Malabar Princess} sur le mont-Blanc en 1950
\autocite{CFDLD,Mollaret1993,SDSM2013}.

Cepndant, cette organisation semi-professionnelle montrera rapidement
ses limites, la faute à des difficultés de financement
\footnote{Problème que \bsc{Devies} avait déjà soulevé dans son
  article de 1946 \autocite{Devies1946}}, de recrutement, mais
également à cause de la hausse de fréquentation des massifs Alpins
\autocite{CFDLD}. Mais c'est en 1956, avec \emph{l'affaire Vincendon
  et Henry} que les autorités prirent concience des limites de cette
organisation bénévole des secours en montagne.

Le 22 décembre 1956 deux étudiants passionnés d'alpinisme, Jean
\bsc{Vincendon} et François \bsc{Henry} entament l’ascension du mont
Blanc \autocite{Ballu1997}. Les deux hommes cherchent l'exploit, à une
époque où l'alpinisme hivernal est encore balbutiant. Ils décident
cependant de redescendre vers Chamonix et d'abandonner leur ascension
le 24, par crainte du mauvais temps à venir. Cependant, leur rencontre
avec la cordée du guide italien Walter \bsc{Bonatti,} déjà auréolé
d'acenssions notables comme la première du K2 en 1954
\footnote{\bsc{Bonatti} ne fera cependant pas parti des deux
  alpinistes ayant atteint le sommet
  \autocite{ContributeursWikipedia2020a}.} ou l'ouverture en solitaire
d'une voie dans la face sud-ouest des Drus en 1955 \footnote{Cette
  ascension n'est pas la première de la face ouest des Drus, celle-ci
  ayant été effectuée trois ans auparavant
  \autocite{ContributeursWikipedia2020}. Cependant l'ouverture en
  solitaire par \bsc{Bonatti} d'une voie au centre de la paroi et
  considérablement plus exposée est considérée comme l'une des plus
  importantes réalisations de l'alpinisme, qualifiée
  \enquote{\textins{d'}exploit magnifique et inspiré} par
  \textcite{Robbins2000}.}, les incite à reprendre l’ascension. Les
deux cordées, fidèles à leur itinéraire initial, se séparent le
lendemain. Bien qu'empruntant des itinéraires distincts, les deux
groupes se font surprendre par la nuit et bivouaquent sous la
tempête. Le lendemain \bsc{Bonatti} et son client atteignent le refuge
Vallot, alors que \bsc{Vincendon} et \bsc{Henry,} épuisés, prennent la
décision de descendre directement. Lors de la descente les deux
alpinistes se perdent et se retrouvent coincés sur une
\gls{vire}. Exténués et surchargés de matériel ils sont incapables de
rebrouser chemin et passent une seconde nuit dehors.

Dans la vallée, les secours, prévenus dès le 25 décembre par un
proche, tardent à se mettre en place. La société chamoniarde de
secours en montagne, regoupant l'\ac{ehm}, la compagnie des guides de
Chamonix et de l'école nationale de ski alpin n'arrive pas à organiser
une opération de secours, les guides refusant de s'aventurer en
haute-montagne en plein hiver et en pleine tempête. C'est finalement
l'\ac{ehm} qui prendra la direction des secours. Un premier
hélicoptère militaire est envoyé le 27 décembre, sans résultat, mais
les alpinistes sont repérés, à la longue-vue, un peu plus tard. La
configuration spatiale de l'événement favorise effervescence
médiatique, les accidentés sont visibles depuis Chamonix. De nombreux
médias nationaux, tel que la radio \emph{Europe 1,} nouvellement crée,
commencent à couvrir l'événement et les secouristes doivent opérer
sous une pression médiatique alors inédite.

Les victimes étant localisées, une opération de secours héliportée est
organisée le 28. Mais l'utilisation d'un appareil inhadapté au vol en
montagne rend l'approche impossible. Le pilote indique aux alpinistes
de se rapprocher d'un plateau, situé en amont, seul endroit où un
atterrissage est envisageable. Cependant la météo empêche tout nouveau
vol et le directeur de l'\ac{ehm}, commandant de fait des opérations,
refuse de risquer de nouvelles vies en envoyant une caravane terrestre
en pleine tempête. C'est finalement le 31 décembre qu'un nouveau vol
sera possible. Les deux alpinistes sont rejoints par un hélicoptère de
l'armée, qui s'écrase lors de son atterrissage. Les deux pilotes sont
grièvements bléssés. Les deux secouristes présents mettent alors
\bsc{Viencendon} et \bsc{Henry} à l'abri, dans la carcasse de
l'hélicoptère et prennent la décision de remonter en priorité les
pilotes au refuge Vallot, avant de redescendre, le lendemain, pour
secourir les deux alpinistes. Cependant, les quatres hommes restent au
refuge jusqu'a leur évacuation, le 3 janvier. Ne constatant plus aucun
signe de vie, le commandant des opérations décide alors d'abandonner
les recherches.

Cette affaire, qui fut un véritable choc pour l'opignon publique,
marqua un point de rupture dans la conception française du secours en
montagne. Le système de secours basé sur le volontariat avait montré
ses limites et sera rapidement replacé par un système géré par l'état
\autocite{Ballu1997}.

\paragraph{Le secours en montagne comme service public}
\label{par:1-1-1-1-2}

\begin{displayquote}
  \og Il a fallu le spectacle de guides professionneles chamoniards
  qui laissèrent périr deux jeunes étudiants en perdition faute
  d'avoir pu s'organiser pour que l'État décide de mettre en place ses
  propres services \textelp{}\fg{} \autocite{Descamps2018}
\end{displayquote}

C'est à partir de 1958, après le choc de \emph{l'affaire Vincendon et
  Henry,} que les services publics de secours en montagne se mettront
en place. Les gendarmes et policiers, auparavant participants
occasionnels à des opérations de secours
\autocite{Mollaret2016,CFDLD}, en sont officiellement chargés par la
circulaire interministérielle du 21 août 1958 \footnote{Circulaire \no
  1272 du 21 août 1958 relative à la mise en œuvre du secours en
  montagne. \label{fn:circulaire_21_aout_58}}. Deux unitées
spécialisées dans le secours en montagne sont alors créées, le
\ac{pghm} \footnote{Alors le Groupe Spécialisé en Haute-montange} et
de la \ac{crsm} \autocite{Halle2007}. La formation de ses premiers
professionels du secours en montagne est alors assurée par des
institutions pré-existantes, comme l'\ac{ehm} ou le \ac{cenas} fondé
en 1955 \autocite{Mezin2016}.

L'action de ces deux corps est régulée au niveau local par des
dispositions spécifiques définies dans les plans \ac{orsec}
\footnote{ORganisation des SECours jusqu'en 2006, puis Organisation de
  la Réponse de la Sécurité Civile.}. Lesquels, préxistants, ont étés
étendus aux secours en montagne. Dans le département de l'Isère, par
exemple, est mis en place, dès 1958, un régime d'alternance. La
\ac{crsm} et le \ac{pghm} sont chargés des secours une semaine sur
deux, ce qui permet d'alterner astreinte et entrainement ou repos. Ce
régime sera par la suite étendu à la majorité des départements Alpins,
même si de nombreuses spécifités locales existent
\autocite{Halle2007}. À partir de 1985 un nouveau corps de secours en
montagne fera son apparition les \ac{grimp} un groupe d'intervention
de pompiers spécialisé dans les milieux périlleux
\autocite{CFDLD}. Les modalités de cohabitations entre ces trois corps
changement en fonction des départements et des plans \ac{orsec}.
% TODO: Completer
Cette cohabitation entre trois corps distincs a pu créer des
situations conflictuelles par le passé \autocite{Soule2002,
  Ganser2012}.

À la faveur de la professionnalisation, les techniques et méthodes de
secours en montagne vont connaitre une rapide évolution. L'utilisation
de l'outil héliporté, maladroite lors de \emph{l'affaire Viencendon et
  Henry,} sera considérablement perfectionée au cours des années
soixante. Notamment grâce à l'introduction de l'\emph{alouette
  \bsc{iii}} en 1962, un hélicoptère parfaitement adapté au vol en
montagne et qui sera utilisée jusqu'en 2009 \autocite{Elie2006,
  Lafond, Lafond2011b}. L'utilisation de cet outil va évoluer avec le
développement de solutions techniques, tel que le treuil, remplacant
avantageusement les échelles de courdes suspendues sous le
fuselage. Cet outil, introduit en 1965 et utilisé pour la première
fois en 1967 lors d'un sauvetagne au Grépon \autocite{Lafond2011a},
permet dans un premier temps de treuiller des civières, évitant à
l'hélicoptère de se poser. Plus tard, des utilisations plus
ambitieuses sont envisagées, comme en 1972 lorsqu'un équipage évacue
deux alpinistes en difficulté sur la face ouest des Drus
\autocite{Ministere2013}. L'outil héliporté se révéle d'une efficacité
redoutable et dès 1972 l'ensemble des zones montagneuses françaises
seront à portée d'hélicoptère \autocite{CFDLD}. Les secours héliportés
occuperont une place de plus en plus importante dans les opérations de
secours en montagne, passant de 50 \% des 420 secours de 1965
\autocite{CFDLD} à plus de 90 \% des secours aujourd'hui
\autocite{Halle2007}.

Une autre avancée majeure permise par la professionalisation des
secours a été leur médicalisation. Si des alpinistes--médecins ont pu
participer aux secours, c'est avant tout des actions volontaires, qui
vont commencer a se structurer dans le courant des années 70 avec la
création des \ac{samu} \autocite{Halle2007}, notamment par le biais de
permanances médicales bénévoles, comme celle mise en place en Isère en
1971 \autocite{Rocourt2014}. Parallèlement, l'armée apportera son
soutien à partir de 1975 avec le détachement de médecins du
contingent. À partir des années 80 que la présence des médecins dans
les équipages va se généraliser \autocite{CFDLD}. Notamment sous
l'impulsion de l'article 2 de la loi de 1986 relative à l'aide
médicale d'urgence \footnote{L’ordonnance n° 2000-548 du 15 juin 2000
  abroge les articles 1 et 2 de la loi \no 86--11 et crée les articles
  L6311-1 et L6311-2 du code de la santé public de contenu identique.}
\autocite{Rocourt2014} qui précise que le rôle de l'aide médicale
d'urgence est \enquote{\textins{d'}assurer aux malades \textelp{} en
  quelque endroit qu'ils se trouvent, les soins d'urgence appropriés à
  leur état.} À partir de 1992, après le désengagement de l'armée, la
médicalisation des secours en montagne est entièrement gérée par le
\ac{samu} ou directement par les pompiers si ce sont ces derniers qui
interviennent \autocite{Rocourt2014, Halle2007}.

Un autre aspect important induit par la professionalisation des
secours et leur gratuité. Les frais d'interventions des \ac{pghm}, des
\ac{crsm} ou des pompiers ne sont pas facturés aux victimes. Ce n'est
cependant pas le cas des soins hospitaliers ou pré-hospitaliers liés
au secours, qui eux sont facturés, comme toute intervention du
\ac{samu}. La gratuité des opérations de secours a cependant souvent
été discutée ou nuancée \autocite{CFDLD, Halle2007, Magne2017}. Dans
un premier temps par la loi, dite Montagne, de 1985, qui autorise les
communes à déléguer les opérations de secours sur domaine skiable à
des structures privées et a demander le remboursement des frais de
secours aux victimes. Cependant la circulaire du 22 septembre 1987,
limite cette disposition à seulement deux activitées, le ski alpin et
le ski de fond. Cette limitation a cependant été retirée en 2002
\footnote{Loi \no 2002-276, relative a la démocratie de proximité.}
\autocite{Magne2017}. Cet événement a été interprété par les
professionels comme une remise en cause du principe de gratuité des
secours, ce qui a été infirmé par Michèle \bsc{Alio-Marie} en 2007,
alors ministre de l'intérieur : \enquote{La gratuité des secours est
  l’un des grands principes de solidarité de la vie en montagne}
\autocite{Descamps2018}.

Les dernières modifications conséquentes à l'organisaition des secours
en montagne français seront faites en 2011 par la circulaire dite
\bsc{Kihl} \footnote{Circulaire du 6 juin 2011 relative aux
  orientations générales pour la mise en œuvre des moyens publics
  concourant au secours en montagne et sa formalisation dans le cadre
  d’une disposition spécifique
  \ac{orsec}. \label{circ:khil}}\multiplefootnoteseparator\footnote{Cette
  circulaire abroge la circulaire interministèrielle d'aout 1958
  (cf. \autoref{fn:circulaire_21_aout_58}), jusqu'ici texte de
  référence.}. Cette dernière vise a rationaliser l'organisation
nationale des secours en montange, notamment en clarifiant les
\enquote{\textelp{} les modalités d\textins{e} coopération normée
  entre les différentes entités \textelp{}}
\footnote{\emph{Avant-propos,} pages 2--3 de la circulaire \bsc{Khil}
  (cf. \autoref{circ:khil}).}, lesquelles avaient été complexifiées
par l'arrivée des pompiers dans le secours en montagne. Cette
circulaire est aujourd'hui le texte de référence quant à
l'organisation du secours en montagne.

\begin{figure}
  \centering
   \begin{tikzpicture}[%
  ev/.style={font=\tiny, text centered, inner sep=2pt, fill=white}
  ]
  % Création de la Frise
  \begin{scope}
    % Axe dégradé à gauche
    \path[draw,-,path fading=west] (-.5,0) -- (0,0);
    % Axe principal
    \path[draw,->] (0,0) --++ (12,0);

    % Trait vertical tous les .75 cm (10 ans)
    \foreach \x in {0,.75,...,11.75}{
      \path[draw] (\x,-.05) --(\x,.05);
    }

    % Étiquettes de décénine
    \node[below, font=\tiny] at (0, -.05) {1860};
    \node[below, font=\tiny] at (.75, -.05) {1870};
    \node[below, font=\tiny] at (1.5, -.05) {1880};
    \node[below, font=\tiny] at (2.25, -.05) {1890};
    \node[below, font=\tiny] at (3, -.05) {1900};
    \node[below, font=\tiny] at (3.75, -.05) {1910};
    \node[below, font=\tiny] at (4.5, -.05) {1920};
    \node[below, font=\tiny] at (5.25, -.05) {1930};
    \node[below, font=\tiny] at (6, -.05) {1940};
    \node[below, font=\tiny] at (6.75, -.05) {1950};
    \node[below, font=\tiny] at (7.5, -.05) {1960};
    \node[below, font=\tiny] at (8.25, -.05) {1970};
    \node[below, font=\tiny] at (9, -.05) {1980};
    \node[below, font=\tiny] at (9.75, -.05) {1990};
    \node[below, font=\tiny] at (10.5, -.05) {2000};
    \node[below, font=\tiny] at (11.25, -.05) {2010};  
  \end{scope}

  % Étiquettes partie supérieure
  \foreach \date/\text/\y [evaluate=\date as \x using (\date - 1860)*.075] in {
    1958/{Intégration des secours en montagne au plan \ac{orsec}}/{4.5},   
    1947/{Création de la commission des secours en montagne de la \ac{ffm}}/{3.25},
    1985/{Création des \ac{gmsp}}/{3},	    
    1897/{Création des Sauveteurs volontaires du Salève}/{2.5},
    1956/{Accident de Vincendon et Henry}/{2.25},		
    1865/{Accident du Cervin}/{1.5},
    2002/{loi \no 2002-276 relative a la démocratie de proximité}/{2},
    1929/{Création d'un comité de secours Savoyard}/{1.75},
    1970/{Avalanche à Val-d'Isère}/{1.5},
    1932/{Création d'un comité de secours Briançonnais}/{.5},
    1874/{Création du \ac{caf}}/{.5},
    1987/{Circulaire du 22 septembre 1987}/{.5}
  } {
    \node[shape=circle,fill=black, scale=.25] (a1) at (\x,0) {};
    \path[draw, -|] (a1) --++ (0,\y) node[ev, above] {\begin{varwidth}{3.5cm}\centering\text\\(\date)\end{varwidth}};
  }

  % Étiquettes partie inférieure
  \foreach \date/\text/\y [evaluate=\date as \x using (\date - 1860)*.075] in {
    1986/{Loi \no 86--11 relative à l'aide médicale urgente et aux transports sanitaires}/{3.75},
    1956/{Premier secours aéroporté}/{3},
    1958/{Création du \ac{pghm}}/{2},
    1910/{Création du comité de secours en montagne du Dauphiné}/{2},
    2007/{La ministre de l'Intérieur confirme la gratuité des secours}/{2},
    1945/{Création de la \ac{ffm}}/{1.25},
    2011/{Circulaire Kihl}/{1.},
    1888/{Création des régiments alpins}/{.5},
    1972/{Couverture aérienne complète}/{.5},
    1932/{Création de l'\ac{ehm}}/{.5}
  } {
    \node[shape=circle,fill=black, scale=.25] (a1) at (\x,0) {};
    \path[draw,-|] (\x, -.4) --++ (0,-\y) node[ev, below] {\begin{varwidth}{3.5cm}\centering\text\\(\date)\end{varwidth}};
  }
\end{tikzpicture}
   \caption{Chronologie des principaux événements de
     l'histoire fraçaise des secours en montagne.}
  \label{fig:frise_chronologique}
\end{figure}

\subsubsection{Organisation contemporaine}
\label{subsubsec:1-1-1-2}

Si la professionnalisation des secours a permis la mise en place d'un
système nettement plus efficace est rationel que ce qui était en place
jusqu'ici, l'organisation actuelle n'en demeure pas moins
complexe. D'une part au niveau national, pusique de nombreux acteurs
différents participent à ces opérations. Les équipes de secours
réunissent des professionels rattachés à administrations différentes
auxquels s’ajoutent toutes les structures participant au traitement
des alertes. D'autre part, de nombreuses spécificités locales
existent. En effet, la législation laisse une importante latitude aux
préfets dans l'organisation des secours en montagne. Par conséquent,
les acteurs concernés et l'organisation des secours varient, parfois
grandement, entre départements.

\paragraph{Échelle nationale}

% Cadre Légal
La législation nationale désigne les préfets de département comme les
responsables de l'élaboration et des secours en montagne, par le biais
de la rédaction des plans \ac{orsec}. De plus le préfet doit également
veiller à la bonne application du plan de secours en montagne, c'est
pourquoi il est systématiquement \emph{directeur des opérations de
  secours} en montagne \acp{dos}. Ce statut n'impose cependant pas au
préfet de participer \emph{directement} aux opérations de
secours. Leur conduite incombe au \emph{commandant des opérations de
  secours} \acp{cos}, dont le processus de désignation change en
fonction de \emph{la nature de l'opération.}

% Nature des opérations
Les \emph{opérations simples} de secours en montagne sont les plus
fréquentes. Il s'agit d'interventions effectuées dans un temps court,
un espace peu étendu et n'impliquant que peu de secouristes. Dans ce
cas le commandement est assuré par le \emph{chef de caravane}
\acp{cc}, qui est nécessairement un secouriste de l'\ac{usem}
d'astreinte, généralement son membre le plus expérimenté qui agit
alors sous couvert du chef de son unité. Si une opération de secours
prend plus d'ampleur elle est qualifiée \emph{d'opération complexe.}
Cette catégorie d'opération nécessite une coordination entre de
nombreux acteurs et peu se dérouler sur des durées plus grandes que
celles d'une \emph{opération simple.} Dans ce cas le \ac{cos} est
directement désigné par le préfet \enquote{à partir d’une liste
  annuelle de cadres issus des unités spécialisées ou détenteurs des
  compétences spécifiques \textelp{}} \footnote{Section 4.2 page 8 de
  la circulaire \bsc{Khil} (cf. \autoref{circ:khil}).}. Dans les faits
cette fonction est souvent assurée par le commandant de l'\ac{usem}
d'astreinte. Enfin, une opération est qualifiée \emph{d'envergure}
lorsqu'elle est d'une importance et d'une complexité telle que le
secours en montagne n'en est qu'une composante parmi d'autres. Dans ce
cas c'est le \emph{directeur départemental des services d'incendies et
  de secours} \acp{ddsis} qui est nommé \ac{cos}, assisté d'un
\emph{chef d'opérations montagne} a qui est généralement délégué le
commandement de cette composante.

% Organisation géographique
C'est également au niveau national que sont fixés les domaines
d'invertion des \ac{usem}. La législation distingue deux espaces
différents intervention, le \emph{domaine montagne} et les
\emph{domaines skiables.} Les seconds sont placés sous la
responsabilité des maires qui, comme nous l'indiquions précédement
(cf. \autoref{par:1-1-1-1-2}), ont la liberté de déleguer les secours
à des opérateurs privés, généralement l'exploitant de la station. Le
rôle des \ac{usem} est d'assurer le secours en \emph{domaine
  montagne,} ce qui n'exclu pas des interventions dans les
\emph{domaines skiables} en cas de nécessité, qu'il revient au préfet
de définir.

% Organisation Administravie
La législation nationale fait des \ac{codis} de chaque département le
point central du traitement des alertes. Ces structures ont à charge
la gestion des appels et la coordination des acteurs du secours. Par
conséquent chaque appel de demande d'assistance en montagne fait
auprès d'un autre opérateur que le \ac{codis} départemental
\footnote{C'est-à-dire si le numéro d'urgence composé n'est ni le 112,
  ni le 18.}, comme le \ac{samu} ou les \ac{usem}\footnote{Les
  permanences des \ac{usem} possèdent des numéros à dix chiffres qui
  sont parfois utilisés pour contacter les secours. Leur usage est
  cependant fortement déconseillé, leur validité n'étant que locale.},
doit être systématiquement trasféré au \ac{codis}. C'est aux
\ac{codis} de définir si la demande d'assistance s'inscrit dans le
cadre des secours en montagne et ce en fonction des règles fixées dans
le département.  Si les \ac{codis} centralisent la gestion des alertes
il ne leur incombe cependant pas de prendre unilatéralement les
décisions quant à la gestion des opérations de secours. En effet, le
cadre légal leur impose, dès lors que l'opérations a été qualifiée de
\emph{secours en montagne,} de mettre en place une conférence avec
tous les acteurs concernés par ce type d'opération. La décision de la
médicalisation est à la charge du \ac{samu} et celle de l'engagement
des moyens héliportés est prise collectivement.

\paragraph{Spécificités locales}

% Composition, répartition des équipes, des hélicoptères
La principale différence entre les département dotés d'un plan de
secours en montagne se situe au niveau des \ac{usem} y
participant. Dans la plupart des départements alpins, le secours est
organisé selon une alternance hebdomadaire
(cf. \autoref{tab:organisation_secours_departements}). Comme nous
l'expliquions précédement (cf. \ref{par:1-1-1-1-2}) cette organisation
a été appliquée pour la première fois depuis 1958. Cependant cette
alternance ne concerne pas nécessairement tout le département. Dans
les Hautes-Alpes, par exemple, trois situations particulières sont
définies. Si le secours a lieu en zone de haute-montagne ou de moyenne
montagne accesible uniquement par hélicoptère, le secours est à la
charge de l'\ac{usem} de permanance, située à Briançon. Dans le cas où
la victime est dans une zone de moyenne montagne facilement accessible
le secours est à la charge du \ac{gmsp} de Gap. Cette organisation
permet d'adapter le dispositif d'invtervention à la topographie du
département, les zones de haute-montagne étant principalement situées
dans le briançonnais, au nord-est du département. D'autres département
ont des organisations moins fréquentes. Dans les alpes de
haute-Provence le secours en montagne est à la charge seule du
\ac{pghm} de Jausiers dans la vallée de L'ubaye. Enfin, pour des
raisons historiques, qle département de la Haute-Savoie est organisé
suivant un régime unique.  Dans la majorité du département les secours
sont réalisés par un équipage mixte, toujours composé de secouristes
du \ac{pghm} et de pompiers du \ac{gmsp}. Ce n'est toutefois pas le
cas pour de massif du mont Blanc, où les secours sont uniquement
assurés par le \ac{pghm} \autocite{Halle2007,Boillot2017}.

\begin{table}
  \centering
  \begin{tabular}{L{6cm}L{8cm}}
  \toprule
  \multicolumn{1}{c}{\bfseries Départements} & \multicolumn{1}{c}{\bfseries Organisation} \\
  \midrule
  Hautes-Alpes (05), Alpes-Maritimes (06), Isère (38), Savoie (73) & Alternance hebdomadaire \ac{crsm}, \ac{pghm} (selon un calendrier national).\\
  Alpes de Haute-Provence (04) & \ac{pghm} uniquement.\\
  Haute-Savoie (74) & Collaboration \ac{pghm}, \ac{gmsp}, sauf dans la région de Chamonux (\ac{gmsp} uniquement) et dans le massif du Mont-Blanc (\ac{pghm} uniquement). \\
  \bottomrule
\end{tabular}
  \caption{Corps mobilisés pour le secours en montagne dans les
    départements alpins.}
  \label{tab:organisation_secours_departements}
\end{table}

% Répartition des hélicoptères
Un autre différence notable entre les départements est la nature des
moyens héliortés à la disposition des secouristes. Pour des raisons
historiques les appareils destinés au secours en montagne sont
répartis entre deux acteurs, la sécurité civile par le biais de son
\emph{groupement hélicoptère} \acp{ghsc} et la gendarmerie par le
biais de sa \emph{force aérienne} \acp{fag}. Cette différence à
principalement des conséquences administratives, les \emph{Dragons} et
les \emph{Choucas} \footnote{\emph{Dragon} et \emph{Choucas} sont,
  respectivement, les indicatifs radio des hélicoptères du \ac{ghsc}
  et de la \ac{fag}. Ce dernier est généralement postfixé par le
  numéro du département de rattachement de l'hélicoptère, \eg
  \emph{Choucas 05,} \emph{Dragon 38.}} déployés en montagne
correspondant au même modèle d'appareil \footnote{\emph{EC145} ou
  \emph{H145} depuis qu'\emph{Eurocopter} est \emph{devenu Airbus
    Helicopter.}}. La répartition des hélicoptères est fixée au niveau
national, certains départements n'ont accès qu'a un seul appareil,
comme les Hautes-Alpes et les Alpes-de-Haute-Provence (appareils des
\ac{fag}), d'autres à deux, comme l'Isère (deux appareils du
\ac{ghsc}), la Savoie et la Haute-Savoie (avec un appareil des
\ac{fag} et un du \ac{ghsc} par département).

\paragraph{Le cas Isérois}

Dans le cas spécifique du département de l'Isère les dispositions
spécifiques au secours en montagne du plan \ac{orsec} datent de 2016
et ont été modifiées pour la dernière fois en 2018.

% Organisation administrative
Comme l'indique la réglementation nationale, le \ac{dos} des
opérations de secours est le préfet de département.  C'est au
\ac{codis} de traiter et de centraliser les appels.

% Organisation Géographique
Comme pour le département des Hautes-Alpes plusieures zones
d'inverventions sont définies. Le plan de secours en montagne
\acp{pms} ne s'applique qu'en zone montagneuse, \ie la partie sud-est
du département (cf. \autoref{crt:isere_pms}). Les interventions dans
la zone d'application du \ac{pms}, à l'exception des stations de ski,
relèvent de la compétance exclusive des unités de secours en montagne
\acp{usem}.

\begin{carte}
  \centering
  \missingfigure{Carte}
  \caption{Zone d'application des dispositions spécifiques au secours
    en montagne du plan \ac{orsec} en Isère.}
  \label{crt:isere_pms}
\end{carte}

% Acteurs dans le département
Dans le département de l'Isère le \ac{pghm} et la \ac{crsm} sont les
seules \ac{usem}. Ces deux coprs fonctionnent selon le principe
d'alternance hebdomadaire, commun à beaucoup de départements. Les
pompiers ne participent donc pas aux secours. Cependant 

% Adaptation du dispositif à la saison
Le département de l'Isère à également comme particularité d'adapter le
dispositif de secours en montagne aux flux touristiques
\footnote{C'est également le cas en Savoie.}. L'aérodrome du Versoud,
en périphérie de Grenoble est le point central des opérations de
secours. Un Hélicoptère et son équipage, une équipe de l'\ac{usem}
d'astreinte et un médecin du \ac{smur} y sont présents tout au long de
l'année. Cette équipe est complétée à la haute-saison (été et hiver)
par une seconde permanence à l'altiport de l'Alpe d'Huez. Y sont
présents, comme pour le Versoud, un hélicoptère et son équipage, une
équipe de secouristes et un médecin. Enfin, durant l'été, une
troisième équipe est déployés au poste de secours de la Bérarde, point
de départ d'un grand nombre de randonées dans l'Oisans et les
Écrins. La mise en place de ces deux permanences de haute-saison est
néanmoins contrainte à une diponibilité suffisante des moyens humains
et matériels, si ceux-ci ne sont pas suffisants le maintient de la
base du Versoud est prioritaire.

% Opérateurs privés

% Les interactions avec les autres départements
Enfin il est a noter que des disposition de collaboration entre
départements sont présentes dans le plan \ac{orsec}

En Isère, par exemple, les pompiers ne participent pas aux opérations
de secours en montagne \footnote{Il convient cependant de nuancer
  cette affirmation. Le plan \ac{orsec} en vigeur en Isère prévoit que
  le chef des opérations de secours \acp{cos} puisse faire une demande
  de détachement de moyens, dont les pompiers, lorsque nécessaire.}.

\subsection{Méthodologie de recherche de victime}
\label{susec:1-1-2}

Jusqu'a présent nous avons traité les opérations de secours en
montagne comme \enquote{un tout}, sans en distinguer les différentes
phases. Pourtant les opérations de secours sont décomposables en deux
phases clairement distincte. La première est celle du traitement de
l'alerte, de la discution entre acteurs du secours. C'est durant cette
phase que serons décidées les modalités d'intervention et de
médicalisation. La seconde phase est celle du secours à proprement
parler, du déplacement des équipes et de l'évacuation de la
victime. Notre travail se focalise sur la première phase, celle de la
préparation de l'intervention et plus particulièrement sur la question
de la localisation.


\begin{verbatim}
Selon, Halle 2007, le déclanchement même de l'alerte peut-être
problématique. 
\end{verbatim}

\subsubsection{La solution \emph{Gend'Loc}}
\label{subsec:1-1-2-1}

Mis en œuvre à partir de février 2013

Une première solution à ce problème a été proposée avec le
Développement de \emph{Gend'Loc,} un outil facilitant la localisation
des vicitimes en montagne.

\paragraph{Présentation de l'outil}

\paragraph{Les limites de la solution}

Bien qu'extrèmement efficace cet outil ne peut pas apporter de
solution satisfaisante à toutes les situations.

On peut identifier trois cas principaux où la solution \emph{Gend'Loc}
n'est pas adaptée.

% Pas de gps
Le premier cas est celui où la victime ne peut pas utiliser la
solution pour des raisons techniques. Cet outil nécessite d'avoir
accès à un \emph{smartphone} doté d'un récepteur GPS. Si la majorité
de français dispose aujourd'hui de ce type de
téléphone\todo[inline]{Note + lien arcep} \footnote{} il n'est pas
exclu que la victime n'en dispose pas, ce qui rend le recours à
\emph{Gend'Loc} impossible.

% Requêrant n'est pas la victime
Il est également possible que les secours soient contactés par tiers
qui n'est pas à proximité de la victime. Les secouristes ne disposent
donc que de son numéro de téléphone ce qui ne permet pas la
localisation de la victime. Toutefois il peut être utile de localiser
le requérant pour approximier la position de la victime.

% Victime incapable d'utiliser la solution
Enfin, dans certains cas la victime ne peut tout simplement pas
effectuer d'opérations complexes avec son téléphone.


\subsubsection{Cas inlocalisables}
\label{subsec:1-1-2-2}

\paragraph{Le cas \enquote{fil rouge}}

\texttt{Présenter une image de la zone du fil rouge ?}

%%% Local Variables:
%%% mode: latex
%%% TeX-master: "../../../../main"
%%% End:
