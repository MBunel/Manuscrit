Ce chapitre nous a permis de présenter la manière dont les \emph{zones
  de localisation} construites par notre méthodologie seront
modélisées. Notre volonté de prendre en compte l'imprécision des
relations de localisation et de la répercuter sur les zones de
localisation impose de recourir a une représentation des objets
géographiques différente de celles généralement utilisées dans le
domaine des \ac{sig}.

Différents modèles théoriques permettent de modéliser
\emph{l'imprécision.} Nous avons choisi d'utiliser la \emph{théorie
  des sous-ensembles flous,} qui offre un cadre théorique permettant
de traiter tous les aspects que nous avons identifiés comme
essentiels, à savoir la possibilité de représenter des \emph{zones de
  localisation} de formes et d'imprécision très diverses et la
possibilité de réaliser les opérations de fusion définies dans la
méthodologie, tout en garantissant l'interprétabilité des résultats.

Pour sélectionner une implémentation de la théorie des
\emph{sous-ensembles flous,} nous avons confronté les deux
implémentations qui nous semblaient être les plus intéressantes,
l'approche par \emph{alpha-cut} et l'approche raster, lors de la
modélisation d'un \emph{indice de localisation} réel issu du cas
\emph{fil rouge} : \enquote{Je suis sous une ligne électrique trois
  brins}. Les deux implémentations permettent une modélisation de
qualité similaire, mais avec des contraintes différentes. Là où
l'approche par \emph{alpha-cuts} est plus difficile à mettre en place,
mais plus sobre, l'approche raster est rapide à utiliser et à étendre,
mais plus couteuse en temps de calcul et en volume de données. Au
terme de cette comparaison, nous nous sommes donc orientés vers
l'implémentation raster, qui nous permet d’implémenter plus rapidement
notre méthodologie de \emph{spatialisation} et, par conséquent, de
traiter un maximum de \emph{relations de localisations} et donc de
types d'indices utilisés dans les alertes.

%%% Local Variables:
%%% mode: latex
%%% TeX-master: "../../../main"
%%% End:
