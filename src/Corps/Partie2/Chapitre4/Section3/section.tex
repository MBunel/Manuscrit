Les \emph{principes de modélisation} décrits, nous allons à présent
formaliser notre méthode de construction de \emph{zones de
  localisation probables}. L'objectif n'est cependant pas d'en
présenter tous les aspects, mais d'en décrire le fonctionnement global
et les interactions entre les différentes phases de la méthode. Ainsi,
les considérations les plus avancées, comme la prise en compte de
\emph{l'incertitude} ou de \emph{l'imprécision,} ou les points
spécifiques à chaque phase particulière de la méthode, seront
détaillés dans des chapitres dédiés. Ce chapitre ne se destine donc
qu'à présenter le socle de notre méthode.

Dans le \autoref{chap:02}, nous avons ébauché une méthode de
construction d'une \emph{zone de localisation probable} à partir
\emph{d'une description de position,} en deux étapes : la
\emph{spatialisation} et la \emph{fusion.} Cette présentation n'est
cependant satisfaisante qu'en première approximation. En effet, les
principes de modélisation introduits dans ce chapitre (et plus
particulièrement le \emph{principe de décomposition}) complexifient la
méthode en y ajoutant de nouvelles étapes, ne pouvant être distinguées
avec ces anciennes définitions. C'est pourquoi ces dernières seront
complétées et revues au fur et à mesure de cette partie.

La méthode que nous proposons est désormais décomposable en trois
phases : la \emph{décomposition,} la \emph{spatialisation} et la
\emph{fusion,} elles-mêmes composées de plusieurs étapes. La
\autoref{fig:methodo_1} présente une synthèse de l'ensemble de cette
méthode.

\subsection{La phase de \emph{décomposition}}

La phase de \emph{décomposition} (qui sera détaillée dans le
\autoref{chap:05}) est un ensemble de trois étapes permettant de
passer d'un ensemble \emph{d'indices de localisation} à un ensemble
décomposé de ces \emph{indices,} prêts à être \emph{spatialisés.}
Chacune des étapes de la phase de \emph{décomposition} produit des
nouveaux \emph{indices de localisation.} Les étapes de cette phase
sont interchangeables, sans que cela n'influe sur le résultat final, à
l'exception de la première étape qui ne peut être effectuée qu'en
premier.

\subsubsection{La décomposition de \emph{l'ensemble des indices de
    localisation}}

La première de ces étapes est la décomposition de \emph{l'ensemble des
  indices de localisation} (\(I\)), en ces différents \emph{indices de
  localisation} (\(i_r\)). Cette décomposition est réalisée par les
secouristes et nous est transmise par le biais de l'interface de
géovisualisation (\autoref{fig:methodo_1}) :

\begin{equation}
  \label{eq:ens_ind_loc}
  I = \{i_1, i_2, \dots, i_r \}
\end{equation}

La \emph{décomposition} de l'ensemble \(I\) permet d'extraire les
différents \emph{indices de localisation} (\(i_r\)), à partir desquels
serons construites les \emph{zones de localisation compatibles} durant
la phase de \emph{spatialisation.} Comme on peut le voir sur la
\autoref{fig:methodo_1}, les indices issus de la décomposition
effectuée lors de la première étape du processus sont fusionnés lors
de la dernière étape de la \emph{phase de fusion.} Ainsi les
différents \emph{indices de localisation} sont traités indépendamment,
conformément au principe de modélisation précédemment mentionné.

\subsubsection{La \emph{décomposition} des \emph{objets de référence indéfinis}}

Comme nous l'avons indiqué lors de leur formalisation, un \emph{indice
  de référence} peut contenir de nombreux \emph{objets de référence}
différents. Si \emph{l'indice de localisation} qui nous est transmis
contient plus d'un objet \footnote{Ou plus d'un n-uplet, dans le cas
  où la relation modélisée est bi ou n-aire.}, alors \emph{l'objet de
  référence} est indéfini et \emph{l'indice de localisation} doit être
\emph{spatialisé} pour chacun d'entre eux.

L'objectif de la \emph{décomposition} des \emph{objets de référence
  indéfinis} est donc de transformer un \emph{indice de localisation}
comportant plusieurs \emph{objets de référence} concurrents en un
ensemble \emph{d'indices de localisation} contenant chacun un seul
\emph{objet de référence} (ou un seul n-uplet), c'est-à-dire une des
instances de l'ensemble des \emph{objets de référence possibles.}
L'étape de décomposition des \emph{objets de référence indéfinis} peut
être formalisée de la manière suivante :

\begin{equation}
  Dec_{Or}(i) = \{(S,Rl,or_i) \mid S \in , Rl \in , \forall or_i \in Or \} 
\end{equation}

Avec \(Dec_{Or}(i)\) la fonction de \emph{décomposition} de l'ensemble
des \emph{objets de référence,} \(S\) le \emph{sujet} de \emph{l'indice
  de localisation} décomposé, \(Rl\) son ensemble de \emph{relations
  de localisation,} \(Or\) l'ensemble initial des \emph{objets de
  référence} et \(or_i\) un objet de référence donné, appartenant
initialement à \(Or\).

\subsubsection{La \emph{décomposition} des \emph{relations de localisation}}

La troisième et dernière étape de cette phase est la
\emph{décomposition} des \emph{relations de localisation.} C'est
durant cette étape qu'est appliqué le \emph{principe de décomposition}
des \emph{relations de localisation} que nous avons déjà
présenté. Comme nous l'avons expliqué, cette étape consiste à
décomposer chaque \emph{relation de localisation} en un ensemble de
composantes sémantiquement indépendantes, les \emph{relations de
  localisation atomiques.} Comme pour les deux précédentes étapes,
cette décomposition n'est pas toujours nécessaire, \emph{la relation
  de localisation utilisée} pouvant déjà être atomique (\eg
\enquote{proche}). Cette étape peut être formalisée de la manière
suivante :

\begin{equation}
  Dec_{Rl}(i) = \{(S,Rla_i,Or) \mid S \in, \forall Rla_i \in Rl,  Or \in \} 
\end{equation}

Avec \(Dec_{Rl}(i)\) la fonction de \emph{décomposition} des
\emph{relations de localisation,} \(S\) le \emph{sujet} de
\emph{l'indice de localisation} décomposé, \(Rl\) la \emph{relation de
  localisation} et \(Rla_i\) une des relations de \emph{localisation
  atomiques décomposant} la \emph{relation} \(Rl\) et \(Or\)
l'ensemble initial des \emph{objets de référence.}

La principale différence entre cette étape et les précédentes est
qu'il est nécessaire de disposer d'une information supplémentaire pour
savoir comment l'effectuer. En effet, lorsqu'un ensemble d'indice de
localisation nous est transmis, les différents \emph{indices de
  localisation} qui le composent sont déjà identifiés, puisque saisis
tels quels par le secouriste, de la même manière que les différents
\emph{objets de référence,} mais contrairement à la
\emph{décomposition des relations de localisation.} En effet, notre
souhait est de permettre au secouriste de sélectionner la
\emph{relation de localisation} correspondant à la description du
requérant à partir d'une liste pré-définie de \emph{relations de
  localisation} (cf. principe de modélisation explicite des
connaissances). Ainsi, le secouriste n'a pas à manipuler directement
les \emph{relations de localisation atomiques.} Il est donc nécessaire
de connaître la \emph{décomposition des relations de localisation} et
de s'y référer lors de cette étape. Ce processus sera détaillé plus
longuement dans le \autoref{chap:05}.

La \emph{phase de décomposition} consistant en l’application
successive de la fonction de \emph{décomposition} des \emph{objets de
  référence indéfinis} et de la fonction de décomposition des
\emph{relations de localisation} à tous les \emph{indices de
  localisation} contenu dans l'ensemble \(I\), peut donc être
formalisée de la manière suivante :

\begin{equation}
  Dec(I) = \{ i \mid i \in Dec_{Rl}(j), j \in  Dec_{Or}(k), k \in I\} 
\end{equation}

Avec \(Dec(I)\) la fonction de décomposition, \(Dec_{Rl}(j)\) la
fonction de \emph{décomposition des relations de localisation,}
\(Dec_{Or}(k)\) la fonction de \emph{décomposition des objets de
  référence indéfinis }et \(I\) l'ensemble des \emph{indices de
  localisation.}

\subsubsection{Illustration de la phase de décomposition}
\label{subsec:4-3-3-4}

Pour illustrer l'ensemble de la \emph{phase de décomposition,}
imaginons que \emph{l'ensemble des indices de localisation} (\(I\)
saisi par un secouriste à partir de la description suivante :
\enquote{je suis proche du \emph{Pic de Jean Ray} et suis sur une
  crête}. Ce dernier est composé de deux \emph{indices de
  localisation,} séparés (par le secouriste) lors de la première étape
de la \emph{phase de décomposition.} Les deux \emph{indices de
  localisation :} \enquote{je suis proche du \emph{Pic de Jean Ray}}
et \enquote{je suis sous une crête} peuvent être alors traités
indépendamment :

\begin{quote}
  \begin{tikzpicture}[%
    grow via three points={one child at (0.5,-0.7) and
      two children at (0.5,-0.7) and (0.5,-1.4)},
    edge from parent path={(\tikzparentnode.south) |- (\tikzchildnode.west)}
    ]
    \node[anchor=south west,text=RdBu-9-1] {Et} child
    {node[anchor=south west] {\enquote{je suis proche du \emph{Pic de
            Jean Ray}}}} child {node[anchor=south west] {\enquote{je
          suis sous une crête}} };
  \end{tikzpicture}
\end{quote}

La seconde étape de la \emph{phase de décomposition} fragmente chaque
\emph{indice de localisation} en fonction des \emph{objets de
  référence.} Le premier \emph{indice de localisation} (\enquote{je
  suis proche du \emph{Pic de Jean Ray}}) se réfère à un objet nommé
et unique, il n'est donc pas nécessaire de le décomposer,
contrairement au second \emph{indice de localisation} pour lequel
l'ensemble des objets de type \enquote{crête} situés dans la \ac{zir}
\footnote{Voir dans un espace plus réduit si le secouriste décide,
  compte-tenu de ses connaissances et de la situation, de réduire la
  zone de recherche pour cette catégorie d'objets.} peuvent
correspondre. Admettons que six crêtes aient été sélectionnées :
\emph{l'indice de localisation} qui nous sera transmis sera alors de
la forme \enquote{je suis sous la \emph{crête de Roche Motte} ou je
  suis sous la \emph{crête de Font Froide} ou je suis sous la
  \emph{crête de Serre Chapelle} ou je suis sous la \emph{crête des
    Barres} ou je suis sous la \emph{crête du Petit Puy}}. Ainsi, au
sortir de cette seconde étape de décomposition, on dispose de \(n\)
\emph{indices de localisation} (première décomposition), eux-mêmes
divisés en plusieurs \emph{indices de localisation,} en fonction du
nombre \emph{d'objets de référence} à prendre en compte :

\begin{quote}
  \begin{tikzpicture}[%
    grow via three points={one child at (0.5,-0.7) and
      two children at (0.5,-0.7) and (0.5,-1.4)},
    edge from parent path={(\tikzparentnode.south) |- (\tikzchildnode.west)}
    ]
    \node[anchor=south west,text=RdBu-9-1] {Et}
    child {node[anchor=south west] {\enquote{je suis proche du \emph{Pic
            de Jean Ray}}}}
    child {node[anchor=south west, text=RdBu-9-9] {Ou}
      child{node[anchor=south west] {\enquote{je suis sous la
            \emph{crête de Roche Motte}}}}
      child{node[anchor=south west]{\enquote{je suis
            sous la \emph{crête de Font Froide}}} 
      }
      child{node[anchor=south west]{\enquote{je suis
            sous la \emph{crête de Serre Chapelle}}} 
      }
      child{node[anchor=south west]{\enquote{je suis
            sous la \emph{crête des Barres}}}  
      }
      child{node[anchor=south west]{\enquote{je suis
            sous la \emph{crête du Petit Puy}}}} 
    };
  \end{tikzpicture}
\end{quote}

Enfin, la dernière étape de la phase de \emph{décomposition,}
nécessite la définition préalable d'un ensemble de \emph{relations de
  localisation atomiques,} ce qui sera abordé dans le
\autoref{chap:05}. Cependant, pour aller au bout de cet exemple, nous
allons considérer que la relation de localisation \enquote{proche} est
une \emph{relation spatiale atomique} (hypothèse déjà utilisée
précédemment) et que la relation de localisation \enquote{sous} se
décompose en deux \emph{relations spatiales atomiques,}
\emph{proximité} et \emph{différence d'altitude négative}
\footnote{Comme nous le verrons ultérieurement cette décomposition est
  fortement simplificatrice. Cet exemple n'est donc pas à considérer
  comme représentatif de la décomposition de la \emph{relation de
    localisation} \enquote{sous}, mais bien comme une illustration de
  la phase de décomposition dans son ensemble.}. Ainsi, le premier
\emph{indice de localisation} n'est pas décomposé, puisque nous
considérons que la \emph{relation de localisation} \enquote{proche}
est atomique, contrairement à \emph{l'ensemble des indices de
  localisation} issus de la décomposition de l'indice : \enquote{je
  suis sous une crête}, qui voient tous leur relation de localisation
(\enquote{sous}) décomposée. Ainsi, au terme de la \emph{phase de
  décomposition,} l'ensemble des \emph{indices de localisation :}
\enquote{je suis proche du \emph{Pic de Jean Ray}} et \enquote{je suis
  sur une crête} devient :

\begin{quote}
  \begin{tikzpicture}[%
    grow via three points={one child at (0.5,-0.7) and
      two children at (0.5,-0.7) and (0.5,-1.4)},
    edge from parent path={(\tikzparentnode.south) |- (\tikzchildnode.west)}
    ]
    \node[anchor=south west,text=RdBu-9-1] {Et}
    child {node[anchor=south west] {\enquote{je suis proche du \emph{Pic
            de Jean Ray}}}}
    child {node[anchor=south west, text=RdBu-9-9] {Ou}
      child{node[anchor=south west, text=RdBu-9-1] {Et}
        child{node[anchor=south west]{\enquote{je suis proche de la
              \emph{crête de Roche Motte}}}} 
        child{node[anchor=south west]{\enquote{je suis à une altitude
              inférieure à la \emph{crête de Roche Motte}}}}
      }
      child[missing]{}
      child{node[anchor=south west,text=RdBu-9-1]{Et}
        child{node[anchor=south west]{\enquote{je suis proche de la
              \emph{crête de Font Froide}}}} 
        child{node[anchor=south west]{\enquote{je suis à une altitude
              inférieure à la \emph{crête de Font Froide}}}} 
      }
      child[missing]{}
      child{node[anchor=south west,text=RdBu-9-1]{Et}
        child{node[anchor=south west]{\enquote{je suis proche de la
              \emph{crête de Serre Chapelle}}}} 
        child{node[anchor=south west]{\enquote{je suis à une altitude
              inférieure à la \emph{crête de Serre Chapelle}}}} 
      }
      child[missing]{}
      child{node[anchor=south west,text=RdBu-9-1]{Et}
        child{node[anchor=south west]{\enquote{je suis proche de la
              \emph{crête des Barres}}}} 
        child{node[anchor=south west]{\enquote{je suis à une altitude
              inférieure à la \emph{crête des Barres}}}} 
      }
      child[missing]{}
      child{node[anchor=south west,text=RdBu-9-1]{Et}
        child{node[anchor=south west]{\enquote{je suis proche de la
              \emph{crête du Petit Puy}}}} 
        child{node[anchor=south west]{\enquote{je suis à une altitude
              inférieure à la \emph{crête du Petit Puy}}}} 
      }
    };
  \end{tikzpicture}
\end{quote}

Comme le montre cette longue énumération, \emph{le processus de
  décomposition,} même lorsqu'il est appliqué à un petit
\emph{ensemble d'indices de localisation} (ici deux), peut générer de
nombreux \emph{indices de localisation décomposés,} qu'il est
difficile d'énumérer. Nous sommes cependant convaincu que les apports
permis par cette approche (\eg spatialisation indépendante,
décomposition sémantique) compensent la multiplication des
\emph{indices de localisation.}

\subsection{La phase de \emph{spatialisation}}

Une fois que les \emph{indices de localisation} ont été entièrement
décomposés il est possible de procéder à leur \emph{spatialisation.}
Comme nous l'avons expliqué précédemment, la phase de
\emph{spatialisation} consiste à transformer un \emph{indice de
  localisation} décomposé en une \emph{zone de localisation
  compatible.}
% 
Il s'agit de la seule étape de notre méthode qui ne modifie pas le
nombre d'objets traité (la phase de \emph{décomposition} les augmente
et celle de \emph{fusion} les réduit) et qui crée un nouveau type de
données : des objets géographiques sous la forme de \emph{zones de
  localisation compatibles.}
%
Pour construire la \emph{zone de localisation compatible}
correspondant à un \emph{indice de localisation} donné, il est
nécessaire d'identifier les positions pour lesquelles \emph{l'indice
  de localisation} est vrai. On peut donc imaginer une fonction
\(\text{\textsf{S}}_{Rla}\) destinée à la \emph{spatialisation} d'une
\emph{relation de localisation atomique} donnée. Cette fonction, que
l'on peut définir de la manière suivante :
%
\begin{equation}
  \text{\textsf{S}}_{Rla}(p, o) = \left \{
    \begin{array}{l@{\,}l}
      1\ \text{si}\ p\ \text{est une \enquote{bonne position}}\\
      0\ \text{sinon}\\
    \end{array}
  \right.
\end{equation}
%
revoie une valeur booléenne pour chaque position \(p\) indiquant si
cette position valide la \emph{relation de localisation atomique}
\(Rla\) par rapport à l'objet de référence \(o\). On peut alors
définir la \emph{zone de localisation compatible} issue de la
\emph{spatialisation} d'un \emph{indice de localisation} décomposé
\(i\) comme l'ensemble des positions \(p\) appartenant à la \emph{zone
  initiale de recherche} et pour lesquelles la fonction \(S_{rla}\)
est vraie :

\begin{equation}
  \text{\textsf{ZLC}}_i = \{p \mid \text{\textsf{S}}_{Rla}(p, o) = 1 \wedge p
  ∈ \text{\textsf{ZIR}} \wedge o ∈ i\}
\end{equation}

Avec \(\text{\textsf{ZLC}}_i\) la \emph{zone de localisation
  compatible} pour l'indice de localisation $i$, \(S_{Rla}\) la
fonction de \emph{spatialisation} de ce même indice et \(p\) un point
appartenant à la \emph{zone initiale de recherche,}
\(\text{\textsf{ZIR}}\), elle-même un sous-ensemble de
\(\mathbb{R}^2\), défini par le secouriste. Toute la difficulté de la
phase de \emph{spatialisation} tient en la définition de cette
fonction de spatialisation \(S_{Rla}\). La méthode de définition d'une
telle fonction sera présentée dans le \autoref{chap:07}.

Au terme de la phase de \emph{spatialisation} on dispose d'un grand
ensemble de \emph{zones de localisation compatible}
(\(\text{\textsf{ZLC}}\)), comptant autant d'éléments qu'il y a
d'indices au terme de la \emph{phase de décomposition.} Cet ensemble
est définissable de la manière suivante :

\begin{equation}
  \text{\textsf{ZLC}} = \{\text{\textsf{ZLC}}_i \mid i \in Dec(I)\}
\end{equation}

Avec \(\text{\textsf{ZLC}}_i\) la \emph{zone de localisation
  compatible} issue de la \emph{spatialisation} de \emph{l'indice de
  localisation décomposé} \(i\), \(Dec\) la fonction de décomposition
et \(I\) l'ensemble des \emph{indices de localisation.} Ces
différentes \emph{zones de localisation} contiennent l'ensemble des
informations produites par notre méthode, mais leur (potentiel) grand
nombre rend leur exploitation directe, sinon impossible, du moins
délicate, il est donc nécessaire de les fusionner pour obtenir une
seule zone, la \ac{zlp}.

\subsection{La phase de fusion}

La \emph{phase de fusion} des \emph{zones de localisation} se
décompose en trois étapes, chacune analogue à l'une des étapes de la
\emph{phase de décomposition,} comme indiqué par des traits pointillés
sur la \autoref{fig:methodo_1}. Cette phase permet de fusionner les
\emph{zones de localisation compatibles} créées lors de la
\emph{spatialisation} en une seule, la zone de localisation probable,
correspondant à l'ensemble des indices de localisation. La sémantique
de chaque étape de fusion et l'approche proposée seront détaillées
dans le \autoref{chap:08}

De toutes les étapes de notre méthodologie, la phase de \emph{fusion}
est, de loin, la plus simple. Son objectif est de combiner les
différentes \emph{zones de localisation compatibles} créées lors de la
phase de \emph{spatialisation} en une seule, la \emph{zone de
  localisation probable} (\ac{zlp}).

\subsubsection{La \emph{fusion} des \emph{relations de localisation
    atomiques}}

La première étape de la \emph{phase de fusion} est la \emph{fusion}
des \emph{zones de localisation compatibles} issues de la
\emph{spatialisation.} Cette étape répond donc à l'étape de
\emph{décomposition des relations de localisation} de \emph{la phase
  de décomposition,} c'est-à-dire que le nombre de \emph{zones de
  localisation} en résultant est identique au nombre \emph{d'indices
  de localisation} avant la \emph{décomposition des relations de
  localisation} (\autoref{fig:methodo_1}).

Conformément au \emph{principe de décomposition,} les \emph{relations
  de localisation} se décomposent en une série de \emph{relations de
  localisation atomiques} liées par une relation de conjonction. Par
conséquent, pour valider une \emph{relation de localisation} il est
nécessaire de valider toutes les \emph{relations de localisation
  atomiques} qui la composent. La méthode de \emph{fusion} des
\emph{relations de localisation atomiques} ne doit donc retourner que
les positions présentes dans toutes les \emph{zones de localisations}
qu'elle fusionne. Comme chaque \emph{zone de localisation} peut être
formalisée comme l'ensemble des points validant une \emph{relation de
  localisation} on peut construire la \emph{zone de localisation
  compatible} correspondant à une \emph{relation de localisation}
composée en intersectant les \emph{zones de localisation compatibles}
correspondant à la \emph{spatialisation} des \emph{relations de
  localisation atomiques.} Ainsi, la \emph{zone de localisation}
correspondant à la fusion de plusieurs \emph{relations de
  localisation} correspond à :

\begin{equation}
  \text{\textsf{ZLC}}_{o_i} = ⋂ \text{\textsf{ZLC}}_{{Rla}_j}
\end{equation}

Avec \(\text{\textsf{ZLC}}_{o_i}\), la \emph{zone de localisation
  compatible} correspondant à \emph{l'objet de référence} \(oᵢ\) et
\(\text{\textsf{ZLC}}_j\) la \emph{zone de localisation compatible}
correspondant à la \emph{spatialisation,} pour l'objet de référence
\(o_i\), de la \emph{relation de localisation atomique} \(Rla_j\).

\subsubsection{La \emph{fusion} des \emph{objets de référence
    indéfinis}}

La seconde étape de cette phase est la \emph{fusion} des différents
\emph{objets de référence} candidats. Cette étape répond à
\emph{l'étape de décomposition des objets de référence indéfinis.}
Comme nous l'indiquions lors de la présentation de la \emph{phase de
  décomposition,} les étapes de \emph{décomposition des relations de
  localisation} et de décomposition des \emph{objets de référence
  indéfinies} sont interchangeables et il en va de même pour les
étapes de \emph{fusion} leur correspondant. Ainsi cette étape de
fusion pourrait être interchangée avec l'étape, précédemment
présentée, de \emph{fusion des relations de localisation atomiques,}
sans que cela n'impacte le résultat de la modélisation.

La \emph{fusion des objets de référence indéfinis} se distingue des
autres étapes de \emph{fusion} sa forme disjonctive. Comme nous
l'avons indiqué lors de la présentation de \emph{l'étape de
  décomposition des objets de référence indéfinis,} les différents
\emph{objets de référence} candidats (pour un même indice de
localisation) sont concurrents. Ainsi pour que \emph{l'indice de
  localisation spatialisé} soit valable en une position il est
nécessaire qu'il soit vrai pour au moins un des \emph{objets de
  référence} et non pour tous (comme c'est le cas lors de la
\emph{fusion} des \emph{relations de localisation atomiques}). La
\emph{zone de localisation compatible} correspondant à un \emph{indice
  de localisation} dont \emph{l'objet de référence} est indéfini peut,
par conséquent, être construite en faisant l'union de l'ensemble des
\emph{zones de localisation compatibles} construites en
\emph{spatialisant} l'indice de localisation pour chaque objet de
référence candidat :

\begin{equation}
  \text{\textsf{ZLC}}_{i} = ⋃_{i \in I} \text{\textsf{ZLC}}_{o_i}
\end{equation}

Avec \(\text{\textsf{ZLC}}_{i}\) la \emph{zone de localisation
  compatible} correspondant à un \emph{indice de localisation} \(i\)
et \(\text{\textsf{ZLC}}_{o_i}\) la \emph{zone de localisation
  compatible} pour l'objet \(i\) de l'indice de localisation \(i\).

\subsubsection{La \emph{fusion} des \emph{indices de localisation}}

La dernière étape de \emph{la phase de fusion} est celle de la
\emph{fusion} des \emph{indices de localisation} en vue de construire
la \emph{zone de localisation probable,} \ie le résultat final de
notre méthode. Contrairement aux \emph{zones de localisation
  compatibles,} la \emph{zone de localisation probable} est toujours
unique, même si elle peut être fragmentée ou vide (mais ce cas traduit
une erreur dans la description de la position, dans sa saisie ou sa
\emph{spatialisation}).

Comme lors de l'étape de \emph{fusion} des \emph{relations de
  localisation atomiques,} les différentes \emph{zones de localisation
  compatibles} sont \emph{fusionnées} à l'aide d'une intersection
ensembliste. En effet, la \emph{zone de localisation probable} est
définie comme l'ensemble des positions appartenant à toutes les
\emph{zones de localisation compatibles.} On peut donc exprimer la
zone de localisation probable de la manière suivante :

\begin{equation}
  \text{\textsf{ZLP}}_I = ⋂_{i \in I} \text{\textsf{ZLC}}ᵢ
\end{equation}

Avec \textsf{ZLP} la \emph{zone de localisation probable,} \(I\),
l'ensemble des \emph{indices de localisation} pris en compte, \(i\) un
\emph{indice de localisation} et \textsf{ZLP} un \emph{zone de
  localisation compatible.}

\subsubsection{Illustration de la \emph{phase de fusion}}

On peut illustrer l'ensemble de la \emph{phase de fusion} à l'aide de
l'exemple également utilisé pour décrire la \emph{phase de
  décomposition} (\autoref{subsec:4-3-3-4}). Au terme de la
\emph{phase de décomposition} on dispose de onze \emph{indices de
  localisation} décomposés. Durant la phase de \emph{spatialisation}
une \ac{zlc} est construite, indépendamment, pour chacun
d'entre-eux. Au début de la phase de fusion on dispose donc de onze
\ac{zlc} (comme le montre la partie supérieure de la
\autoref{fig:ex_fus_rsa}) qu'il va falloir fusionner de sorte à
obtenir une \ac{zlp}.

La \autoref{fig:ex_fus_rsa} propose une illustration de la première
étape de la phase de \emph{fusion,} la fusion des \emph{relations de
  localisation atomiques.}  L'objectif de cette étape est de regrouper
les \emph{zones} issues de la \emph{spatialisation} \emph{d'indices de
  localisation} créés suite à la décomposition d'une \emph{relation de
  localisation.} Les \emph{zones} spatialisées à partir
\emph{d'indices} issus de la même décomposition sont alors fusionnées
indépendamment à l'aide d'un opérateur d'intersection permettant de
construire une nouvelle \emph{zone de localisation} ne contenant que
les positions appartenant aux deux \emph{zones de localisations}
fusionnées.  si l'on reprend l'exemple précédent on obtiendra, après
la \emph{fusion} des \emph{relations de localisation atomiques,} sept
\emph{zones de localisation compatibles,} correspondant aux
\emph{indices de localisation :} \enquote{je suis proche du \emph{Pic
    de Jean Ray}}, \enquote{je suis sur la \emph{crête de Roche
    Motte}}, \enquote{je suis sur la \emph{crête de Font Froide}},
\enquote{je suis sur la \emph{crête de Serre Chapelle}}, \enquote{je
  suis sur la \emph{crête des Barres}}, et \enquote{je suis sur la
  \emph{crête du Petit Puy}}.

\begin{figure}
  \centering
  \begin{tikzpicture}
  \matrix [matrix of nodes,
  anchor=west,
  nodes={minimum size=.75cm}
  ] (zla0) at (0,0)
  {
    \draw[ffa,ffc] (0,0) circle (.3cm);\\
  };

  \matrix [matrix of nodes,
  anchor=west,
  nodes={minimum size=.75cm},
  right=.75cm of zla0
  ] (zla1)
  {
    \draw[ffa,ffc] (0,0) circle (.3cm);&
    \draw[ffa,ffc] (0,0) circle (.3cm);\\
  };

  \matrix [matrix of nodes,
  anchor=west,
  nodes={minimum size=.75cm},
  right=.75cm of zla1
  ] (zla2)
  {
    \draw[ffa,ffc] (0,0) circle (.3cm);&
    \draw[ffa,ffc] (0,0) circle (.3cm);\\
  };
  \matrix [matrix of nodes,
  anchor=west,
  nodes={minimum size=.75cm},
  right=.75cm of zla2
  ] (zla3)
  {
    \draw[ffa,ffc] (0,0) circle (.3cm);&
    \draw[ffa,ffc] (0,0) circle (.3cm);\\
  };
  \matrix [matrix of nodes,
  anchor=west,
  nodes={minimum size=.75cm},
  right=.75cm of zla3
  ] (zla4)
  {
    \draw[ffa,ffc] (0,0) circle (.3cm);&
    \draw[ffa,ffc] (0,0) circle (.3cm);\\
  };
  \matrix [matrix of nodes,
  anchor=west,
  nodes={minimum size=.75cm},
  right=.75cm of zla4
  ] (zla5)
  {
    \draw[ffa,ffc] (0,0) circle (.3cm);&
    \draw[ffa,ffc] (0,0) circle (.3cm);\\
  };


  \matrix [matrix of nodes,
  anchor=north,
  nodes={minimum size=.75cm},
  below=2cm of zla0
  ] (zlb0)
  {
    \draw[ffa,ffc] (0,0) circle (.3cm);\\
  };

  \matrix [matrix of nodes,
  anchor=north,
  nodes={minimum size=.75cm},
  below=2cm of zla1
  ] (zlb1)
  {
    \draw[ffa,ffc] (0,0) circle (.3cm);\\
  };

  \matrix [matrix of nodes,
  anchor=north,
  nodes={minimum size=.75cm},
  below=2cm of zla2
  ] (zlb2)
  {
    \draw[ffa,ffc] (0,0) circle (.3cm);\\
  };
  \matrix [matrix of nodes,
  anchor=north,
  nodes={minimum size=.75cm},
  below=2cm of zla3
  ] (zlb3)
  {
    \draw[ffa,ffc] (0,0) circle (.3cm);\\
  };
  \matrix [matrix of nodes,
  anchor=north,
  nodes={minimum size=.75cm},
  below=2cm of zla4
  ] (zlb4)
  {
    \draw[ffa,ffc] (0,0) circle (.3cm);\\
  };
  \matrix [matrix of nodes,
  anchor=north,
  nodes={minimum size=.75cm},
  below=2cm of zla5
  ] (zlb5)
  {
    \draw[ffa,ffc] (0,0) circle (.3cm);\\
  };


  % Accolades
  \foreach \m in {0,1,...,5} {
    \draw (zla\m.south west) |- ($(zla\m.south west)!0.5!(zla\m.south
    east) + (0,-.1)$) -| (zla\m.south east)  node[pos=0, yshift=.2]
    (zla\m-g) {};
    % 
    \draw (zlb\m.north west) |- ($(zlb\m.north west)!0.5!(zlb\m.north
    east) + (0,.1)$) -| (zlb\m.north east)  node[pos=0, yshift=.2]
    (zlb\m-g) {};
    % 
    \draw[->>, black,] ([yshift=-.2cm]zla\m.south) -- ([yshift=.2cm]zlb\m.north);
  }

\draw[|-, black] ([yshift=1.25cm]zla1-1-2.north) -- (zla1-1-2.north);
\draw[|-, black] ([yshift=1.25cm]zla2-1-2.north) -- (zla2-1-2.north);
\draw[|-, black] ([yshift=1.25cm]zla3-1-2.north) -- (zla3-1-2.north);
\draw[|-, black] ([yshift=1.25cm]zla4-1-2.north) -- (zla4-1-2.north);  
\draw[|-, black] ([yshift=1.25cm]zla5-1-2.north) -- (zla5-1-2.north);

  
  \node[above=0cm of zla0.north, anchor=south, align=center,
  font=\tiny, text width=1.75cm]{\enquote{Je suis proche du \emph{Pic
        Jean Ray}}};
  
  \node[above=0cm of zla1-1-1.north, anchor=south, align=center,
  font=\tiny, text width=1.75cm, fill=white]{\enquote{Je suis proche de la \emph{crête de
        Roche Motte}}};
    \node[above=1.25cm of zla1-1-2.north, anchor=south, align=center,
  font=\tiny, text width=2.5cm]{\enquote{Je suis à une altitude
      supérieure à la \emph{crête de
        Roche Motte}}};
  
  \node[above=0cm of zla2-1-1.north, anchor=south, align=center,
  font=\tiny, text width=1.75cm, fill=white]{\enquote{Je suis proche de la \emph{crête de Font
        Froide}}};
    \node[above=1.25cm of zla2-1-2.north, anchor=south, align=center,
  font=\tiny, text width=2.5cm]{\enquote{Je suis à une altitude
      supérieure à la \emph{crête de Font
        Froide}}};
  
  \node[above=0cm of zla3-1-1.north, anchor=south, align=center,
  font=\tiny, text width=1.75cm, fill=white]{\enquote{Je suis proche de la \emph{crête de
        Serre Chapelle}}};
    \node[above=1.25cm of zla3-1-2.north, anchor=south, align=center,
  font=\tiny, text width=2.5cm]{\enquote{Je suis à une altitude
      supérieure à la \emph{crête de
        Serre Chapelle}}};
  
  \node[above=0cm of zla4-1-1.north, anchor=south, align=center,
  font=\tiny, text width=1.75cm, fill=white]{\enquote{Je suis proche de la \emph{crête
        des Barres}}};
    \node[above=1.25cm of zla4-1-2.north, anchor=south, align=center,
  font=\tiny, text width=2.5cm]{\enquote{Je suis sous la \emph{crête des Barres}}};
  
  \node[above=0cm of zla5-1-1.north, anchor=south, align=center,
  font=\tiny, text width=1.75cm, fill=white]{\enquote{Je suis proche de la \emph{crête du
        Petit Puy}}};
    \node[above=1.25cm of zla5-1-2.north, anchor=south, align=center,
  font=\tiny, text width=2.5cm]{\enquote{Je suis à une altitude
      supérieure à la \emph{crête du Petit Puy}}};


  
  \node[below=0cm of zlb0.south, anchor=north, align=center,
  font=\tiny, text width=1.75cm]{\enquote{Je suis proche du \emph{Pic
        Jean Ray}}};
  
  \node[below=0cm of zlb1.south, anchor=north, align=center,
  font=\tiny, text width=1.75cm]{\enquote{Je suis  la \emph{crête de
        Roche Motte}}};
  
  \node[below=0cm of zlb2.south, anchor=north, align=center,
  font=\tiny, text width=1.75cm]{\enquote{Je suis sous la \emph{crête de Font
        Froide}}};
  
  \node[below=0cm of zlb3.south, anchor=north, align=center,
  font=\tiny, text width=1.75cm]{\enquote{Je suis sous la \emph{crête de
        Serre Chapelle}}};
  
  \node[below=0cm of zlb4.south, anchor=north, align=center,
  font=\tiny, text width=1.75cm]{\enquote{Je suis sous la \emph{crête des Barres}}};
  
  \node[below=0cm of zlb5.south, anchor=north, align=center,
  font=\tiny, text width=1.75cm]{\enquote{Je suis sous la \emph{crête du
        Petit Puy}}};

    \node[fill=white,align=center, font=\large\sffamily] at ($(zla0.south west)!0.5!(zlb5.north
    east)$){\emph{Fusion} des \emph{relations de localisation atomiques}};
\end{tikzpicture}
  \caption{Illustration de l'étape de \emph{fusion des relations de
      localisation atomiques}}
  \label{fig:ex_fus_rsa}
\end{figure}

L'étape de fusion des objets de référence indéfinis est, quant à elle,
illustrée par la \autoref{fig:ex_fus_obj}. Comme précédemment on
regroupe les \emph{zones de localisation} en fonction de la
décomposition dont elles sont issues et on les combine en parallèle
avec un opérateur ensembliste, ici l'union. Les \emph{zones de
  localisation} résultantes sont alors composées des positions
appartenant à, au moins, une des zones fusionnées. Au terme de cette
étape de \emph{fusion,} on obtient donc, deux \emph{zones de
  localisation compatibles.} La première (inchangée depuis la
\emph{spatialisation}) correspond à \emph{l'indice de localisation}
\enquote{je suis proche du \emph{Pic de Jean Ray}}. Comme
\emph{l'objet de référence} est unique aucune \emph{fusion} n'est à
effectuer. La zone correspondant au second \emph{indice de
  localisation,} \enquote{je suis sur une crête}, correspond, quant à
elle, à l'union des \emph{zones de localisation compatibles}
correspondant à chacune des une des crêtes candidates.

\begin{figure}
  \centering
  \begin{tikzpicture}
  \matrix [matrix of nodes,
  anchor=west,
  nodes={minimum size=.75cm}
  ] (zla0) at (0,0)
  {
    \draw[ffa,ffc] (0,0) circle (.3cm);\\
  };

  \matrix [matrix of nodes,
  anchor=west,
  nodes={minimum size=.75cm},
  right=.75cm of zla0
  ] (zla1)
  {
    \draw[ffa,ffc] (0,0) circle (.3cm);&
    \draw[ffa,ffc] (0,0) circle (.3cm);&
    \draw[ffa,ffc] (0,0) circle (.3cm);&
    \draw[ffa,ffc] (0,0) circle (.3cm);&
    \draw[ffa,ffc] (0,0) circle (.3cm);\\
  };

  \matrix [matrix of nodes,
  anchor=north,
  nodes={minimum size=.75cm},
  below=2cm of zla0
  ] (zlb0)
  {
    \draw[ffa,ffc] (0,0) circle (.3cm);\\
  };

  \matrix [matrix of nodes,
  anchor=north,
  nodes={minimum size=.75cm},
  below=2cm of zla1
  ] (zlb1)
  {
    \draw[ffa,ffc] (0,0) circle (.3cm);\\
  };

 
  % Accolades
  \foreach \m in {0,1} {
    \draw (zla\m.south west) |- ($(zla\m.south west)!0.5!(zla\m.south
    east) + (0,-.1)$) -| (zla\m.south east)  node[pos=0, yshift=.2]
    (zla\m-g) {};
    % 
    \draw (zlb\m.north west) |- ($(zlb\m.north west)!0.5!(zlb\m.north
    east) + (0,.1)$) -| (zlb\m.north east)  node[pos=0, yshift=.2]
    (zlb\m-g) {};
    % 
    \draw[->>, black,] ([yshift=-.2cm]zla\m.south) -- ([yshift=.2cm]zlb\m.north);
  }

\draw[|-, black] ([yshift=1.25cm]zla1-1-2.north) -- (zla1-1-2.north)
node[pos=.55,fill=white, minimum size=.75cm]{};

\draw[|-, black] ([yshift=2.5cm]zla1-1-3.north) -- (zla1-1-3.north)
node[pos=.3,fill=white, minimum size=1cm]{};

\draw[|-, black] ([yshift=1.25cm]zla1-1-4.north) -- (zla1-1-4.north)
node[pos=.55,fill=white, minimum size=.75cm]{};

  
  \node[above=0cm of zla0.north, anchor=south, align=center,
  font=\tiny, text width=1.75cm]{\enquote{Je suis proche du \emph{Pic
        Jean Ray}}};
  
  \node[above=0cm of zla1-1-1.north, anchor=south, align=center,
  font=\tiny, text width=1.75cm]{\enquote{Je suis sous la \emph{crête de
        Roche Motte}}};
    \node[above=1.25cm of zla1-1-2.north, anchor=south, align=center,
  font=\tiny, text width=1.75cm]{\enquote{Je suis sous la \emph{crête de
        Font Froide}}};
      \node[above=2.5cm of zla1-1-3.north, anchor=south, align=center,
  font=\tiny, text width=1.75cm]{\enquote{Je suis sous la \emph{crête de
        Serre Chapelle}}};
      \node[above=1.25cm of zla1-1-4.north, anchor=south, align=center,
  font=\tiny, text width=1.75cm]{\enquote{Je suis sous la \emph{crête des Barres}}};
      \node[above=0cm of zla1-1-5.north, anchor=south, align=center,
  font=\tiny, text width=1.75cm]{\enquote{Je suis sous la \emph{crête
        du Petit Puy}}};
  


    \node[below=0cm of zlb0.south, anchor=north, align=center,
  font=\tiny, text width=1.75cm]{\enquote{Je suis proche du \emph{Pic
        Jean Ray}}};
  
  \node[below=0cm of zlb1.south, anchor=north, align=center,
  font=\tiny, text width=1.75cm]{\enquote{Je sous une \emph{crête}}};
  

    \node[fill=white,align=center, font=\large\sffamily] at ($(zla1.south east)!0.5!(zlb0.north
    west)$){\emph{Fusion} des \emph{objets de référence}};
\end{tikzpicture}
  \caption{Illustration de l'étape de \emph{fusion des objets de
      référence indéfinis}}
  \label{fig:ex_fus_obj}
\end{figure}

la troisième et dernière étape de la \emph{phase de fusion} correspond
à la fusion des différents \emph{indices de localisation} et à la
construction de la \emph{zone de localisation probable}
(\autoref{fig:ex_fus_inf}). À la fin de la seconde étape de la phase
de fusion, on dispose d'une \emph{zone de localisation compatible}
pour chaque \emph{indice de localisation} contenu dans
\emph{l'ensemble des indices de localisation} \(I\)
(\autoref{fig:methodo_1}). Pour obtenir une seule zone à partir de ces
différentes \emph{zones de localisation compatibles} on utilise une
nouvelle fois un opérateur d'intersection qui permet de construire une
zone contenant les positions qui appartiennent à toutes les \ac{zlc}
fusionnées. Ainsi, dans notre exemple, la \emph{zone de localisation
  probable} correspond à \emph{l'intersection} de deux \emph{zones de
  localisation compatibles,} celle construite à partir de
\emph{l'indice de localisation} \enquote{je suis proche du \emph{Pic
    de Jean Ray}} et celle construite à partir de \emph{l'indice de
  localisation} \enquote{je suis sur une crête}.

\begin{figure}
  \centering
  \begin{tikzpicture}
  \matrix [matrix of nodes,
  anchor=west,
  nodes={minimum size=.75cm}
  ] (zla0) at (0,0)
  {
    \draw[ffa,ffc] (0,0) circle (.3cm);\\
  };

  \matrix [matrix of nodes,
  anchor=west,
  nodes={minimum size=.75cm},
  right=.75cm of zla0
  ] (zla1)
  {
    \draw[ffa,ffc] (0,0) circle (.3cm);&
    \draw[ffa,ffc] (0,0) circle (.3cm);&
    \draw[ffa,ffc] (0,0) circle (.3cm);&
    \draw[ffa,ffc] (0,0) circle (.3cm);&
    \draw[ffa,ffc] (0,0) circle (.3cm);&
    \draw[ffa,ffc] (0,0) circle (.3cm);\\
  };

  \matrix [matrix of nodes,
  anchor=north,
  nodes={minimum size=.75cm},
  below=2cm of zla0
  ] (zlb0)
  {
    \draw[ffa,ffc] (0,0) circle (.3cm);\\
  };

  \matrix [matrix of nodes,
  anchor=north,
  nodes={minimum size=.75cm},
  below=2cm of zla1
  ] (zlb1)
  {
    \draw[ffa,ffc] (0,0) circle (.3cm);\\
  };

 
  % Accolades
  \foreach \m in {0,1} {
    \draw (zla\m.south west) |- ($(zla\m.south west)!0.5!(zla\m.south
    east) + (0,-.1)$) -| (zla\m.south east)  node[pos=0, yshift=.2]
    (zla\m-g) {};
    % 
    \draw (zlb\m.north west) |- ($(zlb\m.north west)!0.5!(zlb\m.north
    east) + (0,.1)$) -| (zlb\m.north east)  node[pos=0, yshift=.2]
    (zlb\m-g) {};
    % 
    \draw[->>, black,] ([yshift=-.2cm]zla\m.south) -- ([yshift=.2cm]zlb\m.north);
  }

% \draw[|-, black] ([yshift=1.25cm]zla1-1-2.north) -- (zla1-1-2.north) node[pos=.85,fill=white]{};
% \draw[|-, black] ([yshift=1.25cm]zla2-1-2.north) -- (zla2-1-2.north) node[pos=.85,fill=white]{};
% \draw[|-, black] ([yshift=1.25cm]zla3-1-2.north) -- (zla3-1-2.north)
% node[pos=.55,fill=white, minimum size=.75cm]{};
% \draw[|-, black] ([yshift=1.25cm]zla4-1-2.north) -- (zla4-1-2.north);  
% \draw[|-, black] ([yshift=1.25cm]zla5-1-2.north) -- (zla5-1-2.north);

  
  \node[above=0cm of zla0.north, anchor=south, align=center,
  font=\tiny, text width=1.75cm]{\enquote{Je suis proche du \emph{Pic
        Jean Ray}}};
  
  \node[above=0cm of zla1-1-1.north, anchor=south, align=center,
  font=\tiny, text width=1.75cm]{\enquote{Je suis proche de la \emph{crête de
        Roche Motte}}};
    \node[above=1.25cm of zla1-1-2.north, anchor=south, align=center,
  font=\tiny, text width=2.25cm]{\enquote{Je suis à une altitude
      supérieure à la \emph{crête de
        Roche Motte}}};
      \node[above=1.25cm of zla1-1-3.north, anchor=south, align=center,
  font=\tiny, text width=2.25cm]{\enquote{Je suis à une altitude
      supérieure à la \emph{crête de
        Roche Motte}}};
      \node[above=1.25cm of zla1-1-4.north, anchor=south, align=center,
  font=\tiny, text width=2.25cm]{\enquote{Je suis à une altitude
      supérieure à la \emph{crête de
        Roche Motte}}};
      \node[above=1.25cm of zla1-1-5.north, anchor=south, align=center,
  font=\tiny, text width=2.25cm]{\enquote{Je suis à une altitude
      supérieure à la \emph{crête de
        Roche Motte}}};
      \node[above=1.25cm of zla1-1-6.north, anchor=south, align=center,
  font=\tiny, text width=2.25cm]{\enquote{Je suis à une altitude
      supérieure à la \emph{crête de
        Roche Motte}}};
  


    \node[below=0cm of zlb0.south, anchor=north, align=center,
  font=\tiny, text width=1.75cm]{\enquote{Je suis proche du \emph{Pic
        Jean Ray}}};
  
  \node[below=0cm of zlb1.south, anchor=north, align=center,
  font=\tiny, text width=1.75cm]{\enquote{Je sous une \emph{crête}}};
  

    \node[fill=white,align=center, font=\large\sffamily] at ($(zla1.south east)!0.5!(zlb0.north
    west)$){\emph{Fusion} des \emph{indices de localisation}};
\end{tikzpicture}
  \caption{Illustration de l'étape de fusion des \emph{indices de
      localisation}}
  \label{fig:ex_fus_inf}
\end{figure}


\begin{landscape}
  \begin{figure}[!h]
    \centering
    \begin{tikzpicture}
  \tikzset{
    acc/.style={decorate,decoration={brace,raise=0cm,amplitude=.1cm}},
    accm/.style={acc,decoration={mirror}},
    acc3/.style={acc,decoration={amplitude=.2cm}},
    acc3m/.style={acc3,decoration={mirror}}
  }
  
  %\draw[step=.5,black,thin] (0,0) grid (22,-14);

%% Matrices
\node[font=\bfseries, baseline] (I) at (0,0) {$I$};

\matrix [matrix of math nodes,row sep=0.1cm,column sep=0.1cm,
anchor=south west, nodes={anchor=base, baseline}] (is) at
([xshift=1.5cm]I.north east)
{i_1\\i_2\\\vdots{}\\i_n\\};

\matrix [matrix of math nodes,row sep=0.1cm,column sep=0.1cm,
anchor=south west, nodes={baseline}] (is_o) at
([xshift=1.5cm,yshift=.5cm]is-1-1.east) {o_1\\o_2\\\vdots{}\\o_n\\};

\matrix [matrix of math nodes,row sep=0.1cm,column sep=1.5cm,
anchor=south west, nodes={minimum height=5mm}] (is_o_r) at ([xshift=1.5cm,yshift=.5cm]is_o-1-1.east)
{
  \text{RSA}_1&\text{ZLC}_{rsa1}\\
  \text{RSA}_2&\text{ZLC}_{rsa_2}\\
  \vdots{}&\vdots{}\\
  \text{RSA}_n&\text{ZLC}_{rsa_n}\\};

\matrix [matrix of math nodes,row sep=0.1cm,column sep=0.1cm,
anchor=west, nodes={baseline}] (zlb) at ([xshift=1.5cm]is_o_r.east  |- is_o)
{ZLC_{o_1}\\ZLC_{o_2}\\\vdots{}\\ZLC_{o_n}\\};

\matrix [matrix of math nodes,row sep=0.1cm,column sep=0.1cm,
anchor=west, nodes={anchor=base, baseline}] (zlc) at ([xshift=1.5cm]zlb.east  |- is)
{ZLC_1\\ZLC_2\\\vdots{}\\ZLC_n\\};

\node [anchor=west] (zlp) at ([xshift=1.5cm]zlc.east |- I)
{ZLP};


%% Accolades 
\begin{scope}
  \draw[acc3m] (is.north west) -- (is.south west);
  \draw[acc3m] (is_o.north west) -- (is_o.south west);
  \draw[acc3m] (is_o_r.north west) -- (is_o_r.south west);

  \draw[acc] (is-1-1.north east) -- (is-1-1.south east);
  \draw[acc] (is_o-1-1.north east) -- (is_o-1-1.south east);
  \draw[acc] (is_o_r-1-1.north east) -- (is_o_r-1-1.south east);

  \draw[accm] (is_o_r-1-2.north west) -- (is_o_r-1-2.south west);
  \draw[accm] (zlb-1-1.north west) -- (zlb-1-1.south west);
  \draw[accm] (zlc-1-1.north west) -- (zlc-1-1.south west);

  \draw[acc3] (is_o_r.north east) -- (is_o_r.south east);
  \draw[acc3] (zlb.north east) -- (zlb.south east);
  \draw[acc3] (zlc.north east) -- (zlc.south east);
\end{scope}


\begin{scope}
  \draw[->, black] (I.east) -- (is.west) node[pos=0,below] {1} node[pos=1,below]{1..*};

  \draw[->, black] (is-1-1.east) -- (is_o.west) node[pos=0,below] {1} node[pos=1,below]{1..*}; 
  
  \draw[->, black] (is_o-1-1.east) --
  (is_o_r.west) node[pos=0,below]
  {1} node[pos=1,below]{1..*}; 
  
  \draw[->, black] (is_o_r-1-1.east) -- (is_o_r-1-2.west) node[pos=0,below]
  {1} node[pos=1,below]{1}; 
  
  \draw[->, black] (is_o_r.east) -- (zlb-1-1.west) node[pos=0,below] {1..*} node[pos=1,below]{1}; 
  
  \draw[->, black]
  (zlb.east) -- (zlc-1-1.west) node[pos=0,below] {1..*} node[pos=1,below]{1}; 
  
  \draw[->, black] (zlc.east) -- (zlp.west) node[pos=0,below] {1..*} node[pos=1,below]{1}; 
\end{scope}


\begin{scope}
    \draw[->, black, dotted] ([xshift=.5cm]I.east) -- ([xshift=-.5cm]zlp.west) node[pos=0,below]
  {1} node[pos=1,below]{1}; 

  \draw[->, black, dotted] ([xshift=.5cm]is.east) -- ([xshift=-.5cm]zlc.west) node[pos=0,below]
  {1} node[pos=1,below]{1}; 

  \draw[->, black, dotted] ([xshift=.5cm]is_o.east) -- ([xshift=-.5cm]zlb.west) node[pos=0,below]
  {1} node[pos=1,below]{1}; 
\end{scope}

%% XX

\begin{scope}[below of=I]
  \draw[acc3m] (0,0) --++ (8,0) node[pos=.5, yshift=-.5cm] {\emph{Décomposition}};
  \draw[acc3m] (8,0) --++ (3,0) node[pos=.5, yshift=-.5cm] {\emph{Spatialisation}};
  \draw[acc3m] (13.4,0) --++ (2,0) node[pos=.5, yshift=-.5cm] {\emph{Fusion}};
\end{scope}

\end{tikzpicture}
    \caption{Organisation générale de la méthode de construction de la
    \emph{zone de localisation probable} à partir d'un ensemble
    d'indices de localisation}
    \label{fig:methodo_1}
  \end{figure}
\end{landscape}


%%% Local Variables:
%%% mode: latex
%%% TeX-master: "../../../../main"
%%% End:
