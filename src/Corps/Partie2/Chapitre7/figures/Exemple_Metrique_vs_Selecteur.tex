\begin{tikzpicture}
  % Arrow
  \begin{scope}
    \path[draw, ->, shorten >=5pt, shorten <=5pt] (2,1) --++ (4,0) node[pos=.5, font=\footnotesize, fill=white]
    {\itshape rasterisation};
    
    \path[draw, ->, shorten >=5pt, shorten <=5pt] (7,2) |- (12,3)
    node[pos=.75, font=\footnotesize, fill=white] (labsel1)
    {\itshape sélection};
    % 
    \node[anchor=north west,baseline, text width=2.5cm, font=\tiny] at
    ([xshift=1.5ex, yshift=1ex]labsel1.south west) {des pixels dont la
    matrice 4IM est : \(\left[\begin{smallmatrix}V&V\\V&F\\\end{smallmatrix}\right]\)};

    \path[draw, ->, shorten >=5pt, shorten <=5pt] (7,0) |- (12,-1)
    node[pos=.75, font=\footnotesize, fill=white] (labsel2)
    {\itshape sélection};
    % 
    \node[anchor=north west,baseline, text width=2.5cm, font=\tiny] at
    ([xshift=1.5ex, yshift=1ex]labsel2.south west) {des pixels dont la
    matrice 4IM est : \(\left[\begin{smallmatrix}F&F\\F&V\\\end{smallmatrix}\right]\)};
  \end{scope}
  % Représentation objet de référence rasterisé
  \begin{scope}
    \begin{scope}
      \foreach \x in {0,1}{
        \foreach \y in {0,1}
        \path[draw, line width=.01mm] (\x,\y) rectangle (\x +1, \y + 1);
      }
      % Limite zir
      \path[ffc, black] (0,0) rectangle (2,2);
      % Pixels rasterisation
      \path[ffa] (1,1) rectangle (2,2);
    \end{scope}
    \node[text width=2.5cm, align=center, anchor=north,
    font=\footnotesize, fill=white] at (1,-.25)
    {\itshape Objet de référence rastérisé};
  \end{scope}
  
  % Représentation métrique
  \begin{scope}[xshift=6cm]
    % \path[ffa] (1,.25) rectangle (1.25, .5);
    \begin{scope}
      \foreach \x in {0,1}{
        \foreach \y in {0,1}
        \path[draw, line width=.01mm] (\x,\y) rectangle (\x +1, \y + 1);
      }
      % 
      \node[font=\small, align=center,anchor=center] at (.5, .5)
      {\(\left[\begin{smallmatrix}F&F\\F&V\\\end{smallmatrix}\right]\)};
      \node[font=\small, align=center,anchor=center] at (.5, 1.5)
      {\(\left[\begin{smallmatrix}F&F\\F&V\\\end{smallmatrix}\right]\)};
      \node[font=\small, align=center,anchor=center] at (1.5, .5)
      {\(\left[\begin{smallmatrix}F&F\\F&V\\\end{smallmatrix}\right]\)};
      \node[font=\small, align=center,anchor=center] at (1.5, 1.5)
      {\(\left[\begin{smallmatrix}V&V\\V&F\\\end{smallmatrix}\right]\)};
    \end{scope}
    \path[ffc, black] (0,0) rectangle (2,2);
    \node[text width=4.5cm, align=center, anchor=north,
    font=\footnotesize, fill=white] at (1,-.25)
    {\itshape Métrique};
  \end{scope}

    \begin{scope}[xshift=12cm, yshift=2cm]
    \begin{scope}
      \foreach \x in {0,1}{
        \foreach \y in {0,1}
        \path[draw, line width=.01mm] (\x,\y) rectangle (\x +1, \y + 1);
      }
      \path[ffa] (1,1) rectangle (2,2);
      \path[ffc] (1,1) rectangle (2,2);
    \end{scope}
    \node[text width=2.5cm, align=center, anchor=north, font=\footnotesize] at (1,-.25)
    {\itshape Zone de localisation compatible \normalfont \textcolor{RdBu-9-1}{\textsf{A}}};
  \end{scope}
  
  \begin{scope}[xshift=12cm, yshift=-2cm]
    \begin{scope}
      \foreach \x in {0,1}{
        \foreach \y in {0,1}
        \path[draw, line width=.01mm] (\x,\y) rectangle (\x +1, \y + 1);
      }

      \begin{scope}
        \fill[ffa2] (0,0) rectangle (2,1);
        \fill[ffa2] (0,1) rectangle (1,2);
      \end{scope}
      
      \path[ffc2] (0,0) --++(0,2) --++(1,0) --++(0,-1) --++ (1,0)
      --++(0,-1) --cycle;
    \end{scope}
    \node[text width=2.5cm, align=center, anchor=north, font=\footnotesize] at (1,-.25)
    {\itshape Zone de localisation compatible \normalfont \textcolor{RdBu-9-9}{\textsf{B}}};
  \end{scope}
\end{tikzpicture}