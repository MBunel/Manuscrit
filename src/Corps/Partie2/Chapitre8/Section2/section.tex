

\subsection{La théorie des possibilités}


La théorie des \emph{sous-ensembles flous} ne permet pas, du moins
dans sa formulation initiale, de modéliser l'incertitude d'une
connaissance, seulement son \emph{imprécision.}

\textcite{Zadeh1978} a donc proposé en 1978 la \emph{théorie des
  possibilités,} qui permet d'étendre la théorie des sous-ensembles
flous en permettant la modélisation conjointe de \emph{l'imprécision}
et de \emph{l'incertitude.}

La \emph{théorie des possibilités} ne doit pas être confondue avec la
\emph{théorie des probabilités.} Si ces théories modélisent toutes
deux \emph{l'incertitude,} elles utilisent un formalisme et ont des
propriétés différentes. Cependant, ces deux théories restent
comparables, en plus de s’inscrire toutes deux dans le cadre plus
général \autocite{Bouchon-Meunier1995}, celui de la théorie des
fonctions de croyances \autocite{Shafer1976}.

Comme la \emph{théorie des probabilité} (et donc des \emph{fonctions
  de croyance}), la \emph{théorie des possibilités} permet évaluer la
\enquote{chance} qu'un événement donné se produise. Ces événements
sont regroupés dans un ensemble fini \(X\), appelé \emph{ensemble des
  événements.} La \emph{théorie des possibilités} permet d'attribuer à
chaque événement défini dans \(X\) un coefficient, compris entre 0 et
1, évaluant la possibilité de l'événement. Une valeur de 1 indiquant
que l'événement est \enquote{tout à fait possible}
\autocite[p. 43]{Bouchon-Meunier2007} et une valeur nulle, qu'il est
impossible. Ce coefficient est attribué par une fonction \(Π\), nommée
\emph{mesure de possibilité,} qui l'attribue à chaque élément de
l'ensemble des parties de \emph{l'ensemble des événements}
(\(Π : P(X) → [0,1]\)). La fonction \(Π\) doit nécessairement :

\begin{itemize}
\item Attribuer un coefficient nul à \emph{l'ensemble vide} :
  \(Π(∅)=0\)
\item Attribuer un coefficient de 1 à \emph{l'ensemble des événements
    :} \(Π(X)=1\)
\item Définir la \emph{possibilité} conjonctive de deux événements
  (\(A\) et \(B\)) comme la \emph{possibilité} de l'événement le plus
  possible : \(∀ A,B ∈ P(X),\ Π(A ∪ B) = \max(Π(A),\ Π(B))\)
\end{itemize}

Par déduction, la possibilité disjonctive de deux événements est
toujours inférieure à la \emph{possibilité} de l'événement le moins
possible :
%
\begin{equation}
  ∀ A,\ B ∈ P(X),\ Π(A ∩ B) ≤ \min(Π(A),\ Π(B))
\end{equation}
%
Par conséquent, deux événements peuvent avoir chacun une possibilité
non nulle, mais leur occurrence simultanée peut être nulle
(\( Π(A ∪ B) = 0\)) n'implique pas que \(Π(A)\) et \(Π(B)\) soient
égaux à 0). De plus, la \emph{mesure de possibilité} est
\emph{monotone} relativement à l'inclusion des parties de \(X\),
c'est-à-dire que si un événement \(A\) est inclus dans un événement
\(B\), alors la possibilité de \(B\) est supérieure à celle de \(A\) :
%
\begin{equation}
  A \subseteq B, Π(A) ≤ Π(B)
\end{equation}
%
Enfin, tous les événements de \(X\) ou leur complémentaires sont
\enquote{tout à fait possibles} :
\begin{equation}
  ∀ A ∈ P(X),\ \max(Π(A),\ Π(A^C)) = 1  
\end{equation}
%
Par conséquent, la somme des possibilité des ces deux événements peut
dépasser 1 :
%
\begin{equation}
  Π(A) + Π(A^C) ≥ 1
\end{equation}

Lorsqu'une \emph{mesure de possibilité} attribue à chaque
\emph{événement} de \(X\) \footnote{Et par conséquent pour tous les
  événements de l'ensemble des parties de \emph{l'ensemble des
    événements} \(P(X)\), puisque ce dernier est constructible à
  partir de l'union des éléments élémentaires de \(X\).} un
coefficient de possibilité on parle de mesure \enquote{totalement
  définie} \autocite{Bouchon-Meunier2007}. Pour attribuer un
\emph{coefficient de possibilité} à chaque événement de \(X\) on
définit une \emph{distribution de possibilité} (\(π : X → [0,1]\)),
dont le supremum pour un événement \(x\) de \(X\) doit être égal à 1 :
%
\begin{equation}
  \sup_{x ∈ X}π(x)=1
\end{equation}


Une fonction d'appartenance normalisée est donc une distribution de
possibilité.


Dans le cas où \emph{l'incertitude} (\(i\)) est homogène, la
\emph{distribution de possibilité} (\(π\)) de cette proposition est
obtenue en tronquant la base de la fonction d'appartenance par la
droite d'ordonnée \(i\) \autocite{Bouchon-Meunier2007}. La
\emph{distribution de possibilité} (\(π'\)) combinant incerti XX est
donc :
%
\begin{equation}
  π'(x) = \max(π(x),\ i).  
\end{equation}


La confiance peut se définir en amont de la spatialisation.

Diminuer la confiance c'est augmenter le degré d'appartenance minimal

Proposer une version améliorée de la fig finale du chapitre 4 avec la
confiance.

Dire que les fonctions présentées reviennent a considérer que la
confiance est absolue

Parler des seuils de la confiance (3) et des valeurs associées ou
contraintes.


\subsection{La modélisation de la confiance d'un \emph{indice de
    localisation}}

Pour permettre la prise en compte de la confiance


La confiance donnée par le secouriste portant sur \emph{l'indice de
  localisation} et non sur l'une de ces composantes (\eg \emph{objet
  de référence,} \emph{relation de localisation,} \emph{etc.}), le
degré de confiance est distribué de manière homogène, chaque position
de la \ac{zir} se voyant attribuer la même valeur.


La modification des degrés d'appartenance des positions est
équivalente à une modification de la forme de la fonction
d'appartenance utilisée pour \emph{spatialiser l'indice de
  localisation} (\autoref{fig:fnc_incert}). Comme pour les
\emph{modifieurs} (\autoref{chap:07}), la variation du seuil de
confiance modifie la forme générale de la courbe, mais ne change pas
les paramètres du \emph{fuzzyficateur,} que sont l'écartement
(\(\delta\)) et la valeur de référence (\(v\)). Toutefois,
contrairement aux \emph{modifieurs,} la modification de la confiance à
également pour effet de changer la valeur de seuils de la fonction
d'appartenance. En effet, si la pente des fonctions appartenance est
conservée lors du rehaussement de la valeur minimale, ce n'est pas le
cas de la valeur où la fonction d'appartenance atteint la valeur
nulle, cette dernière étant rehaussée par la diminution de la
confiance.

\begin{figure}
  \centering  \subfloat[\label{fig:fnc_app_inc}]{\input{../figures/fnc_app_inc.tex}}\hfill
  \subfloat[\label{fig:fnc_app_inc_2}]{\input{../figures/fnc_app_inc_2.tex}}

  \subfloat[\label{fig:fnc_app_inc_3}]{\input{../figures/fnc_app_inc_3.tex}}\hfill
  \subfloat[\label{fig:fnc_app_inc_4}]{\input{../figures/fnc_app_inc_4.tex}}
  %
  \caption[Représentation des \emph{fonctions d'appartenance} de
  différents \emph{fuzzyfieurs} avec une
  \emph{incertitude}]{Représentation des \emph{fonctions
      d'appartenance} des \emph{fuzzyfieurs}
    \protect\onto[orla]{Eq\-Val} \protect\subref{fig:fnc_app_inc},
    \protect\onto[orla]{Sup\-Val} \protect\subref{fig:fnc_app_inc_2},
    \protect\onto[orla]{Inf\-Val} \protect\subref{fig:fnc_app_inc_3}
    et \protect\onto[orla]{Eq\-Val} avec le \emph{modifieur}
    \protect\onto[orla]{Not} \protect\subref{fig:fnc_app_inc_4}, avec
    un degré de \emph{certitude} (\(i\)) variable (respectivement
    0,8, 0,4, 0,2 et 0,5).}
  \label{fig:fnc_incert}
\end{figure}

On peut illustrer les effets de cette modélisation en comparant le
résultat de la \emph{spatialisation} d'un même \emph{indice de
  localisation} avec une confiance variable. La
\autoref{fig:zlc_cert_vs_inceret} illustre la \emph{spatialisation} de
la \emph{relation de localisation atomique} \onto[orla]{Pres\-De}
(comme pour les figures \ref{fig:tiers-exclu} et
\ref{fig:non-contradiction}) à partir du même pixel, avec la même
fonction d'appartenance. Cependant, alors qu'une confiance absolue est
donnée au premier cas, c'est une confiance partielle (0,4) qui est
donnée à la seconde modélisation. Par conséquent, la valeur minimale
que peut prendre une position est de 0,6. Si cette approche permet de
XXXXX, c'est par le jeu des opérations inter-zones qu'elle prend tout
son sens.

En rehaussant la valeur du degré d'appartenance minimal à la \ac{zlc},
la m

\begin{figure}
  \centering
  \subfloat[]{\begin{tikzpicture}
  \begin{scope}[xshift=-2cm, local bounding box=fuzz]
    \begin{scope}
      \foreach \x in {0,.25,...,1.75}{ \foreach \y in {0,.25,...,1.75}
        {
          % Calcul et représentation floue distance euclidienne
          \pgfmathsetmacro\dist{sqrt((\x-0.125)^2+(\y-1.375)^2)}
          % Fuzzyfication de la distance
          \pgfmathsetmacro\fuzzy{%
            ifthenelse(\dist < .25,1,%
            ifthenelse(\dist < 2,-0.57*\dist+1.14,0)%
            )
          }
          % Calcul du rayon à partir de la fuzzyfication
          \pgfmathsetmacro\radius{\fuzzy*2.5}
          \fill[RdBu-9-1] (\x+.125,.125+\y) circle (\radius pt);
        }
      }
      \path[ffc] (0,0) rectangle (2,2);
    \end{scope}
  \end{scope} 
\end{tikzpicture}}\hspace{2cm}
  \subfloat[]{\input{../figures/zlc_incertaine.tex}}
  \caption{Modélisation d'un même \emph{relation de localisation
      atomique} (\protect\onto[orla]{Pres\-De}), à partir du même
    \emph{objet de référence} avec deux certitudes différentes,
    absolue et partielle (0,4).}
  \label{fig:zlc_cert_vs_inceret}
\end{figure}


\begin{figure}
  \centering
  \subfloat[]{\begin{tikzpicture}
  \begin{scope}[xshift=-2cm]
    \begin{scope}
      \foreach \x in {0,.25,...,1.75}{ \foreach \y in {0,.25,...,1.75}
        {
          % Calcul et représentation floue distance euclidienne
          \pgfmathsetmacro\dist{sqrt((\x-0.125)^2+(\y-1.375)^2)}
          % Fuzzyfication de la distance
          \pgfmathsetmacro\fuzzy{%
            ifthenelse(\dist < .25,1,%
            ifthenelse(\dist < 2,-0.57*\dist+1.14,0)%
            )
          }
          % Calcul du rayon à partir de la fuzzyfication
          \pgfmathsetmacro\radius{max(\fuzzy, 0.6)*2.5}
          \fill[black] (\x+.125,.125+\y) circle (\radius pt);
        }
      }
      \path[ffc, black] (0,0) rectangle (2,2);
    \end{scope}
  \end{scope} 
  
  \begin{scope}[xshift=1.2cm]
    \begin{scope}
      \foreach \x in {0,.25,...,1.75}{ \foreach \y in {0,.25,...,1.75}
        {
          % Calcul et représentation floue distance euclidienne
          \pgfmathsetmacro\dist{sqrt((\x-0.125)^2+(\y-1.375)^2)}
          % Fuzzyfication de la distance
          \pgfmathsetmacro\fuzzy{%
            ifthenelse(\dist < .25,1,%
            ifthenelse(\dist < 2,-0.57*\dist+1.14,0)%
            )
          }
          % Calcul du rayon à partir de la fuzzyfication
          \pgfmathsetmacro\radius{max(\fuzzy, 0.2)*2.5}
          \fill[RdBu-9-1] (\x+.125,.125+\y) circle (\radius pt);
        }
      }
      \path[ffc] (0,0) rectangle (2,2);
    \end{scope}
  \end{scope}

    \begin{scope}[xshift=6.8cm]
    \begin{scope}
      \foreach \x in {0,.25,...,1.75}{ \foreach \y in {0,.25,...,1.75}
        {
          % Calcul et représentation floue distance euclidienne
          \pgfmathsetmacro\dist{sqrt((\x-0.125)^2+(\y-1.375)^2)}
          % Fuzzyfication de la distance
          \pgfmathsetmacro\fuzzy{%
            ifthenelse(\dist < .25,1,%
            ifthenelse(\dist < 2,-0.57*\dist+1.14,0)%
            )
          }
          % Calcul du rayon à partir de la fuzzyfication
          \pgfmathsetmacro\radius{max(\fuzzy, 0.6)*2.5}
          \fill[RdBu-9-9] (\x+.125,.125+\y) circle (\radius pt);
        }
      }
      \path[ffc2] (0,0) rectangle (2,2);
    \end{scope}
  \end{scope}

  \node[scale=5] at (-3.2,1) {\(\top(\)};
  \node[scale=5] at (.45,0) {\(,\)};
  \node[scale=5] at (4.7,1) {\() = \)};
  
\end{tikzpicture}}

  \subfloat[]{\begin{tikzpicture}
  \begin{scope}[xshift=-2cm]
    \begin{scope}
      \foreach \x in {0,.25,...,1.75}{ \foreach \y in {0,.25,...,1.75}
        {
          % Calcul et représentation floue distance euclidienne
          \pgfmathsetmacro\dist{sqrt((\x-0.125)^2+(\y-1.375)^2)}
          % Fuzzyfication de la distance
          \pgfmathsetmacro\fuzzy{%
            ifthenelse(\dist < .25,1,%
            ifthenelse(\dist < 2,-0.57*\dist+1.14,0)%
            )
          }
          % Calcul du rayon à partir de la fuzzyfication
          \pgfmathsetmacro\radius{max(\fuzzy, 0.6)*2.5}
          \fill[RdBu-9-1] (\x+.125,.125+\y) circle (\radius pt);
        }
      }
      \path[ffc] (0,0) rectangle (2,2);
    \end{scope}
  \end{scope} 
  
  \begin{scope}[xshift=1cm]
    \begin{scope}
      \foreach \x in {0,.25,...,1.75}{ \foreach \y in {0,.25,...,1.75}
        {
          % Calcul et représentation floue distance euclidienne
          \pgfmathsetmacro\dist{sqrt((\x-0.125)^2+(\y-1.375)^2)}
          % Fuzzyfication de la distance
          \pgfmathsetmacro\fuzzy{%
            ifthenelse(\dist < .25,1,%
            ifthenelse(\dist < 2,-0.57*\dist+1.14,0)%
            )
          }
          % Calcul du rayon à partir de la fuzzyfication
          \pgfmathsetmacro\radius{max(\fuzzy, 0.8)*2.5}
          \fill[RdBu-9-1] (\x+.125,.125+\y) circle (\radius pt);
        }
      }
      \path[ffc] (0,0) rectangle (2,2);
    \end{scope}
  \end{scope}

    \begin{scope}[xshift=6.4cm]
    \begin{scope}
      \foreach \x in {0,.25,...,1.75}{ \foreach \y in {0,.25,...,1.75}
        {
          % Calcul et représentation floue distance euclidienne
          \pgfmathsetmacro\dist{sqrt((\x-0.125)^2+(\y-1.375)^2)}
          % Fuzzyfication de la distance
          \pgfmathsetmacro\fuzzy{%
            ifthenelse(\dist < .25,1,%
            ifthenelse(\dist < 2,-0.57*\dist+1.14,0)%
            )
          }
          % Calcul du rayon à partir de la fuzzyfication
          \pgfmathsetmacro\radius{max(\fuzzy, 0.6)*2.5}
          \fill[RdBu-9-1] (\x+.125,.125+\y) circle (\radius pt);
        }
      }
      \path[ffc] (0,0) rectangle (2,2);
    \end{scope}
  \end{scope}

  \node[scale=5] at (-3.2,1) {\(\top(\)};
  \node[scale=5] at (.45,2-\baselineskip) {\(,\)};
  \node[scale=5] at (4.5,1) {\() = \)};
  
\end{tikzpicture}}
  \caption{Influence de l'incertitude d'un \emph{indice de
      localisation} sur la \emph{zone de localisation} résultant de la
    fusion de deux \ac{zlc}.}
  \label{fig:intersection}
\end{figure}



%%% Local Variables:
%%% mode: latex
%%% TeX-master: "../../../../main"
%%% End:
