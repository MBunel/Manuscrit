\subsection{Considérations générales}

Le recours à une implémentation raster (\autoref{chap:06}) a deux conséquences importantes. 

\tdi{Importance du raster de base et de la maille}
Tout d'abord il est nécessaire de définir une résolution de travail, qui influencera sur la précision et la durée du calcul.

\tdi{Importance de la zir}
Ensuite il est néssaire de définir la \ac{zir}


\subsection{Les parties du processus de \emph{spatialisation}}

\tdi{Présenter les 3 grandes étapes de la démarche, leur rôle et leur justification}

\tdi{méthode de spatialisation = rasterisation + calcul métrique +
  sélection floue}

\begin{figure}
  \centering
  \begin{tikzpicture}
  \begin{scope}
    \path[ffa] (0,0) rectangle (2,2);
    \path[ffc] (0,0) rectangle (2,2);
    \node[text width=3cm, align=center, anchor=center] at (1,-.75)
    {\footnotesize \itshape Zone initiale de recherche};
  \end{scope}

  \begin{scope}[yshift=2.5cm]
    \fill[ffa] (1.25,.7) circle [radius=7pt];
    \path[ffc] (1.25,.7) circle [radius=7pt];
    \node[text width=3cm, align=center, anchor=center] at (1,.15)
    {\footnotesize \itshape Objet de référence};
  \end{scope}

  
  \path[draw, ->] (2.5,1) --++ (1,0)  node[pos=.5, above]
  {\footnotesize \itshape rasterisation};
  
  \begin{scope}[xshift=4.25cm]
    \begin{scope}
      \foreach \x in {0,.25,...,1.75}{
        \foreach \y in {0,.25,...,1.75}
        \path[draw, line width=.01mm] (\x,\y) rectangle (\x +.25, \y + .25);
      }
      % Pixels
      \path[ffa] (1,.25) rectangle (1.25, .5);
      \path[ffa] (1.25,.25) rectangle (1.5, .5);
      \path[ffa] (.75,.5) rectangle (1, .75);
      \path[ffa] (1,.5) rectangle (1.25, .75);
      \path[ffa] (1.25,.5) rectangle (1.5, .75);
      \path[ffa] (1.5,.5) rectangle (1.75, .75);
      \path[ffa] (.75,.75) rectangle (1, 1);
      \path[ffa] (1,.75) rectangle (1.25, 1);
      \path[ffa] (1.25,.75) rectangle (1.5,1);
      \path[ffa] (1.5,.75) rectangle (1.75,1);
      \path[ffc] (0,0) rectangle (2,2);
    \end{scope}
     \node[text width=3cm, align=center, anchor=center] at (1,-.75)
     {\footnotesize \itshape Zone de localisation compatible};
  \end{scope}
  
  % \path[draw, ->] (8.5,1) --++ (1,0)  node[pos=.5, above]
  % {\footnotesize \itshape spatialisation};
  
  \begin{scope}[xshift=8.5cm]
    \begin{scope}
      \foreach \x in {0,.25,...,1.75}{
        \foreach \y in {0,.25,...,1.75}
        \path[draw, line width=.01mm] (\x,\y) rectangle (\x +.25, \y +
        .25);
        \node[circle, inner sep=0pt,minimum size=4pt, fill] (c) at (\x+.125,\y+.125) {};
      }
      \path[ffc] (0,0) rectangle (2,2);
    \end{scope}
  \end{scope}
  
  \begin{scope}[xshift=12.75cm]
    \begin{scope}
      \foreach \x in {0,.25,...,1.75}{
        \foreach \y in {0,.25,...,1.75}
        \path[draw, line width=.01mm] (\x,\y) rectangle (\x +.25, \y + .25);
      }
      \path[ffc] (0,0) rectangle (2,2);
    \end{scope}
  \end{scope}
\end{tikzpicture}
  \caption{Méthode de \emph{spatialisation}}
  \label{fig:methodo_spatialisation}
\end{figure}

\tdi{rasterisation, pourquoi + justifications}%
Dans un premier temps il est nécessaire de
transformer \emph{l'objet de référence} en raster, de manière ce qu'il
puisse être traité. Cette partie est nommée \emph{rasterisation} est,
comme nous le verrons dans la partie qui lui est dédiée
(\autoref{chap:07-sec2}), réalisable de plusieurs façons.

\tdi{métrique}%
La seconde partie de la \emph{spatialisation} est le calcul de la métrique.

\tdi{Fuzzyficaiton}%
Enfin, l'étape dite de \emph{fuzzyfication,} transforme les valeurs de la métrique en un degré d'appartenance.

%%% Local Variables:
%%% mode: latex
%%% TeX-master: "../../../../main"
%%% End:
