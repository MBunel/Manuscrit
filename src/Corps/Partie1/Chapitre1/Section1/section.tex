Le secours en montagne est aujourd'hui un service public majeur, et
fortement ancré dans inconscient collectif.

La simplicité de suface cache cependant une grande complexité. Il
s'agit d'une institution compliquée.

L'objectif de cette partie est de décrire le secours en montagne et
son fonctionnement

\subsection{Le secours en montagne dans les Alpes françaises}
\label{subsec:1-1-1}


\begin{verbatim}
Expliquer que le secours comme on le voit auj, Professionel, héliporté
et médicalisé est quelque chose qui s'est construit avec le
temps. Pendant longtemps c'était des amateurs et la présence
systématique de médecins est assez récente (Halle)
\end{verbatim}

\subsubsection{Historique}
\label{subsubsec:1-1-1-1}

Nous ne souhaitons pas entreprendre un travail d'historien, très
éloigné de nos compétances. Cette partie n'est pas construitre pour
répondre à une problématique historique, mais plutôt pour dresser un
état des lieux du secours en montagne.

Malheuresement ces événements nous sont apparus peu documentés.

Les événements principaux de l'histoire des secours en montagne
français peuvent etre retrouvés sur la figure
\ref{fig:frise_chronologique} page \pageref{fig:frise_chronologique}.


\paragraph{Le proto-secours en montagne}
\label{par:1-1-1-1-1}

Jusqu'a la première moitiée du \bsc{xx}\up{e} siècle, le secours en
montagne est l’œuvre de montagnards, auto-organisés et comptant sur la
solidarité de leurs compagnons de cordée. Les alpinistes se regroupent
en \emph{comités de secours} associatifs et agissant à l'échelle
locale.

Étonnamment le premier de ces comités n'est pas créé dans un
centre de l'alpinisme, comme Chamonix, mais dans les préalpes avec la
fondation, en 1897, des \emph{sauveteurs volontaires du Salève}.

L'histoire des secours en montagne est indissociable de celle de
l'alpinisme. Si l

\begin{displayquote}
  \og Pour se distraire en inventant l'art de gravir des montagnes,
  les grands bourgeois britaniques achetèrent les services de \og
  guides \fg locaux, une aubaine devene rapidement un métier, avec ses
  codes \textelp{}\fg{} \footnote{Dernier de cordée}
\end{displayquote}

\begin{verbatim}
premiers secours col du saint-bernard
\end{verbatim}


Avant secours occasionels et non organisés

1888 Création des régiments alpins

1897 les ``sauveteurs volontaires du  Salève sont le premier groupement
de secouristes en france. Sans doute à cause d'une pression touristique.

1910 Le syndicat d'initiatives de Grenoble fonde le comité de secours
en montagne du Dauphiné

1911 L'article « l'accident de l'aiguille du plan » dresse un aperçu
des problèmes liés à une organisation des secours basés sur le
volontariat.

C'est au tournant des années 30 que seront crées la majorité des
commités de secours. Avec, notamment, la création d'un commité de
secours à Annecy en 1928, à Chamonix en 1929, à Briançon en 1932 ou à
Pau en 1936. \texttt{Source la montagne 46}

\begin{verbatim}
1928 Création du comité de secours annécien

1929 ou 27 selon sopurces Le commité de secours du dauphiné devient la
société dauphinoise de secours en montagne. Son organisation est
décrite dans la montagne.

1932 Création de l'\ac{ehm}

1932 Création d'un commité de secours à Briançon source la montagne 46

1936 Création d'un commité de secours à Pau
\end{verbatim}


1946 Lucien Devies publie l'orgaganisation des secours en montagne. Il
met en évidence le retard de la France en manière de secours en
montagne comparé à l'Allemangne, l'autriche et la Suisse. Il appelle
les autorités publiques à prendre le controle des secours -> le
secours en montagne comme service public.

\paragraph{Le début d'une organisation nationale}

Ces propositions aboutissent, en 1947, lorsqu'est créée la commision
de secours de la fédération française de montagne \acp{ffm} dont
\bsc{Devies} devient président. L'objectif de cette nouvelle
institution est de fédérer les différents commités locaux de secours
en montagne pour proposer des réponses plus efficaces aux situations
de crise. Cette organisation permettra la mise en place de secours
d'envergure, comme en 1948, lorsqu'une centaine de secouristes
participent au sauvetage de deux alpinistes coincés dans la face sud,
d'une verticalité extrème, du Pavé dans le massif des
Écrins\footfullcite{Romanaz2018} ou en 1949 lors du sauvetage de
l'Olan\footfullcite{Mollaret1993}. Cette commission dirigera également
les secours lors des crashs du \emph{Canadian Pilgrim} sur l'Obiou et
du \emph{Malabar Princess} sur le mont-Blanc en
1950\footfullcite{CFDLD,Mollaret1993,SDSM2013}.

Cepndant, cette organisation semi-professionnelle montrera rapidement
ses limites, la faute à des difficultés de financement, de
recrutement, mais également à cause de la hausse de fréquentation des
massifs Alpins\footfullcite{CFDLD}. Mais c'est avec \emph{l'affaire
  Vincendon et Henry} que les autorités prirent concience des limites
d'une organisation bénévole des secours en montagne.

Le 22 décembre 1956 deux étudiants passionnés d'alpinisme, Jean
\bsc{Vincendon} et François \bsc{Henry} entament l’ascension du mont
Blanc\footfullcite{Ballu1997}. Les deux hommes cherchent l'exploit, à
une époque où l'alpinisme hivernal est encore balbutiant. Ils décident
cependant de redescendre vers Chamonix et d'abandonner leur ascension
le 24, par crainte du mauvais temps à venir. Cependant, leur rencontre
avec la cordée du guide italien Walter \bsc{Bonatti,} déjà auréolé
d'acenssions notables comme la première du K2 en
1954\footnote{\bsc{Bonatti} n'atteindra cependant pas le sommet.} ou
l'ouverture en solitaire d'une voie dans la face sud-ouest des Drus en
1955, les incite à reprendre l’ascension. Les deux cordées, fidèles à
leur itinéraire initial, se séparent le lendemain. Bien qu'empruntant
des itinéraires distincts, les deux groupes se font surprendre par la
nuit et bivouaquent sous la tempête. Le lendemain \bsc{Bonatti} et son
client atteignent le refuge Vallot, alors que \bsc{Vincendon} et
\bsc{Henry,} épuisés, prennent la décision de descendre
directement. Lors de la descente les deux alpinistes se perdent et se
retrouvent coincés sur une vire. Exténués et surchargés de matériel
ils sont incapables de rebrouser chemin et passent une seconde nuit
dehors.

Dans la vallée, les secours, prévenus dès le 25 décembre par un
proche, tardent à se mettre en place. La société chamoniarde de
secours en montagne, regoupant l'\ac{ehm}, la compagnie des guides de
Chamonix et de l'école nationale de ski alpin n'arrive pas à organiser
une opération de secours, les guides refusant de s'aventurer en
haute-montagne en plein hiver et en pleine tempête. C'est finalement
l'\ac{ehm} qui prendra la direction des secours. Un premier
hélicoptère militaire est envoyé le 27 décembre, sans résultat, mais
les alpinistes sont repérés, à la longue-vue, un peu plus tard. La
configuration spatiale de l'événement favorise effervescence
médiatique, les accidentés sont visibles depuis Chamonix. De nombreux
médias nationaux, tel que la radio \emph{Europe 1,} nouvellement crée,
commencent à couvrir l'événement et les secouristes doivent opérer
sous une pression médiatique alors inédite.

Les victimes étant localisées, une opération de secours héliportée est
organisée le 28. Mais l'utilisation d'un appareil inhadapté au vol en
montagne rend l'approche impossible. Le pilote indique aux alpinistes
de se rapprocher d'un plateau, situé en amont, seul endroit où un
atterrissage est envisageable. Cependant la météo empêche tout nouveau
vol et le directeur de l'\ac{ehm}, commandant de fait des opérations,
refuse de risquer de nouvelles vies en envoyant une caravane terrestre
en pleine tempête. C'est finalement le 31 décembre qu'un nouveau vol
sera possible. Les deux alpinistes sont rejoints par un hélicoptère de
l'armée, qui s'écrase lors de son atterrissage. Les deux pilotes sont
grièvements bléssés. Les deux secouristes présents mettent alors
\bsc{Viencendon} et \bsc{Henry} à l'abri, dans la carcasse de
l'hélicoptère et prennent la décision de remonter en priorité les
pilotes au refuge Vallot, avant de redescendre, le lendemain, pour
secourir les deux alpinistes. Cependant, les quatres hommes restent au
refuge jusqu'a leur évacuation, le 3 janvier. Ne constatant plus aucun
signe de vie, le commandant des opérations décide alors d'abandonner
les recherches.

Cette affaire, qui fut un véritable choc pour l'opignon publique,
marqua un point de rupture dans la conception française du secours en
montagne. Le système de secours basé sur le volontariat avait montré
ses limites et sera rapidement replacé par un système géré par
l'état\footfullcite{Ballu1997}.

\paragraph{Le secours en montagne comme service public}
\label{par:1-1-1-1-2}

\begin{displayquote}
  \og Il a fallu le spectacle de guides professionneles chamoniards
  qui laissèrent périr deux jeunes étudiants en perdition faute
  d'avoir pu s'organiser pour que l'État décide de mettre en place ses
  propres services \textelp{}\fg{}\footfullcite{Descamps2018}
\end{displayquote}

C'est à partir de 1958, après le choc de \emph{l'affaire Vincendon et
  Henry,} que les services publics de secours en montagne se mettront
en place. Les gendarmes et policiers, auparavant participants
occasionnels à des opérations de
secours\footfullcite{Mollaret2016,CFDLD}, en sont officiellement
chargés par la circulaire interministérielle du 21 août
1958\footnote{Circulaire \no 1272 du 21 août 1958 relative à la mise
  en œuvre du secours en montagne. Cette dernière sera abrogée en 2011
  par la circulaire du 6 juin 2011, dite circulaire \bsc{Kihl.}}. Deux
unitées spécialisées dans le secours en montagne sont alors créées, le
\ac{pghm}\footnote{Alors le Groupe Spécialisé en Haute-montange} et de
la \ac{crsm} \autocite{Halle2007}. La formation de ses premiers
professionels du secours en montagne est alors assurée par des
institutions pré-existantes, comme l'\ac{ehm} ou le \ac{cenas} fondé
en 1955 \autocite{Mezin}.

L'action de ces deux corps est régulée au niveau local par des
dispositions spécifiques définies dans les plans
\ac{orsec}\footnote{ORganisation des SECours jusqu'en 2006, puis
  Organisation de la Réponse de la Sécurité Civile.}. Lesquels,
préxistants, ont étés étendus aux secours en montagne. Dans le
département de l'Isère, par exemple, est mis en place, dès 1958, un
régime d'alternance. La \ac{crsm} et le \ac{pghm} sont chargés des
secours une semaine sur deux, ce qui permet d'alterner astreinte et
entrainement ou repos. Ce régime sera par la suite étendu à la
majorité des départements Alpins \autocite{Halle2007}.

À la faveur de la professionnalisation, les techniques et méthodes de
secours en montagne vont connaitre une rapide évolution. L'utilisation
de l'outil héliporté, maladroite lors de \emph{l'affaire Viencendon et
  Henry,} sera considérablement perfectionée au cours des années
soixante. Notamment grâce à l'introduction de l'\emph{alouette
  \bsc{iii}} en 1962, un hélicoptère parfaitement adapté au vol en
montagne et qui sera utilisé jusqu'en 2009\footcite{Elie2006,
  Lafond, Lafond2011b}. L'utilisation de cet outil va évoluer avec le
développement de solutions techniques, tel que le treuil, remplacant
avantageusement les échelles de courdes suspendues sous le
fuselage. Cet outil, introduit en 1965 et utilisé pour la première
fois en 1967 lors d'un sauvetagne au Grépon\footfullcite{Lafond2011a},
permet dans un premier temps de treuiller des civières, évitant à
l'hélicoptère de se poser. Plus tard, des utilisations plus
ambitieuses sont envisagées, comme en 1972 lorsqu'un équipage évacue
deux alpinistes en difficulté sur la face ouest des
Drus\footfullcite{Ministere2013}.L'outil héliporté se révéle d'une
efficacité redoutable et dès 1972 l'ensemble des zones montagneuses
seront à portée d'hélicoptère\footfullcite{CFDLD}. Les secours
héliportés occuperont une place de plus en plus importante dans les
opérations de secours en montagne, passant de 50 \% des 420 secours de
1965\footfullcite{CFDLD} à plus de 90 \% des secours aujourd'hui.

Une autre avancée majeure, mais plus tardive, à également été permise
par la professionalisation des secours, leur médicalisation.


\texttt{Gratuité}

\begin{verbatim}
1980 généralisation de la présence d'un médecin

1985 GMPS

1985-1987 gratuité des secours

2002 mise en cause de la gratuité

2007 Prise de position d'alio-marie sur la gratuité des secours

Poids de l'hélico 80 \% en moyenne sur la période 2015-2019 en domaine
montagne selon le \ac{snosm}.
\end{verbatim}

\begin{figure}
  \centering
   \begin{tikzpicture}[%
  ev/.style={font=\tiny, text centered, inner sep=2pt, fill=white}
  ]
  % Création de la Frise
  \begin{scope}
    % Axe dégradé à gauche
    \path[draw,-,path fading=west] (-.5,0) -- (0,0);
    % Axe principal
    \path[draw,->] (0,0) --++ (12,0);

    % Trait vertical tous les .75 cm (10 ans)
    \foreach \x in {0,.75,...,11.75}{
      \path[draw] (\x,-.05) --(\x,.05);
    }

    % Étiquettes de décénine
    \node[below, font=\tiny] at (0, -.05) {1860};
    \node[below, font=\tiny] at (.75, -.05) {1870};
    \node[below, font=\tiny] at (1.5, -.05) {1880};
    \node[below, font=\tiny] at (2.25, -.05) {1890};
    \node[below, font=\tiny] at (3, -.05) {1900};
    \node[below, font=\tiny] at (3.75, -.05) {1910};
    \node[below, font=\tiny] at (4.5, -.05) {1920};
    \node[below, font=\tiny] at (5.25, -.05) {1930};
    \node[below, font=\tiny] at (6, -.05) {1940};
    \node[below, font=\tiny] at (6.75, -.05) {1950};
    \node[below, font=\tiny] at (7.5, -.05) {1960};
    \node[below, font=\tiny] at (8.25, -.05) {1970};
    \node[below, font=\tiny] at (9, -.05) {1980};
    \node[below, font=\tiny] at (9.75, -.05) {1990};
    \node[below, font=\tiny] at (10.5, -.05) {2000};
    \node[below, font=\tiny] at (11.25, -.05) {2010};  
  \end{scope}

  % Étiquettes partie supérieure
  \foreach \date/\text/\y [evaluate=\date as \x using (\date - 1860)*.075] in {
    1956/{Accident de Vincendon et Henry}/{4.5},
    1947/{Création de la commission des secours en montagne de la \ac{ffm}}/{3.25},
    1985/{Création des \ac{gmsp}}/{3},
    1958/{Plan \ac{orsec}}/{2.5},
    1897/{Création des Sauveteurs volontaires du Salève}/{2.5},
    1865/{Accident du Cervin}/{1.5},
    2002/{Amendement à la loi sur la démocratie de proximité}/{2},
    1929/{Création d'un comité de secours Savoyard}/{1.75},
    1970/{Avalanche à Val-d'Isère}/{1.5},
    1932/{Création d'un comité de secours Briançonnais}/{.5},
    1874/{Création du \ac{caf}}/{.5},
    1987/{Circulaire du 22 septembre 1987}/{.5}
  } {
    \node[shape=circle,fill=black, scale=.25] (a1) at (\x,0) {};
    \path[draw, -|] (a1) --++ (0,\y) node[ev, above] {\begin{varwidth}{3.5cm}\centering\text\\(\date)\end{varwidth}};
  }

  % Étiquettes partie inférieure
  \foreach \date/\text/\y [evaluate=\date as \x using (\date - 1860)*.075] in {
    1980/{Généralisation de la présence des médecins dans les équipes de secours}/{3.75},
    1956/{Premier secours aéroporté}/{3},
    1958/{Création du \ac{pghm}}/{2},
    1910/{Création du comité de secours en montagne du Dauphiné}/{2},
    2007/{La ministre de l'Intérieur confirme la gratuité des secours}/{1.75},
    1945/{Création de la \ac{ffm}}/{1.25},
    1888/{Création des régiments alpins}/{.5},
    1972/{Couverture aérienne complète}/{.5},
    1932/{Création de l'\ac{ehm}}/{.5}
  } {
    \node[shape=circle,fill=black, scale=.25] (a1) at (\x,0) {};
    \path[draw,-|] (\x, -.4) --++ (0,-\y) node[ev, below] {\begin{varwidth}{3.5cm}\centering\text\\(\date)\end{varwidth}};
  }
\end{tikzpicture}
   \caption{Chronologie des principaux événements de
     l'histoire fraçaise des secours en montagne.}
  \label{fig:frise_chronologique}
\end{figure}


\subsubsection{Organisation contemporaine}
\label{subsubsec:1-1-1-2}

Aujourd'hui le secours en montagne est géré par trois crops de métiers
différents.

\begin{verbatim}
Trois corps principaux

Deux organisations différentes, pistes et hors-piste.
\end{verbatim}


\paragraph{Le secours en station}

\begin{verbatim}
voir : https://www.pompiers.fr/actualites/ou-sapplique-la-gratuite-des-secours-en-montagne
\end{verbatim}


\paragraph{Le secours hors-piste}



\begin{quotation}
  «~Ce système de la mixité s'est progressivement imposé, mais il
  n'est pas transposable et devrait deumerer une spécificité
  haut-savoyarde.~»
\end{quotation}

\begin{table}
  \centering
  \begin{tabular}{L{6cm}L{8cm}}
  \toprule
  \multicolumn{1}{c}{\bfseries Départements} & \multicolumn{1}{c}{\bfseries Organisation} \\
  \midrule
  Hautes-Alpes (05), Alpes-Maritimes (06), Isère (38), Savoie (73) & Alternance hebdomadaire \ac{crsm}, \ac{pghm}\\
  Alpes-de-Haute-Provence (04) & \ac{pghm} uniquement\\
  Haute-Savoie (74) & Collaboration \ac{pghm}, \ac{gmsp}, sauf dans dans le massif du Mont-Blanc (\ac{pghm} uniquement)\\
  \bottomrule
\end{tabular}
  \caption{Corps mobilisés pour le secours en montagne dans les
    départements alpins.}
  \label{tab:organisation_secours_departements}
\end{table}

\subsection{Méthodologie du secours en montagne}
\label{susec:1-1-2}

\begin{verbatim}
Selon, Halle 2007, le déclanchement même de l'alerte peut-être
problématique. 
\end{verbatim}

\subsection{Limites Méthodologiques}
\label{subsec:1-1-3}

\texttt{Présenter une image de la zone du fil rouge ?}
%%% Local Variables:
%%% mode: latex
%%% TeX-master: "../../../../main"
%%% End:
