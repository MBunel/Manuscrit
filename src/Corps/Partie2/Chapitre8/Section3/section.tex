Pour illustrer le déroulement de la \emph{phase de fusion} nous nous
fonderons sur l'exemple de la construction de la \ac{zlp}
correspondant à \emph{l'ensemble des indices de localisation}
\enquote{Je suis à l'extérieur, au sud de la forêt (qui a été
  spatialisé dans le \autoref{chap:07}) et je suis proche d'un refuge}
(\autoref{fig:exemple_final_fusion}). Pour les besoins de l'exemple,
nous considérerons que le secouriste a attribué une confiance absolue
au premier \emph{indice} \footnote{C'est le choix qui a implicitement
  été fait lors de la \emph{spatialisation} de cet \emph{indice} dans
  le \autoref{chap:05}.} et une confiance moyenne (\ie un coefficient
de \si{0,0}) au second. Le second \emph{indice de localisation}
utilisant un \emph{objet de référence indéfini} (\eg \enquote{un
  refuge}) nous avons considéré qu'il n'existait que deux objets
candidats, utilisés pour la spatialisation des \ac{zlc}
\textcolor{RdBu-9-8}{\textsf{C}} et \textcolor{RdBu-9-9}{\textsf{D}}
(\autoref{fig:exemple_final_fusion}).

Au terme de la \emph{phase de spatialisation} (\autoref{chap:07}) nous
disposons donc de quatre \ac{zlc} :
% 
\begin{enumerate*}[label=(\arabic*)]
\item la \ac{zlc} \textcolor{RdBu-9-1}{\textsf{A}}, spatialisant la
  \emph{relation de localisation atomique} \onto[orla]{Hors\-De} pour
  \emph{l'objet de référence} forêt,
\item la \ac{zlc} \textcolor{RdBu-9-2}{\textsf{B}}, spatialisant la
  \emph{relation de localisation atomique} \onto[orla]{Sud\-De},
  également pour \emph{l'objet de référence} forêt,
\item la \ac{zlc} \textcolor{RdBu-9-8}{\textsf{C}}, spatialisant la
  \emph{relation de localisation atomique} \onto[orla]{Pres\-De} pour
  la première des deux instances de type \enquote{refuge} identifiées,
  et
\item la \ac{zlc} \textcolor{RdBu-9-9}{\textsf{D}}, spatialisant la
  \emph{relation de localisation atomique} \onto[orla]{Pres\-De} pour
  le second refuge.
\end{enumerate*}

\subsection{La \emph{fusion} des \emph{relations de localisation
    atomiques}}

La première étape de la \emph{phase de fusion} consiste en une
intersection des \ac{zlc} spatialisant une \emph{relation de
  localisation atomique} décomposant la même \emph{relation de
  localisation,} pour le même \emph{objet de référence,} issu du même
\emph{indice de localisation.} Dans l'exemple illustré par la
\autoref{fig:exemple_final_fusion} on retrouve deux configurations
différentes. Dans la partie supérieure de la figure se situent les
\ac{zlc} \textcolor{RdBu-9-1}{\textsf{A}} et
\textcolor{RdBu-9-2}{\textsf{B}}, spatialisant deux \emph{relations de
  localisation atomiques} traitant du même \emph{indice de
  localisation} (\enquote{Je suis à l'extérieur, au sud de la forêt}),
du même \emph{objet de référence} (\enquote{forêt}) et de la même
\emph{relation de localisation}
(\onto[orl]{Au\-Sud\-De\-Ex\-ter\-ne}). Ces deux \emph{zones de
  localisation compatibles} doivent donc être \emph{fusionnées,} de
manière a construire la \ac{zlc} spatialisant la \emph{relation de
  localisaiton} \onto[orl]{Au\-Sud\-De\-Ex\-ter\-ne} pour
\emph{l'objet de référence} \enquote{forêt}, de \emph{l'indice de
  localisation} : \enquote{Je suis à l'extérieur, au sud de la
  forêt}. Cette nouvelle \emph{zone de localisation} est construite en
\emph{intersectant} les \ac{zlc} \textcolor{RdBu-9-1}{\textsf{A}} et
\textcolor{RdBu-9-2}{\textsf{B}}, ce qui équivaut à construire une
\emph{zone} constituée de toutes les positions de la \ac{zir} validant
l'assertion \enquote{Cette position est au sud de XX} \emph{et}
l'assertion \enquote{Cette position est à l'extérieur de
  XX}. Conformément aux opérateurs précédemment sélectionnés, cette
intersection est réalisée en prenant, pour chaque position de la
\ac{zir}, la valeur minimale des deux degrés d'appartenance, on
obtient alors la \ac{zlc} XXXXX.

Bien que provenant du même \emph{indice de localisation} (\enquote{Je
  suis proche d'un refuge}) les \ac{zlc}
\textcolor{RdBu-9-8}{\textsf{C}} et \textcolor{RdBu-9-9}{\textsf{D}}
n'ont pas à être fusionnées lors de cette étape. En effet elles sont
toutes deux construites à partir d'une \emph{relation de localisation
  atomique} (\onto[orla]{Proche\-De}), il n'est donc pas nécessaire
d'intersecter ces deux zones.

\subsection{La \emph{fusion} des \emph{objets de référence indéfinis}}

La seconde étape de la \emph{phase de fusion} est la fusion des
\emph{objets de référence indéfinis.} Son objectif est de regrouper
les \ac{zlc} spatialisant la même \emph{relation de localisation,} du
même \emph{indice de localisation} mais pour des instances différentes
du même type \emph{d'objet de référence.} Comme pour la fusion des
\emph{relations de localisation atomiques,} cette étape ne concerne
qu'une partie de l'exemple. Dans la partie supérieure de la
\autoref{fig:exemple_final_fusion}, un seul \emph{objet de référence}
est mentionné (\enquote{xxx}), il n'y a donc pas d'alternatives qu'il
serait nécessaire de regrouper, contrairement à la seconde partie de
la figure. Comme nous l'avons vu, les \ac{zlc}
\textcolor{RdBu-9-8}{\textsf{C}} et \textcolor{RdBu-9-9}{\textsf{D}}
\emph{spatialisent} la même \emph{relation de localisation atomique}
(et ne nécessitent donc pas d'être fusionnées lors de la première
étape de la phase de spatialisation) mais pour deux instances
différentes d'un même type \emph{d'objet de référence.} Ces deux
\ac{zlc} doivent donc être \emph{fusionnées,} de manière a obtenir la
\emph{zone de localisaiton} spatialisant \emph{l'indice de
  localisation} initial, c'est-à-dire avant la \emph{décomposition des
  objets de référence} et la \emph{décomposition des relations de
  localisation.} Pour ce faire on crée l'union des deux \ac{zlc},
conformément à la méthode précédemment définie, ce qui équivaut à
construire une \ac{zlc} constituée de toutes les positions
\emph{proches} d'au moins un refuge. Compte-tenu des opérateurs
choisis, cette union est réalisée à l'aide de la \emph{t-conorme} de
\textcite{Zadeh1965}. Le degré d’appartenance d'une position donnée, à
l'union des \ac{zlc} \textcolor{RdBu-9-8}{\textsf{C}} et
\textcolor{RdBu-9-9}{\textsf{D}} correspond donc au maximum de son
degré d'appartenance pour ces deux \ac{zlc}.

\subsection{La \emph{fusion} des indices de \emph{localisation}}

Une fois que l'on dispose d'une \ac{zlc} par \emph{indice de
  localisation,} on peut les regrouper pour obtenir la \ac{zlp}. Dans
l'exemple de la \autoref{fig:exemple_final_fusion}, cette dernière
étape de \emph{fusion} combine deux \ac{zlc}, celle correspondant à
\emph{l'indice de localisation} \enquote{je suis au sud et à
  l'extérieur de XXX} et celle correspondant à l'indice de localisation
\enquote{Je suis proche d'un refuge}. Comme lors de l'étape de la
fusion des \emph{relations de localisation atomiques} on cherche à
construire la \emph{zone de localisation} correspondant à ces deux
indices, c'est-à-dire la zone composée de toutes les positions qui
valident l'assertion \enquote{Cette position est au sud et à
  l'extérieur de XXXX} \emph{et} l'assertion \enquote{Cette zone est
  proche d'un refuge}. Pour ce faire on construit, comme pour la
première étape de cette phase, l'intersection de ces deux \ac{zlc},
avec la \emph{t-norme} de \textcite{Zadeh1965}. Le degré
d’appartenance d'une position donnée à la \ac{zlp} est donc égal au
minimum de son degré d'appartenance aux \ac{zlc} spatialisant les deux
\emph{indices de localisation.}

Cette étape n'est cependant pas un décalque de l'étape de fusion des
\emph{relations de localisation atomiques} appliquée à un autre
regroupement de \ac{zlc}. Il est en effet nécessaire d'y adjoindre la
prise en compte de la \emph{certitude} des \emph{indices de
  localisation.} Comme nous l'avons précédemment expliqué, la
\emph{certitude} est une grandeur s'appliquant uniformément à
l'ensemble des positions de la \ac{zir} utilisée pour
\emph{spatialiser} un \emph{indice de localisation} donné.

Après la réalisation de cette dernière étape on obtient la \ac{zlp}.

Comme on peut le voir avec cet exemple XXXXX

\begin{landscape}
  \begin{figure}[H] \centering
    \begin{tikzpicture}
  % grille temp
  % \draw[step=1.0,black,thin] (0,0) grid (20,-12);
  % Arrow

  \newcommand{\calcangle}[4]{%
    % #1 nom de la varaible renvoyée
    % #2 coordonnées de l'objet de référence
    % #3 coordonnées du point visé
    % #4 coordonées de la position testée
    \pgfmathanglebetweenlines{#2}{#3}{#2}{#4}%
    \global\let#1\pgfmathresult%
  }

  
  \begin{scope}
    % \path[draw, -, shorten >=5pt, shorten <=5pt] (-1,0) -- (-5,-2);
    % \path[draw, ->,shorten >=5pt, shorten <=5pt] (8.25,3) -- (10,3);
    % \path[draw, -,shorten >=12pt, shorten <=5pt] (1,-3) |- (3.65,-1);
    % \path[draw, ->,shorten >=5pt, shorten <=5pt] (8.25,-1) -- (10,-1);
  \end{scope}


  %%%%%%%%%%%% 
  % Colonne 1
  %%%%%%%%%%%% 
  % 
  % 
  % Zone de localisation compatible 1
  \begin{scope}[local bounding box=zlc1]
    \begin{scope}
      \foreach \x in {0,.25,...,1.75}{ \foreach \y in {0,.25,...,1.75}
        {
          % Calcul de l'angle
          \calcangle{\angle}{\pgfpoint{0.125cm}{1.375cm}}{\pgfpoint{1.75cm}{0cm}}{\pgfpoint{\x cm}{\y cm}}
          % Fuzzyfication de l'angle
          \pgfmathsetmacro\fuzzy{%
            ifthenelse(\angle < 90,(90-\angle)/90,
            ifthenelse(\angle > 270,(\angle-270)/90,0)
            )
          }
          % Calcul du rayon à partir de la fuzzyfication
          \pgfmathsetmacro\radius{\fuzzy*2.5}
          \fill[RdBu-9-1] (\x+.125,.125+\y) circle (\radius pt);
        }
      }
      \path[ffc] (0,0) rectangle (2,2);
    \end{scope}
    \node[text width=3cm, align=center, anchor=north, font=\footnotesize] at (1,-.25)
    {\itshape Zone de localisation compatible \normalfont \textcolor{RdBu-9-1}{\textsf{A}}};
  \end{scope}
  % 
  % ZLC 2
  \begin{scope}[yshift=-3.5cm,local bounding box=zlc2]
    \begin{scope}
      \foreach \x in {0,.25,...,1.75}{
        \foreach \y in {0,.25,...,1.75}
        {
          % Calcul et représentation floue distance euclidienne
          \pgfmathsetmacro\dist{sqrt((\x-0.125)^2+(\y-1.375)^2)}
          % Fuzzyfication de la distance
          \pgfmathsetmacro\fuzzy{%
            ifthenelse(\dist < .25,1,%
            ifthenelse(\dist < 2,-0.57*\dist+1.14,0)%
            )
          }
          % Calcul du rayon à partir de la fuzzyfication
          \pgfmathsetmacro\radius{(1-\fuzzy)*2.5}
          \fill[RdBu-9-2] (\x+.125,.125+\y) circle (\radius pt);
        }
      }
      \path[ffc, RdBu-9-2] (0,0) rectangle (2,2);
    \end{scope}
    \node[text width=3cm, align=center, anchor=north, font=\footnotesize] at (1,-.25)
    {\itshape Zone de localisation compatible \normalfont \textcolor{RdBu-9-2}{\textsf{B}}};
  \end{scope}
  % 
  % ZlC 3
  \begin{scope}[yshift=-8cm,local bounding box=zlc3]
    \begin{scope}
      \foreach \x in {0,.25,...,1.75}{
        \foreach \y in {0,.25,...,1.75}
        {
          % Calcul et représentation floue distance euclidienne
          \pgfmathsetmacro\dist{sqrt((\x-0.125)^2+(\y-1.375)^2)}
          % Fuzzyfication de la distance
          \pgfmathsetmacro\fuzzy{%
            ifthenelse(\dist < .25,1,%
            ifthenelse(\dist < 2,-0.57*\dist+1.14,0)%
            )
          }
          % Calcul du rayon à partir de la fuzzyfication
          \pgfmathsetmacro\radius{max(\fuzzy,0.3)*2.5}
          \fill[RdBu-9-8] (\x+.125,.125+\y) circle (\radius pt);
        }
      }
      \path[ffc2, RdBu-9-8] (0,0) rectangle (2,2);
    \end{scope}
    \node[text width=3cm, align=center, anchor=north, font=\footnotesize] at (1,-.25)
    {\itshape Zone de localisation compatible \normalfont \textcolor{RdBu-9-8}{\textsf{C}}};
  \end{scope}
  % 
  % Zlc 4
  \begin{scope}[yshift=-11.5cm,local bounding box=zlc4]
    \begin{scope}
      \foreach \x in {0,.25,...,1.75}{
        \foreach \y in {0,.25,...,1.75}
        {
          % Calcul et représentation floue distance euclidienne
          \pgfmathsetmacro\dist{sqrt((\x-2.125)^2+(\y-1.875)^2)}
          % Fuzzyfication de la distance
          \pgfmathsetmacro\fuzzy{%
            ifthenelse(\dist < .25,1,%
            ifthenelse(\dist < 2,-0.57*\dist+1.14,0)%
            )
          }
          % Calcul du rayon à partir de la fuzzyfication
          \pgfmathsetmacro\radius{max(\fuzzy,0.3)*2.5}
          \fill[RdBu-9-9] (\x+.125,.125+\y) circle (\radius pt);
        }
      }
      \path[ffc2] (0,0) rectangle (2,2);
    \end{scope}
    \node[text width=3cm, align=center, anchor=north, font=\footnotesize] at (1,-.25)
    {\itshape Zone de localisation compatible \normalfont \textcolor{RdBu-9-9}{\textsf{D}}};
  \end{scope}
  %%%%%%%%%%%% 
  % Colonne 2
  %%%%%%%%%%%% 
  \begin{scope}[xshift=6cm, yshift=-1.75cm,local bounding box=zlc5]
    \begin{scope}
      \foreach \x in {0,.25,...,1.75}{
        \foreach \y in {0,.25,...,1.75}
        {
          % Calcul de l'angle
          \calcangle{\angle}{\pgfpoint{0.125cm}{1.375cm}}{\pgfpoint{1.75cm}{0cm}}{\pgfpoint{\x cm}{\y cm}}
          % Fuzzyfication de l'angle
          \pgfmathsetmacro\fuzzya{%
            ifthenelse(\angle < 90,(90-\angle)/90,
            ifthenelse(\angle > 270,(\angle-270)/90,0)
            )
          }
          %
          % Calcul et représentation floue distance euclidienne
          \pgfmathsetmacro\dist{sqrt((\x-0.125)^2+(\y-1.375)^2)}
          % Fuzzyfication de la distance
          \pgfmathsetmacro\fuzzyb{%
            ifthenelse(\dist < .25,1,%
            ifthenelse(\dist < 2,-0.57*\dist+1.14,0)%
            )
          }
          %
          \pgfmathsetmacro\radius{min(\fuzzya,1-\fuzzyb)*2.5}
          \fill[RdBu-9-1] (\x+.125,.125+\y) circle (\radius pt);
        }
      }
      \path[ffc] (0,0) rectangle (2,2);
    \end{scope}
    \node[text width=3cm, align=center, anchor=north, font=\footnotesize] at (1,-.25)
    {\itshape Zone de localisation compatible \normalfont \textcolor{RdBu-9-1}{\textsf{AB}}};
  \end{scope}

  \begin{scope}[xshift=6cm, yshift=-8cm,local bounding box=zlc6]
    \begin{scope}
      \foreach \x in {0,.25,...,1.75}{
        \foreach \y in {0,.25,...,1.75}
        {
          % Calcul et représentation floue distance euclidienne
          \pgfmathsetmacro\dist{sqrt((\x-0.125)^2+(\y-1.375)^2)}
          % Fuzzyfication de la distance
          \pgfmathsetmacro\fuzzy{%
            ifthenelse(\dist < .25,1,%
            ifthenelse(\dist < 2,-0.57*\dist+1.14,0)%
            )
          }
          % Calcul du rayon à partir de la fuzzyfication
          \pgfmathsetmacro\radius{max(\fuzzy,0.3)*2.5}
          \fill[RdBu-9-8] (\x+.125,.125+\y) circle (\radius pt);
        }
      }
      \path[ffc2, RdBu-9-8] (0,0) rectangle (2,2);
    \end{scope}
    \node[text width=3cm, align=center, anchor=north, font=\footnotesize] at (1,-.25)
    {\itshape Zone de localisation compatible \normalfont \textcolor{RdBu-9-8}{\textsf{C}}};
  \end{scope}
  
  \begin{scope}[xshift=6cm, yshift=-11.5cm]
    \begin{scope}
      \foreach \x in {0,.25,...,1.75}{
        \foreach \y in {0,.25,...,1.75}
        {
          % Calcul et représentation floue distance euclidienne
          \pgfmathsetmacro\dist{sqrt((\x-2.125)^2+(\y-1.875)^2)}
          % Fuzzyfication de la distance
          \pgfmathsetmacro\fuzzy{%
            ifthenelse(\dist < .25,1,%
            ifthenelse(\dist < 2,-0.57*\dist+1.14,0)%
            )
          }
          % Calcul du rayon à partir de la fuzzyfication
          \pgfmathsetmacro\radius{max(\fuzzy,0.3)*2.5}
          \fill[RdBu-9-9](\x+.125,.125+\y) circle (\radius pt);
        }
      }
      \path[ffc2] (0,0) rectangle (2,2);
    \end{scope}
    \node[text width=3cm, align=center, anchor=north, font=\footnotesize] at (1,-.25)
    {\itshape Zone de localisation compatible \normalfont \textcolor{RdBu-9-9}{\textsf{D}}};
  \end{scope}
  %%%%%%%%%%% 
  % Colonne 3
  %%%%%%%%%%%% 
  \begin{scope}[xshift=12cm, yshift=-1.75cm,local bounding box=zlc7]
    \begin{scope}
      \foreach \x in {0,.25,...,1.75}{
        \foreach \y in {0,.25,...,1.75}
        {
          % Calcul de l'angle
          \calcangle{\angle}{\pgfpoint{0.125cm}{1.375cm}}{\pgfpoint{1.75cm}{0cm}}{\pgfpoint{\x cm}{\y cm}}
          % Fuzzyfication de l'angle
          \pgfmathsetmacro\fuzzya{%
            ifthenelse(\angle < 90,(90-\angle)/90,
            ifthenelse(\angle > 270,(\angle-270)/90,0)
            )
          }
          %
          % Calcul et représentation floue distance euclidienne
          \pgfmathsetmacro\dist{sqrt((\x-0.125)^2+(\y-1.375)^2)}
          % Fuzzyfication de la distance
          \pgfmathsetmacro\fuzzyb{%
            ifthenelse(\dist < .25,1,%
            ifthenelse(\dist < 2,-0.57*\dist+1.14,0)%
            )
          }
          %
          \pgfmathsetmacro\radius{min(\fuzzya,1-\fuzzyb)*2.5}
          \fill[RdBu-9-1] (\x+.125,.125+\y) circle (\radius pt);
        }
      }
      \path[ffc] (0,0) rectangle (2,2);
    \end{scope}
    \node[text width=3cm, align=center, anchor=north, font=\footnotesize] at (1,-.25)
    {\itshape Zone de localisation compatible \normalfont \textcolor{RdBu-9-1}{\textsf{AB}}};
  \end{scope}
  % 
  \begin{scope}[xshift=12cm, yshift=-9.75cm,local bounding box=zlc8]
    \begin{scope}
      \foreach \x in {0,.25,...,1.75}{
        \foreach \y in {0,.25,...,1.75}
        {
          \pgfmathsetmacro\dista{sqrt((\x-0.125)^2+(\y-1.375)^2)}
          % Fuzzyfication de la distance
          \pgfmathsetmacro\fuzzya{%
            ifthenelse(\dista < .25,1,%
            ifthenelse(\dista < 2,-0.57*\dista+1.14,0)%
            )
          }
          % Calcul et représentation floue distance euclidienne
          \pgfmathsetmacro\distb{sqrt((\x-2.125)^2+(\y-1.875)^2)}
          % Fuzzyfication de la distance
          \pgfmathsetmacro\fuzzyb{%
            ifthenelse(\distb < .25,1,%
            ifthenelse(\distb < 2,-0.57*\distb+1.14,0)%
            )
          }
          \pgfmathsetmacro\radius{max(max(\fuzzya, 0.3), max(\fuzzyb, 0.3))*2.5}
          \fill[RdBu-9-9](\x+.125,.125+\y) circle (\radius pt);
        }
      }
      \path[ffc2] (0,0) rectangle (2,2);
    \end{scope}
    \node[text width=3cm, align=center, anchor=north, font=\footnotesize] at (1,-.25)
    {\itshape Zone de localisation compatible \normalfont \textcolor{RdBu-9-9}{\textsf{CD}}};
  \end{scope}
  %%%%%%%%%%% 
  % Colonne 4
  %%%%%%%%%%%% 
  \begin{scope}[xshift=18cm, yshift=-5.75cm,,local bounding box=zlc9]
    \begin{scope}
      \foreach \x in {0,.25,...,1.75}{
        \foreach \y in {0,.25,...,1.75}
        {          
          % Calcul de l'angle
          \calcangle{\angle}{\pgfpoint{0.125cm}{1.375cm}}{\pgfpoint{1.75cm}{0cm}}{\pgfpoint{\x cm}{\y cm}}
          % Fuzzyfication de l'angle
          \pgfmathsetmacro\fuzzya{%
            ifthenelse(\angle < 90,(90-\angle)/90,
            ifthenelse(\angle > 270,(\angle-270)/90,0)
            )
          }
          %
          % Calcul et représentation floue distance euclidienne
          \pgfmathsetmacro\dist{sqrt((\x-0.125)^2+(\y-1.375)^2)}
          % Fuzzyfication de la distance
          \pgfmathsetmacro\fuzzyb{%
            ifthenelse(\dist < .25,1,%
            ifthenelse(\dist < 2,-0.57*\dist+1.14,0)%
            )
          }
          \pgfmathsetmacro\dista{sqrt((\x-0.125)^2+(\y-1.375)^2)}
          % Fuzzyfication de la distance
          \pgfmathsetmacro\fuzzyc{%
            ifthenelse(\dista < .25,1,%
            ifthenelse(\dista < 2,-0.57*\dista+1.14,0)%
            )
          }
          % Calcul et représentation floue distance euclidienne
          \pgfmathsetmacro\distb{sqrt((\x-2.125)^2+(\y-1.875)^2)}
          % Fuzzyfication de la distance
          \pgfmathsetmacro\fuzzyd{%
            ifthenelse(\distb < .25,1,%
            ifthenelse(\distb < 2,-0.57*\distb+1.14,0)%
            )
          }
          \pgfmathsetmacro\radius{min(\fuzzya, 1-\fuzzyb, max(\fuzzyc, 0.3), max(\fuzzyd, 0.3))*2.5}          
          \fill[black] (\x+.125,.125+\y) circle (\radius pt);
        }
      }
      \path[ffc2, color=black] (0,0) rectangle (2,2);
    \end{scope}
    \node[text width=3cm, align=center, anchor=north, font=\footnotesize] at (1,-.25)
    {\itshape Zone de localisation probable};
  \end{scope}
  % Accolades
  \begin{scope}
    \draw (zlc1.north east) -| ($(zlc1.south east)!0.5!(zlc2.north
    east) + (.1,0)$) |- (zlc2.south east) node[pos=0, yshift=.2] (acc-1)
    {};

    \draw (zlc3.north east) -| ($(zlc3.south east)!0.5!(zlc4.north
    east) + (.1,0)$) |- (zlc4.south east) node[pos=0, yshift=.2] (acc-2)
    {};

    \draw (zlc5.north east) -| ($(zlc5.north east)!0.5!(zlc5.south
    east) + (.1,0)$) |- (zlc5.south east) node[pos=0, yshift=.2] (acc-3)
    {};

        \draw (zlc6.north east) -| ($(zlc6.south east)!0.5!(zlc7.north
    east) + (.1,0)$) |- (zlc7.south east) node[pos=0, yshift=.2] (acc-4)
    {};
    
  \end{scope}
  % Annotations
  \begin{scope}
    \begin{scope}
      \begin{scope}
        \path[ffc] (1.75cm,1.75cm) rectangle (2cm,2cm);
      \end{scope}
      \begin{scope}[yshift=-3.5cm]
        \path[ffc] (1.75cm,1.75cm) rectangle (2cm,2cm);
      \end{scope}
      \begin{scope}[xshift=6cm, yshift=-1.75cm]
        \path[ffc] (1.75cm,1.75cm) rectangle (2cm,2cm);
      \end{scope}
      \node[align=center,anchor=center] (zi) at (4,-1) {$\top(a,b)=c$};
      \path[->, draw] (2,2) -- (zi.west);
      \path[->, draw] (2,-1.5) -- (zi.west);
      \path[->, draw] (zi.east) -- (7.75,.25);
    \end{scope}
    % 
    \begin{scope}
      \begin{scope}[xshift=6cm, yshift=-8cm]
        \path[ffc] (1.75cm,1.75cm) rectangle (2cm,2cm);
      \end{scope}
      \begin{scope}[xshift=6cm, yshift=-11.5cm]
        \path[ffc] (1.75cm,1.75cm) rectangle (2cm,2cm);
      \end{scope}
      \begin{scope}[xshift=12cm, yshift=-9.75cm]
        \path[ffc] (1.75cm,1.75cm) rectangle (2cm,2cm);
      \end{scope}
      \node[align=center,anchor=center] (zou) at (12,-6) {$\bot(\textcolor{RdBu-9-8}{\textsf{C}},\textcolor{RdBu-9-9}{\textsf{D}})=\textcolor{RdBu-9-9}{\textsf{CD}}$};
      \path[->, draw] (8,-6) -- (zou.west);
      \path[->, draw] (8,-9.5) -- (zou.west);
      \path[->, draw] (zou.east) -- (13.75,-7.75);
    \end{scope}
    % 
    \begin{scope}
      \begin{scope}[xshift=12cm, yshift=-1.75cm]
        \path[ffc] (1.75cm,1.75cm) rectangle (2cm,2cm);
      \end{scope}
      \begin{scope}[xshift=18cm, yshift=-5.75cm]
        \path[ffc] (1.75cm,1.75cm) rectangle (2cm,2cm);
      \end{scope}
      \node[align=center,anchor=center] (zu) at (16,-4) {$\top(a,b)=c$};
      \path[->, draw] (14,.25) -- (zu.west);
      \path[->, draw] (14,-7.75) -- (zu.west);
      \path[->, draw] (zu.east) -- (19.75,-3.75);
    \end{scope}
  \end{scope}
\end{tikzpicture}
    \caption{Gros exemple}
    \label{fig:exemple_final_fusion}
  \end{figure}
\end{landscape}


%%% Local Variables: %%% mode: latex %%% TeX-master:
"../../../../main" %%% End:
