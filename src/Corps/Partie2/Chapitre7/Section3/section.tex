La seconde partie du processus de \emph{spatialisation} est le
\emph{calcul de la métrique.} C'est durant cette phase qu'est calculée
la mesure utilisée pour quantifier une grandeur que l'on veut
représentative de la sémantique de la \emph{relation de localisation
  atomique} \emph{spatialisée.} Les \emph{métriques} sont plus
nombreuses et diverses que les \emph{rasterisers.}

\subsection{Classification}

Le concept de \enquote{\emph{métrique}} est assez vaste et de
nombreuses XXX 

Les \emph{métriques} que nous sommes amenés à construire peuvent être
de nature et porter sur ds phénomènes assez différents. Toutefois on
peut identifier une certaine récurrence dans leurs
caractéristiques.

Tout d'abord certaines \emph{métriques} quantifient des phénomènes
exprimés par rapport à \emph{l'objet de référence.}  C'est par exemple
le cas des matrices 4IM, utilisées précédemment pour introduire la
méthodologie de \emph{spatialisation.} En effet, la valeur de la
matrice calculée en un pixel dépend de la position et de la forme de
\emph{l'objet de référence,} si ce dernier change, la \emph{métrique}
sera caduque. Cependant des \emph{métriques} comme la pente ou
l'altitude, utilisées pour \emph{spatialiser} des \emph{indices de
  localisation} tels que : \enquote{Je suis à \SI{2500}{\meter}} ou
\enquote{Nous sommes dans une zone de forte pente}, ne dépendent pas
d'un \emph{objet de référence.} On parle dans ce cas de
\emph{métriques intrinsèques,} que l'on oppose aux \emph{métriques
  extrinsèques,} telles que la distance à \emph{l'objet de référence}
ou une relation topologique.

Un second critère peut être utilisé pour distinguer les
\emph{métriques.} En effet, certaines d'entre elles nécessitent un
paramétrage pour être calculées. C'est par exemple le cas d'une
\emph{métrique} comme le temps de marche, qu'il est nécessaire de
paramétrer en fonction de la vitesse de déplacement.
%
Il peut cependant être délicat de différentier ce qui est rattaché à
la paramétrisation et ce qui se rattache à la métrique
%
Cependant une
grande partie des \emph{métriques} ne nécessitent pas de paramétrage,
c'est notamment le cas de la distance à \emph{l'objet de référence} ou
de l'altitude, déjà citées.

Ces deux critères se combinent et (\autoref{fig:type_metriques})

\subsection{Les différentes \emph{métriques}}

Parmi tous les exemples que nous avons présentés, une \emph{métrique}
apparait régulièrement, la \emph{distance planimétrique} à
\emph{l'objet de référence.} Le processus de calcul de la
\emph{métrique} consiste à calculer, pour chaque pixel de la \ac{zir},
la distance minimale à \emph{l'objet de référence,} c'est-à-dire la
distance au pixel, appartenant à \emph{l'objet de référence} (défini
lors de la \emph{rasterisation}), le plus proche. Cette métrique a été
utilisée lors de la \emph{spatialisation} de \emph{l'indice de
  localisation} \enquote{Je suis sous une ligne électrique}
(\autoref{fig:ISO_DIST_HT}), mais elle est également utilisable pour
spatialiser des relations topologiques, tout en prenant en compte
\emph{l'imprécision} (\eg deux objets très proches, mais qui ne
partagent pas de positions, peuvent, du point vue du requérant, être
considérés comme en contact). Cette \emph{métrique} est un exemple
caractéristique des \emph{métriques extrinsèques} et \emph{non
  paramétriques.} Il est, en effet, nécessaire de la calculer pour
chaque \emph{objet de référence,} mais elle n'accepte pas de
\emph{paramètres.}  On pourrait cependant en ajouter, par exemple en
ajoutant un paramètre permettant d'employer d'autres distances, comme
la distance de Manhattan, de Minkowski ou de Tchebychev. Mais ces
distances alternatives, trop éloignées de la perception humaine des
distances, ne présentent pas d’intérêt pour notre cas d'application.

Une autre \emph{métrique,} également employée lors de la comparaison
des implémentations (\autoref{chap:06}), est la différence d'altitude
entre une position et \emph{l'objet de référence.} Il s'agit d'une
\emph{métrique extrinsèque,} prenant en paramètre la méthode de calcul
du l'altitude de référence, utilisée pour la comparaison. Comme nous
l'expliquions précédemment (\autoref{}), plusieurs méthodes peuvent
être employées pour sélectionner l'altitude à comparer. On peut par
exemple calculer la différence d'altitude à partir de la valeur
minimale, maximale ou moyenne de \emph{l'objet de référence,} ce qui
peut permettre une \emph{spatialisation} plus ou moins
stricte. Cependant, comme nous l'indiquions dans le \autoref{},
l'usage de ces valeurs de référence n'est pas pertinent pour des
objets très étendus, comme des lignes électriques ou des forêts. C'est
pourquoi nous avons choisi de calculer, par défaut, cette différence à
partir du pixel de l'objet de référence, le plus proche. Dans cette
configuration la \emph{métrique} \onto{Delta\-Alt} exprime alors une
différence d'altitude contextuelle et non globale comme elle le ferait
en calculant la différence par rapport au maximum ou au minimum. La
\autoref{fig:metrique_delta_alt} donne un exemple ---~extrait de la
spatialisation du \emph{fil rouge} (\autoref{chap:10}) et de la
comparaison des implémentations (\autoref{chap:06})~--- du calcul de
cette \emph{métrique.} La différence d'altitude est ici calculée par
rapport à l'altitude du pixel de la ligne électrique le plus
proche. Une amélioration possible de cette \emph{métrique} serait
d'ajouter de nouveaux paramètres permettant de comparer l'altitude par
rapport à des extremums locaux. Par exemple, en prenant pour point de
comparaison le minimum (ou le maximum ou la moyenne) de la zone la
plus proche de \emph{l'objet de référence.} Cette approche permettrait
une plus grande finesse de \emph{spatialisation,} en permettant des
configurations plus fines. Cependant la solution retenue par défaut à
l'avantage d'être suffisamment robuste pour fonctionner dans toutes
les situations que nous avons rencontrés et ce sans ajustements
\emph{ad hoc.} La métrique \onto{Delta\-Alt} peut également être
utilisée pour calculer une différence par rapport à une altitude
donnée explicitement, comme dans \emph{l'indice de localisation,}
\enquote{Il est à \SI{300}{\meter} d'altitude.}

\begin{figure}
  \centering
  \begin{tikzpicture}
  \tikzset{et/.style={above,font=\footnotesize\vphantom{Ag}}}
  % 
  \node[inner sep=0pt, anchor=south west] (image) at (0,0){\includegraphics[angle=90]{./figures/Metrique_delta_alt.png}};
  % 
  \begin{scope}
    \node (P2) at ([yshift=-.5cm]image.south east) {};
    \node (P1) at ([yshift=-.5cm]image.south west) {};
    % 
    \node[anchor=west, font=\footnotesize\vphantom{Ag}, text width=8cm] at
    (P1 |- 0cm,-1cm) {Différence d'altitude avec la ligne électrique:};
    % 
    \begin{scope}
      \foreach \x [evaluate=\xshift using 1+\x/10, evaluate=\rad using (\x * -.0008) + .05] in {0,...,50}
      {
        \draw[fill=RdBu-9-9,draw=none, below] ([xshift=\xshift cm, yshift=-1.5cm]P1) circle [radius=\rad cm];
      }
      \foreach \x [evaluate=\xshift using 6+\x/10, evaluate=\rad using (\x * .0008) + .01] in {0,...,50}
      {
        \draw[fill=RdBu-9-1,draw=none, below] ([xshift=\xshift cm, yshift=-1.5cm]P1) circle [radius=\rad cm];
      }
      % 
      \path(1,-2) --++ (10,0)
      node[et,pos=0] {$<$ \SI{-500}{\meter}}
      node[et,pos=.5] {\SI{0}{\meter}}
      node[et,pos=1] {$>$ \SI{500}{\meter}};
    \end{scope}
    % Échelle
    \draw[-] (P2 |- -1cm,-1cm) --++ (-1,0) node[et,pos=.5] {\SI{500}{\meter}};
    % Légende détaillée
    \path (P1) -- (P2) node[pos=.5, yshift=-2cm] {\tiny Pour la légende détaillée du fond topographique voir \autoref{anx:topo_leg}. Sources: BD TOPO 2018, BD ALTI 2018.}; 
  \end{scope}
\end{tikzpicture}
  \caption{Exemple d'une \emph{métrique} issue de la
    \emph{spatialisation} du \emph{fil rouge,} la différence
    d'altitude par rapport au pixel le plus proche de \emph{l'objet de
      référence.}}
  \label{fig:metrique_delta_alt}
\end{figure}

Pour \emph{spatialiser} les \emph{relations de localisation atomiques}
directionnelles, la \emph{métrique} \onto{Ecart\-Angulaire} a été
développée. Il s'agit d'une \emph{métrique} extrinsèque et
paramétrique. \onto{Ecart\-Angulaire} est destinée à mesurer l'écart à
une orientation donnée, fixée par un paramètre \enquote{angle} et
exprimée à partir de \emph{l'objet de référence.} Cette
\emph{métrique} est employée pour \emph{spatialiser} l'ensemble des
\emph{relations de localisation atomiques} décrivant une orientation
générale, comme les relations de cardinalité (\eg
\onto[orla]{Au\-Nord\-De}). La \autoref{fig:metrique_ecart_angulaire},
donne un exemple de cet \emph{métrique.} À partir d'un point donné (au
centre de la figure) et d'une direction (l'angle supérieur droit), on
construit une demi-droite faisant office de direction de
référence. Puis, pour chaque pixel, on calcul l'écart angulaire entre
la direction de référence et la demi-droite reliant \emph{l'objet de
  référence} a ce point. Sur la \autoref{fig:metrique_ecart_angulaire}
l'importance de cet angle est figurée avec une variation de taille, et
son orientation par une variation de couleur. Cette \emph{métrique}
peut-être utilisée pour \emph{spatialiser} différentes \emph{relations
  de localisation atomiques.}

\begin{figure}
  \centering
  \input{../figures/Metrique_ecart_angulaire.tex}
  \caption{Exemple du calcul de la \emph{métrique}
    \protect\onto{Ecart\-Angulaire} pour un point et une direction
    donnée.}
  \label{fig:metrique_ecart_angulaire}
\end{figure}

Bien qu'elle puisse être utilisée à cet effet, la \emph{métrique}
\onto{Ecart\-Angulaire}, n'est pas très adaptée à la
\emph{spatialisation} de \emph{relations de localisations} qualifiant
un déplacement vers un \emph{objet de référence,} comme
\onto[orl]{Dans\-La\-Direction\-De}. Un déplacement dans une direction
globale pouvant conduire à s'éloigner ponctuellement de cette
direction sans que l'on puisse considérer que la \emph{relation de
  localisation spatialisée} soit fausse. Pour prendre en considération
ces cas nous avons développé une \emph{métrique} \emph{ad hoc,}
\onto{Direction\-De}, destinée à quantifier l'éloignement à un
déplacement dans une direction donnée.


%
(\autoref{fig:metrique_direction_de})

\begin{figure}
  \centering
  \begin{tikzpicture}
  \tikzset{et/.style={above,font=\footnotesize\vphantom{Ag}}}
  %
  \node[inner sep=0pt, anchor=south west] (image) at (0,0){\includegraphics{./figures/Metrique_part_lac_visible.png}};
  %
  \begin{scope}
    \node (P2) at ([yshift=-.5cm]image.south east) {};
    \node (P1) at ([yshift=-.5cm]image.south west) {};
    %
    \foreach \x [evaluate=\xshift using \x/10, evaluate=\rad using (\x * .0004) + .01] in {0,...,100}
    {
      \draw[fill=black,draw=none, below] ([xshift=\xshift cm, yshift=-.5cm]P1) circle [radius=\rad cm];
    }
    %
    \path(P1 |- 0cm,-1cm) --++ (10,0)
    node[et,pos=0] {0}
    node[et,pos=.1] {0,1}
    node[et,pos=.2] {0,2}
    node[et,pos=.3] {0,3}
    node[et,pos=.4] {0,4}
    node[et,pos=.65] {0,65}
    node[et,pos=1] {1};
    % Échelle
    \draw[-] (P2 |- -1cm,-1cm) --++ (-1,0) node[et,pos=.5] {\SI{500}{\meter}};
    % Légende détaillée
    \path (P1) -- (P2) node[pos=.5, yshift=-1cm] {\tiny Pour la légende détaillée du fond topographique voir \autoref{anx:topo_leg}. Sources: BD TOPO 2018, BD ALTI 2018.}; 
  \end{scope}
\end{tikzpicture}
  \caption{Exemple du calcul de la \emph{métrique}
    \protect\onto{Direction\-De}}
  \label{fig:metrique_direction_de}
\end{figure}

D'autres \emph{métriques,} plus spécifiques (\ie employées par un
petit nombre de \emph{relations de localisation atomiques}) ont
également été développées. C'est par exemple le cas de
\onto{Temps\-De\-Marche} ou \onto{Part\-Visible}, respectivement
utilisées pour la \emph{spatialisation} des \emph{relations de
  localisation} \onto[orla]{Cible\-Voit\-Site} (et son opposée
\onto[orla]{Site\-Voit\-Cible}) et \onto[orla]{A\-Distance\-Temps}
\footnote{Ces trois \emph{relations de localisation atomiques} peuvent
  être utilisées directement. Elles sont par conséquent également
  présentes dans \ac{orl} (\autoref{anx:orl_dic}).}. La
\emph{métrique} \onto{Temps\-De\-Marche} est une \emph{métrique}
extrinsèque, sans paramètre. Sa fonction est de calculer la durée
minimale de déplacement nécessaire pour atteindre chaque pixel de la
\ac{zir} à partir de l'objet de référence. Pour calculer cette
\emph{métrique} nous avons employé le modèle de \textcite{Tobler1993}
(voir \autoref{eq:marche_tobler} et \autoref{fig:modeles_marche}), ce
dernier présentant l'avantage d'être robuste et facilement
paramétrable en fonction de la nature du terrain.

La \emph{métrique} \onto{Part\-Visible} est également une
\emph{métrique} extrinsèque et non paramétrique. Elle est utilisée
pour spatialiser les relations de visibilité. Le cas de cette
\emph{métrique} est un peu particulier car elle ne 
%
On peut voir une illustration de cette \emph{métrique} sur la
\autoref{fig:metrique_part_lac}. Dans cet exemple la taille du figuré
représente la part de \emph{l'objet de référence} (le lac au centre de
l'image) qui est visible depuis chaque pixel.

\begin{figure}
  \centering
  \input{../figures/Metrique_part_lac_visible.tex}
  \caption{Exemple d'une \emph{métrique} issue de la
    \emph{spatialisation} du \emph{fil rouge :} la part de la surface
    visible d'un lac donné.}
  \label{fig:metrique_part_lac}
\end{figure}

\begin{table}
  \centering
  \begin{tabular}{>{\bfseries}R{3cm}C{5cm}C{5cm}}
  \toprule
  &
   \multicolumn{1}{c}{\bfseries Paramétriques} &
   \multicolumn{1}{c}{\bfseries Non Paramétriques} \\
  \midrule
  Extrinsèques &&Pente\\
  Intrinsèques &&\\
  \bottomrule
\end{tabular}

  \caption{Types de métriques}
  \label{fig:type_metriques}
\end{table}

%%% Local Variables:
%%% mode: latex
%%% TeX-master: "../../../../main"
%%% End:
