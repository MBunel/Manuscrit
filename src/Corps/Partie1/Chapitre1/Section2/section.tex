\subsection{Origine du projet}
\label{subsec:1-2-1}

L’alerte présentée ci-dessus a mis en évidence les limites de la
méthode de localisation des victimes utilisée par le \ac{pghm} et le
risque qu'elles font courir aux victimes. Dans le cas où la solution
\emph{Gend'Loc} n'est pas utilisable, les secouristes doivent opter
pour une localisation manuelle qui, si elle fonctionne la plupart du
temps, peu être inefficace et conduire a la perte d'un temps précieux
---~comme c'est le cas ici. Une réflexion a donc été amorcée au sein
du \ac{pghm} du Grenoble pour trouver des solutions permettant
d'améliorer la gestion de ces cas, qui bien que minoritaires, sont
problématiques. Une des voies suivie a été d'organiser des
collaborations avec des laboratoires de recherche pour mettre en place
une réflexion autour de ces questions.

Une de ces collaborations a abouti à création du projet de recherche
pluridisciplinaire Choucas \footnote{\url{http://choucas.ign.fr/}}, du
sein du quel notre travail de thèse s'inscrit. Ce projet de recherche
implique quatre institutions partenaires :
%
\begin{enumerate*}[label=(\arabic*)]
\item Le \emph{laboratoire en sciences et technologies de
    l'information géographique} (LASTIG)
  \footnote{\url{https://www.umr-lastig.fr/}} de l'institut national
  de l'information géographique et forestière (IGN), qui regroupe des
  chercheurs spécialisés dans la modélisation, la qualification,
  l'enrichissement et la visualisation des données géographiques;
\item l'équipe \emph{probabilités et statistiques} du
  \emph{laboratoire de mathématiques et de leurs applications}
  \footnote{\url{https://lma-umr5142.univ-pau.fr/}} (LMAP), dont les
  membres sont spécialisés dans la recherche autour des modèles
  probabilistes et de l'inférence statistique;
\item L'équipe STeamer \footnote{\url{http://steamer.imag.fr/}} du
  \emph{laboratoire d'informatique de Grenoble} (LIG), travaillant à
  la conception d'outils et de méthodes destinés à la construction de
  systèmes d'information spatio-temporelles; et
\item du \ac{pghm} de Grenoble qui est à la fois un contributeur et
  l'utilisateur final des résultats du projet Choucas
  \autocite{Choucas2020}.
\end{enumerate*}

\subsection{Problématique du projet}
\label{subsec:1-2-2}

L'objectif premier du projet Choucas et de développer des méthodes et
des outils permettant aux secouristes de localiser plus facilement,
rapidement et efficacement des victimes en montagne et ainsi éviter de
reproduire la situation de l'alerte présentée ci-dessus, où la victime
n'a été retrouvée que par chance. De part son statut de
\enquote{déclencheur} du projet de recherche Choucas, mais surtout en
raisons de ses spécificités qui en font un cas d'école
(\autoref{subsec:1-1-2-3}), cette alerte aucune une place centrale
dans le projet Choucas, au point d'être baptisée \enquote{\emph{fil
    rouge}} \autocite{OlteanuRaimond2017}.

À la suite de l'alerte \emph{fil rouge} un processus de réflexion,
visant à identifier les lacunes du processus de localisation des
victimes, a été mis en place. Ce dernier a permis d'identifier
plusieurs problèmes. Le premier d'entre eux est qu'il est difficile,
voir impossible, de traiter une partie des informations données par le
requérant. Les raisons peuvent être multiples. Par exemple le
requérant peut donner une information qui pourrait se révéler
discriminante, mais qu'il est très difficile d'exploiter
manuellement. C'est par exemple le cas lorsque, dans le \emph{fil
  rouge,} la victime indique \enquote{voir une partie de plan
  d'eau}. Si relativement peu de positions correspondent à cet énoncé,
il n'est pas exploitable lors d'une localisation manuelle, puisqu'il
serait nécessaire d'arriver à construire manuellement des zones de
visibilité. Un autre problème se pose lorsque le requérant décrit sa
position à l'aide d'un objet non nommé, mais défini par son type ou sa
nature. C'est par exemple le cas lorsque la victime indique être
\enquote{sous une route} ou être \enquote{partie \textelp{} en
  direction d'une station de ski}. Dans ce cas de très nombreuses
positions peuvent correspondre à une même description et il est
difficile pour un opérateur humain de toutes les identifier et les
vérifier.

Un autre point problématique, bien que moins critique que le
précédent, est qu'il est difficile pour les secouristes de traiter
différentes hypothèses de localisation de la victime en parallèle,
surtout lorsque ces dernières sont complexes. Les secouristes ont donc
tendance à se focaliser sur une même hypothèse, celle qui semble la
plus probable.

\enquote{À l'usage, les idées se valorisent indûment}
\autocite{Bachelard1934}


La première d'entre-t-elle est la richesse sémantique des données à la
disposition des secouristes. En effet, une partie des descriptions
utilisées par les victimes pour décrire leur position peut se référer
à des objets qui n'existent pas dans les bases de données
géographiques utilisées (ou utilisables). De plus, si certains objets
sont présents dans les bases leur qualification n'est pas forcément
suffisamment précise pour que l'on puisse leur correspondre une
description qui en serait donnée par la victime. Il est, par exemple,
possible qu'un chemin soit décrit comme \enquote{pentu}, \enquote{en
  lacets} ou encore \enquote{en balcon}. Ces qualificatifs ne se
retrouvent pas dans les bases de données géographiques et il n'est
donc pas possible de sélectionner des objets suivant ces
qualificatifs. L'enrichissement de cette base de données permettrait
donc d'élargir la richesse des sélections




Le développement d'une solution destinée à rechercher ces
différentes possibilités et de les combiner entre elles, permettrait
de faciliter le travail de localisation. Toutefois nous ne souhaitons
pas que cette solution se substitue au secouriste, ces solutions sont
conçues pour être des outils d'aide à la décision.

\subsection{Objectifs scientifiques du projet}
\label{subsec:1-2-3}

La réponse à la problématique du projet s'est organisée autour de
quatre objectifs scientifiques principaux.

\subsubsection{Structuration des données issues de sources textuelles
  hétérogènes}
\label{subsec:1-2-3-1}

Le premier de ces objectifs est l'extraction et la structuration de
données issues de sources textuelles. Comme nous l’expliquions
précédemment les bases de données géographiques ne contiennent pas
l'ensemble des objets utilisés pour décrire une position en
montagne. Si on y trouve la plupart des sentiers et des sommets, il
est moins fréquent de trouver des représentations de barres rocheuses,
ou de vires, alors qu'il s'agit de points de repères utiles sur le
terrain. L'ajout de ce type d'objet dans la base de données utilisées
par les secours permettrait d’affiner ou de faciliter la localisation
des victimes.

Pour ce faire, il a été proposé d'extraire ce type d'information à
partir de descriptions textuelles d'itinéraires de randonnée. En
effet, la majorité des informations disponible sur les itinéraires de
randonnée est présentée (et diffusée) sous forme de texte. Notamment
par le biais de \emph{topoguides,} de sites collaboratifs
\footnote{Comme le site web \emph{camp2camp.}}, ou encore de
blogs. Les itinéraires y sont généralement présentés sous la forme
d'un texte décrivant principalement les points de bifurcation et les
objets permettant de les repérer, \eg \enquote{prendre à droite à la
  bifurcation située au niveau d'un gros rocher.} Or certains de ces
objets peuvent être absents des bases de données géographiques, que ce
soit à cause de leur taille ou de leur type. Ainsi, certains objets
saillants peuvent être absents des bases de données de référence. Le
développement d'une méthode automatisée d'extraction de ce type
d'objets permettrait donc de compléter les données avec un ensemble
d'objets géographiques utiles dans ce contexte.

\subsubsection{Raisonnement spatial qualitatif flou}
\label{subsec:1-2-3-2}

Le second objectif majeur de ce projet de recherche est l’élaboration
de méthodes permettant d'automatiser le processus de localisation à
partir des informations qui en sont données par les
victimes. C'est-à-dire arriver à identifier la ou les position⋅s
correspondant à une description de, \eg \enquote{je suis en forêt}. Le
principal verrou scientifique de cet objectif et la nature imprécise
du langage naturel, \ie qu'une même description peut être utilisée
dans des contextes différents ou pour décrire des positions qui le
sont tout autant. Il est donc nécessaire de prendre cet aspect en
considération et d'adapter la méthodologie pour qu'elle puisse prendre
en compte l'imprécision.

Un second point essentiel est que, comme nous l'expliquions
précédemment, un parti pris essentiel du projet Choucas est de
développer des solutions \emph{ne se substituant pas} au secouriste,
mais l’assistant. La solution développée doit alors être conçue pour
prendre en compte les ajustements du secouriste et être capable d'être
utilisée en direct par les secouristes lors de la phase de
localisation.

C'est dans cet axe que s'inscrit notre travail doctoral et nous
détaillerons donc cet objectif dans les chapitres suivants.

\subsubsection{Géovisualisation de données multidimensionnelles et
  imparfaites pour la prise de décision}
\label{subsec:1-2-3-3}

Un autre point abordé par ce projet de recherche est la question de la
visualisation des données géographiques et du développement
d'interfaces homme-machine. Pour utiliser les outils d'aide au
raisonnement proposés dans note thèse (et présentés ci-dessus), le
secouriste doit disposer d'une interface adaptée, permettant de
décrire la localisation de la victime et de visualiser les résultats
de l'outil d'aide à la localisation. 

Cette interface de géovisualisation est également destinée à afficher
les résultats produits par l'outil d'aide à la localisation. Cet outil
devant prendre en compte la nature imprécise des descriptions en
langage naturel, il est nécessaire que l'interface puisse représenter
ces résultats sans omettre leur nature imprécise.  

Également confrontée à la question de l'imprécision du langage
naturel, le développement de cette interface nécessite un
questionnement autour de la représentation de données de cette nature
qui seront produites par l'outil. Cet objectif nécessite une réflexion
autour de la question de la représentation de données imprécises,
comme celles qui seront produites lors de notre travail de thèse.

\subsubsection{Intégration de sources hétérogènes spatialisables}
\label{subsec:1-2-3-4}

Enfin le dernier objectif du projet Choucas consiste à travailler à la
structuration de l'ensemble des données dont disposent les
secouristes. Comme nous l'expliquions lors de notre description du
processus de localisation utilisé par les secouristes
(\ref{subsec:1-1-2-1}), ces derniers disposent d'un important corpus
de données. Mais ces dernières sont de types divers et ne sont pas
organisées. Il est, par conséquent difficile de trouver une
information utilise au processus de localisation sans savoir où
chercher. La structuration de ces données offrirait la possibilité de
les interroger conjointement et ainsi de disposer d'un maximum
d'informations, ce qui améliora le processus de localisation.

Le principal verrou scientifique de cette tâche est la prise en compte
de la contradiction entre sources. En effet, deux sources différentes
peuvent placer un même sommet à des endroits différents, il est donc
nécessaire d'être capable de traiter ces cas.

% Tableau synthétique
\begin{landscape}
\begin{table}[H]
  \centering
  \begin{tabular}{p{.15\textheight}>{\small}L{.35\textheight}>{\small}L{.4\textheight}}
  \toprule
  \multicolumn{1}{c}{\normalsize\bfseries Objectif scientifique} &
  \multicolumn{1}{c}{\normalsize\bfseries Verrous scientifiques} &
  \multicolumn{1}{c}{\normalsize\bfseries Apports envisagés} \\ \midrule
  % Lot 1
  \nameref{subsec:1-2-3-1}
& \begin{minipage}[t]{.35\textheight}
    \begin{itemize}
    \item Annotation, catégorisation et désambiguïsation des entités nommées
    \item Extraction et interprétation des relations sémantiques
    \end{itemize}
  \end{minipage} & \begin{minipage}[t]{.4\textheight}
    \begin{itemize}
    \item Définition d'une méthode d'annotation des entités nommées
    \item Définition d'une méthode permettant l'interprétation des
      relations sémantiques
    \end{itemize}
  \end{minipage} \\
  \addlinespace[.5cm]
  %
  \nameref{subsec:1-2-3-2}
  {\par\footnotesize\hspace{.25cm}$\longrightarrow$~Chapitre \ref{chap:02}}
& \begin{minipage}[t]{.35\textheight}
    \begin{itemize}
    \item Spatialisation des indices de localisation
    \item Modélisation des objets géographiques imprécis
    \end{itemize}
  \end{minipage}& \begin{minipage}[t]{.4\textheight}
    \begin{itemize}
    \item Définition d'une méthode de spatialisation
    \item Définition d'une méthode de prise en compte de l'imprécision
      des objets géographiques
    \end{itemize}
  \end{minipage} \\
  \addlinespace[.5cm]
  % Lot 
  \nameref{subsec:1-2-3-3}
& \begin{minipage}[t]{.35\textheight}
    \begin{itemize}
    \item Définir des méthodes permettant d'assister le secouriste
      dans sa prise de décision
    \item Visualisation de données imprécises
    \end{itemize}
  \end{minipage}& \begin{minipage}[t]{.4\textheight}
    \begin{itemize}
    \item Construction d'une interface d'aide à la décision
    \item Définition de méthodes permettant de prendre en compte
      l'imprécision lors de la visualisation
    \end{itemize}
  \end{minipage} \\
  \addlinespace[.5cm]
  %
  \nameref{subsec:1-2-3-4}
& \begin{minipage}[t]{.35\textheight}
    \begin{itemize}
    \item Structuration d'un référentiel de données métier
    \item Appareillement de données hétérogènes
    \end{itemize}
  \end{minipage}& \begin{minipage}[t]{.4\textheight}
    \begin{itemize}
    \item Définition de méthodes d'appareillement
    \item Construction d'un référentiel de données métier
    \end{itemize}
  \end{minipage}\\
  \bottomrule
\end{tabular}

  \caption{Synthèse des verrous et des apports attendus pour chaque
    objectif scientifique du projet Choucas}
  \label{tab:synthese_objectifs_choucas}
\end{table}
\end{landscape}

%%% Local Variables:
%%% mode: latex
%%% TeX-master: "../../../../main"
%%% End:
