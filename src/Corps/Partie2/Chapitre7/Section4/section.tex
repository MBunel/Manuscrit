Une fois que la \emph{métrique} identifiée pour la
\emph{spatialisation} de la \emph{relation de localisation atomique} a
été calculée, il est nécessaire de la \emph{fuzzyfier} pour construire
la \emph{zone de localisation compatible.} Comme nous l'expliquions
lors de la définition de la \emph{méthode de spatialisation}
(\autoref{chap:07-sec1}), la \emph{fuzzyfication} a deux rôles, le
premier ---~illustré par la
\autoref{fig:Exemple_Metrique_vs_Selecteur}~--- consiste à
sélectionner les pixels pour lesquels la valeur de la \emph{métrique}
correspond à ce qui est attendu au sein de la \ac{zlc}, le second est
produire une \emph{zone de localisation compatible} floue,
transcrivant \emph{l'imprécision} de la \emph{relation de localisation
  atomique.}

\subsection{la méthode de \emph{fuzzyfication}}

Un simple filtre binaire, comme celui utilisé pour l'exemple de la
\autoref{fig:Exemple_Metrique_vs_Selecteur}, ne permet pas de remplir
ces deux fonctions, nous devons donc étendre cette méthode pour lui
permettre de générer des \ac{zlc} floues. Pour ce faire on peut se
tourner vers la sélection floue

La \autoref{fig:importance_fuzzyfication} présente une version
améliorée de l'exemple de la
\autoref{fig:Exemple_Metrique_vs_Selecteur}. La \emph{métrique} a été
remplacée par la distance euclidienne
%
Les valeurs de la \emph{métrique} sont alors filtrées à l'aide de deux
fonctions d'appartenance, transformant les valeurs de la
\emph{métrique} \footnote{Dont le domaine est \(\mathbb{R}^+\).} en un
\emph{degré d'appartenance.} La première \emph{fonction
  d'appartenance} (définissant la \ac{zlc}
\textcolor{RdBu-9-1}{\textsf{A}}) attribue un degré d'appartenance
maximal lorsque la distance \footnote{Exprimée en centimètres à
  l'échelle de la feuille.} est inférieure ou égale à
\SI{0,25}{\centi\meter} (soit le côté d'un pixel), puis le degré
d'appartenance décroit jusqu'à atteindre 0 lorsque la distance est de
\SI{2}{\centi\meter} ou plus. La seconde \emph{fonction
  d'appartenance} (cf. \ac{zlc} \textcolor{RdBu-9-9}{\textsf{B}})
attribue un degré nul jusqu'à un éloignement de
\SI{0,75}{\centi\meter}, puis croit jusqu'à atteindre la valeur de 1
pour les éloignements supérieurs ou égaux à
\SI{1,50}{\centi\meter}. Comme on peut le constater, les \emph{zones
  de localisation compatibles}, \textcolor{RdBu-9-1}{\textsf{A}} et
\textcolor{RdBu-9-9}{\textsf{B}}, résultant de ces deux
\emph{fuzzyfications} ont une structure spatiale différente, pouvant
approcher la sémantique de certaines \emph{relations de localisation
  atomiques.} La \emph{zone de localisation compatible}
\textcolor{RdBu-9-1}{\textsf{A}} peut, par exemple, être rapprochée de
la sémantique de la \emph{relation de localisation atomique}
\onto[orla]{Pres\-De}, alors que la \ac{zlc}
\textcolor{RdBu-9-9}{\textsf{B}} renvoie plutôt à une modélisation
floue d'une \emph{relation} telle que \onto[orla]{Exterieur\-De}.
%
Ainsi, comme le montre cet exemple, on peut étendre la logique de
\emph{spatialisation} en trois étapes utilisée pour construire la
\autoref{fig:Exemple_Metrique_vs_Selecteur} à un cadre flou.

% forme = sémantique, seuils = implémentation
Une autre conclusion que l'on peut tirer de la
\autoref{fig:importance_fuzzyfication} est la faible importance de la
valeur des seuils, comparativement à la forme globale de la
\emph{fonction d'appartenance.}



Comme nous le mentions précédemment (\autoref{chap:05}), le choix de
la forme de la fonction d'appartenance est essentiel à la
\emph{spatialisation,} puisqu'il
%
Ainsi, la sémantique est dans la forme de la fonction d'appartenance.
%
La question de la définition des seuils de la fonction d'appartenance
utilisée est également importante. Toutefois, il s'agit

\begin{figure}
  \centering
  \begin{tikzpicture}
  \def\decalageX{-.2}
  \def\decalageY{-.2}
  % Arrow
  \begin{scope}
    \path[draw, -, shorten >=12pt, shorten <=5pt] (1,2) |- (3.65,3);
    \path[draw, ->,shorten >=5pt, shorten <=5pt] (8.25,3) -- (10,3);
    \path[draw, -,shorten >=12pt, shorten <=5pt] (1,0) |- (3.65,-1);
    \path[draw, ->,shorten >=5pt, shorten <=5pt] (8.25,-1) -- (10,-1);
  \end{scope}
  % Représentation métrique
  \begin{scope}[local bounding box=met]
    % \path[ffa] (1,.25) rectangle (1.25, .5);
    \begin{scope}
      \foreach \x in {0,.25,...,1.75}{
        \foreach \y in {0,.25,...,1.75}
        {
          % Traçage pixels 
          \path[draw, line width=.01mm] (\x,\y) rectangle (\x +.25, \y +
          .25);
          % Calcul et représentation distance euclidienne
          \pgfmathsetmacro\radius{sqrt((\x-1.125)^2+(\y-.375)^2)*1.25}
          \fill[black] (\x+.125,.125+\y) circle (\radius pt);
        }
      }
      \path[ffc, black] (0,0) rectangle (2,2);
    \end{scope}
    \node[text width=3cm, align=center, anchor=north, font=\footnotesize,fill=white] at (1,-.25)
    {\itshape Métrique};
  \end{scope}
  % Fuzzyfication 1
  \begin{scope}[xshift=10cm, yshift=2cm,local bounding box=fuzz]
    \begin{scope}
      \foreach \x in {0,.25,...,1.75}{
        \foreach \y in {0,.25,...,1.75}
        {
          % Traçage pixels 
          \path[draw, line width=.01mm] (\x,\y) rectangle (\x +.25, \y +
          .25);
          % Calcul et représentation floue distance euclidienne
          \pgfmathsetmacro\dist{sqrt((\x-1.125)^2+(\y-.375)^2)}
          % Fuzzyfication de la distance
          \pgfmathsetmacro\fuzzy{%
            ifthenelse(\dist < .25,1,%
            ifthenelse(\dist < 2,-0.57*\dist+1.14,0)%
            )
          }
          % Calcul du rayon à partir de la fuzzyfication
          \pgfmathsetmacro\radius{\fuzzy*2.5}
          \fill[RdBu-9-1] (\x+.125,.125+\y) circle (\radius pt);
        }
      }
      \path[ffc] (0,0) rectangle (2,2);
    \end{scope}
    \node[text width=3cm, align=center, anchor=north, font=\footnotesize] at (1,-.25)
    {\itshape Zone de localisation compatible \normalfont \textcolor{RdBu-9-1}{A}};
  \end{scope} 
  % Fuzzyfication 2
  \begin{scope}[xshift=10cm, yshift=-2cm,local bounding box=fuzz2]
    \begin{scope}
      \foreach \x in {0,.25,...,1.75}{
        \foreach \y in {0,.25,...,1.75}
        {
          % Traçage pixels 
          \path[draw, line width=.01mm] (\x,\y) rectangle (\x +.25, \y +
          .25);
          % Calcul et représentation floue distance euclidienne
          \pgfmathsetmacro\dist{sqrt((\x-1.125)^2+(\y-.375)^2)}
          % Fuzzyfication de la distance
          \pgfmathsetmacro\fuzzy{%
            ifthenelse(\dist < .75,0,%
            ifthenelse(\dist < 1.75,\dist-.75,1)%
            )
          }
          % Calcul du rayon à partir de la fuzzyfication
          \pgfmathsetmacro\radius{\fuzzy*2.5}
          \fill[RdBu-9-9] (\x+.125,.125+\y) circle (\radius pt);
        }
      }
      \path[ffc2] (0,0) rectangle (2,2);
    \end{scope}
    \node[text width=3cm, align=center, anchor=north, font=\footnotesize] at (1,-.25)
    {\itshape Zone de localisation compatible \normalfont \textcolor{RdBu-9-9}{B}};
  \end{scope}
  \begin{scope}[xshift=3.75cm,yshift=2.5cm, scale=.5]
    % Courbe
    \begin{scope}[transparency group]
      % fond
      \begin{scope}
        \path[ffa]  (0,2) -- (1, 2)  -- (8,0) -- (0,0) -- cycle;
      \end{scope}
      % bords
      \begin{scope}
        \path[ffc] (0,2) -- (1,2) -- (8,0);
        \path[ffc_fade] (8,0) -- (9,0) ;
      \end{scope}
    \end{scope}
    % Axes X, Y
    \begin{scope}
      % Axe X
      \begin{scope}
        % Axe
        \draw[->] (0, \decalageX) --++ (9, 0) coordinate (x axis);
        % Graduations
        \foreach \n/\t in {0/,{0},1/{0,25},2/{},3/{},4/{},5/{},6/{},7/{},8/{}}
        {
          \draw[-] (\n, \decalageX - .05) --++ (0, .1);
          \node[below, font=\tiny] at (\n, \decalageX - .05) {\t};
        }
        % label
        \node[below left, font=\tiny] at (x axis) {\itshape Distance
          \normalfont (cm)};
      \end{scope}
      % Axe Y
      \begin{scope}
        % Axe
        \draw[-] (\decalageY ,0) --++ (0, 2) coordinate (y axis);
        % Graduations
        \foreach \n/\t in {0/{0},2/{1}}
        {
          \draw[-] (\decalageY -.05, \n) --++ (.1, 0);
          \node[left, font=\tiny] at (\decalageY -.05, \n) {\t};
        }
        % Label
        \node[above, font=\tiny] at (y axis) {$\mu$};
      \end{scope}
    \end{scope}
  \end{scope}
  \begin{scope}[xshift=3.75cm,yshift=-1.5cm, scale=.5]
    % Courbe
    \begin{scope}[transparency group]
      % fond
      \begin{scope}
        \path[ffa2]  (3,0) -- (6, 2)  -- (8,2) -- (8,0) -- cycle;
        \path[ffa2_fade]  (8,2) -- (9, 2)  -- (9,0) -- (8,0) -- cycle;
      \end{scope}
      % bords
      \begin{scope}
        \path[ffc2] (0,0) -- (3,0) -- (6, 2) -++ (3,0) ;
        \path[ffc2_fade] (8,2) -- (9,2) ;
      \end{scope}
    \end{scope}
    % Axes X, Y
    \begin{scope}
      % Axe X
      \begin{scope}
        % Axe
        \draw[->] (0, \decalageX) --++ (9, 0) coordinate (x axis);
        % Graduations
        \foreach \n/\t in {0/,{0},1/{0.25},2/{},3/{},4/{},5/{},6/{},7/{},8/{}}
        {
          \draw[-] (\n, \decalageX - .05) --++ (0, .1);
          \node[below, font=\tiny] at (\n, \decalageX - .05) {\t};
        }
        % label
        \node[below left, font=\tiny] at (x axis) {\itshape Distance
          \normalfont (cm)};
      \end{scope}
      % Axe Y
      \begin{scope}
        % Axe
        \draw[-] (\decalageY ,0) --++ (0, 2) coordinate (y axis);
        % Graduations
        \foreach \n/\t in {0/{0},2/{1}}
        {
          \draw[-] (\decalageY -.05, \n) --++ (.1, 0);
          \node[left, font=\tiny] at (\decalageY -.05, \n) {\t};
        }
        % Label
        \node[above, font=\tiny] at (y axis) {$\mu$};
      \end{scope}
    \end{scope}
  \end{scope}
\end{tikzpicture}
  \caption{Illustration de la construction de \emph{zones de
      localisation compatibles} à partir d'une même \emph{métrique} et
    de \emph{fuzzyficateurs} différents.}
  \label{fig:importance_fuzzyfication}
\end{figure}

\subsection{Les différents \emph{fuzzyficateurs}}

Nous proposons la définition de trois fonctions de
\emph{fuzzyfication} principales, déclinées en plusieurs variantes,
portant le total des \emph{fuzzyficateurs} à sept.

Le premier d'entre-eux est le \emph{fuzzyficateur} \onto{Eq\-Val}, qui
défini une fonction d'appartenance triangulaire (Figure
\ref{fig:select_eqval_b}). L'objectif de ce \emph{fuzzyfieur} est de
décrire des situations où la \emph{zone de localisation compatible}
est définie par un ensemble de pixels dont la valeur de la métrique
est égale à une valeur donnée. C'est par exemple le cas pour une
\emph{relation de localisation atomique} comme
\onto[orla]{Distance\-Quantitative\-Planimétrique}, définissant une
\ac{zlc} située à un éloignement donné de \emph{l'objet de référence}
(\eg \enquote{à \SI{50}{\meter}}), ou pour la \emph{relation}
\onto[orla]{Alt\-Eq}, définissant une \ac{zlc} située à une altitude
donnée. La forme triangulaire de cette \emph{fonction d'appartenance,}
permet d'attribuer un degré d'appartenance non nul aux pixels ayant
une valeur de la \emph{métrique} proche de celle attendue. Ainsi, si
l'on souhaite construire la \ac{zlc} à 100 mètres de \emph{l'objet de
  référence,} on pourra, en fonction des seuils fixés, inclure des
positions situées à 75 ou 110 mètres de \emph{l'objet de référence.}
De la même manière ce \emph{fuzzyfieur} permet de prendre en
considération des pixels situés au dessus, ou en dessous, d'une
altitude donnée. La \enquote{largeur} de la fonction est définie par
un paramètre \(delta\), indiquant l'écart entre XX et XX. Ce paramètre
définit les seuils

La fonction d'appartenance du \emph{fuzzyfieur} \onto{Eq\-Val} est
donc définie de la manière suivante :

\begin{equation}
  \label{eq:eq_val}
  \def\arraystretch{1.35}
  f(x,v,\delta) = \left\{
    \begin{array}{cl}
      \dfrac{x - x_1}{\Delta x} & \text{si}\ v - \frac{\delta}{2} ≤ x ≤ v\\
      \dfrac{x_2 - x}{\Delta x} & \text{si}\ v ≤ x ≤ v + \frac{\delta}{2}\\
      0 & \text{si}\ x < v-\frac{\delta}{2}\ \text{ou}\ x > v+\frac{\delta}{2}\\
    \end{array}
  \right.
\end{equation}

Avec \(x\) la valeur de la \emph{métrique,} \(v\) la valeur de
référence et \(\delta\) l'écart xxx.

Au fil de la \emph{spatialisation} de nouvelles \emph{relations de
  localisation atomiques} il s'est avéré que nous recourions
fréquemment à des métriques relatives, comme la différence d'altitude
par rapport à \emph{l'objet de référence,} c'est pourquoi nous avons
construit un second \emph{fuzzyfieur,} \onto{EqVal0}, spécialisant le
premier \autoref{fig:select_eqval_0}. Il s'agit également d'une
\emph{fonction de fuzzyfication} triangulaire et symétrique, dont la
\enquote{largeur} est définie par un paramètre \(\delta\). Cependant
son paramètre \(v\), indiquant la valeur de la \emph{métrique} sur
laquelle la courbe est centrée est fixé à 0, sa fonction
d'appartenance est donc : \(f(x,\delta) = f(x,0,\delta)\).. Nous avons
également défini une seconde variante de ce \emph{fuzzyfieur,}
\onto{Eq\-Angle}, fortement similaire à \onto{Eq\-Val}, mais destinée
aux cas spécifiques où la \emph{métrique} est un angle. Ce
\emph{fuzzyfieur} permet donc de traiter ces valeurs en prenant en
compte leur périodicité. \todo{Ce sélecteur ne devrait pas exister,
  c'est à la fonction python d'identifier si ce qu'on lui donne est
  angulaire ou non, pas à l'ontologie. Voir si je l'enlève}.

\begin{figure}
  \centering
  \subfloat[\emph{Fonction d'appartenance} pour \texttt{Eq\-Val}]{
    \begin{tikzpicture}[scale=.7]
  \def\decalageX{-.2}
  \def\decalageY{-.2}
  % Courbe
  \begin{scope}[transparency group]
    % fond
    \begin{scope}
      \path[ffa]  (3,0) -- (4.5, 2) -- (6,0) -- cycle;
    \end{scope}
    % bords
    \begin{scope}
      \path[ffc] (1,0) -- (3,0) -- (4.5, 2) -- (6,0) -- (8,0) ;
      \path[ffc_fade_m] (0,0) -- (1,0) ;
      \path[ffc_fade] (8,0) -- (9,0) ;
    \end{scope}
  \end{scope}
  % Axes X, Y
  \begin{scope}
    % Axe X
    \begin{scope}
      % Axe
      \draw[<->] (0, \decalageX) --++ (9, 0) coordinate (x axis);
      % Graduations
      \foreach \n/\t in {1/{},2/{},3/{},4/{},5/{},6/{},7/{},8/{}}
      {
        \draw[-] (\n, \decalageX - .05) --++ (0, .1);
        \node[below, font=\footnotesize] at (\n, \decalageX - .05) {\t};
      }
      % label
      \node[below left, yshift=-.1cm, font=\small] at (x axis) {\itshape Métrique};
    \end{scope}
    % Axe Y
    \begin{scope}
      % Axe
      \draw[-] (\decalageY ,0) --++ (0, 2) coordinate (y axis);
      % Graduations
      \foreach \n/\t in {0/{0},2/{1}}
      {
        \draw[-] (\decalageY -.05, \n) --++ (.1, 0);
        \node[left, font=\footnotesize] at (\decalageY -.05, \n) {\t};
      }
      % Label
      \node[above] at (y axis) {$\mu$};
    \end{scope}
  \end{scope}
  \begin{scope}
    % Seuil 1
    \draw[ffc,line width=.5] (3,\decalageY) -- (3,0);
    \draw[fill, RdBu-9-1] (3,\decalageY) circle (2pt);
    \draw[fill, RdBu-9-1] (3,0) circle (2pt);
    % Seuil 2
    \draw[ffc,line width=.5] (4.5,\decalageY) -- (4.5,2);
    \draw[fill, RdBu-9-1] (4.5,\decalageY) circle (2pt);
    \draw[fill, RdBu-9-1] (4.5,2) circle (2pt);
    \node[above] at (4.5,2) {\(v\)};
    % Seuil 3
    \draw[ffc,line width=.5] (6,\decalageY) -- (6,0);
    \draw[fill, RdBu-9-1] (6,\decalageY) circle (2pt);
    \draw[fill, RdBu-9-1] (6,0) circle (2pt);
    \draw[|-|] (3,-.7cm) --++(3,0) node[pos=.5, fill=white, inner
    sep=1pt, font=\small] {$\delta$};
  \end{scope}
\end{tikzpicture}

    \label{fig:select_eqval_b}
  }
  
  \subfloat[\emph{Fonction d'appartenance} pour \texttt{Eq\-Val\-0}]{
    \begin{tikzpicture}[scale=.7]
  \def\decalageX{-.2}
  \def\decalageY{-.2}
  % Courbe
  \begin{scope}[transparency group]
    % fond
    \begin{scope}
      \path[ffa]  (3,0) -- (4.5, 2) -- (6,0) -- cycle;
    \end{scope}
    % bords
    \begin{scope}
      \path[ffc] (1,0) -- (3,0) -- (4.5, 2) -- (6,0) -- (8,0) ;
      \path[ffc_fade_m] (0,0) -- (1,0) ;
      \path[ffc_fade] (8,0) -- (9,0) ;
    \end{scope}
  \end{scope}
  % Axes X, Y
  \begin{scope}
    % Axe X
    \begin{scope}
      % Axe
      \draw[<->] (0, \decalageX) --++ (9, 0) coordinate (x axis);
      % Graduations
      \foreach \n/\t in {1/{},2/{},3/{},4/{},5/{},6/{},7/{},8/{}}
      {
        \draw[-] (\n, \decalageX - .05) --++ (0, .1);
        \node[below, font=\footnotesize] at (\n, \decalageX - .05) {\t};
      }
      % label
      \node[below left, yshift=-.1cm, font=\small] at (x axis) {\itshape Métrique};
    \end{scope}
    % Axe Y
    \begin{scope}
      % Axe
      \draw[-] (\decalageY ,0) --++ (0, 2) coordinate (y axis);
      % Graduations
      \foreach \n/\t in {0/{0},2/{1}}
      {
        \draw[-] (\decalageY -.05, \n) --++ (.1, 0);
        \node[left, font=\footnotesize] at (\decalageY -.05, \n) {\t};
      }
      % Label
      \node[above] at (y axis) {$\mu$};
    \end{scope}
  \end{scope}
  \begin{scope}
    % Seuil 1
    \draw[ffc,line width=.5] (3,\decalageY) -- (3,0);
    \draw[fill, RdBu-9-1] (3,\decalageY) circle (2pt);
    \draw[fill, RdBu-9-1] (3,0) circle (2pt);
    % Seuil 2
    \draw[ffc,line width=.5] (4.5,\decalageY) -- (4.5,2);
    \draw[fill, RdBu-9-1] (4.5,\decalageY) circle (2pt);
    \draw[fill, RdBu-9-1] (4.5,2) circle (2pt);
    \node[above] at (4.5,2) {\(v\)};
    % Seuil 3
    \draw[ffc,line width=.5] (6,\decalageY) -- (6,0);
    \draw[fill, RdBu-9-1] (6,\decalageY) circle (2pt);
    \draw[fill, RdBu-9-1] (6,0) circle (2pt);
    \draw[|-|] (3,-.7cm) --++(3,0) node[pos=.5, fill=white, inner
    sep=1pt, font=\small] {$\delta$};
  \end{scope}
\end{tikzpicture}

    \label{fig:select_eqval_0}
  }\hfill  
  \subfloat[\emph{Fonction d'appartenance} pour \texttt{Eq\-Angle}]{
    \begin{tikzpicture}[scale=.7]
  \def\decalageX{-.2}
  \def\decalageY{-.2}
  % Courbe
  \begin{scope}[transparency group]
    % fond
    \begin{scope}
      \path[ffa]  (3,0) -- (4.5, 2) -- (6,0) -- cycle;
    \end{scope}
    % bords
    \begin{scope}
      \path[ffc] (1,0) -- (3,0) -- (4.5, 2) -- (6,0) -- (8,0) ;
      \path[ffc_fade_m] (0,0) -- (1,0) ;
      \path[ffc_fade] (8,0) -- (9,0) ;
    \end{scope}
  \end{scope}
  % Axes X, Y
  \begin{scope}
    % Axe X
    \begin{scope}
      % Axe
      \draw[<->] (0, \decalageX) --++ (9, 0) coordinate (x axis);
      % Graduations
      \foreach \n/\t in {1/{},2/{},3/{},4/{},5/{},6/{},7/{},8/{}}
      {
        \draw[-] (\n, \decalageX - .05) --++ (0, .1);
        \node[below, font=\footnotesize] at (\n, \decalageX - .05) {\t};
      }
      % label
      \node[below left, yshift=-.1cm, font=\small] at (x axis) {\itshape Métrique};
    \end{scope}
    % Axe Y
    \begin{scope}
      % Axe
      \draw[-] (\decalageY ,0) --++ (0, 2) coordinate (y axis);
      % Graduations
      \foreach \n/\t in {0/{0},2/{1}}
      {
        \draw[-] (\decalageY -.05, \n) --++ (.1, 0);
        \node[left, font=\footnotesize] at (\decalageY -.05, \n) {\t};
      }
      % Label
      \node[above] at (y axis) {$\mu$};
    \end{scope}
  \end{scope}
  \begin{scope}
    % Seuil 1
    \draw[ffc,line width=.5] (3,\decalageY) -- (3,0);
    \draw[fill, RdBu-9-1] (3,\decalageY) circle (2pt);
    \draw[fill, RdBu-9-1] (3,0) circle (2pt);
    % Seuil 2
    \draw[ffc,line width=.5] (4.5,\decalageY) -- (4.5,2);
    \draw[fill, RdBu-9-1] (4.5,\decalageY) circle (2pt);
    \draw[fill, RdBu-9-1] (4.5,2) circle (2pt);
    \node[above] at (4.5,2) {\(v\)};
    % Seuil 3
    \draw[ffc,line width=.5] (6,\decalageY) -- (6,0);
    \draw[fill, RdBu-9-1] (6,\decalageY) circle (2pt);
    \draw[fill, RdBu-9-1] (6,0) circle (2pt);
    \draw[|-|] (3,-.7cm) --++(3,0) node[pos=.5, fill=white, inner
    sep=1pt, font=\small] {$\delta$};
  \end{scope}
\end{tikzpicture}

    \label{fig:select_eqval_ang}
  }  
  \caption{\emph{Fonctions d'appartenance} du \emph{fuzzyficateur}
    \protect\onto{Eq\-Val} et de ses dérivés \protect\onto{Eq\-Val\-0}
    et \protect\onto{Eq\-Angle}.}
  \label{fig:select_eqval}
\end{figure}

Le second \emph{fuzzyfieur,} \onto{Sup\-Val}, est destiné à
représenter des situations où les valeurs de la \emph{métrique}
doivent être supérieures à un seuil \(v\). On considère que toutes les
valeurs supérieures ou égale à ce seuil ont un degré d'appartenance
maximal. À l'inverse on tolère des valeurs inférieures à ce seuil,
dans la mesure d'un paramètre \(\delta\). Ce fuzzyfieur prend la forme
d'une fonction d'appartenance linéaire (Figure
\ref{fig:select_supval_b}), dont l'équation est :

% \dfrac{\Delta y}{\Delta x}x + \dfrac{x_2y_1 - x_1y_2}{\Delta x}

\begin{equation}
  \label{eq:sup_val}
  \def\arraystretch{1.25}
  f(x,v,\delta) = \left\{
    \begin{array}{cl}
      1 & \text{si}\ x > v \\
      \dfrac{x - x_1}{\Delta x} & \text{si}\  v - \delta ≤ x ≤ v \\
      0 & \text{si}\ x < v - \delta\\
    \end{array}
  \right.
\end{equation}

Avec \(x\) la valeur de la \emph{métrique,} \(v\) la valeur référence
et \(\delta\) xxxxx.

Comme pour le \emph{fuzzyfieur} \onto{Eq\-Val} une variante dont la
valeur de référence est fixée à 0 a été définie, \onto{Sup\-Val\-0} et dont la fonction d'appartenance est donc : \(f(x,\delta) = f(x,0,\delta)\).

\begin{figure}
  \centering
  \subfloat[\emph{Fonction d'appartenance} pour \texttt{Sup\-Val}]{
    \begin{tikzpicture}[scale=.7]
  \def\decalageX{-.2}
  \def\decalageY{-.2}
  % Courbe
  \begin{scope}[transparency group]
    % fond
    \begin{scope}
      \path[ffa]  (3,0) -- (6, 2)  -- (8,2) -- (8,0) -- cycle;
      \path[ffa_fade]  (8,2) -- (9, 2)  -- (9,0) -- (8,0) -- cycle;
    \end{scope}
    % bords
    \begin{scope}
      \path[ffc_fade_m] (0,0) -- (2,0) ;
      \path[ffc] (2,0) -- (3,0) -- (6, 2) -++ (3,0) ;
      \path[ffc_fade] (8,2) -- (9,2) ;
    \end{scope}
  \end{scope}
  % Axes X, Y
  \begin{scope}
    % Axe X
    \begin{scope}
      % Axe
      \draw[<->] (0, \decalageX) --++ (9, 0) coordinate (x axis);
      % Graduations
      \foreach \n/\t in {1/{},2/{},3/{},4/{},5/{},6/{},7/{},8/{}}
      {
        \draw[-] (\n, \decalageX - .05) --++ (0, .1);
        \node[below, font=\footnotesize] at (\n, \decalageX - .05) {\t};
      }
      % label
      \node[below left, yshift=-.1cm, font=\small] at (x axis) {\itshape Métrique};
    \end{scope}
    % Axe Y
    \begin{scope}
      % Axe
      \draw[-] (\decalageY ,0) --++ (0, 2) coordinate (y axis);
      % Graduations
      \foreach \n/\t in {0/{0},2/{1}}
      {
        \draw[-] (\decalageY -.05, \n) --++ (.1, 0);
        \node[left, font=\footnotesize] at (\decalageY -.05, \n) {\t};
      }
      % Label
      \node[above] at (y axis) {$\mu$};
    \end{scope}
  \end{scope}
  \begin{scope}
    \draw (6,\decalageY) -- (6,2);
    \draw[fill] (6,\decalageY) circle (1pt);
    \draw[fill] (6,2) circle (1pt);
    \draw[fill] (3,0) circle (1pt);
    \node[above] at (6,2) {P};
    %
    \draw[|-|] (3,-.7cm) --++(3,0) node[pos=.5, fill=white, inner
    sep=1pt, font=\small] {$\delta$};
  \end{scope}
\end{tikzpicture}

    \label{fig:select_supval_b}
  }\hfill
  \subfloat[\emph{Fonction d'appartenance} pour \texttt{Sup\-Val\-0}]{
    \begin{tikzpicture}[scale=.7]
  \def\decalageX{-.2}
  \def\decalageY{-.2}
  % Courbe
  \begin{scope}[transparency group]
    % fond
    \begin{scope}
      \path[ffa]  (3,0) -- (6, 2)  -- (8,2) -- (8,0) -- cycle;
      \path[ffa_fade]  (8,2) -- (9, 2)  -- (9,0) -- (8,0) -- cycle;
    \end{scope}
    % bords
    \begin{scope}
      \path[ffc_fade_m] (0,0) -- (2,0) ;
      \path[ffc] (2,0) -- (3,0) -- (6, 2) -++ (3,0) ;
      \path[ffc_fade] (8,2) -- (9,2) ;
    \end{scope}
  \end{scope}
  % Axes X, Y
  \begin{scope}
    % Axe X
    \begin{scope}
      % Axe
      \draw[<->] (0, \decalageX) --++ (9, 0) coordinate (x axis);
      % Graduations
      \foreach \n/\t in {1/{},2/{},3/{},4/{},5/{},6/{},7/{},8/{}}
      {
        \draw[-] (\n, \decalageX - .05) --++ (0, .1);
        \node[below, font=\footnotesize] at (\n, \decalageX - .05) {\t};
      }
      % label
      \node[below left, yshift=-.1cm, font=\small] at (x axis) {\itshape Métrique};
    \end{scope}
    % Axe Y
    \begin{scope}
      % Axe
      \draw[-] (\decalageY ,0) --++ (0, 2) coordinate (y axis);
      % Graduations
      \foreach \n/\t in {0/{0},2/{1}}
      {
        \draw[-] (\decalageY -.05, \n) --++ (.1, 0);
        \node[left, font=\footnotesize] at (\decalageY -.05, \n) {\t};
      }
      % Label
      \node[above] at (y axis) {$\mu$};
    \end{scope}
  \end{scope}
  \begin{scope}
    \draw (6,\decalageY) -- (6,2);
    \draw[fill] (6,\decalageY) circle (1pt);
    \draw[fill] (6,2) circle (1pt);
    \draw[fill] (3,0) circle (1pt);
    \node[above] at (6,2) {P};
    %
    \draw[|-|] (3,-.7cm) --++(3,0) node[pos=.5, fill=white, inner
    sep=1pt, font=\small] {$\delta$};
  \end{scope}
\end{tikzpicture}

    \label{fig:select_supval_0}
  }
  \caption{\emph{Fonctions d'appartenance} du \emph{fuzzyficateur}
    \protect\onto{Sup\-Val} et de son dérivé
    \protect\onto{Sup\-Val\-0}.}
  \label{fig:select_supval}
\end{figure}

Le dernier \emph{fuzzyfieur,} \onto{Inf\-Val}, est l'exact opposé de
\onto{Sup\-Val}. Son objectif est de représenter des situations où
l'on souhaite sélectionner les pixels dont la \emph{métrique} est
inférieure à un seuil \(v\), donné. Toutes les valeurs inférieures à
ce seuil ont un degré d'appartenance maximal, qui diminue linéairement
lorsque la valeur de la métrique dépasse le seuil \(v\), jusqu’à
devenir nul à \(v+\delta\). Ce \emph{fuzzyfieur} est défini à l'aide
de la fonction d'appartenance suivante :

\begin{equation}
  \label{eq:inf_val}
  \def\arraystretch{1.25}
   f(x,v,\delta) = \left\{
    \begin{array}{cl}
      1 & \text{si}\ x < v  \\
      \dfrac{x_2-x}{\Delta x} & \text{si}\ v ≤ x ≤ v + \delta \\
      0 & \text{si}\ x < v + \delta\\
    \end{array}
  \right.
\end{equation}

Où \(x\) est la valeur de la \emph{métrique,} \(v\) la valeur référence et \(\delta\) l'écart xxx.

De la même manière que les autres \textrm{fuzzyfieurs,}
\onto{Inf\-Val} dispose d'une variante, \onto{Inf\-Val\-0}, centrée
sur 0 est dont la fonction d'appartenance est donc : \(f(x,\delta) = f(x,0,\delta)\).

\begin{figure}
  \centering
  \subfloat[\emph{Fonction d'appartenance} pour \texttt{Inf\-Val}]{
    \begin{tikzpicture}[scale=.7]
  \def\decalageX{-.2}
  \def\decalageY{-.2}
  % Courbe
  \begin{scope}[transparency group]
    % fond
    \begin{scope}
      \path[ffa_fade_m]  (0,2) -- (1, 2)  -- (1,0) -- (0,0) -- cycle;
      \path[ffa]  (1,2) -- (3, 2)  -- (6,0) -- (1,0) -- cycle;
    \end{scope}
    % bords
    \begin{scope}
      \path[ffc_fade_m] (0,2) -- (2,2) ;
      \path[ffc] (2,2) -- (3,2) -- (6,0) -++ (3,0) ;
      \path[ffc_fade] (8,0) -- (9,0) ;
    \end{scope}
  \end{scope}
  % Axes X, Y
  \begin{scope}
    % Axe X
    \begin{scope}
      % Axe
      \draw[<->] (0, \decalageX) --++ (9, 0) coordinate (x axis);
      % Graduations
      \foreach \n/\t in {1/{},2/{},3/{},4/{},5/{},6/{},7/{},8/{}}
      {
        \draw[-] (\n, \decalageX - .05) --++ (0, .1);
        \node[below, font=\footnotesize] at (\n, \decalageX - .05) {\t};
      }
      % label
      \node[below left, yshift=-.1cm, font=\small] at (x axis)
      {\itshape Métrique};
    \end{scope}
    % Axe Y
    \begin{scope}
      % Axe
      \draw[-] (\decalageY ,0) --++ (0, 2) coordinate (y axis);
      % Graduations
      \foreach \n/\t in {0/{0},2/{1}}
      {
        \draw[-] (\decalageY -.05, \n) --++ (.1, 0);
        \node[left, font=\footnotesize] at (\decalageY -.05, \n) {\t};
      }
      % Label
      \node[above] at (y axis) {$\mu$};
    \end{scope}
  \end{scope}
  \begin{scope}
    \draw (3,\decalageY) -- (3,2);
    \draw[fill] (3,\decalageY) circle (1pt);
    \draw[fill] (3,2) circle (1pt);
    \draw[fill] (6,0) circle (1pt);
    \node[above] at (3,2) {P};
    % 
    \draw[|-|] (3,-.7cm) --++(3,0) node[pos=.5, fill=white, inner
    sep=1pt, font=\small] {$\delta$};
  \end{scope}
\end{tikzpicture}

  }\hfill
  \subfloat[\emph{Fonction d'appartenance} pour \texttt{Inf\-Val\-0}]{
    \begin{tikzpicture}[scale=.7]
  \def\decalageX{-.2}
  \def\decalageY{-.2}
  % Courbe
  \begin{scope}[transparency group]
    % fond
    \begin{scope}
      \path[ffa_fade_m]  (0,2) -- (1, 2)  -- (1,0) -- (0,0) -- cycle;
      \path[ffa]  (1,2) -- (3, 2)  -- (6,0) -- (1,0) -- cycle;
    \end{scope}
    % bords
    \begin{scope}
      \path[ffc_fade_m] (0,2) -- (2,2) ;
      \path[ffc] (2,2) -- (3,2) -- (6,0) -++ (3,0) ;
      \path[ffc_fade] (8,0) -- (9,0) ;
    \end{scope}
  \end{scope}
  % Axes X, Y
  \begin{scope}
    % Axe X
    \begin{scope}
      % Axe
      \draw[<->] (0, \decalageX) --++ (9, 0) coordinate (x axis);
      % Graduations
      \foreach \n/\t in {1/{},2/{},3/{},4/{},5/{},6/{},7/{},8/{}}
      {
        \draw[-] (\n, \decalageX - .05) --++ (0, .1);
        \node[below, font=\footnotesize] at (\n, \decalageX - .05) {\t};
      }
      % label
      \node[below left, yshift=-.1cm, font=\small] at (x axis)
      {\itshape Métrique};
    \end{scope}
    % Axe Y
    \begin{scope}
      % Axe
      \draw[-] (\decalageY ,0) --++ (0, 2) coordinate (y axis);
      % Graduations
      \foreach \n/\t in {0/{0},2/{1}}
      {
        \draw[-] (\decalageY -.05, \n) --++ (.1, 0);
        \node[left, font=\footnotesize] at (\decalageY -.05, \n) {\t};
      }
      % Label
      \node[above] at (y axis) {$\mu$};
    \end{scope}
  \end{scope}
  \begin{scope}
    \draw (3,\decalageY) -- (3,2);
    \draw[fill] (3,\decalageY) circle (1pt);
    \draw[fill] (3,2) circle (1pt);
    \draw[fill] (6,0) circle (1pt);
    \node[above] at (3,2) {P};
    % 
    \draw[|-|] (3,-.7cm) --++(3,0) node[pos=.5, fill=white, inner
    sep=1pt, font=\small] {$\delta$};
  \end{scope}
\end{tikzpicture}

  }
  \caption{\emph{Fonctions d'appartenance} du \emph{fuzzyficateur}
    \protect\onto{Inf\-Val} et de son dérivé
    \protect\onto{Inf\-Val\-0}.}
  \label{fig:select_infval}
\end{figure}


% \begin{wrapfigure}[10]{o}{7.5cm}
%   \centering
%   \begin{tikzpicture}[scale=.7]
  \def\decalageX{-.2}
  \def\decalageY{-.2}
  % Courbe
  \begin{scope}[transparency group]
    % fond
    \begin{scope}
      \path[ffa_fade_m]  (0,2) -- (1, 2)  -- (1,0) -- (0,0) -- cycle;
      \path[ffa]  (1,2) -- (3, 2)  -- (6,0) -- (1,0) -- cycle;
    \end{scope}
    % bords
    \begin{scope}
      \path[ffc_fade_m] (0,2) -- (2,2) ;
      \path[ffc] (2,2) -- (3,2) -- (6,0) -++ (3,0) ;
      \path[ffc_fade] (8,0) -- (9,0) ;
    \end{scope}
  \end{scope}
  % Axes X, Y
  \begin{scope}
    % Axe X
    \begin{scope}
      % Axe
      \draw[<->] (0, \decalageX) --++ (9, 0) coordinate (x axis);
      % Graduations
      \foreach \n/\t in {1/{},2/{},3/{},4/{},5/{},6/{},7/{},8/{}}
      {
        \draw[-] (\n, \decalageX - .05) --++ (0, .1);
        \node[below, font=\footnotesize] at (\n, \decalageX - .05) {\t};
      }
      % label
      \node[below left, yshift=-.1cm, font=\small] at (x axis)
      {\itshape Métrique};
    \end{scope}
    % Axe Y
    \begin{scope}
      % Axe
      \draw[-] (\decalageY ,0) --++ (0, 2) coordinate (y axis);
      % Graduations
      \foreach \n/\t in {0/{0},2/{1}}
      {
        \draw[-] (\decalageY -.05, \n) --++ (.1, 0);
        \node[left, font=\footnotesize] at (\decalageY -.05, \n) {\t};
      }
      % Label
      \node[above] at (y axis) {$\mu$};
    \end{scope}
  \end{scope}
  \begin{scope}
    \draw (3,\decalageY) -- (3,2);
    \draw[fill] (3,\decalageY) circle (1pt);
    \draw[fill] (3,2) circle (1pt);
    \draw[fill] (6,0) circle (1pt);
    \node[above] at (3,2) {P};
    % 
    \draw[|-|] (3,-.7cm) --++(3,0) node[pos=.5, fill=white, inner
    sep=1pt, font=\small] {$\delta$};
  \end{scope}
\end{tikzpicture}

%   \caption{Fonction d'appartenance utilisée pour définir le
%     \emph{fuzzyfieur} \protect\onto{Inf\-Val}}.
% \end{wrapfigure}

% \begin{wrapfigure}[10]{o}{7.5cm}
%   \centering
%   \begin{tikzpicture}[scale=.7]
  \def\decalageX{-.2}
  \def\decalageY{-.2}
  % Courbe
  \begin{scope}[transparency group]
    % fond
    \begin{scope}
      \path[ffa_fade_m]  (0,2) -- (1, 2)  -- (1,0) -- (0,0) -- cycle;
      \path[ffa]  (1,2) -- (3, 2)  -- (6,0) -- (1,0) -- cycle;
    \end{scope}
    % bords
    \begin{scope}
      \path[ffc_fade_m] (0,2) -- (2,2) ;
      \path[ffc] (2,2) -- (3,2) -- (6,0) -++ (3,0) ;
      \path[ffc_fade] (8,0) -- (9,0) ;
    \end{scope}
  \end{scope}
  % Axes X, Y
  \begin{scope}
    % Axe X
    \begin{scope}
      % Axe
      \draw[<->] (0, \decalageX) --++ (9, 0) coordinate (x axis);
      % Graduations
      \foreach \n/\t in {1/{},2/{},3/{},4/{},5/{},6/{},7/{},8/{}}
      {
        \draw[-] (\n, \decalageX - .05) --++ (0, .1);
        \node[below, font=\footnotesize] at (\n, \decalageX - .05) {\t};
      }
      % label
      \node[below left, yshift=-.1cm, font=\small] at (x axis)
      {\itshape Métrique};
    \end{scope}
    % Axe Y
    \begin{scope}
      % Axe
      \draw[-] (\decalageY ,0) --++ (0, 2) coordinate (y axis);
      % Graduations
      \foreach \n/\t in {0/{0},2/{1}}
      {
        \draw[-] (\decalageY -.05, \n) --++ (.1, 0);
        \node[left, font=\footnotesize] at (\decalageY -.05, \n) {\t};
      }
      % Label
      \node[above] at (y axis) {$\mu$};
    \end{scope}
  \end{scope}
  \begin{scope}
    \draw (3,\decalageY) -- (3,2);
    \draw[fill] (3,\decalageY) circle (1pt);
    \draw[fill] (3,2) circle (1pt);
    \draw[fill] (6,0) circle (1pt);
    \node[above] at (3,2) {P};
    % 
    \draw[|-|] (3,-.7cm) --++(3,0) node[pos=.5, fill=white, inner
    sep=1pt, font=\small] {$\delta$};
  \end{scope}
\end{tikzpicture}

%   \caption{Fonction d'appartenance utilisée pour définir le
%     \emph{fuzzyfieur} \protect\onto{Inf\-Val\-0}.}
% \end{wrapfigure}


\subsection{Les \emph{modifieurs}}

La seule utilisation des \emph{fuzzyfieurs} définis jusqu'à présent ne
permet cependant pas de traiter toutes les situations possibles. En
effet, il n'est pas vraiment possible de formaliser des situations où
l'on modélise 

C'est pour traiter ce genre de situations que nous avons développé la
notion de \emph{modifieurs.} Les \emph{modifieurs} sont ---~de la même
que les \emph{rasterisers,} les \emph{métriques} et les
\emph{fuzzyficateurs}~--- des concepts formalisant la sémantique d'une
\emph{relation spatiale atomique} et correspondant à un algorithme
appliqué lors de la \emph{phase de spatialisation.} Cependant, ils
possèdent la particularité d'être facultatifs, la plupart des
\emph{relations de localisation atomiques} en sont dépourvues, et
cumulatifs, une même \emph{relation} pouvant en utiliser
plusieurs. Les \emph{modifieurs} sont appliqués à la fin de la
\emph{spatialisation} (\autoref{fig:methodo_spatialisation}), après la
\emph{fuzzyfication.}  Dans cette configuration la \emph{zone de
  localisation compatible} n'est donc pas le résultat de la
\emph{fuzzyfication} \footnote{Comme l'indique la
  \autoref{fig:methodo_spatialisation}, qui présente la \emph{méthode
    de spatialisation} sans modifieurs.}, mais de l’application des
\emph{modifieurs} au résultat de la \emph{fuzzyfication.}

\tdi{Modifeur not}
Il est, par exemple, impossible de traiter une situation où la
sémantique de la \emph{relation de localisation atomique} impose de
construire une \ac{zlc} à partir de la 

Le \emph{modifieur} \onto{Not} rempli justement ce rôle.

Conformément à l'opérateur de négation défini dans le cadre de
\emph{logique floue} (\autoref{eq:comp}, \autoref{chap:03}).

Le \emph{modifieur} \onto{Not} retourne donc le complémentaire de
la \emph{zone de localisation compatible} qui lui est transmisse.

S'il ne modifie pas directement les \emph{fonctions d'appartenance,}
son application aux \emph{fuzzyfieurs} précédemment définis revient à
définir trois nouvelles \emph{fonctions d'appartenance}
(\autoref{fig:fnc_not}), complémentaires de celles déjà définies

\begin{figure}
  \centering
  \subfloat[\emph{Fonction d'appartenance} pour \texttt{Eq\-Val} avec
  le \emph{modifieur} \texttt{Not}]{
    \begin{tikzpicture}[scale=.7]
  \def\decalageX{-.2}
  \def\decalageY{-.2}
  % Courbe
  \begin{scope}[transparency group]
    % fond
    \begin{scope}
      \path[ffa2]  (1,0)--(1,2) --(3,2) -- (4.5, 0) -- (6,2) --(8,2) --(8,0)-- cycle;
      \path[ffa2_fade_m]  (0,2) -- (1, 2) -- (1,0) --(0,0) -- cycle;
      \path[ffa2_fade]  (8,2) -- (9, 2) -- (9,0) --(8,0) -- cycle;
    \end{scope}
    % bords
    \begin{scope}
      \path[ffc2] (1,2) -- (3,2) -- (4.5, 0) -- (6,2) -- (8,2) ;
      \path[ffc2_fade_m] (0,2) -- (1,2) ;
      \path[ffc2_fade] (8,2) -- (9,2) ;
    \end{scope}
  \end{scope}
  % Axes X, Y
  \begin{scope}
    % Axe X
    \begin{scope}
      % Axe
      \draw[<->] (0, \decalageX) --++ (9, 0) coordinate (x axis);
      % Graduations
      \foreach \n/\t in {1/{},2/{},3/{},4/{},5/{},6/{},7/{},8/{}}
      {
        \draw[-] (\n, \decalageX - .05) --++ (0, .1);
        \node[below, font=\footnotesize] at (\n, \decalageX - .05) {\t};
      }
      % label
      \node[below left, yshift=-.1cm, font=\small] at (x axis) {\itshape Métrique};
    \end{scope}
    % Axe Y
    \begin{scope}
      % Axe
      \draw[-] (\decalageY ,0) --++ (0, 2) coordinate (y axis);
      % Graduations
      \foreach \n/\t in {0/{0},2/{1}}
      {
        \draw[-] (\decalageY -.05, \n) --++ (.1, 0);
        \node[left, font=\footnotesize] at (\decalageY -.05, \n) {\t};
      }
      % Label
      \node[above] at (y axis) {$\mu$};
    \end{scope}
  \end{scope}
  \begin{scope}
    % Seuil 1
    \draw[ffc2,line width=.5] (3,\decalageY) -- (3,2);
    \draw[fill, RdBu-9-9] (3,\decalageY) circle (2pt);
    \draw[fill, RdBu-9-9] (3,2) circle (2pt);
    % Seuil 2
    \draw[ffc2,line width=.5] (4.5,\decalageY) -- (4.5,0);
    \draw[fill, RdBu-9-9] (4.5,\decalageY) circle (2pt);
    \draw[fill, RdBu-9-9] (4.5,0) circle (2pt);
    \node[above] at (4.5,0) {\(v\)};
    % Seuil 3
    \draw[ffc2,line width=.5] (6,\decalageY) -- (6,2);
    \draw[fill, RdBu-9-9] (6,\decalageY) circle (2pt);
    \draw[fill, RdBu-9-9] (6,2) circle (2pt);
        \draw[|-|] (3,-.7cm) --++(3,0) node[pos=.5, fill=white, inner
    sep=1pt, font=\small] {$\delta$};
  \end{scope}
\end{tikzpicture}

    \label{fig:fnc_not_eq_val}
  }
  
  \subfloat[\emph{Fonction d'appartenance} pour \texttt{Sup\-Val} avec
  le \emph{modifieur} \texttt{Not}]{
    \begin{tikzpicture}[scale=.7]
  \def\decalageX{-.2}
  \def\decalageY{-.2}
  % Courbe
  \begin{scope}[transparency group]
    % fond
    \begin{scope}
      \path[ffa2_fade_m]  (0,2) -- (1, 2)  -- (1,0) -- (0,0) -- cycle;
      \path[ffa2]  (1,2) -- (3, 2)  -- (6,0) -- (1,0) -- cycle;
    \end{scope}
    % bords
    \begin{scope}
      \path[ffc2_fade_m] (0,2) -- (2,2) ;
      \path[ffc2] (2,2) -- (3,2) -- (6,0) -++ (3,0) ;
      \path[ffc2_fade] (8,0) -- (9,0) ;
    \end{scope}
  \end{scope}
  % Axes X, Y
  \begin{scope}
    % Axe X
    \begin{scope}
      % Axe
      \draw[<->] (0, \decalageX) --++ (9, 0) coordinate (x axis);
      % Graduations
      \foreach \n/\t in {1/{},2/{},3/{},4/{},5/{},6/{},7/{},8/{}}
      {
        \draw[-] (\n, \decalageX - .05) --++ (0, .1);
        \node[below, font=\footnotesize] at (\n, \decalageX - .05) {\t};
      }
      % label
      \node[below left, yshift=-.1cm, font=\small] at (x axis) {\itshape Métrique};
    \end{scope}
    % Axe Y
    \begin{scope}
      % Axe
      \draw[-] (\decalageY ,0) --++ (0, 2) coordinate (y axis);
      % Graduations
      \foreach \n/\t in {0/{0},2/{1}}
      {
        \draw[-] (\decalageY -.05, \n) --++ (.1, 0);
        \node[left, font=\footnotesize] at (\decalageY -.05, \n) {\t};
      }
      % Label
      \node[above] at (y axis) {$\mu$};
    \end{scope}
  \end{scope}
  \begin{scope}
    % Seuil 1
    \draw[ffc2,line width=.5] (3,\decalageY) -- (3,2);
    \draw[fill, RdBu-9-9] (3,\decalageY) circle (2pt);
    \draw[fill, RdBu-9-9] (3,2) circle (2pt);
    % Seuil 2
    \draw[ffc2,line width=.5] (6,\decalageY) -- (6,0);
    \draw[fill, RdBu-9-9] (6,\decalageY) circle (2pt);
    \draw[fill, RdBu-9-9] (6,0) circle (2pt);
    \node[above] at (6,0) {\(v\)};

    \draw[|-|] (3,-.7cm) --++(3,0) node[pos=.5, fill=white, inner
    sep=1pt, font=\small] {$\delta$};
  \end{scope}
\end{tikzpicture}

    \label{fig:fnc_not_sup_va}
  }\hfill  
  \subfloat[\emph{Fonction d'appartenance} pour \texttt{Inf\-Val} avec
  le \emph{modifieur} \texttt{Not}]{
    \begin{tikzpicture}[scale=.7]
  \def\decalageX{-.2}
  \def\decalageY{-.2}
  % Courbe
  \begin{scope}[transparency group]
    % fond
    \begin{scope}
      \path[ffa2]  (3,0) -- (6, 2)  -- (8,2) -- (8,0) -- cycle;
      \path[ffa2_fade]  (8,2) -- (9, 2)  -- (9,0) -- (8,0) -- cycle;
    \end{scope}
    % bords
    \begin{scope}
      \path[ffc2_fade_m] (0,0) -- (2,0) ;
      \path[ffc2] (2,0) -- (3,0) -- (6, 2) -++ (3,0) ;
      \path[ffc2_fade] (8,2) -- (9,2) ;
    \end{scope}
  \end{scope}
  % Axes X, Y
  \begin{scope}
    % Axe X
    \begin{scope}
      % Axe
      \draw[<->] (0, \decalageX) --++ (9, 0) coordinate (x axis);
      % Graduations
      \foreach \n/\t in {1/{},2/{},3/{},4/{},5/{},6/{},7/{},8/{}}
      {
        \draw[-] (\n, \decalageX - .05) --++ (0, .1);
        \node[below, font=\footnotesize] at (\n, \decalageX - .05) {\t};
      }
      % label
      \node[below left, yshift=-.1cm, font=\small] at (x axis)
      {\itshape Métrique};
    \end{scope}
    % Axe Y
    \begin{scope}
      % Axe
      \draw[-] (\decalageY ,0) --++ (0, 2) coordinate (y axis);
      % Graduations
      \foreach \n/\t in {0/{0},2/{1}}
      {
        \draw[-] (\decalageY -.05, \n) --++ (.1, 0);
        \node[left, font=\footnotesize] at (\decalageY -.05, \n) {\t};
      }
      % Label
      \node[above] at (y axis) {$\mu$};
    \end{scope}
  \end{scope}
    \begin{scope}
    % Seuil 1
    \draw[ffc2,line width=.5] (3,\decalageY) -- (3,0);
    \draw[fill, RdBu-9-9] (3,\decalageY) circle (2pt);
    \draw[fill, RdBu-9-9] (3,0) circle (2pt);
    % Seuil 2
    \draw[ffc2,line width=.5] (6,\decalageY) -- (6,2);
    \draw[fill, RdBu-9-9] (6,\decalageY) circle (2pt);
    \draw[fill, RdBu-9-9] (6,2) circle (2pt);
    \node[above] at (3,0) {\(v\)};

    \draw[|-|] (3,-.7cm) --++(3,0) node[pos=.5, fill=white, inner
    sep=1pt, font=\small] {$\delta$};
  \end{scope}
\end{tikzpicture}

    \label{fig:fnc_not_inf_va}
  }  
  \caption{Fonctions d'appartenance des \emph{fuzzyfieurs}
    \protect\onto{Eq\-Val}, \protect\onto{Sup\-Val}, et
    \protect\onto{Inf\-Val} après application du \emph{modifieur}
    \protect\onto{Not}.}
  \label{fig:fnc_not}
\end{figure}

\tdi{resampling}
Le second \emph{modifieur} que nous ayons défini est XXXX

Sa fonction est de 

\begin{figure}
  \centering
  \includegraphics[width=\textwidth]{../../../../tmp/modifieur.jpeg}
  \caption{Modifieur}
  \label{fig:methode_modifieur}
\end{figure}

\tdi{Parler de l'ordre}

%%% Local Variables:
%%% mode: latex
%%% TeX-master: "../../../../main"
%%% End:
