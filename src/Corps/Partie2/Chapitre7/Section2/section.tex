La \emph{rasterisation} des \emph{objets de référence}

La première partie de la méthode de spatialisation est la
\emph{rasterisation des objets de référence.}

\tdi{méthode de rasterisation = sélection partie objet ref +
  rasterisation}

\tdi{Méthode de rasterisation fixe, seule la sélection varie}

\subsection{Rasterisation et sémantique ??}

\tdi{Expliquer comment on les sélectionne et leur importance sur la
  sémantique}

\tdi{parler des consiérations autour de la rasterisation importance de
  la zir, de la résolution, méthode de sélection des pixels, etc}

La \autoref{fig:rasterisation_sel_pixels} illustre ces deux approches. Un même \emph{objet de référence} est \emph{rastérisé,} d'une part en sélectionnant uniquement les pixels dont le centroïde est à l'intérieur de \emph{l'objet de référence} (aboutissant au raster \textcolor{RdBu-9-1}{\textsf{A}}), d'autre part en sélectionnant tous les pixels touchant \emph{l'objet de référence} (ce qui donne le raster \textcolor{RdBu-9-9}{\textsf{B}}). Comme on peut le voir la seconde solution abouti à une zone plus large, la zone résultant de la \emph{rasterisation} est alors plus importante que l'emprise réelle de l'objet. Cependant cette seconde approche à l'avantage de toujours aboutir à un raster non nul. En effet dans certaines configurations, il est possible qu'un \emph{objet de référence,} de surface non nulle, ne contienne aucun centroïde de pixel. Ce qui conduit à produire un raster vide. Ce phénomène se produit lorsque la maille utilisée pour la \emph{rasterisation} est trop large, comparativement aux objets que l'on souhaite \emph{rasteriser.} Ce problème n'est donc lié qu'à une mauvaise paramétrisation de la rasterisation et il peut être corrigé par une augmentation de la résolution du raster cible.

Concrètement cela peut se produire lors de la rastérisation de bâtiments, comme l'illustre la \autoref{fig:rasterisation_cas_limite}. En effet si l'on souhaite travailler sur une zone assez étendue (de l'ordre du XX) il sera nécessaire d'adopter une maille de taille intermédiaire (XX), ce qui est supérieur à la taille de nombreux batiments. Ainsi, une rasterisation avec de tels paramètres conduira à oublier de nombreux objets.

\begin{figure}
  \centering
  \begin{tikzpicture}
  % Arrow
  \begin{scope}
    \path[draw, ->, shorten >=5pt, shorten <=5pt] (1,2) |- (8,3)
    node[pos=.75, font=\footnotesize, fill=white] (labRastA)
    {\itshape rasterisation};
    %
    \node[anchor=north west,baseline, text width=3.5cm, font=\tiny] at
    ([xshift=1.5ex, yshift=1ex]labRastA.south west) {avec sélection
      des pixels dont le centroïde est à l'intérieur de \emph{l'objet
        de référence}};
    
    \path[draw, ->, shorten >=5pt, shorten <=5pt] (1,0) |- (8,-1)
    node[pos=.75, font=\footnotesize, fill=white] (labRastB)
    {\itshape rasterisation};
    %
    \node[anchor=north west,baseline, text width=3.5cm, font=\tiny] at
    ([xshift=1.5ex, yshift=1ex]labRastB.south west) {avec sélection de
    tous les pixels intersectant \emph{l'objet de référence}};
  \end{scope}
  % Représentation objet de référence
  \begin{scope}[local bounding box=obr]
    \fill[ffa, pattern color=black] (1,1) circle [radius=15pt];
    \path[ffc, black] (1,1) circle [radius=15pt];
    \path[ffc, black] (0,0) rectangle (2,2);
    \node[text width=3cm, align=center, anchor=north,
    font=\footnotesize, fill=white] at (1,-.25)
    {\itshape Objet de référence};
  \end{scope}
  
  % Représentation objet de référence rasterisé
  \begin{scope}[xshift=8cm, yshift=2cm]
    \begin{scope}
      \foreach \x in {0,.25,...,1.75}{
        \foreach \y in {0,.25,...,1.75}
        \path[draw, line width=.01mm] (\x,\y) rectangle (\x +.25, \y + .25);
      }
      \path[ffa] (.5,.5) rectangle (1.5, 1.5);
      \path[ffc, black, dashed] (1,1) circle [radius=15pt];
      \path[ffc] (0,0) rectangle (2,2);
    \end{scope}
    \node[text width=3cm, align=center, anchor=north, font=\footnotesize] at (1,-.25)
    {\itshape Objet de référence rastérisé \normalfont \textcolor{RdBu-9-1}{\textsf{A}}};
  \end{scope}
  
  \begin{scope}[xshift=8cm, yshift=-2cm]
    \begin{scope}
      \foreach \x in {0,.25,...,1.75}{
        \foreach \y in {0,.25,...,1.75}
        \path[draw, line width=.01mm] (\x,\y) rectangle (\x +.25, \y + .25);
      }
      % Pixels rasterisation
      %\path[ffc] (1,1) circle [radius=15pt];
      \path[ffa2] (.5,.5) rectangle (1.5, 1.5);
      \path[ffa2] (.25,.75) rectangle (.5, 1.25);
      \path[ffa2] (1.5,.75) rectangle (1.75, 1.25);
      \path[ffa2] (.75,1.5) rectangle (1.25, 1.75);
      \path[ffa2] (.75,.25) rectangle (1.25,.5);
      \path[ffc2, black, dashed] (1,1) circle [radius=15pt];
      \path[ffc2] (0,0) rectangle (2,2);
    \end{scope}
    \node[text width=3cm, align=center, anchor=north, font=\footnotesize] at (1,-.25)
    {\itshape Objet de référence rastérisé \normalfont \textcolor{RdBu-9-9}{\textsf{B}}};
  \end{scope}
\end{tikzpicture}
  \caption{d}
  \label{fig:rasterisation_sel_pixels}
\end{figure}
    
Pour notre application nous avons pris la décision de toujours travailler à l'aide d'une rasterisation sélectionnant tous les pixels touchant \emph{l'objet de référence} (\eg la \emph{rasterisation} \textcolor{RdBu-9-9}{\textsf{B}}, \autoref{fig:rasterisation_sel_pixels}). Cette solution à l’avantage de toujours renvoyer une rasterization non vide 

\begin{figure}
  \centering
  \begin{tikzpicture}
  \begin{scope}
    \foreach \x in {0,.5,...,1.5}{
      \foreach \y in {0,.5,...,1.5}{
        \pgfmathsetmacro\xc{\x +.25}
        \pgfmathsetmacro\yc{\y +.25}
        % 
        \path[draw, line width=.01mm] (\x,\y) rectangle (\x +.5, \y +
        .5);
        \fill[black] (\xc,\yc) circle [radius=0.5 pt];
      }
    }
    \path[ffc, black] (0,0) rectangle (2,2);
    \fill[ffa, pattern color=black,rotate around={30:(1,1)}] (.8,.9) --++ (.4,0) --++ (0,.2) --++ (-.4,0) --cycle;
    \path[ffc, black,rotate around={30:(1,1)}] (.8,.9) --++ (.4,0) --++ (0,.2) --++ (-.4,0) --cycle;
  \end{scope}
  \path[<->, draw] (0,2.2) -- (.5,2.2) node[pos=.5, above, font=\tiny] {\SI{25}{\meter}};  
\end{tikzpicture}
  \caption{d}
  \label{fig:rasterisation_cas_limite}
\end{figure}

\subsection{Les méthodes de rasterisation}

\tdi{Détailler les différentes méthodes de rasterisation définies
  figure \ref{fig:methode_rasterisation} + exemple}


La plus évidente des solutions consiste à rasteriser l'ensemble de
\emph{l'objet de référence.}


Mais il également possible de ne rasteriser qu'une partie de
\emph{l'objet de référence,} comme sa frontière ou son centroide
(\autoref{tab:methode_rasterisation}).

\begin{table}
  \centering
  \begin{tabular}{>{\bfseries}R{2cm}C{3cm}C{3cm}C{3cm}}
  \toprule
  & \multicolumn{1}{c}{\bfseries Ponctuel} &
   \multicolumn{1}{c}{\bfseries Linéaire} &
   \multicolumn{1}{c}{\bfseries Région} \\
  \addlinespace
  & \tikz{\fill[black] (.65,.65) circle [radius=1pt];\path[ffc, black] (0,0) rectangle (2,2);}&
  \tikz{\path[ffc,black] (.3,.3) ..controls(.25,1.75) and (1.375,.25) .. (1.7,1.7); \path[ffc,black] (0,0) rectangle (2,2);} &
  \tikz{\fill[ffa, pattern color=black] (1.125,.65) circle [radius=10pt];\path[ffc,black](1.125,.65) circle [radius=10pt];\path[ffc,black](0,0) rectangle (2,2);}\\
  %
  \midrule
  \addlinespace
  Objet
  & \tikz{
    \begin{scope}
      \foreach \x in {0,.25,...,1.75}{
        \foreach \y in {0,.25,...,1.75}
        \path[draw, line width=.01mm] (\x,\y) rectangle (\x +.25, \y + .25);
      }\path[ffa] (.5,.5) rectangle (.75,.75);
      \path[ffc] (0,0) rectangle (2,2);
    \end{scope}
    }
                                           & \tikz{
                                             \begin{scope}
                                               \foreach \x in {0,.25,...,1.75}{
                                                 \foreach \y in {0,.25,...,1.75}
                                                 \path[draw, line width=.01mm] (\x,\y) rectangle (\x +.25, \y + .25);
                                               }
                                               \foreach \i/\j in {.25/.25,.25/.5,.25/.75,.5/.75,.5/1,.75/.75,.75/1,1/.75,1/1,1.25/1,1.5/1,1.5/1.25,1.5/1.5}{
                                                 \path[ffa] (\i,\j) rectangle (\i + .25, \j +.25);
                                               }
                                               \path[ffc] (0,0) rectangle (2,2);
                                             \end{scope}
                                             } &\tikz{
                                                 \begin{scope}
                                                   \foreach \x in {0,.25,...,1.75}{
                                                     \foreach \y in {0,.25,...,1.75}
                                                     \path[draw, line width=.01mm] (\x,\y) rectangle (\x +.25, \y + .25);
                                                   }
                                                   \path[ffa] (.75,.25)
                                                   rectangle (1.5,
                                                   1);\path[ffc] (0,0) rectangle (2,2);
                                                 \end{scope}
                                                 }\\
  \addlinespace
  Frontière&\tikz{
             \begin{scope}
               \foreach \x in {0,.25,...,1.75}{
                 \foreach \y in {0,.25,...,1.75}
                 \path[draw, line width=.01mm] (\x,\y) rectangle (\x +.25, \y + .25);
               }
               \path[ffa] (.5,.5) rectangle (.75,.75);
               \path[ffc] (0,0) rectangle (2,2);
             \end{scope}
             }
                                           & \tikz{
                                             \begin{scope}
                                               \foreach \x in {0,.25,...,1.75}{
                                                 \foreach \y in {0,.25,...,1.75}
                                                 \path[draw, line width=.01mm] (\x,\y) rectangle (\x +.25, \y + .25);
                                               }
                                               \path[ffa] (.25,.25) rectangle (.5, .5);
                                               \path[ffa] (1.5,1.5) rectangle (1.75, 1.75);
                                               \path[ffc] (0,0) rectangle (2,2);
                                             \end{scope}
                                             } &\tikz{
                                                 \begin{scope}
                                                   \foreach \x in {0,.25,...,1.75}{
                                                     \foreach \y in {0,.25,...,1.75}
                                                     \path[draw, line width=.01mm] (\x,\y) rectangle (\x +.25, \y + .25);
                                                   }
                                                   % Pixels rasterisation
                                                   \path[ffa] (.75,.25)
                                                   rectangle (1.5,
                                                   .5);  
                                                   \path[ffa] (.75,.5)
                                                   rectangle (1,.75);
                                                   \path[ffa] (1.25,.5)
                                                   rectangle (1.5,.75);
                                                   \path[ffa] (.75,.75)
                                                     rectangle (1.5,1);
                                                   \path[ffc] (0,0) rectangle (2,2);
                                                 \end{scope}
                                                 }\\
%
  \addlinespace
%
  Centroide&\tikz{
             \begin{scope}
               \foreach \x in {0,.25,...,1.75}{
                 \foreach \y in {0,.25,...,1.75}
                 \path[draw, line width=.01mm] (\x,\y) rectangle (\x +.25, \y + .25);
               }
               \path[ffa] (.5,.5) rectangle (.75,.75);
               \path[ffc] (0,0) rectangle (2,2);
             \end{scope}
             }
                                           & \tikz{
                                             \begin{scope}
                                               \foreach \x in {0,.25,...,1.75}{
                                                 \foreach \y in {0,.25,...,1.75}
                                                 \path[draw, line width=.01mm] (\x,\y) rectangle (\x +.25, \y + .25);
                                               }
                                               \path[ffa] (1,1) rectangle (1.25, 1.25);
                                               \path[ffc] (0,0) rectangle (2,2);
                                             \end{scope}
                                             } &\tikz{
                                                 \begin{scope}
                                                   \foreach \x in {0,.25,...,1.75}{
                                                     \foreach \y in {0,.25,...,1.75}
                                                     \path[draw, line width=.01mm] (\x,\y) rectangle (\x +.25, \y + .25);
                                                   }
                                                   \path[ffa] (1,.5)
                                                   rectangle (1.25, .75);
                                                   \path[ffc] (0,0) rectangle (2,2);
                                                 \end{scope}
                                                 }\\
  \bottomrule    
\end{tabular}
  \caption{Méthodes de rasterisation}
  \label{tab:methode_rasterisation}
\end{table}

\tdi{Parler rapidement des méthodes définies mais non utilisées (bbox,
  conxhull)}

D'autres méthodes plus spécifiques sont également envisageables.

%%% Local Variables:
%%% mode: latex
%%% TeX-master: "../../../../main"
%%% End:
