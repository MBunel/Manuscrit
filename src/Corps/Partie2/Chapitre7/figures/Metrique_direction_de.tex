\begin{tikzpicture}
  \tikzset{et/.style={above,font=\footnotesize\vphantom{Ag}}}
  %
  \node[inner sep=0pt, anchor=south west] (image) at (0,0){\includegraphics{./figures/Metrique_part_lac_visible.png}};
  %
  \begin{scope}
    \node (P2) at ([yshift=-.5cm]image.south east) {};
    \node (P1) at ([yshift=-.5cm]image.south west) {};
    %
    \foreach \x [evaluate=\xshift using \x/10, evaluate=\rad using (\x * .0004) + .01] in {0,...,100}
    {
      \draw[fill=black,draw=none, below] ([xshift=\xshift cm, yshift=-.5cm]P1) circle [radius=\rad cm];
    }
    %
    \path(P1 |- 0cm,-1cm) --++ (10,0)
    node[et,pos=0] {0}
    node[et,pos=.1] {0,1}
    node[et,pos=.2] {0,2}
    node[et,pos=.3] {0,3}
    node[et,pos=.4] {0,4}
    node[et,pos=.65] {0,65}
    node[et,pos=1] {1};
    % Échelle
    \draw[-] (P2 |- -1cm,-1cm) --++ (-1,0) node[et,pos=.5] {\SI{500}{\meter}};
    % Légende détaillée
    \path (P1) -- (P2) node[pos=.5, yshift=-1cm] {\tiny Pour la légende détaillée du fond topographique voir \autoref{anx:topo_leg}. Sources: BD TOPO 2018, BD ALTI 2018.}; 
  \end{scope}
\end{tikzpicture}