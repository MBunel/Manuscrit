Dans cette troisième et dernière partie nous allons présenter
l’application de la méthode présentée et définie dans la
\autoref{part:02} à des cas réels, auxquels les secouristes du
\ac{pghm} de Grenoble ont été confrontés. Nous présenterons chaque
alerte et détaillerons la manière dont les indications données par les
requérants ont été formalisés en des relations de localisation
atomiques. Cette partie ne se compose que d'un seul chapitre, le \ref{chap:09},
au sein duquel nous présenterons la spatialisation de deux alertes.

%%% Local Variables:
%%% mode: latex
%%% TeX-master: "../../main"
%%% End:
