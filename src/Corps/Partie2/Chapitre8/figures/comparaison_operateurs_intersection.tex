\begin{tikzpicture}
  \def\decalageX{-.2}
  \def\decalageY{-.2}

  \newcommand{\calcangle}[4]{%
    % #1 nom de la varaible renvoyée
    % #2 coordonnées de l'objet de référence
    % #3 coordonnées du point visé
    % #4 coordonées de la position testée
    \pgfmathanglebetweenlines{#2}{#3}{#2}{#4}%
    \global\let#1\pgfmathresult%
  }

  % Arrow
  % \begin{scope}
  %   \path[draw, -, shorten >=12pt, shorten <=5pt] (1,3) |- (3.65,3);
  %   \path[draw, ->,shorten >=5pt, shorten <=5pt] (8.25,3) -- (10,3);
  %   \path[draw, -,shorten >=12pt, shorten <=5pt] (1,-3) |- (3.65,-1);
  %   \path[draw, ->,shorten >=5pt, shorten <=5pt] (8.25,-1) -- (10,-1);
  % \end{scope}
  % Fuzzyfication 1 (Sud)
  \begin{scope}[xshift=-2cm, local bounding box=fuzz]
    \begin{scope}
      \foreach \x in {0,.25,...,1.75}{ \foreach \y in {0,.25,...,1.75}
        {
          % Calcul de l'angle
          \calcangle{\angle}{\pgfpoint{1cm}{1cm}}{\pgfpoint{1cm}{0cm}}{\pgfpoint{\x cm}{\y cm}}
          % Fuzzyfication de l'angle
          \pgfmathsetmacro\fuzzy{%
            ifthenelse(\angle < 90,(90-\angle)/90,
            ifthenelse(\angle > 270,(\angle-270)/90,0)
            )
          }
          % Calcul du rayon à partir de la fuzzyfication
          \pgfmathsetmacro\radius{\fuzzy*2.5}
          \fill[RdBu-9-1] (\x+.125,.125+\y) circle (\radius pt);
        }
      }
      \path[ffc] (0,0) rectangle (2,2);
    \end{scope}
    \node[text width=3cm, align=center, anchor=north, font=\footnotesize] at (1,-.25)
    {\itshape Zone de localisation compatible \normalfont \textcolor{RdBu-9-1}{\textsf{A}}};
  \end{scope} 
  % Fuzzyfication 2
  \begin{scope}[xshift=2cm,local bounding box=fuzz2]
    \begin{scope}
      \foreach \x in {0,.25,...,1.75}{
        \foreach \y in {0,.25,...,1.75}
        {
          % Calcul de l'angle
          \calcangle{\angle}{\pgfpoint{1cm}{1cm}}{\pgfpoint{2cm}{1cm}}{\pgfpoint{\x cm}{\y cm}}
          % Fuzzyfication de l'angle
          \pgfmathsetmacro\fuzzy{%
            ifthenelse(\angle < 90,(90-\angle)/90,
            ifthenelse(\angle > 270,(\angle-270)/90,0)
            )
          }
          % Calcul du rayon à partir de la fuzzyfication
          \pgfmathsetmacro\radius{\fuzzy*2.5}
          \fill[RdBu-9-9] (\x+.125,.125+\y) circle (\radius pt);
        }
      }
      \path[ffc2] (0,0) rectangle (2,2);
    \end{scope}
    \node[text width=3cm, align=center, anchor=north, font=\footnotesize] at (1,-.25)
    {\itshape Zone de localisation compatible \normalfont \textcolor{RdBu-9-9}{\textsf{B}}};
  \end{scope}
  % Fusion 1
  \begin{scope}[xshift=-6cm, yshift=-4cm,local bounding box=fus1]
    \begin{scope}
      \foreach \x in {0,.25,...,1.75}{
        \foreach \y in {0,.25,...,1.75}
        {
          % Calcul des angles
          \calcangle{\angleA}{\pgfpoint{1cm}{1cm}}{\pgfpoint{1cm}{0cm}}{\pgfpoint{\x
              cm}{\y cm}}
          % Fuzzyfication des angles
          \pgfmathsetmacro\fuzzyA{%
            ifthenelse(\angleA < 90,(90-\angleA)/90,
            ifthenelse(\angleA > 270,(\angleA-270)/90,0)
            )
          }
          \calcangle{\angleB}{\pgfpoint{1cm}{1cm}}{\pgfpoint{2cm}{1cm}}{\pgfpoint{\x cm}{\y cm}}
          \pgfmathsetmacro\fuzzyB{%
            ifthenelse(\angleB < 90,(90-\angleB)/90,
            ifthenelse(\angleB > 270,(\angleB-270)/90,0)
            )
          }
          % fusion (zadeh)
          \pgfmathsetmacro\fus{min(\fuzzyA, \fuzzyB)}          
          % Calcul du rayon à partir de la fuzzyfication
          \pgfmathsetmacro\radius{\fus*2.5}
          \fill[RdBu-9-9] (\x+.125,.125+\y) circle (\radius pt);
        }
      }
      \path[ffc2] (0,0) rectangle (2,2);
    \end{scope}
    \node[text width=3cm, align=center, anchor=north, font=\footnotesize] at (1,-.25)
    {\itshape Intersection avec la \emph{t-norme} de \bsc{Zadeh}};
  \end{scope}
  % Fusion 2
  \begin{scope}[xshift=-2cm, yshift=-4cm,local bounding box=fus2]
    \begin{scope}
      \foreach \x in {0,.25,...,1.75}{
        \foreach \y in {0,.25,...,1.75}
        {
          % Calcul des angles
          \calcangle{\angleA}{\pgfpoint{1cm}{1cm}}{\pgfpoint{1cm}{0cm}}{\pgfpoint{\x
              cm}{\y cm}}
          % Fuzzyfication des angles
          \pgfmathsetmacro\fuzzyA{%
            ifthenelse(\angleA < 90,(90-\angleA)/90,
            ifthenelse(\angleA > 270,(\angleA-270)/90,0)
            )
          }
          \calcangle{\angleB}{\pgfpoint{1cm}{1cm}}{\pgfpoint{2cm}{1cm}}{\pgfpoint{\x cm}{\y cm}}
          \pgfmathsetmacro\fuzzyB{%
            ifthenelse(\angleB < 90,(90-\angleB)/90,
            ifthenelse(\angleB > 270,(\angleB-270)/90,0)
            )
          }
          % fusion (luka)
          \pgfmathsetmacro\fus{max(\fuzzyA + \fuzzyB - 1, 0)}          
          % Calcul du rayon à partir de la fuzzyfication
          \pgfmathsetmacro\radius{\fus*2.5}
          \fill[RdBu-9-9] (\x+.125,.125+\y) circle (\radius pt);
        }
      }
      \path[ffc2] (0,0) rectangle (2,2);
    \end{scope}
    \node[text width=3cm, align=center, anchor=north, font=\footnotesize] at (1,-.25)
    {\itshape Intersection avec la \emph{t-norme} de \bsc{Łukasiewicz}};
  \end{scope}
  % Fusion 3
  \begin{scope}[xshift=2cm, yshift=-4cm,local bounding box=fus3]
    \begin{scope}
      \foreach \x in {0,.25,...,1.75}{
        \foreach \y in {0,.25,...,1.75}
        {
          % Calcul des angles
          \calcangle{\angleA}{\pgfpoint{1cm}{1cm}}{\pgfpoint{1cm}{0cm}}{\pgfpoint{\x
              cm}{\y cm}}
          % Fuzzyfication des angles
          \pgfmathsetmacro\fuzzyA{%
            ifthenelse(\angleA < 90,(90-\angleA)/90,
            ifthenelse(\angleA > 270,(\angleA-270)/90,0)
            )
          }
          \calcangle{\angleB}{\pgfpoint{1cm}{1cm}}{\pgfpoint{2cm}{1cm}}{\pgfpoint{\x cm}{\y cm}}
          \pgfmathsetmacro\fuzzyB{%
            ifthenelse(\angleB < 90,(90-\angleB)/90,
            ifthenelse(\angleB > 270,(\angleB-270)/90,0)
            )
          }
          % fusion (zadeh)
          \pgfmathsetmacro\fus{\fuzzyA * \fuzzyB}          
          % Calcul du rayon à partir de la fuzzyfication
          \pgfmathsetmacro\radius{\fus*2.5}
          \fill[RdBu-9-9] (\x+.125,.125+\y) circle (\radius pt);
        }
      }
      \path[ffc2] (0,0) rectangle (2,2);
    \end{scope}
    \node[text width=3cm, align=center, anchor=north, font=\footnotesize] at (1,-.25)
    {\itshape Intersection avec la \emph{t-norme} probabiliste};
  \end{scope}
  % Fusion 4
  \begin{scope}[xshift=6cm, yshift=-4cm,local bounding box=fus4]
    \begin{scope}
      \foreach \x in {0,.25,...,1.75}{
        \foreach \y in {0,.25,...,1.75}
        {
          % Calcul des angles
          \calcangle{\angleA}{\pgfpoint{1cm}{1cm}}{\pgfpoint{1cm}{0cm}}{\pgfpoint{\x
              cm}{\y cm}}
          % Fuzzyfication des angles
          \pgfmathsetmacro\fuzzyA{%
            ifthenelse(\angleA < 90,(90-\angleA)/90,
            ifthenelse(\angleA > 270,(\angleA-270)/90,0)
            )
          }
          \calcangle{\angleB}{\pgfpoint{1cm}{1cm}}{\pgfpoint{2cm}{1cm}}{\pgfpoint{\x cm}{\y cm}}
          \pgfmathsetmacro\fuzzyB{%
            ifthenelse(\angleB < 90,(90-\angleB)/90,
            ifthenelse(\angleB > 270,(\angleB-270)/90,0)
            )
          }
          % fusion (drast)
          \pgfmathsetmacro\fus{%
            ifthenelse(\fuzzyA > 0.999, \fuzzyB,
            ifthenelse(\fuzzyB > 0.999, \fuzzyA, 0)
            )
          }          
          % Calcul du rayon à partir de la fuzzyfication
          \pgfmathsetmacro\radius{\fus*2.5}
          \fill[RdBu-9-9] (\x+.125,.125+\y) circle (\radius pt);
        }
      }
      \path[ffc2] (0,0) rectangle (2,2);
    \end{scope}
    \node[text width=3cm, align=center, anchor=north, font=\footnotesize] at (1,-.25)
    {\itshape Intersection avec la \emph{t-norme} drastique};
  \end{scope}
\end{tikzpicture}