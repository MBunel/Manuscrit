\subsection{Retranscription de l'alerte}
\begin{dialogue}
  \speak{Standard} PGHM de l'Isère, bonjour.
  \Req Oui bonjour, je suis forestier
  à Huez… dans l'Oisan donc… On vient de constater… un parapentiste qui s'est… qui s'est… qui est tombé dans les arbres… là… sur le Roset, donc heu… sur le GR 54 entre le Roset et le pont de Sarenne.
  \speak{Standard} Ouais… Alors attendez un instant… quittez pas…
  \Sec Oui, je vous écoute.
  \Req Oui, bonjour. C'est donc un parapentiste… là… qui a chuté, heu… dans le arbres, heu… sur le GR 54, entre le… le village du Roset et le pont de Sarenne… Donc en face Huez…
  \Sec D'accord… et qui… C'est vous qui l'avez vu ?
  \Req Beh on… on le voit… on vient de le voir de chez nous… on est à quoi… aller trois-cent mètres à vol d'oiseau…
  \Sec D'accord.
  \Req … ouais, on voit la voile qui est dans… dans les arbres… On voit pas trop bouge, quoi.
  \Sec D'accord. Il a chuté de haut ?
  \Req \direct{en aparté} Il était haut, Flo ?
  \speak{Forestier} Non il était pas très haut, visiblement.
  \Sec D'accord. Et vous l'avez vu partir en torche ? Ou c'est…
  \Req Non, non, il est pas parti en torche. Il… il était plutôt à flanc de montagne et il a accroché les arbres en fait.
  \Sec D'accord… Ok… Heu…
  \Req Une voile blanche et orange.
  \Sec Une voile blanche et orange… C'était y'a combien de temps ?
  \Req C'était y'a… Y'a cinq minutes à peu pret.
  \Sec D'accord. Là vous le voyez avec…
  \Req Avec les jumelles on a regardé, on voit personne, on voit pas bouger la voile.
  \Sec D'acccord. Heu… Vous êtes monsieur ?
  \Req \textins{anonymisation}
  \Sec D'accord. Heu… Donc c'est vers Huez… Alors nous on est en train… en même temps… on a notre carte, heu… Vous précisément vous êtes où, de là ou vous voyez la…
  \Req Au coeur du village d'Huez.
  \Sec D'accord.
  \Req Dans la partie basse du village. A peu près.
  \Sec D'accord… \direct{en chuchotant} Partie basse…
  \Req On est à côté de l'église en fait.
  \Sec D'accord…
  \Req Pour plus de précision.
  \Sec D'accord, Ok… D'accord… Vous estimez à combien, à peu près, la distance de là où vous…
  \Req Trois-cent mètres à vol d'oiseau.
  \Sec Trois-cent mètres à vol d'oiseau ? Ouais, d'accord. Et… un peu en contre-bas, c'est ça ?
  \Req Non, il est légèrement plus haut que nous.
  \Sec D'accord…
  \Req On est quatorze-cent, lui y doit être à… à quinze-cent mètres à peu près…
  \Sec Alors…
  \Req … Il est… Il… pff… Ouais, il est vingt, trente mètres au dessus du, du GR 54.
  \Sec D'accord, vingt à trente mètres au dessus… Ben là je regarde la carte en même temps…
  \Req Et donc entre le Roset et… heu… le, le pont de Sarrenne. Si vous voulez y'a une partie… relativement plate… heu… qui va en direction du pont de Sarenne, là, avant de descendre en direction de la vallée…
  \Sec D'accord…
  \Req … en venant du Roset.
  \Sec D'accord. Ouais, donc, alors, le pont de Sarrenne, ok, c'est bon j'ai trouvé sur la carte, et vous dites…
  \Req Le Roset c'est le premier hammeau après.
  \Sec Le Roset, ouais.
  \Req Voila… Ben il est à mi-parcours entre les deux, à peu près.
  \Sec D'accord, Ok.
  \Req On le voit super bien…
  \Sec D'accord…
  \Req … Il est super visible. Il a une voile super visible.
  \Sec D'accord. Et ben… parfait. Vous, vous… Vous venez de nous appeler directement ?
  \Req Heu non, j'ai essayé d'appeler l'altiport avant… à l'Alpe \textins{d'Huez}… 
  \Sec D'accord.
  \Req … mais ça répondait pas… et puis on a cherché sur internet votre numéro…
  \Sec D'accord. Là… Heu… D'accord… Heu… Ce que je vais faire c'est qu'on va se mettre en conférence…
  \Req Ouais…
  \Sec Avec le SAMU centre… \direct{Enregistrement coupé}
\end{dialogue}
