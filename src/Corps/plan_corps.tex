%%%%%%%%%%%%%%%%%%%%%%%%%%%%%%%%%%%%%%%%%%%%%
% Introduction générale
%%%%%%%%%%%%%%%%%%%%%%%%%%%%%%%%%%%%%%%%%%%%%
\addchap{Introduction générale}
\label{part:int}
\subimport{Introduction/}{plan_introduction}
%%%%%%%%%%%%%%%%%%%%%%%%%%%%%%%%%%%%%%%%%%%%%
% Première partie
%%%%%%%%%%%%%%%%%%%%%%%%%%%%%%%%%%%%%%%%%%%%%
\setpartpreamble[u]{%
  \dictum[Walter \bsc{Bonatti}]{\enquote{Je me suis toujours senti
      extrêmement fragile face aux éléments : d'un côté, un squelette
      avec de la chair autour ; de l'autre, les forces auxquelles on
      se frotte, le rocher, la glace, la tempête.}}%

  \dictum[Royal \bsc{Robbins}]{\enquote{A first ascent is a creation
      in the same sense as is a painting or a song.}}%
}
\part{Le secours en montagne et la question du traitement de positions exprimées dans un référentiel indirect}
\label{part:01}
\subimport{Partie1/}{plan_partie}
%%%%%%%%%%%%%%%%%%%%%%%%%%%%%%%%%%%%%%%%%%%%%
% Seconde partie
%%%%%%%%%%%%%%%%%%%%%%%%%%%%%%%%%%%%%%%%%%%%%
\setpartpreamble[u]{%
  \dictum[Pierre \bsc{Dac}]{\enquote{Pour voir loin, il faut y regarder de près.}}%
}
\part{Vers la définition d'une méthode de construction d'une
  \emph{zone de localisation probable} à partir d'une description de
  position}
\label{part:02}
%\subimport{Partie2/}{plan_partie}
%%%%%%%%%%%%%%%%%%%%%%%%%%%%%%%%%%%%%%%%%%%%%
% Troisième partie
%%%%%%%%%%%%%%%%%%%%%%%%%%%%%%%%%%%%%%%%%%%%%
\part{Partie3}
\label{part:03}
%\subimport{Partie3/}{plan_partie}
%%%%%%%%%%%%%%%%%%%%%%%%%%%%%%%%%%%%%%%%%%%%%
% Conclusion générale
%%%%%%%%%%%%%%%%%%%%%%%%%%%%%%%%%%%%%%%%%%%%%
\addchap{Conclusion Générale}
\label{part:cnl}
\subimport{Conclusion/}{plan_conclusion}

%%% Local Variables:
%%% mode: latex
%%% TeX-master: "../../../../Thèse/Manuscrit/src/main"
%%% End:
