Comme nous l'expliquions en introduction, cette thèse s'inscrit dans
le cadre d'un projet de recherche pluridisciplinaire dont l'objectif
principal est de développer des méthodes et des outils facilitant la
localisation de personnes blessées ou perdues en montagne. Notre
travail de recherche s'inscrit donc, à la fois dans le contexte métier
du secours en montagne mais également dans le contexte organisationnel
du projet de recherche Choucas.

L'objectif de ce premier chapitre est de présenter ces deux contextes
et leurs spécificités pour permettre au lecteur d'appréhender au mieux
ce travail doctoral. La présentation du secours en montagne proposée
ici, a pour but de présenter son fonctionnement, son organisation
actuelle et l'historique de cette activité, mais aussi de décrire la
méthode de localisation des victimes utilisée par les secouristes. La
présentation du projet de recherche Choucas, insistera, quant à elle,
sur ses différents objectifs scientifiques du projet et leurs
interactions. Cela nous permettra de distinguer les questions qui
serons abordées dans notre thèse de celles qui ne le seront pas, car
rattachées à d'autres objectifs de recherche.

Dans la première partie de ce chapitre nous présenterons donc les
secours en montagne et leur organisation. Quant à la seconde partie de
ce chapitre elle abordera la question de l'inscription de cette thèse
au sein du projet de recherche Choucas.

%%% Local Variables:
%%% mode: latex
%%% TeX-master: "../../../main"
%%% End:
