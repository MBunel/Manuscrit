\subsection{Le secours en montagne dans les alpes occidentales}
\label{subsec:1-1-1}

Comment arrive t'on a l'organisation actuelle, secrous public, partagé
entre plusieurs corps…


\begin{verbatim}
Expliquer que le secours comme on le voit auj, Professionel, héliporté
et médicalisé est quelque chose qui s'est construit avec le
temps. Pendant longtemps c'était des amateurs et la présence
systématique de médecins est assez récente (Halle)
\end{verbatim}

\subsubsection{Historique}
\label{subsubsec:1-1-1-1}

Nous ne souhaitons pas entreprendre un travail d'historien, très
éloigné de nos compétances. Cette partie n'est pas construitre pour
répondre à une problématique historique, mais plutôt pour dresser un
état des lieux du secours en montagne.

Malheuresement ces événements nous sont apparus peu documentés.

\paragraph{Le proto-secours en montagne}
\label{par:1-1-1-1-1}

L'histoire des secours en montagne est indissociable de celle de
l'alpinisme. Si l

\begin{verbatim}
premiers secours col du saint-bernard
\end{verbatim}


Dans les premiers temps les secours sont avant tout l’œuvre des
alpinistes eux-mêmes qui se regroupent dans des \emph{comités de
  secours} associatifs et agissant à l'échelle locale. Étonnamment le
premier de ces comités n'est pas créé dans un centre de l'alpinisme,
comme Chamonix, mais dans les préalpes avec la fondation, en 1897, des
\emph{sauveteurs volontaires du Salève}.

\begin{verbatim}
Avant secours occasionels et non organisés

1888 Création des régiments alpins

1897 les ``sauveteurs volontaires du  Salève sont le premier groupement
de secouristes en france. Sans doute à cause d'une pression touristique.

1910 Le syndicat d'initiatives de Grenoble fonde le comité de secours
en montagne du Dauphiné

1911 L'article « l'accident de l'aiguille du plan » dresse un aperçu
des problèmes liés à une organisation des secours basés sur le
volontariat.
\end{verbatim}

C'est au tournant des années 30 que seront crées la majorité des
commités de secours. Avec, notamment, la création d'un commité de
secours à Annecy en 1928, à Chamonix en 1929, à Briançon en 1932 ou à
Pau en 1936. \texttt{Source la montagne 46}

\begin{verbatim}
1928 Création du comité de secours annécien

1929 ou 27 selon sopurces Le commité de secours du dauphiné devient la
société dauphinoise de secours en montagne. Son organisation est
décrite dans la montagne.

1932 Création de l'\ac{ehm}

1932 Création d'un commité de secours à Briançon source la montagne 46

1936 Création d'un commité de secours à Pau
\end{verbatim}


1946 Lucien Devies publie l'orgaganisation des secours en montagne. Il
met en évidence le retard de la France en manière de secours en
montagne comparé à l'Allemangne, l'autriche et la Suisse. Il appelle
les autorités publiques à prendre le controle des secours -> le
secours en montagne comme service public.

\paragraph{Le début d'une organisation nationale}

\begin{verbatim}
1947 Création de la comission de secours au sein de la ffm
\end{verbatim}


En 1947, deux ans après la création de la fédération française de
montagne \acp{ffm}, une commission de secours en montagne est crée et
Lucien \bsc{Devies} en est élu président. Cette commission fédére les
différents comités locaux de secours en montagne.

En 1948, cette commission effectue sa première opération
d'envergure. Deux alpinistes sont coincés dans la face sud, d'une
verticalité extrème, du Pavé dans le massif des Écrins. 

\begin{verbatim}
1948 Secours au Pavé, aide de l'\ac{ehm}. Première grosse opération de la ffm

1950 premières participations volontaires de gendarmes et de crs à des
opérations de secours.

1950 Secours de l'abiou

1955 La société Chamoniarde de secours en montagne obtient, pâr la
protection civile un hélicoptète, utilisé dès 1956.

1956 Création d'un centre d'entrainement en montangne des crs à
chamoinx.

fin 1956 Accident de viencendon et Henry. Organisation des secours par
l'ehm. Grand choc dans la société. Début de transition vers un secours
public.
\end{verbatim}

\paragraph{Le secours en montagne comme service public}
\label{par:1-1-1-1-2}

\begin{verbatim}
1958 le secours devient un service public, création du plan ORSEC

Formation des premiers gendarmes par l'ehm
\end{verbatim}


La décénie suivante vera la mise en place progressive de l'outil
aéroporté. L'alouette 3, mise en service en 1963 sera, jusqu'a son
remplacement en 2009, l'unique hélicoptère a la disposition des
secouristes. L'outil héliporté sera peu à peu amélioré avec le
développement de solutions techniques, tel que le treuil, dont la
première utilisation identifiée date de 1967, le treuillage de
civivères, ou les opérations avec secouriste suspendu. À partir de
1972 l'ensemble des zones montagneuses seront à portée
d'hélicoptère. Les secours héliportés occuperons une place de plus en
plus importante dans les opérations de secours en montagne, jusqu'a
atteindre, une part de plus de 80 \% des secours durant la décennie
actuelle.

\begin{verbatim}
1960 Mise en place progressive outil aéroporté

1963 mise en service de l'alouette 3

1967 premier treuillage

1972 Toutes le zones montagneuses françaises sont couvertes par un
hélicoptère

Parler de l'introduction des civières, des treuils etc.

1980 généralisation de la présence d'un médecin

1985 GMPS

1985-1987 gratuité des secours

2002 mise en cause de la gratuité

2007 Prise de position d'alio-marie sur la gratuité des secours

Poids de l'hélico 80 \% en moyenne sur la période 2015-2019 en domaine
montagne selon le \ac{snosm}.
\end{verbatim}

\begin{figure}
  \centering
   \begin{tikzpicture}[%
  ev/.style={font=\tiny, text centered, inner sep=2pt, fill=white}
  ]
  % Création de la Frise
  \begin{scope}
    % Axe dégradé à gauche
    \path[draw,-,path fading=west] (-.5,0) -- (0,0);
    % Axe principal
    \path[draw,->] (0,0) --++ (12,0);

    % Trait vertical tous les .75 cm (10 ans)
    \foreach \x in {0,.75,...,11.75}{
      \path[draw] (\x,-.05) --(\x,.05);
    }

    % Étiquettes de décénine
    \node[below, font=\tiny] at (0, -.05) {1860};
    \node[below, font=\tiny] at (.75, -.05) {1870};
    \node[below, font=\tiny] at (1.5, -.05) {1880};
    \node[below, font=\tiny] at (2.25, -.05) {1890};
    \node[below, font=\tiny] at (3, -.05) {1900};
    \node[below, font=\tiny] at (3.75, -.05) {1910};
    \node[below, font=\tiny] at (4.5, -.05) {1920};
    \node[below, font=\tiny] at (5.25, -.05) {1930};
    \node[below, font=\tiny] at (6, -.05) {1940};
    \node[below, font=\tiny] at (6.75, -.05) {1950};
    \node[below, font=\tiny] at (7.5, -.05) {1960};
    \node[below, font=\tiny] at (8.25, -.05) {1970};
    \node[below, font=\tiny] at (9, -.05) {1980};
    \node[below, font=\tiny] at (9.75, -.05) {1990};
    \node[below, font=\tiny] at (10.5, -.05) {2000};
    \node[below, font=\tiny] at (11.25, -.05) {2010};  
  \end{scope}

  % Étiquettes partie supérieure
  \foreach \date/\text/\y [evaluate=\date as \x using (\date - 1860)*.075] in {
    1956/{Accident de Vincendon et Henry}/{4.5},
    1947/{Création de la commission des secours en montagne de la \ac{ffm}}/{3.25},
    1985/{Création des \ac{gmsp}}/{3},
    1958/{Plan \ac{orsec}}/{2.5},
    1897/{Création des Sauveteurs volontaires du Salève}/{2.5},
    1865/{Accident du Cervin}/{1.5},
    2002/{Amendement à la loi sur la démocratie de proximité}/{2},
    1929/{Création d'un comité de secours Savoyard}/{1.75},
    1970/{Avalanche à Val-d'Isère}/{1.5},
    1932/{Création d'un comité de secours Briançonnais}/{.5},
    1874/{Création du \ac{caf}}/{.5},
    1987/{Circulaire du 22 septembre 1987}/{.5}
  } {
    \node[shape=circle,fill=black, scale=.25] (a1) at (\x,0) {};
    \path[draw, -|] (a1) --++ (0,\y) node[ev, above] {\begin{varwidth}{3.5cm}\centering\text\\(\date)\end{varwidth}};
  }

  % Étiquettes partie inférieure
  \foreach \date/\text/\y [evaluate=\date as \x using (\date - 1860)*.075] in {
    1980/{Généralisation de la présence des médecins dans les équipes de secours}/{3.75},
    1956/{Premier secours aéroporté}/{3},
    1958/{Création du \ac{pghm}}/{2},
    1910/{Création du comité de secours en montagne du Dauphiné}/{2},
    2007/{La ministre de l'Intérieur confirme la gratuité des secours}/{1.75},
    1945/{Création de la \ac{ffm}}/{1.25},
    1888/{Création des régiments alpins}/{.5},
    1972/{Couverture aérienne complète}/{.5},
    1932/{Création de l'\ac{ehm}}/{.5}
  } {
    \node[shape=circle,fill=black, scale=.25] (a1) at (\x,0) {};
    \path[draw,-|] (\x, -.4) --++ (0,-\y) node[ev, below] {\begin{varwidth}{3.5cm}\centering\text\\(\date)\end{varwidth}};
  }
\end{tikzpicture}
   \caption{Chronologie des principaux événements de
     l'histoire fraçaise des secours en montagne.}
  \label{fig:frise_chronologique}
\end{figure}


\subsubsection{Organisation contemporaine}
\label{subsubsec:1-1-1-2}

\begin{verbatim}
Trois corps principaux

Deux organisations différentes, pistes et hors-piste.
\end{verbatim}


\paragraph{Le secours en station}

\begin{verbatim}
voir : https://www.pompiers.fr/actualites/ou-sapplique-la-gratuite-des-secours-en-montagne
\end{verbatim}


\paragraph{Le secours hors-piste}



\begin{quotation}
  «~Ce système de la mixité s'est progressivement imposé, mais il
  n'est pas transposable et devrait deumerer une spécificité
  haut-savoyarde.~»
\end{quotation}

\begin{table}
  \centering
  \begin{tabular}{L{6cm}L{8cm}}
  \toprule
  \multicolumn{1}{c}{\bfseries Départements} & \multicolumn{1}{c}{\bfseries Organisation} \\
  \midrule
  Hautes-Alpes (05), Alpes-Maritimes (06), Isère (38), Savoie (73) & Alternance hebdomadaire \ac{crsm}, \ac{pghm}\\
  Alpes-de-Haute-Provence (04) & \ac{pghm} uniquement\\
  Haute-Savoie (74) & Collaboration \ac{pghm}, \ac{gmsp}, sauf dans dans le massif du Mont-Blanc (\ac{pghm} uniquement)\\
  \bottomrule
\end{tabular}
  \caption{Corps mobilisés pour le secours en montagne dans les
    départements alpins.}
  \label{tab:organisation_secours_departements}
\end{table}

\subsection{Méthodologie du secours en montagne}
\label{susec:1-1-2}

\begin{verbatim}
Selon, Halle 2007, le déclanchement même de l'alerte peut-être
problématique. 
\end{verbatim}

\subsection{Limites Méthodologiques}
\label{subsec:1-1-3}

\texttt{Présenter une image de la zone du fil rouge ?}
%%% Local Variables:
%%% mode: latex
%%% TeX-master: "../../../../main"
%%% End:
