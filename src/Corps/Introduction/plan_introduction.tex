On pourrait commencer toutes les thèses en géomatique de la même
manière : l'informatisation de la société et plus spécifiquement des
données géographiques a révolutionné notre rapport à l'espace et a sa
représentation archétypale, la carte, \emph{etc.} Pourtant 

Toutefois, des descriptions de configurations spatiales parfaitement
courantes et inéligibles pour le locuteur humain, comme \enquote{La
  boulangerie à côté de chez moi fait les meilleurs croissants de
  Paris}, peuvent être extrêmement difficiles à interpréter
informatiquement. Les raisons sont multiples, mais 



La transformation d'une description de position en une zone définie
par des coordonéess

%-> geo-information
\emph{Geographic Information Retrieval} \autocite{Jones2008}


%-> transition secours en montagne "le problème devient grave"

La difficulté de cette opération peut devenir contraignante
lorsque la rapidité de la localisation est indispensable, notamment
dans les secours a la personne. Plusieurs travaux ont donc cherché a
méliorer le processus de 

\autocite{DosSantosFerreria2019} 

% -> transition choucas
Cette difficulté de localisation des victimes est d'autant plus
important lorsqu'il est nécessaire de localiser des personnes perdues
dans des espaces offrant peu de points de repères. C'est par exemple
le cas pour les unités de secours en montagne \acp{usem}, qui sont
régulièrement amenés à secourir des personnes partiellement ou
totalement perdues en montagne.

Pour ces unités le problème de la difficulté de localisation est
accentué par ces implications. Les \ac{usem} disposant d'un nombre
limité de moyens humains et matériels, la focalisation sur une alerte
peut impacter les autres cas.

Pour faciliter la localisation des victimes en montagne les
secouristes du \ac{pghm} de Grenoble ont développé une solution,
Gend'Loc, permettant de localiser le requérant à l'aide du GPS intégré
a son téléphone portable. Cette solution à l'avantage de permettre une
localisation rapide et assez facile à utiliser pour le requérant, mais
elle a inconvénient majeur d'être dépendante du téléphone du requérant
et de son bon fonctionnement. De nombreux facteurs peuvent rendre
cette solution inopérante, notamment lorsque le requérant ne dispose
pas d'un téléphone avec GPS, lorsque les secours sont contactés par un
tiers ou lorsque la victime n'arrive pas à effectuer la manipulation
nécessaire.

Dans ces situations les secouristes n'ont d'autres choix que
d'identifier la position manuellement, en recoupant les informations
qui leur sont données par téléphone par le requérant avec leurs
connaissances personnelles et les données géographiques qu'ils ont à
leur disposition. Le résultat de ce processus est assez variable, si
les secouristes peuvent parvenir à identifier très rapidement la
position de la victime a l'aide de quelques éléments très
discriminants, il est parfois difficile pour la victime de donner des
informations qui soient suffisamment claires ou discriminantes pour
que les secouristes puissent les interpréter. Dans ces conditions, la
phase de localisation peut devenir extrêmement longue et critique.

Le projet de recherche Choucas a été construit en vue de développer
des solutions permettant de faciliter le traitement de ces alertes
spécifiques. L'hypothèse première de ce projet est que la géomatique
et ses méthodes peuvent aider à identifier les zones correspondant aux
informations données par les victimes. En effet, de nombreuses
méthodes qui ont été proposées dans le domaine de la géomatique
pourraient être employés a cet effet. Par exemple les travaux autour
des questions de visibilité pourraient être exploités pour identifier
la position d'un requérant décrivant son champ de vision, les travaux
sur la modélisation de trajectoire pourraient être utilisés pour
localiser une personne indiquant qu'elle a suivi un itinéraire donné,
\emph{etc.}

Un tel objectif n'est cependant pas réalisable par la \enquote{simple}
concaténation de méthodes préexistantes et impose de travailler à
l'étude et a l’enrichissement des données utilisables par les
secouristes, de définir des solutions leur permettant d’interagir avec
les solutions développées au sein du projet, d'étudier les subtilités
et les spécificités des descriptions de localisation en milieux
montagneux, \emph{etc.} 

% Présentaiont thèse
Notre thèse se rattache spécifiquement à ce dernier point. Plus
précisément notre objectif est de proposer des solutions permettant de
construire les zones correspondant aux descriptions de positions
données par les requérants. Notre objectif n'est cependant pas de
créer un \enquote{secouriste virtuel}, réalisant l'ensemble des tâches
effectuées manuellement pas ces derniers, mais de proposer des
solutions d'assistance, facilitant l'identification des zones
correspondant à une description de position.

% inteprétation sémantique

% spatialisaiton

% fusion
Il est également nécessaire de se demander comment regrouper toutes
les spatialisations d'une alerte pour construire la zone correspondant
à l'ensemble des descriptions données par le réquérant et don à la
position de la victime.

% prise en compte imprécision

% incertitude



% Pas de Tal
Notre propos n'est cependant pas de travailler sur le \emph{traitement
  automatisé des langues} (TAL), 

%\begin{figure}
%   \centering
%   \includegraphics{./figures/Le_couloir_Rochette.jpg}
%   \caption{\emph{Le couloir,} Jean-Marc \bsc{Rochette,} 2010.}
%   \label{fig:couloir_rochette}
% \end{figure}

\addsec{Organisation du manuscrit}

Notre mémoire de thèse est organisé en trois parties, totalisant neufs
chapitres. Dans la \autoref{part:01} nous présenterons le cadre
général dans lequel notre travail s'inscrit. Le premier chapitre sera
dédié a la présentation du contexte applicatif et organisationnel de
cette thèse. Nous détaillerons le rôle et le fonctionnement des
secours en montagne, les problèmes soulevés par la localisation des
victimes et la manière dont le projet CHOUCAS souhaite y répondre. Le
second chapitre est destiné à présenter le contexte de scientifique et
la problématisation de notre thèse. Enfin, dans le troisième chapitre
nous dresserons un état de l'art sur les questions principales de ce
travail.

La \autoref{part:02} est destinée à présenter notre méthode de
transformation des positions exprimées dans un référentiel indirect en
des zones de localisation. Elle est constituée de cinq chapitres. Le
premier chapitre de cette partie présente l’organisation générale de
notre méthode. Les chapitres suivants approfondissent des points
spécifiques de cette méthode. Dans le \autoref{chap:05} nous
détaillerons le fonctionnement de la première phase de la méthode,
\emph{la décomposition.} Le \autoref{chap:06} sera quant à lui
consacré a la définition d'un modèle permettant la représentation
d'objets géographiques imprécis, tels que ceux produits par notre
méthode. Dans le \autoref{chap:07} nous présenterons la seconde phase
de notre méthode, \emph{la spatialisation.} Le dernier chapitre de
cette seconde partie sera consacré à la présentation de la dernière
phase de notre méthode, la \emph{fusion.}

Enfin, la \autoref{part:03} de ce manuscrit présente l’application de
la méthode définie a plusieurs alertes réelles. Cette partie ne
contient qu'un seul chapitre, le neuvième, où notre méthode est
appliquée à la modélisation de trois alertes réelles.


%%% Local Variables:
%%% mode: latex
%%% TeX-master: "../../main"
%%% End:
