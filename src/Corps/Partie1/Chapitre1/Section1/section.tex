\subsection{Le secours en montagne dans les alpes occidentales}
\label{subsec:1-1-1}

Comment arrive t'on a l'organisation actuelle, secrous public, partagé
entre plusieurs corps…

\subsubsection{Historique}
\label{subsubsec:1-1-1-1}

Nous ne souhaitons pas entreprendre un travail d'historien, très
éloigné de nos compétances. Cette partie n'est pas construitre pour
répondre à une problématique historique, mais plutôt pour dresser un
état des lieux du secours en montagne.

Malheuresement ces événements nous sont apparus peu documentés.

\paragraph{Le proto-secours en montagne}
\label{par:1-1-1-1-1}

1888 Création des régiments alpins

1897 les ``sauveteurs volontaires du Salève sont le premier groupement
de secouristes en france.

1910 Le syndicat d'initiatives de Grenoble fonde le comité de secours
en montagne du Dauphiné

1911 L'article « l'accident de l'aiguille du plan » dresse un aperçu
des problèmes liés à une organisation des secours basés sur le
volontariat.

1928 Création du comité de secours annécien

1929 Le commité de secours du dauphiné devient la société dauphinoise
de secours en montagne. Son organisation est décrite dans la montagne.

1932 Création de l'\ac{ehm}

1933 Création d'un commité de secours à Briançon

1936 Création d'un commité de secours à Pau

1946 Lucien Devies publie l'orgaganisation des secours en montagne. Il
met en évidence le retard de la France en manière de secours en
montagne comparé à l'Allemangne, l'autriche et la Suisse. Il appelle
les autorités publiques à prendre le controle des secours -> le
secours en montagne comme service public.

1948 Secours au Pavé, aide de l'\ac{ehm}

1950 premières participations volontaires de gendarmes et de crs à des
opérations de secours.

1950 Secours de l'abiou

1955 La société Chamoniarde de secours en montagne obtient, pâr la
protection civile un hélicoptète, utilisé dès 1956.

1956 Création d'un centre d'entrainement en montangne des crs à
chamoinx.

fin 1956 Accident de viencendon et Henry. Organisation des secours par
l'ehm. Grand choc dans la société. Début de transition vers un secours
public.


\paragraph{La circulaire du 21 août 1958 et la mise en place des plans ORSECs}
\label{par:1-1-1-1-2}

1958 le secours devient un service public, création du plan ORSEC

Formation des premiers gendarmes par l'ehm

1960 Mise en place progressive outil aéroporté

1972 Toutes le zones montagneuses françaises sont couvertes par un
hélicoptère

1980 généralisation de la présence d'un médecin

1985 GMPS

1985-1987 gratuité des secours

2002 mise en cause de la gratuité

2007 Prise de position d'alio-marie sur la gratuité des secours

Poids de l'hélico 80 \% en moyenne sur la période 2015-2019 en domaine
montagne selon le \ac{snosm}.

\begin{figure}
  \centering
   \begin{tikzpicture}[%
  ev/.style={font=\tiny, text centered, inner sep=2pt, fill=white}
  ]
  % Création de la Frise
  \begin{scope}
    % Axe dégradé à gauche
    \path[draw,-,path fading=west] (-.5,0) -- (0,0);
    % Axe principal
    \path[draw,->] (0,0) --++ (12,0);

    % Trait vertical tous les .75 cm (10 ans)
    \foreach \x in {0,.75,...,11.75}{
      \path[draw] (\x,-.05) --(\x,.05);
    }

    % Étiquettes de décénine
    \node[below, font=\tiny] at (0, -.05) {1860};
    \node[below, font=\tiny] at (.75, -.05) {1870};
    \node[below, font=\tiny] at (1.5, -.05) {1880};
    \node[below, font=\tiny] at (2.25, -.05) {1890};
    \node[below, font=\tiny] at (3, -.05) {1900};
    \node[below, font=\tiny] at (3.75, -.05) {1910};
    \node[below, font=\tiny] at (4.5, -.05) {1920};
    \node[below, font=\tiny] at (5.25, -.05) {1930};
    \node[below, font=\tiny] at (6, -.05) {1940};
    \node[below, font=\tiny] at (6.75, -.05) {1950};
    \node[below, font=\tiny] at (7.5, -.05) {1960};
    \node[below, font=\tiny] at (8.25, -.05) {1970};
    \node[below, font=\tiny] at (9, -.05) {1980};
    \node[below, font=\tiny] at (9.75, -.05) {1990};
    \node[below, font=\tiny] at (10.5, -.05) {2000};
    \node[below, font=\tiny] at (11.25, -.05) {2010};  
  \end{scope}

  % Étiquettes partie supérieure
  \foreach \date/\text/\y [evaluate=\date as \x using (\date - 1860)*.075] in {
    1958/{Intégration des secours en montagne au plan \ac{orsec}}/{4.5},   
    1947/{Création de la commission des secours en montagne de la \ac{ffm}}/{3.25},
    1985/{Création des \ac{gmsp}}/{3},	    
    1897/{Création des Sauveteurs volontaires du Salève}/{2.5},
    1956/{Accident de Vincendon et Henry}/{2.25},		
    1865/{Accident du Cervin}/{1.5},
    2002/{loi \no 2002-276 relative a la démocratie de proximité}/{2},
    1929/{Création d'un comité de secours Savoyard}/{1.75},
    1970/{Avalanche à Val-d'Isère}/{1.5},
    1932/{Création d'un comité de secours Briançonnais}/{.5},
    1874/{Création du \ac{caf}}/{.5},
    1987/{Circulaire du 22 septembre 1987}/{.5}
  } {
    \node[shape=circle,fill=black, scale=.25] (a1) at (\x,0) {};
    \path[draw, -|] (a1) --++ (0,\y) node[ev, above] {\begin{varwidth}{3.5cm}\centering\text\\(\date)\end{varwidth}};
  }

  % Étiquettes partie inférieure
  \foreach \date/\text/\y [evaluate=\date as \x using (\date - 1860)*.075] in {
    1986/{Loi \no 86--11 relative à l'aide médicale urgente et aux transports sanitaires}/{3.75},
    1956/{Premier secours aéroporté}/{3},
    1958/{Création du \ac{pghm}}/{2},
    1910/{Création du comité de secours en montagne du Dauphiné}/{2},
    2007/{La ministre de l'Intérieur confirme la gratuité des secours}/{2},
    1945/{Création de la \ac{ffm}}/{1.25},
    2011/{Circulaire Kihl}/{1.},
    1888/{Création des régiments alpins}/{.5},
    1972/{Couverture aérienne complète}/{.5},
    1932/{Création de l'\ac{ehm}}/{.5}
  } {
    \node[shape=circle,fill=black, scale=.25] (a1) at (\x,0) {};
    \path[draw,-|] (\x, -.4) --++ (0,-\y) node[ev, below] {\begin{varwidth}{3.5cm}\centering\text\\(\date)\end{varwidth}};
  }
\end{tikzpicture}
   \caption{Chronologie des principaux événements de
     l'histoire fraçaise des secours en montagne.}
  \label{fig:frise_chronologique}
\end{figure}


\subsubsection{Organisation contemporaine}
\label{subsubsec:1-1-1-2}

Trois corps principaux

Deux organisations différentes, pistes et hors-piste.

\paragraph{Le secours en station}

voir : https://www.pompiers.fr/actualites/ou-sapplique-la-gratuite-des-secours-en-montagne

\paragraph{Le secours hors-piste}



\begin{quotation}
  «~Ce système de la mixité s'est progressivement imposé, mais il
  n'est pas transposable et devrait deumerer une spécificité
  haut-savoyarde.~»
\end{quotation}

\begin{table}
  \centering
  \begin{tabular}{L{6cm}L{8cm}}
  \toprule
  \multicolumn{1}{c}{\bfseries Départements} & \multicolumn{1}{c}{\bfseries Organisation} \\
  \midrule
  Hautes-Alpes (05), Alpes-Maritimes (06), Isère (38), Savoie (73) & Alternance hebdomadaire \ac{crsm}, \ac{pghm} (selon un calendrier national).\\
  Alpes de Haute-Provence (04) & \ac{pghm} uniquement.\\
  Haute-Savoie (74) & Collaboration \ac{pghm}, \ac{gmsp}, sauf dans la région de Chamonux (\ac{gmsp} uniquement) et dans le massif du Mont-Blanc (\ac{pghm} uniquement). \\
  \bottomrule
\end{tabular}
  \caption{Corps mobilisés pour le secours en montagne dans les
    départements alpins.}
  \label{tab:organisation_secours_departements}
\end{table}


\subsection{Méthodologie du secours en montagne}
\label{susec:1-1-2}

\subsection{Limites Méthodologiques}
\label{subsec:1-1-3}

\texttt{Présenter une image de la zone du fil rouge ?}
%%% Local Variables:
%%% mode: latex
%%% TeX-master: "../../../../main"
%%% End:
