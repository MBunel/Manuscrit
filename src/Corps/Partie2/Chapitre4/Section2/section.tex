Dans le \autoref{chap:2} nous identifions deux étapes centrales de
notre méthode permettant de construire une \emph{zone de localisation
  probable} correspondant à \emph{une description de position,} la
\emph{spatialisation} et \emph{la fusion.} Cette présentation n'est
cependant satisfaisante qu'en première approximation. En effet, les
principes introduits dans ce chapitre (et plus particulièrement le
principe de décomposition) complexifient la méthodologie en y ajoutant
de nouvelles étapes, comme différentes phases de décomposition et de
fusion. Ainsi nos précédentes définitions (\autoref{chap:02}) de la
\emph{spatialisation} et de la \emph{fusion} doivent êtres complétées,
ce qui sera fait au fur et à mesure de cette partie.

Dans cette partie nous allons présenter une première formalisation de
notre méthodologie. L'objectif n'est cependant pas de présenter tous
les aspects des étapes de \emph{décomposition,} de
\emph{spatialisation} et de \emph{fusion} chaque étape de la méthode
étant détaillée dans un chapitre dédié (respectivement les chapitres
\ref{chap:05}, \ref{chap:07} et \ref{chap:08}). Nous ne parlerons donc
pas de certains aspects avancés de la modélisation, comme la prise en
compte de l'imprécision (cf. \autoref{chap:07}) ou de l'incertitude
(cf.). La méthode ici présentée est la plus générique possible  

La \autoref{fig:methodo_1} présente une synthèse visuelle de la
méthodologie que nous avons définie.

\subsection{La phase de décomposition}

La \emph{décomposition} est la première phase de notre méthode. C'est
un ensemble de trois étapes permettant de passer d'un ensemble
\emph{d'indices de localisation} à un ensemble décomposé de ces
indices, prêt à être \emph{spatialisé.} Seule une de ces étapes (la
dernière) découle du principe de \emph{décomposition} des \emph{relations de localisation.}


La première étape de cette phase est celle de la décomposition de
\emph{l'ensemble des indices de localisation}, \(I\)
(\autoref{fig:methodo_1}). Cet ensemble contient tous les
\emph{indices de localisation} qui nous sont transmis par les
secouristes, par le biais de l'interface de géovisualisation. Comme
l'indique la \autoref{fig:diag_acti_secours}, c'est la réception de
cet ensemble qui déclenche le processus de construction de la
\emph{zone de localisation probable.} La décomposition de l'ensemble
\(I\) permet d'extraire les différents \emph{indices de localisation}
(\(i_n\) sur la \autoref{fig:methodo_1}), à partir desquels serons
construites les\emph{ zones de localisation compatibles.} Cette
transformation était qualifée de \emph{spatialisation} dans le
\autoref{chap:02}, cependant, comme le montre la
\autoref{fig:methodo_1}, ce n'est pas à cette étape qu'apparaissent
les objets géographiques, sous la forme des \emph{zones de
  localisation compatibles,} mais après deux nouvelles étapes de
\emph{décomposition.} C'est pourquoi nous préférons réserver l'usage
de ce terme à l'étape centrale de la méthode, la création d'objets
géographiques. L'étape de décomposition de \emph{l'ensemble des
  indices de décomposition,} ne présente pas de difficultés techniques
ou d'apport scientifique, mais elle permet la \emph{spatialisation
  autonome des indices de localisation.} Comme on peut le voir sur la
\autoref{fig:methodo_1}, les indices décomposés lors de la première
étape du processus sont fusionnés au dernier moment, conformément au
principe de modélisation autonome.

\tdi{équations}

La seconde étape de la phase de décomposition aborde un point que
nous n'avons pas encore présenté.
%
les \emph{indices de localisation} contiennent toujours un (ou
plusieurs, si la relation de localisation est n-aire) \emph{objet de
  référence.} Cependant ce dernier peut être une instance clairement
définie, que l'on peut alors facilement identifier (\eg
\enquote{Grenoble}, \enquote{Le bois des Ayes}, \emph{etc.}) ou un
objet décrit par sa nature, qu'il est alors impossible d'identifier
sans informations supplémentaires (\eg \enquote{une ville},
\enquote{un bois}). Par conséquent, lorsque le secouriste identifie
les objets correspondant à la description du requérant (conformément
au principe \texttt{XXXX}, cf. \ref{sec:4-1-1-1}) il peut sélectionner
un ou plusieurs objets. Si \emph{l'indice de localisation} qui nous
est transmis contient plus d'un objet (dans le cas où la relation
modélisée n'est pas bi ou n-aire) alors \emph{l'objet de référence}
est indéfini est \emph{l'indice de localisation} doit être
\emph{spatialisé} pour chacun d'entre eux et la position correspondant
à \emph{l'indice de localisation} et l'une des positions possibles.
La seconde étape de la \emph{phase de décomposition} permet donc de
transformer un \emph{indice de localisation} comportant plusieurs
\emph{objets de référence} concurrents en un ensemble \emph{d'indices
  de localisation} portant chacun sur un des \emph{objets de référence
  possible.} Prenons l'exemple de \emph{l'indice de localisation}
\enquote{je suis sur une crête}. Le secouriste peut alors sélectionner
à l'aide de l'interface l'ensemble des objets géographiques
\enquote{crêtes} situés dans la ZIR \footnote{Voir dans un espace plus
  réduit si le secouriste décide, compte-tenu de ses connaissances et
  de la situation, de réduire la zone de recherche de cette catégorie
  d'objets.}. Admettons qu'elles soient au nombre de six,
\emph{l'indice de localisation} qui nous sera transmis sera donc de la
forme \enquote{je suis sur la \emph{crête de Roche Motte} ou je suis
  sur la \emph{crête de Font Froide} ou je suis sur la \emph{crête de
    Serre Chapelle} ou je suis sur la \emph{crête des Barres} ou je
  suis sur la \emph{crête du Petit Puy}}. Ainsi au sortir de cette
seconde étape de décomposition on dispose de \(i\) \emph{indices de
  localisation} (première décomposition), eux-mêmes divisés en
plusieurs \emph{indices de localisation,} en fonction du nombre
\emph{d'objets de référence} à prendre en compte.

La troisième est dernière phase de décomposition est celle de la
décomposition des \emph{relations spatiales.}

Ce principe sera détaillé dans le \autoref{chap:05}

\subsubsection{objets de ref}

\subsubsection{La décomposition des relations spatiales}

\subsection{Spatialisation des \emph{indices de localisation}}

\subsubsection{La décomposition des relations de localisation}

\subsubsection{La spatialisation des relations de localisation atomiques}


Faite pour chaque indice (chap2)
Nécessite des méthodes spécifiques (eda)
Plusieurs objets de référence

Ce principe sera détaillé dans le \autoref{chap:07}

Pour rappel, l'opération de \emph{spatialisation} consiste à
construire une zone (dite de \emph{localisation compatible})
correspondant à \emph{l'indice de localisation}
(\autoref{fig:obj_spa}). Ainsi on peut définir une \emph{zone de
localisation compatible} (pour un \emph{indice de localisation} donné)
comme l'ensemble des positions où \emph{l'indice de localisation} est
vrai, soit :


\begin{equation}
  \text{\textsf{ZLC}}_i = \{(x,y) \ |\ \text{\textsf{RL}}_i(x,y) \wedge (x,y) ∈ \mathbb{R}^2\}
\end{equation}

Avec \textsf{ZLC} la \emph{zone de localisation compatible,} \(i\) un
\emph{indice de localisation,} \((x,y)\) un couple de réels formant
une position et \(\text{\textsf{RL}}_i(x,y)\) une fonction vraie si la
position \((x,y)\) est dans la \emph{zone de localisation compatible}
correspondant à \emph{l'indice de localisation} \(i\).

\subsection{Fusion}

De toutes les étapes de notre méthodologie, la phase de \emph{fusion}
est, de loin, la plus simple. Son objectif est de combiner les
différentes \emph{zones de localisation compatibles} créées lors de la
phase de \emph{spatialisation} en une seule, la \emph{zone de
  localisation probable} (ZLP).

Ce principe sera détaillé dans le \autoref{chap:08}



Contrairement à la
décomposition de l'ensemble des \emph{indices de localisation}
(l'étape précédente) on ne s'attend pas ici à ce que toutes les
possibilités soient vraies, mais seulement l'une d'entre-elles (d'où
la formulation contenant des \enquote{ou} et non des \enquote{et})


\begin{equation}
  \text{\textsf{ZLP}}_I = ⋂_{i \in I} \text{\textsf{ZLC}}ᵢ
\end{equation}

Avec \textsf{ZLP} la \emph{zone de localisation probable,} \(I\),
l'ensemble des \emph{indices de localisation} pris en compte, \(i\) un
\emph{indice de localisation} et \textsf{ZLP} un \emph{zone de
  localisation compatible.}

\begin{landscape}
  \begin{figure}[H]
    \centering
    \begin{tikzpicture}
  \tikzset{
    acc/.style={decorate,decoration={brace,raise=0cm,amplitude=.1cm}},
    accm/.style={acc,decoration={mirror}},
    acc3/.style={acc,decoration={amplitude=.2cm}},
    acc3m/.style={acc3,decoration={mirror}}
  }
  
  %\draw[step=.5,black,thin] (0,0) grid (22,-14);

%% Matrices
\node[font=\bfseries, baseline] (I) at (0,0) {$I$};

\matrix [matrix of math nodes,row sep=0.1cm,column sep=0.1cm,
anchor=south west, nodes={anchor=base, baseline}] (is) at
([xshift=1.5cm]I.north east)
{i_1\\i_2\\\vdots{}\\i_n\\};

\matrix [matrix of math nodes,row sep=0.1cm,column sep=0.1cm,
anchor=south west, nodes={baseline}] (is_o) at
([xshift=1.5cm,yshift=.5cm]is-1-1.east) {o_1\\o_2\\\vdots{}\\o_n\\};

\matrix [matrix of math nodes,row sep=0.1cm,column sep=1.5cm,
anchor=south west, nodes={minimum height=5mm}] (is_o_r) at ([xshift=1.5cm,yshift=.5cm]is_o-1-1.east)
{
  \text{RSA}_1&\text{ZLC}_{rsa1}\\
  \text{RSA}_2&\text{ZLC}_{rsa_2}\\
  \vdots{}&\vdots{}\\
  \text{RSA}_n&\text{ZLC}_{rsa_n}\\};

\matrix [matrix of math nodes,row sep=0.1cm,column sep=0.1cm,
anchor=west, nodes={baseline}] (zlb) at ([xshift=1.5cm]is_o_r.east  |- is_o)
{ZLC_{o_1}\\ZLC_{o_2}\\\vdots{}\\ZLC_{o_n}\\};

\matrix [matrix of math nodes,row sep=0.1cm,column sep=0.1cm,
anchor=west, nodes={anchor=base, baseline}] (zlc) at ([xshift=1.5cm]zlb.east  |- is)
{ZLC_1\\ZLC_2\\\vdots{}\\ZLC_n\\};

\node [anchor=west] (zlp) at ([xshift=1.5cm]zlc.east |- I)
{ZLP};


%% Accolades 
\begin{scope}
  \draw[acc3m] (is.north west) -- (is.south west);
  \draw[acc3m] (is_o.north west) -- (is_o.south west);
  \draw[acc3m] (is_o_r.north west) -- (is_o_r.south west);

  \draw[acc] (is-1-1.north east) -- (is-1-1.south east);
  \draw[acc] (is_o-1-1.north east) -- (is_o-1-1.south east);
  \draw[acc] (is_o_r-1-1.north east) -- (is_o_r-1-1.south east);

  \draw[accm] (is_o_r-1-2.north west) -- (is_o_r-1-2.south west);
  \draw[accm] (zlb-1-1.north west) -- (zlb-1-1.south west);
  \draw[accm] (zlc-1-1.north west) -- (zlc-1-1.south west);

  \draw[acc3] (is_o_r.north east) -- (is_o_r.south east);
  \draw[acc3] (zlb.north east) -- (zlb.south east);
  \draw[acc3] (zlc.north east) -- (zlc.south east);
\end{scope}


\begin{scope}
  \draw[->, black] (I.east) -- (is.west) node[pos=0,below] {1} node[pos=1,below]{1..*};

  \draw[->, black] (is-1-1.east) -- (is_o.west) node[pos=0,below] {1} node[pos=1,below]{1..*}; 
  
  \draw[->, black] (is_o-1-1.east) --
  (is_o_r.west) node[pos=0,below]
  {1} node[pos=1,below]{1..*}; 
  
  \draw[->, black] (is_o_r-1-1.east) -- (is_o_r-1-2.west) node[pos=0,below]
  {1} node[pos=1,below]{1}; 
  
  \draw[->, black] (is_o_r.east) -- (zlb-1-1.west) node[pos=0,below] {1..*} node[pos=1,below]{1}; 
  
  \draw[->, black]
  (zlb.east) -- (zlc-1-1.west) node[pos=0,below] {1..*} node[pos=1,below]{1}; 
  
  \draw[->, black] (zlc.east) -- (zlp.west) node[pos=0,below] {1..*} node[pos=1,below]{1}; 
\end{scope}


\begin{scope}
    \draw[->, black, dotted] ([xshift=.5cm]I.east) -- ([xshift=-.5cm]zlp.west) node[pos=0,below]
  {1} node[pos=1,below]{1}; 

  \draw[->, black, dotted] ([xshift=.5cm]is.east) -- ([xshift=-.5cm]zlc.west) node[pos=0,below]
  {1} node[pos=1,below]{1}; 

  \draw[->, black, dotted] ([xshift=.5cm]is_o.east) -- ([xshift=-.5cm]zlb.west) node[pos=0,below]
  {1} node[pos=1,below]{1}; 
\end{scope}

%% XX

\begin{scope}[below of=I]
  \draw[acc3m] (0,0) --++ (8,0) node[pos=.5, yshift=-.5cm] {\emph{Décomposition}};
  \draw[acc3m] (8,0) --++ (3,0) node[pos=.5, yshift=-.5cm] {\emph{Spatialisation}};
  \draw[acc3m] (13.4,0) --++ (2,0) node[pos=.5, yshift=-.5cm] {\emph{Fusion}};
\end{scope}

\end{tikzpicture}
    \caption{Méthodologie}
    \label{fig:methodo_1}
  \end{figure}
\end{landscape}


%%% Local Variables:
%%% mode: latex
%%% TeX-master: "../../../../main"
%%% End:
