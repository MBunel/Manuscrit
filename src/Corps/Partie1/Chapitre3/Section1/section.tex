\tdi{Voir comment citer les 2 naive physics manifestos}


\subsection{Éléments généraux sur la localisation par référencement
  indirect et formalisation d'une relation de localisation}

% Mise au point vocabulaire
Comme nous l'expliquions lors de la présentation du concept
\emph{d'élément de localisation} (cf. \ref{}), ce dernier est
composé de trois éléments :
%
\begin{enumerate*}[label=(\alph*)]
\item \label{i:site} un \emph{sujet,}
\item \label{i:cible} un \emph{objet de référence} et
\item une \emph{relation de localisation,}
\end{enumerate*}
%
qui, combinés, permettent d'exprimer un \emph{référencement indirect.} 
Nous n'avons cependant pas pris le temps de détailler ces trois
concepts, dont la compréhension est nécessaire au développement de
méthodes de \emph{spatialisation.}

\subsubsection{Les concepts de \emph{sujet} et \emph{d'objet de référence}}

Les concepts de \emph{sujet} et \emph{d'objet de référence} sont
présents dans toutes les formalisations des \emph{référencements
  spatiaux indirects.} Tous les auteurs considèrent que ce type de
description nécessite deux objets spatiaux, dont la position de l'un
(le \emph{sujet}) est décrite par rapport à la position de l'autre
(\emph{l'objet de référence}), qui est supposée connue.

Toutefois, le vocabulaire utilisé pour décrire ces concepts ne fait
pas l'objet d'un consensus, comme le montre la synthèse proposée par
\textcite{RetzSchmidt1988}.  Les travaux francophones se rattachant au
domaine de la linguistique, comme ceux de
\textcite{Vandeloise1986,Borillo1998, Aurnague1997, Mathet2000},
parlent de \emph{site} et de \emph{cible}, pour désigner,
respectivement l'\emph{objet de référence} et le \emph{sujet.} La
littérature anglophone parle quant à elle de \emph{primary object} et
de \emph{reference object} ou de \emph{figure} et de \emph{ground}.

\subsubsection{\emph{Relations spatiales} ou \emph{relations de
    localisation ?}}

Le terme de \enquote{relation spatiale}, massivement utilisé dans la
littérature 

\textcite{Duchene2019}, fait la distinction entre \emph{relations de
  localisation} et \emph{relations spatiales.}

Toutefois, ces expressions n'utilisant pas explicitement de
\emph{prépositions spatiales} sont considérés comme des
\emph{relations spatiales} par les linguistes comme
\textcite{Vandeloise1986}, leur sémantique les ramenant à des notions
spatiales.

C'est par exemple le cas de la notion \emph{d'intervisibilité,} dont
l'énoncé, de la forme : \enquote{Je vois un lac}, ne fait pas appel à
une \emph{préposition spatiale.}

% Relations ternaires
Par ailleurs, contrairement à ce que pourraient laisser penser les
différents exemples \emph{d'indices de localisation} présentés
jusqu'ici, ces derniers peuvent contenir plus d'un \emph{objet de
  référence} \autocite{Clementini2013}. Ce cas, relativement peu
fréquent, se présente lorsque la \emph{relation de localisation}
utilisée n'est pas \emph{binaire,} c'est-à-dire qu'elle n'implique pas
qu'un \emph{sujet} et qu'un \emph{objet de référence.} Le cas le plus
fréquent est celui de la \emph{relation de localisation}
\enquote{entre} qui implique, au minimum, deux \emph{objets de
  référence}. Si l'on peut, en effet, dire que \enquote{La poste est
  entre le café et la banque}, il est incorrect d'utiliser cette
\emph{relation de localisation} avec un seul \emph{objet de
  référence.} La phrase \enquote{Je suis entre l'université}, par
exemple, est fautive et incompréhensible. Cependant, touts les
\emph{référencements indirects} liant trois objets, ne sont pas
nécessairement composés d'une \emph{relation de localisation
  ternaire.} \textcite{Duchene2019} identifient deux cas de ce
type. D'une part l'un des objets peut être le support des deux autres,
c'est par exemple lorsqu'une position est décrite sur un itinéraire,
comme dans la phrase \enquote{le parking avant le pont}. De même, des
phrases comme \enquote{Il est à l'angle de la rue Saint-Jacques et le
  la rue Pierre et Marie Curie}. Dans ces deux cas on peut recombiner
l'énoncé en un ensemble \emph{d'indices de localisation} utilisant des
\emph{relations de localisation binaires.} Par exemple,
\enquote{l'angle de la rue Saint-Jacques et de la rue Pierre et Marie
  Curie} peut être considéré comme un seul \emph{objet de référence,}
cette description de position peut s'interpréter comme deux indices de
localisation : \enquote{Je suis à $\phi$ et $\phi$ est à
  l'intersection de la rue Saint-Jacques et de la rue Pierre et Marie
  Curie}. Ces exemples n’intègrent donc pas de \emph{relations de
  localisation ternaires.} À l'inverse, des phrases ne contenant qu'un
seul \emph{objet de référence} peuvent utiliser des \emph{relations de
  localisation ternaires.} On pourra par exemple dire : \enquote{La
  chèvre est entre les arbres}, dans ce cas, seul un \emph{objet de
  référence,} \enquote{les arbres}, est présent, mais celui-ci est
composite, \textcite{Aurnague1993} parlent alors de
\enquote{\emph{collection}}. 


\subsubsection{Contexte d'interprétation et cadre de référence}

\tdi{citer Spatial representation and updating: Evidence from
  neuropsychological investigations}

\tdi{citer The Role of Context in the Interpretation of Natural
  Language Location Descriptions}

Certaines \emph{relations de localisation} ne peuvent être
interprétées (et donc spatialisées) qu'avec une connaissance
supplémentaire sur la configuration de la position décrite. Par
exemple des \emph{relations de localisations} comme \enquote{devant}
ou \enquote{à gauche} ne peuvent être interprétées sans la
connaissance d'un \emph{référentiel de directions,} ou \emph{cadre de
  référence} \autocite{Duchene2019}.

Le \emph{cadre de référence} n'est pas nécessairement explicité dans
un énoncé.

Par exemple, dans la phrase \enquote{il est à ma gauche}, le cadre de
\emph{référence} est explicité. L'article définit \enquote{ma} indique
que le locuteur utilise sa ligne de vue comme référence.

\subsubsection{Modélisation formelle d'une relation}

\paragraph{Modèles morphologiques}



\paragraph{Modèles sémantiques}

Cette seconde catégorie de modèles formalise les \emph{relations de
  localisations} indépendamment de toute considération linguistique
et morphologique.

\textcite{Aurngague1997} proposent

% 1
\autocite{Bateman2010}

\paragraph{\enquote{ISO-Space}, un modèle pivot}

Le modèle normalisé \enquote{ISO-Space}, proposé par
\textcite{Pustejovsky2017}, a pour objectif d'être un pivot entre les
\emph{modèles morphologiques,} destinés à l'annotation et les
\emph{modèles sémantiques,} construits pour interpréter les
\emph{relations de localisation.}

Ce modèle 


\subsection{Classification des \emph{relations de localisation}}

\subsubsection{Critères de classification}

\subsubsection{Classifications trouvées dans la littérature}

interne vs externe \autocite{Borillo1998}

\subsection{\texttt{TITRE}}

\subsubsection{La notion d'ontologie}

La notion \emph{d'ontologie}

\tdi{Gruber (1993), comme la spécification d’une conceptualisation partagée d’un domaine.}
\tdi{(Guarino, 1998) distingue trois niveaux d’ontologies}

\tdi{Présentation de la logique de description et des extensions floues}
\tdi{Présentation OWL}

\subsubsection{Les taxonomies et ontologies de relations de localisations
  existantes}

La définition d'ontologies des relations spatiales

\textcite{Duchene2019} ont identifié 4 ontologies qui se destinent à
la modélisation de \emph{relations de localisation.}

Bien que traitant des mêmes relations spatiales, ces ontologies sont
souvent peu génériques et adaptée qu'au domaine d’application pour
lequel elles ont été initialement conçues \autocite{Hudelot2008a}.

% Bateman
La plus récente, et probablement la plus ambitieuse, d'entre-elles est
l'ontologie GUM-Space (aussi connue sous le nom d'OntoSpace), déjà
présentée \autocite{Bateman2010}.
%
Cette ontologie est l'extension spatiale de l'ontologie GUM
(\emph{Generalized Upper Model}), destinée

% Ontoi 2

% Onto 3


% Hudelot 
Enfin, l'ontologie FRSO (\emph{Fuzzy Spatial Relation Ontology}),
proposée par \textcite{Hudelot2008a}, a pour objectif de permettre
l’interprétation d'images et plus spécifiquement la segmentation
automatique d'images médicales. Les auteurs justifient le
développement d'une ontologie des \emph{relations de localisation} par
l'inadaptation des ontologies déjà existantes pour l'interprétation
d'images.

La particularité principale de l'ontologie FRSO est qu'elle est basée
sur une extension floue de la \emph{logique de description.}

Dans cette ontologies sont distinguées 8 relations directionnelles, 3
relations de diostance




\subsection{Travaux sur différentes familles de relations}


Nous allons à présent détailler les différents travaux proposés pour
modéliser des familles spécifiques de \emph{relations de
  localisation.}



\subsubsection{Les relations topologiques}

Les relations topologiques 

Ces différentes familles de \emph{relations de localisation} se
distinguent également par leur complexité. Comme l'indiquent
\textcite{Aurnague1997}, les travaux de \textcite{Piaget1948} sur le
développement de la perception spatiale chez l'enfant ont mis en
évidence que les premières notions développées étaient celles
relatives à la topologie, suivies par les concepts relevant de la
perception des distances, finalement suivies par la notion
d'orientation.

Comme l'indique \textcite{Duchene2019}, ces relations sont
\enquote{invariantes par déformation continue de l’espace}.


\tdi{citer navive geography sur la topologie avant les distances}

\paragraph{Modèles computationnels}

Dans le domaine des \ac{sig}, deux grandes \enquote{familles} de
modèles topologiques coexistent.

La première est celle du modèle RCC8, proposé par
\textcite{Randell1992} et de ses dérivés.

Le modèle RCC-8 est un modèle formalisant les \emph{relations
  topologiques} entre régions de deux dimensions. Dans ce modèle, les
régions sont modélisées comme des \emph{ensembles de points}


Huit relations topologiques et méréologiques y sont définies. Ces
dernières sont mutuellement exclusives, par conséquent une et une
seule d'entre-elles est vérifiée, quelque soit la configuration.

Ce modèle permet également de raisonner sur les relations
topologiques, à l'aide d'une \emph{table de transition.}

\tdi{citer A More Expressive 3D Region Connection Calculus pour l'EDA
  RCC}
\tdi{Ajouter ler JEPD et si les extensions le valident}

% RCC5
% Bennett, 1994, Spatial reasoning with propositional logics
\textcite{Bennett1994} propose un premier modèle dérivé du travail de
\textcite{Randell1992}, le modèle RCC-5, réduisant, comme son nom
l'indique, le nombre der \emph{relations topologiques} modélisées à
5. Les relations DC et EC sont regroupées en une nouvelle relation, DR
(\enquote{\emph{discrete}}), indiquant que deux régions ne partagent
pas leurs aires. De même, les relations TPP et NTPP (et leurs
inverses) sont combinées en une seule relation (et son inverse), PP,
pour \enquote{\emph{proper parts}}. En réduisant le nombre de
\emph{relations topologiques} modélisées le modèle RCC-5 perd en
expressivité, cependant l'objectif annoncé pour ce modèle n'est pas
d’enrichir la sémantique du modèle RCC-8 (ce qui est le but de la
majorité des extensions), mais de faciliter le raisonnement automatisé
\textcite{Bennett1994}.

\begin{figure}
  \centering
  \usetikzlibrary{calc}
\begin{tikzpicture}

  \begin{scope}[local bounding box=scope1]
    \path[ffa] (0,0) circle [radius=15pt];
    \path[ffa2] (1.25,0) circle [radius=15pt];
    
    \path[ffc] (0,0) circle [radius=15pt] node{A};
    \path[ffc2] (1.25,0) circle [radius=15pt] node{B};
  \end{scope}

  \begin{scope}[local bounding box=scope2, shift={($(scope1.east)+(1cm,0)$)}]
    \path[ffa] (0,0) circle [radius=15pt];
    \path[ffa2] (30pt,0) circle [radius=15pt];
    
    \path[ffc] (0,0) circle [radius=15pt] node{A};
    \path[ffc2] (30pt,0) circle [radius=15pt] node{B};
  \end{scope}

  \begin{scope}[local bounding box=scope3, shift={($(scope2.east)+(1cm,0)$)}]
    \path[ffa] (0,0) circle [radius=15pt];
    \path[ffa2] (0.6,0) circle [radius=15pt];
    
    \path[ffc] (0,0) circle [radius=15pt] node{A};
    \path[ffc2] (0.6,0) circle [radius=15pt] node{B};
  \end{scope}

  \begin{scope}[local bounding box=scope4, shift={($(scope3.east)+(1cm,0)$)}]
    \path[ffa] (0,0) circle [radius=15pt];
    \path[ffa2] (-5pt,0) circle [radius=10pt];
    
    \path[ffc] (0,0) circle [radius=15pt] node{A};
    \path[ffc2] (-5pt,0) circle [radius=10pt] node{B};
  \end{scope}

  \begin{scope}[local bounding box=scope5, shift={($(scope4.east)+(1cm,0)$)}]
    \path[ffa] (0,0) circle [radius=15pt]; node {A}; % Aire
    \path[ffa2] (0,0) circle [radius=10pt];
    
    \path[ffc] (0,0) circle [radius=15pt] node{A};
    \path[ffc2] (0,0) circle [radius=10pt] node{B};
  \end{scope}

  \begin{scope}[local bounding box=scope6, shift={($(scope5.east)+(1cm,0)$)}]
    \path[ffa2] (0,0) circle [radius=15pt];
    \path[ffa] (-5pt,0) circle [radius=10pt];
    
    \path[ffc2] (0,0) circle [radius=16pt] node{B};
    \path[ffc] (-5pt,0) circle [radius=10pt] node{A};
    
  \end{scope}

  \begin{scope}[local bounding box=scope7, shift={($(scope6.east)+(1cm,0)$)}]
    \path[ffa2] (0,0) circle [radius=15pt];
    \path[ffa] (0,0) circle [radius=10pt];

    \path[ffc2] (0,0) circle [radius=15pt] node{B};
    \path[ffc] (0,0) circle [radius=10pt] node{A};
  \end{scope}

  \begin{scope}[local bounding box=scope8, shift={($(scope7.east)+(1cm,0)$)}]
    \path[ffa2] (0,0) circle [radius=15pt];
    \path[ffa] (0,0) circle [radius=15pt];
    
    \path[ffc] (0,0) circle [radius=15pt] node{A};
    \path[ffc2] (0,0) circle [radius=15pt] node{B};
  \end{scope}
\end{tikzpicture}
  \caption{Les relations topologiques des modèles RCC-8 et RCC-5}
  \label{fig:RCC}
\end{figure}

% RCC23 A.G.  Cohn, B.  Bennet, J.  Dooday, and N.M.  Gotts,
% Qualitative Spatial Representation and Reasoning with the Region
% Connection Calculus, GeoInformatica 1(1), pp 1-44, 1997.
Pour améliorer la modélisation des relations topologiques avec des
régions concaves, \textcite{Cohn1997} proposerons une extension à 23
relations du modèle RCC, définissant le modèle RCC-23. La relation
\emph{externally connected} (EC) est décomposée en 9 nouvelles
relations, permettant, par exemple, de faire la distinction entre un
contact avec inclusion dans l'enveloppe convexe et sans. De même, la
relation \emph{disconnected} (DC) est décomposée en 8 nouvelles
relations, ce qui permet de distinguer une situation où un objet
\enquote{enveloppe} l'autre. Un modèle RCC-62 sera également proposé
pour répondre à ce même problème. Contrairement au modèle RCC-23, le
modèle RCC-62 traite les régions en distinguant leur \emph{intérieur}
de leur \emph{frontière} et de leur \emph{extérieur,} ce qui permet de
définir plus de relations topologiques, au prix d'une complexification
conséquente.

\begin{table}
  \centering
  \subfloat[Exemple de décomposition de la relation topologique EC]{
  \begin{tabular}{rC{5.5cm}C{5.5cm}}
    \toprule
    & OUTSIDE\_OUTSIDE\_EC & P\_INSIDE\_INSIDEi\_EC \\
    \midrule
    EC & \tikz{
         \begin{scope}
           \draw[ffa] (-15pt,5pt) -- (10pt,5pt) -- (10pt,0) arc
           (90:270:20pt) --++ (0,-5pt) --++(-25pt,0) -- cycle;
           \draw[ffc] (-15pt,5pt) -- (10pt,5pt) -- (10pt,0) arc
           (90:270:20pt) --++ (0,-5pt) --++(-25pt,0) -- cycle;
         \end{scope}
         \begin{scope}[xshift=20pt,xscale=-1]
           \draw[ffa2] (-15pt,5pt) -- (10pt,5pt) -- (10pt,0) arc
           (90:270:20pt) --++ (0,-5pt) --++(-25pt,0) -- cycle;
           \draw[ffc2] (-15pt,5pt) -- (10pt,5pt) -- (10pt,0) arc
           (90:270:20pt) --++ (0,-5pt) --++(-25pt,0) -- cycle;
         \end{scope}
         }&
            \tikz{
            \begin{scope}
              \draw[ffa] (-15pt,5pt) -- (10pt,5pt) -- (10pt,0) arc
              (90:270:20pt) --++ (0,-5pt) --++(-25pt,0) -- cycle;
              \draw[ffc] (-15pt,5pt) -- (10pt,5pt) -- (10pt,0) arc
              (90:270:20pt) --++ (0,-5pt) --++(-25pt,0) -- cycle;
            \end{scope}
            \begin{scope}[yshift=-20pt]
              \path[ffa2] (0,0) circle [radius=10pt];
              \path[ffc2] (0,0) circle [radius=10pt] node {B};
            \end{scope}
            } \\
    \bottomrule    
  \end{tabular}
  \label{tab:RCC8_vs_RCC23_1}
}

\subfloat[Exemple de décomposition de la relation topologique DC]{
  \begin{tabular}{rC{5.55cm}C{5.5cm}}
    \toprule
    & INTSIDE\_OUTSIDEi\_DC & P\_OUTSIDE\_OUTSIDE\_EC \\
    \midrule
    DC & \tikz{
         \begin{scope}
           \draw[ffa] (-15pt,5pt) -- (10pt,5pt) -- (10pt,0) arc
           (90:270:20pt) --++ (0,-5pt) --++(-25pt,0) -- cycle;
           \draw[ffc] (-15pt,5pt) -- (10pt,5pt) -- (10pt,0) arc
           (90:270:20pt) --++ (0,-5pt) --++(-25pt,0) -- cycle;
         \end{scope}
         \begin{scope}[xshift=2.5pt,yshift=-20pt]
           \path[ffa2] (0,0) circle [radius=10pt];
           \path[ffc2] (0,0) circle [radius=10pt] node {B};
         \end{scope}
         }&
            \tikz{
            \begin{scope}
              \draw[ffa] (-15pt,5pt) -- (10pt,5pt) -- (10pt,0) arc
              (90:270:20pt) --++ (0,-5pt) --++(-25pt,0) -- cycle;
              \draw[ffc] (-15pt,5pt) -- (10pt,5pt) -- (10pt,0) arc
              (90:270:20pt) --++ (0,-5pt) --++(-25pt,0) -- cycle;
            \end{scope}
            \begin{scope}[xshift=30pt,yshift=-20pt]
              \path[ffa2] (0,0) circle [radius=10pt];
              \path[ffc2] (0,0) circle [radius=10pt] node {B};
            \end{scope}
            } \\
    \bottomrule    
  \end{tabular}
  \label{tab:RCC8_vs_RCC23_2}
}


  \caption{Extrait des nouvelles relations topologiques proposées par
le modèle RCC23, d'après XXXXXXX.}
  \label{tab:RCC8_vs_RCC23}
\end{table}

% Exstention 3D
% Extension temporelle : Wolter
Le modèle RCC sera également étendu d'autres manières, notamment pour
prendre en compte la troisième dimension avec les modèles RCC-3D,
VRCC-3D et VRCC-3D+ ou la temporalité avec les travaux de XXXX
couplant le modèle RCC-8 à l'algèbre temporelle d'Allen. Ces
extensions s'éloignent cependant du cadre de ce travail.

% 4IM et dérivés
La seconde \enquote{famille} de modèle des relations topologiques est
celle du \emph{modèle des 4 intersections}
\autocite[4IM,][]{Egenhofer1989} et de ses dérivés, qui, comme pour le
modèle RCC-8, sont nombreux. Ces deux modèles partagent leurs
fondements. Tous deux s’appuient sur le paradigme des \emph{ensembles
  de points} pour modéliser les objets géométriques et définissent les
\emph{relations topologiques} à partir d'opérations ensemblistes sur
ces ensembles de points. De plus, dans les modèles RCC-8
\autocite{Randell1992} et 4IM, les relations topologiques définies
sont exhaustives et mutuellement disjointes (JEPD). Les deux modèles
ont cependant quelques différences, la première est que le modèle 4IM
et ses dérivés ne permettent pas de raisonner à partir des
\emph{relations topologiques,} il n'est donc pas possible d'inférer
une relation à partir de relations topologiques et d'objets
connus. Cependant, le \emph{modèle des intersections} permet de
modéliser des \emph{relations topologiques} entre des régions, des
lignes et des points, contrairement au modèle RCC-8 qui se limite aux
régions.

Dans la première version du \emph{modèle des intersections,} le modèle
4IM, 8 \emph{relations tolopologiques}, équivalentes à celles du
modèle RCC-8 \autocite{Duchene2019}, sont définies
\textcite{Egenhofer1989,Egenhofer1990,Egenhofer1991a}. Celles-ci sont
définies à partir d'une matrice 4 valeurs booléenes, détaillant les
intersections entre les composantes de deux objets géométriques \(a\)
et \(b\). La première d'entre elles est \emph{l'intérieur} de l'objet,
notée \(aᵒ\) ou \(bᵒ\), en fonction de l'objet concerné. La seconde,
notée \(δa\) ou \(δb\), est la \emph{frontière} de l'objet. Ces deux
ensembles, \emph{intérieur} et \emph{frontière,} sont définis pour
chaque type de géométrie, par exemple la frontière d'une ligne est un
ensemble de deux points, le premier et le dernier de la ligne. Pour un
point les deux notions sont équivalentes. La construction de la
\emph{matrice }d'intersection consiste alors en l'étude des
intersections, deux à deux, de la \emph{frontière} et de l'intérieur
de deux objets géométriques :

\begin{equation}
  \label{eq:matrice_4IM}
  \text{4IM}(a,b) =
  \begin{bmatrix}
    aᵒ ∩ bᵒ ≠ ∅ & aᵒ ∩ δb ≠ ∅ \\
    δa ∩ bᵒ ≠ ∅ & δa ∩ δb ≠ ∅ \\
  \end{bmatrix}
\end{equation}

Si l'intersection entre deux ensembles donnés est vide, alors la
valeur inscrite dans la matrice est \enquote{\(F\)}, dans le cas
contraire, la valeur est \enquote{\(V\)}. Par exemple, une
\emph{relation topologique} \emph{d'inclusion} (nomée \emph{overlap,}
dans le modèle 4IM et correspondant au prédicat PO du modèle RCC-8,
cf. figure \ref{fig:RCC}) correspond à la matrice :
%
\(\left[
  \begin{smallmatrix}
    V&V\\
    V&V\\
  \end{smallmatrix}
\right]\),
%
étant donné que les quatres \emph{intersections} définies dans le
modèle 4IM ne sont pas nulles. La \emph{matrice d'intersections} de ce
modèle permet de définir 16 configurations différentes. Cependant la
moitié d'entre-elles ne sont pas réalisables, comme par exemple la
matrice :
%
\(\left[
  \begin{smallmatrix}
    F&V\\
    V&F\\
  \end{smallmatrix}
\right]\),
%
qui décrit une situation où les deux intérieurs et les deux frontières
ne s'intersectent pas, mais où la frontière de chaque objet intersecte
l'intérieur de chaque objet. Comme pour le modèle RCC-8, les 8
configurations possibles sont toutes nomées (\emph{disjoint, contains,
  inside, equal, meet, covers, coveredBy} et \emph{overlap}), ce qui
rend les \emph{relations topologiques} définies plus intelligible et
facile à manipuler que des matrices de valeurs booléenes.

Dans le modèle 9IM, une extension du modèle 4IM proposée par
\textcite{Egenhofer1991}, la \emph{matrice des intersection} est
étendue par l'ajout d'un nouvel ensemble, \emph{l'extérieur} (\ie tous
les points qui n'appartiennent ni à l'intérieur, ni à la frontière de
l'objet), noté \(aᵉ\) (ou \(bᵉ\) en fonction de l'objet). La matrice
initiale est alors étendue en une matrice de 9 valeurs booléennes :

\begin{equation}
  \label{eq:matrice_9IM}
  \text{9IM}(a,b) =
  \begin{bmatrix}
    aᵒ ∩ bᵒ ≠ ∅ & aᵒ ∩ δb ≠ ∅ & aᵒ ∩ bᵉ ≠ ∅ \\
    δa ∩ bᵒ ≠ ∅ & δa ∩ δb ≠ ∅ & δa ∩ bᵉ ≠ ∅ \\
    aᵉ ∩ bᵒ ≠ ∅ & aᵉ ∩ δb ≠ ∅ & aᵉ ∩ bᵉ ≠ ∅ \\
  \end{bmatrix}
\end{equation}

Ce nouveau modèle ne permet pas de modéliser des configurations
nouvelles entre deux régions, mais il permet de distinguer plus
finement les configurations incluant une ligne, comme le montre la
figure \ref{tab:4IM_vs_9IM}.

\begin{table}
  \centering
  \subfloat[Exemple de décomposition de la relation topologique \emph{overlap}]{
  \begin{tabular}{rC{5.5cm}C{5.5cm}}
    \toprule
    & \scriptsize \(\begin{bmatrix}
      V & V & F \\
      V & V & F \\
      V & F & F \\
    \end{bmatrix}\) & \scriptsize  \(\begin{bmatrix}
      V & V & V \\
      V & V & V \\
      V & V & V \\
    \end{bmatrix}\) \\
    \midrule
    \scriptsize \(\text{overlap} ≡ \begin{bmatrix}
      V & V \\
      V & V \\
    \end{bmatrix}\)
    & \tikz{
      \path[ffa] (0,0) circle [radius=15pt];
      \path[ffc] (0,0) circle [radius=15pt] node {A};
      \path[ffc, draw=RdBu-9-9,o-o,shorten >=-3pt,shorten <=-3pt]
      (0,15pt).. controls (0pt, 35pt) and (30pt, 30pt) .. (10pt,0);
      }
    & \tikz{
      \path[ffa] (0,0) circle [radius=15pt];
      \path[ffc] (0,0) circle [radius=15pt] node {A};
      \path[ffc, draw=RdBu-9-9,o-o,shorten >=-3pt,shorten <=-3pt]
      (0,15pt) arc (90:0:15pt) -- (7.5pt,0);
      } \\
    \bottomrule    
  \end{tabular}
  \label{tab:RCC8_vs_RCC23_1}
}

\subfloat[Exemple de décomposition de la relation topologique \emph{meet}]{
  \begin{tabular}{rC{5.5cm}C{5.5cm}}
    \toprule
    &  \scriptsize \(\begin{bmatrix}
      F & F & V \\
      F & V & F \\
      F & V & F \\
    \end{bmatrix}\) & \scriptsize  \(\begin{bmatrix}
      F & F & V \\
      F & V & V \\
      V & F & V \\
    \end{bmatrix}\) \\
    \midrule
    \scriptsize \(\text{meet} ≡ \begin{bmatrix}
      F & F \\
      F & V \\
    \end{bmatrix}\)
    & \tikz{\path[ffa] (0,0) circle [radius=15pt] node {A};}&
                                                              \tikz{\path[ffa] (0,0) circle [radius=15pt] node {B};} \\
    \bottomrule    
  \end{tabular}
  \label{tab:RCC8_vs_RCC23_2}
}


  \caption{Exemple des raffinements de \emph{relations topologiques}
    permis par le modèle 9IM, d'après \textcite{Egenhofer2011}.}
  \label{tab:4IM_vs_9IM}
\end{table}

Le modèle 9IM sera lui-même étendu par le modèle DE-9IM proposé par
\textcite{Clementini1993}. Modèle qui sera par ailleurs normalisé par
l'OGC et par l'ISO, ce qui a conduit à son adoption massive dans les
\ac{sig} \autocite{Strobl2008}. Comme son sigle le laisse supposer, ce
nouveau modèle conserve la matrice à 9 valeurs du modèle 9IM,
cependant les valeurs qui y sont inscrites ne sont plus limitées aux
booléens. En effet, le \emph{dimensions extend nine-intersection
  model} (DE-9IM) ajoute la possibilité de spécifier la
\emph{dimension} des intersections. La matrice d'intersection de deux
objets \(a\) ou \(b\) est alors :

\begin{equation}
  \label{eq:matrice_DE9IM}
  \text{DE-9IM}(a,b) =
  \begin{bmatrix}
    \text{dim}(aᵒ ∩ bᵒ)&\text{dim}(aᵒ ∩ δb)&\text{dim}(aᵒ ∩ bᵉ)\\
    \text{dim}(δa ∩ bᵒ)&\text{dim}(δa ∩ δb)&\text{dim}(δa ∩ bᵉ)\\
    \text{dim}(aᵉ ∩ bᵒ)&\text{dim}(aᵉ ∩ δb)&\text{dim}(aᵉ ∩ bᵉ)\\
  \end{bmatrix}
\end{equation}

Avec \(\text{dim}(x)\), une fonction renvoyant la dimension de
l'intersection, soit 0 pour un point, 1 pour une ligne et 2 pour une
région. Malgrès cette modification substentielle, le modèle DE-9IM
reste compatible avec le modèle 9IM, les matrices numériques du modèle
DE-9IM étant faciles à convertir en matrices binaires, compatibles
avec le modèle 9IM. Comme les autres extensions présentées, cette
extension du modèle 9IM augmente l'expressivité du \emph{modèle des
  intersections} et offre, par exemple, la possibilité de distinguer
un contact ponctuel, d'un contact en une ligne, \ie de distinguer un
voisinage de Von Neuman, d'un voisinage de Moore
(\autoref{tab:9IM_vs_DE9IM}). Cependant, cette nouvelle proposition se
heurte aux mêmes limites que les modèles RCC-23 et RCC-62, à savoir le
grand nombre de cas possibles. Conscient de se problème
\textcite{Clementini1993} complétent le modèle DE-9IM avec le
\emph{calculus based model} (CMB). Ce second modèle propose de définir
cinq \emph{prédicats topologiques} (\emph{in, cross, overlap, touch,
  et disjoint}), correspondant à des configurations particulières de
\emph{matrices d'intersection}, plus faciles à apprender et a
manipuler.

\begin{table}
  \centering
  \begin{tabular}{R{2.5cm}C{4cm}C{4cm}}
  \toprule
  & \scriptsize \(\begin{bmatrix}
    F & F & 2 \\
    F & \mathbf{0} & 1 \\
    2 & 1 & 2 \\
  \end{bmatrix}\) & \scriptsize  \(\begin{bmatrix}
    F & F & 2 \\
    F & \mathbf{1} & 1 \\
    2 & 1 & 2 \\
  \end{bmatrix}\) \\
  \midrule
  \scriptsize \(\begin{bmatrix}
    F & F & V \\
    F & V & V \\
    V & V & V \\
  \end{bmatrix}\)
  & \tikz{
    \path[ffa,, rotate around={45:(0,0)}] (0,0) rectangle ++(1,1);
    \path[ffc,, rotate around={45:(0,0)}] (0,0) rectangle ++(1,1) node[pos=.5,color=RdBu-9-1] {A};
    \path[ffa2, rotate around={45:(1.41,0)}] (1.41,0) rectangle ++ (1,1);
    \path[ffc2, rotate around={45:(1.41,0)}] (1.41,0) rectangle ++ (1,1) node[pos=.5,color=RdBu-9-9] {B};

    }
      & \tikz{
        \path[ffa] (0,0) rectangle (1,1);
        \path[ffc] (0,0) rectangle (1,1) node[pos=.5,color=RdBu-9-1] {A};
        \path[ffa2] (1,0) rectangle (2,1);
        \path[ffc2] (1,0) rectangle (2,1) node[pos=.5,color=RdBu-9-9]
        {B};
        }
  \\
  \bottomrule    
\end{tabular}



  \caption{Exemple des raffinements de \emph{relations topologiques}
    permis par le modèle DE-9IM.}
  \label{tab:9IM_vs_DE9IM}
\end{table}

Comme pour le modèle RCC-8, des extensions tridimensionnelles du
\emph{modèle des intersection} ont été
proposées. \textcite{DelaLosa2000} propose, par exemple, une extension
du modèle 9IM à la troisième dimension. Si le format et le processus
de construction de la \emph{matrice d'intersection} demeurent
identiques, l'ajout du volume comme primitive topologique conduit à
l'apparition de nouvelles relations topologiques qui étaient
irréalisables, comme par exemple, la relation correspondant à la
matrice: 
%
\(\left[
  \begin{smallmatrix}
    F&V&F\\
    F&F&V\\
    V&V&V\\
  \end{smallmatrix}
\right]\),
%
décrivant une situation où la frontière d'une géométrie intersecte
l'intérieur de la seconde géométrie, sans qu'il y ait intersection de
leurs intérieurs, configuration uniquement possible en 3D,
correspondant, par exemple, à l'intersection d'une anse et d'un
panier.

\subsubsection{Les relations de distance}

La notion de \emph{relation de distance} regroupe de nombreux aspects
de  

\tdi{parler de la notion de réseau support
  \url{https://www.cairn.info/revue-flux-2016-3-page-33.htm?contenu=bibliographie}
  pailhous sur le fait que les plus proches chemins ne sont pas
  nécessairement une bonne approximation}

Les relations de distance permettent de décrire l'éloignement entre le
\emph{sujet} et un \emph{objet de référence.} Comme les relations
directionnelles elles peuvent être exprimées de manière quantitative
(\eg \enquote{Je suis à 50 mètres}) ou qualitatives (\eg \enquote{Je
  suis proche}). Ce

\tdi{Geographic Space and Time are Tightly Coupled}
\tdi{Distances are Asymmetric}
\tdi{Distance Inferences are Local, Not Global}
\tdi{Distances Don’t Add Up Easily}

\enquote{La distance \textelp{} semble \textelp{} numérique par
  essence. Toutefois \textelp{} la notion sous-jacente essentielle est
  en fait un ordre \textelp{}}, \textcite{Aurnague1997}.

\tdi{deux aspects: réseau et distance}

Types de distance, euclidienne, manhattan, réseaux Algo réseau,
Dijkstra, A*, driving distance, hausdorf

\tdi{parler recalcul réseau Bunel2016, Berli2017 ?}

\paragraph{Distances métriques quantitatives}

L'expression de distances sous leur forme quantitative, correspond à
la manière la plus classique d'exprimer une relation de distance.

On peut supposer, qu'une fois la

Cependant, l'expression de distances sous une forme quantitative
n'implique pas nécessairement précision. 

Toutefois, l'utilisation de \emph{relations de distance quantitatives}
n'implique pas nécessairement une plus grande précision. Les distances

\tdi{attraction des nombres ronds : \url{https://www.persee.fr/doc/rfsoc_0035-2969_1961_num_2_3_5938}}

Le phénomène \emph{d'attraction des nombres ronds}

La mesure d'une distance entre deux objets 



\paragraph{Distances métriques qualitatives}

\tdi{Robinson, V.: Interactive machine acquisition of a fuzzy spatial
  relation. Comput.  Geosci. 16, 857–872 (1990). Robinson, V.:
  Individual and multipersonal fuzzy spatial relations acquired using
  human-machine interaction. Fuzzy Sets Syst. 113(1), 133–145 (2000)}

\tdi{Spatial Natural Language Generation for Location Description in
  Photo Captions Mark M. Hall 1 , Christopher B. Jones 2( B ) , and
  Philip Smart 2}

\paragraph{Distances-temps}

\tdi{Borillo pas de différence fondamentatle entre distance et temps
  de marche, voir aussi Naive geography}

\tdi{Bonne ref An analysis of probability of area techniques for
  missing persons in Yosemite National Park}

\tdi{équation fausses, vérifier}

PLa description d'une \emph{relation de localisation}

L'évaluation des zones atteignables en une durée donnée, à partir
d'une position donnée, nécessite de disposer d'un \emph{modèle de
  vitesse,} permettant d'estimer la vitesse de déplacement en fonction
de la nature du terrain, ou du moyen de déplacement.

\tdi{Dire qu'il n'est pas nécessaire d'utiliser des modèles avancés,
  par exemple en SMA; MAGS Project:
Multi-agent GeoSimulation and Crowd Simulation}

Certaines utilisation ne nécessitern

% Naismith
Le plus ancien d'entre eux est le modèle de
\textcite{Naismith1892}. Comme l'indique \textcite{Duchene2019}, ce
modèle ne consiste qu'en une phrase, concluant la présentation d'une
sortie dans un journal d'alpinisme. Cette règle énonce qu'une bonne
estimation du temps de marche consisterait à compter une heure pour 3
miles (\SI{5}{\kilo\meter}) en distance planaire et d'y ajouter une
heure de marche par tranche de \SI{2000}{ft} (\SI{600}{\meter}) de
dénivelé positif \autocite{Naismith1892}. Avec ce modèle, la vitesse
de marche (\(V\)) peut s'exprimer en fonction de la pente (\(S\)), de
la manière suivante:

\begin{equation}
  \label{eq:marche_naismith}
  V = \dfrac{5 × 0,6}{5 + 0,6 × S}
\end{equation}

Avec \(S\) la pente, dont l'expression est:

\begin{equation}
 S = \dfrac{Δh}{Δx} = \tan θ
\end{equation}

Où \(θ\) est l'angle de la pente, \(Δh\) le dénivelé et \(Δx\) la
distance planimétrique. Cette formule peut être généralisée pour
s'adapter à différentes valeurs de vitesses planimétriques (\(Vₓ\)) et
ascensionnelles (\(Vₕ\), en \si{\kilo\meter\per\hour}):

\begin{equation}
  \label{eq:marche_naismith_f}
  V = \dfrac{Vₓ × Vₕ}{Vₓ + Vₕ × S}
\end{equation}

Une première correction de ce modèle sera proposée par
\textcite{Aitken1977}, qui suggère d'utiliser la règle définie par
\autocite{Naismith1892} pour estimer la vitesse de marche sur sentiers
et de réduire la vitesse planimétrique (\(Vₓ\)) à
\SI{4}{\kilo\meter\per\hour} pour estimer une vitesse de marche
hors-sentier.

% Langmuir
Les modèles d'\textcite{Naismith1892,Aitken1977} ne permettent,
cependant, pas d'estimer la vitesse de marche en descente. Cet ajout
sera du fait de \textcite{Langmuir1984}, qui propose d'étendre le
modèle de \textcite{Naismith1892} en :
%
\begin{enumerate*}[label=(\alph*)]
\item \emph{réduisant} le temps de marche estimé de 10 minutes, tous les
300 mètres de dénivelé négatif, si la pente est \emph{faible} (\ie
comprise entre -5 et -12 degrés) et
\item en \emph{augmentant} le temps de marche estimé de 10 minutes,
  tous les 300 mètres de dénivelé négatif, si la pente est
  \emph{forte} (\ie inférieure à -12 degrés).
\end{enumerate*}
%
Dans la quatrième édition de son ouvrage de 1984,
\textcite{Langmuir2013} proposera un second modèle, plus adapté à
l'estimation de la vitesse de marche d'un groupe. La vitesse de marche
est réduite à \SI{4}{\kilo\meter\per\hour} et les montées sont plus
pénalisées, une heure de marche est ajoutée par tranche de
\SI{450}{\meter} de dénivelé positif. Ces deux versions corrigées du
modèle de \autocite{Naismith1892} par \textcite{Langmuir1984,
  Langmuir2013} peuvent s'exprimer de la manière suivante :

\begin{equation}
  \label{eq:marche_langumir}
  \def\arraystretch{1.75}
  V = \left\{
    \begin{array}{ll}
      \dfrac{Vₓ × V_{h1}}{Vₓ + V_{h1} × S} & \text{si}\ θ > -5° \\
      \dfrac{Vₓ × V_{h2}}{Vₓ - V_{h2} × S} & \text{si}\ -12° ≤ θ ≤ -5° \\
      \dfrac{Vₓ × V_{h2}}{Vₓ + V_{h2} × S} & \text{si}\ θ < -12°\\
    \end{array}
  \right.
\end{equation}

Avec \(V_{h1}\) la vitesse ascensionnelle (\SI{600}{\meter\per\hour}
dans \textcite{Langmuir1984} et \SI{450}{\meter\per\hour} dans
\textcite{Langmuir2013}), \(V_{h2}\) ascensionnelle en descente. Avec
ces corrections il devient possible d'évaluer le temps de marche
nécessaire pour effectuer un parcours complet, descente
comprise.
% utilisé par R.cost
Cependant le modèle de \bsc{Langmuir-Naismith} à un défaut majeur, son
comportement pour les faibles pentes
(cf. \autoref{fig:modeles_marche}). En effet les vitesses de marche en
descente estimées sont très importantes et en fort décalage avec la
tendance observée pour les courbes définies pour les montées et les
grandes descentes. Cet effet est particulièrement marqué dans la
formulation originale du modèle, où les vitesses estimées peuvent
approcher les \SI{12}{\kilo\meter\per\hour}, une vitesse de marche
très élevée, même dans des conditions optimales
\autocite{Kerouanton2020}. Les vitesses de marche estimées par la
seconde version du modèle \autocite{Langmuir2013} semblent plus
réalistes, la vitesse maximale étant d'environ
\SI{7}{\kilo\meter\per\hour}. Cependant, la rupture entre l'estimation
pour les fortes descentes et celle pour les faibles descentes
subsiste, puisque si la vitesse estimée pour une pente de 12 degrés
est d'environ \SI{7}{\kilo\meter\per\hour}, celle pour une pente de 13
degrés est d'environ \SI{2,5}{\kilo\meter\per\hour}. Malgré ses
limites, le modèle \bsc{Langmuir-Naismith} propose une caractéristique
intéressante, les vitesses de marches maximales ne sont pas atteintes
sur le plat, mais dans les descentes, suffisamment pentues pour
soulager le randonneur, mais pas assez prononcées pour être pénibles.

% Tobler
Cette caractéristique est partagée avec le modèle de
\textcite{Tobler1993}, où la vitesse de marche est exprimée en
fonction de la pente seule. Bien que présenté sous une forme
analytique, ce modèle se base sur des données empiriques provenant de
\textcite{Imhof1950} \autocite{Tobler1993}.

\begin{equation}
  \label{eq:marche_tobler}
  V = g × 6 × e^{-3,5 × \left| S + 0,05 \right|}
\end{equation}

Quant à \(g\), il s'agit d'un coefficient de
pondération, permettant de faire varier l'estimation de la vitesse en
fonction d'éléments exogènes, comme le mode de
déplacement. \textcite{Tobler1993} en propose 3 valeurs :
%
\begin{enumerate*}[label=(\alph*)]
\item \(g = 1\), le cas standard, pour estimer un temps de marche sur
  sentier,
\item \(g = 0,6\), lorsque la marche est faite hors-sentier et
\item  \(g = 1.25\), pour un déplacement à cheval.
\end{enumerate*}

% Modèle de Ress
\textcite{Rees2004} a proposé d'estimer la vitesse de marche en
fonction de la pente à l'aide d'une fonction quadratique :
\begin{equation}
  V = \dfrac{1}{a + b × S + c × S²}
\end{equation}

Avec \(S\) la pente. Les valeurs des coefficients \(a\), \(b\) et
\(c\), respectivement \SIlist{0,75;0,09;14,6}{\second\per\meter}, ont
été estimés par \textcite{Rees2004} à l'aide

% Modèle Irmisher
\textcite{Irmischer2017,Kerouanton2020} proposent, dans la lignée de
\textcite{Tobler1993}, deux modèles de marche construits à partir de
données empiriques.

% Modèle Jobe White

% Sentiers ou terrain ?

\begin{figure}
  \centering \pgfmathdeclarefunction{naismith}{1}{%
  \pgfmathparse{1 / ((1/#1) + tan(x) / 0.6)}%
}

\pgfmathdeclarefunction{langmuir1}{1}{%
  \pgfmathparse{(1 / ((1/#1) + abs(tan(x)) * (0.16 /
    0.3)))}%
}

\pgfmathdeclarefunction{langmuir2}{1}{%
  \pgfmathparse{(1 / ((1/#1) - abs(tan(x)) * (0.16 /
    0.3)))}%
}

\pgfmathdeclarefunction{tobler}{1}{%
  \pgfmathparse{#1 * 6 * exp(-3.5 * abs(tan(x) + 0.05))}%
}

\pgfmathdeclarefunction{ress}{0}{%
  \pgfmathparse{6*exp(-3.5*abs(x*0.05))}%
}

\begin{tikzpicture}
  \begin{axis}[
    width=12cm,
    height=7cm,
    grid=major,
    xlabel=Pente,
    ylabel=Vitesse de marche (\si{\kilo\meter\per\hour}),
    every axis plot post/.append style={
      mark=none,domain=-60:60,samples=500,smooth},
    ymax=12,
    axis x line=bottom,
    axis y line*=left,
    enlargelimits=upper,
    legend style={
      at={(0.5,-0.25)},
      anchor=north,
      legend columns=3,
      font=\small,
      draw=none},
    ]

    %\addplot[restrict x to domain={0:60}, thick] {naismith(5)};
    %\addlegendentry{\textcite{Naismith1892}}
    
    \addplot[restrict x to domain={0:60}, thick] {naismith(4)};
    %\addlegendentry{\textcite{Naismith1892}}

    \addplot[restrict x to domain={-60:-12}, thick] {langmuir1(5)};
    \addplot[restrict x to domain={-12:-5}, thick] {langmuir2(5)};
    \addplot[restrict x to domain={-5:0}, thick] {5};
    \addplot[restrict x to domain={0:60}, thick] {naismith(5)};

    %\draw[black] (axis cs:-12,2) -- (axis cs:-12,12);
    %\draw[black] (axis cs:-5,7) -- (axis cs:-5,5);
    
    \addplot[thick] {tobler(1)};
    %\addlegendentry{\textcite{Tobler1993}, \(g=1\)}
    
    \addplot[thick] {tobler(0.6)};
    %\addlegendentry{\textcite{Tobler1993}, \(g=0,6\)}
    
    %\addplot {ress};

  \end{axis}
\end{tikzpicture}
  \caption{Vitesse de marche estimée en fonction de la pente par les
    différents modèles de marche proposés dans la littérature.}
  \label{fig:modeles_marche}
\end{figure}


\subsubsection{Les relations projectives}

\enquote{\textelp{} qui nécessitent recours à un espace projectif
  \textelp{}} \autocite[p. 18]{Duchene2019}

\paragraph{Les relations XXXXe}

\paragraph{Les relations directionnelles}


\textcite{Frank,} recense deux catégories, les modèles à secteur (ou
\emph{cone-based}), dérivés du travail de \textcite{Peuquet1987} et
les modèles utilisant des demi-plans, dont la première utilisation a
été proposée par XXXXXX.


% Cone based



% Demi plan

%
Une autre catégorie de modèle 

% Allen based


% Modélisation floues

Enfin, comme pour les modèles topologiques, des extensions
tridimensionnelles de ces modèles ont été proposées.



\paragraph{Les relations d'orientation relative}

% Définition
Les relations d'orientation relatives sont définies par
\textcite{Duchene2019} comme des relations
\enquote{concern\textins{ant} l'écart angulaire entre deux objets
  ayant une direction privilégiée d'allongement et pouvant à ce titre
  être assimilés à des segments \textelp{}}.

% Exemple

% Travaux





\subsubsection{Les relations de visibilité}

Comme pour les \enquote{durées de déplacement}, détaillées ci-dessus,
les \enquote{relations de visibilité} ne ne s'expriment pas à l'aide
d'une préposition spatiale, comme \enquote{sur} ou \enquote{dans}, il
s'agit donc d'une \emph{relation de localisation,} mais pas d'une
\emph{relation spatiale.}

La question de la visibilité est importante dans le domaine du SIG



\tdi{citer \url{https://journals.openedition.org/cybergeo/27862}}



%%% Local Variables:
%%% mode: latex
%%% TeX-master: "../../../../main"
%%% End:
