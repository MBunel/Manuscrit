Dans le \autoref{chap:2} nous identifions deux étapes centrales de
notre méthode permettant de construire une \emph{zone de localisation
  probable} correspondant à \emph{une description de position,} la
\emph{spatialisation} et \emph{la fusion.} Cette présentation n'est
cependant satisfaisante qu'en première approximation. En effet, les
principes introduits dans ce chapitre (et plus particulièrement le
principe de décomposition) complexifient la méthodologie en y ajoutant
de nouvelles étapes, comme différentes phases de décomposition et de
fusion. Ainsi nos précédentes définitions (\autoref{chap:02}) de la
\emph{spatialisation} et de la \emph{fusion} doivent êtres complétées,
ce qui sera fait au fur et à mesure de cette partie.

Dans cette partie nous allons présenter une première formalisation de
notre méthodologie. L'objectif n'est cependant pas de présenter tous
les aspects des étapes de \emph{décomposition,} de
\emph{spatialisation} et de \emph{fusion} chaque étape de la méthode
étant détaillée dans un chapitre dédié (respectivement les chapitres
\ref{chap:05}, \ref{chap:07} et \ref{chap:08}). Nous ne parlerons donc
pas de certains aspects avancés de la modélisation, comme la prise en
compte de l'imprécision (cf. \autoref{chap:07}) ou de l'incertitude
(cf.). La méthode ici présentée est la plus générique possible  

La \autoref{fig:methodo_1} présente une synthèse visuelle de la
méthodologie que nous avons définie.

\subsection{La phase de décomposition}

La \emph{décomposition} est la première phase de notre méthode. C'est
un ensemble de trois étapes permettant de passer d'un ensemble
\emph{d'indices de localisation} à un ensemble décomposé de ces
indices, prêt à être \emph{spatialisé.} Seule une de ces étapes (la
dernière) découle du principe de \emph{décomposition} des \emph{relations de localisation.}


La première étape de cette phase est celle de la décomposition de
\emph{l'ensemble des indices de localisation}, \(I\)
(\autoref{fig:methodo_1}). Cet ensemble contient tous les
\emph{indices de localisation} qui nous sont transmis par les
secouristes, par le biais de l'interface de géovisualisation. Comme
l'indique la \autoref{fig:diag_acti_secours}, c'est la réception de
cet ensemble qui déclenche le processus de construction de la
\emph{zone de localisation probable.} La décomposition de l'ensemble
\(I\) permet d'extraire les différents \emph{indices de localisation}
(\(i_n\) sur la \autoref{fig:methodo_1}), à partir desquels serons
construites les\emph{ zones de localisation compatibles.} Cette
transformation était qualifée de \emph{spatialisation} dans le
\autoref{chap:02}, cependant, comme le montre la
\autoref{fig:methodo_1}, ce n'est pas à cette étape qu'apparaissent
les objets géographiques, sous la forme des \emph{zones de
  localisation compatibles,} mais après deux nouvelles étapes de
\emph{décomposition.} C'est pourquoi nous préférons réserver l'usage
de ce terme à l'étape centrale de la méthode, la création d'objets
géographiques. L'étape de décomposition de \emph{l'ensemble des
  indices de décomposition,} ne présente pas de difficultés techniques
ou d'apport scientifique, mais elle permet la \emph{spatialisation
  autonome des indices de localisation.} Comme on peut le voir sur la
\autoref{fig:methodo_1}, les indices décomposés lors de la première
étape du processus sont fusionnés au dernier moment, conformément au
principe de modélisation autonome.

La seconde étape de la phase de décomposition aborde un point que
nous n'avons pas encore présenté.
%
les \emph{indices de localisation} contiennent toujours un (ou
plusieurs, si la relation de localisation est n-aire) \emph{objet de
  référence.} Cependant ce dernier peut être une instance clairement
définie, que l'on peut alors facilement identifier (\eg
\enquote{Grenoble}, \enquote{Le bois des Ayes}, \emph{etc.}) ou un
objet décrit par sa nature, qu'il est alors impossible d'identifier
sans informations supplémentaires (\eg \enquote{une ville},
\enquote{un bois}). Dans ce second cas



Ce principe sera détaillé dans le \autoref{chap:05}

\subsubsection{objets de ref}

\subsubsection{La décomposition des relations spatiales}

\subsection{Spatialisation des \emph{indices de localisation}}

\subsubsection{La décomposition des relations de localisation}

\subsubsection{La spatialisation des relations de localisation atomiques}


Faite pour chaque indice (chap2)
Nécessite des méthodes spécifiques (eda)
Plusieurs objets de référence

Ce principe sera détaillé dans le \autoref{chap:07}

Pour rappel, l'opération de \emph{spatialisation} consiste à
construire une zone (dite de \emph{localisation compatible})
correspondant à \emph{l'indice de localisation}
(\autoref{fig:obj_spa}). Ainsi on peut définir une \emph{zone de
localisation compatible} (pour un \emph{indice de localisation} donné)
comme l'ensemble des positions où \emph{l'indice de localisation} est
vrai, soit :


\begin{equation}
  \text{\textsf{ZLC}}_i = \{(x,y) \ |\ \text{\textsf{RL}}_i(x,y) \wedge (x,y) ∈ \mathbb{R}^2\}
\end{equation}

Avec \textsf{ZLC} la \emph{zone de localisation compatible,} \(i\) un
\emph{indice de localisation,} \((x,y)\) un couple de réels formant
une position et \(\text{\textsf{RL}}_i(x,y)\) une fonction vraie si la
position \((x,y)\) est dans la \emph{zone de localisation compatible}
correspondant à \emph{l'indice de localisation} \(i\).

\subsection{Fusion}

De toutes les étapes de notre méthodologie, la phase de \emph{fusion}
est, de loin, la plus simple. Son objectif est de combiner les
différentes \emph{zones de localisation compatibles} créées lors de la
phase de \emph{spatialisation} en une seule, la \emph{zone de
  localisation probable} (ZLP).

Ce principe sera détaillé dans le \autoref{chap:08}


\begin{equation}
  \text{\textsf{ZLP}}_I = ⋂_{i \in I} \text{\textsf{ZLC}}ᵢ
\end{equation}

Avec \textsf{ZLP} la \emph{zone de localisation probable,} \(I\),
l'ensemble des \emph{indices de localisation} pris en compte, \(i\) un
\emph{indice de localisation} et \textsf{ZLP} un \emph{zone de
  localisation compatible.}

\begin{landscape}
  \begin{figure}[H]
    \centering
    \begin{tikzpicture}
  \usetikzlibrary{matrix, fit}
  \usetikzlibrary{decorations.pathreplacing, calc}

  \tikzset{
    acc/.style={decorate,decoration={brace,raise=-.2,amplitude=0cm}},
    accm/.style={acc,decoration={mirror}},
    acc3/.style={acc,decoration={amplitude=0cm}},
    acc3m/.style={acc3,decoration={mirror}},
    cardinalite/.style={font=\footnotesize\ttfamily}
  }


  \newcommand\irregularcircle[2]{% radius, irregularity
  let \n1 = {(#1)+rand*(#2)} in
  +(0:\n1)
  \foreach \a in {20,40,...,350}{
    let \n1 = {(#1)+rand*(#2)} in
    -- +(\a:\n1)
  } -- cycle
}

  
  %\draw[step=.5,black,thin] (0,0) grid (22,-14);

%% Matrices
\node[font=\bfseries, baseline] (I) at (0,0) {$I$};

\matrix [matrix of math nodes,
anchor=south west, nodes={anchor=base, baseline}] (is) at
([xshift=1cm,yshift=1.5cm]I.north east)
{i_1&i_2&\cdots{}&i_n\\};

\matrix [matrix of math nodes,
anchor=south west, nodes={baseline}] (is_o) at
([xshift=1cm,yshift=1.5cm]is-1-1.north east) {o_1&o_2&\cdots{}&o_n\\};

\matrix [matrix of math nodes,
anchor=south west, nodes={minimum height=.75cm, font=\footnotesize}] (is_o_r) at
([xshift=1cm,yshift=1.5cm]is_o-1-1.north east)
{\text{RSA}_1&\cdots{}&\text{RSA}_n\\};

\matrix [matrix of nodes, column sep=.25cm,
anchor=west, nodes={anchor=center, minimum height=.75cm}] (zla) at
([xshift=1cm]is_o_r.east)
{
  \draw[ffa,ffc] (0, 0) circle (.2cm);&
  \draw[ffa,ffc] (0, 0) circle (.2cm);&
  \node{$\cdots{}$};&
  \draw[ffa,ffc] (0, 0) circle (.2cm);\\
};

% \text{ZLC}_{rsa1}\\ 
% \text{ZLC}_{rsa_2}\\
% ts{}\\              
% \text{ZLC}_{rsa_n}\\

\matrix [matrix of math nodes,
anchor=west, nodes={baseline}] (zlb) at ([xshift=1.5cm]is_o_r.east  |- is_o)
{\text{\textsf{ZLC}}_{o_1}&ZLC_{o_2}&\cdots{}&ZLC_{o_n}\\};

\matrix [matrix of math nodes,
anchor=west, nodes={anchor=base, baseline}] (zlc) at ([xshift=1.5cm]zlb.east  |- is)
{ZLC_1&ZLC_2&\cdots{}&ZLC_n\\};

\node [anchor=west] (zlp) at ([xshift=1.5cm]zlc.east |- I)
{ZLP};

%% Accolades
\begin{scope}
  \draw (I.north west) |- ($(I.north west)!0.5!(I.north east) + (0,.1)$) -| (I.north east) node[pos=0, yshift=.2]
  (I-p) {};

  \draw (zlp.north west) |- ($(zlp.north west)!0.5!(zlp.north east) + (0,.1)$) -| (zlp.north east) node[pos=0, yshift=.2]
  (zlp-p) {};
  
\foreach \m in {is,is_o,is_o_r,zla,zlb,zlc} {
  \draw (\m.south west) |- ($(\m.south west)!0.5!(\m.south
  east) + (0,-.1)$) -| (\m.south east)  node[pos=0, yshift=.2]
  (\m-g) {};
  \draw (\m-1-1.north west) |- ($(\m-1-1.north west)!0.5!(\m-1-1.north east) + (0,.1)$) -| (\m-1-1.north east) node[pos=0, yshift=.2]
  (\m-p) {};
}

\end{scope}

% %% Accolades 
% \begin{scope}
%   \draw (is.south west) |- ($(is.south west)!0.5!(is.south
%   east) + (0,-.1)$) -| (is.south east);

%   \draw[acc] (is-1-1.north west) -- (is-1-1.north east) node[midway, yshift=.2]
%   (dsq) {};
  
%   \draw[acc3m] (is_o.south west) -- (is_o.south east) node[midway] (dsqd) {};


%   \draw[acc] (is_o-1-1.north west) -- (is_o-1-1.north east);

%   \draw[acc3m] (is_o_r.south west) -- (is_o_r.south east);  
%   \draw[acc] (is_o_r-1-1.north west) -- (is_o_r-1-1.north east) node[midway, yshift=.2]
%   (temp1) {};;
  
%   \draw[acc3m] (temp.south west) -- (temp.south east);  
%   \draw[acc] (temp-1-1.north west) -- (temp-1-1.north east) node[midway, yshift=.2]
%   (temp2) {};;
  
  
%   \draw[acc] (zlb-1-1.north west) -- (zlb-1-1.north east);
%   \draw[acc3m] (zlb.south west) -- (zlb.south east);
  
%   \draw[acc] (zlc-1-1.north west) -- (zlc-1-1.north east);
%   \draw[acc3m] (zlc.south west) -- (zlc.south east);


% \end{scope}

\begin{scope}
  % Décomposition
  \draw[-<<, black,] (I-p) |- ($(I-p)!0.5!(is-g)$) node[pos=0,above
  right, cardinalite] {1} node[pos=1, fill=white] {\tiny
    \emph{Décomposition} 1} -| (is-g) 
    node[pos=1,below right,cardinalite]{1..*};

  \draw[-<<, black,] (is-p) |- ($(is-p)!0.5!(is_o-g)$) node[pos=0,above
  right, cardinalite] {1} node[pos=1, fill=white] {\tiny
    \emph{Décomposition} 2} -| (is_o-g) 
  node[pos=1,below right,cardinalite]{1..*};

  \draw[-<<, black,] (is_o-p) |- ($(is_o-p)!0.5!(is_o_r-g)$) node[pos=0,above
  right, cardinalite] {1} node[pos=1, fill=white] {\tiny
    \emph{Décomposition} 3} -| (is_o_r-g) 
  node[pos=1,below right,cardinalite]{1..*};

  % Spatialisation
    \draw[->, black,] (is_o_r-p) |- ($(is_o_r-p)!0.5!(zla-p) + (0,.75)$) node[pos=0,above
  right, cardinalite] {1} node[pos=1, fill=white] {\tiny
    \emph{Spatialisation}} -| (zla-p) 
  node[pos=1,above right,cardinalite]{1};


  % Fusion
  \draw[->>, black,] (zla-g) |- ($(zla-g)!0.5!(zlb-p)$) node[pos=0,below
  right, cardinalite] {1..*} node[pos=1, fill=white] {\tiny
    \emph{Fusion} 1} -| (zlb-p) 
  node[pos=1,above right,cardinalite]{1};
  
  \draw[->>, black,] (zlb-g) |- ($(zlb-g)!0.5!(zlc-p)$) node[pos=0,below
  right, cardinalite] {1..*} node[pos=1, fill=white] {\tiny
    \emph{Fusion} 2} -| (zlc-p) 
  node[pos=1,above right,cardinalite]{1};

    \draw[->>, black,] (zlc-g) |- ($(zlc-g)!0.5!(zlp-p)$) node[pos=0,below
  right, cardinalite] {1..*} node[pos=1, fill=white] {\tiny
    \emph{Fusion} 3} -| (zlp-p) 
  node[pos=1,above right,cardinalite]{1};  
\end{scope}


% \begin{scope}
%   \draw[->, black] (I.east) -- (is.west) node[pos=0,below] {1} node[pos=1,below]{1..*};

%   \draw[->, black] (is-1-1.east) -- (is_o.west) node[pos=0,below] {1} node[pos=1,below]{1..*}; 
  
%   \draw[->, black] (is_o-1-1.east) --
%   (is_o_r.west) node[pos=0,below]
%   {1} node[pos=1,below]{1..*}; 
  
%   \draw[->, black] (is_o_r-1-1.east) -- (is_o_r-1-2.west) node[pos=0,below]
%   {1} node[pos=1,below]{1}; 
  
%   \draw[->, black] (is_o_r.east) -- (zlb-1-1.west) node[pos=0,below] {1..*} node[pos=1,below]{1}; 
  
%   \draw[->, black]
%   (zlb.east) -- (zlc-1-1.west) node[pos=0,below] {1..*} node[pos=1,below]{1}; 
  
%   \draw[->, black] (zlc.east) -- (zlp.west) node[pos=0,below] {1..*} node[pos=1,below]{1}; 
% \end{scope}


\begin{scope}
  \draw[->, black, dotted] ([xshift=.5cm]I.east) -- ([xshift=-.5cm]zlp.west) node[pos=0,below,cardinalite]
  {1} node[pos=1,below,cardinalite]{1}; 

  \draw[->, black, dotted] ([xshift=.5cm]is.east) -- ([xshift=-.5cm]zlc.west) node[pos=0,below,cardinalite]
  {1} node[pos=1,below,cardinalite]{1}; 

  \draw[->, black, dotted] ([xshift=.5cm]is_o.east) -- ([xshift=-.5cm]zlb.west) node[pos=0,below,cardinalite]
  {1} node[pos=1,below,cardinalite]{1}; 
\end{scope}

%% XX

% \begin{scope}[below of=I]
%   \draw[acc3m] (0,0) --++ (8,0) node[pos=.5, yshift=-.5cm] {\emph{Décomposition}};
%   \draw[acc3m] (8,0) --++ (3,0) node[pos=.5, yshift=-.5cm] {\emph{Spatialisation}};
%   \draw[acc3m] (13.4,0) --++ (2,0) node[pos=.5, yshift=-.5cm] {\emph{Fusion}};
% \end{scope}

  % \path[ffa] (0,0) rectangle ++(2,-12);

  % \path[ffa] (3.5,0) rectangle ++(2,-5);
  % \path[ffa] (3.5,-5.5) rectangle ++(2,-5);
  % \node[font=\bfseries] at (4.5, -11.5) {$\vdots$};

  % \path[ffa] (5,-1) rectangle ++(2,-2);
  % \path[ffa] (5,-4.5) rectangle ++(2,-2);
  % \path[ffa] (5,-8) rectangle ++(2,-2);

  % %seg
  % \path[ffa] (7.5,0) rectangle ++(2,-1.25);
  % \path[ffa] (7.5,-1.75) rectangle ++(2,-1.25);
  

  
  % \path[ffa] (7.5,-3.5) rectangle ++(2,-3);
  % \path[ffa] (7.5,-7) rectangle ++(2,-3);

  % \path[ffa] (10,0) rectangle ++(2,-3);
  % \path[ffa] (10,-3.5) rectangle ++(2,-3);
  % \path[ffa] (10,-7) rectangle ++(2,-3);
  

  %\path[ffa] (20,0) rectangle ++(2,-12);


\end{tikzpicture}
    \caption{Méthodologie}
    \label{fig:methodo_1}
  \end{figure}
\end{landscape}


%%% Local Variables:
%%% mode: latex
%%% TeX-master: "../../../../main"
%%% End:
