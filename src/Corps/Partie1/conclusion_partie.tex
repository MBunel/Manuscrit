Comme nous l'avons montré dans le premier chapitre de cette thèse, le
secours en montagne est une activité essentielle dans la pratique des
sports de montagne et bien que les opérations de secours soit
efficaces et ont permi réduire considérablement la mortalié dans les
sports en mointagne tout en accompagnant l'explosion des pratiques
touristiques en montagne, il s'agit d'opérations critiques, mobilisant
un personel peu nomnbreux et disposant de moyens, certes important,
mais pas illimités. Si les opérations de secours sont maitrisées grâce
à l'expertise technique et l'expérience des secouristes,
%
la première partie des opérations de secours, \ie la phase de
localisation de la victime est une étape critique. En effet, si dans
la majorité des cas les secouristes n'éprouvent que peu de
difficultées à localiser les victimes et encore moins depuis la
généralisation des outils de géolocalisation (\eg GendLoc), il reste
des situations où les victimes sont difficiles à localiser, ce qui
conduit à une perte de temps, dont les conséquences peuvent s'avérer
dramatiques (\autoref{chap:1}).

Le projet de recherche Choucas, issu d'une collaboration scientifique
entre le \ac{pghm} de Grenoble, les UMRs LaSTIG et LIG et l'EA LIUPPA,
souhaite élaborer des solutions d'aide à la localisation de
victimes. Comme indiqué dans le \autoref{chap:1}, les solutions
développées au sein de ce projet n'ont pas pour but d'automatiser la
localisation (et donc se substituer aux secouristes) mais d'apporter
des solutions d'aide à la décision, facilitant la définition et la
vérification des hypothèses de localisation énoncées par les
secouristes. Pour ce faire, quatre axes de travail ont été identifiés
(\autoref{tab:synthese_objectifs_choucas}). Le premier d'entre eux
consiste à enrichir les données dont disposent les secouristes,
notamment en exploitant les corpus de récits de randonnées mis en
ligne par les alpinistes. Ces nouvelles données compléterons les
données dont disposent déjà les secouristes. Ces dernières peuvent
être numériques (\eg bases de données géographiques) ou analogiques
(\eg plans de pistes de ski, \emph{topoguides,} etc.)
%
Pour faciliter le traitement de ces différentes données, un second
objectif du projet Choucas est de travailler à leur intégration,
permettant aux secouristes de travailler efficacement avec ces
données, très hétérogènes.
%
Le troisième objectif de ce projet, auquel notre thèse se rattache,
consiste, quant à lui, à développer des modèles de raisonnement,
permettant d'identifier la position correspondant à une description de
position.
%
Enfin, le dernier objectif de projet consiste à travailler, d'une part
sur la géovisualisation des résultats produits par le troisième
objectif, mais également à produire une interface permettant aux
secouristes de formaliser leurs hypothèses.

Dans le second chapitre nous avons détaillé les objectifs de notre
thèse et expliqué que la transformation d'une position décrite en une
zone de coordonnées connues ne pouvait ce faire qu'en deux étapes. Une
première dite de \emph{spatialisation,} où les différents
\emph{indices de localisation} qui composent une alerte sont
transformés en des \emph{zones de localisation compatibles} et une
seconde étape de \emph{fusion} où les différentes \emph{zones de
  localisation compatibles} d'une même alerte sont combinée pour
former la \emph{zone de localisation probable} de l'alerte, \ie la
zone où tous les \emph{indices de localisation} sont vérifiés. Nous
avons identifié plusieurs verrous scientifiques à la réalisation de
ces deux objectifs. Le premier d'entre eux est la définition d'une
méthode de \emph{spatialisation} qui soit adaptée aux différents
\emph{indices de localisation,} \ie qui soit capable de construire une
\emph{zone de localisation compatible} pour des \emph{indices de
  localisation} aussi variés que : \enquote{Je sur une route} ou
\enquote{je vois une forêt}.  Un autre problème est que la sémantique
des \emph{indices de localisation} peut être difficile à
identifier. En effet, comme nous l'avons expliqué dans le chapitre 2
et dans la seconde partie du chapitre 3, ces derniers peuvent être
fortement \emph{imprécis.}  \emph{L'imprécision} des \emph{indices de
  localisation} implique de développer une méthode de spatialisation
qui soit capable de la prendre en compte dans le but d'améliorer la
qualité des \emph{zones de localisation compatibles,} c'est l'objet du
second verrou scientifique de cette thèse.  Un troisième point
essentiel est la notion \emph{d'incertitude}, également abordée dans
les chapitres 2 et 3. En effet, les secouristes doivent pouvoir
énoncer un doute sur la véracité des \emph{indices de localisation}
qui leur sont transmis. La principale difficulté posée par la prise en
compte de \emph{l'incertitude} est que la méthode développée devra
cohabiter avec la méthode de prise en compte de \emph{l'imprécision}
et, plus généralement, avec les méthodes de \emph{spatialisation.}  De
plus nous devons développer une méthode de \emph{fusion} des
\emph{zones de localisation compatibles.} Cette dernière doit pouvoir
cohabiter avec les méthodes développées pour prendre en compte
\emph{l'imprécision} et \emph{l'incertitude.} Enfin, nous souhaitons
proposer une méthode permettant d'évaluer les \emph{zones de
  localisation probables} construites.


% Transition partie 2
La prochaine partie de ce travail sera consacrée à la mise en œuvre
des concepts présentés jusqu'ici. Avec notamment la définition de la
méthodologie de la thèse et la présentation des concepts dédiés.


%%% Local Variables:
%%% mode: latex
%%% TeX-master: "../../main"
%%% End: