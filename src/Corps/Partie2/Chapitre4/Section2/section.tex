Dans le \autoref{chap:2} nous identifions deux étapes centrales de
notre méthode permettant de construire une \emph{zone de localisation
  probable} correspondant à \emph{une description de position,} la
\emph{spatialisation} et \emph{la fusion.} Cette présentation n'est
cependant satisfaisante qu'en première approximation. En effet, les
principes introduits dans ce chapitre (et plus particulièrement le
principe de décomposition) complexifient la méthodologie en y ajoutant
de nouvelles étapes, comme différentes phases de décomposition et de
fusion. Ainsi nos précédentes définitions (\autoref{chap:02}) de la
\emph{spatialisation} et de la \emph{fusion} doivent êtres complétées,
ce qui sera fait au fur et à mesure de cette partie.

Dans cette partie nous allons présenter une première formalisation de
notre méthodologie. L'objectif n'est cependant pas de présenter tous
les aspects des étapes de \emph{décomposition,} de
\emph{spatialisation} et de \emph{fusion} chaque étape de la méthode
étant détaillée dans un chapitre dédié (respectivement les chapitres
\ref{chap:05}, \ref{chap:07} et \ref{chap:08}). Nous ne parlerons donc
pas de certains aspects avancés de la modélisation, comme la prise en
compte de l'imprécision (cf. \autoref{chap:07}) ou de l'incertitude
(cf.). La méthode ici présentée est la plus générique possible  

La \autoref{fig:methodo_1} présente une synthèse visuelle de la
méthodologie que nous avons définie.

\subsection{La phase de \emph{décomposition}}

La \emph{décomposition} est la première phase de notre méthode. C'est
un ensemble de trois étapes permettant de passer d'un ensemble
\emph{d'indices de localisation} à un ensemble décomposé de ces
indices, prêt à être \emph{spatialisé.} Seule une de ces étapes (la
dernière) découle du principe de \emph{décomposition} des
\emph{relations de localisation.}

Ce principe sera détaillé dans le \autoref{chap:05}

\subsubsection{La décomposition de \emph{l'ensemble des indices de
    localisation}}

La première étape de cette phase est celle de la décomposition de
\emph{l'ensemble des indices de localisation}, \(I\)
(\autoref{fig:methodo_1}). Cet ensemble contient tous les
\emph{indices de localisation} qui nous sont transmis par les
secouristes, par le biais de l'interface de géovisualisation. Comme
l'indique la \autoref{fig:diag_acti_secours}, c'est la réception de
cet ensemble qui déclenche le processus de construction de la
\emph{zone de localisation probable.} La décomposition de l'ensemble
\(I\) permet d'extraire les différents \emph{indices de localisation}
(\(i_n\) sur la \autoref{fig:methodo_1}), à partir desquels serons
construites les\emph{ zones de localisation compatibles.} Cette
transformation était qualifée de \emph{spatialisation} dans le
\autoref{chap:02}, cependant, comme le montre la
\autoref{fig:methodo_1}, ce n'est pas à cette étape qu'apparaissent
les objets géographiques, sous la forme des \emph{zones de
  localisation compatibles,} mais après deux nouvelles étapes de
\emph{décomposition.} C'est pourquoi nous préférons réserver l'usage
de ce terme à l'étape centrale de la méthode, la création d'objets
géographiques. L'étape de décomposition de \emph{l'ensemble des
  indices de décomposition,} ne présente pas de difficultés techniques
ou d'apport scientifique, mais elle permet la \emph{spatialisation
  autonome des indices de localisation.} Comme on peut le voir sur la
\autoref{fig:methodo_1}, les indices décomposés lors de la première
étape du processus sont fusionnés au dernier moment, conformément au
principe de modélisation autonome.

\begin{equation}
  I = \{i_1, i_2, \dots, i_n \}
\end{equation}

\subsubsection{La décomposition des \emph{objets de référence indéfinis}}

La seconde étape de la phase de décomposition aborde un point que
nous n'avons pas encore présenté.
%
Contrairement à la première étape de \emph{la phase de décomposition,}
la seconde et la troisième phase peuvent être interverties, sans que
cela n'impacte la modélisation. L'ordre de présentation que nous
utilisons ici correspond à celui que nous avons implémenté
(cf. \autoref{chap:08}), mais également à une progression que nous
estimons plus simple à présenter.
%
les \emph{indices de localisation} contiennent toujours un (ou
plusieurs, si la relation de localisation est n-aire) \emph{objet de
  référence.} Cependant ce dernier peut être une instance clairement
définie, que l'on peut alors facilement identifier (\eg
\enquote{Grenoble}, \enquote{Le bois des Ayes}, \emph{etc.}) ou un
objet décrit par sa nature, qu'il est alors impossible d'identifier
sans informations supplémentaires (\eg \enquote{une ville},
\enquote{un bois}). Par conséquent, lorsque le secouriste identifie
les objets correspondant à la description du requérant (conformément
au principe \texttt{XXXX}, cf. \ref{sec:4-1-1-1}) il peut sélectionner
un ou plusieurs objets. Si \emph{l'indice de localisation} qui nous
est transmis contient plus d'un objet (dans le cas où la relation
modélisée n'est pas bi ou n-aire) alors \emph{l'objet de référence}
est indéfini est \emph{l'indice de localisation} doit être
\emph{spatialisé} pour chacun d'entre eux et la position correspondant
à \emph{l'indice de localisation} et l'une des positions possibles.
La seconde étape de la \emph{phase de décomposition} permet donc de
transformer un \emph{indice de localisation} comportant plusieurs
\emph{objets de référence} concurrents en un ensemble \emph{d'indices
  de localisation} portant chacun sur un des \emph{objets de référence
  possible.} Prenons l'exemple de \emph{l'indice de localisation}
\enquote{je suis sur une crête}. Le secouriste peut alors sélectionner
à l'aide de l'interface l'ensemble des objets géographiques
\enquote{crêtes} situés dans la ZIR \footnote{Voir dans un espace plus
  réduit si le secouriste décide, compte-tenu de ses connaissances et
  de la situation, de réduire la zone de recherche de cette catégorie
  d'objets.}. Admettons qu'elles soient au nombre de six,
\emph{l'indice de localisation} qui nous sera transmis sera donc de la
forme \enquote{je suis sur la \emph{crête de Roche Motte} ou je suis
  sur la \emph{crête de Font Froide} ou je suis sur la \emph{crête de
    Serre Chapelle} ou je suis sur la \emph{crête des Barres} ou je
  suis sur la \emph{crête du Petit Puy}}. Ainsi au sortir de cette
seconde étape de décomposition on dispose de \(i\) \emph{indices de
  localisation} (première décomposition), eux-mêmes divisés en
plusieurs \emph{indices de localisation,} en fonction du nombre
\emph{d'objets de référence} à prendre en compte.

\subsubsection{La décomposition des \emph{relations de localisation}}

La troisième est dernière étape de cette phase est celle de la
décomposition des \emph{relations de localisation.} C'est durant cette
étape qu'est appliqué le \emph{principe de décomposition} des
\emph{relations de localisation.} Comme nous l'expliquions lors de la
présentation de ce principe, cette étape consiste à décomposer les
\emph{relations de localisation} en un ensemble de composantes
sémantiquement indépendantes, les \emph{relations de localisation
  atomiques.} Comme pour les deux précédentes étapes, cette
décomposition n'est pas toujours nécessaire, \emph{la relation de
  localisation utilisée} pouvant être atomique (\eg
\enquote{proche}). La principale différence entre cette étape et les
précédentes est qu'il est nécessaire de disposer d'une information
supplémentaire pour savoir comment l'effectuer. En effet lorsqu'un
ensemble d'indice de localisation nous est transmis, les différents
\emph{indices de localisation} qui le composent sont déjà identifiés,
puisque saisis tels quels par le secouriste, de la même manière les
différents \emph{objets de référence.} Par conséquent, les deux
premières étapes de la \emph{phase de décomposition} ne présentent pas
de difficultés particulières, contrairement à la \emph{décomposition
  des relations de localisation.} En effet note souhait est de
permettre au secouriste de sélectionner la \emph{relation de
  localisation} correspondant à la description du requérant à partir
d'une liste pré-définie de \emph{relations de localisation}
(cf. principe de modélisation explicite des connaissances). Ainsi le
secouriste n'a pas à manipuler directement les \emph{relations de
  localisation atomiques.} Il est donc nécessaire de connaitre la
\emph{décomposition des relations de localisation} et de s'y référer
lors de cette étape. Ce processus sera détaillé plus longuement dans
le \autoref{chap:05}.


Pour illustrer l'ensemble de la \emph{phase de décomposition,}
imagions que \emph{l'ensemble des indices de localisation} (\(I\)
saisi par un secouriste à partir de la description suivante :
\enquote{je suis proche du \emph{Pic de Jean Ray} et suis sur une
  crête}. Ce dernier est composé de deux \emph{indices de
  localisation,} séparés lors de la première étape de la \emph{phase
  de décomposition.} Les deux \emph{indices de localisation :}
\enquote{je suis proche du \emph{Pic de Jean Ray}} et \enquote{je suis
  sur une crête} peuvent être alors traités indépendamment. La seconde
étape de la \emph{phase de décomposition} fragmente chaque
\emph{indice de localisation} en fonction des \emph{objets de
  référence.} Le premier \emph{indice de localisation} (\enquote{je
  suis proche du \emph{Pic de Jean Ray}}) se réfère à un objet nommé
et unique, il n'est donc pas nécessaire de le décomposer,
contrairement au second \emph{indice de localisation,} que l'on peut
décomposer en six nouveaux \emph{indices de localisation,} comme
illustré précédement : \enquote{je suis sur la \emph{crête de Roche
    Motte} ou je suis sur la \emph{crête de Font Froide} ou je suis
  sur la \emph{crête de Serre Chapelle} ou je suis sur la \emph{crête
    des Barres} ou je suis sur la \emph{crête du Petit Puy}.} Enfin,
la dernière étape de la phase de \emph{décomposition,} nécessite la
définition préalable d'un ensemble de \emph{relations de localisation
  atomiques,} ce qui sera abordé dans le \autoref{chap:05}. Cependant,
pour aller au bout de cet exemple nous allons considérer que la
relation de localisation \enquote{proche} est une \emph{relation
  spatiale atomique} (hypothèse déjà utilisée précédemment) et que la
relation de localisation \enquote{sur} se décompose en deux
\emph{relations spatiales atomiques,} \emph{proximité} et
\emph{différence d'altitude positive} \footnote{Comme nous le verrons
  ultérieurement cette décomposition est fortement
  simplificatrice. Cet exemple n'est donc pas à considérer comme
  représentatif de la décomposition de la \emph{relation de
    localisation} \enquote{sur}, mais bien comme une illustration de
  la phase de décomposition dans son ensemble.}. Ainsi, le premier
\emph{indice de localisation} n'est pas décomposé, puisque nous
considérons que la \emph{relation de localisation} \enquote{proche}
est atomique, contrairement à \emph{l'ensemble des indices de
  localisation} issus de la décomposition de l'indice : \enquote{je
  suis sur une crête}, qui voient tous leur relation de localisation
(\enquote{sur}) décomposée. Ainsi, au terme de la \emph{phase de
  décomposition,} l'ensemble des \emph{indices de localisation :}
\enquote{je suis proche du \emph{Pic de Jean Ray}} et \enquote{je suis
  sur une crête} devient :
%
\begin{quote}
  \enquote{je suis proche du \emph{Pic de Jean Ray}} et ((\enquote{je
    suis proche de la \emph{crête de Roche Motte}} et \enquote{je suis
    à une altitude supérieure à la \emph{crête de Roche Motte}}) ou
  (\enquote{je suis proche de la \emph{crête de Font Froide}} et
  \enquote{je suis à une altitude supérieure à la \emph{crête de Font
      Froide}}) ou (\enquote{je suis proche de la \emph{crête de Serre
      Chapelle}} et \enquote{je suis à une altitude supérieure à la
    \emph{crête de Serre Chapelle}}) ou (\enquote{je suis proche de la
    \emph{crête des Barres}} et \enquote{je suis à une altitude
    supérieure à la \emph{crête des Barres}}) ou (\enquote{je suis
    proche de la \emph{crête du Petit Puy}} et \enquote{je suis à une
    altitude supérieure à la \emph{crête du Petit Puy}})).
\end{quote}
%
Comme le montre cette longue énumération, \emph{le processus de
  décomposition,} même lorsqu'il est appliqué à un petit
\emph{ensemble d'indices de localisation} (ici deux), peut générer de
nombreux \emph{indices de localisation décomposés}.

Au terme de ces trois phases de décomposition nous
disposons donc d'un ensemble \emph{d'indices de localisation}




\subsection{La phase de \emph{spatialisation}}


Faite pour chaque indice (chap2)
Nécessite des méthodes spécifiques (eda)
Plusieurs objets de référence

Ce principe sera détaillé dans le \autoref{chap:07}

Pour rappel, l'opération de \emph{spatialisation} consiste à
construire une zone (dite de \emph{localisation compatible})
correspondant à \emph{l'indice de localisation}
(\autoref{fig:obj_spa}). Ainsi on peut définir une \emph{zone de
localisation compatible} (pour un \emph{indice de localisation} donné)
comme l'ensemble des positions où \emph{l'indice de localisation} est
vrai, soit :


\begin{equation}
  \text{\textsf{ZLC}}_i = \{(x,y) \ |\ \text{\textsf{RL}}_i(x,y) \wedge (x,y) ∈ \mathbb{R}^2\}
\end{equation}

Avec \textsf{ZLC} la \emph{zone de localisation compatible,} \(i\) un
\emph{indice de localisation,} \((x,y)\) un couple de réels formant
une position et \(\text{\textsf{RL}}_i(x,y)\) une fonction vraie si la
position \((x,y)\) est dans la \emph{zone de localisation compatible}
correspondant à \emph{l'indice de localisation} \(i\).

\subsection{La phase de fusion}

De toutes les étapes de notre méthodologie, la phase de \emph{fusion}
est, de loin, la plus simple. Son objectif est de combiner les
différentes \emph{zones de localisation compatibles} créées lors de la
phase de \emph{spatialisation} en une seule, la \emph{zone de
  localisation probable} (ZLP).

Ce principe sera détaillé dans le \autoref{chap:08}


\subsubsection{La \emph{fusion} des \emph{relations de localisation
    atomiques}}

\subsubsection{La \emph{fusion} des \emph{objets de référence
    indéfinis}}

\tdi{Revoir titre}

\subsubsection{La \emph{fusion} des \emph{indices de localisation}}


Contrairement à la
décomposition de l'ensemble des \emph{indices de localisation}
(l'étape précédente) on ne s'attend pas ici à ce que toutes les
possibilités soient vraies, mais seulement l'une d'entre-elles (d'où
la formulation contenant des \enquote{ou} et non des \enquote{et})


\begin{equation}
  \text{\textsf{ZLP}}_I = ⋂_{i \in I} \text{\textsf{ZLC}}ᵢ
\end{equation}

Avec \textsf{ZLP} la \emph{zone de localisation probable,} \(I\),
l'ensemble des \emph{indices de localisation} pris en compte, \(i\) un
\emph{indice de localisation} et \textsf{ZLP} un \emph{zone de
  localisation compatible.}

\begin{landscape}
  \begin{figure}[H]
    \centering
    \begin{tikzpicture}
  \tikzset{
    acc/.style={decorate,decoration={brace,raise=0cm,amplitude=.1cm}},
    accm/.style={acc,decoration={mirror}},
    acc3/.style={acc,decoration={amplitude=.2cm}},
    acc3m/.style={acc3,decoration={mirror}}
  }
  
  %\draw[step=.5,black,thin] (0,0) grid (22,-14);

%% Matrices
\node[font=\bfseries, baseline] (I) at (0,0) {$I$};

\matrix [matrix of math nodes,row sep=0.1cm,column sep=0.1cm,
anchor=south west, nodes={anchor=base, baseline}] (is) at
([xshift=1.5cm]I.north east)
{i_1\\i_2\\\vdots{}\\i_n\\};

\matrix [matrix of math nodes,row sep=0.1cm,column sep=0.1cm,
anchor=south west, nodes={baseline}] (is_o) at
([xshift=1.5cm,yshift=.5cm]is-1-1.east) {o_1\\o_2\\\vdots{}\\o_n\\};

\matrix [matrix of math nodes,row sep=0.1cm,column sep=1.5cm,
anchor=south west, nodes={minimum height=5mm}] (is_o_r) at ([xshift=1.5cm,yshift=.5cm]is_o-1-1.east)
{
  \text{RSA}_1&\text{ZLC}_{rsa1}\\
  \text{RSA}_2&\text{ZLC}_{rsa_2}\\
  \vdots{}&\vdots{}\\
  \text{RSA}_n&\text{ZLC}_{rsa_n}\\};

\matrix [matrix of math nodes,row sep=0.1cm,column sep=0.1cm,
anchor=west, nodes={baseline}] (zlb) at ([xshift=1.5cm]is_o_r.east  |- is_o)
{ZLC_{o_1}\\ZLC_{o_2}\\\vdots{}\\ZLC_{o_n}\\};

\matrix [matrix of math nodes,row sep=0.1cm,column sep=0.1cm,
anchor=west, nodes={anchor=base, baseline}] (zlc) at ([xshift=1.5cm]zlb.east  |- is)
{ZLC_1\\ZLC_2\\\vdots{}\\ZLC_n\\};

\node [anchor=west] (zlp) at ([xshift=1.5cm]zlc.east |- I)
{ZLP};


%% Accolades 
\begin{scope}
  \draw[acc3m] (is.north west) -- (is.south west);
  \draw[acc3m] (is_o.north west) -- (is_o.south west);
  \draw[acc3m] (is_o_r.north west) -- (is_o_r.south west);

  \draw[acc] (is-1-1.north east) -- (is-1-1.south east);
  \draw[acc] (is_o-1-1.north east) -- (is_o-1-1.south east);
  \draw[acc] (is_o_r-1-1.north east) -- (is_o_r-1-1.south east);

  \draw[accm] (is_o_r-1-2.north west) -- (is_o_r-1-2.south west);
  \draw[accm] (zlb-1-1.north west) -- (zlb-1-1.south west);
  \draw[accm] (zlc-1-1.north west) -- (zlc-1-1.south west);

  \draw[acc3] (is_o_r.north east) -- (is_o_r.south east);
  \draw[acc3] (zlb.north east) -- (zlb.south east);
  \draw[acc3] (zlc.north east) -- (zlc.south east);
\end{scope}


\begin{scope}
  \draw[->, black] (I.east) -- (is.west) node[pos=0,below] {1} node[pos=1,below]{1..*};

  \draw[->, black] (is-1-1.east) -- (is_o.west) node[pos=0,below] {1} node[pos=1,below]{1..*}; 
  
  \draw[->, black] (is_o-1-1.east) --
  (is_o_r.west) node[pos=0,below]
  {1} node[pos=1,below]{1..*}; 
  
  \draw[->, black] (is_o_r-1-1.east) -- (is_o_r-1-2.west) node[pos=0,below]
  {1} node[pos=1,below]{1}; 
  
  \draw[->, black] (is_o_r.east) -- (zlb-1-1.west) node[pos=0,below] {1..*} node[pos=1,below]{1}; 
  
  \draw[->, black]
  (zlb.east) -- (zlc-1-1.west) node[pos=0,below] {1..*} node[pos=1,below]{1}; 
  
  \draw[->, black] (zlc.east) -- (zlp.west) node[pos=0,below] {1..*} node[pos=1,below]{1}; 
\end{scope}


\begin{scope}
    \draw[->, black, dotted] ([xshift=.5cm]I.east) -- ([xshift=-.5cm]zlp.west) node[pos=0,below]
  {1} node[pos=1,below]{1}; 

  \draw[->, black, dotted] ([xshift=.5cm]is.east) -- ([xshift=-.5cm]zlc.west) node[pos=0,below]
  {1} node[pos=1,below]{1}; 

  \draw[->, black, dotted] ([xshift=.5cm]is_o.east) -- ([xshift=-.5cm]zlb.west) node[pos=0,below]
  {1} node[pos=1,below]{1}; 
\end{scope}

%% XX

\begin{scope}[below of=I]
  \draw[acc3m] (0,0) --++ (8,0) node[pos=.5, yshift=-.5cm] {\emph{Décomposition}};
  \draw[acc3m] (8,0) --++ (3,0) node[pos=.5, yshift=-.5cm] {\emph{Spatialisation}};
  \draw[acc3m] (13.4,0) --++ (2,0) node[pos=.5, yshift=-.5cm] {\emph{Fusion}};
\end{scope}

\end{tikzpicture}
    \caption{Méthodologie}
    \label{fig:methodo_1}
  \end{figure}
\end{landscape}


%%% Local Variables:
%%% mode: latex
%%% TeX-master: "../../../../main"
%%% End:
