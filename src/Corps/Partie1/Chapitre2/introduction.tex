Comme nous l'avons vu dans le chapitre précédent, cette thèse
s'inscrit dans un projet de recherche ---~co-construit avec le
\ac{pghm} de Grenoble~---, dont l'objectif principal est de développer
des solutions destinées à faciliter la localisation de personnes
perdues en montagne. Parmi toutes les recherches nécessaires à la
bonne réalisation de cet objectif, notre travail de thèse se concentre
sur la question de la transformation d'une description de position, en
une zone, que nous qualifions de \enquote{\emph{localisation}}.

Ce chapitre se destine à présenter, plus en détail, nos objectifs et
notre problématique de recherche. Dans un premier temps
(\autoref{sec:2-1}) nous présenterons les différents \emph{objectifs
  scientifiques} de ce travail, avant (\autoref{sec:2-2}) de présenter
la problématique de cette thèse.

%%% Local Variables:
%%% mode: latex
%%% TeX-master: "../../../main"
%%% End:
