Dans ce dernier chapitre nous avons présenté l'implémentation de notre
méthode dans un prototype, Ruitor et détaillé la manière dont nous
avons procédé pour modéliser des alertes réelles. Pour ce faire nous
avons sélectionnée deux alertes : \emph{Grand Veymont} et le \emph{fil
  rouge} et avons cherché à les spatialiser avec les \emph{relations
  de localisation} définies dans l'ontologie \ac{orl}. Nous avons
analysé ces deux alertes de manière à identifier la manière la plus
adaptée de les modéliser, puis avons construit les \emph{zones de
  localisation compatibles} correspondant à chacune d'entre-elles.
