% Acronymes

% newacronym[plural={<acronyme pluriel>},
%		first={<texte à afficher à la première occurrence>},
%		firstplural={<idem, mais au pluriel>}
%]{<label>}{<acronyme>}{<Définition>}

%% Acronymes métier
\newacronym[first={peloton de Gendarmerie de haute-montagne},
firstplural={pelotons de Gendarmerie de haute-montagne
(PGHM)}]{pghm}{PGHM}{Pelotons de Gendarmerie de haute-montagne}

\newacronym[plural={PGM},first={peloton de Gendarmerie de montagne},
firstplural={pelotons de Gendarmerie de montagne (PGM)}]{pgm}{PGM}{Pelotons
de Gendarmerie de montagne}

\newacronym[first={compagnie républicaine de sécurité en montagne},
firstplural={compagnies républicaines de sécurité en montagne (CRS
Montagne)}]{crsm}{CRS Montagne}{Compagnies républicaines de sécurité
en montagne}

\newacronym{gmsp}{GMSP}{Groupes montagne sapeurs-pompiers}

\newacronym{grimp}{GRIMP}{Groupes de reconnaissance et d'intervention en milieux périlleux}
\newacronym{orsec_o}{ORSEC}{Organisation de la réponse de la sécurité civile (jusqu'en 2006)}
\newacronym[first=ORSEC, parent=orsec_o]{orsec}{ORSEC}{Organisation des secours}
\newacronym{pcs}{PCS}{Plan communal de sauvegarde}
\newacronym{caf}{CAF}{Club alpin français}
\newacronym{codis}{CODIS}{Centre opérationel départemental d'incendies et de secours}
\newacronym{usem}{USEM}{Unités de secours en montagne}
\newacronym{pms}{PMS}{Plan de secours en montagne}
\newacronym{samu}{SAMU}{Service d'aide médicale urgente}
\newacronym{smur}{SMUR}{Service mobile d'urgence et de réanimation}
\newacronym{cos}{COS}{Chef des opérations de secours}
\newacronym{dos}{DOS}{Directeur des opérations de secours}
\newacronym{cc}{CC}{Chef de caravane}
\newacronym{ffm}{FFM}{Fédération française de montagne}
\newacronym{ehm}{EHM}{École de haute-montagne (École militaire de haute-montagne à partir de 1964)}
\newacronym{cenas}{CENAS}{Centre national d'entrainement à l'alpinisme et au ski}
\newacronym{snosm}{SNOSM}{Système national d'observation de la sécurité en montagne}

%% Acronymes SIG
\newacronym{mnt}{MNT}{modèle numérique de terrain}

% Glossaire

\newglossaryentry{DZ}{name={\emph{Drop zone}},description={Zone
d'atterissage}}
\newacronym[first=DZ, parent=DZ]{DZa}{DZ}{\emph{Drop
zone}}

\newglossaryentry{vire}{name={vire},description={\enquote{Replat étroit le long d'un escarpement montagneux}
\cite{Tresor2020}}}