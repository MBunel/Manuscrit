La phase de \emph{spatialisation,} au cœur de notre méthode, n'est pas
subdivisée en plusieurs étapes, contrairement aux phases de
\emph{décomposition} (\autoref{chap:05}) et de \emph{fusion}
(\autoref{chap:08}). Cependant, la \emph{méthode} développée pour la
\emph{spatialisation} est composée de plusieurs parties, correspondant
à des traitements computationnels effectués séquentiellement, et
toutes porteuses d'une partie de la sémantique de la \emph{relation de
  localisation atomique} modélisée.

Une \emph{relation spatiale atomique} est donc définie par trois
concepts : un \emph{rasteriser,} qui décrit la partie de \emph{l'objet
  de référence} à laquelle se rapporte la \emph{relation de
  localisation atomique} et définit la manière de \emph{rasteriser}
\emph{l'objet de référence.} La \emph{métrique,} qui est une grandeur
permettant d'approcher la sémantique de la \emph{relation de
  localisation atomique} et le \emph{fuzzyfieur,} qui est une méthode
permettant de sélectionner les pixels de la \ac{zir}, en fonction de
la valeur de \emph{la métrique,} correspondant aux valeurs attendues
au sein de la \ac{zlc}. À ces trois concepts, peut s'ajouter un (ou
des) \emph{modifieur,} qui permet de transformer la sémantique obtenue
par l'association d'un \emph{rasteriser,} d'une \emph{métrique} et
d'un \emph{fuzzyficateur,} par exemple en l'inversant.

La définition de ces différents concepts et de leurs liens avec les
\emph{relations de localisation atomiques} est fait au sein de
\emph{l'ontologie} \ac{orla}, également utilisée pour représenter la
\emph{décomposition} des \emph{relations de localisation} en des
\emph{relations de localisation atomiques} (\autoref{chap:05}).

La \emph{spatialisation} d'une \emph{relation de localisation
  atomique} est donc réalisée en appliquant successivement, dans
l'aire délimitée par la \ac{zir}, le \emph{rasteriser} à \emph{l'objet
  de référence} traité, en calculant la \emph{métrique,} puis en la
\emph{fuzzyfiant.} On obtient alors une représentation raster de la
\emph{zone de localisation compatible,} où chaque pixel se voit
attribuer un \emph{degré,} quantifiant son appartenance à la \ac{zlc}.


%%% Local Variables:
%%% mode: latex
%%% TeX-master: "../../../main"
%%% End:
