\begin{tabular}{>{\small}L{.3\textwidth}>{\footnotesize}p{.4\textwidth}}
  \toprule \multicolumn{1}{c}{\bfseries Colonne} &
  \multicolumn{1}{c}{\normalsize \bfseries Contenu} \\ \midrule
% Sémantique des relations spatiales
  \addlinespace
  Identifiant de l'extrait & Permet d'identifier une expression\\
  Identifiant de l'expression & Permet d'identifier une expression au
                                sein d'un extrait en comprenant plusieurs\\
  Extrait & Verbatim de la phrase transcrite\\
  Confiance & Permet au transcripteur de saisir sa confiance en
              \emph{l'indice de localisation} (uniquement si la saisie
              est faite par un secouriste)\\
  Timestamp & Début de l'extrait dans le fichier audio source\\
  Locuteur & Permet d'identifier le locuteur (\eg secouriste,
             requérant, témoin, \emph{etc.})\\
  Verbe & Verbe utilisé dans \emph{l'indice de localisation}\\
  Modifieur du verbe & Modifieur du verbe (\eg marcher \emph{vite})\\
  Sujet & Sujet de \emph{l'indice de localisation}\\
  Modifieur du sujet & Modifieur du \emph{sujet}\\
  \emph{Relation de localisation} & \emph{Relation de localisation}
                                    utilisée dans l'extrait\\
  Modifieur de la \emph{relation de localisation} &  Modifieur de la \emph{relation de localisation} (\eg \emph{très} loin)\\
  \emph{Objet de référence} & Nom ou type de \emph{l'objet de référence}
                              (cette colonne peut être multipliée si la
                              \emph{relation de localisation} est bi ou n-aire)\\
  Modifieur de \emph{l'objet de référence} & Modifieur de
                                             \emph{l'objet de
                                             référence} (\eg
                                             \enquote{un grand lac}. Cette colonne peut être multipliée si la
                                             \emph{relation de localisation} est bi ou n-aire)\\
  Commentaires & Champ permettant au transcripteur de commenter sa
                 saisie ou les indications données par le requérant
                 (erreurs potentielles, fautes de prononciation, \emph{etc.}) \\
  \bottomrule
\end{tabular}
