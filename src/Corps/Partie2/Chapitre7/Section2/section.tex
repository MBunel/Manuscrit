La première partie de la méthode de spatialisation est la
\emph{rasterisation des objets de référence.}

\tdi{méthode de rasterisation = sélection partie objet ref +
  rasterisation}

\tdi{Méthode de rasterisation fixe, seule la sélection varie}

\subsection{Rasterisation et sémantique ??}

\tdi{Expliquer comment on les sélectionne et leur importance sur la
  sémantique}

\tdi{parler des consiérations autour de la rasterisation importance de
  la zir, de la résolution, méthode de sélection des pixels, etc}


\subsection{Les méthode de rasterisation}

\tdi{Détailler les différentes méthodes de rasterisation définies
  figure \ref{fig:methode_rasterisation} + exemple}


La plus évidente des solutions consiste à rasteriser l'ensemble de
\emph{l'objet de référence.}


Mais il également possible de ne rasteriser qu'une partie de
\emph{l'objet de référence,} comme sa frontière ou son centroide
(\autoref{tab:methode_rasterisation}).

\begin{table}
  \centering
  \begin{tabular}{>{\bfseries}R{2cm}C{3cm}C{3cm}C{3cm}}
  \toprule
  & \multicolumn{1}{c}{\bfseries Ponctuel} &
   \multicolumn{1}{c}{\bfseries Linéaire} &
   \multicolumn{1}{c}{\bfseries Région} \\
  \addlinespace
  & \tikz{\fill[black] (.65,.65) circle [radius=1pt];\path[ffc, black] (0,0) rectangle (2,2);}&
  \tikz{\path[ffc,black] (.3,.3) ..controls(.25,1.75) and (1.375,.25) .. (1.7,1.7); \path[ffc,black] (0,0) rectangle (2,2);} &
  \tikz{\fill[ffa, pattern color=black] (1.125,.65) circle [radius=10pt];\path[ffc,black](1.125,.65) circle [radius=10pt];\path[ffc,black](0,0) rectangle (2,2);}\\
  %
  \midrule
  \addlinespace
  Objet
  & \tikz{
    \begin{scope}
      \foreach \x in {0,.25,...,1.75}{
        \foreach \y in {0,.25,...,1.75}
        \path[draw, line width=.01mm] (\x,\y) rectangle (\x +.25, \y + .25);
      }\path[ffa] (.5,.5) rectangle (.75,.75);
      \path[ffc] (0,0) rectangle (2,2);
    \end{scope}
    }
                                           & \tikz{
                                             \begin{scope}
                                               \foreach \x in {0,.25,...,1.75}{
                                                 \foreach \y in {0,.25,...,1.75}
                                                 \path[draw, line width=.01mm] (\x,\y) rectangle (\x +.25, \y + .25);
                                               }
                                               \foreach \i/\j in {.25/.25,.25/.5,.25/.75,.5/.75,.5/1,.75/.75,.75/1,1/.75,1/1,1.25/1,1.5/1,1.5/1.25,1.5/1.5}{
                                                 \path[ffa] (\i,\j) rectangle (\i + .25, \j +.25);
                                               }
                                               \path[ffc] (0,0) rectangle (2,2);
                                             \end{scope}
                                             } &\tikz{
                                                 \begin{scope}
                                                   \foreach \x in {0,.25,...,1.75}{
                                                     \foreach \y in {0,.25,...,1.75}
                                                     \path[draw, line width=.01mm] (\x,\y) rectangle (\x +.25, \y + .25);
                                                   }
                                                   \path[ffa] (.75,.25)
                                                   rectangle (1.5,
                                                   1);\path[ffc] (0,0) rectangle (2,2);
                                                 \end{scope}
                                                 }\\
  \addlinespace
  Frontière&\tikz{
             \begin{scope}
               \foreach \x in {0,.25,...,1.75}{
                 \foreach \y in {0,.25,...,1.75}
                 \path[draw, line width=.01mm] (\x,\y) rectangle (\x +.25, \y + .25);
               }
               \path[ffa] (.5,.5) rectangle (.75,.75);
               \path[ffc] (0,0) rectangle (2,2);
             \end{scope}
             }
                                           & \tikz{
                                             \begin{scope}
                                               \foreach \x in {0,.25,...,1.75}{
                                                 \foreach \y in {0,.25,...,1.75}
                                                 \path[draw, line width=.01mm] (\x,\y) rectangle (\x +.25, \y + .25);
                                               }
                                               \path[ffa] (.25,.25) rectangle (.5, .5);
                                               \path[ffa] (1.5,1.5) rectangle (1.75, 1.75);
                                               \path[ffc] (0,0) rectangle (2,2);
                                             \end{scope}
                                             } &\tikz{
                                                 \begin{scope}
                                                   \foreach \x in {0,.25,...,1.75}{
                                                     \foreach \y in {0,.25,...,1.75}
                                                     \path[draw, line width=.01mm] (\x,\y) rectangle (\x +.25, \y + .25);
                                                   }
                                                   % Pixels rasterisation
                                                   \path[ffa] (.75,.25)
                                                   rectangle (1.5,
                                                   .5);  
                                                   \path[ffa] (.75,.5)
                                                   rectangle (1,.75);
                                                   \path[ffa] (1.25,.5)
                                                   rectangle (1.5,.75);
                                                   \path[ffa] (.75,.75)
                                                     rectangle (1.5,1);
                                                   \path[ffc] (0,0) rectangle (2,2);
                                                 \end{scope}
                                                 }\\
%
  \addlinespace
%
  Centroide&\tikz{
             \begin{scope}
               \foreach \x in {0,.25,...,1.75}{
                 \foreach \y in {0,.25,...,1.75}
                 \path[draw, line width=.01mm] (\x,\y) rectangle (\x +.25, \y + .25);
               }
               \path[ffa] (.5,.5) rectangle (.75,.75);
               \path[ffc] (0,0) rectangle (2,2);
             \end{scope}
             }
                                           & \tikz{
                                             \begin{scope}
                                               \foreach \x in {0,.25,...,1.75}{
                                                 \foreach \y in {0,.25,...,1.75}
                                                 \path[draw, line width=.01mm] (\x,\y) rectangle (\x +.25, \y + .25);
                                               }
                                               \path[ffa] (1,1) rectangle (1.25, 1.25);
                                               \path[ffc] (0,0) rectangle (2,2);
                                             \end{scope}
                                             } &\tikz{
                                                 \begin{scope}
                                                   \foreach \x in {0,.25,...,1.75}{
                                                     \foreach \y in {0,.25,...,1.75}
                                                     \path[draw, line width=.01mm] (\x,\y) rectangle (\x +.25, \y + .25);
                                                   }
                                                   \path[ffa] (1,.5)
                                                   rectangle (1.25, .75);
                                                   \path[ffc] (0,0) rectangle (2,2);
                                                 \end{scope}
                                                 }\\
  \bottomrule    
\end{tabular}
  \caption{Méthodes de rasterisation}
  \label{tab:methode_rasterisation}
\end{table}

\tdi{Parler rapidement des méthodes définies mais non utilisées (bbox,
  conxhull)}

D'autres méthodes plus spécifiques sont également envisageables.

%%% Local Variables:
%%% mode: latex
%%% TeX-master: "../../../../main"
%%% End:
