Ce chapitre est dédié à la phase de \emph{spatialisation,} étape primordiale et centrale de notre méthode de transformation d'un référencement spatial indirect en référencement spatial direct (\autoref{fig:methodo_1}).
%
% Rappeler la définition de la spatialisation
Son objectif est de créer la \emph{zone de localisation compatible} correspondant à un \emph{indice de localisation décomposé.} Pour réaliser cette construction nous avons défini une méthode en trois parties successives, la \emph{rasterisation,} le calcul de la \emph{métrique} et la \emph{fuzzyfication,} qui seront présentées dans la première partie de ce chapitre, puis respectivement détaillées dans les parties 2, 3 et 4 de ce chapitre. Enfin, nous questionnerons sur la manière dont les connaissances nécessaires à la mise en place de cette méthode de spatialisation doivent être représentées.

%%% Local Variables:
%%% mode: latex
%%% TeX-master: "../../../main"
%%% End:
