\subfloat[Exemple de décomposition de la relation topologique EC]{
  \begin{tabular}{rC{5.5cm}C{5.5cm}}
    \toprule
    & OUTSIDE\_OUTSIDE\_EC & P\_INSIDE\_INSIDEi\_EC \\
    \midrule
    EC & \tikz{\path[ffa] (0,0) circle [radius=15pt] node {A};}&
                                                                 \tikz{\path[ffa] (0,0) circle [radius=15pt] node {B};} \\
    \bottomrule    
  \end{tabular}
  \label{tab:RCC8_vs_RCC23_1}
}

\subfloat[Exemple de décomposition de la relation topologique DC]{
  \begin{tabular}{rC{5.55cm}C{5.5cm}}
    \toprule
    & INTSIDE\_OUTSIDEi\_DC & P\_OUTSIDE\_OUTSIDE\_EC \\
    \midrule
    DC & \tikz{
         \path[ffa] (0,10);}&
                              \tikz{\path[ffa] (0,0) circle [radius=15pt] node {B};} \\
    \bottomrule    
  \end{tabular}
  \label{tab:RCC8_vs_RCC23_2}
}

