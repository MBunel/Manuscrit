
\subsection{Présentation de l'alerte}
\label{subsec:9-3-1}


\begin{figure}
  \centering
  \input{./figures/ZIR_moucherotte.tex}
  \caption{Zone initiale de recherche de l'alerte
    \enquote{Moucherotte}}
  \label{fig:zir_moucherotte}
\end{figure}

\subsubsection{Identification des indices de localisation}
\label{subsec:9-3-1-1}

Section

\begin{dialogue}
  % 
  \Sec \nex{1-1} Donc vous étiez au sommet du Moucherotte à 17h30.
  %
  \Req Ouais.
  %
  \Sec \nex{2-1} Vous êtes redescendue par le chemin que
  vous avez pris à la montée ?
  %
  \Req \nex{3-1} Oui, ben j’ai essayé de le trouver, je l’ai pas
  trouvé, \nex{4-1} donc j’ai viré un peu à gauche, \nex{5-1} sur ce
  grand couloir enseigné. \nex{6-1, 6-2} Je crois qu’il y a des gens
  qui redescendent par Lens-en-Vercors, \nex{7-1, 7-2} mais je suis
  montée de Sain-Nizier, du haut du tremplin.
\end{dialogue}
% 


\begin{dialogue}
  % 
  \Sec \nex{8-1} Vous êtes gardée à Saint-Nizier ?
  %
  \Req \nex{9-1} Saint-Nizier ouais, le haut du tremplin
\end{dialogue}


\begin{dialogue}
  \Req \nex{10-1} Mais là je suis complètement paumée, \nex{10-2} dans
  les cailloux. \nex{11-1} Jusqu’à maintenant j’étais dans la forêt,
  \nex{11-2} un peu de neige. \nex{12-1} Mais là, j’ai failli me tuer
  en glissant \nex{12-2} parce qu’il y a plein de petits morceaux de
  cailloux.
\end{dialogue}

\begin{dialogue}
  \Req \nex{13-1} En partant du sommet du Moucherotte, donc j’ai viré
  un peu à gauche et \nex{14-1} après j'ai coupé pour \nex{14-2}
  descendre, descendre, descendre, descendre et \nex{15-1} j’ai pas
  retrouvé mon sentier et \nex{15-2} je suis certainement descendue
  beaucoup trop bas et là \nex{16-1} j’ai plus dutout la force de
  remonter, je suis trop descendue.
\end{dialogue}

\begin{dialogue}
  \Req \nex{17-1} Mon but c’était de descendre, descendre, descendre
  vers les maison qui sont en bas, \nex{18-1} mais je là j’ai une
  grosse barre rocheuse à droite, \nex{19-1} j’ai plein de petits
  cailloux très glissants, \nex{19-2} c’est très raide. \nex{20-1} Je
  suis assise là, au milieu de nulle part.
\end{dialogue}


\begin{dialogue}
  \Sec \nex{21-1} Dans l’idée, vous êtes repartie en direction du
  tremplin ? On est d’accord ? Vous avez emprunté le même chemin ?
  % 
  \Req \nex{21-2} Voila. Oui. \nex{22-1} Mais, pas par la droite où
  \nex{22-2} apparemment y’a deux sortes de chemins, pour le
  rejoindre. \nex{23-1} Parce-qu’à un moment donné y’a une espèce de
  cabane qu’on voit quand on est au Moucherotte. C’est pas
  là. \nex{24-1} Moi je suis allée plutôt sur la gauche. J’ai fait
  quoi, 200 mètres à gauche et \nex{24-2} puis j’ai coupé à la
  perpendiculaire en redescendant.
  %
  \Sec \nex{25-1} Et vous avez jamais croisé un chemin ?
  %
  \Req \nex{25-2} J’étais arrivée par là et j’ai pas retrouvé de
  chemin.
\end{dialogue}

\begin{dialogue}
  \Sec \nex{26-1} Vous êtes donc plutôt rentrée dans la forêt là ?
  %
  \Req \nex{26-2} Là je suis un peu à la sortie. \nex{26-3} J’ai fait
  un peu d’escalade là, \nex{26-3} parce que c’est très
  rocheux. \nex{26-4} Je suis dans une partie où il y a plein d’arbres
  tous secs et pas trop d’arbres. Je suis complétement à découvert
\end{dialogue}


\begin{dialogue}
  \Req \nex{27-1} Et juste à ma droite y’a une grosse, grosse barre
  rocheuse. \nex{28-1} Ouais et devant moi j’ai la plaine et
  \nex{28-2} à ma gauche j’ai quand même des sapins.
\end{dialogue}


\begin{dialogue}
  \Sec \nex{29-1} Et donc, dans la plaine qu’est-ce que vous voyez ?
  %
  \Req \nex{30-1} Eh ben je vois un peu d’eau.
  %
  \Sec \nex{30} Vous voyez des habitations ? Vous voyez une route ?
  %
  \Req \nex{30-3} Un petit peu, mais les habitations c’est très
  parses. \nex{30-4} Et juste devant moi y’a un groupe de 4, 5 maisons
  ou des fermes.
\end{dialogue}

\subsection{Modélisation de l'alerte}
\label{subsec:9-3-2}

\subsubsection{Décomposition des indices de localisation}
\label{subsec:9-3-2-2}

\subsubsection{Spatialisation des indices de localisation}
\label{subsec:9-3-2-3}

\subsubsection{Fusion des zones de localisation compatibles}
\label{subsec:9-3-2-4}

\subsection{Critique de la modélisation}
\label{subsec:9-3-3}


