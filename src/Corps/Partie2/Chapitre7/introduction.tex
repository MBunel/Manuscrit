\tdi{Rappeler la définition de la spatialisation}

\tdi{La spatialisation en plusieurs étapes}

La phase de \emph{spatialisation} est centrale dans notre méthode de transformation d'un référencement spatial indirect en référencement spatial direct. D'une part car elle s'en situe a mis-chemin (cf. \autoref{fig:methodo_1}), mais surtout car c'est lors de cette étape que sont créés les premiers objets spatiaux, les \emph{zones de localisation compatibles.} 

Si la phase de \emph{spatialisation} n'est composée d'aucune étape ---~contrairement à la \emph{décomposition} et la \emph{fusion}~---, elle est organisée suivant trois parties distinctes que nous allons présenter.

La première partie de ce chapitre présentera la \emph{phase de spatialisation} et recensera les différentes parties, \emph{rasterisation des objets de référence}, \emph{calcul des métriques} et \emph{fuzzyfication,} qui seront respectivement détaillées dans les parties 2, 3 et 4 de ce chapitre. Enfin, nous présenterons la manière dont ces connaissances sont représentées.

%%% Local Variables:
%%% mode: latex
%%% TeX-master: "../../../main"
%%% End:
