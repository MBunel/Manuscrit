Dans le \autoref{chap:2} nous ébauchions une méthode de construction
d'une \emph{zone de localisation probable} à partir \emph{d'une
  description de position,} en deux étapes : la \emph{spatialisation}
et la \emph{fusion.} Cette présentation n'est cependant satisfaisante
qu'en première approximation. En effet, les principes de modélisation
introduits dans ce chapitre (et plus particulièrement le
\emph{principe de décomposition}) complexifient la méthodologie en y
ajoutant de nouvelles étapes, ne pouvant être distinguées avec ces
anciennes définitions. C'est pourquoi ces dernières seront complétées
et revues au fur et à mesure de cette partie.

Nous allons à présent formaliser notre méthode de construction de
\emph{zones de localisation probables}. L'objectif n'est cependant pas
d'en présenter tous les aspects, mais d'en décrire le fonctionnement
global et les interactions entre les différentes phases de la
méthode. Ainsi les considérations les plus avancées, comme la prise en
compte de \emph{l'incertitude} ou de \emph{l'imprécision,} ou les
points spécifiques à une phase particulière de la méthodologie seront
détaillés dans des chapitres dédiés. Ce chapitre ne se destine donc qu'à
présenter le socle de notre méthode.

La méthode que nous proposons (et dont la \autoref{fig:methodo_1}
présente une synthèse visuelle) est décomposable en trois phases, la
\emph{décomposition,} la \emph{spatialisation} et la \emph{fusion.}

\subsection{La phase de \emph{décomposition}}

La phase de \emph{décomposition} (détaillé dans le \autoref{chap:05})
est un ensemble de trois étapes permettant de passer d'un ensemble
\emph{d'indices de localisation} à un ensemble décomposé de ces
indices, prêts à être \emph{spatialisés.}

\subsubsection{La décomposition de \emph{l'ensemble des indices de
    localisation}}

La première de ces étapes est la décomposition de \emph{l'ensemble des
  indices de localisation}, \(I\) (\autoref{fig:methodo_1}). Cet
ensemble contient tous les \emph{indices de localisation} (\(i_n\))
qui nous sont transmis par les secouristes, par le biais de
l'interface de géovisualisation :

\begin{equation}
  I = \{i_1, i_2, \dots, i_n \}
\end{equation}

La \emph{décomposition} de l'ensemble \(I\) permet d'extraire les
différents \emph{indices de localisation} (\(i_n\), à partir desquels
serons construites les \emph{zones de localisation compatibles.} Cette
transformation était qualifée de \emph{spatialisation} dans le
\autoref{chap:02}, cependant, comme le montre la
\autoref{fig:methodo_1}, ce n'est pas à cette étape qu'apparaissent
les objets géographiques, sous la forme des \emph{zones de
  localisation compatibles,} mais après deux nouvelles étapes de
\emph{décomposition.} C'est pourquoi nous préférons réserver l'usage
de ce terme à l'étape centrale de la méthode, la création d'objets
géographiques.

Cette étape de \emph{décomposition} ne présente pas de difficultés
techniques ou scientifique, mais elle est nécessaire à l’application
du \emph{principe de spatialisation autonome des indices de
  localisation.} Comme on peut le voir sur la \autoref{fig:methodo_1,}
les indices décomposés lors de la première étape du processus sont
fusionnés lors de la dernière étape de la \emph{phase de fusion.}
Ainsi les différents \emph{indices de localisation} sont traités
indépendamment, conformément au principe de modélisation précédemment
mentionné. On peut illustrer ce principe avec l'exemple de
\emph{l'ensemble d'indices de localisation} : \enquote{je suis proche
  du \emph{Pic de Jean Ray} et sur une crête.} Ce dernier est composé
de deux \emph{indices de localisation,} liés par une conjonction. Le
but de cette étape de \emph{décomposition} est de les séparer pour en
permettre le traitement autonome. Au terme de cette étape on dispose
ainsi de deux \emph{indices de localisation} (\enquote{je suis proche
  du \emph{Pic de Jean Ray}} et \enquote{je suis sur une crête}),
indépendamment transmis à l'étape suivante de décomposition.

\subsubsection{La décomposition des \emph{objets de référence indéfinis}}

Contrairement à la première étape de \emph{la phase de décomposition,}
la seconde et la troisième phase peuvent être interverties, sans que
cela n'impacte la modélisation. L'ordre de présentation que nous
utilisons ici correspond à celui que nous avons implémenté
(cf. \autoref{chap:08}), mais également à une progression que nous
estimons plus simple à présenter.

Les \emph{indices de localisation} contiennent toujours un (ou
plusieurs, si la relation de localisation est n-aire) \emph{objet de
  référence.} Cependant ce dernier peut être une instance clairement
définie, que l'on peut alors facilement identifier (\eg
\enquote{Grenoble}, \enquote{Le bois des Ayes}, \emph{etc.}) ou un
objet décrit par sa nature, qu'il est alors impossible d'identifier
sans informations supplémentaires (\eg \enquote{une ville},
\enquote{un bois}). Par conséquent, lorsque le secouriste identifie
les objets correspondant à la description du requérant (conformément
au principe \texttt{TITRE principe}, cf. \ref{sec:4-1-1-1}) il peut
sélectionner un ou plusieurs objets. Si \emph{l'indice de
  localisation} qui nous est transmis contient plus d'un objet (dans
le cas où la relation modélisée n'est pas bi ou n-aire) alors
\emph{l'objet de référence} est indéfini est \emph{l'indice de
  localisation} doit être \emph{spatialisé} pour chacun d'entre eux et
la position correspondant à \emph{l'indice de localisation} et l'une
des positions possibles.  La seconde étape de la \emph{phase de
  décomposition} permet donc de transformer un \emph{indice de
  localisation} comportant plusieurs \emph{objets de référence}
concurrents en un ensemble \emph{d'indices de localisation} portant
chacun sur un des \emph{objets de référence possible.}

Prenons l'exemple de \emph{l'indice de localisation} \enquote{je suis
  sur une crête}. Le secouriste peut alors sélectionner à l'aide de
l'interface l'ensemble des objets géographiques \enquote{crêtes}
situés dans la ZIR \footnote{Voir dans un espace plus réduit si le
  secouriste décide, compte-tenu de ses connaissances et de la
  situation, de réduire la zone de recherche de cette catégorie
  d'objets.}. Admettons qu'elles soient au nombre de six,
\emph{l'indice de localisation} qui nous sera transmis sera donc de la
forme \enquote{je suis sur la \emph{crête de Roche Motte} ou je suis
  sur la \emph{crête de Font Froide} ou je suis sur la \emph{crête de
    Serre Chapelle} ou je suis sur la \emph{crête des Barres} ou je
  suis sur la \emph{crête du Petit Puy}}. Ainsi au sortir de cette
seconde étape de décomposition on dispose de \(i\) \emph{indices de
  localisation} (première décomposition), eux-mêmes divisés en
plusieurs \emph{indices de localisation,} en fonction du nombre
\emph{d'objets de référence} à prendre en compte.

\subsubsection{La décomposition des \emph{relations de localisation}}

La troisième est dernière étape de cette phase est celle de la
décomposition des \emph{relations de localisation.} C'est durant cette
étape qu'est appliqué le \emph{principe de décomposition} des
\emph{relations de localisation.} Comme nous l'expliquions lors de la
présentation de ce principe, cette étape consiste à décomposer les
\emph{relations de localisation} en un ensemble de composantes
sémantiquement indépendantes, les \emph{relations de localisation
  atomiques.} Comme pour les deux précédentes étapes, cette
décomposition n'est pas toujours nécessaire, \emph{la relation de
  localisation utilisée} pouvant être atomique (\eg
\enquote{proche}). La principale différence entre cette étape et les
précédentes est qu'il est nécessaire de disposer d'une information
supplémentaire pour savoir comment l'effectuer. En effet lorsqu'un
ensemble d'indice de localisation nous est transmis, les différents
\emph{indices de localisation} qui le composent sont déjà identifiés,
puisque saisis tels quels par le secouriste, de la même manière les
différents \emph{objets de référence.} Par conséquent, les deux
premières étapes de la \emph{phase de décomposition} ne présentent pas
de difficultés particulières, contrairement à la \emph{décomposition
  des relations de localisation.} En effet note souhait est de
permettre au secouriste de sélectionner la \emph{relation de
  localisation} correspondant à la description du requérant à partir
d'une liste pré-définie de \emph{relations de localisation}
(cf. principe de modélisation explicite des connaissances). Ainsi le
secouriste n'a pas à manipuler directement les \emph{relations de
  localisation atomiques.} Il est donc nécessaire de connaitre la
\emph{décomposition des relations de localisation} et de s'y référer
lors de cette étape. Ce processus sera détaillé plus longuement dans
le \autoref{chap:05}.

Pour illustrer l'ensemble de la \emph{phase de décomposition,}
imagions que \emph{l'ensemble des indices de localisation} (\(I\)
saisi par un secouriste à partir de la description suivante :
\enquote{je suis proche du \emph{Pic de Jean Ray} et suis sur une
  crête}. Ce dernier est composé de deux \emph{indices de
  localisation,} séparés lors de la première étape de la \emph{phase
  de décomposition.} Les deux \emph{indices de localisation :}
\enquote{je suis proche du \emph{Pic de Jean Ray}} et \enquote{je suis
  sur une crête} peuvent être alors traités indépendamment. La seconde
étape de la \emph{phase de décomposition} fragmente chaque
\emph{indice de localisation} en fonction des \emph{objets de
  référence.} Le premier \emph{indice de localisation} (\enquote{je
  suis proche du \emph{Pic de Jean Ray}}) se réfère à un objet nommé
et unique, il n'est donc pas nécessaire de le décomposer,
contrairement au second \emph{indice de localisation,} que l'on peut
décomposer en six nouveaux \emph{indices de localisation,} comme
illustré précédement : \enquote{je suis sur la \emph{crête de Roche
    Motte} ou je suis sur la \emph{crête de Font Froide} ou je suis
  sur la \emph{crête de Serre Chapelle} ou je suis sur la \emph{crête
    des Barres} ou je suis sur la \emph{crête du Petit Puy}.} Enfin,
la dernière étape de la phase de \emph{décomposition,} nécessite la
définition préalable d'un ensemble de \emph{relations de localisation
  atomiques,} ce qui sera abordé dans le \autoref{chap:05}. Cependant,
pour aller au bout de cet exemple nous allons considérer que la
relation de localisation \enquote{proche} est une \emph{relation
  spatiale atomique} (hypothèse déjà utilisée précédemment) et que la
relation de localisation \enquote{sur} se décompose en deux
\emph{relations spatiales atomiques,} \emph{proximité} et
\emph{différence d'altitude positive} \footnote{Comme nous le verrons
  ultérieurement cette décomposition est fortement
  simplificatrice. Cet exemple n'est donc pas à considérer comme
  représentatif de la décomposition de la \emph{relation de
    localisation} \enquote{sur}, mais bien comme une illustration de
  la phase de décomposition dans son ensemble.}. Ainsi, le premier
\emph{indice de localisation} n'est pas décomposé, puisque nous
considérons que la \emph{relation de localisation} \enquote{proche}
est atomique, contrairement à \emph{l'ensemble des indices de
  localisation} issus de la décomposition de l'indice : \enquote{je
  suis sur une crête}, qui voient tous leur relation de localisation
(\enquote{sur}) décomposée. Ainsi, au terme de la \emph{phase de
  décomposition,} l'ensemble des \emph{indices de localisation :}
\enquote{je suis proche du \emph{Pic de Jean Ray}} et \enquote{je suis
  sur une crête} devient :
%
\begin{quote}
  \enquote{je suis proche du \emph{Pic de Jean Ray}} et ((\enquote{je
    suis proche de la \emph{crête de Roche Motte}} et \enquote{je suis
    à une altitude supérieure à la \emph{crête de Roche Motte}}) ou
  (\enquote{je suis proche de la \emph{crête de Font Froide}} et
  \enquote{je suis à une altitude supérieure à la \emph{crête de Font
      Froide}}) ou (\enquote{je suis proche de la \emph{crête de Serre
      Chapelle}} et \enquote{je suis à une altitude supérieure à la
    \emph{crête de Serre Chapelle}}) ou (\enquote{je suis proche de la
    \emph{crête des Barres}} et \enquote{je suis à une altitude
    supérieure à la \emph{crête des Barres}}) ou (\enquote{je suis
    proche de la \emph{crête du Petit Puy}} et \enquote{je suis à une
    altitude supérieure à la \emph{crête du Petit Puy}})).
\end{quote}
%
Comme le montre cette longue énumération, \emph{le processus de
  décomposition,} même lorsqu'il est appliqué à un petit
\emph{ensemble d'indices de localisation} (ici deux), peut générer de
nombreux \emph{indices de localisation décomposés,} qu'il est
difficile d'énumérer. Nous sommes cependant convaincus que les apports
permis par cette approche (\eg spatialisation indépendante,
décomposition sémantique) compensent la multiplication des
\emph{indices de localisation.}

\subsection{La phase de \emph{spatialisation}}

Une fois que les \emph{indices de localisation} ont été entièrement
décomposés il est possible de procéder à leur \emph{spatialisation.}
Comme nous expliquions précédemment, la phase de spatialisation
consiste à transformer un \emph{indice de localisation} en une
\emph{zone de localisation compatible.}
%
Il s'agit de la seule étape de notre méthode qui ne modifie pas le
nombre d'objets traité (la phase de \emph{décomposition} les augmente
et celle de \emph{fusion} les réduit) et qui crée un nouveau type de
données, des objets géographiques sous la forme de \emph{zones de
  localisation compatibles.}

Pour construire la \emph{zone de localisation compatible}
correspondant à un \emph{indice de localisation} donné il est
nécessaire d'identifier les positions pour lesquelles \emph{l'indice
  de localisation} est vrai :

\begin{equation}
  \text{\textsf{ZLC}}_i = \{p \ |\ \text{\textsf{S}}_i(p) \wedge p ∈ \text{\textsf{ZIR}}\}
\end{equation}

Avec \(\text{\textsf{ZLC}}_i\) la \emph{zone de localisation
  compatible} pour l'indice de localisation $i$, \(S_i\) la fonction
de \emph{spatialisation} de ce même indice et \(p\) un point
appartenant à la \emph{zone initiale de recherche,}
\(\text{\textsf{ZIR}}\), elle-même un sous-ensemble de
\(\mathbb{R}^2\), défini par le secouriste. L'objectif de la fonction
de \emph{spatialisation} \(S_i\) est de vérifier si le point $p$ est
situé dans la \emph{zone de localisation compatible,} \ie que cette
position valide l'indice de localisation \(i\). Toute la difficulté de
la phase de \emph{spatialisation} tien en la définition de cette
fonction de spatialisation \(S\), dont la méthode de définition sera
présentée dans le \autoref{chap:07}.

Au terme de la phase de \emph{spatialisation} on dispose d'autant de
\emph{zones de localisation} qu'il y a d'indices au terme de la
\emph{phase de décomposition.} Ces différentes \emph{zones de
  localisation} contiennent l'ensemble des informations produites par
notre méthode, mais leur (potentiel) grand nombre rend leur
exploitation directe, sinon impossible délicate, il est donc
nécessaire de les fusionner pour obtenir une seule zone, la ZLP.
 
\subsection{La phase de fusion}

La \emph{phase de fusion} des \emph{zones de localisation} se
décompose en trois étapes, chacune analogue à l'une des étapes de la
\emph{phase de décomposition} (cf. \autoref{fig:methodo_1}). Cette
phase permet de fusionner les \emph{zones de localisation compatibles}
créées lors de la \emph{spatialisation} en une seule, la zone de
localisation probable, correspondant à l'ensemble des indices de
localisation. La particularité de cette phase est que la sémantique de
fusion varie en fonction de l'étape. 



De toutes les étapes de notre méthodologie, la phase de \emph{fusion}
est, de loin, la plus simple. Son objectif est de combiner les
différentes \emph{zones de localisation compatibles} créées lors de la
phase de \emph{spatialisation} en une seule, la \emph{zone de
  localisation probable} (ZLP).

Ce principe sera détaillé dans le \autoref{chap:08}


\subsubsection{La \emph{fusion} des \emph{relations de localisation
    atomiques}}

La première étape de la \emph{phase de fusion} est la \emph{fusion}
des \emph{zones de localisation compatibles} issues de la
\emph{spatialisation.} Cette étape répond donc à l'étape de
\emph{décomposition des relations de localisation} de \emph{la phase
  de décomposition,} \ie que le nombre de \emph{zones de localisation}
en résultant est identique au nombre \emph{d'indices de localisation}
avant la \emph{décomposition des relations de localisation}
(cf. \autoref{fig:methodo_1}).

Conformément au \emph{principe de décomposition,} les \emph{relations
  de localisation} se décomposent en une série de \emph{relations de
  localisation atomiques} liées par des conjonctions. Par conséquent
pour valider une \emph{relation de localisation} il est nécessaire de
valider toutes les \emph{relations de localisation atomiques} qui la
composent. La méthode de \emph{fusion} des \emph{relations de
  localisation atomiques} ne doit donc retourner que les positions
présentes dans toutes les \emph{zones de localisations} qu'elle
fusionne. Comme chaque \emph{zone de localisation} peut être
formalisée comme l'ensemble des points validant une \emph{relation de
  localisation} on peut construire la \emph{zone de localisation
  compatible} correspondant à une \emph{relation de localisation}
composée en intersectant les \emph{zones de localisation compatibles}
correspondant à la \emph{spatialisation} des \emph{relations de
  localisation atomiques.} Ainsi, la \emph{zone de localisation}
correspondant à la fusion de plusieurs \emph{relations de
  localisation} correspond à :

\begin{equation}
  \text{\textsf{ZLC}}_{o_i} = ⋂ \text{\textsf{ZLC}}_{rsa_i}
\end{equation}

Avec \(\text{\textsf{ZLC}}_{o_i}\), la \emph{zone de localisation
  compatible} correspondant à \emph{l'objet de référence} \(oᵢ\) et
\(\text{\textsf{ZLC}}_{rsa_i}\) la \emph{zone de localisation
  compatible} correspondant à la \emph{spatialisation} de la
\emph{relation de localisation atomique} \(i\). Conformément à cette
formalisation, si l'on reprend l'exemple de \emph{l'ensemble d'indices
  de localisation :} \enquote{je suis proche du \emph{Pic de Jean Ray}
  et suis sur une crête}, on obtiendra, après l'étape de \emph{fusion}
des \emph{relations de localisation atomiques,} sept \emph{zones de
  localisation compatibles,} correspondant aux \emph{indices de
  localisation :} \enquote{je suis proche du \emph{Pic de Jean Ray}},
\enquote{je suis sur la \emph{crête de Roche Motte}}, \enquote{je suis
  sur la \emph{crête de Font Froide}}, \enquote{je suis sur la
  \emph{crête de Serre Chapelle}}, \enquote{je suis sur la \emph{crête
    des Barres}}, et \enquote{je suis sur la \emph{crête du Petit
    Puy}}.

\subsubsection{La \emph{fusion} des \emph{objets de référence
    indéfinis}}

La seconde étape de cette phase est la \emph{fusion} des différents
\emph{objets de référence} candidats. Cette étape répond à
\emph{l'étape de décomposition des objets de référence indéfinis.}
Comme nous l'indiquions lors de la présentation de la \emph{phase de
  décomposition,} les étapes de \emph{décomposition des relations de
  localisation} et de décomposition des \emph{objets de référence
  indéfinies} sont interchangeables et il en va de même pour les
étapes de \emph{fusion} leur correspondant. Ainsi cette étape de
fusion pourrait être interchangée avec l'étape, précédemment
présentée, de \emph{fusion des relations de localisation atomiques,}
sans que cela n'impacte le résultat de la modélisation.

La \emph{fusion des objets de référence indéfinis} se distingue des
autres étapes de \emph{fusion} sa forme conjonctive. Comme nous
l'indiquions lors de la présentation de \emph{l'étape de décomposition
  des objets de référence indéfinis,} les différents \emph{objets de
  référence} candidats (pour un même indice de localisation) sont
concurrents. Ainsi pour que \emph{l'indice de localisation spatialisé}
soit valable en une position il est nécessaire qu'il soit vrai pour au
moins un des \emph{objets de référence} et non pour tous (comme c'est
le cas lors de la \emph{fusion} des \emph{relations de localisation
  atomiques}). La \emph{zone de localisation compatible} correspondant
à un \emph{indice de localisation} dont \emph{l'objet de référence}
est indéfini peut, par conséquent, être construite en faisant l'union
de l'ensemble des \emph{zones de localisation compatibles} construites
en \emph{spatialisant} l'indice de localisation pour chaque objet de
référence candidat :

\begin{equation}
  \text{\textsf{ZLC}}_{i} = ⋃_{i \in I} \text{\textsf{ZLC}}_{o_i}
\end{equation}

Avec \(\text{\textsf{ZLC}}_{i}\) la \emph{zone de localisation
  compatible} correspondant à un \emph{indice de localisation} \(i\)
et \(\text{\textsf{ZLC}}_{o_i}\) la \emph{zone de localisation
  compatible} pour l'objet \(i\) de l'indice de localisation \(i\).
Appliqué à l'exemple précédent on obtient, au terme de cette étape de
\emph{fusion,} deux \emph{zones de localisation compatibles.} La
première (inchangée depuis la \emph{spatialisation}) correspond à
\emph{l'indice de localisation} \enquote{je suis proche du \emph{Pic
    de Jean Ray}}. Comme \emph{l'objet de référence} est unique aucune
\emph{fusion} n'est à effectuer, contrairement au second \emph{indice
  de localisation,} \enquote{je suis sur une crête} qui correspond à
l'union des \emph{zones de localisation compatibles} correspondant à
chacune des une des crêtes candidates.

\subsubsection{La \emph{fusion} des \emph{indices de localisation}}

La dernière étape de \emph{la phase de fusion} est celle de la
\emph{fusion} des \emph{indices de localisation} en vue de construire
la \emph{zone de localisation probable,} \ie le résultat final de
notre méthode. Contrairement aux \emph{zones de localisation
  compatibles,} la \emph{zone de localisation probable} est toujours
unique, même si elle peut être fragmentée ou vide (mais ce cas traduit
une erreur dans la description de la position, dans sa saisie ou sa
\emph{spatialisation}).

Comme lors de l'étape de \emph{fusion} des \emph{relations de
  localisation atomiques,} les différentes \emph{zones de localisation
  compatibles} sont \emph{fusionnées} à l'aide d'une intersection
ensembliste. En effet, la \emph{zone de localisation probable} est
définie comme l'ensemble des positions appartenant à toutes les
\emph{zones de localisation compatibles.} On peut donc exprimer la
zone de localisation probable de la manière suivante :

\begin{equation}
  \text{\textsf{ZLP}}_I = ⋂_{i \in I} \text{\textsf{ZLC}}ᵢ
\end{equation}

Avec \textsf{ZLP} la \emph{zone de localisation probable,} \(I\),
l'ensemble des \emph{indices de localisation} pris en compte, \(i\) un
\emph{indice de localisation} et \textsf{ZLP} un \emph{zone de
  localisation compatible.} Ainsi, dans notre exemple, la \emph{zone
  de localisation probable} correspond à \emph{l'intersection} de deux
\emph{zones de localisation compatibles,} celle construite à partir de
\emph{l'indice de localisation} \enquote{je suis proche du \emph{Pic
    de Jean Ray}} et celle construite à partir de \emph{l'indice de
  localisation} \enquote{je suis sur une crête}.

\begin{landscape}
  \begin{figure}[H]
    \centering
    \begin{tikzpicture}
  \tikzset{
    acc/.style={decorate,decoration={brace,raise=0cm,amplitude=.1cm}},
    accm/.style={acc,decoration={mirror}},
    acc3/.style={acc,decoration={amplitude=.2cm}},
    acc3m/.style={acc3,decoration={mirror}}
  }
  
  %\draw[step=.5,black,thin] (0,0) grid (22,-14);

%% Matrices
\node[font=\bfseries, baseline] (I) at (0,0) {$I$};

\matrix [matrix of math nodes,row sep=0.1cm,column sep=0.1cm,
anchor=south west, nodes={anchor=base, baseline}] (is) at
([xshift=1.5cm]I.north east)
{i_1\\i_2\\\vdots{}\\i_n\\};

\matrix [matrix of math nodes,row sep=0.1cm,column sep=0.1cm,
anchor=south west, nodes={baseline}] (is_o) at
([xshift=1.5cm,yshift=.5cm]is-1-1.east) {o_1\\o_2\\\vdots{}\\o_n\\};

\matrix [matrix of math nodes,row sep=0.1cm,column sep=1.5cm,
anchor=south west, nodes={minimum height=5mm}] (is_o_r) at ([xshift=1.5cm,yshift=.5cm]is_o-1-1.east)
{
  \text{RSA}_1&\text{ZLC}_{rsa1}\\
  \text{RSA}_2&\text{ZLC}_{rsa_2}\\
  \vdots{}&\vdots{}\\
  \text{RSA}_n&\text{ZLC}_{rsa_n}\\};

\matrix [matrix of math nodes,row sep=0.1cm,column sep=0.1cm,
anchor=west, nodes={baseline}] (zlb) at ([xshift=1.5cm]is_o_r.east  |- is_o)
{ZLC_{o_1}\\ZLC_{o_2}\\\vdots{}\\ZLC_{o_n}\\};

\matrix [matrix of math nodes,row sep=0.1cm,column sep=0.1cm,
anchor=west, nodes={anchor=base, baseline}] (zlc) at ([xshift=1.5cm]zlb.east  |- is)
{ZLC_1\\ZLC_2\\\vdots{}\\ZLC_n\\};

\node [anchor=west] (zlp) at ([xshift=1.5cm]zlc.east |- I)
{ZLP};


%% Accolades 
\begin{scope}
  \draw[acc3m] (is.north west) -- (is.south west);
  \draw[acc3m] (is_o.north west) -- (is_o.south west);
  \draw[acc3m] (is_o_r.north west) -- (is_o_r.south west);

  \draw[acc] (is-1-1.north east) -- (is-1-1.south east);
  \draw[acc] (is_o-1-1.north east) -- (is_o-1-1.south east);
  \draw[acc] (is_o_r-1-1.north east) -- (is_o_r-1-1.south east);

  \draw[accm] (is_o_r-1-2.north west) -- (is_o_r-1-2.south west);
  \draw[accm] (zlb-1-1.north west) -- (zlb-1-1.south west);
  \draw[accm] (zlc-1-1.north west) -- (zlc-1-1.south west);

  \draw[acc3] (is_o_r.north east) -- (is_o_r.south east);
  \draw[acc3] (zlb.north east) -- (zlb.south east);
  \draw[acc3] (zlc.north east) -- (zlc.south east);
\end{scope}


\begin{scope}
  \draw[->, black] (I.east) -- (is.west) node[pos=0,below] {1} node[pos=1,below]{1..*};

  \draw[->, black] (is-1-1.east) -- (is_o.west) node[pos=0,below] {1} node[pos=1,below]{1..*}; 
  
  \draw[->, black] (is_o-1-1.east) --
  (is_o_r.west) node[pos=0,below]
  {1} node[pos=1,below]{1..*}; 
  
  \draw[->, black] (is_o_r-1-1.east) -- (is_o_r-1-2.west) node[pos=0,below]
  {1} node[pos=1,below]{1}; 
  
  \draw[->, black] (is_o_r.east) -- (zlb-1-1.west) node[pos=0,below] {1..*} node[pos=1,below]{1}; 
  
  \draw[->, black]
  (zlb.east) -- (zlc-1-1.west) node[pos=0,below] {1..*} node[pos=1,below]{1}; 
  
  \draw[->, black] (zlc.east) -- (zlp.west) node[pos=0,below] {1..*} node[pos=1,below]{1}; 
\end{scope}


\begin{scope}
    \draw[->, black, dotted] ([xshift=.5cm]I.east) -- ([xshift=-.5cm]zlp.west) node[pos=0,below]
  {1} node[pos=1,below]{1}; 

  \draw[->, black, dotted] ([xshift=.5cm]is.east) -- ([xshift=-.5cm]zlc.west) node[pos=0,below]
  {1} node[pos=1,below]{1}; 

  \draw[->, black, dotted] ([xshift=.5cm]is_o.east) -- ([xshift=-.5cm]zlb.west) node[pos=0,below]
  {1} node[pos=1,below]{1}; 
\end{scope}

%% XX

\begin{scope}[below of=I]
  \draw[acc3m] (0,0) --++ (8,0) node[pos=.5, yshift=-.5cm] {\emph{Décomposition}};
  \draw[acc3m] (8,0) --++ (3,0) node[pos=.5, yshift=-.5cm] {\emph{Spatialisation}};
  \draw[acc3m] (13.4,0) --++ (2,0) node[pos=.5, yshift=-.5cm] {\emph{Fusion}};
\end{scope}

\end{tikzpicture}
    \caption{Méthodologie}
    \label{fig:methodo_1}
  \end{figure}
\end{landscape}


%%% Local Variables:
%%% mode: latex
%%% TeX-master: "../../../../main"
%%% End:
