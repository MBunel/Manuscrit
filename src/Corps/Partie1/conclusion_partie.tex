Comme nous l'avons montré dans le premier chapitre de cette thèse, le
secours en montagne est une activité essentielle dans la pratique des
sports de montagne et bien que les opérations de secours soit
efficaces et ont permi réduire considérablement la mortalié dans les
sports en mointagne tout en accompagnant l'explosion des pratiques
touristiques en montagne, il s'agit d'opérations critiques, mobilisant
un personel peu nomnbreux et disposant de moyens, certes important,
mais pas illimités. Si les opérations de secours sont maitrisées grâce
à l'expertise technique et l'expérience des secouristes, la première
partie des opérations de qsecours, à savboir la phase de localisation
de la victime est toujours une étape critique. En effet, si dans la
majorité des cas les secouristes n'éprouvent que peu de difficultées à
localiser les victimes et encore moins depuis la généralisation des
outils de géolocatisation, type GendLoc, il reste des situations où
les victimes sont difficiles à localiser, ce qui conduit à une perte
de temps pouvant avoir des effets dramatiques. La localisation des
victimes est donc une étape critique.

Le projet de recherche Choucas propose de développer des solutions
assistant les secouristes durant cette phase de localisation, dans le
but de faciliter la localisation des victimes et ainsi gagner du temps
et limiter le temps d'occupations de moyens humains et techniques
(hélicoptères d'interventions) limités. Pour ce faire quatre axes de
travail principaux ont étés identifiés
(\autoref{tab:synthese_objectifs_choucas}). Le premier d'entre-eux
consiste à enrichir les données dont disposent les secouristes,
notamment en exploitant les corpus de récits de randonnées mis en
ligne par les alpinistes. Le second de ces objectifs est de travailler
à l'intégration de ces nouvelles données avec celles dont disposent
déjà les secouristes, mais également à les harmoniser. Le troisième
objectif consiste à travailler à la géovisualisation des Enfin, le
dernier objectif, auquel notre thèse est rattachée, consiste à
développer des modèles de raisonnement, permettant d'identifier la
position correspondant à une description de position.

Dans le second chapitre nous avons expliqué que la transformation
d'une position décrite en une zone de cordonnées connues ne pouvait ce
faire qu'en deux étapes. Une première dite de \emph{spatialisation,}
où les différents \emph{indices de localisation} qui composent une
alerte sont transformés en des \emph{zones de localisation
  compatibles} et une seconde étape de \emph{fusion} où les
différentes \emph{zones de localisation compatibles} d'une même alerte
sont combinée pour former la \emph{zone de localisation probable} de
l'alerte, \ie la zone où tous les \emph{indices de localisation} sont
vérifiés. Nous avons identifié plusieurs verrous scientifiques à la
réalisation de ces deux objectifs.  Le premier d'entre-eux est la
définition d'une méthode de \emph{spatialisation} qui soit adaptée aux
différents \emph{indices de localisation,} \ie qui soit capable de
construire une \emph{zone de localisation compatible} pour des
\emph{indices de localisation} aussi variés que : \enquote{Je sur une
  route} ou \enquote{je vois une forêt}.  Un autre problème est que la
sémantique des \emph{indices de localisation} peut être difficile à
identifier. En effet, comme nous l'avons expliqué dans le chapitre 2
et dans la seconde partie du chapitre 3, ces derniers peuvent être
fortement \emph{imprécis.}  \emph{L'imprécision} des \emph{indices de
  localisation} implique de développer une méthode de spatialisation
qui soit capable de la prendre en compte dans le but d'améliorer la
qualité des \emph{zones de localisation compatibles,} c'est l'objet du
second verrou scientifique de cette thèse.  Un troisième point
essentiel est la notion \emph{d'incertitude}, également abordée dans
les chapitres 2 et 3. En effet, les secouristes doivent pouvoir
énoncer un doute sur la véracité des \emph{indices de localisation}
qui leur sont transmis. La principale difficulté posée par la prise en
compte de \emph{l'incertitude} est que la méthode développée devra
cohabiter avec la méthode de prise en compte de \emph{l'imprécision}
et, plus généralement, avec les méthodes de \emph{spatialisation.}  De
plus nous devons développer une méthode de \emph{fusion} des
\emph{zones de localisation compatibles.} Cette dernière doit pouvoir
cohabiter avec les méthodes développées pour prendre en compte
\emph{l'imprécision} et \emph{l'incertitude.}  Enfin, nous souhaitons
proposer une méthode permettant d'évaluer les \emph{zones de
  localisation probables} construites.


% Transition partie 2


%%% Local Variables:
%%% mode: latex
%%% TeX-master: "../../main"
%%% End: