Comme nous l'expliquions lors de la présentation du projet Choucas,
dans le chapitre précédent (voir \ref{subsec:1-2-3}, page
\pageref{subsec:1-2-3}), notre thèse s'inscrit dans une réflexion plus
générale, autour de l'aide à la localisation de personnes perdues en
montagne. Plus spécifiquement, notre travail consiste à élaborer une
solution permettant de délimiter la zone correspondant à une
description orale de position.

La réalisation de cet objectif principal nécessite de répondre à plu

On trouvera une synthèse des objectifs scientifiques, ainsi que de
leurs verrous et des contributions attendues dans le
\autoref{tab:synthese_objectifs} page
\pageref{tab:synthese_objectifs}.

\subsection{La spatialisation d'indices de localisation}
\label{subsec:2-1-1}

% int
Le premier objectif de cette thèse est de développer une méthode
permettant d'identifier une position à partir de sa
description.

% Présentation de l'objectif
Comme nous l'expliquions précédement, notre thèse s'inscrit dans un
axe de recherche du projet Choucas dont l'objectif est de faciliter
les secouristes à identifier une position décrite.
%
Cette tâche nécessite d'arriver à définir des méthodes permettant de
construire une zone à partir de descriptions, telles que \enquote{je
  suis sous une route.} Cette tâche nécessite donc de travailler à
l'identification de la sémantique des termes utilisés pour décrire une
possition, puis d'en proposer une modélisation, à même de pouvoir
systématiser la construction d'une \enquote{zone} à partir d'une
description.

% Pas nécessaire de faire de l'analyse du langage naturel
Comme nous le mentionnons dans le chapitre précédent (\todo{mettre res
  du passage où j'explique cet aspect dans le chapitre 1}), notre
objectif n'est pas de déveloper une solution se sustituant aux
secouristes, mais de leur proposer une solution s'intégrant dans leur
processus métier actuel. Cela implique que nous pouvons toujours
compter sur la présence du secouriste pour effectuer certaines
actions. Parmi celles-ci, l'écoute et la compréhension du discours du
requérant. En effet, il n'est pas nécessaire, voire souhaitable
\footnote{Tout du moins du point de vue des pprofessionels du
  secours.}, d'automatiser entièrement le processus de localisation de
la victime. Un tel objectif nécessiterait par ailleurs de développer
des solutions d'analyse automatisée du langage parlé, en vue d'en
extraire les relations de localisation.

Cette étape n'est pas nécessaire dans notre cas, puisque nous pouvons
confier cette tache de \enquote{sémantisation} de la description orale
d'une position au secouriste. 

Notre travail se limitera donc à identifier les \enquote{types} de
descriptions utilisées dans notre contexte, de les exprimer dans un
vocabulaire controlé, puis d'en proposer une modélisation a même d'en
permettre la spatialisation \todo{Attention terme non défini, attendre
  chapitre 4 avant de l'introduire}.

\subsubsection{Contexte scientifique}

La question de l'indentification de positions décrites à fait l'objet
de différents travaux, aux visées applicatives diverses.


% Réflexions autour de la sémantique des relations spatiales

Enfin, dans une thématique plus proche de notre travail,  

Naive geography

Mateh

Moncla



Vasardini
Bloch
Brézil icc\autocite{Egenhofer1995}

vandeloise, kratch, augenarge, borillio 

\paragraph{Formalisation des relations spatiales}
\autocite{Vandeloise1986,Borillo1998,Bateman2010}
\autocite{Kracht2002}
\autocite{Aurnague1993}
\autocite{Mark1999}
\autocite{Freksa2018}
\autocite{Carlson2005}
\autocite{Lang1991}

% vérifier
\autocite{Hois2008}
\autocite{Frank1992}
\autocite{Moratz2008,Cohn2001}
\autocite{Kordjamshidi2012}
\autocite{Matsakis2010}
\autocite{Frank1997}
\autocite{Talmy2005}
\autocite{Freeman1975}
\autocite{Zelinsky-Wibbelt1993}
\autocite{Carlson2004}
\autocite{Gui-Wu2009}

\paragraph{Applications}
\autocite{Xu2007a,Xu2007,Xu2006}
\autocite{Wolter2018}
\autocite{Dittrich2015}
\autocite{Vanegas2011}
\autocite{Du2016}
\autocite{Hornsby2009}
\autocite{Mathet2000}
\autocite{Hudelot2008a}
\autocite{Bloch1996}
\autocite{Hudelot2008}
\autocite{Levit2007}
\autocite{Evans2008}
\autocite{Takemura2012}
\autocite{Shariff1998}
\autocite{Skubic2004}
\autocite{Hall2015}
\autocite{Leopold2015}
\autocite{Denis1997}

Purves "geographic information revieal''

\subsubsection{Verrous}

\paragraph{Peu de données et Faible recoupement}

présentés en détail dans le \autoref{chap:05}

Un second problème, intimement lié au manque de données, est leur
faible recoupement. En effet, les enregistrements d'alerte dont nous
disposons ne concerent jamais les mêmes zones. Et les mêmes points


\paragraph{Disparité des descriptions de localisation}

Deux personnes ne décrivent pas une position de la même façon

\paragraph{Descriptions non exhaustives}

Pas de présuppositions sur ce qui est dit.

le requérant ne dit pas tout

-> complexifie le raisonnement

\subsubsection{Apports}

\subsection{Prise en compte de l'imprécision}
\label{subsec:2-1-2}

La question de la spatialisation nécessiter cependant d'en s'en poser
une seconde, celle de la prise en compte de l'imprécision. En effet,
les positions que nous cherchons à spatialiser possèdent la
particularité d'être exprimées en langage naturel. Orn bien que nous
ne travaillons pas directement à l'interprétation du langage naturel,
\ie que nous ene faisons pas de TAL, nous sommes tout de même
confrontés à certaines de ces limites.

La principale d'entre-elle est la question de l'imprécison du langage
naturel. En effet, le langage naturel n'est pas un outil idéal pour
exprimer clairement des propositions, celle-ci sont toujours
soumisses à l'interpréatation et donc imprécises. Il en va
nécessairement de même pour les propositions décrivants dune ou des
positions? Ces dernières s'expriment en langage natuurel et son donc
soumisses à l'iumprécisions.

Dans les faits ,cela implique qu'une même phrase, un même mot,
peutvent -être utilisés pour décrire des localisations dans l'espace
différentes, voir même totalement incompatibles.

Cette nature du langage naturel peut poser plusieurs soucis. D'une
part il est impossible d'utiliser la langue comme un outil de
raisonement précis. C'est d'ailleurs ce constat qui a conduit les
philosophes analytiques, tels que Fredge ou Russel à travailler à la
formalisation de la logique, outil à même d'affranchir la réflexion
des limites du langage naturel.



\subsubsection{Contexte scientifique}

Travail sur le flou
Russell
Varzi
Zadeh
Dilo ?


\subsubsection{Verrous}

\paragraph{Quantification de l'imprécision}

\subsection{Prise en compte du jugement du secouriste}
\label{subsec:2-1-3}

\subsubsection{Contexte scientifique}

Travaux sur l'incertitude

Ana-Maria

\subsubsection{Verrous}

\paragraph{Quantification de l'incertitude}

\paragraph{inscription dans le contexte métier}

% Le secouriste doir pouvoir manipuler assez 'facilement' ces notions

\subsubsection{Apports envisagés}

\subsection{Agrégation d'indices}
\label{subsec:2-1-4}

La structure des alertes fait apparaitre un autre problème. En effet,
comme nous l'avon vu lors de présentation du fil rouge (\ref{} page
\pageref{} \todo{ajouter ref présentation du fil rouge}) les
requérants décrivent généralement leur position à l'aide de
différentes phrases, décrivant autant d'aspects de leur position
actuelle. Chacune de ces descriptions, constitue, ce que nous appelons
un \emph{indice de localisation.} Une description de localisation est
donc composée d'au moins un, généralement plus, \emph{indice de
  localisation}. Ce sont ces derniers que nous cherchons à
\emph{spatialiser.} Toutefois, la spatialisation indivoduelle de ces
indices de localisation, ne donne qu'une indication partielle sur la
zone de localisation de la vicitime. Il est donc nécessaire de
combiner la spatilisation des différents indices de localisation, afin
de proposer la meilleure \emph{zone de localisation compatible}
possible.

La méthode développée à cet effet, doit conserver les caractéristiques
que nous voulons voir validées par la méthode de spatilisation
présentée ci-dessus. C'est-à-dire, que la méthode d'aggrégation doit
prendre en compte la nature imprécise des résultats de la méthode de
spatilisation des indices. De plus, elle doit être a-même de prendre
en compte l'évaluation de la plausibilité de l'indice faite par le
secouriste.

\subsubsection{Contexte scientifique}

\tdi{je ne sais quoi mettre ici, je ne suis pas sur d'avoir des
  références sur ce sujet.}

\subsubsection{Verrous}

Le processus d'agrégation des indices que nous souhaitons développer
correspond à un stade éloigné de notre méthodologie, \ie qu'il est
impacté par les choix effectués en amont. C'est notamment le cas de
ceux portant sur la spatilisation des différents indices.

\paragraph{Compatibilité avec les méthodes de spatialisation}

Comme nous l'expliquions lors de la présentation des objectifs
\ref{subsec:2-1-2} et \ref{subsec:2-1-3}, nous souhaitons que la prise
en compte de \emph{l'imprécision du langage naturel} et de la
\emph{plausibilité des indices} soit intrinsèque à la méthode de
spatialisation des \emph{indices de localisation} (objectif
\ref{subsec:2-1-1}). Il est par conséquent nécessaire que la méthode
d'agrégation des indices soit à même de conserver ces caractérisques.
% 
Or, ces deux composantes sont, par définition, variables. Deux indices
issus d'une même alerte peuvent avoir une plausibilité considérée
comme différente par le secouriste. Ce peut être, par exemple, le cas
si un requérant décrit sa position comme étant : \enquote{proche d'un
  chalet et probablement au sud du sommet}. Dans ce cas le premier
indice de localisation : \enquote{je suis proche d'un chalet}, serait
surement, considéré comme \emph{plus plausible} que le second :
\enquote{je suis probablement au sud sommet}. La méthode d'agrégation
des indices de localisation doit donc être capable de prendre en
considération cette différence entre les deux indices, par exemple les
pondérant. De même, la différence de précision entre deux indices doit
être prise en compte lors du processus d'agrégation.

\paragraph{\texttt{titre ??}}
%\paragraph{Contexte différent}

% La commutativité de l'agrégation
% Différence des indices
% Traitement 'indépendant' de la localisation

Un autre point important est que le processus d'agrégation sera
potentiellement utilisé plusieurs fois lors du traitement d'une
alerte. En effet, les \emph{indices de localisation} sont donnés par
le requérant, au fur et à mesure du processus de localisation, parfois
en répose aux questions des secouristes. L'ensemble des indices de
localisation modélisés est donc amené à évoluer, comme la \emph{zone
  de localisation compatible} qui résulte de leur agrégation. La
méthode d’agrégation doit donc prendre cet aspect en considération et
nous devons veiller à ce que les résultats de ce processus ne soient
pas sensibles

\tdi{revoir}

\paragraph{La gestion des conflits}

\tdi{C'est pertinent ? Il n'y a pas vraiment de solution dans ma
  méthodo pour les prendre en compte.}

Un second verrou, fortement lié à la question de la modélisation de la
plausibilité est celui de la gestion des conflits entre indices. Il
n'est, en effet, pas exclu que deux indices d'une même alerte se
contredisent, rendant dès lors impossible la construction de la
\emph{zone de localisation probable.} La prise en compte de la
plausibilité des indices permet, en partie, de contourner ce
problème. Cependant, l'estimation de la plausibilité des indices reste
à la charge du secouriste.



\subsection{Évaluation des résulats}
\label{subsec:2-1-5}

Enfin, notre dernier objectif est de proposer une méthode d'évaluation
de la (ou des) \emph{zone(s) de localisation probable} \todo{Veiller à
  ce que le terme soit bien défini auparavant} construite(s). Cet
objectif nécessite définir des critères d'évaluation pertinants et
adaptés au contexye appliquatif de ce travail.

\subsubsection{Contexte scientifique}

\subsubsection{Verrous}

\paragraph{Pas d'évaluation supervisée}
Impossibilité de connaitre la réponse lors de l'évaluation


% Tableau synthétique
\begin{table}[h]
  \centering
  \begin{tabular}{p{.2\textheight}>{\small}L{.35\textheight}>{\small}L{.35\textheight}} \toprule
\multicolumn{1}{c}{\bfseries Objectif scientifique} &
\multicolumn{1}{c}{\normalsize\bfseries Verrous} & \multicolumn{1}{c}{\normalsize\bfseries
Apports envisagés} \\ \midrule
% Sémantique des relations spatialesss
  \addlinespace
  \nameref{subsec:2-1-1}
{\par\footnotesize\hspace{.25cm}$\longrightarrow$~Chapitre
\ref{chap:07}} & \begin{minipage}[t]{.35\textheight}
    \begin{itemize}
    \item Variation sémantique des \emph{relations de localisation}
    \item Faible redondance des \emph{indices de localisation}
    \end{itemize}
  \end{minipage} & \begin{minipage}[t]{.35\textheight}
    \begin{itemize}
    \item Recensement :
      \begin{itemize}
      \item des \emph{relations de localisation}
      \item des \emph{objets de référence}
      \end{itemize}
    \item Identification de la sémantique des \emph{relations de
        localisation}
    \item Définition d'une méthode de \emph{spatialisation}
    \end{itemize}
  \end{minipage} \\
  %\addlinespace[.5cm]
  %
  % Prise en compte imprécision
  \nameref{subsec:2-1-2}
{\par\footnotesize\hspace{.25cm}$\longrightarrow$~Chapitre
\ref{chap:07}} & \begin{minipage}[t]{.35\textheight}
    \begin{itemize}
    \item Intégration de \emph{l'imprécision} à la \emph{spatialisation}
    \item Quantification de \emph{l'imprécision} des \emph{relations
        de localisation}
    \end{itemize}
  \end{minipage} & \begin{minipage}[t]{.35\textheight}
    \begin{itemize}
    \item Définition d'une méthodes :
      \begin{itemize}
      \item de prise en compte de \emph{l’imprécision} des
        \emph{relations de localisation}
      \item de quantification de \emph{l'imprécision} des
        \emph{relations de localisation}
      \end{itemize}
    \end{itemize}
  \end{minipage} \\
  %\addlinespace[.5cm]
  %
  % Incertitude
  \nameref{subsec:2-1-3}
{\par\footnotesize\hspace{.25cm}$\longrightarrow$~Chapitre
\ref{chap:08}} & \begin{minipage}[t]{.35\textheight}
    \begin{itemize}
    \item Intégration de \emph{l'incertitude} à la
      \emph{spatialisation}
    \item Quantification de \emph{l'incertitude} des \emph{indices de
        localisation}
    \end{itemize}
  \end{minipage}& \begin{minipage}[t]{.35\textheight}
    \begin{itemize}
    \item Définition de méthodes :
      \begin{itemize}
      \item de prise en compte de \emph{l’incertitude} des
        \emph{indices de localisation}
      \item de quantification de \emph{l'incertitude} des
        \emph{indices de localisation}
      \end{itemize}
    \end{itemize}
  \end{minipage} \\
  %\addlinespace[.5cm]
  %
   \nameref{subsec:2-1-4}
{\par\footnotesize\hspace{.25cm}$\longrightarrow$~Chapitre
\ref{chap:08}} & \begin{minipage}[t]{.35\textheight}
    \begin{itemize}
    \item Compatibilité avec la spatialisation
    \item Gestion des conflits
    \end{itemize}
  \end{minipage}& \begin{minipage}[t]{.35\textheight}
    \begin{itemize}
    \item Définition d'une méthode de fusion
    \end{itemize}
  \end{minipage} \\
  %\addlinespace[.5cm]
  %
  \nameref{subsec:2-1-5}
{\par\footnotesize\hspace{.25cm}$\longrightarrow$~Chapitre
\ref{chap:11}} & \begin{minipage}[t]{.35\textheight} \small
    \begin{itemize}
    \item Risque d'évaluer les \emph{indices}.
    \item Intégration dans le contexte métier.
    \end{itemize}
  \end{minipage}& \begin{minipage}[t]{.35\textheight}
    \begin{itemize}
    \item Développement d'une méthode d'évaluation des \emph{zones de
        localisation}
    \end{itemize}
  \end{minipage}\\
  \bottomrule
\end{tabular}

  \caption{Synthèse des verrous et des apports attendus pour chaque
    objectif scientifique de la thèse}
  \label{tab:synthese_objectifs}
\end{table}

%%% Local Variables:
%%% mode: latex
%%% TeX-master: "../../../../main"
%%% End:
