\dict{\onto[orl]{ADroiteDe}}{La cible est à droite de l’objet de
  référence. Ce concept nécessite de connaître l’orientation de la
  cible}{\enquote{sur la droite quand on regarde le haut de la
    station.}}%
\dict{\onto[orl]{AGaucheDe}}{La cible est à gauche de l’objet de
  référence. Ce concept nécessite de connaître l’orientation de la
  cible}{\enquote{Juste sur la gauche, il y a une autre dent, je ne
    sais pas comment elle s’appelle.}}%
\dict{\onto[orl]{AvoirASaDroite}}{La cible est munie d’un référentiel
  des directions intrinsèque : elle a une gauche et une droite (et un
  avant, un arrière), par exemple parce que c’est un être animé muni
  d’un visage. Dans ce référentiel des directions, et plus précisément
  sur l’axe gauche-droite de ce référentiel, le site est situé à
  droite de la cible (donc la cible a le site à sa droite, au sens
  commun de l’expression). Le référentiel des directions doit être
  modélisé explicitement (à préciser : comment)...}{}%
\dict{\onto[orl]{AvoirASaGauche}}{La cible est munie d’un référentiel
  des directions intrinsèque : elle a une gauche et une droite (et un
  avant, un arrière), par exemple parce que c’est un être animé muni
  d’un visage. Dans ce référentiel des directions, et plus précisément
  sur l’axe gauche-droite de ce référentiel, le site est situé à
  gauche de la cible (donc la cible a le site à sa gauche, au sens
  commun de l’expression). Le référentiel des directions doit être
  modélisé explicitement (à préciser : comment)...}{}%
\dict{\onto[orl]{AvoirDerriereSoit}}{La cible est munie d’un
  référentiel des directions intrinsèque : elle a un avant et un
  arrière, (et une gauche, une droite), par exemple parce que c’est un
  être animé muni d’un visage. Dans ce référentiel des directions, et
  plus précisément sur l’axe avant-arrière de ce référentiel, le site
  est situé derrière la cible (donc la cible a le site derrière elle,
  au sens commun de l’expression). Le référentiel des directions doit
  être modélisé explicitement (à préciser : comment)...}{}%
\dict{\onto[orl]{AvoirDevantSoit}}{La cible est munie d’un référentiel
  des directions intrinsèque : elle a un avant et un arrière, (et une
  gauche, une droite), par exemple parce que c’est un être animé muni
  d’un visage. Dans ce référentiel des directions, et plus précisément
  sur l’axe avant-arrière de ce référentiel, le site est situé devant
  la cible (donc la cible a le site devant elle, au sens commun de
  l’expression). Le référentiel des directions doit être modélisé
  explicitement (à préciser : comment)...}{\enquote{devant moi j’ai la
    plaine, et à gauche j’ai quand même des sapins.}}%
\dict{\onto[orl]{A\-Extremite\-De}}{La \emph{cible} est située à
  l'extrémité du \emph{site}. Ce qui impose une relation de
  contenance.}{\enquote{Il est à l'extrémité ouest d'une espèce de
    terrasse en béton.}}%
\dict{\onto[orl]{A\-La\-Frontiere\-De}}{Le \emph{site} a une emprise
  spatiale linéaire ou surfacique. La \emph{cible} est
  \enquote{proche} de ses frontières. La \emph{cible} peut
  indifféremment se situer à l’intérieur
  (\onto[orl]{Dans\-Pla\-ni\-mé\-tri\-que}) du \emph{site}, ou pas. Si
  elle est à l’intérieur, elle est alors dans le complémentaire de la
  zone de l’espace qui vérifier la relation \onto[orl]{Au\-Milieu\-De}
  (mais \onto[orl]{A\-La\-Frontiere\-De} et \onto[orl]{Au\-Milieu\-De}
  ne couvrent pas nécessairement tout le \emph{site} à eux deux). Très
  probablement, la distance à considérer par rapport aux frontières
  dépend de la nature du \emph{site} ou de sa taille.}{\enquote{Je
    suis un peu à la sortie de la forêt.}}%
\dict{\onto[orl]{CibleVoitSite}}{Le \emph{site} est visible depuis la
  \emph{cible}}.{\enquote{Je suis vraiment en montagne, je vois les
    plaines, je vois un grand découvert devant moi.}}%
\dict{\onto[orl]{DansLaPartieBasseDe}}{La \emph{cible} est située dans
  la partie basse du \emph{site}.}{}%
\dict{\onto[orl]{DansLaPartieEstDe}}{La \emph{cible} se situe dans le
  \emph{site} et dans sa partie est.}{}%
\dict{\onto[orl]{DansLaPartieOuestDe}}{La \emph{cible} se situe dans
  le \emph{site} et dans sa partie ouest.}{\enquote{Il est à
    l'extrémité ouest d'une espèce de terrasse en béton.}}%
\dict{\onto[orl]{DansLaPartieDeXLaPlusProcheDeY}}{Le \emph{site} 1
  ($X$) est un surfacique (en 2,5D) le \emph{site} 2 ($Y$) est situé
  en dehors du \emph{site} 1 et permet de délimiter une portion du
  \emph{site} 1 qui est \enquote{la partie la plus proche} du
  \emph{site} 2. La \emph{cible} est alors située dans le \emph{site}
  1 (\onto[orl]{Dans\-Planimétrique}), dans cette \enquote{partie la
    plus proche} du \emph{site} 2.}{\enquote{Dans la Combe de la
    Glière côté télésiège.}}%
\dict{\onto[orl]{DansLaPartieHauteDe}}{La \emph{cible} est située dans
  la partie haute du \emph{site}. Cela suppose de pouvoir définir une
  \enquote{partie haute} du \emph{site}.}{}%
\dict{\onto[orl]{DansLaPartieNordDe}}{La \emph{cible} se situe dans le
  \emph{site} et dans sa partie nord.}{}%
\dict{\onto[orl]{DansLaPartieSudDe}}{La \emph{cible} se situe dans le
  \emph{site} et dans se partie sud.}{}%
\dict{\onto[orl]{Dans\-Planimétrique}}{}{\enquote{Non, on est en
    forêt.}}%
\dict{\onto[orl]{Proximal}}{La \emph{cible} est dans le \emph{site} ou
  à proximité immédiate, qui peut être une proximité
  fonctionnelle.}{\enquote{Elle est à 50m, je suis sur un promontoire
    car le réseau ne passait pas.}}%
\dict{\onto[orl]{Si\-tue\-Sur\-Iti\-ne\-rai\-re\-Ou\-Re\-seau\-Sup\-port}}{La
  \emph{cible} se situe sur un réseau ou un itinéraire. Le
  \enquote{sur} a ici un sens fonctionnel. Le \emph{site} peut être un
  élément de réseau au sens large : réseau parcourable à pied, réseau
  de pistes de ski, voies d’escalade ou d’alpinisme, réseau
  hydrographique (\eg en kayak ou canyoning), itinéraire de ski de
  randonnée, \emph{etc.} Aucun apriori n’est considéré sur la
  modélisation géométrique du réseau dans les données. Il n’y a pas
  nécessairement de connexion topologique entre la \emph{cible} et le
  \emph{site} : la \emph{cible} peut se trouver de fait légèrement
  éloignée de l’élément de réseau (\eg cas d’une aire de pique-nique
  accesible depuis le réseau de randonnée pédestre).}{\enquote{On est
    sur le GR 54.}}%
\dict{\onto[orl]}{DansLaDirectionDe}{}{}%
\dict{\onto[orl]{AEstDe}}{La cible se situe globalement à l’est du
  site, sans préciser si elle est à dans la partie est du site ou
  disjointe du site et à l’est de lui.}{}%
\dict{\onto[orl]{AEstDeExterne}}{La cible se situe globalement à l’est
  et est disjointe du site.}{}%
\dict{\onto[orl]{DansLaPartieEstDe}}{La cible se situe globalement à
  l’est et est disjointe du site.}{}%
\dict{\onto[orl]{AOuestDe}}{La cible se situe globalement à l’ouest du
  site, sans préciser si elle est à dans la partie ouest du site ou
  disjointe du site et à son ouest.}{\enquote{oui, et là on est à
    l'Ouest.}}%
\dict{\onto[orl]{AOuestDeExterne}}{La cible se situe globalement à
  l’ouest et est disjointe du site.}{\enquote{on est versant ouest,
    côté vercors intérieur}}%
\dict{\onto[orl]{DansLaPartieOuestDe}}{La cible se situe dans le site
  et dans se partie ouest.}{\enquote{il est à l'extrémité ouest d'une
    espèce de terrasse en béton.}}%
\dict{\onto[orl]{AuNordDe}}{La cible se situe globalement au nord du
  site, sans préciser si elle est à dans la partie nord du site ou
  disjointe du site et à son nord.}{}%
\dict{\onto[orl]{AuNordDeExterne}}{La cible se situe globalement au
  nord et est disjointe du site.}{\enquote{non, côté Nord (du Pas de
    la Ville), pardon.}}%
\dict{\onto[orl]{DansLaPartieNordDe}}{La cible se situe dans le site
  et dans se partie nord.}{}%
\dict{\onto[orl]{AuSudDe}}{La cible se situe globalement au sud du
  site, sans préciser si elle est à dans la partie sud du site ou
  disjointe du site et à son sud.}{}%
\dict{\onto[orl]{AuSudDeExterne}}{La cible se situe globalement au sud
  et est disjointe du site.}{\enquote{Je suis entre le grand veymont
    et Pas de la Ville, tout à fait, coté sud.}}%
\dict{\onto[orl]{DansLaPartieSudDe}}{La cible se situe dans le site et
  dans se partie sud.}{}%
\dict{\onto[orl]{HorsDePlanimetrique}}{}{}%
\dict{\onto[orl]{ApresJalonSurItineraire}}{La cible est située sur un
  itinéraire (une relation SitueSurItineraireOuReseauSupport peut être
  instanciée), et après le site qui est un jalon de cet
  itinéraire. Cela suppose d’avoir défini un référentiel des
  directions associé à l’itinéraire. Ce référentiel est souvent lié au
  sens de progression sur l’itinéraire (l’avant est vers la
  destination, l’arrière vers l’origine).}{\enquote{après le parking
    de la Villette.}}%
\dict{\onto[orl]{AueDssusJalonSurItineraire}}{La cible est au dessus
  du site et sur un chemin menant au site}{\enquote{\textelp{}sur le
    sentier qui mène à la cascade de l’oursière, haut dessus de la
    cascade de l’oursière}}%
\dict{\onto[orl]{AvantJalonSurItineraire}}{La cible est située sur un
  itinéraire (une relation SitueSurItineraireOuReseauSupport peut être
  instanciée), et avant le site qui est un jalon de cet
  itinéraire. Cela suppose d’avoir défini un référentiel des
  directions associé à l’itinéraire. Ce référentiel est souvent lié au
  sens de progression sur l’itinéraire (l’avant est vers la
  destination, l’arrière vers l’origine).}{}%
\dict{\onto[orl]{SousJalonSurItineraire}}{La cible est située sur un
  chemin menant au site et sous le site}{\enquote{on est sous le
    déversoir du lac de belledonne}}%
\dict{\onto[orl]{DeAutreCoteDeParRapportA}}{La cible est de l’autre
  côté du site 1 par rapport au site 2. Autrement dit, la cible et le
  site 2 sont de part et d’autre du site 1. C’est une situation
  opposée à DuMemeCoteQueParRapportA, où la cible et le site1 sont
  situés du même côté du site 2. Comme pour DuMemeCoteQueParRapportA,
  cela suppose qu’on peut partitionner l’espace en deux zones appelées
  les deux « côtés » du site 1. Selon la forme et la nature du site 1
  (et son contexte spatial) : soit le site le site 1 a intrinsèquement
  deux côtés (un col, une crête, une rivière, une ville située sur une
  rivière)... Soit, pour deux sites assimilables à des ponctuels et
  sans contexte spatial particulier, c’est le couple (site 1, site 2)
  qui permet de définir deux côtés au site 1 : un côté qui contient le
  site 1, et un côté qui ne le contient pas. Typiquement la limite est
  alors la perpendiculaire au segment joignant le site 1 au site 2,
  passant par le site 1.}{\enquote{Les rochers de l’homme, vous
    descendiez de l’autre côté plutôt ?}}%
\dict{\onto[orl]{DuMemeCoteQueParRapportA}}{La cible située du même
  côté que le site 1, par rapport au site 2. Cela suppose qu’on peut
  partitionner l’espace en deux zones appelées les deux « côtés » du
  site 2. Selon la forme et la nature du site 2 (et son contexte
  spatial) : soit le site le site 2 a intrinsèquement deux côtés (un
  col, une crète, une rivère, une ville située sur une rivière)… Soit,
  pour deux sites assimilables à des ponctuels et sans contexte
  spatial particulier, c’est le couple (site 1, site 2) qui permet de
  définir deux côtés au site 2 : un côté qui contient le site 1, et un
  côté qui ne le contient pas. Typiquement la limite est alors la
  perpendiculaire au segment joignant les site 1 au site 2, passant
  par le site 2. DuMemeCoteQueParRapportA est l’exact opposé de
  DeLAutreCoteDeParRapportA.}{\enquote{c'est côté lac Robert}}%
\dict{\onto[orl]{EntreXetY}}{Il faut DEUX sites ici (site1 et
  site2). La cible est située entre les deux. Le cadre de référence
  est particulièrement important ici. Il faudra le modéliser pour
  pouvoir spatialiser une relation comme celle-là. Par exemple, si les
  deux sites peuvent être considérés comme de faible extension
  spatiale (donc assimilables à des ponctuels), le cadre de référence
  peut inclure un ou plusieurs tracés à une dimension qui relient les
  deux sites : tracé d’un itinéraire, etc. (ou une ligne droite en
  l’absence d’information). Si l’extension spatiale est plus
  importante, on peut définir une zone « entre les deux » comme dans
  le modèle 5IM (Clementini et Billen 2006). Mais attention aux cas
  déictiques comme « je vois Z entre X et Y » ou « nagez entre les
  deux poteaux » de Bateman et al. 2010.}{\enquote{Là on est dans le
    secteur entre… par les pistes forestières, entre cordéac et
    pellafol}}%
\dict{\onto[orl]{ALaMemeAltitudeQue}}{La cible est située à la même
  altitude que le site. Le site peut être une altitude absolue (=>
  donc plus ou moins une courbe de niveau ou un « plan de niveau »),
  ou un objet dont l’altitude sert donc de référence.}{\enquote{c’est
    à 2300 mètres d’altitude}}%
\dict{\onto[orl]{AuDessusALAplombDe}}{}{}%
\dict{\onto[orl]{AuDessusAltitude}}{La cible a une altitude supérieure
  à celle du site. La distance entre le site est la cible n’est pas
  contraignante.}{\enquote{Au-dessus de Percolin, y’a la cabane du
    berger de Bellefont et on est 50 mètres au-dessus}}%
\dict{\onto[orl]{AuDessusProche}}{La cible est proche et a une
  altitude supérieure à celle du site.}{\enquote{juste au-dessus de
    Bernin}}%
\dict{\onto[orl]{SousALAplombDe}}{La cible est à une altitude
  supérieure à celle du site et site et le site est plus ou moins
  situé sur la ligne de plus grande pente qui passe par la cible, ou
  inversement (NB : pas facile de calculer la ligne de plus grande
  pente passant par la cible sans connaître la localisation de la
  cible... pourtant, si le site est en fond de vallée c’est bien le
  site qui est situé sur la ligne de plus grande pente passant par la
  cible, et non l’inverse...). Typiquement, site et cible
  appartiennent à une même vallée (pb de niveau de granularité...) et
  sont situés à peu près au même niveau longitudinalement.}{}%
\dict{\onto[orl]{SousAltitude}}{La cible a une altitude inférieure à
  celle du site. Sa distance au site n’est pas
  contrainte.}{\enquote{je pense pas qu'ils soient passés par les
    vires en-dessous}}%
\dict{\onto[orl]{SousProcheDe}}{La cible est proche et a une altitude
  inférieure à celle du site.}{\enquote{On est à 100 m ou 150, ou 200
    m du col, vraiment juste en-dessous}}%
\dict{\onto[orl]{SousRecouvertPar}}{La cible est sous le site et elle
  est recouverte par ce dernier (ex. "je suis sous un
  pont")}{\enquote{avec un mec coincé sous un arbre}}%
\dict{\onto[orl]{ADistanceQuantitativeRelative}}{La cible est située à
  une certaine distance du site. Cette distance est quantitative et
  est exprimée par une métrique dépendant du site (eg. À un pâté de
  maisons). Il n’y a pas de contrainte sur la dimension de la
  distance, elle peut être planimétrique, altimétrique ou les deux
  .}{\enquote{et lui est à une longueur en-dessous (du dos d'âne),
    donc à une petite longueur du haut du couloir de la Meije}}%
\dict{\onto[orl]{ADistanceTemps}}{La cible est à une certaine distance
  du site. Cette distance est exprimée en temps de parcours. Il est
  par conséquent nécessaire de connaître le mode de déplacement ou la
  vitesse de déplacement.}{}%
\dict{\onto[orl]{ATempsDeMarche}}{La cible est à tel temps de marche
  du site, en marchant depuis le site vers la cible. (temps de marche
  en propriété de la relation).}{\enquote{on est à 10 minutes du
    sommet de la Bastille}}%
\dict{\onto[orl]{AuxAlentoursDe}}{La cible est suffisamment proche du
  site pour qu’on puisse considérer que le site reste un point de
  repère qui a un sens. La proximité qui pourrait s’en déduire est
  généralement dépendante d’un « rayonnement » qui pourrait êtrre
  affecté au site (lié à sa renommée, à sa taille typiquemnet pour un
  lieu habité, à sa saillance...). Il peut y avoir connexion
  topologique entre la cible et le site, ou pas. Exemple : « je suis
  dans le coin du mont-Blanc ».}{\enquote{dans le massif de la
    chartreuse}}%
\dict{\onto[orl]{DistanceQuantitativePlanimetrique}}{La cible est à
  telle distance du site, exprimée dans une unité de longueur (pas un
  temps de marche/accès). La distance considérée est planimétrique. Si
  relief en pente « pas trop forte », ça peut être le long de la
  pente, mais il ne s’agit pas d’un dénivelé.}{\enquote{j'estime à 800
    m (du Pas de la Ville) je crois, à peu près, à vol d'oiseau}}%
\dict{\onto[orl]{PresDe}}{}{\enquote{vers midi, il était près du
    sommet et dans le brouillard}}%
\dict{\onto[orl]{SiteVoitCible}}{Le site voit la cible (vision
  active). La position de l'objet à localiser (ex. la victime) est
  visible depuis une autre position connue (ex. un refuge depuis
  lequel un témoin contacte les secours).}{}%

% \dict{\onto[orl]{ADroiteDe} $\sqsubseteq$
% \onto[orl]{RelationSpatialeDeDirection}}{}{}%



%%% Local Variables:
%%% mode: latex
%%% TeX-master: "../../main"
%%% End:
