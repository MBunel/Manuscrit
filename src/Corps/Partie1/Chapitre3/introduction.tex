Ce chapitre se destine à dresser un état de l'art des objectifs
scientifiques de la thèse, présentés au chapitre précédent. Nous
allons nous concentrer sur deux points principaux, qui nous semblent
être, à la fois les objectifs les plus difficiles à traiter, mais
également les plus importants de ce travail de doctorat, à savoir la
question de la \emph{spatialisation} des \emph{indices de
  localisation} et la prise en compte de \emph{l'imprécision} des
\emph{indices de localisation.}

La première partie de cet état de l'art sera donc consacrée à la
notion de \emph{relation de localisation} et aux solutions proposées
dans la littérature pour les modéliser. Nous détaillerons les concepts
de \emph{sujet,} \emph{objet de référence} et \emph{relation de
  localisation,} introduits dans le chapitre précédent. Nous
présenterons ensuite les différentes classifications des
\emph{relations de localisation} qui ont été proposées dans la
littérature. Puis, nous détaillerons les différents modèles qui ont
été proposés dans la littérature pour modéliser, interpréter ou
spatialiser ces \emph{relations de localisation.}

La seconde partie de cet état de l'art présentera la notion
\emph{d'imperfection} (également abordée dans le chapitre précédent)
et ses trois composantes (\emph{imprécision,} \emph{incertitude} et
\emph{l'incomplétude}). Nous donnerons une définition détaillée de ces
concepts et de leurs équivalents spatiaux, puis nous présenterons les
différentes théories les formalisant. Enfin, nous nous focaliserons
sur \emph{l'imprécision spatiale} et présenterons les modèles proposés
dans la littérature pour construire des \emph{objets géographiques
  imprécis,} dans le but d'identifier une méthode nous permettant de
modéliser l'imprécision des \emph{indices de localisation} et des
\emph{zones de localisation compatibles} et \emph{probables} en
résultant.

%%% Local Variables:
%%% mode: latex
%%% TeX-master: "../../../main"
%%% End:
