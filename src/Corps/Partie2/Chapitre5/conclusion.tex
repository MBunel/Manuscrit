La phase de décomposition des \emph{indices de localisation} se
compose de trois étapes (décomposition de l'ensemble des indices de
localisation, décomposition des objets de référence non nommées et
décomposition des relations de localisation), dont seule la dernière
soulève des problèmes techniques et scientifiques. En effet, là où les
indices de localisation et les objets de référence non nommés sont
saisis de manière décomposée et où la décomposition n'est donc qu'une
formalité, les \emph{relations de localisation} sont sélectionnées
sous une forme composée, conçue pour être manipulable par
l'utilisateur. L’application de la décomposition des \emph{relations
  de localisation} nécessite donc de disposer d'une base de règles,
formalisant le processus de décomposition à suivre pour chaque
\emph{relation de localisation} spatialisable.

Nous avons choisi de construire cette base de règles à l'aide de deux
ontologies \emph{ad hoc} et fortement liées : \ac{orl} et
\ac{orla}. La première est une ontologie des \emph{relations de
  localisations} qui propose de recenser et de définir toutes les
\emph{relations de localisation} pertinentes dans notre
contexte. L'ontologie des \emph{relations de localisation atomiques}
défini quant à elle l'ensemble des \emph{relations de localisations
  atomiques} est la manière dont elles se composent pour créer les
relations de localisation définies dans \ac{orl}. Le peuplement de ces
ontologies s'est fait en deux temps. Tout d'abord nous avons identifié
et définit toutes les \emph{relations de localisations} pertinentes
dans notre contexte. Pour ce faire nous avons retranscrit et analysé
un grand nombre d'enregistrements d'alertes, dans le but d'identifier
les \emph{relations de localisation} utilisées par les requérants pour
décrire leur position. Nous avons ensuite défini les \emph{relations
  de localisation} permettant de modéliser ces alertes. Les
\emph{relations de localisation atomiques} ont quant à elles étés
définies au fil de l'eau, en fonction des relations de localisation
utilisées dans les alertes traitées.

%%% Local Variables:
%%% mode: latex
%%% TeX-master: "../../../main"
%%% End:
