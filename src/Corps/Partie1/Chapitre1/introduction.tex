Comme nous l'expliquions en introduction (\autoref{prt:int}), cette
thèse s'inscrit dans le cadre d'un projet de recherche
pluridisciplinaire dont l'objectif principal est de développer des
méthodes et des outils facilitant la localisation de personnes
bléssées ou perdues en montagne. Notre travail de recherche d'inscrit
donc, à la fois dans le contexte métier du secours en montagne mais
également dans un contexte organisationnel, celui d'un projet de
recherche pluridisciplinaire.

L'objectif de ce premier chapitre est de présenter ces deux contextes,
leurs sécificités ayant nécessairement impacté l'ensemble de ce
travail doctoral. La présentation du secours en montagne proposée ici,
a pour but de présenter son fonctionnement, son organisation actuelle
et l'historique de cette activité mais surtout de décrire la démarche
de localisation des victimes utilisée par les secouristes, question
qui est au cœur de ce travail de recherche. La présentation du projet
de recherche dans lequel nous nous inscrivons, insistera, quant à elle
sur ses différents objectifs scientiques et leurs interactions. Ce qui
nous permettra de distinguer les questions qui serons abordées
dans notre thèse de celles qui ne le seront pas, car rattachées à
d'autres objectifs de recherche.

La première partie de ce chapitre se concentrera donc sur les la
présentation du secours en montagne et son organisation. Quant à la
seconde partie de ce chapitre elle abordera la question de
l'inscription de cette thèse au sein du projet de recherche Choucas.

%%% Local Variables:
%%% mode: latex
%%% TeX-master: "../../../main"
%%% End:
