L’application de notre méthode à des cas réels nécessite d'aborder
trois questions supplémentaires.

Il est tout d'abord nécessaire de procéder à l'implémentation de la
méthode que nous avons définie.

Puis il est nécessaire de sélectionner les alertes à modéliser

Enfin, il convient de définir les données utilisées lors de la
spatialisation.

\subsection{Présentation des alertes}



L'ensemble des retranscriptions et des \emph{templates} de
retranscription qui y sont associés sont présentés dans
l'\autoref{anx:retrans}.


\subsubsection{Statistiques}

Pour justifier le choix des alertes nous avons fait des stats 


\subsection{L'implémentation de la méthode de spatialisation}

Pour permettre la confrontation de notre méthode à la réalité nous
avons développé un prototype, baptisé \emph{Ruitor} \footnote{XXXX},
implémentant l'ensemble de nos propositions.
%
L'ensemble des modélisations que nous avons présentées jusqu'ici (à
l'exceptions de celles présentes dans le \autoref{chap:06}) ont été
réalisées avec ce logiciel.

\emph{Ruitor} a été entièrement développé à l'aide du langage de
programmation python, dans sa version 3.5, de septembre 2015
\autocite{VanRossum2009}. Ce langage a été sélectionné pour deux
raisons principales :
%
\begin{enumerate*}[label=(\arabic*)]
\item D'une part car c'est un langage de programmation avec lequel
  nous avons une certaine expérience \autocite{Bunel2017b,Bunel2017c},
  ce qui nous a permis de limiter le temps de développement consacré à
  la résolution de problèmes qui auraient pu être causés par une
  mauvaise maitrise du langage.
\item De plus il s'agit d'un langage populaire et répandu, notamment
  en géomatique, que ce soit dans la recherche ou dans le milieu
  professionnel. De nombreuses bibliothèques utiles pour
  l'implémentation de notre méthode sont donc développées dans ce
  langage.
\end{enumerate*}
%
Le choix d'un langage de développement ne fixe pas à lui seul le cadre
du développement. En effet, plusieurs approches de développement,
utilisant le même langage, sont envisageables. Une première solution
serait d'employer le \emph{cadriciel} fourni par les différents SIG
professionnels, comme QGis ou ArcGis. Cette solution permet de
disposer des nombreuses fonctions de manipulation, d'analyse et de
visualisation de ces logiciels. Cette approche a cependant deux
défauts, dont l'un majeur. Tout d'abord il est nécessaire de se
familiariser avec ces fonctions spécifiques, ce qui peut nécessiter un
temps d’apprentissage conséquent, bien que cette remarque soit valable
pour n'importe quelle bibliothèque logicielle. De plus, le choix
d'utiliser le \emph{cadriciel} d'un SIG donné, impose une forte
dépendance à ce logiciel, ce qui peut considérablement augmenter la
difficulté du déploiement d'un logiciel fondé sur ces technologies. À
la réflexion, l'utilisation du \emph{cadriciel} d'un SIG nous semble
plus adapté à un développement de \emph{scripts} ---~de petits
logiciels conçus pour une tâche spécifique et déclenchés manuellement
par l'utilisateur~---, étendant les fonctionnalités d'un SIG, plutôt
que d'un logiciel unique, fonctionnant de manière quasi-automatisée.

Le rejets des \emph{cadriciels} d'un SIG n'impose cependant pas de
rejeter toute bibliothèque de manipulations de données géométrique.


Nous avons donc choisi d'utiliser une solution totalement indépendante
des logiciels de SIG, uniquement fondée sur des bibliothèques de bas
niveau.





Les trois phases de notre méthode traitant d'objets différents (\ie
des \emph{indices de localisation} dans la phase de décomposition et
des \ac{zlc} dans la phase de spatialisation et de fusion), il en
résulte que les bibliothèques qui 

% Implémentation de la phase de décomposition

Lors de la présentation de la \emph{phase de décomposition}
(\autoref{chap:05}) nous avions indiqué que difficulté de mise en
place de cette phase était concentrée sur la dernière étape, celle de
la décomposition des \emph{relations de localisation,} les
\emph{indices de localisation} et les \emph{objets de référence} étant
déjà décomposés lors de leur saisie par le secouriste. Par corolaire,
implémentation des deux premières étapes de la phase de décomposition
ne pose pas, non plus, de problèmes particuliers.

Au début de la phase de décomposition ---~et donc de la méthode dans
son ensemble~--- nous disposons de la requête du secouriste
formalisée
%
Pour le développement du prototype nous avons mis en pla


% Implémentation des phases de spatialisation \& de fusion

Les objets géographiques sont manipulés (pour les \emph{objets de
  référence}) et construits (pour les zones de localisation
compatibles et probables) à partir de l'étape de spatialisation.

La mise en place de ces étapes nécessite donc de mettre en place de
nouveaux outils pour manipuler les géométries.

Notre travail faisant une grande utilisation de rasters, la
bibliothèque permettant leur manipulation prend une place centrale
lors de cette étape.

Nous avons construit notre implémentation autour de la bibliothèque
\emph{numpy,} conçue pour le calcul scientifique et plus précisément
la manipulation optimisée de matrices volumineuses
\autocite{vanderWalt2011}. Le type de données défini par \emph{numpy}
\footnote{Le \emph{ndarray,} pour
  \foreigntextquote{english}{N-dimensional array} (Tableau
  multidimensionnel).} fait office de standard du calcul scientifique
dans l'univers du python, il est donc à la base de nombreuses
bibliothèques de calcul scientifique.  La bibliothèque \emph{numpy}
permet d'effectuer de nombreux calculs avancés, toutefois certaines
méthodes spécifiques n'y figurent pas, c'est pourquoi nous avons
également fait appel aux bibliothèques \emph{SciPy} et
\emph{scikit-image} \autocite{vanderWalt2014,Virtanen2020} qui
proposent des outils supplémentaires tout en opérant de à partir des
types de données définis par \emph{numpy.}

Owlready \autocite{Lamy2017}

\subsection{Données}

L'ensemble des \emph{objets de référence} utilisés pour la
spatialisation des indices de localisation proviennent de la même base
de données, la BDTOPO, produite par l'IGN. Cette dernière
%
Pour nos modélisations nous avons travaillé exclusivement à partir du
millésime de 2018


Une base de données comme la BDTOPO, ne propose pas une représentation
exhaustive et à l'échelle unitaire du territoire qu'elle
cartographie. Tous les objets du territoire n'y figurent pas et les
géométries qui les représentent ne sont que des abstractions
simplifiée de la forme réelle de l'objet.

Le principal problème d'une telle base de donnée est donc que certains
objets du réels, pouvant donc être utilisés comme \emph{objets de
  référence} dans un \emph{indice de localisation,} n'y figurent pas
nécessairement.


La BDALTI est la composante altimétrique du référentiel à grande
échelle de l'IGN.



%%% Local Variables:
%%% mode: latex
%%% TeX-master: "../../../../main"
%%% End:
