La construction d'une zone à partir de la description qui en est
données par le requérant est une tache complexe, qui nécessite de
prendre en compte de nombreux éléments que nous n'avons pas abordés
lors de la présentation des objectifs scientifiques du projet
Choucas. De plus, durant le premier chapitre de ce manuscrit nous
avons fait emploi de nombreux termes, inhérents au projet, sans pour
autant les définir. Nous allons a présent clarifier ces différents
termes (\ref{subsec:2-1}), avant de détailler les concepts généraux de
cette thèse (\ref{subsec:2-2})

\subsection{Définition des termes du sujet}
\label{subsec:2-1}

% Alertes
Jusqu'à présent nous avons abondamment employé le terme
\enquote{d'alerte} pour désigner à la fois une opération de secours
donnée, sa phase de localisation et l'ensemble des éléments décrivant
une position donnés par le requérant. Dans cadre du projet Choucas ce
terme prend une définition plus restrictive, qui est formalisée dans
\emph{l’ontologie d'alerte Choucas}
\autocite[\ac{oac},][]{Viry2019}. Dans l'ontologie \ac{oac} le terme
\enquote{alerte} est défini comme l'ensemble des \emph{indices de
  localisation} à la disposition des secours pour identifier la
position d'une victime donnée, pour une opération de secours
donnée. Une alerte est donc caractérisée par un ensemble
\emph{d'indices de localisation,} mais également par sa temporalité et
l'identité de la \emph{victime} (et du \emph{requérant,} s'ils sont
différents). L'objectif de cette thèse est de construire les zones
correspondant à la position de la victime à partir de la description
qui en est faite dans une alerte donnée, \ie à partir des différents
\emph{indices de localisation} qui la composent.

% Indices de localisation
Ces derniers correspondent à l'ensemble des éléments donnés par le
requérant pour décrire une position. Par exemple, \enquote{je suis
  sous une route} ou \enquote{il était à La Grave} sont deux
\emph{indices de localisation,} bien qu'ils soient très différents. En
effet, le premier exemple décrit la position actuelle du locuteur
(\enquote{je}), alors que le second décrit la position d'un tiers
(\enquote{il}) à un instant révolu. On pourrait, dès lors, penser que
le second exemple n'est pas représentatif des \emph{indices de
  localisation} apparaissant dans les alertes traitées, leur objectif
commun étant de décrire la position d'une victime (et non d'un tiers)
au moment de l'appel (et non il y a plusieurs heures) et tout autre
élément pourrait sembler, au mieux, inutile. Pourtant, il n'est pas
rare que le requérant soit amené à décrire la position de la victime à
partir de la sienne, donnant un ensemble d'indices de localisation de
la forme :
%
\begin{enumerate*}[label=(\alph*)]
\item \enquote{Je suis dans un refuge},
\item \enquote{la victime est en face de moi, sur le versant opposé}.
\end{enumerate*}
% 
De plus des \emph{indices de localisation} décrivant des positions
passées peuvent s'avérer utiles, comme dans le cas du \emph{fil rouge}
où de nombreux \emph{indices de localisation} décrivent des positions
passées (\eg \enquote{La victime est partie de Bourg-d'Oisans},
cf. \autoref{subsec:1-1-2-3}). Le nombre \emph{d'indices de
  localisation} présents dans \emph{une alerte} est très variable, le
\emph{fil rouge}, par exemple, en contient une dizaine. S'il peut,
théoriquement, être de un, ce n'est jamais le cas en pratique, les
secouristes disposent toujours de plusieurs \emph{indices de
  localisation,} même si l'un d'entre eux est extrêmement précis,
comme des coordonnées GPS.
 
% Relations de localisation
Les \emph{indices de localisation} sont eux-mêmes des objets composés
de plusieurs éléments, un \emph{sujet} (\enquote{je}, \enquote{la
  maison}), qui est l'élément dont la position est décrite ; un
\emph{objet de référence} (\enquote{la maison}, \enquote{moi}), qui
est l'élément à partir duquel est définie la position du \emph{sujet}
et une \emph{relation de localisation} (\enquote{en face},
\enquote{sous}) qui définit la relation entre \emph{sujet} et
\emph{objet de référence.} C'est en interprétant ces trois éléments
que l'on peut construire la \emph{zone de localisation} correspondant
à une description de position, processus que nous nommons
\emph{spatialisation.}

%TODO
%\tdi{revoir titre}
\subsection{Présentation des concepts du sujet}
\label{subsec:2-2}

% TODO
% \tdi{Vérifier que la spatialisation est bien définie}

% \tdi{Introduire la spatialisation avec la citation :
%   \enquote{\textelp{} La localisation d'une cible par rapport à un
%     site peut être déterminée également à partir des coordonnées d'un
%     troisième intervenant, l'observateur \textelp{}}
%   \autocite[p. 19]{Borillo1998}}

À cause de la nature composite des \emph{alertes,} la construction de
la \emph{zone de localisation} correspondant à une description de
position ne peut se faire directement. Il est nécessaire de procéder
en deux étapes, la première destinée à \emph{spatialiser} les
différents \emph{indices de localisation} et la seconde combinant les
résultats des différentes \emph{spatialisations} pour aboutir à la
\emph{zone de localisation} finale, \ie la \emph{fusion} des résultats
de la \emph{spatialisation.} On est donc amenées, pour une même
alerte, à définir plusieurs \footnote{À l'exception du cas, déjà
  mentionné, où le requérant ne donne qu'un seul \emph{indice de
    localisation.}  Toutefois, c'est un cas que l'on rencontre pas en
  pratique.}  \emph{zones de localisation,} une pour chaque
\emph{indice de localisation,} auxquelles s'ajoute une dernière zone,
construite par la \emph{fusion} des résultats de la
\emph{spatialisation.}  L'ontologie \ac{oac} \autocite{Viry2019} fait
donc la distinction entre les \emph{zones de localisation
  compatibles,} construites par la spatialisation des \emph{indices de
  localisation} et la \emph{zone de localisation probable,} qui
résulte de leur \emph{fusion.}

% Spatialisation
Le processus de \emph{spatialisation} d'un \emph{indice de
  localisation} est illustré par la figure \ref{fig:obj_spa}. Dans cet
exemple, \emph{l'indice de localisation} \enquote{je suis proche d'une
  maison} est \emph{spatialisé} à l'aide d'une zone tampon. On
considère que la zone \enquote{proche d'une maison} correspond à
toutes positions situées à moins d'une distance fixée de \emph{l'objet
  de référence}, la zone résultante est donc un disque. Cette
\emph{spatialisation} est simple, voire simpliste et il ne faut pas
voir cette illustration comme une représentation de la méthode qui
sera effectivement utilisée pour \emph{spatialiser} cet indice, mais
plutôt comme une illustration du processus de \emph{spatialisation.}

\begin{figure}[hb]
  \centering
  \begin{tikzpicture}

\node[text width=2.5cm, align=center] (0,0) {\small\enquote{Je suis proche d'une maison}};

\path[draw, ->] (2,0) --++ (2.5,0)  node[pos=.5, above] {\footnotesize \itshape spatialisation};

\begin{scope}[xshift=6.5cm]
\path[ffa] (0,0) circle [radius=30pt];
\path[ffc] (0,0) circle [radius=30pt];

\node[circle, inner sep=0pt,minimum size=4pt, fill] (c) at (0,0) {};
\node[circle, inner sep=0pt,minimum size=8pt] (c2) at (0,0) {};
\node[anchor=west] (m) at  (1.75,1) {\footnotesize \itshape objet de référence};
\path[draw, ->] (m.west) -- (c2);

\node[anchor=west, text width=3cm] (m2) at  (1.75,0) {\footnotesize \itshape zone de localisation compatible};
\path[draw, ->] (m2.west) --++ (-.5,0);
\end{scope}
\end{tikzpicture}
  \caption{Illustration du processus de \emph{spatialisation} d'un
    \emph{indice de localisation.}}
  \label{fig:obj_spa}
\end{figure}

% Fusion
\emph{La spatialisation} est une étape qui doit être répétée pour tous
les \emph{indices de localisation} donnés par le requérant, on dispose
ainsi d'autant de \emph{zones de localisation compatibles} que
d'indices. Chacun d'entre eux donne une information sur la position de
la victime, qui, si l'on exclut le cas où certains indices sont faux,
se situe dans une zone où tous les \emph{indices de localisation} sont
vérifiés, \ie que la \emph{zone de localisation probable} est située à
l'intersection de toutes les \emph{zones de localisation compatibles.}
%
% Exemple
Prenons pour exemple une version simplifiée du \emph{fil rouge}
(cf. \autoref{subsec:1-1-2-3}), composée de seulement deux
\emph{indices:} \enquote{la victime voit une partie de plan d'eau} et
\enquote{elle vient de passer du soleil à l'ombre.}  La \emph{zone de
  localisation probable} correspondant à cette description est, à la
fois une zone à partir de laquelle on peut apercevoir une partie de
plan d'eau (\ie que le premier \emph{indice de localisation} est
validé), mais aussi une zone qui vient de passer du soleil à l'ombre
(\ie le second \emph{indice de localisation} est validé). Par
conséquent la \emph{zone de localisation probable} correspond à la
zone validant les deux \emph{indices de localisation} de
\emph{l'alerte}, \ie l'intersection des deux \emph{zones de
  localisation compatibles} leur correspondant, comme l'illustre la
figure \ref{fig:obj_fus}. Deux \emph{zones de localisation
  compatibles} y sont représentées, lesquelles sont produites par le
processus de \emph{spatialisation} décrit par la figure
\ref{fig:obj_spa}. La méthode de \emph{fusion} combine ces deux zones
pour en construire une seule, la \emph{zone de localisation probable,}
correspondant à l'intersection des deux \emph{zones de localisation
  compatibles}. Toutefois, comme pour la figure \ref{fig:obj_spa},
cette représentation est grandement simplifiée, notamment car les deux
\emph{zones de localisation compatibles} sont similaires et partagent
le même \emph{objet de référence,} ce qui n'est généralement pas le
cas, comme le montre l'extrait du cas \emph{fil rouge} pris pour
exemple ci-dessus. De plus, les différents \emph{indices de
  localisation} ne sont pas donnés de manière concomitante, la
description de position s'étoffe au fur et à mesure de la conversation
et le résultat \emph{fusion} doit donc être mis à jour.

\begin{figure}
  \centering
  \begin{tikzpicture}
  % ZLC 1
  \path[ffa] (0,0) circle [radius=30pt]; % Aire
  \path[ffc] (0,0) circle [radius=30pt]; % Contour
  % ZLC 2
  \path[ffa2, pattern color=RdBu-9-3] (.25,-.5) ellipse (40pt and 20pt); % Aire 
  \path[ffc2, draw=RdBu-9-2] (.25,-.5) ellipse (40pt and 20pt); % Contour
  % Objet de ref
  % 1
  \node[circle, inner sep=0pt,minimum size=4pt, fill] (c) at (0,0) {}; % visible
  \node[circle, inner sep=0pt,minimum size=8pt] (c2) at (0,0) {}; % fictif
  % 2
  \node[circle, inner sep=0pt,minimum size=4pt, fill] (c1) at (.25,-.5) {}; % visible
  \node[circle, inner sep=0pt,minimum size=8pt] (c12) at (.25,-.5) {}; % fictif
  % Légende
  \node[anchor=east] (m) at  (-1.5,.25) {\footnotesize \itshape objets de référence};
  \path[draw, ->] (m.east) -- (c2);
   \path[draw, ->] (m.east) -- (c12);
  \node[anchor=north west, text width=3cm] (m2) at (1.75,-.4) {\footnotesize
    \itshape zones de localisation compatibles};
  \path[draw, ->] (m2.west) --++ (-1.7,0);
  \path[draw, ->] (m2.west) -- (1.2,-.7);
  % Fléche + texte
  \path[draw, ->] (2.25,0) --++ (2.75,0)  node[pos=.5, above] {\footnotesize \itshape fusion};
  % Fusion
  \begin{scope}[xshift=6.75cm]
    \path[draw, dashed] (0,0) circle [radius=30pt]; % Contour
    \path[draw, dashed] (.25,-.5) ellipse (40pt and 20pt);
    \begin{scope}
      \clip (.25,-.5) ellipse (40pt and 20pt);
      \fill[ffa2] (0,0) circle [radius=30pt];
      \path[ffc2] (0,0) circle [radius=30pt];
    \end{scope}
    \begin{scope}
      \clip (0,0) circle [radius=30pt];
      \path[ffc2] (.25,-.5) ellipse (40pt and 20pt);
    \end{scope}
    \node[circle, inner sep=0pt,minimum size=4pt, fill] (c) at (0,0)
    {};
    \node[circle, inner sep=0pt,minimum size=4pt, fill] (c) at (.25,-.5) {};
    \node[anchor=west, text width=3cm] (m2) at  (1.25,.75) {\footnotesize \itshape zone de localisation probable};
    \path[draw, ->] (m2.south west) --++ (-.6,-.4);
  \end{scope}
\end{tikzpicture}
  \caption{Illustration du processus de construction de la \emph{zone
      de localisation probable} par la \emph{fusion} des \emph{zones
      de localisation compatibles.}}
  \label{fig:obj_fus}
\end{figure}

La \emph{spatialisation} des \emph{indices de localisation} en des
\emph{zones de localisation compatibles} et leur \emph{fusion} en une
\emph{zone de localisation probable} sont les deux étapes majeures
nécessaires à la transformation d'une description de localisation en
une \emph{zone de localisation} et sont, par conséquent, au cœur des
deux objectifs principaux de cette thèse, à savoir le développement de
méthodes de \emph{spatialisation} et de \emph{fusion} qui soient
adaptées à notre contexte applicatif. En effet, les deux exemples
présentés font ignorent tous les problèmes que l'on peut rencontrer
dans des cas réels, comme la fausseté des \emph{indices de
  localisation,} le fait que ces derniers ne soient pas connus de
manière concomitante ou la difficulté d'interpréter la sémantique des
\emph{relations de localisation,} etc. Les méthodes de
\emph{spatialisation} et de \emph{fusion} devront donc prendre en
compte ces différents enjeux pour être applicables à notre contexte.


%%% Local Variables:
%%% mode: latex
%%% TeX-master: "../../../../main"
%%% End:
