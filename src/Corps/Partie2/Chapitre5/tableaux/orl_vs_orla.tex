\begin{tabular}{r>{\small}p{.35\textwidth}>{\small}p{.35\textwidth}}
  \toprule & \multicolumn{1}{c}{\ac{orl}} &
  \multicolumn{1}{c}{\ac{orla}} \\ \midrule
  \addlinespace
  Objectif & Recense et définit les \emph{relations de localisation} utilisées
  pour décrire une position dans le contexte de la localisation de
  personnes en montagne & Définit les \emph{relations de localisation
                          atomiques,} la décomposition des relations
  définies dans \ac{orl} et formalise le processus de
                          \emph{spatialisation} des \emph{relations
                          spatiales atomiques.}\\
  Contenu & Définition de 51 \emph{relations de localisation,}
            regroupées en 11 classes abstraites. & Définition de XX
                                                   \emph{relations de
                                                   localisation
                                                   atomiques,}
                                                   décomposant XX des
                                                   51 \emph{relations
                                                   de localisation}
                                                   définies dans \ac{orl}.\\
%  Hiérarchie & Blo & \\
  Modélisation & L'ensemble des concepts sont définis pour être facilement
       différentiables et proches de la perception humaine des
       localisations dans l'espace & Les \emph{relations de
                                     localisation atomiques} ne sont
                                     pas conçues pour être manipulées
                                     directement par les utilisateurs.
  Elles généralement plus abstraites que les \emph{relations de
                                     localisation} qu'elles
                                     décomposent.\\ 
  \bottomrule
\end{tabular}
