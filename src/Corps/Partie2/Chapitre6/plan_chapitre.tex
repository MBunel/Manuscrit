% Citation début de chapitre
\dictum[Pierre Dac]{\enquote{La synthèse du monde peut se résumer en ces deux mots : oui et non}}%

\chaptertoc{}

\addsec{Introduction}

Introduction chapitre 6

\tdi{Modélisation = Modèle théorique + implémentation}

\tdi{Cette partie ne concerne que les zlp et les zlc. Les objets de
  référence sont modélisés de façon nette. Mais bloch l'a fait. Le
  flou aux objets est une extension possible.}

\section{Critères de choix de la modélisation}

\tdi{On veut un modèle qui permet de prendre en compte des relations
  de localisation traduisant des limites assez nettes (en dessous de)
  et des relations avec des limites très continues (proches [voir
  concept similaire dans l'ontologie]).}

\tdi{Possibilité de construire des zones de localisation issues de
  relations de localisation non linéaire (en dessous de).}

\tdi{Possibilité de fusionner (intersection et union). Nécessité que
l'union et l'intersection soient commutatives.}

\tdi{Nécessité que les résultats soient facilement
  interprétables. Sémantique des résultats et pas trop de zones (par
  exemple beaucoup d'alpha cuts).}

\tdi{Stabilité du modèle ? (par exemple si l'on fait des intersection
  avec le modèle egg-yolk on doit choisir ce qu'est le jaune. C'est
  l'intersection des jaunes ? ou il suffit qu'il y a ait un jaune?
  est-ce qu'il faut introduire des valeurs intermédiaires (gris,
  lorsqu'on intersecte jaune et blanc) -> mais dans ce cas on sort du
  egg yolk, le modèle n'est pas stable alors). Le résultat doit
  pouvoir s'exprimer avec la même modélisation sans perte de
  sémantique et de finesse de modélisation.}

\missingfigure{Figure explicitant le point 5}

\section{Choix du modèle thèorique}

\subsection{Le choix du flou}

\tdi{Les modèles discrets (exacts, ensembles épais, autres ?)}

-Trop simples, pas de possibilité de modèliser des variations
importantes de degré d'appartenance. \textbf{Variation non linéaires
  (importantes avec le relief})

\missingfigure{comparaison décroissance degré d'appartenace en
  fonction de la pente. Voir si possible de le faire avec une arte
  topo et un vrai relief. Illustrer l'hypothèse implicite d'évolution
  linéaire entre les frontières dans les modèles exacts.}

\tdi{Modèles non linéaires et non flous}

\tdi{Modèle flous plus éprovés dans la littérature}

\tdi{Dire ques les modèles flous sont assez au point sur les opération
  de fusion avec les opérateurs flous qu'il définissent.}

\section{Choix de l'implémentation}

\tdi{Préciser les modèles validés par les deux implémentations}


\tdi{OK : On veut un modèle qui permet de prendre en compte des relations
  de localisation traduisant des limites assez nettes (en dessous de)
  et des relations avec des limites très continues (proches [voir
  concept similaire dans l'ontologie]).}

\tdi{OK : Possibilité de construire des zones de localisation issues de
  relations de localisation non linéaire (en dessous de).}


\subsection{Modélisation par Alpha-cuts}

\texttt{citer de Runz, zoglami}

\missingfigure{Illustration modélisation concrète par alpha-cuts}

\subsection{Modélisation par rasters}

\texttt{citer Bloch, Vanegass, takemura}

\missingfigure{Illustration modélisation concrète par raster
  (introduction de la représentation par rasters).}

\subsection{Comparaison des approches et choix approche raster}

\paragraph{Comparaison des approches}

\tdi{Difficulté de mise en place}

Au moment de la sélection d'une méthode de modélisation nous avons
estimé que le développement d'une méthode utilisant des vecteurs
serait bien plus coutueuses. Sur ce point nous avons préféré les
rasters. De plus il a été bc plus difficile d'aboutir à une
modélisation vectorielle permettant la comparaison (citer RIG).

\tdi{Difficulté d'extension}

Il semble plus difficile de développer des méthodes de spatialisation
très différentes en vecteur qu'en raster.

Il est également plus compliqué de tester de nouveaux couples
d'opérateurs flous en vecteur.

\tdi{Tendance à la fragmentation des alpha-cuts}

\tdi{Temps de calcul}

% Critère moyennement pertinent
Le critère du temps de calcul n'est pas fondamental pour ce
travail. Toutefois il nous semble pertinent de préciser que les deux
méthodes différent sur ce point.

Avantage raster si le nombre de cellules n'est pas trop
important. Mais le temps de calcul augmente rapidement. L'approche
vecteur est moins rapide mais le code est moins (et plus
difficilement) optimisé. Le résultat est assez mitigé.

\tdi{Volume des données}

Avantage vecteur. Le raster tend à prendre beaucoup de place, encore
plus si l'on souhaite stoquer les objets de référence en raster ou
sauvegarder des résutlats intermédiaires.

\tdi{Qualité de la modélisation}

Comparable, l'article de la RIG n'a pas permis de mettre en évidence
des différences notables. Par construction l'approche vectorielle est
plus précise, mais sa précision dépend aussi des données (MNT pour
l'altitude). À l'échelle du pixel les résultats sont comparables.

\tdi{OK pour les deux: Stabilité du modèle ? (par exemple si l'on fait
  des intersection avec le modèle egg-yolk on doit choisir ce qu'est
  le jaune. C'est l'intersection des jaunes ? ou il suffit qu'il y a
  ait un jaune?  est-ce qu'il faut introduire des valeurs
  intermédiaires (gris, lorsqu'on intersecte jaune et blanc) -> mais
  dans ce cas on sort du egg yolk, le modèle n'est pas stable
  alors). Le résultat doit pouvoir s'exprimer avec la même
  modélisation sans perte de sémantique et de finesse de
  modélisation.}


\paragraph{Choix du raster}

\tdi{Plus facile à développer}

\tdi{Plus facile à étendre}

\tdi{Plus rapide d'arriver à spatialiser -> possibilité de travailler
  sur plus d'alertes}

\tdi{Plus facile à utiliser (pas de fragmentation)}

Conséquences :

\tdi{Nécessité de choix des opérateurs flous}

\tdi{Nécessité de définir une résolution}

\addsec{Conclusion}

Conclusion chapitre 6

%%% Local Variables:
%%% mode: latex
%%% TeX-master: "../../../main"
%%% End:
