\addsec{Générale}

L'objectif de cette thèse était de développer une méthode permettant
la transformation d'une description de position en une zone, exprimée
dans un référentiel connu.
%
Pour répondre a cette question nous avons proposé une méthode
permettant de transformer une description de position en une zone de
localisation compatible.


Comme nous l'avons montré dans les deux premiers chapitres de cette
thèse, de nombreuses



La méthode proposée a l'avantage de 




\addsec{Contributions}

La méthode que nous proposons se fonde sur 

\addsec{Perspectives}

Comme l'ont montré les applications de note méthode a des cas réels
(\autoref{chap:09}), de nombreuses pistes d'évolutions sont
envisageables.

On peut identifier deux grandes catégories d'améliorations
potentielles, celles plutôt techniques, visant à offrir une meilleure
utilisabilité de notre prototype dans le contexte métier du secours en
montagne et les perspectives scientifiques.

% Ajout de relations

Portée sonore

Passage ombre

antenne téléphonique

Une première perspective d'amélioration serait de proposer de
nouvelles relations de localisation adaptées aux besoins métier. Le
\autoref{chap:09} et plus particulièrement la modélisation du
\emph{fil rouge} nous a permis de constater que certains indices très
spécifiques comme \enquote{la victime entend des voitures} ou
\enquote{le téléphone de la victime est connectée à une antenne
  téléphonique donnée}, n'étaient pas correctement modélisés avec les
concepts que nous avions développés.
%
Nous avons donc proposé de compléter l'ontologie des relations de
localisation spécifiquement dédiées a la modélisation de ce type
d'indices
%
La spatialisation de telles relations de localisation est également
assez 


 %L'évaluation de la qualité des \ac{zlp}

Comme l'étude du \emph{fil rouge} l'a montré (\autoref{chap:09}), la
\emph{zone de localisation probable} produite par notre méthode peut,
en fonction des indices de localisation, avoir des configurations très
différentes. Elle peut être d'un bloc, ou fragmentée. Dans cette
seconde situation,  e

% La sélection des seuils

Une autre limite de la méthode proposée est la paramétrisation des
fuzzyficateurs. Comme nous l'avons montré dans le \autoref{chap:09}
cette paramétrisation doit être effectuée manuellement, en fonction de
considérations empiriques. Or, si l'utilisation de fuzzyficateurs
spécialisés comme \onto[orla]{Sup\-Val\-0} ou \onto[orla]{Eq\-Val\-0}
facilite l'utilisation de certaines relations de localisation (\eg
\onto[orl]{Alt\-Inf}} il est toujours nécessaire de fixer la valeur
d'un paramètre \(\delta\). Si cette tâche n'est que peu contraignante
pour l'exercice de modélisation effectué dans le \autoref{chap:09}, il
nous semble difficile de demander aux secouristes de paramétrer les
fuzzyficateurs durant la phase de secours.

Une première solution serait de définir un paramétrage par défaut des
fuzzyficateurs, mais cette solution n'est pas envisageable, un même
fuzzyficateur pouvant être couplé avec plusieurs métriques, qui ne
partagent donc pas leur unités.
%
On peut raffiner cette solution en rattachant la paramétrisation à la
relation de localisation et non plus au fuzzyficateur. Pour donner un
exemple, avec cette solution toute utilisation de la relation de
localisation \onto[orl]{Sous\-Altitude} utilisera le même paramétrage
implicite du fuzzyficateur \onto[orla]{Inf\-Val\-0}, mais ce
paramétrage pourra être différent lorsque ce fuzzyficateur est appelé
par une autre relation de localisation, comme
\onto[orla]{Dans\-Planimetrique}. Bien qu'elle soit plus
satisfaisante, cette seconde solution ne règle pas tous les problèmes
qui peuvent être posés par la question de la paramétrisation. Peut-on,
par exemple, considérer que la relation de localisation
\onto[orl]{Pres\-De} traduit le même ordre de grandeur lorsqu'elle
s'applique à une maison ou à une ville ? Des relations de localisation
comme \onto[orl]{Aux\-Alentours\-De} intègrent même cette notion dans
leur définition. Il nous semble donc indispensable de prendre
également en compte la taille, voir la saillance, de l'objet de
référence lors de la spatialisation.

Notre méthode de spatialisation permet par ailleurs, grâce aux
principes de décomposition et de modélisation autonome
(\autoref{chap:04}) de faire varier ce paramétrage pour chaque objet
de référence, ce qui permettrait d'aboutir à une modélisation beaucoup
plus fine.


% La sélection de la maille



% Seuils de la confiance

Système d'inférence flou ?

% Objets de référence flous

\tdi{Parler de l'avantage de manipuler directement des objets flous,
  par exemple pour la durée de déplacement (bourg d'oisans fil rouge}




%%% Local Variables:
%%% mode: latex
%%% TeX-master: "../../main"
%%% End:
