\begin{tabular}{p{.2\textheight}L{.35\textheight}L{.35\textheight}}
  \toprule
  \multicolumn{1}{c}{\bfseries Objectif scientifique} &
  \multicolumn{1}{c}{\bfseries Verrous scientifiques} &
  \multicolumn{1}{c}{\bfseries Apports envisagés} \\ \midrule
  % Lot 1
  \nameref{subsec:1-2-3-1}
& \begin{minipage}[t]{.35\textheight}
    \begin{itemize}
    \item Annotation, catégorisation et désambiguïsation des entités nommées
    \item Extraction et interprétation des relations sémantiques
    \end{itemize}
  \end{minipage} & \begin{minipage}[t]{.35\textheight}
    \begin{itemize}
    \item Définition d'une méthode d'annotation des entités nommées
    \item Définition d'une méthode permettant l'interprétation des
      relations sémantiques
    \end{itemize}
  \end{minipage} \\
  \addlinespace[.5cm]
  %
  \nameref{subsec:1-2-3-2}
  {\par\footnotesize\hspace{.25cm}$\longrightarrow$~Chapitre \ref{chap:02}}
& \begin{minipage}[t]{.35\textheight}
    \begin{itemize}
    \item Spatialisation des indices de localisation
    \item Modélisation des objets géographiques imprécis
    \end{itemize}
  \end{minipage}& \begin{minipage}[t]{.35\textheight}
    \begin{itemize}
    \item Définition d'une méthode de spatialisation
    \item Définition d'une méthode de prise en compte de l'imprécision
      des objets géographiques
    \end{itemize}
  \end{minipage} \\
  \addlinespace[.5cm]
  % Lot 
  \nameref{subsec:1-2-3-3}
& \begin{minipage}[t]{.35\textheight}
    \begin{itemize}
    \item Définir des méthodes permettant d'assister le secouriste
      dans sa prise de décision
    \item Visualisation de données imprécises
    \end{itemize}
  \end{minipage}& \begin{minipage}[t]{.35\textheight}
    \begin{itemize}
    \item Construction d'une interface d'aide à la décision
    \item Définition de méthodes permettant de prendre en compte
      l'imprécision lors de la visualisation
    \end{itemize}
  \end{minipage} \\
  \addlinespace[.5cm]
  %
  \nameref{subsec:1-2-3-4}
& \begin{minipage}[t]{.35\textheight}
    \begin{itemize}
    \item Structuration d'un référentiel de données métier
    \item Appareillement de données hétérogènes
    \end{itemize}
  \end{minipage}& \begin{minipage}[t]{.35\textheight}
    \begin{itemize}
    \item Définition de méthodes d'appareillement
    \item Construction d'un référentiel de données métier
    \end{itemize}
  \end{minipage}\\
  \addlinespace
  \bottomrule
\end{tabular}
