\addsec{Résumé}

Plus de 15 000 demandes de secours en montagne sont recensées chaque
année, sur le territoire français, en majorité durant les périodes
estivales et hivernales. Lorsque ces demandes concernent des
intervention à l'extérieur des dommaines skiables elles sont réalisées
par des corps spécialisés, comme les pelotons de Gendarmerie en
haute-montagne (PGHM). Ces secouristes spécialisés sont chargés de
l'opération de secours en tant que telle, mais également de sa
préparation, ce qui implique notamment de délimiter la zone
d'intervention et donc d'identifier la position de la victime. Cette
étape peut s'avérer difficile, car elle nécessite que le requérant
décrive le plus précisément possible sa position, malgré la fatigue,
la panique ou sa méconnaissance de la région. Grace à la
multiplication des téléphones dotés de GPS, des solutions de
geo-localisation plus fiables et précises ont été développées, mais
les secouristes doivent encore procéder fréquement à une localisaiton
manuelle, s'appuyant uniquement sur le discours du requêrant, leur
connaissances et leur expérience.

L'objectif de cette thèse est de proposer une méthode permettant
d'assister les secouristes dans la situation où la position du
requérant ne peut être identifiée que manuellement. Nous proposons de
développer une méthode permettant d'identifier les zones correspondant
à une description orale de position (ex. « Je suis à côté d'un lac »),
c'est-à-dire permettant de transformer une position exprimée dans un
référentiel indirect (une description orale) en une position exprimée
dans un référentiel direct, c'est-à-dire décrite par des coordonées,
qu'il est alors possible de cartographier. Le développement d'une
telle méthode se heurte a de nombreux verrous scientifiques, comme la
prise en compte de l'imprécision inhérente au langage naturel, des
potentielles erreurs de description ou l’identification de la
sémantique des prépositions utilisées pour décrire une position en
milieu montagneux.

\noident \textbf{Mots clés :} Truc, Machin bidude\par

\addsec{abstract}

Plus de 15 000 demandes de secours en montagne sont recensées chaque
année, sur le territoire français, en majorité durant les périodes
estivales et hivernales. Lorsque ces demandes concernent des
intervention à l'extérieur des dommaines skiables elles sont réalisées
par des corps spécialisés, comme les pelotons de Gendarmerie en
haute-montagne (PGHM). Ces secouristes spécialisés sont chargés de
l'opération de secours en tant que telle, mais également de sa
préparation, ce qui implique notamment de délimiter la zone
d'intervention et donc d'identifier la position de la victime. Cette
étape peut s'avérer difficile, car elle nécessite que le requérant
décrive le plus précisément possible sa position, malgré la fatigue,
la panique ou sa méconnaissance de la région. Grace à la
multiplication des téléphones dotés de GPS, des solutions de
geo-localisation plus fiables et précises ont été développées, mais
les secouristes doivent encore procéder fréquement à une localisaiton
manuelle, s'appuyant uniquement sur le discours du requêrant, leur
connaissances et leur expérience. 

L'objectif de cette thèse est de proposer une méthode permettant
d'assister les secouristes dans la situation où la position du
requérant ne peut être identifiée que manuellement. Nous proposons de
développer une méthode permettant d'identifier les zones correspondant
à une description orale de position (ex. « Je suis à côté d'un lac »),
c'est-à-dire permettant de transformer une position exprimée dans un
référentiel indirect (une description orale) en une position exprimée
dans un référentiel direct, c'est-à-dire décrite par des coordonées,
qu'il est alors possible de cartographier. Le développement d'une
telle méthode se heurte a de nombreux verrous scientifiques, comme la
prise en compte de l'imprécision inhérente au langage naturel, des
potentielles erreurs de description ou l’identification de la
sémantique des prépositions utilisées pour décrire une position en
milieu montagneux.

\noident \textbf{Keywords :} Truc, Machin bidude\par