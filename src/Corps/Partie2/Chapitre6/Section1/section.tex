
\tdi{On veut un modèle qui permet de prendre en compte des relations
  de localisation traduisant des limites assez nettes (en dessous de)
  et des relations avec des limites très continues (proches [voir
  concept similaire dans l'ontologie]).}

S'il est nécessaire de prendre de prendre en compte l'imprécision lors
de la \emph{spatialisation} des \emph{relations de localisation,} il
est également 

Tout d'abord il est nécessaire que la modélisation retenue permette de
rendre compte des variations de \emph{précision} inhérentes aux
différentes \emph{relations de localisation.} En effet, là où
certaines des relations définies dans \ac{orl} 

\tdi{Possibilité de construire des zones de localisation issues de
  relations de localisation non linéaire (en dessous de).}

De manière analogue, la progression de \emph{l'imprécision} peu être
assez différente en fonction de la \emph{relation de localisation}
modélisée. Une \emph{relation} comme \onto[orl]{Près\-De} voit, toutes
choses égales par ailleurs, son \textsl{degré de validité} diminuer
linéairement avec l'éloignement à \emph{l'objet de référence.} Alors
qu'une relation comme \onto[orl]{Au\-Des\-sus\-Al\-ti\-tu\-de}, dont
la validité dépend du relief, pourra évoluer de manière différente et
non linéaire.
%
La méthode développée devra donc arriver à modéliser aussi bien ces
deux cas.

\missingfigure{Figure explicitant les points 1 \& 2}

\tdi{Possibilité de fusionner (intersection et union). Nécessité que
l'union et l'intersection soient commutatives.}

La méthodologie de construction des \emph{zones de localisation}
présentée dans le \autoref{chap:04} et plus précisément la \emph{phase
  de fusion,} impose d'effectuer de combiner des \emph{zones de
  localisation} par des unions et des intersections. La modélisation
retenue doit donc permettre de réaliser ces opérations, mais également
garantir leur commutativité. En effet, l'utilisation d'un opérateur
d'union ou d'intersection non commutatif aurait pour conséquence de
modifier le résultat de la \emph{phase de fusion} en fonction de
l'ordre de traitement des \ac{zlc}. Par exemple, si l'on souhaite
construire la \emph{zone de localisation probable}
($\text{\textsl{zlp}}$) correspondant à l'ensemble d'indices de
localisation : \enquote{Je suis à côté d'une maison et dans une
  forêt}. Pour ce faire il est nécessaire d'identifier les positions
situées dans les \emph{zone de localisation compatibles} correspondant
au premier ($\text{\textsf{zlc}}_{i1}$) et au second
($\text{\textsf{zlc}}_{i2}$) indice, à l'aide d'un opérateur
d'intersection. Si cet opérateur n'est pas commutatif, alors
l'intersection de $\text{\textsf{zlc}}_{i1}$ avec
$\text{\textsf{zlc}}_{i2}$ pourra donner un résultat différent de
l'intersection de $\text{\textsf{zlc}}_{i2}$ avec
$\text{\textsf{zlc}}_{i1}$.

\tdi{Nécessité que les résultats soient facilement
  interprétables. Sémantique des résultats et pas trop de zones (par
  exemple beaucoup d'alpha cuts).}

Nous avons déjà présenté l'importance que nous accordons à
l'interprétabilité des résultats (\emph{principe d'intégration dans le
  contexte métier}).
%
Par conséquent nous souhaitons que la méthodologie retenue produise
des résultats qui soient interprétables.

\tdi{Stabilité du modèle ? (par exemple si l'on fait des intersection
  avec le modèle egg-yolk on doit choisir ce qu'est le jaune. C'est
  l'intersection des jaunes ? ou il suffit qu'il y a ait un jaune?
  est-ce qu'il faut introduire des valeurs intermédiaires (gris,
  lorsqu'on intersecte jaune et blanc) -> mais dans ce cas on sort du
  egg yolk, le modèle n'est pas stable alors). Le résultat doit
  pouvoir s'exprimer avec la même modélisation sans perte de
  sémantique et de finesse de modélisation.}

\missingfigure{Figure explicitant le point 5}

%%% Local Variables:
%%% mode: latex
%%% TeX-master: "../../../../main"
%%% End:
