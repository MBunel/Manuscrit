\begin{tikzpicture}
  \tikzset{et/.style={above,font=\footnotesize\vphantom{Ag}}}
  % 
  \node[inner sep=0pt, anchor=south west] (image) at (0,0){\includegraphics{./figures/AlphaCut_zenith.png}};
  % 
  \begin{scope}
    \node (P2) at ([yshift=-.5cm]image.south east) {};
    \node (P1) at ([yshift=-.5cm]image.south west) {};

    \node (rect) [anchor=north west, minimum width=1cm,minimum
    height=.25cm] at ([yshift=-.25cm]P1) {}; \path[draw=RdBu-9-1, line
    width=.4mm](rect.west) --([xshift=-1ex]rect.south) -- ([xshift=1ex]rect.north)
    -- (rect.east);
    \node[anchor=west, font=\tiny\vphantom{Ag}, text width = 4cm] at
    ([xshift=1ex]rect.east) {Ligne électrique utilisée comme
      \emph{objet de référence}};

    \node (rect2) [anchor=north west, minimum width=1cm,minimum
    height=.25cm] at ([xshift=5.5cm,yshift=-.25cm]P1)
    {};

    \node[anchor=west, minimum width=.5cm,minimum
    height=.25cm, fill=RdBu-9-8] (rect2-1) at (rect2.west) {};

    \node[anchor=south west, minimum width=.5cm,minimum
    height=.25cm, fill=RdBu-9-9] (rect2-2) at (rect2-1.north west) {};
    
    \node[anchor=north west, minimum width=.5cm,minimum
    height=.25cm, fill=RdBu-9-7] (rect2-3) at (rect2-1.south west) {};

    \node[anchor=west, font=\fontsize{5}{5.5}\selectfont] at
    ([xshift=.5ex]rect2-1.east) {$0< \mu < 1$};
    
    \node[anchor=west, font=\fontsize{5}{5.5}\selectfont] at
    ([xshift=.5ex]rect2-2.east) {$\mu = 1$};
    
    \node[anchor=west, font=\fontsize{5}{5.5}\selectfont] at
    ([xshift=.5ex]rect2-3.east) {$\mu > 0$};

    \draw[decorate,decoration={brace}] ([xshift=7.5ex]rect2-2.north
    east) -- ([xshift=7.5ex]rect2-3.south east);
    
    \node[anchor=west, font=\tiny\vphantom{Ag}, text width = 4cm] at
    ([xshift=6ex]rect2.east) {\emph{Alpha-cuts} pour les valeurs du
      \emph{degré d'appartenance} ($\mu$) à la \ac{zlp}};

    % Échelle
    \draw[-] (P2 |- -1cm,-1cm) --++ (-1,0) node[et,pos=.5] {\SI{500}{\meter}};
    % Légende détaillée
    \path (P1) -- (P2) node[pos=.5, yshift=-1cm] {\tiny Pour la légende détaillée du fond topographique voir \autoref{anx:topo_leg}. Sources: BD TOPO 2018, BD ALTI 2018.}; 
  \end{scope}
\end{tikzpicture}