Nous avons identifié cinq objectifs principaux pour ce travail de
recherche.

On trouvera une synthèse des objectifs scientifiques, ainsi que de
leurs verrous et des contributions attendues dans le
\autoref{tab:synthese_objectifs} page
\pageref{tab:synthese_objectifs}.

\subsection{Modélisation de localisations indirectes}
\label{subsec:2-1-1}

Le premier objectif de cette thèse est de développer une méthode
permettant d'identifier une position à partir de sa
description.

% Présentation de l'objectif
Comme nous l'expliquions précédement, notre thèse s'inscrit dans un
axe de recherche du projet Choucas dont l'objectif est de faciliter
les secouristes à identifier une position décrite. Cette tâche
nécessite d'arriver à définir des méthodes permettant de construire
une zone à partir de descriptions, telles que \enquote{je suis sous
  une route.} Cette tâche nécessite donc de travailler à
l'identification de la sémantique des termes utilisés pour décrire une
possition, puis d'en proposer une modélisation, à même de pouvoir
systématiser la construction d'une \enquote{zone} à partir d'une
description.

% Pas nécessaire de faire de l'analyse du langage naturel
Comme nous le mentionnons dans le chapitre précédent (\todo{mettre res
  du passage où j'explique cet aspect dans le chapitre 1}), notre
objectif n'est pas de déveloper une solution se sustituant aux
secouristes, mais de leur proposer une solution s'intégrant dans leur
processus métier actuel. Cela implique que nous pouvons toujours
compter sur la présence du secouriste pour effectuer certaines
actions. Parmi celles-ci, l'écoute et la compréhension du discours du
requérant. En effet, il n'est pas nécessaire, voire souhaitable
\footnote{Tout du moins du point de vue des pprofessionels du
  secours.}, d'automatiser entièrement le processus de localisation de
la victime. Un tel objectif nécessiterait par ailleurs de développer
des solutions d'analyse automatisée du langage parlé, en vue d'en
extraire les relations de localisation.

Cette étape n'est pas nécessaire dans notre cas, puisque nous pouvons
confier cette tache de \enquote{sémantisation} de la description orale
d'une position au secouriste. 

Notre travail se limitera donc à identifier les \enquote{types} de
descriptions utilisées dans notre contexte, de les exprimer dans un
vocabulaire controlé, puis d'en proposer une modélisation a même d'en
permettre la spatialisation \todo{Attention terme non défini, attendre
  chapitre 4 avant de l'introduire}.

\subsubsection{Contexte scientifique}

La question de l'indentification de positions décrites à fait l'objet
de différents travaux, aux visées applicatives diverses.



Enfin, dans une thématique plus proche de notre travail,  

Vasardini
Bloch
Brézil icc

\subsubsection{Verrous}

\paragraph{Peu de données et Faible recoupement}

présentés en détail dans le \autoref{chap:05}

Un second problème, intimement lié au manque de données, est leur
faible recoupement. En effet, les enregistrements d'alerte dont nous
disposons ne concerent jamais les mêmes zones. Et les mêmes points


\paragraph{Disparité des descriptions de localisation}

Deux personnes ne décrivent pas une position de la même façon

\paragraph{Descriptions non exhaustives}

Pas de présuppositions sur ce qui est dit.

le requérant ne dit pas tout

-> complexifie le raisonnement

\subsubsection{Apports}

\subsection{Prise en compte de l'imprécision}
\label{subsec:2-1-2}

La question de la spatialisation nécessiter cependant d'en s'en poser
une seconde, celle de la prise en compte de l'imprécision. En effet,
les positions que nous cherchons à spatialiser possèdent la
particularité d'être exprimées en langage naturel. Orn bien que nous
ne travaillons pas directement à l'interprétation du langage naturel,
\ie que nous ene faisons pas de TAL, nous sommes tout de même
confrontés à certaines de ces limites.

La principale d'entre-elle est la question de l'imprécison du langage
naturel. En effet, le langage naturel n'est pas un outil idéal pour
exprimer clairement des propositions, celle-ci sont toujours
soumisses à l'interpréatation et donc imprécises. Il en va
nécessairement de même pour les propositions décrivants dune ou des
positions? Ces dernières s'expriment en langage natuurel et son donc
soumisses à l'iumprécisions.

Dans les faits ,cela implique qu'une même phrase, un même mot,
peutvent -être utilisés pour décrire des localisations dans l'espace
différentes, voir même totalement incompatibles.

Cette nature du langage naturel peut poser plusieurs soucis. D'une
part il est impossible d'utiliser la langue comme un outil de
raisonement précis. C'est d'ailleurs ce constat qui a conduit les
philosophes analytiques, tels que Fredge ou Russel à travailler à la
formalisation de la logique, outil à même d'affranchir la réflexion
des limites du langage naturel.



\subsubsection{Contexte scientifique}

Travail sur le flou
Russell
Varzi
Zadeh
Dilo ?


\subsubsection{Verrous}

\paragraph{Quantification de l'imprécision}

\subsection{Prise en compte du jugement du secouriste}
\label{subsec:2-1-3}

\subsubsection{Contexte scientifique}

Travaux sur l'incertitude

Ana-Maria

\subsubsection{Verrous}

\paragraph{Quantification de l'incertitude}


\subsubsection{Apports envisagés}

\subsection{Agrégation d'indices}
\label{subsec:2-1-4}

\subsubsection{Contexte scientifique}

\subsubsection{Verrous}

\paragraph{Contexte différent}

\paragraph{Conflits}

\subsection{Évaluation des résulats}
\label{subsec:2-1-5}

Enfin, le dernier objectif scientifique de cette thèse consiste à
définir une méthode d'évaluation des résultats produits.

L'objectif de cette évaluation est de permettre une adaptation des
résultats en fonction 

\subsubsection{Contexte scientifique}

\subsubsection{Verrous}

\paragraph{Pas d'évaluation supervisée}


% Tableau synthétique
\begin{table}[h]
  \centering
  \begin{tabular}{L{5cm}L{4.5cm}L{4.5cm}} \toprule
\multicolumn{1}{c}{\bfseries Objectifs scientifiques} &
\multicolumn{1}{c}{\bfseries Verrous} & \multicolumn{1}{c}{\bfseries
Apports envisagés} \\ \midrule
  
  \nameref{subsec:2-1-1}
{\par\footnotesize\hspace{.25cm}$\longrightarrow$~Chapitre
\ref{chap:5}} & \begin{minipage}{4cm}
    \begin{itemize}
    \item This is item 1
    \item This is item 2
    \end{itemize}
  \end{minipage}& \begin{minipage}{4cm} \bigskip
    \begin{itemize}
    \item This is item 1
    \item This is item 2
    \item This is item 3
    \item This is item 3
    \item This is item 3
    \end{itemize} \bigskip
  \end{minipage} \\
  
  \nameref{subsec:2-1-2}
{\par\footnotesize\hspace{.25cm}$\longrightarrow$~Chapitre
\ref{chap:5}} & \begin{minipage}{4cm}
    \begin{itemize}
    \item This is item 1
    \item This is item 2
    \item This is item 3
    \end{itemize}
  \end{minipage} & \begin{minipage}{4cm}
    \begin{itemize}
    \item This is item 1
    \item This is item 2
    \item This is item 3
    \end{itemize}
  \end{minipage} \\
  
  \nameref{subsec:2-1-3}
{\par\footnotesize\hspace{.25cm}$\longrightarrow$~Chapitre
\ref{chap:5}} & \begin{minipage}{4cm}
    \begin{itemize}
    \item This is item 1
    \item This is item 2
    \end{itemize}
  \end{minipage}& \begin{minipage}{4cm}
    \begin{itemize}
    \item This is item 1
    \item This is item 2
    \item This is item 3
    \end{itemize}
  \end{minipage} \\
  
   \nameref{subsec:2-1-4}
{\par\footnotesize\hspace{.25cm}$\longrightarrow$~Chapitre
\ref{chap:9}} & \begin{minipage}{4cm}
    \begin{itemize}
    \item This is item 1
    \item This is item 2
    \end{itemize}
  \end{minipage}& \begin{minipage}{4cm}
    \begin{itemize}
    \item This is item 1
    \item This is item 2
    \item This is item 3
    \end{itemize}
  \end{minipage} \\
  
  \nameref{subsec:2-1-5}
{\par\footnotesize\hspace{.25cm}$\longrightarrow$~Chapitre
\ref{chap:11}} & \begin{minipage}{4cm}
    \begin{itemize}
    \item This is item 1
    \item This is item 2
    \end{itemize}
  \end{minipage}& \begin{minipage}{4cm}
    \begin{itemize}
    \item This is item 1
    \item This is item 2
    \item This is item 3
    \end{itemize}
  \end{minipage}\\
  
  \bottomrule
\end{tabular}

  \caption{Synthèse des verrous et des apports attendus pour chaque
    objectif scientifique de la thèse}
  \label{tab:synthese_objectifs}
\end{table}

%%% Local Variables:
%%% mode: latex
%%% TeX-master: "../../../../main"
%%% End:
