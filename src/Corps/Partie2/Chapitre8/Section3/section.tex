Pour illustrer le déroulement de la \emph{phase de fusion} nous nous
fonderons sur l'exemple de la construction de la \ac{zlp}
correspondant à \emph{l'ensemble des indices de localisation}
\enquote{Je suis à l'extérieur, au sud de XXX (qui a été spatialisé
  dans le \autoref{chap:07}) et je suis proche d'un refuge}
(\autoref{fig:exemple_final_fusion}). Pour les besoins de l'exemple,
nous considérerons que le secouriste a attribué une confiance absolue
au premier \emph{indice} \footnote{C'est le choix qui a implicitement
  été fait lors de la \emph{spatialisation} de cet \emph{indice} dans
  le \autoref{chap:05}.} et une confiance moyenne (\ie un coefficient
de \si{0,0}) au second. Le second \emph{indice de localisation}
utilisant un \emph{objet de référence indéfini} (\eg \enquote{un
  refuge}) nous avons considéré qu'il n'existait que deux objets
candidats, utilisés pour la spatialisation des \ac{zlc} XX et XX
(\autoref{fig:exemple_final_fusion}).

Au terme de la \emph{phase de spatialisation} (\autoref{chap:07}) nous
disposons donc de quatre \ac{zlc} :
% 
\begin{enumerate*}[label=(\arabic*)]
\item la \ac{zlc} \textcolor{RdBu-9-1}{\textsf{A}}, spatialisant la \emph{relation de localisation
    atomique} \onto[orla]{Hors\-De} pour \emph{l'objet de référence}
  XX,
\item la \ac{zlc} \textcolor{RdBu-9-3}{\textsf{B}}, spatialisant la \emph{relation de localisation
    atomique} \onto[orla]{Sud\-De}, également pour \emph{l'objet de
    référence} XXX,
\item la \ac{zlc} \textcolor{RdBu-9-7}{\textsf{C}}, spatialisant la \emph{relation de localisation
    atomique} \onto[orla]{XXX} pour la première des deux instances de
  type \enquote{refuge} identifiées, avec un degré de certitude de \si{0,0}, et
\item la \ac{zlc} \textcolor{RdBu-9-9}{\textsf{D}}, spatialisant,
  également avec une certitude de \si{0,0}, la \emph{relation de
    localisation atomique} \onto[orla]{XXX} pour le second refuge.
\end{enumerate*}
% 

\subsection{La \emph{fusion} des \emph{relations de localisation
    atomiques}}

La première étape de la \emph{phase de fusion} consiste en une
intersection des \ac{zlc} spatialisant une \emph{relation de
  localisation atomique} décomposant la même \emph{relation de
  localisation,} pour le même \emph{objet de référence,} issu du même
\emph{indice de localisation.} Dans l'exemple illustré par la
\autoref{fig:exemple_final_fusion} on retrouve deux configurations
différentes. Dans la partie supérieure de la figure se situent les
\ac{zlc} \textcolor{RdBu-9-1}{\textsf{A}} et
\textcolor{RdBu-9-3}{\textsf{B}}, spatialisant

\subsection{La \emph{fusion} des \emph{objets de référence indéfinis}}

\subsection{La \emph{fusion} des indices de \emph{localisation}}


\begin{landscape}
  \begin{figure}[H]
    \centering
    \begin{tikzpicture}
  % grille temp
  % \draw[step=1.0,black,thin] (0,0) grid (20,-12);
  % Arrow

  \newcommand{\calcangle}[4]{%
    % #1 nom de la varaible renvoyée
    % #2 coordonnées de l'objet de référence
    % #3 coordonnées du point visé
    % #4 coordonées de la position testée
    \pgfmathanglebetweenlines{#2}{#3}{#2}{#4}%
    \global\let#1\pgfmathresult%
  }

  
  \begin{scope}
    % \path[draw, -, shorten >=5pt, shorten <=5pt] (-1,0) -- (-5,-2);
    % \path[draw, ->,shorten >=5pt, shorten <=5pt] (8.25,3) -- (10,3);
    % \path[draw, -,shorten >=12pt, shorten <=5pt] (1,-3) |- (3.65,-1);
    % \path[draw, ->,shorten >=5pt, shorten <=5pt] (8.25,-1) -- (10,-1);
  \end{scope}


  %%%%%%%%%%%% 
  % Colonne 1
  %%%%%%%%%%%% 
  % 
  % 
  % Zone de localisation compatible 1
  \begin{scope}[local bounding box=zlc1]
    \begin{scope}
      \foreach \x in {0,.25,...,1.75}{ \foreach \y in {0,.25,...,1.75}
        {
          % Calcul de l'angle
          \calcangle{\angle}{\pgfpoint{0.125cm}{1.375cm}}{\pgfpoint{1.75cm}{0cm}}{\pgfpoint{\x cm}{\y cm}}
          % Fuzzyfication de l'angle
          \pgfmathsetmacro\fuzzy{%
            ifthenelse(\angle < 90,(90-\angle)/90,
            ifthenelse(\angle > 270,(\angle-270)/90,0)
            )
          }
          % Calcul du rayon à partir de la fuzzyfication
          \pgfmathsetmacro\radius{\fuzzy*2.5}
          \fill[RdBu-9-1] (\x+.125,.125+\y) circle (\radius pt);
        }
      }
      \path[ffc] (0,0) rectangle (2,2);
    \end{scope}
    \node[text width=3cm, align=center, anchor=north, font=\footnotesize] at (1,-.25)
    {\itshape Zone de localisation compatible \normalfont \textcolor{RdBu-9-1}{\textsf{A}}};
  \end{scope}
  % 
  % ZLC 2
  \begin{scope}[yshift=-3.5cm,local bounding box=zlc2]
    \begin{scope}
      \foreach \x in {0,.25,...,1.75}{
        \foreach \y in {0,.25,...,1.75}
        {
          % Calcul et représentation floue distance euclidienne
          \pgfmathsetmacro\dist{sqrt((\x-0.125)^2+(\y-1.375)^2)}
          % Fuzzyfication de la distance
          \pgfmathsetmacro\fuzzy{%
            ifthenelse(\dist < .25,1,%
            ifthenelse(\dist < 2,-0.57*\dist+1.14,0)%
            )
          }
          % Calcul du rayon à partir de la fuzzyfication
          \pgfmathsetmacro\radius{(1-\fuzzy)*2.5}
          \fill[RdBu-9-2] (\x+.125,.125+\y) circle (\radius pt);
        }
      }
      \path[ffc, RdBu-9-2] (0,0) rectangle (2,2);
    \end{scope}
    \node[text width=3cm, align=center, anchor=north, font=\footnotesize] at (1,-.25)
    {\itshape Zone de localisation compatible \normalfont \textcolor{RdBu-9-2}{\textsf{B}}};
  \end{scope}
  % 
  % ZlC 3
  \begin{scope}[yshift=-8cm,local bounding box=zlc3]
    \begin{scope}
      \foreach \x in {0,.25,...,1.75}{
        \foreach \y in {0,.25,...,1.75}
        {
          % Calcul et représentation floue distance euclidienne
          \pgfmathsetmacro\dist{sqrt((\x-0.125)^2+(\y-1.375)^2)}
          % Fuzzyfication de la distance
          \pgfmathsetmacro\fuzzy{%
            ifthenelse(\dist < .25,1,%
            ifthenelse(\dist < 2,-0.57*\dist+1.14,0)%
            )
          }
          % Calcul du rayon à partir de la fuzzyfication
          \pgfmathsetmacro\radius{max(\fuzzy,0.3)*2.5}
          \fill[RdBu-9-8] (\x+.125,.125+\y) circle (\radius pt);
        }
      }
      \path[ffc2, RdBu-9-8] (0,0) rectangle (2,2);
    \end{scope}
    \node[text width=3cm, align=center, anchor=north, font=\footnotesize] at (1,-.25)
    {\itshape Zone de localisation compatible \normalfont \textcolor{RdBu-9-8}{\textsf{C}}};
  \end{scope}
  % 
  % Zlc 4
  \begin{scope}[yshift=-11.5cm,local bounding box=zlc4]
    \begin{scope}
      \foreach \x in {0,.25,...,1.75}{
        \foreach \y in {0,.25,...,1.75}
        {
          % Calcul et représentation floue distance euclidienne
          \pgfmathsetmacro\dist{sqrt((\x-2.125)^2+(\y-1.875)^2)}
          % Fuzzyfication de la distance
          \pgfmathsetmacro\fuzzy{%
            ifthenelse(\dist < .25,1,%
            ifthenelse(\dist < 2,-0.57*\dist+1.14,0)%
            )
          }
          % Calcul du rayon à partir de la fuzzyfication
          \pgfmathsetmacro\radius{max(\fuzzy,0.3)*2.5}
          \fill[RdBu-9-9] (\x+.125,.125+\y) circle (\radius pt);
        }
      }
      \path[ffc2] (0,0) rectangle (2,2);
    \end{scope}
    \node[text width=3cm, align=center, anchor=north, font=\footnotesize] at (1,-.25)
    {\itshape Zone de localisation compatible \normalfont \textcolor{RdBu-9-9}{\textsf{D}}};
  \end{scope}
  %%%%%%%%%%%% 
  % Colonne 2
  %%%%%%%%%%%% 
  \begin{scope}[xshift=6cm, yshift=-1.75cm,local bounding box=zlc5]
    \begin{scope}
      \foreach \x in {0,.25,...,1.75}{
        \foreach \y in {0,.25,...,1.75}
        {
          % Calcul de l'angle
          \calcangle{\angle}{\pgfpoint{0.125cm}{1.375cm}}{\pgfpoint{1.75cm}{0cm}}{\pgfpoint{\x cm}{\y cm}}
          % Fuzzyfication de l'angle
          \pgfmathsetmacro\fuzzya{%
            ifthenelse(\angle < 90,(90-\angle)/90,
            ifthenelse(\angle > 270,(\angle-270)/90,0)
            )
          }
          %
          % Calcul et représentation floue distance euclidienne
          \pgfmathsetmacro\dist{sqrt((\x-0.125)^2+(\y-1.375)^2)}
          % Fuzzyfication de la distance
          \pgfmathsetmacro\fuzzyb{%
            ifthenelse(\dist < .25,1,%
            ifthenelse(\dist < 2,-0.57*\dist+1.14,0)%
            )
          }
          %
          \pgfmathsetmacro\radius{min(\fuzzya,1-\fuzzyb)*2.5}
          \fill[RdBu-9-1] (\x+.125,.125+\y) circle (\radius pt);
        }
      }
      \path[ffc] (0,0) rectangle (2,2);
    \end{scope}
    \node[text width=3cm, align=center, anchor=north, font=\footnotesize] at (1,-.25)
    {\itshape Zone de localisation compatible \normalfont \textcolor{RdBu-9-1}{\textsf{AB}}};
  \end{scope}

  \begin{scope}[xshift=6cm, yshift=-8cm,local bounding box=zlc6]
    \begin{scope}
      \foreach \x in {0,.25,...,1.75}{
        \foreach \y in {0,.25,...,1.75}
        {
          % Calcul et représentation floue distance euclidienne
          \pgfmathsetmacro\dist{sqrt((\x-0.125)^2+(\y-1.375)^2)}
          % Fuzzyfication de la distance
          \pgfmathsetmacro\fuzzy{%
            ifthenelse(\dist < .25,1,%
            ifthenelse(\dist < 2,-0.57*\dist+1.14,0)%
            )
          }
          % Calcul du rayon à partir de la fuzzyfication
          \pgfmathsetmacro\radius{max(\fuzzy,0.3)*2.5}
          \fill[RdBu-9-8] (\x+.125,.125+\y) circle (\radius pt);
        }
      }
      \path[ffc2, RdBu-9-8] (0,0) rectangle (2,2);
    \end{scope}
    \node[text width=3cm, align=center, anchor=north, font=\footnotesize] at (1,-.25)
    {\itshape Zone de localisation compatible \normalfont \textcolor{RdBu-9-8}{\textsf{C}}};
  \end{scope}
  
  \begin{scope}[xshift=6cm, yshift=-11.5cm]
    \begin{scope}
      \foreach \x in {0,.25,...,1.75}{
        \foreach \y in {0,.25,...,1.75}
        {
          % Calcul et représentation floue distance euclidienne
          \pgfmathsetmacro\dist{sqrt((\x-2.125)^2+(\y-1.875)^2)}
          % Fuzzyfication de la distance
          \pgfmathsetmacro\fuzzy{%
            ifthenelse(\dist < .25,1,%
            ifthenelse(\dist < 2,-0.57*\dist+1.14,0)%
            )
          }
          % Calcul du rayon à partir de la fuzzyfication
          \pgfmathsetmacro\radius{max(\fuzzy,0.3)*2.5}
          \fill[RdBu-9-9](\x+.125,.125+\y) circle (\radius pt);
        }
      }
      \path[ffc2] (0,0) rectangle (2,2);
    \end{scope}
    \node[text width=3cm, align=center, anchor=north, font=\footnotesize] at (1,-.25)
    {\itshape Zone de localisation compatible \normalfont \textcolor{RdBu-9-9}{\textsf{D}}};
  \end{scope}
  %%%%%%%%%%% 
  % Colonne 3
  %%%%%%%%%%%% 
  \begin{scope}[xshift=12cm, yshift=-1.75cm,local bounding box=zlc7]
    \begin{scope}
      \foreach \x in {0,.25,...,1.75}{
        \foreach \y in {0,.25,...,1.75}
        {
          % Calcul de l'angle
          \calcangle{\angle}{\pgfpoint{0.125cm}{1.375cm}}{\pgfpoint{1.75cm}{0cm}}{\pgfpoint{\x cm}{\y cm}}
          % Fuzzyfication de l'angle
          \pgfmathsetmacro\fuzzya{%
            ifthenelse(\angle < 90,(90-\angle)/90,
            ifthenelse(\angle > 270,(\angle-270)/90,0)
            )
          }
          %
          % Calcul et représentation floue distance euclidienne
          \pgfmathsetmacro\dist{sqrt((\x-0.125)^2+(\y-1.375)^2)}
          % Fuzzyfication de la distance
          \pgfmathsetmacro\fuzzyb{%
            ifthenelse(\dist < .25,1,%
            ifthenelse(\dist < 2,-0.57*\dist+1.14,0)%
            )
          }
          %
          \pgfmathsetmacro\radius{min(\fuzzya,1-\fuzzyb)*2.5}
          \fill[RdBu-9-1] (\x+.125,.125+\y) circle (\radius pt);
        }
      }
      \path[ffc] (0,0) rectangle (2,2);
    \end{scope}
    \node[text width=3cm, align=center, anchor=north, font=\footnotesize] at (1,-.25)
    {\itshape Zone de localisation compatible \normalfont \textcolor{RdBu-9-1}{\textsf{AB}}};
  \end{scope}
  % 
  \begin{scope}[xshift=12cm, yshift=-9.75cm,local bounding box=zlc8]
    \begin{scope}
      \foreach \x in {0,.25,...,1.75}{
        \foreach \y in {0,.25,...,1.75}
        {
          \pgfmathsetmacro\dista{sqrt((\x-0.125)^2+(\y-1.375)^2)}
          % Fuzzyfication de la distance
          \pgfmathsetmacro\fuzzya{%
            ifthenelse(\dista < .25,1,%
            ifthenelse(\dista < 2,-0.57*\dista+1.14,0)%
            )
          }
          % Calcul et représentation floue distance euclidienne
          \pgfmathsetmacro\distb{sqrt((\x-2.125)^2+(\y-1.875)^2)}
          % Fuzzyfication de la distance
          \pgfmathsetmacro\fuzzyb{%
            ifthenelse(\distb < .25,1,%
            ifthenelse(\distb < 2,-0.57*\distb+1.14,0)%
            )
          }
          \pgfmathsetmacro\radius{max(max(\fuzzya, 0.3), max(\fuzzyb, 0.3))*2.5}
          \fill[RdBu-9-9](\x+.125,.125+\y) circle (\radius pt);
        }
      }
      \path[ffc2] (0,0) rectangle (2,2);
    \end{scope}
    \node[text width=3cm, align=center, anchor=north, font=\footnotesize] at (1,-.25)
    {\itshape Zone de localisation compatible \normalfont \textcolor{RdBu-9-9}{\textsf{CD}}};
  \end{scope}
  %%%%%%%%%%% 
  % Colonne 4
  %%%%%%%%%%%% 
  \begin{scope}[xshift=18cm, yshift=-5.75cm,,local bounding box=zlc9]
    \begin{scope}
      \foreach \x in {0,.25,...,1.75}{
        \foreach \y in {0,.25,...,1.75}
        {          
          % Calcul de l'angle
          \calcangle{\angle}{\pgfpoint{0.125cm}{1.375cm}}{\pgfpoint{1.75cm}{0cm}}{\pgfpoint{\x cm}{\y cm}}
          % Fuzzyfication de l'angle
          \pgfmathsetmacro\fuzzya{%
            ifthenelse(\angle < 90,(90-\angle)/90,
            ifthenelse(\angle > 270,(\angle-270)/90,0)
            )
          }
          %
          % Calcul et représentation floue distance euclidienne
          \pgfmathsetmacro\dist{sqrt((\x-0.125)^2+(\y-1.375)^2)}
          % Fuzzyfication de la distance
          \pgfmathsetmacro\fuzzyb{%
            ifthenelse(\dist < .25,1,%
            ifthenelse(\dist < 2,-0.57*\dist+1.14,0)%
            )
          }
          \pgfmathsetmacro\dista{sqrt((\x-0.125)^2+(\y-1.375)^2)}
          % Fuzzyfication de la distance
          \pgfmathsetmacro\fuzzyc{%
            ifthenelse(\dista < .25,1,%
            ifthenelse(\dista < 2,-0.57*\dista+1.14,0)%
            )
          }
          % Calcul et représentation floue distance euclidienne
          \pgfmathsetmacro\distb{sqrt((\x-2.125)^2+(\y-1.875)^2)}
          % Fuzzyfication de la distance
          \pgfmathsetmacro\fuzzyd{%
            ifthenelse(\distb < .25,1,%
            ifthenelse(\distb < 2,-0.57*\distb+1.14,0)%
            )
          }
          \pgfmathsetmacro\radius{min(\fuzzya, 1-\fuzzyb, max(\fuzzyc, 0.3), max(\fuzzyd, 0.3))*2.5}          
          \fill[black] (\x+.125,.125+\y) circle (\radius pt);
        }
      }
      \path[ffc2, color=black] (0,0) rectangle (2,2);
    \end{scope}
    \node[text width=3cm, align=center, anchor=north, font=\footnotesize] at (1,-.25)
    {\itshape Zone de localisation probable};
  \end{scope}
  % Accolades
  \begin{scope}
    \draw (zlc1.north east) -| ($(zlc1.south east)!0.5!(zlc2.north
    east) + (.1,0)$) |- (zlc2.south east) node[pos=0, yshift=.2] (acc-1)
    {};

    \draw (zlc3.north east) -| ($(zlc3.south east)!0.5!(zlc4.north
    east) + (.1,0)$) |- (zlc4.south east) node[pos=0, yshift=.2] (acc-2)
    {};

    \draw (zlc5.north east) -| ($(zlc5.north east)!0.5!(zlc5.south
    east) + (.1,0)$) |- (zlc5.south east) node[pos=0, yshift=.2] (acc-3)
    {};

        \draw (zlc6.north east) -| ($(zlc6.south east)!0.5!(zlc7.north
    east) + (.1,0)$) |- (zlc7.south east) node[pos=0, yshift=.2] (acc-4)
    {};
    
  \end{scope}
  % Annotations
  \begin{scope}
    \begin{scope}
      \begin{scope}
        \path[ffc] (1.75cm,1.75cm) rectangle (2cm,2cm);
      \end{scope}
      \begin{scope}[yshift=-3.5cm]
        \path[ffc] (1.75cm,1.75cm) rectangle (2cm,2cm);
      \end{scope}
      \begin{scope}[xshift=6cm, yshift=-1.75cm]
        \path[ffc] (1.75cm,1.75cm) rectangle (2cm,2cm);
      \end{scope}
      \node[align=center,anchor=center] (zi) at (4,-1) {$\top(a,b)=c$};
      \path[->, draw] (2,2) -- (zi.west);
      \path[->, draw] (2,-1.5) -- (zi.west);
      \path[->, draw] (zi.east) -- (7.75,.25);
    \end{scope}
    % 
    \begin{scope}
      \begin{scope}[xshift=6cm, yshift=-8cm]
        \path[ffc] (1.75cm,1.75cm) rectangle (2cm,2cm);
      \end{scope}
      \begin{scope}[xshift=6cm, yshift=-11.5cm]
        \path[ffc] (1.75cm,1.75cm) rectangle (2cm,2cm);
      \end{scope}
      \begin{scope}[xshift=12cm, yshift=-9.75cm]
        \path[ffc] (1.75cm,1.75cm) rectangle (2cm,2cm);
      \end{scope}
      \node[align=center,anchor=center] (zou) at (12,-6) {$\bot(\textcolor{RdBu-9-8}{\textsf{C}},\textcolor{RdBu-9-9}{\textsf{D}})=\textcolor{RdBu-9-9}{\textsf{CD}}$};
      \path[->, draw] (8,-6) -- (zou.west);
      \path[->, draw] (8,-9.5) -- (zou.west);
      \path[->, draw] (zou.east) -- (13.75,-7.75);
    \end{scope}
    % 
    \begin{scope}
      \begin{scope}[xshift=12cm, yshift=-1.75cm]
        \path[ffc] (1.75cm,1.75cm) rectangle (2cm,2cm);
      \end{scope}
      \begin{scope}[xshift=18cm, yshift=-5.75cm]
        \path[ffc] (1.75cm,1.75cm) rectangle (2cm,2cm);
      \end{scope}
      \node[align=center,anchor=center] (zu) at (16,-4) {$\top(a,b)=c$};
      \path[->, draw] (14,.25) -- (zu.west);
      \path[->, draw] (14,-7.75) -- (zu.west);
      \path[->, draw] (zu.east) -- (19.75,-3.75);
    \end{scope}
  \end{scope}
\end{tikzpicture}
    \caption{Gros exemple}
    \label{fig:exemple_final_fusion}
  \end{figure}
\end{landscape}


%%% Local Variables:
%%% mode: latex
%%% TeX-master: "../../../../main"
%%% End:
