La \emph{fusion} est la dernière phase de notre méthodologie
(\autoref{chap:04}). C'est au cours de cette phase que les \emph{zones
  de localisation compatibles} construites durant la \emph{phase de
  spatialisation} (\autoref{chap:07}) sont \emph{fusionnées} de
manière à obtenir une \emph{zone de localisation probable,} figurant
le résultat de la méthodologie pour \emph{l'ensemble des indices de
  localisation} traités.

\subsection{Présentation des \emph{étapes} de la \emph{phase de
    fusion}}

Comme la \emph{phase de décomposition} (\autoref{chap:05}), la phase
de \emph{fusion} est scindable en trois étapes ---~\emph{fusion} des
\emph{relations de localisation atomiques,} \emph{fusion} des
\emph{objets de référence indéfinis} et \emph{fusion} des
\emph{indices de localisation}~--- effectuées successivement sur
différents ensembles de \emph{zones de localisation compatibles.}

Comme nous l'avons expliqué dans le \autoref{chap:04} chacune de ces
étapes est destinée à fusionner des \emph{zones de localisation
  compatibles} spatialisées à partir \emph{d'indices de localisation
  décomposés}

Les étapes de \emph{fusion} et de \emph{décomposition} se répondent
donc, comme l'illustre la \autoref{fig:methodo_1}

\tdi{Dire qu'on fait la comparaison entre rasters}


\begin{figure}
  \centering
  \begin{tikzpicture}
  \matrix [matrix of nodes,
  anchor=west,
  nodes={minimum size=.75cm}
  ] (zla0) at (0,0)
  {
    \draw[ffa,ffc] (0,0) circle (.3cm);&
    \draw[ffa,ffc] (0,0) circle (.3cm);\\
  };

  \matrix [matrix of nodes,
  anchor=west,
  nodes={minimum size=.75cm},
  right=.75cm of zla0
  ] (zla1)
  {
    \draw[ffa,ffc] (0,0) circle (.3cm);&
    \draw[ffa,ffc] (0,0) circle (.3cm);\\
  };


  \matrix [matrix of nodes,
  anchor=west,
  nodes={minimum size=.75cm},
  right=.75cm of zla1
  ] (zla2)
  {
    \draw[ffa,ffc] (0,0) circle (.3cm);&
    \draw[ffa,ffc] (0,0) circle (.3cm);\\
  };

  \matrix [matrix of nodes,
  anchor=west,
  nodes={minimum size=.75cm},
  right=.75cm of zla2
  ] (zla3)
  {
    \draw[ffa,ffc] (0,0) circle (.3cm);&
    \draw[ffa,ffc] (0,0) circle (.3cm);\\
  };


  % Ligne 2
  \matrix [matrix of nodes,
  anchor=north,
  nodes={minimum size=.75cm},
  below=2cm of zla0
  ] (zlb0)
  {
    \draw[ffa,ffc] (0,0) circle (.3cm);\\
  };

  \matrix [matrix of nodes,
  anchor=north,
  nodes={minimum size=.75cm},
  below=2cm of zla1
  ] (zlb1)
  {
    \draw[ffa,ffc] (0,0) circle (.3cm);\\
  };

  \matrix [matrix of nodes,
  anchor=north,
  nodes={minimum size=.75cm},
  below=2cm of zla2
  ] (zlb2)
  {
    \draw[ffa,ffc] (0,0) circle (.3cm);\\
  };
  \matrix [matrix of nodes,
  anchor=north,
  nodes={minimum size=.75cm},
  below=2cm of zla3
  ] (zlb3)
  {
    \draw[ffa,ffc] (0,0) circle (.3cm);\\
  };

  % Ligne 3
  \coordinate (zlb01) at ($(zlb0)!0.5!(zlb1)$);
  \coordinate (zlb23) at ($(zlb2)!0.5!(zlb3)$);

  
  \matrix [matrix of nodes,
  anchor=north,
  nodes={minimum size=.75cm},
  below=2cm of zlb01
  ] (zlc0)
  {
    \draw[ffa,ffc] (0,0) circle (.3cm);\\
  };

  \matrix [matrix of nodes,
  anchor=north,
  nodes={minimum size=.75cm},
  below=2cm of zlb23
  ] (zlc1)
  {
    \draw[ffa,ffc] (0,0) circle (.3cm);\\
  };

  % Ligne 4
    \coordinate (zlc01) at ($(zlc0)!0.5!(zlc1)$);
  
  \matrix [matrix of nodes,
  anchor=north,
  nodes={minimum size=.75cm},
  below=2cm of zlc01
  ] (zlc1)
  {
    \draw[ffa,ffc] (0,0) circle (.3cm);\\
  };


  % Accolades
  \foreach \m in {0,1,...,3} {
    \draw (zla\m.south west) |- ($(zla\m.south west)!0.5!(zla\m.south
    east) + (0,-.1)$) -| (zla\m.south east)  node[pos=0, yshift=.2]
    (zla\m-g) {};
    % 
    % \draw (zlb\m.north west) |- ($(zlb\m.north west)!0.5!(zlb\m.north
    % east) + (0,.1)$) -| (zlb\m.north east)  node[pos=0, yshift=.2]
    % (zlb\m-g) {};
    % 
    % \draw[->>, black,] ([yshift=-.2cm]zla\m.south) -- ([yshift=.2cm]zlb\m.north);
  }


  % \node[fill=white,align=center, font=\large\sffamily] at ($(zla0.south west)!0.5!(zlb5.north
  % east)$){\emph{Fusion} des \emph{relations de localisation atomiques}};
\end{tikzpicture}
  \caption{Méthode de \emph{fusion} d'un ensemble \emph{d'indices de localisation}}
  \label{fig:methodo_fusion}
\end{figure}



\begin{table}
  \centering
   \begin{tabular}{r>{\small}p{.35\textwidth}>{\small}p{.35\textwidth}}
  \toprule & \multicolumn{1}{c}{\ac{orl}} &
  \multicolumn{1}{c}{\ac{orla}} \\ \midrule
  \addlinespace
  Objectif & Recense et définit les \emph{relations de localisation} utilisées
  pour décrire une position dans le contexte de la localisation de
  personnes en montagne & Définit les \emph{relations de localisation
                          atomiques,} la décomposition des relations
  définies dans \ac{orl} et formalise le processus de
                          \emph{spatialisation} des \emph{relations
                          spatiales atomiques.}\\
  Contenu & Définition de 51 \emph{relations de localisation,}
            regroupées en 11 classes abstraites. & Définition de XX
                                                   \emph{relations de
                                                   localisation
                                                   atomiques,}
                                                   décomposant XX des
                                                   51 \emph{relations
                                                   de localisation}
                                                   définies dans \ac{orl}.\\
%  Hiérarchie & Blo & \\
  Modélisation & L'ensemble des concepts sont définis pour être facilement
       différentiables et proches de la perception humaine des
       localisations dans l'espace & Les \emph{relations de
                                     localisation atomiques} ne sont
                                     pas conçues pour être manipulées
                                     directement par les utilisateurs.
  Elles généralement plus abstraites que les \emph{relations de
                                     localisation} qu'elles
                                     décomposent.\\ 
  \bottomrule
\end{tabular}

   \caption{Synthèse des étapes de la \emph{phase de fusion}}
  \label{tab:comparaison_etapes_fusion}
\end{table}


\tdi{Transition Pour faire ces fusions on a besoin d'opérateurs}
%
La théorie des \emph{sous-ensembles flous,} que nous employons pour la
représentation des \emph{zones de localisation} (\autoref{chap:06}),
permet d'effectuer des opérations inter-ensemblistes à l'aide de
plusieurs opérateurs (\autoref{chap:03}). 

\subsection{Opérateurs}

Comme nous l'avons montré dans l'état de l'art (\autoref{sec:3-2}), il
existe plusieurs opérateurs permettant d'étendre les unions et les
intersections ensemblistes à la théorie des sous-ensembles
flous. Cependant, aucun opérateur n'offre la possibilité de conserver
toutes les caractéristiques de la théorie des ensembles et tous
ne possèdent pas les mêmes propriétés. Le choix des opérateurs
d'intersection et de fusion nécessite donc d'identifier les
propriétés nécessaires.

\tdi{Si ne pas passer les caractéristiques des opérateurs ici.}

Les différentes caractéristiques des opérateurs d'intersection et de
fusion ont été présentées dans le \autoref{chap:03}. Cependant nous ne
les avons pas détaillées 


\tdi{Nécessité d'utiliser des opérateurs duals}

Si leurs propriétés différent, ces quatres couples partagent la
caractéristique d'êtres \emph{duals} \autocite{Bouchon-Meunier2007},
ce qui signifie que pour tous \(x\) et \(y\) dont la valeur est
comprise entre 0 et 1, \emph{la négation d'une \emph{t-norme}} est
égale à la \emph{t-conorme} de la négation de \(x\) et de \(y\) et
inversement, soit :

\begin{equation}
  n(⊤(x,y)) = ⊥(n(x), n(y))
\end{equation}

\begin{equation}
  n(⊥(x,y)) = ⊤(n(x), n(y))
\end{equation}

Avec \(n\) la négation.

\tdi{Nécessité des opérateurs continus}

\tdi{Enlever les opérateurs drastiques car ils sont trop sévères et ne
permettent pas de conserver des positions dont le degré d'appartenance
n'est pas au moins un fois de 1. En plus si j'ai n indices il faut
qu'il y ait n-1 indices dont le dégré est de 1.}


\subsubsection{Les principes du tiers-exclu \& de la non-contradiction}

Comme les opérateurs de \textcite{Zadeh1965} la \emph{t-norme} et la
\emph{t-conorme} de \bsc{Łukasiewicz} sont \emph{continues.} De plus,
comme pour tout \(x \in ]0,1[\) : \(⊤_L(x,x) < x\) et
\( ⊥_L(x,x) > x\) ces opérateurs sont dits \emph{archimédiens}
\autocite{Bouchon-Meunier1995}, contrairement aux opérateurs de
\textcite{Zadeh1965}. Une autre caractéristique des opérateurs de
\bsc{Łukasiewicz} est que, contrairement aux opérateurs de
\textcite{Zadeh1965}, ils satisfont les lois de
\emph{non-contradiction} et du \emph{tiers exclu.} Ainsi, conformément
à la loi de \emph{non-contradiction,} l’application de la
\emph{t-norme} de \bsc{Łukasiewicz} à un degré appartenance et à sa
négation est toujours nulle :

\begin{equation}
  ⊤_L(x,1-x) = 0
\end{equation}

Quant à loi du \emph{tiers exclu,} celle-ci conduit à ce que
l’application de \emph{t-conorme} de \bsc{Łukasiewicz} à un degré
appartenance et à sa négation soit toujours égale à 1 :

\begin{equation}
  ⊥_L(x,1-x) = 1
\end{equation}

Par conséquent, l'intersection d'un sous-ensemble flou \(A\) (de
\(X\)) avec son complément \(A^C\) (\autoref{eq:comp}) avec ces
opérateurs est toujours égale à l'ensemble vide :

\begin{equation}
  A \cap A^C = \emptyset
\end{equation}

et l'union de ces deux sous-ensembles avec les opérateurs de
\bsc{Łukasiewicz} est toujours égale à l'ensemble net \(X\):

\begin{equation}
  A \cup A^C = X
\end{equation}

%%%

les opérateurs probabilistes
sont continus, les opérateurs de \textcite{Zadeh1965} ou de
\bsc{Łukasiewicz}. Ils ont également la caractéristique d'être
\emph{archimédiens stricts,} \ie que pour tout \(x < v\) et \(y < w\)
: \(⊤_P(x,y) < ⊤_P(v,w)\) et \(⊥_P(x,y) > ⊥_P(v,w)\)
\autocite{Bouchon-Meunier1995}. Cependant, comme les opérateurs de
\textcite{Zadeh1965}, les opérateur probabilistes ne satisfont pas les
lois de \emph{non-contradiction} et du \emph{tiers exclus.}


\tdi{Loi du tier-exclu}

On peut s'en convaincre à l'aide de la \autoref{fig:tiers-exclu}. La
\autoref{fig:tiers-exclu} représente le résultat de l'union d'un sous
ensemble-flou \textcolor{RdBu-9-1}{\textsf{A}} avec son complémentaire
\textcolor{RdBu-9-9}{\textsf{A\up{C}}}, en fonction de l'opérateur
utilisé. On peut remarquer deux comportements distincts. Les
opérateurs drastiques et de \bsc{Łukasiewicz} produisent un résultat
homogène, où l'ensemble des pixels possèdent un degré d'appartenance
de 1, alors que les opérateurs de \bsc{Zadeh} ou probabilistes
aboutissent à une zone non uniforme et contenant des degrés
d'appartenance inférieurs à 1.

\begin{figure}
  \centering
  \begin{tikzpicture}
  \def\decalageX{-.2}
  \def\decalageY{-.2}

  \newcommand{\calcangle}[4]{%
    % #1 nom de la varaible renvoyée
    % #2 coordonnées de l'objet de référence
    % #3 coordonnées du point visé
    % #4 coordonées de la position testée
    \pgfmathanglebetweenlines{#2}{#3}{#2}{#4}%
    \global\let#1\pgfmathresult%
  }

  % Arrow
  \begin{scope}
    \path[draw, -, shorten >=5pt, shorten <=5pt] (-1,0) -- (-5,-2);
    %\path[draw, ->,shorten >=5pt, shorten <=5pt] (8.25,3) -- (10,3);
    %\path[draw, -,shorten >=12pt, shorten <=5pt] (1,-3) |- (3.65,-1);
    %\path[draw, ->,shorten >=5pt, shorten <=5pt] (8.25,-1) -- (10,-1);
  \end{scope}
  % Fuzzyfication 1 (Sud)
  \begin{scope}[xshift=-2cm, local bounding box=fuzz]
    \begin{scope}
      \foreach \x in {0,.25,...,1.75}{ \foreach \y in {0,.25,...,1.75}
        {
          % Calcul de l'angle
          \calcangle{\angle}{\pgfpoint{1cm}{1cm}}{\pgfpoint{1cm}{0cm}}{\pgfpoint{\x cm}{\y cm}}
          % Fuzzyfication de l'angle
          \pgfmathsetmacro\fuzzy{%
            ifthenelse(\angle < 90,(90-\angle)/90,
            ifthenelse(\angle > 270,(\angle-270)/90,0)
            )
          }
          % Calcul du rayon à partir de la fuzzyfication
          \pgfmathsetmacro\radius{\fuzzy*2.5}
          \fill[RdBu-9-1] (\x+.125,.125+\y) circle (\radius pt);
        }
      }
      \path[ffc] (0,0) rectangle (2,2);
    \end{scope}
    \node[text width=3cm, align=center, anchor=north, font=\footnotesize] at (1,-.25)
    {\itshape Zone de localisation compatible \normalfont \textcolor{RdBu-9-1}{\textsf{A}}};
  \end{scope} 
  % Fuzzyfication 2
  \begin{scope}[xshift=2cm,local bounding box=fuzz2]
    \begin{scope}
      \foreach \x in {0,.25,...,1.75}{
        \foreach \y in {0,.25,...,1.75}
        {
          % Calcul de l'angle
          \calcangle{\angle}{\pgfpoint{1cm}{1cm}}{\pgfpoint{1cm}{0cm}}{\pgfpoint{\x cm}{\y cm}}
          % Fuzzyfication de l'angle
          \pgfmathsetmacro\fuzzy{%
            ifthenelse(\angle < 90,(90-\angle)/90,
            ifthenelse(\angle > 270,(\angle-270)/90,0)
            )
          }
          % Calcul du rayon à partir de la fuzzyfication
          \pgfmathsetmacro\radius{(1-\fuzzy)*2.5}
          \fill[RdBu-9-9] (\x+.125,.125+\y) circle (\radius pt);
        }
      }
      \path[ffc2] (0,0) rectangle (2,2);
    \end{scope}
    \node[text width=3cm, align=center, anchor=north, font=\footnotesize] at (1,-.25)
    {\itshape Zone de localisation compatible \normalfont \textcolor{RdBu-9-9}{\textsf{A\up{C}}}};
  \end{scope}
  % Fusion 1
  \begin{scope}[xshift=-6cm, yshift=-4cm,local bounding box=fus1]
    \begin{scope}
      \foreach \x in {0,.25,...,1.75}{
        \foreach \y in {0,.25,...,1.75}
        {
          % Calcul des angles
          \calcangle{\angle}{\pgfpoint{1cm}{1cm}}{\pgfpoint{1cm}{0cm}}{\pgfpoint{\x
              cm}{\y cm}}
          % Fuzzyfication des angles
          \pgfmathsetmacro\fuzzy{%
            ifthenelse(\angle < 90,(90-\angle)/90,
            ifthenelse(\angle > 270,(\angle-270)/90,0)
            )
          }          
          % fusion (zadeh)
          \pgfmathsetmacro\fus{max(\fuzzy, 1-\fuzzy)}          
          % Calcul du rayon à partir de la fuzzyfication
          \pgfmathsetmacro\radius{\fus*2.5}
          \fill[RdBu-9-9] (\x+.125,.125+\y) circle (\radius pt);
        }
      }
      \path[ffc2] (0,0) rectangle (2,2);
    \end{scope}
    \node[text width=3cm, align=center, anchor=north, font=\footnotesize] at (1,-.25)
    {\itshape Zone de localisation compatible \normalfont \textcolor{RdBu-9-9}{\textsf{B}}};
  \end{scope}
  % Fusion 2
  \begin{scope}[xshift=-2cm, yshift=-4cm,local bounding box=fus2]
    \begin{scope}
      \foreach \x in {0,.25,...,1.75}{
        \foreach \y in {0,.25,...,1.75}
        {
          % Calcul des angles
          \calcangle{\angle}{\pgfpoint{1cm}{1cm}}{\pgfpoint{1cm}{0cm}}{\pgfpoint{\x
              cm}{\y cm}}
          % Fuzzyfication des angles
          \pgfmathsetmacro\fuzzy{%
            ifthenelse(\angle < 90,(90-\angle)/90,
            ifthenelse(\angle > 270,(\angle-270)/90,0)
            )
          }
          % fusion (luka)
          \pgfmathsetmacro\fus{min(\fuzzy + (1-\fuzzy), 0)}          
          % Calcul du rayon à partir de la fuzzyfication
          \pgfmathsetmacro\radius{\fus*2.5}
          \fill[RdBu-9-9] (\x+.125,.125+\y) circle (\radius pt);
        }
      }
      \path[ffc2] (0,0) rectangle (2,2);
    \end{scope}
    \node[text width=3cm, align=center, anchor=north, font=\footnotesize] at (1,-.25)
    {\itshape Zone de localisation compatible \normalfont \textcolor{RdBu-9-9}{\textsf{B}}};
  \end{scope}
  % Fusion 3
  \begin{scope}[xshift=2cm, yshift=-4cm,local bounding box=fus3]
    \begin{scope}
      \foreach \x in {0,.25,...,1.75}{
        \foreach \y in {0,.25,...,1.75}
        {
          % Calcul des angles
          \calcangle{\angle}{\pgfpoint{1cm}{1cm}}{\pgfpoint{1cm}{0cm}}{\pgfpoint{\x
              cm}{\y cm}}
          % Fuzzyfication des angles
          \pgfmathsetmacro\fuzzy{%
            ifthenelse(\angle < 90,(90-\angle)/90,
            ifthenelse(\angle > 270,(\angle-270)/90,0)
            )
          }
          % fusion (zadeh)
          \pgfmathsetmacro\fus{(\fuzzy + (1-\fuzzy)) - \fuzzy * (1-\fuzzy)}          
          % Calcul du rayon à partir de la fuzzyfication
          \pgfmathsetmacro\radius{\fus*2.5}
          \fill[RdBu-9-9] (\x+.125,.125+\y) circle (\radius pt);
        }
      }
      \path[ffc2] (0,0) rectangle (2,2);
    \end{scope}
    \node[text width=3cm, align=center, anchor=north, font=\footnotesize] at (1,-.25)
    {\itshape Zone de localisation compatible \normalfont \textcolor{RdBu-9-9}{\textsf{B}}};
  \end{scope}
  % Fusion 4
  \begin{scope}[xshift=6cm, yshift=-4cm,local bounding box=fus4]
    \begin{scope}
      \foreach \x in {0,.25,...,1.75}{
        \foreach \y in {0,.25,...,1.75}
        {
          % Calcul des angles
          \calcangle{\angle}{\pgfpoint{1cm}{1cm}}{\pgfpoint{1cm}{0cm}}{\pgfpoint{\x
              cm}{\y cm}}
          % Fuzzyfication des angles
          \pgfmathsetmacro\fuzzy{%
            ifthenelse(\angle < 90,(90-\angle)/90,
            ifthenelse(\angle > 270,(\angle-270)/90,0)
            )
          }
          % fusion (drast)
          \pgfmathsetmacro\fus{%
            ifthenelse(\fuzzy < 0.001, (1-\fuzzy),
            ifthenelse((1-\fuzzy) < 0.001, \fuzzy, 1)
            )
          }          
          % Calcul du rayon à partir de la fuzzyfication
          \pgfmathsetmacro\radius{\fus*2.5}
          \fill[RdBu-9-9] (\x+.125,.125+\y) circle (\radius pt);
        }
      }
      \path[ffc2] (0,0) rectangle (2,2);
    \end{scope}
    \node[text width=3cm, align=center, anchor=north, font=\footnotesize] at (1,-.25)
    {\itshape Zone de localisation compatible \normalfont \textcolor{RdBu-9-9}{\textsf{B}}};
  \end{scope}
\end{tikzpicture}
  \caption{fig:tiersexclu}
  \label{fig:tiers-exclu}
\end{figure}

\tdi{Principe de non contradiction}

On peut s'en convaincre à l'aide de la
\autoref{fig:non-contradiction}. La \autoref{fig:non-contradiction}
représente le résultat de l'intersection d'un sous ensemble-flou
\textcolor{RdBu-9-1}{\textsf{A}} avec son complémentaire
\textcolor{RdBu-9-9}{\textsf{A\up{C}}}, en fonction de l'opérateur
utilisé. On peut remarquer deux comportements distincts. Les
opérateurs drastiques et de \bsc{Łukasiewicz} produisent un résultat
homogène, où l'ensemble des pixels possèdent un degré d'appartenance
nul, alors que les opérateurs de \bsc{Zadeh} ou probabilistes
aboutissent à une zone non uniforme et contenant des degrés
d'appartenance non nuls.


\begin{figure}
  \centering
  \begin{tikzpicture}
  % Arrow
  \begin{scope}
    \path[draw, -, shorten >=5pt, shorten <=5pt] (-1,0) -- (-5,-2);
    % \path[draw, ->,shorten >=5pt, shorten <=5pt] (8.25,3) -- (10,3);
    % \path[draw, -,shorten >=12pt, shorten <=5pt] (1,-3) |- (3.65,-1);
    % \path[draw, ->,shorten >=5pt, shorten <=5pt] (8.25,-1) -- (10,-1);
  \end{scope}
  % Fuzzyfication 1 (Sud)
  \begin{scope}[xshift=-2cm, local bounding box=fuzz]
    \begin{scope}
      \foreach \x in {0,.25,...,1.75}{ \foreach \y in {0,.25,...,1.75}
        {
          % Calcul et représentation floue distance euclidienne
          \pgfmathsetmacro\dist{sqrt((\x-0.125)^2+(\y-1.375)^2)}
          % Fuzzyfication de la distance
          \pgfmathsetmacro\fuzzy{%
            ifthenelse(\dist < .25,1,%
            ifthenelse(\dist < 2,-0.57*\dist+1.14,0)%
            )
          }
          % Calcul du rayon à partir de la fuzzyfication
          \pgfmathsetmacro\radius{\fuzzy*2.5}
          \fill[RdBu-9-1] (\x+.125,.125+\y) circle (\radius pt);
        }
      }
      \path[ffc] (0,0) rectangle (2,2);
    \end{scope}
    \node[text width=3cm, align=center, anchor=north, font=\footnotesize] at (1,-.25)
    {\itshape Zone de localisation compatible \normalfont \textcolor{RdBu-9-1}{\textsf{A}}};
  \end{scope} 
  % Fuzzyfication 2
  \begin{scope}[xshift=2cm,local bounding box=fuzz2]
    \begin{scope}
      \foreach \x in {0,.25,...,1.75}{
        \foreach \y in {0,.25,...,1.75}
        {
          % Calcul et représentation floue distance euclidienne
          \pgfmathsetmacro\dist{sqrt((\x-0.125)^2+(\y-1.375)^2)}
          % Fuzzyfication de la distance
          \pgfmathsetmacro\fuzzy{%
            ifthenelse(\dist < .25,1,%
            ifthenelse(\dist < 2,-0.57*\dist+1.14,0)%
            )
          }
          % Calcul du rayon à partir de la fuzzyfication
          \pgfmathsetmacro\radius{(1-\fuzzy)*2.5}
          \fill[RdBu-9-9] (\x+.125,.125+\y) circle (\radius pt);
        }
      }
      \path[ffc2] (0,0) rectangle (2,2);
    \end{scope}
    \node[text width=3cm, align=center, anchor=north, font=\footnotesize] at (1,-.25)
    {\itshape Zone de localisation compatible \normalfont \textcolor{RdBu-9-9}{\textsf{A\up{C}}}};
  \end{scope}
  % Fusion 1
  \begin{scope}[xshift=-6cm, yshift=-4cm,local bounding box=fus1]
    \begin{scope}
      \foreach \x in {0,.25,...,1.75}{
        \foreach \y in {0,.25,...,1.75}
        {
          % Calcul et représentation floue distance euclidienne
          \pgfmathsetmacro\dist{sqrt((\x-0.125)^2+(\y-1.375)^2)}
          % Fuzzyfication de la distance
          \pgfmathsetmacro\fuzzy{%
            ifthenelse(\dist < .25,1,%
            ifthenelse(\dist < 2,-0.57*\dist+1.14,0)%
            )
          }
          \pgfmathsetmacro\fus{min(\fuzzy, 1-\fuzzy)}          
          % Calcul du rayon à partir de la fuzzyfication
          \pgfmathsetmacro\radius{\fus*2.5}
          \fill[black] (\x+.125,.125+\y) circle (\radius pt);
        }
      }
      \path[ffc, black] (0,0) rectangle (2,2);
    \end{scope}
    \node[text width=3cm, align=center, anchor=north, font=\footnotesize] at (1,-.25)
    {\itshape Intersection avec la \emph{t-norme} de \bsc{Zadeh}};
  \end{scope}
  % Fusion 2
  \begin{scope}[xshift=-2cm, yshift=-4cm,local bounding box=fus2]
    \begin{scope}
      \foreach \x in {0,.25,...,1.75}{
        \foreach \y in {0,.25,...,1.75}
        {
          % Calcul et représentation floue distance euclidienne
          \pgfmathsetmacro\dist{sqrt((\x-0.125)^2+(\y-1.375)^2)}
          % Fuzzyfication de la distance
          \pgfmathsetmacro\fuzzy{%
            ifthenelse(\dist < .25,1,%
            ifthenelse(\dist < 2,-0.57*\dist+1.14,0)%
            )
          }
          \pgfmathsetmacro\fus{max(\fuzzy + (1-\fuzzy) -1, 0)}          
          % Calcul du rayon à partir de la fuzzyfication
          \pgfmathsetmacro\radius{\fus*2.5}
          \fill[black] (\x+.125,.125+\y) circle (\radius pt);
        }
      }
      \path[ffc, black] (0,0) rectangle (2,2);
    \end{scope}
    \node[text width=3cm, align=center, anchor=north, font=\footnotesize] at (1,-.25)
    {\itshape Intersection avec la \emph{t-norme} de \bsc{Łukasiewicz}};
  \end{scope}
  % Fusion 3
  \begin{scope}[xshift=2cm, yshift=-4cm,local bounding box=fus3]
    \begin{scope}
      \foreach \x in {0,.25,...,1.75}{
        \foreach \y in {0,.25,...,1.75}
        {
          % Calcul et représentation floue distance euclidienne
          \pgfmathsetmacro\dist{sqrt((\x-0.125)^2+(\y-1.375)^2)}
          % Fuzzyfication de la distance
          \pgfmathsetmacro\fuzzy{%
            ifthenelse(\dist < .25,1,%
            ifthenelse(\dist < 2,-0.57*\dist+1.14,0)%
            )
          }
          \pgfmathsetmacro\fus{\fuzzy * (1-\fuzzy)}          
          % Calcul du rayon à partir de la fuzzyfication
          \pgfmathsetmacro\radius{\fus*2.5}
          \fill[black] (\x+.125,.125+\y) circle (\radius pt);
        }
      }
      \path[ffc, black] (0,0) rectangle (2,2);
    \end{scope}
    \node[text width=3cm, align=center, anchor=north, font=\footnotesize] at (1,-.25)
    {\itshape Intersection avec la \emph{t-norme} probabiliste};
  \end{scope}
  % Fusion 4
  \begin{scope}[xshift=6cm, yshift=-4cm,local bounding box=fus4]
    \begin{scope}
      \foreach \x in {0,.25,...,1.75}{
        \foreach \y in {0,.25,...,1.75}
        {
          % Calcul et représentation floue distance euclidienne
          \pgfmathsetmacro\dist{sqrt((\x-0.125)^2+(\y-1.375)^2)}
          % Fuzzyfication de la distance
          \pgfmathsetmacro\fuzzy{%
            ifthenelse(\dist < .25,1,%
            ifthenelse(\dist < 2,-0.57*\dist+1.14,0)%
            )
          }
          % fusion (drast)
          \pgfmathsetmacro\fus{%
            ifthenelse(\fuzzy > 0.999, (1-\fuzzy),
            ifthenelse((1-\fuzzy)  >0.999, \fuzzy, 0)
            )
          }          
          % Calcul du rayon à partir de la fuzzyfication
          \pgfmathsetmacro\radius{\fus*2.5}
          \fill[black] (\x+.125,.125+\y) circle (\radius pt);
        }
      }
      \path[ffc, black] (0,0) rectangle (2,2);
    \end{scope}
    \node[text width=3cm, align=center, anchor=north, font=\footnotesize] at (1,-.25)
    {\itshape Intersection avec la \emph{t-norme} drastique};
  \end{scope}
\end{tikzpicture}
  \caption{non contradiction}
  \label{fig:non-contradiction}
\end{figure}

La validation des principes du tiers-exclus et de non-contradiction
n'est pas nécessairement une condition \emph{sine qua non} à
l'utilisation d'opérateurs. 



Concepts non excluant

On a pas eu de problèmes du fait que cette loi ne soit pas validée

On peut trouver des exemples réels de cas ou ce n'est pas vrai

Lister les opérateurs qui valident tout

Caractéristiques non excluantes

Discuter du fait que ce soit mieux que ces principes ne soit pas
validés.

\subsubsection{L'archimédianité des opérateurs}

Une autre propriété différenciant les opérateurs est leur
archimédianté.

Mettre une figure de l'intersection d'angle avec les opérateurs
archimédiens (ex. sud et sud-est).

Parler l'exemple de l'union, je suis à côté d'un lac et plus vrai si
on est a côté de deux lac -> wtf

Ce critère discalifie lukacevicz et probabiliste

L'archimédianité a des impliquations sémantiques importantes.

Pour illustrer ce phénomène on peut se questionner sur l'impact de
cette propriété sur la fusion de \emph{zones de localisation
  compatibles.}  Les figures \ref{fig:comparaison_operateurs_union} et
\ref{fig:comparaison_operateurs_intersection} représentent le résultat
de l'union et de l'intersection des deux mêmes \emph{zones de
  localisation compatibles.} La \ac{zlc}
\textcolor{RdBu-9-1}{\textsf{A}} représente la \emph{spatialisation}
de \emph{l'indice de localisation} \enquote{au sud de} et la \ac{zlc}
\textcolor{RdBu-9-9}{\textsf{B}} de \emph{l'indice de localisation}
\enquote{à l'est de}.

\begin{figure}
  \centering
  \begin{tikzpicture}
  \def\decalageX{-.2}
  \def\decalageY{-.2}

  \newcommand{\calcangle}[4]{%
    % #1 nom de la varaible renvoyée
    % #2 coordonnées de l'objet de référence
    % #3 coordonnées du point visé
    % #4 coordonées de la position testée
    \pgfmathanglebetweenlines{#2}{#3}{#2}{#4}%
    \global\let#1\pgfmathresult%
  }

  % Arrow
  \begin{scope}
    \path[draw, -, shorten >=5pt, shorten <=5pt] (-1,0) -- (-5,-2);
    %\path[draw, ->,shorten >=5pt, shorten <=5pt] (8.25,3) -- (10,3);
    %\path[draw, -,shorten >=12pt, shorten <=5pt] (1,-3) |- (3.65,-1);
    %\path[draw, ->,shorten >=5pt, shorten <=5pt] (8.25,-1) -- (10,-1);
  \end{scope}
  % Fuzzyfication 1 (Sud)
  \begin{scope}[xshift=-2cm, local bounding box=fuzz]
    \begin{scope}
      \foreach \x in {0,.25,...,1.75}{ \foreach \y in {0,.25,...,1.75}
        {
          % Calcul de l'angle
          \calcangle{\angle}{\pgfpoint{1cm}{1cm}}{\pgfpoint{1cm}{0cm}}{\pgfpoint{\x cm}{\y cm}}
          % Fuzzyfication de l'angle
          \pgfmathsetmacro\fuzzy{%
            ifthenelse(\angle < 90,(90-\angle)/90,
            ifthenelse(\angle > 270,(\angle-270)/90,0)
            )
          }
          % Calcul du rayon à partir de la fuzzyfication
          \pgfmathsetmacro\radius{\fuzzy*2.5}
          \fill[RdBu-9-1] (\x+.125,.125+\y) circle (\radius pt);
        }
      }
      \path[ffc] (0,0) rectangle (2,2);
    \end{scope}
    \node[text width=3cm, align=center, anchor=north, font=\footnotesize] at (1,-.25)
    {\itshape Zone de localisation compatible \normalfont \textcolor{RdBu-9-1}{\textsf{A}}};
  \end{scope} 
  % Fuzzyfication 2
  \begin{scope}[xshift=2cm,local bounding box=fuzz2]
    \begin{scope}
      \foreach \x in {0,.25,...,1.75}{
        \foreach \y in {0,.25,...,1.75}
        {
          % Calcul de l'angle
          \calcangle{\angle}{\pgfpoint{1cm}{1cm}}{\pgfpoint{2cm}{1cm}}{\pgfpoint{\x cm}{\y cm}}
          % Fuzzyfication de l'angle
          \pgfmathsetmacro\fuzzy{%
            ifthenelse(\angle < 90,(90-\angle)/90,
            ifthenelse(\angle > 270,(\angle-270)/90,0)
            )
          }
          % Calcul du rayon à partir de la fuzzyfication
          \pgfmathsetmacro\radius{\fuzzy*2.5}
          \fill[RdBu-9-9] (\x+.125,.125+\y) circle (\radius pt);
        }
      }
      \path[ffc2] (0,0) rectangle (2,2);
    \end{scope}
    \node[text width=3cm, align=center, anchor=north, font=\footnotesize] at (1,-.25)
    {\itshape Zone de localisation compatible \normalfont \textcolor{RdBu-9-9}{\textsf{B}}};
  \end{scope}
  % Fusion 1
  \begin{scope}[xshift=-6cm, yshift=-4cm,local bounding box=fus1]
    \begin{scope}
      \foreach \x in {0,.25,...,1.75}{
        \foreach \y in {0,.25,...,1.75}
        {
          % Calcul des angles
          \calcangle{\angleA}{\pgfpoint{1cm}{1cm}}{\pgfpoint{1cm}{0cm}}{\pgfpoint{\x
              cm}{\y cm}}
          % Fuzzyfication des angles
          \pgfmathsetmacro\fuzzyA{%
            ifthenelse(\angleA < 90,(90-\angleA)/90,
            ifthenelse(\angleA > 270,(\angleA-270)/90,0)
            )
          }
          \calcangle{\angleB}{\pgfpoint{1cm}{1cm}}{\pgfpoint{2cm}{1cm}}{\pgfpoint{\x cm}{\y cm}}
          \pgfmathsetmacro\fuzzyB{%
            ifthenelse(\angleB < 90,(90-\angleB)/90,
            ifthenelse(\angleB > 270,(\angleB-270)/90,0)
            )
          }
          % fusion (zadeh)
          \pgfmathsetmacro\fus{max(\fuzzyA, \fuzzyB)}          
          % Calcul du rayon à partir de la fuzzyfication
          \pgfmathsetmacro\radius{\fus*2.5}
          \fill[RdBu-9-9] (\x+.125,.125+\y) circle (\radius pt);
        }
      }
      \path[ffc2] (0,0) rectangle (2,2);
    \end{scope}
    \node[text width=3cm, align=center, anchor=north, font=\footnotesize] at (1,-.25)
    {\itshape Zone de localisation compatible \normalfont \textcolor{RdBu-9-9}{\textsf{B}}};
  \end{scope}
  % Fusion 2
  \begin{scope}[xshift=-2cm, yshift=-4cm,local bounding box=fus2]
    \begin{scope}
      \foreach \x in {0,.25,...,1.75}{
        \foreach \y in {0,.25,...,1.75}
        {
          % Calcul des angles
          \calcangle{\angleA}{\pgfpoint{1cm}{1cm}}{\pgfpoint{1cm}{0cm}}{\pgfpoint{\x
              cm}{\y cm}}
          % Fuzzyfication des angles
          \pgfmathsetmacro\fuzzyA{%
            ifthenelse(\angleA < 90,(90-\angleA)/90,
            ifthenelse(\angleA > 270,(\angleA-270)/90,0)
            )
          }
          \calcangle{\angleB}{\pgfpoint{1cm}{1cm}}{\pgfpoint{2cm}{1cm}}{\pgfpoint{\x cm}{\y cm}}
          \pgfmathsetmacro\fuzzyB{%
            ifthenelse(\angleB < 90,(90-\angleB)/90,
            ifthenelse(\angleB > 270,(\angleB-270)/90,0)
            )
          }
          % fusion (luka)
          \pgfmathsetmacro\fus{min(\fuzzyA + \fuzzyB, 1)}          
          % Calcul du rayon à partir de la fuzzyfication
          \pgfmathsetmacro\radius{\fus*2.5}
          \fill[RdBu-9-9] (\x+.125,.125+\y) circle (\radius pt);
        }
      }
      \path[ffc2] (0,0) rectangle (2,2);
    \end{scope}
    \node[text width=3cm, align=center, anchor=north, font=\footnotesize] at (1,-.25)
    {\itshape Zone de localisation compatible \normalfont \textcolor{RdBu-9-9}{\textsf{B}}};
  \end{scope}
  % Fusion 3
  \begin{scope}[xshift=2cm, yshift=-4cm,local bounding box=fus3]
    \begin{scope}
      \foreach \x in {0,.25,...,1.75}{
        \foreach \y in {0,.25,...,1.75}
        {
          % Calcul des angles
          \calcangle{\angleA}{\pgfpoint{1cm}{1cm}}{\pgfpoint{1cm}{0cm}}{\pgfpoint{\x
              cm}{\y cm}}
          % Fuzzyfication des angles
          \pgfmathsetmacro\fuzzyA{%
            ifthenelse(\angleA < 90,(90-\angleA)/90,
            ifthenelse(\angleA > 270,(\angleA-270)/90,0)
            )
          }
          \calcangle{\angleB}{\pgfpoint{1cm}{1cm}}{\pgfpoint{2cm}{1cm}}{\pgfpoint{\x cm}{\y cm}}
          \pgfmathsetmacro\fuzzyB{%
            ifthenelse(\angleB < 90,(90-\angleB)/90,
            ifthenelse(\angleB > 270,(\angleB-270)/90,0)
            )
          }
          % fusion (zadeh)
          \pgfmathsetmacro\fus{(\fuzzyA + \fuzzyB) - \fuzzyA * \fuzzyB}          
          % Calcul du rayon à partir de la fuzzyfication
          \pgfmathsetmacro\radius{\fus*2.5}
          \fill[RdBu-9-9] (\x+.125,.125+\y) circle (\radius pt);
        }
      }
      \path[ffc2] (0,0) rectangle (2,2);
    \end{scope}
    \node[text width=3cm, align=center, anchor=north, font=\footnotesize] at (1,-.25)
    {\itshape Zone de localisation compatible \normalfont \textcolor{RdBu-9-9}{\textsf{B}}};
  \end{scope}
  % Fusion 4
  \begin{scope}[xshift=6cm, yshift=-4cm,local bounding box=fus4]
    \begin{scope}
      \foreach \x in {0,.25,...,1.75}{
        \foreach \y in {0,.25,...,1.75}
        {
          % Calcul des angles
          \calcangle{\angleA}{\pgfpoint{1cm}{1cm}}{\pgfpoint{1cm}{0cm}}{\pgfpoint{\x
              cm}{\y cm}}
          % Fuzzyfication des angles
          \pgfmathsetmacro\fuzzyA{%
            ifthenelse(\angleA < 90,(90-\angleA)/90,
            ifthenelse(\angleA > 270,(\angleA-270)/90,0)
            )
          }
          \calcangle{\angleB}{\pgfpoint{1cm}{1cm}}{\pgfpoint{2cm}{1cm}}{\pgfpoint{\x cm}{\y cm}}
          \pgfmathsetmacro\fuzzyB{%
            ifthenelse(\angleB < 90,(90-\angleB)/90,
            ifthenelse(\angleB > 270,(\angleB-270)/90,0)
            )
          }
          % fusion (drast)
          \pgfmathsetmacro\fus{%
            ifthenelse(\fuzzyA < 0.001, \fuzzyB,
            ifthenelse(\fuzzyB < 0.001, \fuzzyA, 1)
            )
          }          
          % Calcul du rayon à partir de la fuzzyfication
          \pgfmathsetmacro\radius{\fus*2.5}
          \fill[RdBu-9-9] (\x+.125,.125+\y) circle (\radius pt);
        }
      }
      \path[ffc2] (0,0) rectangle (2,2);
    \end{scope}
    \node[text width=3cm, align=center, anchor=north, font=\footnotesize] at (1,-.25)
    {\itshape Zone de localisation compatible \normalfont \textcolor{RdBu-9-9}{\textsf{B}}};
  \end{scope}
\end{tikzpicture}
  \caption{sq}
  \label{fig:comparaison_operateurs_union}
\end{figure}

\begin{figure}
  \centering
  \begin{tikzpicture}
  \def\decalageX{-.2}
  \def\decalageY{-.2}

  \newcommand{\calcangle}[4]{%
    % #1 nom de la varaible renvoyée
    % #2 coordonnées de l'objet de référence
    % #3 coordonnées du point visé
    % #4 coordonées de la position testée
    \pgfmathanglebetweenlines{#2}{#3}{#2}{#4}%
    \global\let#1\pgfmathresult%
  }

  % Arrow
  % \begin{scope}
  %   \path[draw, -, shorten >=12pt, shorten <=5pt] (1,3) |- (3.65,3);
  %   \path[draw, ->,shorten >=5pt, shorten <=5pt] (8.25,3) -- (10,3);
  %   \path[draw, -,shorten >=12pt, shorten <=5pt] (1,-3) |- (3.65,-1);
  %   \path[draw, ->,shorten >=5pt, shorten <=5pt] (8.25,-1) -- (10,-1);
  % \end{scope}
  % Fuzzyfication 1 (Sud)
  \begin{scope}[xshift=-2cm, local bounding box=fuzz]
    \begin{scope}
      \foreach \x in {0,.25,...,1.75}{ \foreach \y in {0,.25,...,1.75}
        {
          % Calcul de l'angle
          \calcangle{\angle}{\pgfpoint{1cm}{1cm}}{\pgfpoint{1cm}{0cm}}{\pgfpoint{\x cm}{\y cm}}
          % Fuzzyfication de l'angle
          \pgfmathsetmacro\fuzzy{%
            ifthenelse(\angle < 90,(90-\angle)/90,
            ifthenelse(\angle > 270,(\angle-270)/90,0)
            )
          }
          % Calcul du rayon à partir de la fuzzyfication
          \pgfmathsetmacro\radius{\fuzzy*2.5}
          \fill[RdBu-9-1] (\x+.125,.125+\y) circle (\radius pt);
        }
      }
      \path[ffc] (0,0) rectangle (2,2);
    \end{scope}
    \node[text width=3cm, align=center, anchor=north, font=\footnotesize] at (1,-.25)
    {\itshape Zone de localisation compatible \normalfont \textcolor{RdBu-9-1}{\textsf{A}}};
  \end{scope} 
  % Fuzzyfication 2
  \begin{scope}[xshift=2cm,local bounding box=fuzz2]
    \begin{scope}
      \foreach \x in {0,.25,...,1.75}{
        \foreach \y in {0,.25,...,1.75}
        {
          % Calcul de l'angle
          \calcangle{\angle}{\pgfpoint{1cm}{1cm}}{\pgfpoint{2cm}{1cm}}{\pgfpoint{\x cm}{\y cm}}
          % Fuzzyfication de l'angle
          \pgfmathsetmacro\fuzzy{%
            ifthenelse(\angle < 90,(90-\angle)/90,
            ifthenelse(\angle > 270,(\angle-270)/90,0)
            )
          }
          % Calcul du rayon à partir de la fuzzyfication
          \pgfmathsetmacro\radius{\fuzzy*2.5}
          \fill[RdBu-9-9] (\x+.125,.125+\y) circle (\radius pt);
        }
      }
      \path[ffc2] (0,0) rectangle (2,2);
    \end{scope}
    \node[text width=3cm, align=center, anchor=north, font=\footnotesize] at (1,-.25)
    {\itshape Zone de localisation compatible \normalfont \textcolor{RdBu-9-9}{\textsf{B}}};
  \end{scope}
  % Fusion 1
  \begin{scope}[xshift=-6cm, yshift=-4cm,local bounding box=fus1]
    \begin{scope}
      \foreach \x in {0,.25,...,1.75}{
        \foreach \y in {0,.25,...,1.75}
        {
          % Calcul des angles
          \calcangle{\angleA}{\pgfpoint{1cm}{1cm}}{\pgfpoint{1cm}{0cm}}{\pgfpoint{\x
              cm}{\y cm}}
          % Fuzzyfication des angles
          \pgfmathsetmacro\fuzzyA{%
            ifthenelse(\angleA < 90,(90-\angleA)/90,
            ifthenelse(\angleA > 270,(\angleA-270)/90,0)
            )
          }
          \calcangle{\angleB}{\pgfpoint{1cm}{1cm}}{\pgfpoint{2cm}{1cm}}{\pgfpoint{\x cm}{\y cm}}
          \pgfmathsetmacro\fuzzyB{%
            ifthenelse(\angleB < 90,(90-\angleB)/90,
            ifthenelse(\angleB > 270,(\angleB-270)/90,0)
            )
          }
          % fusion (zadeh)
          \pgfmathsetmacro\fus{min(\fuzzyA, \fuzzyB)}          
          % Calcul du rayon à partir de la fuzzyfication
          \pgfmathsetmacro\radius{\fus*2.5}
          \fill[RdBu-9-9] (\x+.125,.125+\y) circle (\radius pt);
        }
      }
      \path[ffc2] (0,0) rectangle (2,2);
    \end{scope}
    \node[text width=3cm, align=center, anchor=north, font=\footnotesize] at (1,-.25)
    {\itshape Zone de localisation compatible \normalfont \textcolor{RdBu-9-9}{\textsf{B}}};
  \end{scope}
  % Fusion 2
  \begin{scope}[xshift=-2cm, yshift=-4cm,local bounding box=fus2]
    \begin{scope}
      \foreach \x in {0,.25,...,1.75}{
        \foreach \y in {0,.25,...,1.75}
        {
          % Calcul des angles
          \calcangle{\angleA}{\pgfpoint{1cm}{1cm}}{\pgfpoint{1cm}{0cm}}{\pgfpoint{\x
              cm}{\y cm}}
          % Fuzzyfication des angles
          \pgfmathsetmacro\fuzzyA{%
            ifthenelse(\angleA < 90,(90-\angleA)/90,
            ifthenelse(\angleA > 270,(\angleA-270)/90,0)
            )
          }
          \calcangle{\angleB}{\pgfpoint{1cm}{1cm}}{\pgfpoint{2cm}{1cm}}{\pgfpoint{\x cm}{\y cm}}
          \pgfmathsetmacro\fuzzyB{%
            ifthenelse(\angleB < 90,(90-\angleB)/90,
            ifthenelse(\angleB > 270,(\angleB-270)/90,0)
            )
          }
          % fusion (luka)
          \pgfmathsetmacro\fus{max(\fuzzyA + \fuzzyB - 1, 0)}          
          % Calcul du rayon à partir de la fuzzyfication
          \pgfmathsetmacro\radius{\fus*2.5}
          \fill[RdBu-9-9] (\x+.125,.125+\y) circle (\radius pt);
        }
      }
      \path[ffc2] (0,0) rectangle (2,2);
    \end{scope}
    \node[text width=3cm, align=center, anchor=north, font=\footnotesize] at (1,-.25)
    {\itshape Zone de localisation compatible \normalfont \textcolor{RdBu-9-9}{\textsf{B}}};
  \end{scope}
  % Fusion 3
  \begin{scope}[xshift=2cm, yshift=-4cm,local bounding box=fus3]
    \begin{scope}
      \foreach \x in {0,.25,...,1.75}{
        \foreach \y in {0,.25,...,1.75}
        {
          % Calcul des angles
          \calcangle{\angleA}{\pgfpoint{1cm}{1cm}}{\pgfpoint{1cm}{0cm}}{\pgfpoint{\x
              cm}{\y cm}}
          % Fuzzyfication des angles
          \pgfmathsetmacro\fuzzyA{%
            ifthenelse(\angleA < 90,(90-\angleA)/90,
            ifthenelse(\angleA > 270,(\angleA-270)/90,0)
            )
          }
          \calcangle{\angleB}{\pgfpoint{1cm}{1cm}}{\pgfpoint{2cm}{1cm}}{\pgfpoint{\x cm}{\y cm}}
          \pgfmathsetmacro\fuzzyB{%
            ifthenelse(\angleB < 90,(90-\angleB)/90,
            ifthenelse(\angleB > 270,(\angleB-270)/90,0)
            )
          }
          % fusion (zadeh)
          \pgfmathsetmacro\fus{\fuzzyA * \fuzzyB}          
          % Calcul du rayon à partir de la fuzzyfication
          \pgfmathsetmacro\radius{\fus*2.5}
          \fill[RdBu-9-9] (\x+.125,.125+\y) circle (\radius pt);
        }
      }
      \path[ffc2] (0,0) rectangle (2,2);
    \end{scope}
    \node[text width=3cm, align=center, anchor=north, font=\footnotesize] at (1,-.25)
    {\itshape Zone de localisation compatible \normalfont \textcolor{RdBu-9-9}{\textsf{B}}};
  \end{scope}
  % Fusion 4
  \begin{scope}[xshift=6cm, yshift=-4cm,local bounding box=fus4]
    \begin{scope}
      \foreach \x in {0,.25,...,1.75}{
        \foreach \y in {0,.25,...,1.75}
        {
          % Calcul des angles
          \calcangle{\angleA}{\pgfpoint{1cm}{1cm}}{\pgfpoint{1cm}{0cm}}{\pgfpoint{\x
              cm}{\y cm}}
          % Fuzzyfication des angles
          \pgfmathsetmacro\fuzzyA{%
            ifthenelse(\angleA < 90,(90-\angleA)/90,
            ifthenelse(\angleA > 270,(\angleA-270)/90,0)
            )
          }
          \calcangle{\angleB}{\pgfpoint{1cm}{1cm}}{\pgfpoint{2cm}{1cm}}{\pgfpoint{\x cm}{\y cm}}
          \pgfmathsetmacro\fuzzyB{%
            ifthenelse(\angleB < 90,(90-\angleB)/90,
            ifthenelse(\angleB > 270,(\angleB-270)/90,0)
            )
          }
          % fusion (drast)
          \pgfmathsetmacro\fus{%
            ifthenelse(\fuzzyA > 0.999, \fuzzyB,
            ifthenelse(\fuzzyB > 0.999, \fuzzyA, 0)
            )
          }          
          % Calcul du rayon à partir de la fuzzyfication
          \pgfmathsetmacro\radius{\fus*2.5}
          \fill[RdBu-9-9] (\x+.125,.125+\y) circle (\radius pt);
        }
      }
      \path[ffc2] (0,0) rectangle (2,2);
    \end{scope}
    \node[text width=3cm, align=center, anchor=north, font=\footnotesize] at (1,-.25)
    {\itshape Zone de localisation compatible \normalfont \textcolor{RdBu-9-9}{\textsf{B}}};
  \end{scope}
\end{tikzpicture}
  \caption{sq}
  \label{fig:comparaison_operateurs_intersection}
\end{figure}


\tdi{Raisonnement en termes d'interprétabilité}

On choisi Zadeh

\subsection{La prise en compte de la confiance}
\tdi{voi si c,'est pas mieux de le mettre en section}

La confiance peut se définir en amont de la spatialisation.

Diminuer la confiance c'est augmenter le degré d'appartenance minimal

Proposer une version améliorée de la fig finale du chapitre 4 avec la
confiance.

Dire que les fonctions présentées reviennent a considérer que la
confiance est absolue

Parler des seuils de la confiance (3) et des valeurs associées ou
contraintes.


\begin{figure}
  \centering
  \begin{tikzpicture}[scale=.7]
  \def\decalageX{-.2}
  \def\decalageY{-.2}
  % Courbe
  \begin{scope}[transparency group]
    % fond
    \begin{scope}
      \path[ffa]  (1,.4) -- (3.3,.4) -- (4.5, 2) -- (5.7,.4) -- (8,.4)
      -- (8,0) -- (1,0) -- cycle;
      \path[ffa_fade_m] (0,.4) -- (1,.4) -- (1,0) -- (0,0) -- cycle ;
      \path[ffa_fade] (8,.4) -- (9,.4) -- (9,0) -- (8,0) -- cycle ;
    \end{scope}
    % bords
    \begin{scope}
      \path[ffc] (1,.4) -- (3.3,.4) -- (4.5, 2) -- (5.7,.4) -- (8,.4);
      \path[ffc, dotted] (3.3,.4) -- (3,0);
      \path[ffc, dotted] (5.7,.4) -- (6,0);
      \path[ffc_fade_m] (0,.4) -- (1,.4) ;
      \path[ffc_fade] (8,.4) -- (9,.4) ;
    \end{scope}
  \end{scope}
  % Axes X, Y
  \begin{scope}
    % Axe X
    \begin{scope}
      % Axe
      \draw[<->] (0, \decalageX) --++ (9, 0) coordinate (x axis);
      % Graduations
      \foreach \n/\t in {1/{},2/{},3/{},4/{},5/{},6/{},7/{},8/{}}
      {
        \draw[-] (\n, \decalageX - .05) --++ (0, .1);
        \node[below, font=\footnotesize] at (\n, \decalageX - .05) {\t};
      }
      % label
      \node[below left, yshift=-.1cm, font=\small] at (x axis) {\itshape Métrique};
    \end{scope}
    % Axe Y
    \begin{scope}
      % Axe
      \draw[-] (\decalageY ,0) --++ (0, 2) coordinate (y axis);
      % Graduations
      \foreach \n/\t in {0/{0},2/{1}}
      {
        \draw[-] (\decalageY -.05, \n) --++ (.1, 0);
        \node[left, font=\footnotesize] at (\decalageY -.05, \n) {\t};
      }
      % Label
      \node[above] at (y axis) {$\mu$};
    \end{scope}
  \end{scope}
  \begin{scope}
    % Seuil 1
    \draw[ffc,line width=.5] (3.3,\decalageY) -- (3.3,.4);
    \draw[fill, RdBu-9-1] (3.3,\decalageY) circle (2pt);
    \draw[fill, RdBu-9-1] (3.3,.4) circle (2pt);
    % Seuil 2
    \draw[ffc,line width=.5] (4.5,\decalageY) -- (4.5,2);
    \draw[fill, RdBu-9-1] (4.5,\decalageY) circle (2pt);
    \draw[fill, RdBu-9-1] (4.5,2) circle (2pt);
    \node[above] at (4.5,2) {\(v\)};
    % Seuil 3
    \draw[ffc,line width=.5] (5.7,\decalageY) -- (5.7,.4);
    \draw[fill, RdBu-9-1] (5.7,\decalageY) circle (2pt);
    \draw[fill, RdBu-9-1] (5.7,.4) circle (2pt);
    
    \draw[|-|] (3,-.7cm) --++(3,0) node[pos=.5, fill=white, inner
    sep=1pt, font=\small] {$\delta$};

    \draw[<->, shorten >=2pt,shorten <=2pt] (2,.4cm) --++ (0,1.6cm) node[pos=.5, fill=white, inner
    sep=1pt, font=\small] {$i$};
  \end{scope}
\end{tikzpicture}

  \caption{sq}
  \label{fig:qs}
\end{figure}




\begin{figure}
  \centering
  \missingfigure{Comparaison fusion indices avec et sans incertitude}
\end{figure}




%%% Local Variables:
%%% mode: latex
%%% TeX-master: "../../../../main"
%%% End:
