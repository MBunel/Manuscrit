Comme nous l'avons vu au cours de ce chapitre, notre thèse s'inscrit à
la fois dans le contexte métier du secours en montagne et dans
le contexte organisationnel du projet de recherche Choucas.

Parmi tous les points critiques auxquels les secouristes sont
confrontés dans l'exercice de leur métier, nous nous intéressons à
ceux apparaissant lors la première phase des opérations de secours, la
localisation de la victime. En effet, malgré le développement récent
de la solution de géolocalisation \emph{Gend'Loc,} il est quelquefois
difficile pour les secouristes de localiser une personne perdue ou
blessée en montagne, notamment lorsque la victime ne dispose pas de
smartphone ou que les secouristes ne l'estiment pas capable de
réaliser les manipulations adéquates. Dans ces conditions, les
secouristes procèdent à la localisation manuelle de la victime, en
interprétant les informations qu'elle peut donner sur sa position, ce
qui est moins précis et plus lent qu'une localisation par
\emph{Gend'Loc.} Or la rapidité d'un secours est un facteur
déterminant pour la survie de la victime, mais également pour
l'efficacité du système de secours en montagne dans son ensemble, une
perte de temps sur le traitement d'une alerte pouvant impacter les
alertes suivantes. Le projet Choucas a été construit pour aider les
secouristes, en premier lieu ceux du \ac{pghm} de Grenoble, à traiter
ce type d'alertes. L'objectif de ce projet est, plus précisément, de
développer des méthodes et des outils destinés à assister les
secouristes durant la phase de localisation de la victime. Notamment
en proposant des outils d'aide à la décision, de géovisualisation et
en proposant des méthodes permettant d'enrichir les bases de données
utilisées par les secouristes avec des données métier utiles, telles
que certaines formes spécifiques du relief (\eg vires, barres
rocheuses, \emph{etc.}).

Au sein de ce projet, notre rôle est de travailler à la conception de
méthodes d'aide au raisonnement spatial, permettant de transformer une
description de position (\eg \enquote{je suis sous une cascade}) en
une zone y correspondant, nous parlons alors de
\enquote{spatialisation}. Les détails de cet objectif et de son
contexte scientifique serons présentés dans le chapitre suivant.

%%% Local Variables:
%%% mode: latex
%%% TeX-master: "../../../main"
%%% End:
