
% Définir les concepts importants : alerte, indice de localisation,
% zone de localisation


% Expliciter les objectifs scientifiques
Durant le premier chapitre de ce manuscrit nous avons fait emploi de
nombreux termes, dans une acceptation non scientifique, 

Les termes de \emph{description de position} et de \emph{zone,}
utilisés jusqu'ici, l'étaient, volontairement, dans un sens assez
large.


Nous avons été, jusqu'ici, assez évasifs quant à la nature de ce que
nous qualifions \emph{d'alertes} ou \emph{d'indices de localisation.}
% Alertes
Le terme \emph{d'alerte,} que nous avons abondamment employé depuis le
début de ce manuscrit, désigne à la fois …
%
Dans son acceptation formelle,
%
Une alerte est l'ensemble des \emph{indices de localisation} à la
disposition des secours pour identifier la position d'une victime
donnée. Ainsi ce qui caractérise une alerte est, en plus des
\emph{indices de localisation} qui la composent, son contexte, \ie
l'événement particulier dont elle traite.




% Indices de localisation
Les \emph{indices de localisation} correspondent, quant à eux, à
l'ensemble…
%
Par conséquent, une même alerte est composée d'un
\footnote{\label{note:cas_1_indice}Cas que nous n'avons jamais
  rencontré.} ou plusieurs \emph{indices de localisation} qu'il
convient de spatialiser. Ainsi, pour une même alerte, on est amenées à
\emph{spatialiser} plusieurs \emph{indices de localisation,}
conduisant à la création d'autant de \emph{zones de localisation
  compatibles.} Chacune d'entre elles contient une partir de
l'information donnée par le requérant, il est donc nécessaire de les
combiner, c'est ce processus que nous qualifions de \emph{fusion.}

Ainsi, le \emph{fil rouge,} déjà présenté (\autoref{subsec:1-1-2-3}),
est une \emph{alerte,} composée de XXXX \emph{indices de localisation.}

% Deux objectifs
Ainsi, la construction de la \emph{zone de localisation} correspondant
à la localisation décrite par le requérant, nécessite toujours
\footnote{À l'exception du cas, déjà mentionné, où le requérant ne
  donne qu'un seul \emph{indice de localisation.} Toutefois, c'est un
  cas que l'on rencontre pas en pratique
  (cf. \autoref{note:cas_1_indice}).} un processus en deux
étapes. Lequel commence par la création des \emph{zones de
  localisation compatibles} correspondant à chaque \emph{indice de
  localisation} donné par le requérant et ce termine par la
construction de la \emph{zone de localisation probable} à partir des
\emph{zones de localisation compatibles} construites.

% Spatialisation
La première étape de ce processus correspond à ce que nous appelons
\emph{la spatialisation} d'un \emph{indice de localisation.} C'est un
processus qui transforme un \emph{indice de localisation} en une
\emph{zone de localisation compatible,} comme l'illustre la figure
\ref{fig:obj_spa}.


Lors du traitement d'une alerte il est
nécessaire d'effectuer autant d'opérations de \emph{spatialisation}
qu'il y a \emph{d'indices de localisation} uniques.
%

Cependant, contrairement à ce que la figure laisse
supposer, \emph{l'indice de localisation} utilisé n'est pas contraint
dans sa représentation. La figure \ref{fig:obj_spa} le présente sous
la forme d'une phrase en langage naturel, mais il pourrait en être
autrement. De même, la \emph{zone de localisation} proposée ne l'est
qu'à titre d'illustration.
%
Il est souhaitable que la
\emph{spatialisation} soit un processus déterministe, \ie que ce
processus donnera toujours la même \emph{zone de localisation
  compatible} pour un même \emph{indice de localisation.}

\begin{figure}
  \centering
  \begin{tikzpicture}

\node[text width=2.5cm, align=center] (0,0) {\small\enquote{Je suis proche d'une maison}};

\path[draw, ->] (2,0) --++ (2.5,0)  node[pos=.5, above] {\footnotesize \itshape spatialisation};

\begin{scope}[xshift=6.5cm]
\path[ffa] (0,0) circle [radius=30pt];
\path[ffc] (0,0) circle [radius=30pt];

\node[circle, inner sep=0pt,minimum size=4pt, fill] (c) at (0,0) {};
\node[circle, inner sep=0pt,minimum size=8pt] (c2) at (0,0) {};
\node[anchor=west] (m) at  (1.75,1) {\footnotesize \itshape objet de référence};
\path[draw, ->] (m.west) -- (c2);

\node[anchor=west, text width=3cm] (m2) at  (1.75,0) {\footnotesize \itshape zone de localisation compatible};
\path[draw, ->] (m2.west) --++ (-.5,0);
\end{scope}
\end{tikzpicture}
  \caption{Illustration du processus de \emph{spatialisation} d'un
    \emph{indice de localisation.}}
  \label{fig:obj_spa}
\end{figure}

% Fusion
Chaque \emph{indice de localisation} donné par le requérant donne une
information sur la position de la victime. Si l'on exclut le cas où
certains d'entre-eux sont faux, on peut définir la \emph{zone de
  localisation probable} comme la zone où tous les \emph{indices de
  localisation } sont vérifiés.
% Exemple
Prenons pour exemple une version simplifiée du \emph{fil rouge}
(cf. \autoref{subsec:1-1-2-3}), composée de seulement deux
\emph{indices:} \emph{la victime voit une partie de plan d'eau} et
\emph{vient de passer du soleil à l'ombre.}  La \emph{zone de
  localisation} correspondant à cette description est, à la fois une
zone à partir de laquelle on peut apercevoir une partie de plan d'eau
(\ie que le premier \emph{indice de localisation} est validé), mais
aussi une zone qui vient de passer du soleil à l'ombre (\ie le second
\emph{indice de localisation} est validé). Par conséquent la
\emph{zone de localisation probable} correspond à l'intersection des
deux \emph{zones de localisation compatibles,} comme on peut le voir
sur la figure \ref{fig:obj_fus}. Toutefois, comme pour \emph{la
  spatialisation,} cette représentation est grandement
simplifiée. D'une part car elle considère que les deux \emph{zones de
  localisation compatibles} partagent le même objet de référence, ce
qui n'est généralement pas le cas, comme on peut le voir dans la
présentation du \emph{fil rouge} ou dans son extrait utilisé ici comme
exemple. De plus, comme nous l'indiquions précédemment, cette
illustration occulte le cas 


La construction de la \emph{zone de localisation probable} s'effectue
par le processus de \emph{fusion} des \emph{zones de localisation
  compatibles.} L'objectif de cette étape est 


\begin{figure}
  \centering
  \begin{tikzpicture}
  % ZLC 1
  \path[ffa] (0,0) circle [radius=30pt]; % Aire
  \path[ffc] (0,0) circle [radius=30pt]; % Contour
  % ZLC 2
  \path[ffa2, pattern color=RdBu-9-3] (.25,-.5) ellipse (40pt and 20pt); % Aire 
  \path[ffc2, draw=RdBu-9-2] (.25,-.5) ellipse (40pt and 20pt); % Contour
  % Objet de ref
  % 1
  \node[circle, inner sep=0pt,minimum size=4pt, fill] (c) at (0,0) {}; % visible
  \node[circle, inner sep=0pt,minimum size=8pt] (c2) at (0,0) {}; % fictif
  % 2
  \node[circle, inner sep=0pt,minimum size=4pt, fill] (c1) at (.25,-.5) {}; % visible
  \node[circle, inner sep=0pt,minimum size=8pt] (c12) at (.25,-.5) {}; % fictif
  % Légende
  \node[anchor=east] (m) at  (-1.5,.25) {\footnotesize \itshape objets de référence};
  \path[draw, ->] (m.east) -- (c2);
   \path[draw, ->] (m.east) -- (c12);
  \node[anchor=north west, text width=3cm] (m2) at (1.75,-.4) {\footnotesize
    \itshape zones de localisation compatibles};
  \path[draw, ->] (m2.west) --++ (-1.7,0);
  \path[draw, ->] (m2.west) -- (1.2,-.7);
  % Fléche + texte
  \path[draw, ->] (2.25,0) --++ (2.75,0)  node[pos=.5, above] {\footnotesize \itshape fusion};
  % Fusion
  \begin{scope}[xshift=6.75cm]
    \path[draw, dashed] (0,0) circle [radius=30pt]; % Contour
    \path[draw, dashed] (.25,-.5) ellipse (40pt and 20pt);
    \begin{scope}
      \clip (.25,-.5) ellipse (40pt and 20pt);
      \fill[ffa2] (0,0) circle [radius=30pt];
      \path[ffc2] (0,0) circle [radius=30pt];
    \end{scope}
    \begin{scope}
      \clip (0,0) circle [radius=30pt];
      \path[ffc2] (.25,-.5) ellipse (40pt and 20pt);
    \end{scope}
    \node[circle, inner sep=0pt,minimum size=4pt, fill] (c) at (0,0)
    {};
    \node[circle, inner sep=0pt,minimum size=4pt, fill] (c) at (.25,-.5) {};
    \node[anchor=west, text width=3cm] (m2) at  (1.25,.75) {\footnotesize \itshape zone de localisation probable};
    \path[draw, ->] (m2.south west) --++ (-.6,-.4);
  \end{scope}
\end{tikzpicture}
  \caption{Illustration du processus de construction de la \emph{zone
      de localisation probable} par la \emph{fusion} des \emph{zones
      de localisation compatibles.}}
  \label{fig:obj_fus}
\end{figure}

Ces deux étapes, \emph{spatialisation} et \emph{fusion,} nécessitent
des travaux divers.

notre objectif principal est de construire la
\emph{zone de localisation} correspondant à une description en langage
naturel. Toutefois notre travail ne porte pas directement sur
l'analyse du langage naturel, nous ne cherchons pas à identifier les
\emph{relations de localisation,} ou les \emph{objets de référence} en
direct lors de l'appel téléphonique du requérant aux secours ou à
l'aide d'une retranscription de celui-ci. Notre travail ne porte que
sur la \emph{spatialisation.} Cette étape nécessite de développer des
méthodes de spatialisation permettant construire les \emph{zones de
  localisation} correspondant à chaque \emph{relation de
  localisation,} c'est le propos de notre premier objectif :
scientifique \enquote{\nameref{subsec:2-1-1}}.





%%% Local Variables:
%%% mode: latex
%%% TeX-master: "../../../../main"
%%% End:
