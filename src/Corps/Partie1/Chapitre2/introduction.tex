Comme nous l'avons vu dans le chapitre précédent, cette thèse
s'inscrit dans un projet de recherche a visée applicative dont
l'objectif principal est de développer des solutions destinées à aider
les \emph{unités de secours en montagne} \acp{usem} ---~plus
spécifiquement le \ac{pghm} de Grenoble~--- à localiser des personnes
perdues en montagne. Parmi toutes les problématiques abordées par le
projet de recherche Choucas, notre travail s'inscrit dans celle visant
à développer des méthodes et des outils permettant de passer d'une
description de position à une zone, représentée par des coordonnées
géographiques. La présentation qui a été faite de cette problématique
à, volontairement, été assez évasive. Nous n'en avons pas présenté, ni
le détail, ni le champ de notre travail de thèse et sa place au sein
de cet objectif scientifique.

Ce chapitre se destine à clarifier ces points, notamment en détaillant
la problématique de cette thèse et en présentant les différents
objectifs scientifiques de ce travail. 

La première partie de ce second chapitre sera destinée à présenter la
problématique de cette thèse (\autoref{sec:2-1}). La seconde partie
présentera les objectifs scientifiques de ce travail
(\autoref{sec:2-2}).

%%% Local Variables:
%%% mode: latex
%%% TeX-master: "../../../main"
%%% End:
