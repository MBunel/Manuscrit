% Introduction objectifs scientifiques
La nécessité de concevoir des méthodes de \emph{spatialisation} des
\emph{indices de localisation} et une méthode de \emph{fusion} des
\emph{zones de localisation compatibles} qui soient adaptées aux
contraintes de notre contexte applicatif nous a conduit à identifier
quatre objectifs scientifiques pour cette thèse :
%
\begin{enumerate*}[label=(\arabic*)]
\item la spatialisation des indices de localisation ;
\item la modélisation de l'imprécision des indices de localisation ;
\item la modélisation de l'incertitude ; et
\item la fusion des indices de localisation.
\end{enumerate*}
%
Cette partie se destine à présenter ces différents objectifs.
% Annonce tableau synthèse
On trouvera une synthèse des objectifs scientifiques, ainsi que de
leurs verrous et des contributions attendues, dans le
\autoref{tab:synthese_objectifs} page
\pageref{tab:synthese_objectifs}.

\subsection{La \emph{spatialisation} des \emph{indices de
    localisation}}
\label{subsec:2-1-1}

% Introduction
La construction d'une \emph{zone de localisation compatible} nécessite
de définir des méthodes de \emph{spatialisation} permettant de traiter
des \emph{indices de localisation} aussi différents que :
%
\begin{enumerate*}[label=(\alph*)]
\item \enquote{la victime est sous une route},
\item \enquote{je vois une montagne}, ou
\item \enquote{j'ai marché plusieurs heures}.
\end{enumerate*}
%
Comme nous l'avons expliqué, les \emph{indices de localisation} sont
composites. La position du \emph{sujet} est donnée par la
\emph{relation de localisation,} par rapport à \emph{l'objet de
  référence.} L'élaboration d'une méthode de \emph{spatialisation}
présuppose donc de travailler à l'identification de la sémantique des
\emph{relations de localisation} et d'en proposer une modélisation
formelle.

Comme notre objectif est de proposer des solutions de
\emph{spatialisation} et de \emph{fusion} s'intégrant dans la démarche
actuelle des secouristes, nous pouvons envisager d'exploiter leur
expertise pour réaliser certaines actions, comme l'écoute et l'analyse
du discours du \emph{requérant,} qui seraient difficilement
automatisables par ailleurs. En effet, il n'est ni nécessaire, ni
souhaitable, d'automatiser entièrement le processus de localisation de
la victime, ce qui nécessiterait, par ailleurs, de développer des
solutions d'analyse automatisée du langage oral. Cependant, cette
étape ne concerne pas notre travail et ne sera donc pas abordée ici,
puisque nous choisissons de confier cette tâche de
\enquote{sémantisation} de la description orale d'une position au
secouriste. Notre travail se limitera donc à identifier les
\enquote{types} de descriptions utilisés dans notre contexte, de les
exprimer dans un vocabulaire controlé, puis d'en proposer une
modélisation à même d'en permettre la \emph{spatialisation.}

\subsubsection{Contexte scientifique}

La question de la \emph{spatialisation} d'une position décrite en
langage naturel peut sembler être un sujet dont les applications sont
limitées à des cas très spécifiques, comme l'est notre contexte
applicatif. Cependant, en cherchant à améliorer le traitement de
données géographiques exprimées en langage naturel, ces travaux
œuvrent à faciliter la communication entre machines et êtres
humains. En effet, comme l'illustre le concept de \emph{Naive
  geography}, proposé par \textcite{Egenhofer1995}, la représentation
informatique de positions et plus généralement, d'objets
géographiques, est assez éloignée de la façon dont les hommes et les
femmes conçoivent l'espace et leur localisation. Pour des raisons
techniques, il est beaucoup plus simple, pour un ordinateur,
d'identifier et de manipuler les coordonnées d'un point à la surface
de la terre avec une précision submétrique, que d'identifier la
position correspondant à la description : \enquote{le café sous ma
  maison}, tandis que cette dernière est pourtant beaucoup plus
compréhensible pour un être humain
\autocite{Duchene2019}. L'interprétation de ce type de description se
heurte, en effet, à plusieurs problèmes. Tout d'abord, il est
difficile d'identifier la sémantique des termes utilisés, celle-ci
étant dépendante du contexte. Reprenons l'exemple du
\enquote{café}. Vous aurez très certainement compris en lisant cet
énoncé que \enquote{le café} était un lieu. Toutefois, pour en arriver
à cette conclusion, il a fallu comprendre qu'il s'agissait d'une
métonymie et, par conséquent, que le terme \enquote{café} désigne ici
un lieu et non une boisson psychotrope ou le moment où on la consomme,
ou les grains du caféier. C'est le contexte d’utilisation, où le mot
\enquote{café} est couplé à une \emph{relation de localisation,} qui
permet de comprendre qu'il s'agit d'un objet géographique et donc, que
l'interprétation la plus vraisemblable est que l'on parle ici d'un
bar. De la même manière, le terme \enquote{ma maison}, peut être
interprété comme un synonyme d'\enquote{habitat individuel} ou comme
une nouvelle métonymie désignant \enquote{mon foyer}. Dans ce cas,
c'est l'adjectif possessif \enquote{ma} qui permet de trancher. Comme
le montrent ces deux exemples, bien qu'il soit assez facile pour un
être humain d'interpréter ces deux phrases, il est nécessaire de
prendre en compte beaucoup d'éléments pour permettre à un ordinateur
de les interpréter. Cet exemple porte sur l'interprétation automatisée
du langage naturel dans le cas général, mais les mêmes problèmes
s'appliquent également aux \emph{relations de localisation.} Dans cet
exemple, le \enquote{café} est \enquote{sous} mon domicile, mais
comment identifier les positions qui sont effectivement \enquote{sous
  ma maison} ? Pour ce faire, il est nécessaire d'identifier la
sémantique des \emph{relations de localisation} et notamment de celles
utilisées dans notre contexte applicatif.

\paragraph{La formalisation des \emph{relations de localisation}}

Ce travail d'identification et de formalisation des \emph{relations de
  localisation} a été entrepris par de nombreux chercheurs, comme
\textcite{Vandeloise1986}, qui a proposé une analyse complète de la
sémantique de \emph{relations de localisation} utilisées en
français. Ce travail a été consolidé par les travaux de
\textcite{Borillo1998}, qui portent, quant à eux, sur la morphologie
des descriptions de position, ou de \textcite{Aurnague1993,
  Aurnague1997}, qui iront jusqu'à proposer une formalisation logique
des \emph{relations de localisations.} Des travaux similaires ont été
entrepris pour la langue anglaise, notamment par
\autocite{Kracht2002,Mark1999,Freksa2018,Carlson2005,Lang1991,Matsakis2010}.
En 2010 \textcite{Bateman2010}, proposerons \emph{l'ontologie} GUM,
qui formalise l'ensemble des \emph{relations de localisation.}
L'ensemble de ces travaux permet d'avoir de nombreuses connaissances
sur la sémantique des \emph{relations de localisation.} Pourtant, de
nombreux autres travaux ont eu besoin de spatialiser des
\emph{relations de localisation.}

Parmi les applications de spatialisation des \emph{relations de
  localisation,} on peut retrouver des travaux en analyse d'image
\autocite{Hudelot2008, Hudelot2008a,Bloch1996, Vanegas2011,
  Takemura2012}, en robotique \autocite{Skubic2004} ou encore en
sciences de l'information géographique
\autocite{Xu2007a,Xu2007,Xu2006,Wolter2018,Dittrich2015,Du2016,Hornsby2009,Mathet2000,Hall2015}.


%\autocite{Levit2007}
%\autocite{Evans2008}
%\autocite{Shariff1998}
%\autocite{Leopold2015}
%\autocite{Denis1997}

\subsubsection{Verrous scientifiques}

L’élaboration d'une méthode de \emph{spatialisation} se heurte
néanmoins à deux verrous majeurs, l'un intrinsèquement lié à cet
exercice, l'autre dépendant de notre contexte applicatif.

Le premier verrou est la grande variabilité sémantique des préposition
spatiales. Ces dernières peuvent, en effet, prendre des sens
légèrement différents, en fonction du contexte
\autocite{Bateman2010}. Par exemple, on peut être \enquote{sous
  \emph{un pont}}, ce qui implique une notion de recouvrement, ou
\enquote{sous \emph{une route}}, ce qui n'implique pas de
recouvrement. Ainsi, une même préposition spatiale peut décrire des
relations de localisation différentes, par exemple un \enquote{sous
  avec recouvrement} et un \enquote{sous sans recouvrement}. Les
méthodes de \emph{spatialisation} développées ne pourront pas
directement traiter les prépositions spatiales comme des
\emph{relations de localisation,} mais elle devront s'adapter à ces
différentes \emph{relations de localisations,} au risque de construire
une \emph{zone de localisation compatible} qui ne soit pas
représentative de la description donnée par le \emph{requérant.}

Le second verrou est lié au manque d’exhaustivité et à la redondance
des \emph{indices de localisation} à notre disposition pour développer
et tester nos méthodes de spatialisation. Les seules données dont nous
disposons (et pouvons disposer) sont des enregistrements audio
d'alertes, traitées par différents \ac{pghm}. Or, il existe peu de
redondances entre chacune de ces alertes, ce qui rend leur
confrontation difficile. Par exemple, nous n'avons pas à notre
disposition, deux enregistrements différents, décrivant la même
position, avec les mêmes \emph{indices de localisation.} Nous ne
pouvons donc pas estimer la divergence des descriptions d'une même
position. De même, les \emph{indices de localisation} utilisés sont si
divers, qu'un même \emph{indice de localisation} n'est que rarement
utilisé pour décrire des positions différentes, à l'exception notable
des \emph{indices de localisation} de la forme : \enquote{Je suis
  \emph{sur} \textins{un chemin, une route, un sentier, etc.}}. Il
nous est donc impossible d’élaborer notre méthode de
\emph{spatialisation} uniquement à partir de données issues des
alertes passée.

Pour répondre à ces verrous scientifiques, nous proposons d'identifier
les \emph{relations de localisation} utilisées dans notre contexte
applicatif. Nous émettons l'hypothèse que seul un sous-ensemble des
\emph{relations de localisation} présentes dans la langue française
est utilisé, du moins régulièrement, pour décrire une position en
montagne. L'identification de ces \emph{relations de localisation}
nous permettra, d'une part, de déterminer les spécificités de la
description d'une position en langage naturel dans un contexte
montagnard et, de l'autre, de nous concentrer exclusivement sur la
modélisation de ces \emph{relations de localisation.}

Ce travail sera complété par un travail d'identification de la
sémantique des relations de localisation utilisées dans notre
contexte. Il est, en effet, possible que les \emph{relations de
  localisations} utilisées dans notre contexte, le soient dans un sens
particulier, plus réduit que le sens général. Ainsi, nous allons
compléter le recensement des \emph{relations de localisation}
utilisées dans notre contexte, par un travail d'identification de leur
sémantique, dans ce même contexte.

Ce travail de recensement sera également entrepris pour les
\emph{objets de référence.} Comme pour les \emph{relations de
  localisation,} on peut supposer que seule une petite partie des
objets pouvant être utilisés comme points de repères, le sont
réellement dans ce contexte. On s'attendra, par exemple, à trouver des
\emph{indices de localisation} tels que \enquote{je suis au niveau
  \emph{du sommet}} ou \enquote{la victime est dans \emph{une combe}},
alors que la phrase : \enquote{je suis proche \emph{d'un kiosque}},
semble peu probable. Comme pour les \emph{relations de localisation,}
l'identification et le recensement de ces objets permettront d'avoir,
à la fois une meilleure connaissance des descriptions de positions
dans le contexte du secours en montagne, mais aussi de focaliser notre
travail sur les objets les plus pertinents.

Ces objectifs nous permettront de développer une méthode de
\emph{spatialisation} de ces \emph{indices de localisation} et d'en
proposer une implémentation fonctionnelle. La méthode mise en œuvre
devra être suffisamment générique pour fonctionner avec les
différentes \emph{relations de localisation} et les différents
\emph{objets de référence} identifiés, tout en prenant en compte leurs
spécificités, dans le but de proposer une \emph{spatialisation}
précise et adaptée au contexte du secours en montagne, mais
suffisamment générique pour être enrichie ultérieurement, voire
étendue à des contextes différents.

\subsection{La modélisation de \emph{l'imprécision} des \emph{indices
    de localisation}}
\label{subsec:2-1-2}

% Introduction partie
Le travail d'identification de la sémantique des \emph{relations de
  localisation} se heurte à un autre problème, celui de
l'interprétation du locuteur. En effet, les \emph{relations de
  localisation} que nous cherchons à sémantiser sont des concepts
formulés en langage naturel et donc, soumises à son
\emph{imprécision.}  Par conséquent, la \emph{limite} entre la zone
qui correspond à un \emph{indice de localisation} et celle qui n'y
correspond pas, n'est pas nécessairement, voire jamais, une ligne bien
identifiée. Les conséquences de ce fait sont multiples. D'une part,
cela complique la formalisation des \emph{relations de localisation.}
En effet, comment définir formellement un concept, alors que l'on
peine à en fixer les limites ?  \textcite{Vandeloise1986} se montre
même particulièrement critique\footnote{\enquote{Aucun mot spatial ne
    se prête à une description aussi rigide et des contre-exemples
    peuvent être trouvés à toutes les définitions proposées.}
  \autocite[p. 18]{Vandeloise1986}} envers les travaux proposant une
formalisation des \emph{relations de localisation} à l'aide de la
logique ---~classique~--- du premier ordre. En effet, la logique
\emph{classique} \footnote{C'est-à-dire, telle que formulée au début
  du \bsc{xix}\up{ème} siècle, principalement par \bsc{Frege} dans son
  \emph{Idéographie} (\emph{Begriffsschrift}).}, repose sur un
principe de \emph{bivalence,} qui implique qu'une \emph{proposition}
logique est soit \enquote{vraie}, soit \enquote{fausse}. Formaliser
des \emph{relations de localisation} à l'aide de la logique classique
implique donc de définir une limite claire entre la zone où la
\emph{relation de localisation} est vérifiée et celle où elle ne l'est
pas. Dans les faits, cette limite ne peut être qu'arbitraire. Prenons
pour exemple la \emph{relation de localisation} \enquote{proche de} et
considérons que la distance, seule, est un critère de modélisation
satisfaisant. On pourra dire d'un point adjacent à un bâtiment qu'il
en est \emph{proche.} De même, si l'on s'en éloigne un tout petit peu,
mettons d'un mètre, cette description reste correcte. En prolongeant
ce raisonnement par récurrence, on pourra augurer que toute position,
quelle que soit sa distance à \emph{l'objet de référence,} en est
\emph{proche}. Ce raisonnement, calqué sur le paradoxe sorite
\footnote{Ce paradoxe, formulé par le philosophe Grec Eubulide au
  \bsc{iv}\up{ème} siècle av. J.-C., pose deux prémisses. La première
  est qu'un seul grain de sable ne forme pas un tas de sable. La
  seconde est qu'ajouter un grain de sable à un non-tas n'en fait pas
  un tas. Par récurrence, on aboutit à la conclusion, absurde, que,
  quel que soit le nombre de grains de sable, ils ne peuvent pas
  former de tas \autocite{Sorensen2018,Hyde2018}.}, illustre la
difficulté à fixer une limite \emph{précise} à des concepts qui ne le
sont pas. Pourquoi une position serait-elle proche d'un objet, alors
que celle qui en est éloignée d'un pas ne l'est pas ?
\autocite{Fisher2000} Bien que courante, cette modélisation est une
simplification extrême de la sémantique de la \emph{relation de
  localisation} \enquote{proche de}.

Ces considérations peuvent sembler inutilement complexes, d'autant
plus que notre objectif est applicatif, pourtant elles sont
indispensables. Ne pas tenir compte de \emph{l'imprécision} des
\emph{relations de localisation} pourrait conduire à ignorer des zones
pertinentes ou, au contraire, à proposer une modélisation trop laxiste
en cherchant à éviter les faux positifs. C'est pourquoi il nous semble
indispensable de prendre systématiquement en compte
\emph{l'imprécison} lors de la \emph{spatialisation} des
\emph{relations de localisation.}

\subsubsection{Contexte scientifique}

Si le paradoxe sorite, déjà présenté, a été formulé au
\bsc{iv}\up{ème} siècle av. J.-C., \emph{l'imprécision} ne fut
réellement étudiée qu'à partir de la fin du \bsc{xix}\up{ème} siècle
\autocite{Williamson1994}, avec le développement de la logique
mathématique dont la première formalisation est \emph{l'idéographie}
du mathématicien \textcite{Frege1879} \footnote{Pour une traduction
  française voir \textcite{Frege1999}.}. Ce dernier défend son projet
idéographique avec, en 1882, la publication de \emph{Que la science
  justifie le recours à une idéographie} \autocite{Frege1882}
\footnote{Voir \textcite{Frege2019} pour une traduction
  récente.}. Dans cet article, \bsc{Frege} soutient que
\emph{l'imprécision} du langage naturel est source d'erreurs de
raisonnement et d'interprétation et que, par conséquent, les sciences
ne peuvent s'en contenter. Elles nécessitent un moyen d'expression
plus rigoureux, un langage formel, \emph{l'idéographie.} Cette
\enquote{méfiance} envers le langage naturel est également partagée
par \textcite{Russell1923}, qui propose une formalisation du concept
de \emph{vague,} synonyme \emph{d'imprécision.} Selon sa définition,
\emph{l'imprécision} est intrinsèquement liée à la représentation de
la réalité par un \emph{système de signes,} quel qu'il soit. Ces
derniers ne sont pas \emph{imprécis} dans l'absolu, c'est la relation
entre deux \emph{systèmes de signes} qui est qualifiée de
\emph{précise} ou \emph{d'imprécise.} Cette dernière est qualifiée de
\emph{précise} lorsqu'il existe une bijection entre les deux ensembles
de signes, \ie que chaque signe a un seul et unique équivalent. Par
conséquent, le passage d'un \emph{système de signes} à l'autre est
sans équivoque. Dans le cas contraire, un même signe peut être
\enquote{traduit} de diverses façon, laissant dès lors place à
l'interprétation.

Par exemple, la \emph{relation de localisation} \enquote{sous} en
français a deux traductions en anglais,
\foreignquote{english}{\emph{under}} et
\foreignquote{english}{\emph{below}}, à la sémantique légèrement
différente. \foreignquote{english}{\emph{Under}} implique en effet que
le sujet soit à proximité immédiate, voire recouvert par l'objet de
référence, alors que \foreignquote{english}{below} n'impose pas cette
condition : il s'agit d'un \enquote{sous} plus large et moins
contraint. On dira, par exemple : \foreignquote{english}{The owl is
  under the bed}, car il y a ici une notion de recouvrement, mais
\foreignquote{english}{My car is in the street below}, car la route
n'est pas directement sous la position du locuteur.

% Typologie
Le concept \emph{d'imprécision} est généralement rapproché d'autres
concepts. En intelligence artificielle, par exemple,
\emph{l'imprécision} est définie comme étant une des composantes de
\emph{l'imperfection} avec \emph{l'incertitude,} que nous présenterons
ci-dessous et \emph{l'incomplétude,} désignant le manque de
connaissances
\autocite{Bouchon-Meunier1995,Bouchon-Meunier2007}. D'autres
typologies existent cependant. \textcite{Niskanen1989}, par exemple,
ne parle pas \enquote{\emph{d'imperfection}}, mais de
\enquote{\emph{non-précision}} et découpe cette notion en quatre
composantes : \emph{généralité,} \emph{ambiguïté,} \emph{imprécision}
et \emph{incertitude.}

\paragraph{La modélisation de \emph{l'imprécision}}

Divers cadres théoriques ont été développés pour modéliser
\emph{l'imprécision.} Si la logique classique n'en reconnaît pas
l'existence, d'autres formalisations intègrent directement cette
notion dans leur grammaire. C'est le cas des logiques multivalentes,
comme la logique de \bsc{Łukasiewicz} ou la logique floue
\autocite{Williamson1994,Sorensen2018}. Ces logiques alternatives
permettent de raisonner sur des valeurs de vérité intermédiaires,
entre le \enquote{vrai} et le \enquote{faux}, et permettent de
répondre au paradoxe sorite et autres raisonnements jouant sur
\emph{l'imprécision} du langage naturel. Pour reprendre notre exemple
précédent, si l'on se rapproche peu à peu d'un objet, on finira par
pouvoir décrire notre position comme étant \enquote{proche} de cet
objet. Avec un raisonnement fondé sur la logique classique, on est
contraint de définir arbitrairement la limite entre les positions qui
sont \enquote{proches} de l'objet considéré et celles qui ne le sont
pas, alors que les logiques multivalentes permettent de définir un
entre-deux, une partie de l'espace où les propositions logiques :
\enquote{je suis proche} et \enquote{je ne suis pas proche} sont
toutes les deux fausses ou partiellement vraies, en fonction de
l'approche retenue. \textcite{Tye1994}, par exemple, propose une
modélisation basée sur une logique trivalente, dont les valeurs de
vérités sont : \enquote{vrai}, \enquote{faux} et
\enquote{indéterminé}. Avec cette modélisation, la zone située entre
celle validant la proposition \enquote{je suis proche} et celle ne la
validant pas, a une réponse \enquote{indéterminée} à cette
proposition. Cette approche est assez différente des solutions fondées
sur la logique floue, qui considèrent que la proposition \enquote{je
  suis proche} a un degré de vérité qui décroit au fur et à mesure que
l'on s'éloigne de l'objet en question. D'autres théories permettent
également la prise en compte de \emph{l'imprécision,} comme celle des
ensembles approximatifs de \textcite{Pawlak1982}, qui propose une
modélisation trivalente, similaire à celle de \textcite{Tye1994}
\autocite{Pawlak1991}, ou la théorie des fonctions de croyances
\autocite{Shafer1976} qui propose une modélisation de
\emph{l'imprécision} à travers la composition d'hypothèses singletons
représentant un ensemble de solutions à un problème donné.

\paragraph{\emph{Imprécision} et espace}

Les implications spatiales du concept d'\emph{imprécison} ont
également été spécifiquement étudiées. Les notions de \emph{limite} ou
de \emph{frontière} étant centrales en géographie, l’existence
d'objets difficilement délimitables ne va pas sans poser quelques
problèmes épistémologiques. Dès la fin des années 1970, des géographes
comme \textcite{Gale1976} ou \textcite{Leung1979} ont travaillé à
appliquer la logique floue au problème de la \emph{régionalisation.}
Toutefois, ces travaux ne nous semblent pas avoir eu un impact durable
en géographie et de nouvelles propositions, que nous avons recensées
\autocite{Bunel2020}, sont apparues au cours du temps. C'est
principalement à partir des années 2000 que ces questions théoriques
ont été remises au goût du
jour. \textcite{Varzi2001,Varzi2015,Smith2000,Collins2000} ont, par
exemple, travaillé sur la notion de frontière et son lien avec la
notion \emph{d'imprécision.} La question de la formalisation et de
l'implémentation de modèles théoriques permettant la manipulation
d'objets géographiques \emph{imprécis} est, par ailleurs, un champ de
recherche assez dynamique en sciences de l'information géographique
\autocite{Bunel2020}. De nombreux modèles théoriques ont été proposés,
qu'ils soient basés sur la théorie des ensembles approximatifs
\autocite{Schneider1996,Cohn1996,Clementini1996} ou la théorie des
ensembles flous \autocite{Schneider1999}. La prise en compte et la
modélisation de l'imprécision à travers les formalismes théoriques
mentionnés ci-dessus a concerné de nombreux sujets et thématiques,
comme la recherche en archéologie \autocite{Runz2008a, Zoghlami2016},
en architecture \autocite{Arabacioglu2010}, en cartographie
\autocite{Didelon2009,Didelon2011}, en traitement d'image
\autocite{Brandtberg2002}, pour l'aide à la décision
\autocite{Griot2007,Makropoulos2003} ou en appariement de données
géographiques \autocite{Olteanu2008}. Ces formalismes théoriques ont
également été utilisés pour prendre en compte \emph{l'imprécision}
lors de la \emph{spatialisation} de descriptions de positions,
appliquée à l'interprétation automatisée d'images médicales
\autocite{Takemura2012,Vanegas2011,Bloch1996,Hudelot2008}.

\subsubsection{Verrous scientifiques}

Le premier des verrous à la prise en compte de \emph{l'imprécision}
est l'identification d'une théorie qui permette sa modélisation, tout
en étant compatible avec la solution de \emph{spatialisation} des
\emph{indices de localisation} retenue. Chaque théorie capable de
modéliser \emph{l'imprécision} ayant ses spécificités propres, il sera
nécessaire de s'assurer de leur bonne intégration avec la méthode de
spatialisation qui doit être définie.

Le second verrou est la question de l'évaluation de
\emph{l'imprécision.} En effet, s'il est acquis que les
\emph{relations de localisation} ---~et plus généralement le langage
naturel~--- sont \emph{imprécis,} il reste qu'ils peuvent l'être à des
degrés divers. Pour l'illustrer, nous pouvons utiliser les
\emph{indices de localisation :} \enquote{je suis \emph{proche de}
  Grenoble} et \enquote{je suis \emph{aux alentours de} Grenoble}. Ces
deux indices ont une signification semblable: tous deux renseignent
sur la proximité du locuteur avec la ville de Grenoble. On aurait
toutefois du mal à les considérer comme parfaitement équivalents, la
relation de localisation \enquote{\emph{aux alentours de}} nous
semblant plus vague, c'est-à-dire que l'aire que l'on peut définir
comme étant \enquote{\emph{aux alentours de} Grenoble} est plus
étendue que celle qui en est \emph{proche.} On ne peut pas, pour
autant, considérer que la \emph{relation de localisation}
\enquote{\emph{proche de}} est \emph{précise.} Ces deux relations sont
\emph{imprécises,} mais à des degrés divers et, si l'on peut les
ordonner selon leur \enquote{degré \emph{d'imprécision}}, on ne
saurait quantifier cet écart et donc leurs différents degrés
\emph{d'imprécision.}

Cette quantification est cependant une étape indispensable à la prise
en compte de \emph{l'imprécision} des \emph{relations de
  localisation.} L'analyse de la sémantique des \emph{relations de
  localisation} devra donc être complétée par une estimation de leur
\emph{imprécision} et les méthodes de spatialisation définies devront
prendre en considération ces deux aspects.

\subsection{La modélisation de \emph{l'incertitude} des \emph{indices
    de localisation}}
\label{subsec:2-1-3}

Si la prise en compte de \emph{l'imprécision} des \emph{indices de
  localisation} permet d'améliorer la qualité de leur future
modélisation, cette solution ne permet pas d'en gérer la
plausiblité. En effet, certains \emph{indices de localisation} donnés
par le requérant peuvent être faux ou, du moins, sujets à caution et
ce indépendamment de toute notion de \emph{précision.} La prise en
considération ---~aveugle~--- de ce type d'indices ne peut qu'impacter
négativement la qualité de la \emph{zone de localisation probable}
créée par la \emph{fusion} des \emph{indices de localisation.} Les
secouristes sont cependant, grâce à leur connaissance du terrain et à
leur expérience, à même d'identifier la plupart de ces erreurs et
approximations ou, tout du moins, d’émettre un doute sur la véracité
de certains \emph{indices de localisation.} La prise en compte de ces
connaissances, exogènes à l'indice, nous semble très importante, en
plus d'être un excellent moyen de limiter l'impact d'indices peu
plausibles sur la qualité de la spatialisation.

\subsubsection{Contexte scientifique}

La notion \emph{d'incertitude} est fortement liée à celle,
précédemment présentée, \emph{d'imprécision}. Comme cette dernière,
cette notion qualifie une connaissance. Mais là où la notion
\emph{d'imprécision} désigne une caractéristique intrinsèque de la
connaissance, \emph{l'incertitude} est liée à l'observateur. Cs'est une
caractéristique contextuelle qui qualifie le doute de l'observateur
sur la véracité de la connaissance
\autocite{Bouchon-Meunier1995,Bouchon-Meunier2007}. Ces deux notions
sont souvent confondues, notamment dans la littérature anglophone où
des objets géographiques \emph{imprécis} sont qualifiés
\emph{d'incertains} \footnote{Nous reviendrons sur ces points de
  vocabulaire dans la \autoref{sec:3-2}.}.

La prise en compte de \emph{l'incertitude} n'est pas nécessairement
possible avec les mêmes théories que celles utilisées pour modéliser
l'imprécision. La logique floue, par exemple, permet la modélisation
de \emph{l'imprécision,} mais non de \emph{l'incertitude,} à moins
d'être complétée par la théorie des possibilités
\autocite{Bouchon-Meunier1995,Bouchon-Meunier2007}. Au contraire, la
théorie des fonctions de croyances permet de modéliser
\emph{l'imprécision} et \emph{l'incertitude} sans adaptations
\autocite{Bouchon-Meunier1995,Bouchon-Meunier2007}. Enfin, d'autres
théories ne permettent que la modélisation de \emph{l'incertitude,}
sans prise en compte de \emph{l'imprécision.} C'est par exemple le cas
de la théorie des probabilités, du moins dans leur interprétation
épistémique \autocite{Hajek2019}.

Nous n'avons pas connaissance de travaux portant uniquement sur la
modélisation de \emph{l'incertitude} d'objets géographiques. Cette
dernière est toujours couplée à une modélisation de
\emph{l'imprécision.}

\subsubsection{Verrous scientifiques}

Les verrous à la prise en compte de \emph{l'incertitude} sont
identiques à ceux que nous avons identifiés pour la prise en compte de
\emph{l'imprécision.} Dans les deux cas, en effet, l'objectif est
\enquote{d'enrichir} la \emph{spatialisation} des \emph{indices de
  localisation,} en y apportant la prise en compte de critères qui
nous semblent indispensables, à savoir \emph{l'imprécision des
  relations de localisations} et \emph{l'incertitude} des
\emph{indices de localisation.} Dans les deux cas, nous devons veiller
---~et ce pour les mêmes raisons~---, à ce que les théories et les
méthodes qui vont être utilisées soient compatibles avec la méthode de
\emph{spatialisation,} mais également entre elles.

Le second verrou, également partagé avec \emph{l'imprécision,} est la
question de la quantification. En effet, disposer d'une estimation
chiffrée de \emph{l'incertitude} de \emph{l'indice de localisation}
est nécessaire à sa prise en compte. La question est d'autant plus
délicate que \emph{l'incertitude,} contrairement à
\emph{l'imprécision,} est contextuelle. En effet, la présence du même
\emph{indice de localisation} dans deux alertes différentes n'implique
pas qu'ils aient la même \emph{incertitude,} contrairement à leur
\emph{imprécision} qui sera identique, car intrinsèquement liée à la
sémantique de la \emph{relation de localisation.} Ainsi, l'évaluation
de \emph{l'incertitude} devra être faite pour chaque \emph{indice de
  localisation} d'une alerte donnée, contrairement à l'évaluation de
\emph{l'imprécision} qui peut être faire en amont, lors de
l’identification de la sémantique des \emph{relations de
  localisations} (\autoref{sec:2-1}).

% Descriptions non exhaustives
Le dernier verrou que nous avons identifié est que le processus de
localisation de la victime, basé sur l'interprétation par le
secouriste d'une description de position, n'offre aucune garantie
quant à la \emph{complétude} de cette description. La victime peut,
pour de nombreuses raisons, oublier de donner certains détails de sa
position qui pourraient être pertinents pour les secouristes. Ces
derniers peuvent, bien entendu, poser des questions au requérant ou
lui demander des précisions, mais cela ne garantit pas pour autant
qu'un \emph{indice} important, voire essentiel, ne puisse pas être
oublié. Cette \emph{incomplétude} de la description, que l'on peut
aisément supposer systématique, a pour conséquence de rendre
impossible la supposition de l’existence d'indices, qui n'ont pas été
mentionnés explicitement par le requérant. Par exemple, ce n'est pas
parce que la victime ne précise pas qu'elle est en forêt, qu'elle
n'est effectivement pas en forêt, \ie qu'un \emph{indice} non connu du
secouriste n'est pas nécessairement faux. Par conséquent, il est
hasardeux, d'inférer qu'un \emph{indice} est faux car il n'a pas été
donné par une \emph{description de position.}

\subsection{La \emph{fusion} des \emph{indices de localisation}}
\label{subsec:2-1-4}

Comme nous l'avons expliqué précédemment, la construction de la
\emph{zone de localisation probable} nécessite de combiner les
\emph{zones de localisation compatibles} au cours d'un processus que
nous nommons \emph{fusion.} Son rôle est, en première approche,
d'identifier la zone où tous les \emph{indices de localisation} donnés
par le requérant sont vérifiés. Cependant, les deux objectifs
scientifiques précédents ont montré qu'il était nécessaire de prendre
en compte deux éléments principaux, \emph{l'imprécision} des
\emph{relations de localisation} et \emph{l'incertitude} des
\emph{indices de localisation.} Par conséquent, les \emph{zones} à
fusionner, c'est-à-dire les \emph{zones de localisation compatibles}
seront \emph{imprécises.} La méthode \emph{fusion} que nous allons
élaborer doit donc être capable de prendre en compte ces
caractéristiques. Il est donc nécessaire d'identifier une méthode qui
soit compatible avec la théorie de modélisation de l'imprécision
utilisée pour spatialiser les \emph{indices de localisation} et la
théorie de modélisation de \emph{l'incertitude} utilisée.

\subsubsection{Contexte scientifique}

Si le terme de \emph{fusion} qualifie, dans notre travail, une
opération très spécifique, qui consiste à combiner plusieurs zones
construites en spatialisant des \emph{indices de localisation,}
l'usage de termes de \emph{fusion d'imformations} ou de \emph{fusion
  de données} peut s'appliquer à un grand nombre de méthodes
permettant de réduire l'information. Comme l'indique
\textcite{Castanedo2013}, la fusion d'informations peut s'appliquer à
de nombreux domaines de recherche. Il s'agit par conséquent d'un sujet
abondamment étudié.

Dans un premier temps, des méthodes de classification comme les
\emph{k-means} peuvent être considérées comme de la fusion
d’information, car elles permettent de regrouper les individus selon
un comportement commun. De même, les analyses factorielles permettent
également de fusionner des informations multiples, en synthétisant le
comportement de différentes variables. Les méthodes de \emph{fusion
  des données} ne se limitent pas aux modèles statistiques et d'autres
théories, comme la théorie des fonctions de croyances ou celle des
sous-ensembles flous, permettent de fusionner des informations
provenant de sources diverses. Cette démarche est, par exemple, au
cœur du travail de \textcite{Olteanu2008} qui a été amenée à fusionner
des informations issues de données géographiques et de connaissances
métier dans le but d’appareiller les objets issus de bases de données
différentes, représentant le même objet du monde réel. Dans ce cas, la
fusion des informations est réalisée à l'aide d'un opérateur
particulier de la \emph{théorie des fonctions de croyances}
\autocite{Shafer1976}, utilisée durant tout ce travail pour permettre
la modélisation de \emph{l'imprécision} et de \emph{l'incertitude.}

\subsubsection{Verrous scientifiques}

Le processus de \emph{fusion} des indices est une opération qui se
déroule après la \emph{spatialisation} des différents \emph{indices de
  localisation.} Par conséquent, il s'agit d'une opération qui est
fortement impactée par les choix méthodologiques effectués en amont,
comme, par exemple, celui de la théorie de modélisation de
\emph{l'imprécision }des \emph{relations de localisation.}

Comme nous l'avons expliqué ci-dessus (\ref{subsec:2-1-2} et
\ref{subsec:2-1-3}), nous souhaitons que la prise en compte de
\emph{l'imprécision des relations de localisation} et de
\emph{l'incertitude} des \emph{indices de localisation} soit
intrinsèque à la méthode de \emph{spatialisation} des \emph{indices de
  localisation} (\ref{subsec:2-1-1}). Il est, par conséquent,
nécessaire qu'il en soit de même pour la méthode de \emph{fusion} des
\emph{zones de localisation compatibles}.
% 
Or, ces deux composantes sont, par définition, variables. Deux indices
issus d'une même alerte peuvent avoir une \emph{incertitude} évaluée
différemment par le secouriste. Ce peut être, par exemple, le cas si
un requérant décrit sa position comme étant : \enquote{proche d'un
  chalet et probablement au sud du sommet}. Dans ce cas, le premier
\emph{indice de localisation} : \enquote{je suis proche d'un chalet},
serait sûrement, considéré comme plus \emph{certain} que le second :
\enquote{je suis probablement au sud sommet}. La méthode de
\emph{fusion} des \emph{indices de localisation} doit donc être
capable de prendre en considération cette différence entre les deux
indices, par exemple en les pondérant. De la même manière, des
certitudes différentes entre les indices de localisation doivent
pouvoir être prises en compte lors du processus de \emph{fusion.}

Un autre point important est que le processus de \emph{fusion} sera
potentiellement utilisé plusieurs fois lors du traitement d'une
alerte. En effet, les \emph{indices de localisation} sont donnés par
le requérant, au fur et à mesure de l'appel téléphonique \footnote{Et
  donc du processus de localisation.}, parfois en réponse aux
questions des secouristes. L'ensemble des \emph{indices de
  localisation modélisés} est donc amené à évoluer, comme la
\emph{zone de localisation probable} qui résulte de la \emph{fusion}
des \ac{zlc} les spatialisant. La méthode de \emph{fusion} doit donc
prendre cet aspect en considération et nous devons veiller à ce que
les résultats de ce processus ne soient pas impactés par l'ordre de la
\emph{fusion.}

Un dernier verrou,x fortement lié à la question de la modélisation de
\emph{l'incertitude,} est celui de la gestion des conflits entre
indices. Il n'est, en effet, pas exclu que deux indices d'une même
alerte se contredisent, rendant dès lors impossible la construction de
la \emph{zone de localisation probable.} La prise en compte de
\emph{l'incertitude} des indices permet, en partie, de contourner ce
problème.

% \subsection{L'évaluation des \emph{zones de localisation}}
% \label{subsec:2-1-5}

% TODO
% \tdi{Je mets ici le commentaire d'ana-maria sur cette partie: \\ Cette
%   partie devrait être revue après avoir rédigé le chapitre sur
%   l'évaluation; Il faut être plus précis dans la description de ton
%   objectif scientifique. C'est normal, pour l'instant, comme cette
%   partie n'est pas aboutie tu ne peux pas dire plus que "bonnes
%   zones"; il faut voir ce que tu vas finalement évaluer: \\ 1/ les
%   zones de localisation compatibles --> ce qui implique que tu évalues
%   la méthode de spatialisation \\ 2/ les zones de localisation
%   probables --> ce qui implique que tu évalues le processus de
%   localisation `\\ 3/ tu attribue une note de confiance aux zones de
%   localisations probables --> ce qui implique que tu évalues "la
%   croyance" que la victime se trouve dans une zone de localisation
%   probable }

% Enfin, notre dernier objectif est de proposer une méthode d'évaluation
% des \emph{zones de localisation} construites et ce, qu'il s'agisse des
% \emph{zones de localisation compatibles} ou \emph{probables.} Il est,
% en effet, primordial de savoir si les zones construites à l'aide de
% notre méthodologie de \emph{spatialisation} des \emph{indices de
%   localisation} sont des \enquote{bonnes zones} ou non. Une mesure de
% qualité pertinente offrira la possibilité de juger de la qualité des
% méthodes de \emph{spatialisation} développées. Cependant cet objectif
% nécessite définir des critères d'évaluation pertinants et adaptés au
% contexte applicatif de ce travail.

% \subsubsection{Contexte scientifique}

% La question de l’évaluation des résultats est centrale dans la
% recherche scientifique. Le processus d'évaluation consiste à vérifier
% que les résultats produits sont une bonne approximation de la
% réalité. On peut définir deux grandes catégories de méthodes
% d'évaluation des résultats. La première est \emph{l'évaluation
%   supervisée,} où les résultats, quelle que soit leur forme sont
% comparés à un jeu de données ou d'observations représentatifs de la
% réalité. C'est la confrontation du comportement du modèle avec les
% résultats observés qui permet d'en évaluer la qualité. Généralement,
% tout écart tranche en défaveur du modèle ou de la théorie utilisée,
% cependant l'inverse est également possible, notamment lorsque la
% théorie utilisée est considérée comme très robuste, c'est par exemple
% le cas du \emph{modèle standard de la physique des particules,} que
% les physiciens considèrent comme si éprouvé que les écarts observés
% entre le modèle et les observations sont, par défaut, imputés à des
% erreurs d'observation. À notre connaissance, seule la physique
% théorique dispose de modèles si polis que leurs prédictions théoriques
% sont considérées comme plus fiables que les observations. On peut,
% cependant, également parler d'\enquote{\emph{approche supervisée}}
% lorsque les résultats sont comparés à une référence qui ne correspond
% pas nécessairement à la réalité. \textcite{Olteanu2008}, par exemple,
% a été amenée à évaluer les résultats d'un appariement de données
% géographiques. Pour ce faire les résultats de l’appariement
% automatiques ont été comparés à ceux d'un appariement manuel, faisant
% office de référence. Ce dernier ne correspond donc à aucune
% \enquote{réalité}, il s'agit juste d'un cas idéal, ce qui n’empêche
% pas de considérer cette méthode d’évaluation comme une \emph{approche
%   supervisée.} La seconde méthode d'évaluation existante est
% l'évaluation \emph{non-supervisée.}  Cette catégorie regroupe toutes
% les situations où on ne dispose pas d'observations de référence
% permettant d'évaluer les résultats produits. C'est, par exemple, le
% cas lorsque l'on cherche à évaluer des résultats par des tests
% utilisateurs, comme le propose \textcite{Dumont2018} pour évaluer la
% qualité d'une généralisation cartographique. Ces deux approches ne
% sont cependant pas incompatibles. Par exemple, \textcite{Jolivet2014}
% combine ces deux approches pour évaluer un modèle de simulation. Les
% trajectoires d’animaux construites par le modèle sont, d'une part
% comparées quantitativement aux observations instrumentales
% (\emph{approche supervisée}), mais également évaluées qualitativement
% par des experts de la thématique (\emph{approche non-supervisée}).

% \subsubsection{Verrous scientifiques}

% Le principal verrou scientifique à la mise en place d'une méthode
% d'évaluation, réside dans le fait qu'il est insatisfaisant de nous
% baser uniquement sur une comparaison entre une \enquote{zone estimée},
% \ie la \emph{zone de localisation probable} et la position réele de la
% victime. En effet, l'absence ---~et inversement la présence~--- de la
% position de la victime dans la \emph{zone de localisation probable}
% construite, ne signifie pas nécessairement que la modélisation est
% erronée. Les \emph{indices de localisation} peuvent être incorrects,
% sans pour autant avoir été identifiés comme tels par le secouriste, ou
% le requérant peut se référer à un objet qui n'existe pas dans les
% basses de données utilisées. De plus, avec ce critère, l'évaluation ne
% peut se faire sans la connaissance de la position réelle de la
% victime, ce qui ne permet pas d'envisager la mise en place d'une
% \enquote{évaluation dynamique} des résultats, à même de renseigner le
% secouriste sur la qualité des résultats lors de la phase de
% localisation. L'évalution des résultats ne peut être réalisée qu'après
% avoir pris connaissance de la position de la victime. Une autre
% solution serait de considérer que, moins la \emph{zone de localisation
%   probable} est étendue, plus elle est précise et donc de
% \enquote{meilleure qualitée}. Cependant cette approche n'est pas plus
% satisfaisante.  En effet, la \emph{zone de localisation probable}
% correspond à l'ensemble des \emph{positions} validant la description
% ---~transmise sous la forme \emph{d'indices de localisation}. Or,
% plusieurs zones peuvent correspondre à cette descrition, à fortiori si
% les \emph{indices de localisations} donnés sont peu nombreux ou
% discriminants. Utiliser un critère de \enquote{taille de zone}
% reviendrait donc à estimer la qualité des \emph{indices de
%   localisation} et non celle de leur modélisation.

% L’évaluation de la qualité de la \emph{zone de localisation probable}
% impose donc de développer une méthode à même d'estimer la qualité de
% la modélisation des \emph{indices de localisation} donnés par le
% requérant et non celle des \emph{indices} en eux-mêmes. 

%%% Local Variables:
%%% mode: latex
%%% TeX-master: "../../../../main"
%%% End:
