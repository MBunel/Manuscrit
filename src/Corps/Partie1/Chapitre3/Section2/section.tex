\tdi{Ajouter notion ``espace liminal''
  https://journals.openedition.org/rga/2115}

\tdi{Citation
  https://www.sciencedirect.com/science/article/pii/B9780444886507500287}

Lorsque George Perec relate sa visite d’une maison usonienne1 du
Michigan, il en décrit le délicat cheminement vers l’intérieur :

\begin{quotation}
  « On commençait par suivre un sentier […]. Peu à peu, […] sans qu’à
  aucun instant on ait été en droit d’affirmer avoir perçu quelque
  chose comme une transition […], le sentier devenait […] une allée
  […]. Puis apparaissait […] une toiture […] pratiquement
  indissociable de la végétation […]. Mais en fait, il était déjà trop
  tard pour savoir si l’on était dehors ou dedans. » (Perec, 1974,
  pp. 52-53).
\end{quotation}

Les jeux avec l’environnement et les matériaux, relevés par
l’écrivain, sont les éléments d’un parcours savamment orchestré par
l’architecte pour fondre la construction dans son environnement,
trompant le visiteur et l’empêchant d’identifier une délimitation
claire entre dehors et dedans ; deux espaces tacitement considérés
comme immiscibles. Ainsi, en lieu et place d’une rupture franche,
c’est une transition progressive qui les sépare, rendant la définition
d’une limite, autre qu’arbitraire, impossible.  Or, la délimitation
d’espaces cohérents est centrale dans la réflexion géographique et
l’existence d’objets spatiaux difficilement délimitables ne va pas
sans soulever des questions techniques et épistémologiques
\autocite{Burrough1996b}, auxquelles de nombreux travaux, comme ceux
cherchant à délimiter des espaces à partir de ressentis
\autocite{Arabacioglu2010}, de descriptions de positions (Jones et
al., 2007 ; Wolter et Yousaf, 2018 ; Bunel et al., 2019) ou de cartes
mentales (Dutozia et al., 2014), ont été confrontés. Ces différents
travaux ont pour point commun d’avoir nécessité l’emploi de modèles
avancés, permettant la représentation d’objets spatiaux aux frontières
mal délimitées. Mais le grand nombre de modèles de ce type et
l’absence de consensus pour l’un d’eux rend délicat leur recensement
et leur analyse en vue d’une application originale.  C’est ce travail
de recensement et de classification que nous souhaitons entreprendre
ici, en espérant qu’il puisse servir de point d’entrée à toute
personne amenée à manipuler des objets spatiaux difficilement
délimitables ou souhaitant se familiariser avec ces notions et leur
modélisation. Nous présentons avant tout une catégorisation des
différents modèles proposés dans la littérature, mais nous
présenterons également le cadre conceptuel au quel ils se rattachent
et notamment la notion d’imprécision au centre de ces questions. Les
différentes implémentations des modèles présentés seront également
abordées. Notre ambition est de traiter tous les aspects du problème,
en partant des aspects les plus théoriques et conceptuels (définition
des termes et présentation des théories de rattachement), avant
d’aborder des points plus techniques (implémentations), en passant par
la présentation des différentes modélisations proposées dans la
littérature.  Nous ne souhaitons cependant pas proposer une typologie
originale des différents concepts, la question ayant déjà été
largement traitée (Bouchon-Meunier, 1995 ; Fisher et al., 2006 ;
Devilliers et al., 2019). Nous présenterons également les différentes
théories mathématiques permettant la modélisation de l’imprécison,
comme l’ont récemment proposé Batton-Hubert et Pinet (2019), ainsi que
les implémentations de modèles proposées.  Nous commencerons par
présenter plus abondamment ces notions et définir les différents
concepts, tels que le vague ou l’imprécision, nécessaires à leur
compréhension (partie 2). Puis nous énumérerons les différentes
théories mathématiques permettant de modéliser l’imprécision (partie
3). Enfin nous présenterons les différentes modélisations de
l’imprécision spatiale proposées dans la littérature (partie 4).


\subsection{Les concepts de \emph{vague} et d’\emph{imprécision} spatiale}

Pour commencer, nous allons définir les notions d’\emph{imprécision}
et de \emph{vague} et présenter leurs spécificités lorsqu’elles sont
appliquées au contexte spatial. Nous présenterons d’abord les notions
dans leur ensemble, avant de nous concentrer sur les liens entre
\emph{imprécision} et géographie; puis, nous définirons\emph{
  l’imprécision spatiale,} concept qui sera utilisé tout au long de
cet état de l’art.  L’exercice de définition d’un concept équivoque,
tel que le vague, n’est pas une tâche aisée, car l’on est rapidement
confronté à l’imprécision sémantique du langage naturel. C’est, en
partie, ce constat qui conduisit les philosophes Gottlob Frege
(1848–1925) et Bertrand Russell (1872–1970) à travailler sur une
formalisation mathématique de la logique, à même d’affranchir le
processus de réflexion des ambiguïtés du langage naturel (Williamson,
1994). La réflexion contemporaine sur les notions de précision, de
vague et d’imprécision remonte, selon Williamson (1994), aux travaux
de Russell et plus spécifiquement à la publication, en 1923, de son
article Vagueness (Russell, 1923). Pour Russell, le vague est l’opposé
de la précision. Les deux concepts ont pour domaine tout système de
signes et ne se limitent donc pas au langage naturel. Ils concernent
tout type de représentation (e.g. cartes, photographies, mots) et
n’ont donc de sens que pour qualifier une relation entre deux systèmes
de signes, définie comme précise si bijective, i.e. pour un système de
signes donné, quel que soit le signe considéré, il ne partage sa
signification qu’avec un et un seul signe d’un second système de
signes. À l’inverse, la relation entre deux systèmes de signes est
vague si une représentation a plus d’un (ou aucun) équivalent dans le
second système, laissant dès lors place à l’interprétation. Pour
illustrer le concept de vague, Russell prend l’exemple du mot “rouge”
décrivant une teinte tacitement connue de tous, mais dont on ne peut
qu’arbitrairement fixer les limites. De la même manière, les concepts
de lac et île ne sont pas suffisamment précis pour que leur
dénombrement soit trivial (Sarjakoski, 1996). On pourrait multiplier
les exemples à loisir, car, dans la conception russellienne, aucun
domaine n’échappe au vague ; toute représentation l’est1, à des degrés
divers, et la précision n’est qu’un idéal, hors d’atteinte.  Dans la
littérature, le terme imprécis est régulièrement utilisé comme
synonyme de vague, notamment par Zadeh (1965) et dans une grande
partie des travaux se rattachant à la logique floue. Par métonymie, le
terme flou2 est également employé dans le sens de vague,
imprécis. C’est, par exemple, le cas lorsque Lagacherie et al.  (1996)
parlent de fuzziness ou encore quand Brunet et al. (1992, p. 218)
définissent le flou comme : « la partie d’un système ou d’un espace
dont les contours et les limites sont, soit imparfaitement connus ou
connaissables, soit instables, soit imprécis […]". De nombreux autres
termes sont ponctuellement utilisés, rendant la terminologie confuse
(Tableau 1). Pour rendre notre propos le plus clair possible, nous
n’emploierons le terme flou que pour qualifier des formalisations
fondées sur la théorie des sous-ensembles flous de Zadeh. De plus,
pour rester le plus proche possible du vocabulaire utilisé en
géomatique nous préférerons le terme imprécis à celui de vague.

\begin{table}
  \centering
  \begin{tabular}{rL{6cm}L{6cm}}
  \toprule
  & {\bfseries Anglophone} & {\bfseries Francophone} \\
  \midrule
   {\bfseries Précis} & \emph{Crisp, Sharp, Well defined, Fiat boundaries} & Net
  \\
  {\bfseries Imprécis} & \emph{Fuzzy, Indeterminate, Undefined, Uncertain,
                         Ill-Defined, Unclear, Vagueness, Bona fide boundaries} & Flou, Incertain, Vague
  \\
  \bottomrule
\end{tabular}
  \caption{Termes utilisés dans la littérature comme synonymes de
    précis et d’imprécis}
  \label{tab:termes_vague}
\end{table}

La notion d’imprécision peut être associée à d’autres concepts, comme
l’exactitude chez Russell (1923), que l’on retrouve chez
Bouchon-Meunier (1995, 2007) sous le nom d’incertitude. Ici,
l’incertitude est entendue comme le doute que l’on peut avoir sur la
validité d’une connaissance (Bouchon-Meunier, 1995). L’imprécision et
l’incertitude sont foncièrement liées et varient généralement en sens
inverse (Russell, 1923). Ainsi, si la proposition (1) : « la distance
de la Terre à la Lune est de 384 397 km » est plus précise que la
proposition (2) : « la distance de la Terre à la Lune est d’environ
384 000 km », mais la seconde proposition est plus certaine car, son
cadre de validité (la plage de valeurs de la distance Terre-Lune) est
plus large. On peut en effet, considérer que la proposition (2) reste
vraie pour une distance réelle de 383 500 ou 384 999 km alors que la
moindre variation de l’ordre d’un kilomètre suffit à invalider la
première proposition (1). Dans certains cas, et notamment lorsque ces
notions sont appliquées à des objets spatiaux, il peut être délicat de
distinguer ces deux concepts. Cependant, ils sont fondamentalement
différents : l’imprécision est une caractéristique invariable, alors
que l’incertitude est contextuelle. Pour reprendre l’exemple
précédent, la proposition (2) est imprécise et le restera quel que
soit le contexte, alors que sa certitude dépend des connaissances de
l’observateur. La véracité des propositions (1) et (2) est, toutes
choses égales par ailleurs, invariable, mais la certitude de cette
véracité est contextuelle.  L’imprécision et l’incertitude sont
également associées à la notion d’incomplétude, qui désigne une
connaissance partielle. Ceci est dû au fait qu’un manque de
connaissances peut entraîner des incertitudes, mais également des
imprécisions (Bouchon-Meunier, 1995 ; Bouchon-Meunier, 2007). Pour
Bouchon-Meunier et plus généralement pour la communauté de chercheurs
en intelligence artificielle, la composition de ces trois concepts
définit la notion d’imperfection (Bouchon-Meunier, 1995). Dans la
suite de ce document, nous travaillerons à partir de cette typologie,
même si de nombreuses autres typologies de concepts ont cependant été
proposées, que ce soit dans le domaine de l’intelligence artificielle
ou de la géomatique (Fisher et al.,, 2006).

\subsubsection{Imprécision et géographie}

Les objets et les concepts géographiques n’échappent évidemment pas à
l’imprécision. Ainsi, Russell (1923) mentionnait déjà l’existence
d’objets dont la délimitation spatiale est imprécise, tel que le
système solaire. De nombreux autres objets spatiaux imprécis ont été
identifiés, comme l’illustre l’exercice de définition du Brownfield1,
entrepris par Alker et al. (2000) et relevé par Bennett (2001), qui y
voit un bon exemple de la difficulté d’identifier une délimitation
satisfaisante d’espaces naturels. Le Brownfield, tout comme les forêts
(Bennett, 2001 ; Dilo, 2006 ; Fisher et al.,, 2006), les montagnes
(Varzi, 2001, 2015 ; Fisher et al.,, 2006 ; Chaudhry et Mackaness,
2008), les vallées (Schneider, 2003) ou même le Soleil (Simons, 1999),
appartiennent à cette catégorie d’objets spatiaux dont on ne peut
fixer une limite. Cette énumération pourrait laisser penser que
l’imprécision ne concerne pas les artefacts, pourtant l’expérience de
Perec (Introduction) nuance cette affirmation. Comme l’indique Campari
(1996), l’identification des frontières d’un artefact, n’est pas aisée
puisque dépendante du contexte d’observation. Ainsi, la limite d’une
ville ou d’un village est tout aussi vague que celle d’une zone
frontalière (Varzi, 2001 ; Fisher et al.,, 2006), comme l’illustre la
grande variabilité des définitions du concept de ville.  Tous ces
objets géographiques, généralement qualifiés de vagues (Erwig, 1997),
imprécis (Winter, 2000), flous (Lagacherie et al., 1996) ou d’objets
aux frontières indéterminées (Burrough, 1996), s’opposent aux objets
dits nets (Schneider, 2001) ou précis. Smith et Varzi (1995, 1997,
2000) font usage d’un vocabulaire très différent en opposant les fiat
boundaries, i.e. les frontières précises qu’ils estiment profondément
liées à un processus cognitif, aux bona fide boundaries, caractérisant
les objets spatiaux dont la délimitation est univoque (Varzi,
2015). Pour Couclelis (1996), les objets géographiques nets sont
d’avantage l’exception que la norme. Ce constat est corollaire de
l’avis d’Odd Ambrosetti (1987, p. 200) pour qui "[…] il est
problématique et généralement arbitraire de tracer des limites […]"2,
limites qui, selon Brunet (2001, p. 106), sont "[…] indécises, fuyant
sans cesse devant l’analyse, et même, localement
indécidables ». Dutozia et al. (1994) considèrent, quant à eux, que
"[…] l’espace géographique est par essence flou […]".  Ainsi, si les
concepts présentés jusqu’ici nous semblent actuellement peu utilisés
en géographie, de nombreuses notions et objets entrant dans le champ
d’étude de la discipline y sont fondamentalement liés. C’est notamment
le cas des différents maillages administratifs, comme les régions
(Brennetot et al., 2014), mais également les frontières (Brunet et
al., 1992), les seuils (Brunet et al., 1992 ; Lévy et Lussault, 2013),
les discontinuités (Brunet et al., 1992 ; 1997), les franges (Brunet
et al., 1992), les confins (Brunet et al., 1997) ou encore les fronts
pionniers (Brunet et al., 1992), qui, comme tous les concepts dérivant
de la notion de limite sont généralement définis comme pouvant être
graduels ou progressifs (Brunet et al., 1992 ; Lévy et Lussault,
2013), c’est-à-dire foncièrement imprécis.  La question de la
formalisation des objets spatiaux imprécis a cependant été abordée en
géographie, avant même le développement des systèmes d’information
géographiques (Robinson, 2003). Dans les années 1970 où, à la suite de
l’élaboration de la théorie des sous-ensembles flous (Zadeh, 1965),
plusieurs géographes, rattachés au courant béhavioriste, tels que Gale
(1972 ; 1976), Pipkin (1978) ou Leung (1979, 1987) ont identifié les
problèmes que l’existence d’objets géographiques aux limites
imprécises pouvaient poser à la géographie. Parmi ces problèmes, Gale
(1976) a identifié la question de la régionalisation. Cette
problématique sera également abordée par Rolland-May (1996 ; 1999)
lors de ses travaux sur la définition de territoires de cohérence. Les
travaux béhavioristes aboutiront à la formalisation du concept d’objet
géographique imprécis à l’aide de la théorie des sous-ensembles flous
(Leung, 1987). Ces travaux n’auront, semble-t-il, pas suffi à inscrire
durablement le concept d’imprécision dans le champ de la géographie;
puisque des publications ré-introduisant ce concept en géographie,
apparaîtront régulièrement. C’est notamment le cas de Fisher \& Wood
(1998), Collins et de Varzi (2000), qui se fonderont indépendamment
sur l’exemple de la définition d’une montagne pour introduire cette
notion.  Parallèlement, Christiane Rolland-May (1984 ; 1987) se
fondera notamment sur les travaux de Gale et Leung pour développer le
concept d’espace géographique flou. Cette dénomination qualifie la
formalisation, à l’aide de la théorie des sous-ensembles flous, de
l’espace tel que conceptualisé en géographie. Ce travail permettra à
Rolland-May de proposer une définition formelle de notions courantes
utilisées en géographie, telles que les franges (Rolland-May ,1987),
définies comme la limite floue d’un espace géographique, ou les
discontinuités, décrites comme une configuration particulière
d’ensemble flou (Rolland-May, 2003). Les différents travaux de
Rolland-May autour de la question de l’imprécision en géographie ont
permit à différents chercheurs d’aborder différemment des questions
géographiques (Dutozia et al., 2014), comme, par exemple, de Ruffray
et Hamez (2004), qui feront usage des concepts développés par
Rolland-May pour quantifier la cohérence de territoires, ou Didelon et
al. (2011) qui emploient la logique floue pour exploiter des cartes
mentales.

\subsubsection{L’imprécision spatiale}

Nous proposons d’utiliser le terme d’imprécision spatiale pour décrire
l’application du concept d’imprécision aux objets spatiaux. Par ce
terme, nous entendons qualifier toutes les situations où un objet
spatial, quelle que soit sa nature, voit ses limites difficilement
identifiables. Il s’agit donc d’un concept ne portant que sur la
dimension spatiale et non sur les autres aspects. Par exemple, la
difficulté de délimitation spatiale d’un objet géographique tel que la
forêt entre dans le cadre de l’imprécision spatiale. Ce n’est
cependant pas le cas de la difficulté de définition du concept en
lui-même, il s’agit dans ce cas d’imprécision sémantique. Ainsi,
l’imprécision spatiale n’est qu’un cas spécifique du concept général
d’imprécision, précédemment présenté, mais son cadre d’application et
les spécificités de la question spatiale justifient la définition d’un
nouveau concept.  Pour illustrer ce concept, nous allons nous appuyer
l’exemple de la définition des rives d’un lac artificiel. La Figure 1
est une orthophotographie de la partie ouest du lac du Chambon (Isère)
sur laquelle a été dessinée la limite de l’eau. On peut cependant se
demander si la limite que nous avons tracée est une délimitation
satisfaisante de l’objet lac. En effet, le niveau de l’eau est amené à
bouger au cours du temps. L’orthophotographie permet d’identifier ces
zones, dépourvues de végétation et situées au-delà de la limite
représentée sur la Figure 1. On peut donc tracer une seconde limite,
celle de la zone atteignable par les eaux (Figure 2) et considérer que
c’est ce nouveau tracé qui délimite l’objet lac.

\begin{figure}
  \centering
  %\missingfigure
  \caption{Saisie manuelle de la limite du lac du Chambom}
  \label{fig:lim_champ}
\end{figure}

\begin{figure}
  \centering
  %\missingfigure
  \caption{Saisie manuelle d'une limite alternative}
  \label{fig:lim_champ_alt}
\end{figure}

Toutefois, aucune de ces délimitations n’est réellement
satisfaisante. Peut-on considérer qu’une zone pouvant être découverte
appartient autant à l’objet lac qu’une zone qui est toujours
recouverte d’eau ? À l’inverse, peut-on considérer qu’une zone
intermittemment située sous l’eau n’appartient pas au lac de la même
manière que la forêt située à plusieurs dizaines de mètres de là ?
Cette difficulté de délimitation est liée, comme nous l’expliquions
précédemment, à l’imprécision de l’objet lac. On ne peut en définir
une limite précise autrement qu’arbitrairement. Nous avons cependant
pu tracer deux limites précises1, celle de la zone recouverte d’eau
(Figure 1) et celle de l’étendue maximale du lac (Figure 2). Ces deux
frontières délimitent une aire de transition, entre le lac et son
extérieur (Figure 3), c’est-à-dire la frontière du lac.

\begin{figure}
  \centering
  %\missingfigure
  \caption{Mise en évidence de la limite imprécise du lac}
  \label{fig:lim_champ_imp}
\end{figure}

De même que pour l’imprécision, le concept d’incertitude ne voit pas
sa définition générale impactée par la prise en compte de la dimension
spatiale. Cependant, la multiplicité des termes utilisés dans la
littérature, les contradictions entre auteurs et les représentations
graphiques utilisées pour présenter les concepts sont sources de
nombreuses confusions entre les concepts d’imprécision et
d’incertitude spatiale.

Comme expliqué précédemment, l’incertitude qualifie le doute que l’on
peut avoir sur une connaissance. On peut donc définir l’incertitude
spatiale comme le doute sur la position d’un objet. Ce concept peut
être, tout du moins selon Tøssebro et Nygård (2002), décomposé en deux
éléments : l’incertitude positionnelle et l’incertitude
morphologique. L’incertitude positionnelle qualifie un doute sur la
position d’un objet spatial. Tøssebro et Nygård (2008) prennent comme
exemple l’estimation par sonar de la position d’un sous-marin. Ce cas
offre une bonne opportunité pour distinguer imprécision et incertitude
spatiale. L’objet sous-marin, est, de part sa nature d’artefact, net ;
l’identification de ses frontières, à une échelle donnée, ne pose pas
de problèmes. Cependant, sa position est mal connue, puisqu’estimée à
l’aide d’un outil peu précis qui ne peut qu’estimer une zone de
présence. La position de l’objet est donc incertaine. L’incertitude
morphologique qualifie, quant à elle, un doute sur la forme de
l’objet, c’est-à-dire. sur la position de sa frontière. C’est pourquoi
on peut considérer que l’incertitude morphologique correspond à une
incertitude positionnelle portant uniquement sur la frontière de
l’objet. On peut prendre comme exemple une nappe phréatique, dont
l’étendue ne peut être qu’estimée par des relevés terrain. Il est donc
possible de savoir si la nappe est présente ou non en un point de
mesure, mais non d’en définir la frontière, ce qui se traduit par une
incertitude sur la position de la frontière, une incertitude
morphologique telle que définie par Tøssebro et Nygård. La proximité
des concepts d’incertitude morphologique et positionnelle peut
expliquer pourquoi les autres définitions de l’incertitude spatiale,
notamment issues des travaux de Clementini (2008), Lagacherie et
al. (1996), Freksa et Barkowsky1 (1996) ou Schneider (1999) fusionnent
ces deux notions. Dutton (1992), quant à lui, inclut ces deux notions
dans sa définition de l’incertitude positionnelle.

On peut illustrer la notion d’incertitude spatiale en réutilisant
l’exemple de la délimitation du lac du Chambon (Figures 1 et
2). Lorsque nous avons tracé la limite du niveau maximal de l’eau
(Figure 2), nous nous sommes appuyés sur l’emplacement de la
végétation. Cependant, dans certains cas, notamment pour la rive sud
du lac, il s’est avéré difficile d’identifier la bonne limite et ce à
cause de la présence de certaines poches de végétation. Ainsi, la
frontière sud de l’étendue maximale du lac est incertaine, son tracé
est contestable, mais uniquement à cause de notre manque de
connaissances. Il ne s’agit pas d’imprécision. La Figure 4 représente
une zone au sein de laquelle le tracé exact de la frontière est
incertain, c’est-à-dire qu’au sein de cette zone, tous les tracés sont
envisageables2. Ainsi, le lac du Chambon, rentre dans la catégorie des
objets spatiaux à la fois imprécis et incertains.

\begin{figure}
  \centering
  %\missingfigure
  \caption{Mise en évidence de l'incertitude pour la limite des hautes
  eaux}
  \label{fig:lim_champ_inc}
\end{figure}

L’incertitude et l’imprécision spatiales peuvent cohabiter, comme dans
le cas du lac du Chambon. La Figure 5 donne un aperçu plus théorique
de la différence qu’il peut y avoir entre ces deux
notions. L’incertitude spatiale est représentée par plusieurs
frontières, illustrant le doute sur la position de la limite de
l’objet. L’imprécision spatiale est, quant à elle, représentée par une
bande, marque d’une frontière progressive, non réductible à une ligne.

\begin{figure}
  \centering
  %\missingfigure
  \caption{Distinction entre les notions d’imprécision et
    d'incertitude spatiale (pour des raisons de lisibilité seule la
    frontière de l’objet spatial est représentée).}
  \label{fig:inc_vs_imp}
\end{figure}

De façon similaire à ce qui a été décrit précédemment, l’incertitude
et l’imprécision spatiales sont liées. Par exemple, le fait qu’un
objet géographique soit imprécis complexifie l’identification de sa
frontière, ce qui se traduit par une incertitude morphologique
(Lagacherie et al., 1996). De plus, la précision et la certitude des
attributs d’un objet géographique sont fortement liées à la précision
et à la certitude spatiale de ce même objet et inversement (Mark et
Csillag, 1989). Ainsi, la définition d’un objet géographique à partir
de données imprécises le sera elle-même, et le recueil d’informations
au sein d’un objet géographique dont la frontière est incertaine ne
pourra qu’être une opération qui l’est tout autant.

Différents auteurs ont listé des facteurs expliquant l’apparition de
l’imprécision et de l’incertitude spatiales, comme Freksa et Barkowsky
(1996) ou Dutton (1992) qui identifient quelques facteurs
explicatifs. D’autres travaux (Hadzilacos, 1996 ; Evans et Waters,
2008) vont plus loin dans le détail en proposant une typologie plus
poussée de ces différents facteurs. Enfin, des typologies d’objets
spatiaux imprécis, comme celle proposée par Liu et al. (2019),
permettent d’identifier d’autres causes, inhérentes au processus de
construction des objets spatiaux.

Une première cause de l’imprécision spatiale, correspondant par
ailleurs à la majorité des exemples précédents, est liée à
l’imprécision de la définition. Par exemple, l’objet géographique
montagne, n’est pas (seulement) imprécis à cause d’une quelconque
difficulté technique limitant la précision des mesures, il l’est car
le concept montagne n’est pas suffisamment clair pour permettre la
délimitation précise d’une portion d’espace. C’est généralement
l’imprécision du concept qui rend l’objet géographique imprécis
(Freksa et Barkowsky, 1996). L’imprécision du concept est à distinguer
des définitions concurrentes, ce qu’Evans et Waters (2008) nomment
definitional disagreement1. C’est, par exemple, le cas des frontières
contestées, nécessairement mutuellement exclusives ; qui, même si
définies aussi précisément que possible, ne permettent pas de
construire une frontière unique, sinon en admettant une part
d’incertitude spatiale. Par conséquent, le definitional disagreement,
et plus généralement, l’existence de géométries concurrentes pour un
même individu (Hadzilacos, 1996), sont une source d’incertitude,
inhérente au choix d’une possibilité parmi l’ensemble des possibles.

Cependant, l’imprécision d’une définition peut être souhaitée,
Hadzilacos (1996) parle alors de don't care [boundaries]1. C’est, par
exemple, un cas que l’on retrouve fréquemment lors de la description
en langage naturel d’une position. Un exemple de Bateman et
al. (2010), illustre bien cette situation ; si pour décrire sa
position, une personne dit : « Je suis à la Poste », on ne peut pas en
conclure qu’elle est située à l’intérieur d’une agence postale. En
effet, si la file d’attente sort du bâtiment, cette description sera
toujours valable. La limite de « la Poste » est donc peu précise, mais
dans ce contexte, il n’est pas nécessaire qu’elle le soit
davantage. L’information que le locuteur cherche à communiquer est sa
proximité et son interaction avec une agence postale, et non sa
présence au sein du bâtiment.

La dimension temporelle peut également être une source
d’imprécision. Hadzilacos (1996) et Liu et al. (2019) mentionnent
respectivement l’existence de time-varying [boundaries]1 et de dynamic
boundary objects2 pour qualifier des objets géographiques dont la
frontière varie dans le temps. C’est, par exemple, le cas d’un front
de mer. Pour ce type d’objets, définir une frontière nécessite de
“synthétiser” les différentes évolutions temporelles, ce qui conduit
nécessairement à une frontière imprécise. Dans ce cas l’imprécision
spatiale est un artefact, né de la modélisation atemporelle d’un objet
qui ne l’est pas.

On peut également relever des aspects plus techniques, comme
l’imprécision liée aux instruments de mesure ou au producteur de
données (Follin et al., 2019) et plus généralement au processus de
production de données (Dutton, 1992 ; Evans et Waters, 2008 ; Follin
et al., 2019). D’autres points plus spécifiques peuvent également être
identifiés, comme les limites des modèles de représentation des
données (Dutton, 1992 ; Follin et al., 2019), il est par exemple
impossible de représenter tous les nombres réels informatiquement à
cause de la précision finie des nombres flottants utilisés pour les
figurer.

Enfin, il convient de noter que l’imprécision et l’incertitude peuvent
se transmettre lors de la définition de nouveaux objets à partir de
mesures (Dutton, 1992) ou d’objets géographiques imprécis (Liu et al.,
2019 ; Follin et al., 2019). On peut, dans ce cas, parler
d’imprécision et d’incertitude de second ordre. C’est ce phénomène que
décrivent Liu et al. (2019) lorsqu’ils définissent les
element-clustering objects1, des objets spatiaux imprécis construits
par l’agrégation d’autres objets (flous ou nets), et les
object-referenced objects2, qui sont construits par subdivision
d’objets spatiaux imprécis.

\subsection{La modélisation de l'imprécision spatiale}

La question de la modélisation de l’imprécision a conduit au
développement de plusieurs théories mathématiques, dont la plus connue
est la théorie des sous-ensembles flous (Zadeh, 1965). Nous avons
choisi de nous centrer sur la présentation de cette théorie, car nous
n’avons pas identifié dans la littérature des utilisations de théories
alternatives comme la théorie des fonctions de croyances. Quant aux
travaux basés sur la théorie des probabilités (Tøssebro et Nygård
2002, 2008), ceux-ci traitent de la modélisation de l’incertitude,
c’est pourquoi nous ne les incluons pas dans cet article.

\subsubsection{La théorie des fonctions de croyance}

\tdi{Doit aussi être abordé lors de la présentation de la modélisation
de l'incertitude}

\subsubsection{La théorie des ensembles approximatifs}

\tdi{Doit aussi être abordé lors de la présentation de la modélisation
  de l'incertitude}

\subsubsection{La théorie des sous-ensembles flous}

La théorie des sous-ensembles flous, ou, par abus de langage, théorie
des ensembles flous (Bouchon-Meunier, 2007), proposée par Zadeh en
1965 (Zadeh , 1965), vise à proposer un cadre théorique permettant de
modéliser des appartenances partielles à une classe
d’objets. Bouchon-Meunier (1995) présente les sous-ensembles flous
comme un « assouplissement » des ensembles « classiques », ici
qualifiés de « nets »1 (Smithson et Verkuilen, 2006). La possibilité
de modéliser des appartenances partielles permet à la théorie des
sous-ensembles flous de modéliser l’imprécision des connaissances, ce
qui en fait un candidat idéal pour la modélisation d’objets spatiaux
imprécis.

Un ensemble flou $A$ est défini comme un couple composé d’un ensemble
net $X$ et d’une fonction $f_A$ nommée fonction d’appartenance (Zadeh,
1965) :

\begin{equation}
A = (X, f_A)  
\end{equation}

La fonction $f_A$, associe à chaque élément de X une valeur comprise
dans l’intervalle [0,1], nommée degré d’appartenance. Cette valeur
peut être interprétée comme une mesure de l’appartenance d’un élément
à $X$. Par exemple, si l’on définit $X$ comme l’ensemble des personnes
de grande taille, le degré d’appartenance qualifie l’appartenance
d’une personne à cet ensemble. Une personne mesurant 2,10 m aura un
degré d’appartenance de 1, i.e. qu’elle est considérée comme grande. À
l’inverse, une personne mesurant 1,75 m aura un degré d’appartenance
compris entre 0 et 1 (0,6 par exemple), traduisant une appartenance
partielle à l’ensemble des personnes de grande taille, ce qui revient
à dire qu’il s’agit d’une personne grande, mais pas totalement.

De la même manière, le degré d’appartenance peut illustrer
l’appartenance d’une position à un objet spatial. Par exemple, la
Figure 6 illustre la représentation du lac du Chambon à l’aide d’un
sous-ensemble flou, les positions situées dans la zone d’imprécision
(Figure 3) se voient attribuer un degré d’appartenance inférieur à
1. Ce degré devient nul au-delà de la seconde frontière (Figure 2).

\begin{figure}
  \centering
  %\missingfigure
  \caption{Illustration de la modélisation du lac du Chambon à l’aide
    de la théorie des sous-ensembles flous.}
  \label{fig:champ_flou}
\end{figure}

L’ensemble des éléments ayant un degré d’appartenance non nul permet
de définir le support $S$ de l’ensemble :

\begin{equation}
  S(A) = \{x ∈ X \mid f_A(x) > 0\}
\end{equation}

Avec $X$ un ensemble net, $x$ un élément de l’ensemble $X$. Le noyau
$N$ de l’ensemble correspond, quant à lui, à l’ensemble des éléments
ayant un degré d’appartenance égal à 1 :

\begin{equation}
  N(A) = \{x ∈ X \mid f_A(x) = 1\}
\end{equation}

Si $A$ et $B$, deux sous-ensembles flous d’un même ensemble $X$, $A$
et $B$ sont égaux, si et seulement si, pour tout élément de $X$, le
degré d’appartenance aux sous-ensembles $A$ et $B$ est égal, soit :

\begin{equation}
A = B\ ssi\ ∀x ∈ X, f_A(x) = f_B(x)
\end{equation}

Un sous-ensemble flou $A$ de $X$ est inclus dans un sous-ensemble flou
$B$ de $X$ si et seulement si, pour tout élément $x$ appartenant à
$X$, le degré d’appartenance de $x$ à $A$ est inférieur ou égal à son
degré d’appartenance à $B$ :

\begin{equation}
A ⊆ B\ ssi\ ∀x ∈ X, f_A(x) ≤ f_B(x)
\end{equation}

Le complément $A^c$ d’un sous-ensemble flou $A$ de $X$ a pour fonction
d’appartenance :

\begin{equation}
∀x ∈ X, f_{A^C}(x) = 1 − f_A(x)
\end{equation}

Ainsi, pour reprendre l’exemple précédent, on peut construire le
complément de l’ensemble des personnes de grande taille, l’ensemble
des personnes qui ne sont pas de grande taille. Une personne de 2,10 m
aura un degré d’appartenance nul à ce second ensemble, et une personne
de 1,75 m de 0,4.

Zadeh (1965) définit l’intersection de deux sous-ensembles flous $A$
et $B$ de $X$ comme le sous-ensemble flou $C$ de $X$ dont la fonction
d’appartenance est la suivante :

\begin{equation}
  ∀x ∈ X, f_C (x) = \min(f_A(x), f_B(x))
\end{equation}

De manière analogue, l’union de deux sous-ensembles flous $A$ et $B$
de $X$ est un sous-ensemble $C ∈ F(X)$ dont la fonction d’appartenance
est :

\begin{equation}
    ∀x ∈ X, f_C (x) = \max(f_A(x), f_B(x))
  \end{equation}

  Par exemple, l’ensemble des personnes grandes et âgées peut être
  construit en intersectant l’ensemble des personnes âgées avec celui
  des personnes de grande taille. En utilisant les opérateurs proposés
  par Zadeh (1965), une personne dont le degré d’appartenance à ces
  deux ensembles est respectivement de 0,8 et 0,5, aura un degré
  d’appartenance à l’ensemble des personnes grandes et âgées de 0,5.

  On notera cependant que les opérateurs d’union et d’intersection
  proposés par Zadeh (1965) ne sont pas les seuls à avoir été
  envisagés et d’autres opérateurs, aux caractéristiques diverses, ont
  été proposés dans la littérature1. Cependant, la proposition de
  Zadeh, permettant de conserver « presque toute la structure de la
  théorie classique des ensembles » (Bouchon-Meunier, 2007), fait
  office de standard.

\tdi{Ajouter autres opérateurs}

% Lukasiewicz
% Probabiliste
% Drastique

  \subsection{Modélisations et implémentations}


  Comme l’indiquait Burrough (1996, p.15), « Les objets inexacts
  requièrent des modèles de données inexacts ». L’objectif de cette
  partie est de présenter les différents modèles proposés dans la
  littérature pour modéliser les objets spatiaux flous. Nous
  traiterons aussi bien de modèles uniquement théoriques que de leurs
  implémentations, voire de leurs applications.
  
  Pour présenter ces différents modèles, nous proposons une
  classification ad hoc, distinguant les modèles définis en extension
  de ceux définis en intension. Nous commencerons par expliciter cette
  classification, tout en la confrontant aux catégorisations présentes
  dans la littérature, avant de présenter les différents modèles.

  \subsubsection{Critères de classification}

Un grand nombre de modèles ont été proposés pour permettre la
manipulation d’objets spatiaux imprécis. Ils se distinguent par leur
théorie de rattachement (Partie 3), par la nature des objets spatiaux
modélisables (e.g. points, lignes, surfaces) ou leur implémentation
(e.g. raster, vecteur, ad hoc ou inexistante).

Chacun de ces points peut être utilisé comme critère de classement,
mais à notre connaissance, toutes les catégorisations proposées dans
la littérature se fondent sur le critère de la théorie de
rattachement. Clementini (2008) ou Erwig \& Schneider (1997), par
exemple, identifient trois catégories : les modèles probabilistes, les
modèles flous et les modèles exacts, qui étendent le modèle simple
features1 aux objets spatiaux imprécis. Certains auteurs (Schneider,
2001 ; Schneider, 2008 ; Carniel et al., 2016) y ajoutent la catégorie
des modèles approximatifs, basés sur la théorie des ensembles du même
nom, moins populaire que les précédentes. Enfin Fisher (Fisher, 2003 ;
Fisher et al., 2005 ; Fisher et al., 2006) propose une typologie des
théories combinant la modélisation de l’incertitude spatiale à celle
de l’imprécision spatiale et distinguant les modèles en fonction de
quatre théories de rattachement : probabilités, sous-ensembles flous,
fonctions de croyance et approbation. Bien qu’elle soit explicite et
permette une bonne appréhension de la différence de popularité entre
les différentes théories, cette catégorisation est critiquable, car
elle passe outre un critère qui nous semble fondamental : la nature du
processus de construction.

On peut, suivant la logique du paradigme des ensembles de points2
(Egenhofer et Herring, 1990), concevoir l’espace comme un ensemble
infini de points, représentant autant de positions. Les objets
spatiaux, nécessairement inclus dans cet espace, peuvent dès lors être
conceptualisés comme un ensemble de positions3. Par extension, un
objet spatial imprécis peut être conceptualisé comme un ensemble de
positions dont certaines ont une appartenance partielle à
l’ensemble. La nature du processus de construction décrit la méthode
utilisée pour décider de l’appartenance d’une de ces positions à
l’objet spatial et par extension pour construire un objet
spatial. Nous distinguons deux approches : la construction en
intension et celle en extension. Nous parlons de construction en
extension lorsque l’objet spatial traité est construit par la
sélection d’autres objets spatiaux (e.g. construction d’un département
par l’union des communes le composant). La construction en intension
désigne les cas où un objet spatial est construit par la délimitation
d’un espace (e.g. construction de buffers).

Cette distinction ne doit pas être confondue avec celle, faite en
mathématiques, entre la définition en intension et la définition en
extension, deux concepts qui qualifient la façon dont le contenu d’un
ensemble est exprimé. Il s’agit de deux notions orthogonales, un objet
spatial défini en intension pouvant être construit en extension et
inversement (Tableau 2). La définition en intension consiste à fixer
une ou plusieurs règles décrivant l’appartenance d’un élément à un
ensemble. Par exemple, l’ensemble des géographes anarchistes peut être
définit comme : tous les êtres humains, étudiant la géographie et
favorables à l’anarchisme. De la même manière, on peut définir un
objet géographique en intension. Par exemple, une zone économique
exclusive (ZEE) est définie comme la zone située à moins de 200 miles
marins des côtes d’un pays (Brunet et al., 1992). À l’inverse, la
définition d’un ensemble en extension consiste à lister les différents
éléments y appartenant. Pour reprendre l’exemple précédent, la
définition en extension de l’ensemble géographes anarchistes serait :
Élisée Reclus ; Pierre Kropotkine ; Léon Metchnikoff ; Simon
Springer. Pour un objet spatial, sa construction en extension se
résume à lister les positions ou les objets géographiques appartenant
à l’ensemble, par exemple l’ensemble des régions ultramarines
françaises est : La Guadeloupe ; La Guyane ; La Martinique ; La
Réunion ; Mayotte.

La distinction que nous faisons entre construction en extension et
construction en intension peut sembler équivalente à celle qui a été
faite entre les implémentations raster et vecteur, ou plus
généralement, entre les modèles champs et objets tels que définis par
Couclelis (1992) et Goodchild (1992). En effet, l’utilisation d’un
modèle de type champ nécessite de renseigner, pour chaque élément, par
exemple des pixels dans le cas d’une implémentation raster, son
appartenance à l’objet spatial, ce qui équivaut à une construction en
extension. Inversement, la construction d’un objet vectoriel est
assimilable à une construction en intension. Cependant, le champ des
possibles ne se limite pas à ces deux cas, laissant supposer une
équivalence entre les deux catégorisations. Par exemple, la définition
d’un objet spatial à partir de la sélection d’objets préexistants
(e.g. on souhaite sélectionner les hôtels proches d’une station de
métro donnée), entre dans le cadre du modèle objet tel que défini par
Couclelis (1992), mais impose une construction en extension, i.e. la
sélection d’objets spatiaux à partir d’un ensemble.

\begin{table}
  \centering
  \begin{tabular}{R{3cm}L{5.5cm}L{5.5cm}}
  \toprule
  & {\bfseries Définition en intension} & {\bfseries Définition en extension} \\
  \midrule
   {\bfseries Construction en intension} & Construction d’objets vectoriels à partir de règles (\eg buffer, isolignes) & Construction d’un raster à partir de règles (\eg buffer, seuil de valeur)
  \\
  {\bfseries Construction en extension} & Construction d’objets vectoriels à partir d’une liste d’individus & Construction d’un raster à partir d’une liste d’individus
  \\
  \bottomrule
\end{tabular}
  \caption{Comparaison entre la nature de construction et la nature de définition}
  \label{tab:ext_vs_int}
\end{table}

\subsubsection{Modèles basé sur une construction en intension}

Parmi les modèles construits en intension, on retrouve des
propositions basées sur la théorie des sous-ensemble flous, présentée
précédemment. Cependant, il existe une autre catégorie de modèles,
fréquemment décrits comme « exacts » dans la littérature (Schneider,
2003 ; Bejaoui et al., 2009).

\paragraph{Les modèles \enquote{exacts}}


Parmi les modèles construits en intension, on retrouve des
propositions basées sur la théorie des sous-ensemble flous, présentée
précédemment. Cependant, il existe une autre catégorie de modèles,
fréquemment décrits comme « exacts » dans la littérature (Schneider,
2003 ; Bejaoui et al., 2009).

En 1996, Cohn et Gotts (Cohn et Gotts, 1996) ont proposé le modèle
egg-yolk, qui demeure aujourd’hui la plus connue des solutions de
modélisation de l’imprécision spatiale. Les auteurs proposent de
modéliser des étendues imprécises à l’aide de régions délimitées par
deux frontières. Par analogie avec un œuf au plat, la zone délimitée
par la seconde frontière est baptisée « white » et celle délimitée par
la première frontière, incluse dans « le blanc de l’œuf », est
baptisée « yolk ». Pour poursuivre avec cette analogie, la partie du
« blanc » non incluse dans le « jaune » correspond alors à la partie
imprécise, i.e. dont l’appartenance à la région est contestable,
contrairement à la zone appartenant au « jaune » Dans le cas où ces
deux zones sont confondues, le modèle egg-yolk est équivalent au
modèle simple features et aucune imprécision n’est modélisée.

Parallèlement à ces travaux, Clementini et Di Felice (1996) ont
proposé une modélisation des surfaces imprécises, qui sera par la
suite étendue pour permettre la modélisation de tout type d’objet
spatial imprécis. De la même façon que précédemment, une région vague
possède deux frontières, dont la sémantique est identique à celle du
modèle egg-yolk. Cependant, les propositions de Cohns et Gotts (1996)
et de Clementini et Di Felice (1996) se distinguent par leur
modélisation des relations topologiques (Cohn et Gotts, 1996). Le
modèle egg-yolk est basé sur la théorie RCC1 proposée par Randell et
Cohn (1989) alors que le modèle proposé par Clementini et Di Felice
(1996) est basé sur la topologie des ensembles de points (Egenhofer et
Herring, 1990) tout comme le modèle, similaire, qui sera proposé par
Erwig et Schneider en 1997 (Erwig, 1997). En découle une modélisation
différente des relations topologiques entre deux régions vagues. Pour
Clementini et Di Felice (1996) le modèle egg-yolk se distingue par son
approche topologique du problème, là où eux ont privilégié une
approche géométrique. Clementini proposera ultérieurement une
extension de ce modèle en vue d’y intégrer la modélisation des points
et des polylignes (Clementini, 2005 ; Clementini, 2008), contrairement
au modèle d’Erwig et Schneider (1997), limité aux régions imprécises.

Schneider a proposé une modélisation exacte des objets imprécis1 en
1996. Contrairement aux modèles présentés précédemment, ce dernier
offre, dès sa première itération (Schneider, 1996), la possibilité de
modéliser des points, des lignes et des régions imprécises qui peuvent
être à trous ou composées de plusieurs noyaux. Cette proposition est
basée sur le modèle Realm/Rose de Güting et Schneider (1995), qui
propose de définir des objets spatiaux à partir d’une grille régulière
de points. Chaque objet, quel que soit son type, est construit à
partir d’un ou de plusieurs de ces points, ce qui s’apparente à une
version discrète du paradigme des ensembles de points.

Les modèles précédents ont pour point commun de ne pas permettre la
modélisation des objets partiellement imprécis1, tel que l’on pourrait
conceptualiser le lac du Chambon, dont la limite est précise par
endroits, notamment le long du barrage (Figure 3). Bejaoui et
al. (Bejaoui 2009 ; Bejaoui et al, 2009) proposent donc d’étendre le
modèle egg-yolk à ce type d’objet tout en permettant la modélisation
de points et de lignes imprécises. Pour ce faire, les auteurs
proposent de re-formaliser le modèle à l’aide du paradigme des
ensembles de points, abandonnant le modèle RCC initialement utilisé
par Cohn et Gotts (1996). Mais la principale différence avec les
propositions précédentes n’est pas due aux types d’objets modélisables
ou à la théorie de rattachement, mais au raffinement de la sémantique
des objets imprécis. Ainsi, les lignes imprécises, constituant un seul
type d’objet dans le modèle de Clementini (2005) sont ici décomposées
en neufs classes, en fonction de la nature (imprécise, partiellement
imprécise ou précise) de leur intérieur et de leur frontière. Pour les
régions imprécises, trois catégories sont proposées : les régions
précises, les régions partiellement imprécises, dont certaines parties
sont précises et d’autres non et les régions imprécises. Ce modèle
offre donc une finesse dans la modélisation d’objets spatiaux imprécis
qui était jusqu’ici inaccessible aux modèles exacts, mais au prix
d’une importante complexification du modèle.

\begin{figure}
  \centering
  %\missingfigure
  \caption{Illustration de la modélisation du lac du Chambon avec un
    modèle exact tel que proposé par Bejaoui et al. (2009). Comme pour
    tous les modèles “exacts” la zone d’appartenance totale est
    incluse dans la zone d’appartenance au moins partielle.}
  \label{fig:champ_exact}
\end{figure}

\paragraph{Modèles et implémentation flous}

Définir un objet spatial imprécis à l’aide de la théorie des
sous-ensembles flous nécessite au préalable d’en élaborer un modèle de
représentation. C’est un travail qui, à notre connaissance, n’a été
entrepris que par Schneider (1999 ; 2001) qui a formalisé une version
floue du modèle simple features.

Le modèle de Schneider (1999) définit trois types de géométries
floues : le fpoint, la fline et la fregion, équivalents flous des
types point, line et region du modèle simple features. Comme pour ce
modèle, la proposition de Schneider se base sur le paradigme des
ensembles de points. Les fpoints sont définis comme une union de
points flous1, i.e. un ensemble de positions disjointes, deux à deux,
auxquelles est associée une valeur, le degré d’appartenance de la
position au fpoint. Ainsi, contrairement à un point, un fpoint peut
occuper plus d’une position (Figure 8). Les lignes floues (à ne pas
confondre avec les flines), sont également définies comme un ensemble
de points flous. Cependant, contrairement au fpoint, la construction
de cet ensemble est contrainte et les points flous appartenant à une
ligne floue doivent se trouver sur une même coube continue. De plus,
chaque point flou doit avoir un degré d’appartenance unique,
strictement croissant (ou décroissant) dans le sens de la coube. Les
deux points flous situés aux extrémités d’une ligne floue ont donc,
respectivement, le degré d’appartenance maximal et minimal à la ligne
floue. Les blocs flous sont définis comme l’union d’un nombre fini de
lignes floues, ne pouvant êtres connectées que par leurs extrémités2
et les flines comme une union de blocs flous disjoints. Cette
construction permet aux flines de représenter des objets linéaires
complexes et composés de plusieurs géométries distinctes qui ne
seraient pas représentables par le type line du modèle simple features
(Figure 8). Enfin, les fregions sont définies comme étant un ensemble
de points flous ne comportant pas d’anomalies géométriques
(e.g. superpositions, croisements). Comme pour les fpoints et les
flines, les fregions peuvent modéliser des géométries plus complexes
que le type region du modèle simple features, une fregion pouvant être
composée de plusieurs polygones disjoints (Figure 8). Schneider (2004)
complétera ce modèle en définissant un ensemble de prédicats
permettant de modéliser des relations topologiques entre géométries
floues.

\begin{figure}
  \centering
  %\missingfigure
  \caption{Les trois types géométriques flous proposés par Schneider
    (1999). D’après Schneider (1999).}
  \label{fig:mod_schneider}
\end{figure}

Plusieurs implémentations de ce modèle ont été proposées, notamment
par Kanjilal et al. (2010) ou par Dilo et al. (Dilo, 2006 ; Dilo et
al., 2007). Mais, d’autres travaux ont également proposé des
modélisations floues d’objets spatiaux imprécis, sans rattachement ou
définition explicite d’un modèle théorique. C’est notamment le cas de
Zoghalmi ou de Runz qui proposent une approche fondée sur les
alpha-cuts (de Runz, 2008 ; de Runz et al., 2008 ; Zoghlami, 2013 ;
Zoghlami et al., 2016) ou de Carniel et al. (2016).

Kanjinal et al. (2010), proposent d’implémenter le type fregion défini
par Schneider (1999) à l’aide de « région[s] plateau », définies comme
un ensemble non nul et fini de polygones1 (ou multi-polygones) nets,
représentant une plage de valeurs de degré d’appartenance. Cette
implémentation est semblable à l’approche par alpha-cuts proposée par
de Runz (2008) ou Zoghalmi (2013 ; 2016). Dans la théorie des
sous-ensembles flous, une alpha-cut est définie comme l’ensemble des
éléments d’un sous-ensemble flou ayant un degré d’appartenance
supérieur à un seuil fixé2 (Bouchon-Meunier, 2007). Si les éléments du
sous-ensemble flou sont des positions, une alpha-cut permet de définir
une aire, modélisable par un polygone. Ainsi, les implémentations
proposées par Kanjinal et al. (2010), Zoghalmi (2013 ; 2016) ou de
Runz (2008) ont pour point commun de représenter un sous-ensemble flou
en le discrétisant à l’aide de polygones. Mais ces approches ont une
différence majeure: les contraintes de construction des
polygones. L’approche de Kanjinal et al. (2010) impose aux polygones
constitutif d’une région plateau d’être disjoints ou adjacents deux à
deux, une même position ne peut donc appartenir qu’à un seul polygone,
ce qui n’est pas pour les implémentations proposées par Zoghalmi
(2013 ; 2016) ou de Runz (2008), où une position appartient, par
définition, à toutes les alpha-cuts construites à partir d’un seuil
inférieur à son degré d’appartenance. Cette distinction n’est pas
visible sur la figure 9, les deux approches donnant, pour le même
nombre de polygones, des résultats visuellement
équivalents. Cependant, les trois polygones représentés sur la figure
9 ont une sémantique différente selon l’implémentation choisie. Dans
l’implémentation de Kanjinal et al. (2010), les trois polygones sont
disjoints. Le premier délimite le noyau, le second l’aire ayant un
degré d’appartenance compris entre 1 et 0,25 (exclus) et le dernier
polygone délimitant l’aire dont le degré d’appartenance est
strictement supérieur à 0 et inférieur à 0,25. Avec l’implémentation
par alpha-cuts, les polygones sont superposés, et si le premier
polygone délimite toujours le noyau, le second délimite l’aire dont le
degré d’appartenance est supérieur à 0,25 et le troisième l’aire dont
le degré d’appartenace est non nul3. Par conséquent, si l’on ne
construit que les alpha-cuts du noyau et du support, le résultat de
l’implémentation proposée par Zoghalmi (2013 ; 2016) ou de Runz (2008)
est identique à celui des modèles de Cohn et Gotts (1996) ou de
Clementini et di Felice (1996) présentés précédemment et qui
superposent également la zone d’appartenance à la zone d’appartenance
partielle; ce qui n’est pas le cas pour l’implémentation de Kanjinal
et al. (2010), car les polygones définissant une région plateau y
sont, par définition, disjoints.

\begin{figure}
  \centering
  %\missingfigure
  \caption{Illustration de la modélisation du lac du Chambon avec un modèle flou discrétisé par un ensemble de polygones.}
  \label{fig:champ_polygones}
\end{figure}

Comme pour les implémentations précédentes, l’approche proposée par
Dilo et al. (2007) ne s’applique qu’au cas des régions floues. Les
auteurs proposent d’implémenter ces dernières à l’aide de deux
linéaires, représentant les frontières du support et du noyau et d’un
maillage, servant de support à une interpolation. Les modèles
précédemment présentés imposent un échantillonnage des degrés
d’appartenance, dont la précision varie en fonction du nombre de
régions nettes (Kanjilal et al., 2010) ou d’alpha-cuts utilisées
(Zoghlami et al., 2016). L’implémentation proposée par Dilo et
al. vise à contourner ce problème en offrant la possibilité de
calculer le degré d’appartenance en tout point de la région floue par
interpolation (de manière similaire a la proposition de Tøssebro et
Nygård, 2002). Ils proposent pour cela de définir un maillage à l’aide
d’une triangulation de Delaunay contrainte aux frontières du noyau et
du support, cette dernière devant être post-traitée pour supprimer les
triangles créés dans les trous ou concavités. Le degré d’appartenance
au sous-ensemble flou des points appartenant aux frontières étant
connu, il est ainsi possible de calculer le degré d’appartenance en
tout point à l’aide d’une interpolation triangulaire (Figure 10). La
précision de l’estimation peut être améliorée par l’ajout de points
intermédiaires dont le degré d’appartenance à l’ensemble est connu, ce
qui peut être le cas lorsque, comme proposé par Dilo et al. (2007), la
région floue est définie à partir d’un ensemble de points (les
frontières sont alors définies à l’aide d’enveloppes concaves). Le
principal problème de cette approche est qu’elle conduit à une
importante complexification des opérations inter-ensembles. Là où les
précédents modèles recouraient uniquement à des opérations
géométriques quelconques et à une sélection du plus grand (ou plus
petit, en fonction de l’opération concernée) degré d’appartenance, ce
modèle impose la reconstruction du maillage et son post-traitement, ce
qui peut rendre les opérations inter-ensembles (unions, intersections)
coûteuses.

Schneider (2003) a également proposé une implémentation de son propre
modèle. Contrairement aux implémentations précédemment citées,
celle-ci aborde tous les types formalisés dans le modèle théorique
(fpoint, fline et fregion). Ce travail est assez proche de ce qu’avait
proposé le même auteur avec sa proposition de modèle exact basé sur
l’approche Realm/Rose (Schneider, 1996) (Partie 4.2.1), puisque
l’auteur propose d’implémenter les types spatiaux flous à l’aide d’un
ensemble fini de points répartis régulièrement. L’implémentation du
type fpoint ne présente pas de particularités, il s’agit d’un point
appartenant à l’ensemble des positions possibles et ayant un degré
d’appartenance. De la même manière, l’implémentation du type fline est
proche de la formalisation, la principale différence étant que les
points composant cette dernière doivent nécessairement appartenir à
l’ensemble des positions possibles. Pour le type fregion ce dernier
est composé d’un ensemble de cellules auxquelles est attribué un degré
d’appartenance à la région floue. Les règles présentées dans le modèle
théorique s’appliquent toujours, ainsi les régions avec des anomalies
géométriques ne peuvent pas être modélisées. On peut noter que cette
implémentation des fregions est assez semblable à celle qui sera
proposée plus tardivement par Kanjinal et al. (2010), avec une région
floue modélisée comme un ensemble de polygones nets.

\begin{figure}
  \centering
  %\missingfigure
  \caption{Illustration de la modélisation du lac du Chambon avec un modèle flou tel qu’implémenté par Dilo et al. (2007)}
  \label{fig:champ_dilo}
\end{figure}

Pour finir, on peut également citer l’implémentation proposée par
Carniel et al. (2016). Ces derniers ne proposent qu’une implémentation
des lignes et des points flous, délaissant le cas des régions
floues. Cinq types sont définis et implémentés dans PostGIS par les
auteurs. Le premier d’entre eux est le type générique FuzzyGeometry,
qui se spécialise en FuzzyPoint, FuzzyLine et leurs équivalents
complexes, FuzzyMultiPoint et FuzzyMultiLine. On notera que cette
organisation reprend celle du modèle simple feature. Cette
ressemblance ne se limite pas à cet élément, puisque les types
complexes sont également définis comme des ensembles de types
simples. Les différences avec le modèle simple feature apparaissent
lors de la définition du type FuzzyPoint. Ce dernier est défini comme
un triplet composé de deux coordonnées x, y et d’un degré
d’appartenance. Par extension, le type FuzzyLine est défini comme un
ensemble de FuzzyPoint, ainsi le degré d’appartenance d’une FuzzyLine
est défini par les points qui la composent.

\subsubsection{Modèles basés sur une construction en extension}

Notre présentation des modèles basés sur une construction en extension
est organisée selon une classification ad hoc, basée sur la nature des
objets sélectionnés (Partie 4.3.1) et distinguant les modèles basés
sur une construction en extension du premier ordre (Partie 1.1.1) de
ceux basés sur une construction en extension d’ordre supérieur (Partie
1.1.13).

\paragraph{Critères de classement}

Comme nous l’expliquions précédemment, toute définition d’un objet
spatial par la sélection d’un ou plusieurs objets géographiques est
une construction en extension. On peut distinguer ces modèles selon la
nature des objets spatiaux sélectionnés.

Une première possibilité consiste à définir un objet spatial par la
sélection des positions qu’il occupe. La délimitation d’une zone à
partir d’un ensemble de pixels est un exemple concret de ce type
d’approche, notamment proposée par Zhan (1997). Cet exemple ne doit
cependant pas laisser croire que la catégorisation proposée ici est
fondée sur la nature de l’implémentation, mais bien sur la sélection
de positions dont l’infinité ne peut qu’être approximée par des
modèles champs, quelle que soit la nature de leur
implémentation. Toutefois, l’ensemble des travaux que nous
présenterons s’appuie sur une approche raster.

À l’inverse, on peut imaginer construire des objets spatiaux à partir
d’une sélection d’objets, d’ « agrégats » (Charrre, 1995) de
positions. Cette approche définit des objets spatiaux de manière
semblable aux Elements-Clustering Objects de Liu et al. (2019) (Partie
2.2). Dans ce cas, le sous-ensemble spatialisé peut être défini à
partir d’un groupe d’éléments ne couvrant pas l’intégralité de
l’espace (e.g. sélection à partir d’un réseau routier). Nous
qualifions ce second type de sélection d’ordre supérieur, car elle
s’opère sur des objets complexes, eux-mêmes composés de positions, et
non directement de positions comme c’est le cas pour une sélection
directe de positions, que nous qualifions de sélection de premier
ordre.

\paragraph{Construction en extension du premier ordre}

On peut présenter la définition en extension du premier ordre d’un
sous-ensemble flou spatialisé à l’aide de l’exercice de définition
d’une limite pour le lac du Chambon. La Figure 11 se distingue des
exemples précédents par son découpage en pixels (dont la taille a
volontairement été exagéré). Chacun se voit attribuer un degré
d’appartenance, défini indépendamment. La tâche principale consiste
donc à définir une méthode permettant de calculer le degré
d’appartenance pour chaque pixel. Pour cet exemple, nous avons adopté
une approche simple : le degré d’appartenance décroit avec la distance
au rivage. Le degré d’appartenance est donc de 1 pour les pixels
situés à l’intérieur du lac, puis il décroit en fonction de la
distance du centroïde du pixel au point du rivage le plus proche. Ces
règles peuvent être raffinées autant que nécessaire.

Les travaux de Vanegass et al. (2011) et Takemura et al. (2012), par
exemple, s’intéressent à la modélisation de relations spatiales,
respectivement « entouré de » et « le long de ». L’objectif est de
délimiter la zone validant la relation spatiale considérée,
c’est-à-dire de construire un « paysage flou » selon la terminologie
proposée par Bloch (1996). Ce type de modélisation est nécessairement
impacté par l’imprécision du langage naturel. Une description de
position est nécessairement imprécise, une même relation spatiale
pouvant prendre un sens différent en fonction de l’objet de référence,
du locuteur, etc. (Vandeloise, 1986 ; Borillo, 1998 ; Bateman, 2010),
écueil justifiant le recours à la théorie des sous-ensembles
flous. Ces deux travaux sont fondés sur une méthodologie
similaire. Ils définissent tous deux un ensemble de métriques et de
fonctions permettant de calculer, pour chaque pixel d’un raster, un
degré d’appartenance quantifiant la validité de la relation spatiale
modélisée pour la position considérée.

\begin{figure}
  \centering
  %\missingfigure
  \caption{Illustration de la modélisation du lac du Chambon par un modèle flou et une implémentation raster (i.e. construction en extension de premier ordre).)}
  \label{fig:champ_raster}
\end{figure}

Si ces travaux proposent avant tout une réflexion théorique, d’autres
auteurs utilisent cette même approche pour répondre à des
problématiques plus appliquées. C’est notamment le cas d’Arabacioglu
(2010) qui propose une application de la théorie des sous-ensembles
flous à l’architecture, ou Kurtener et Badenko (2000), Makropoulos et
al. (2003) et Girot (2007) qui se sont penchés sur des questions de
prise de décision appliquées à la gestion du territoire. Comme
précédemment, ces travaux utilisent un raster et calculent pour chaque
pixel un degré d’appartenance à une zone, dont la sémantique varie en
fonction de la problématique. Cependant, le calcul du degré
d’appartenance dans les travaux de Griot (2007), Makropoulos et
al. (2003) et Arabacioglu (2010) est réalisé à l’aide d’un système
d’inférence flou et non directement comme dans les travaux de Vanegass
et al. (2011) et Takemura et al. (2012). Les systèmes d’inférence
flous, pendant flou des systèmes experts,  permettent également de
calculer un degré d’appartenance pour chaque individu (quelle que soit
sa nature), mais à partir de règles logiques formulées en langage
naturel.  Leur utilisation permet donc de prendre en compte
l’imprécision, ici liée à l’expression orale (Makropoulos et al.,
2003 ; Griot, 2007), ou à la quantification de ressentis (Arabacioglu,
2010), mais également de formaliser des connaissances expertes.

La définition en extension du premier ordre d’un sous-ensemble
spatialisé est courante dans la littérature et les publications
présentées ici, l’appliquent à des thématiques diverses, comme
l’interprétation d’images aériennes (Brandtberg, 2002 ; Fonte et
Lodwick, 2005), la gestion du territoire (Griot, 2007 ; Makropoulos et
al., 2003), la qualification de ressentis (Arabacioglu, 2010) ou la
modélisation de relations spatiales floues (Takemura et al., 2012 ;
Vanegas et al., 2011).

\paragraph{Construction en extension d'ordre supérieur}

La construction en extension d’ordre supérieur d’un sous-ensemble flou
spatialisé est probablement l’approche qui nécessite le plus de
s’appuyer sur un exemple concret (Figure 12). Comme nous l’expliquions
précédemment, cette approche nécessite d’attribuer des degrés
d’appartenance à des objets, des agrégats de position. Elle se
distingue donc de la construction en extension du premier ordre, qui
nécessite d’attribuer un degré d’appartenance à des positions (cas
théorique) ou à un pavage régulier, une discrétisation de l’ensemble
des positions (cas pratique, cf. Figure 11). La Figure 12 représente à
la fois l’occupation du sol de la zone traitée et le degré
d’appartenance1 de l’objet à la zone modélisée. Ainsi, on considère
que le polygone du lac appartient totalement au sous-ensemble flou
“lac”, que les berges y appartiennent moyennement et qu’une partie des
forêts y appartient faiblement. Quant aux autres éléments, on
considère qu’ils n’appartiennent pas à la zone modélisée.

\begin{figure}
  \centering
  %\missingfigure
  \caption{Illustration de la définition du lac du Chambon par
    sélection floue d’objets géographiques (i.e. construction en
    extension d’ordre supérieur).}
  \label{fig:champ_raster_sel}
\end{figure}

Ce processus de construction, pendant spatial de la réflexion sur les
requêtes floues (Wang, 1994 ; Moreau et al., 2018), est utilisé par
Duračiová et Chalachanová (2017) qui illustrent leur méthodologie par
la sélection floue « [des] grands parkings proches du stade et ayant
été rénovés il y a environ deux ans ». Leur démarche est fortement
similaire à celle de l’exemple précédent (Figure 12). Les autrices
définissent des règles permettant de quantifier l’appartenance des
objets spatiaux candidats à l’ensemble flou correspondant à la
description. Là où nous n’utilisions qu’un seul critère, la proximité,
ce nouvel exemple nécessite de prendre en compte la proximité mais
aussi la date de la dernière rénovation. Chacun de ces critères permet
de construire un sous-ensemble flou spatialisé (celui des parkings
proches du stade et celui des parking rénovés il y a environ deux ans)
et l’intersection de ces deux ensembles, réalisée à l’aide des
opérateurs flous présentés précédemment (Partie 3) permet de
construire le sous-ensemble flou désiré. Un processus similaire est
utilisé par Bard et al. (2003) dans le cadre de l’évaluation de la
généralisation cartographique. Les objets spatiaux sont ici classés
dans des sous-ensembles flous figurant une description qualitative de
la qualité de la généralisation. Cross et Firat (2000) utilisent quant
à eux une approche similaire pour ajouter la prise en compte de
l’imprécision aux modèles objets des systèmes d’information
géographiques. Dans tous les cas, le sous-ensemble flou construit est
spatialisé, mais cette spatialisation est exogène, puisque dépendante
de la géométrie des objets spatiaux sélectionnés.

% Conclusion de cybergéo

Dans cet état de l’art, nous avons souhaité présenter les différents
concepts permettant de décrire les objets géographiques dont la
délimitation précise est impossible. Les différentes théories et
implémentations recensées offrent de nombreuses possibilités pour
modéliser l’imprécision spatiale. Nous nous sommes particulièrement
penché sur les différentes implémentations que nous avons catégorisées
selon leur méthode de construction. Ainsi, nous distinguons les
implémentations basées sur une construction en intension et les
implémentations basées sur une construction en extension.

La modélisation de l’imprécision spatiale est donc un champ de
recherche riche, qui offre au géographe de nombreux outils théoriques
permettant de travailler efficacement avec les nombreux objets
spatiaux aux frontières imprécises auxquels nous sommes régulièrement
confrontés. Bien que la considération de l’imprécision des objets
spatiaux soit assez ancienne, la formalisation de modèles et encore
plus leur implémentation, est un champ de recherche encore actif comme
le montre, par exemple, la récente réflexion autour de l’expression de
l’imprécision spatiale au sein du format GML (Wei et al., 2017).

Toutefois, l’application de ces différents modèles à une problématique
concrète impose de sélectionner une implémentation avec laquelle
travailler. Dans le cadre de notre travail de recherche, nous avons
été amené à confronter certaines de ces implémentations en vue de
spatialiser une position décrite en langage naturel, et donc,
foncièrement imprécise. Nous avons adopté une démarche basée sur un
principe de décomposition et de spatialisation indépendante des
différentes relations spatiales utilisées pour exprimer une position
(Auteur, 2019). Les contraintes liées à cette démarche, comme la
nécessité de combiner les résultats de différentes spatialisations,
nous ont amené à adopter une approche similaire à celle de Vanegass et
al. (2011) et Takemura et al. (2012), i.e. une modélisation basée sur
la théorie des sous-ensembles flous et implémentée à l’aide de
rasters. Cette solution nous parait être le meilleur compromis entre
minimisation des contraintes techniques et finesse de modélisation
(Auteur, 2019).
 
%%% Local Variables:
%%% mode: latex
%%% TeX-master: "../../../../main"
%%% End:
