\chaptertoc{}

\addsec{Introduction}

Introduction chapitre 10

\section{Alerte 1}
\label{sec:10-1}

\subsection{Présentation de l'alerte}
\label{subsec:10-1-1}

\subsection{Modélisation de l'alerte}
\label{subsec:10-1-2}

\subsubsection{Retranscription et identification des indices}
\label{subsec:10-1-2-1}

\subsubsection{Décomposition des indices}
\label{subsec:10-1-2-2}

\subsubsection{Spatialisation}
\label{subsec:10-1-2-3}

\subsubsection{Agrégation}
\label{subsec:10-1-2-4}


\subsection{Critique de la modélisation}
\label{subsec:10-1-3}

\section{Alerte 2}
\label{sec:10-2}

\subsection{Présentation de l'alerte}
\label{subsec:10-2-1}

\subsection{Modélisation de l'alerte}
\label{subsec:10-2-2}

\begin{verbatim}
Reprendre la structure du chapitre implémentation -> déroulé du code
\end{verbatim}

\subsubsection{Retranscription et identification des indices}
\label{subsec:10-2-2-1}

\subsubsection{Décomposition des indices}
\label{subsec:10-2-2-2}

\subsubsection{Spatialisation}
\label{subsec:10-2-2-3}

\subsubsection{Agrégation}
\label{subsec:10-2-2-4}

\subsection{Critique de la modélisation}
\label{subsec:10-2-3}

\section{Modélisation du \emph{fil rouge}}
\label{sec:10-3}

\subsection{Présentation de l'alerte}
\label{subsec:10-3-1}

\subsection{Modélisation de l'alerte}
\label{subsec:10-3-2}

\subsubsection{Décomposition des indices}
\label{subsec:10-3-2-1}

\subsubsection{Spatialisation}
\label{subsec:10-3-2-2}

\subsubsection{Agrégation}
\label{subsec:10-3-2-3}

Reprendre la structure du chapitre implémentation -> déroulé du code

\subsection{Critique de la modélisation}
\label{subsec:10-3-3}

\section{Synthèse  critique ( conclusion ?) }
\label{sec:10-4}

\addsec{Conclusion}

Conclusion chapitre 10

%%% Local Variables:
%%% mode: latex
%%% TeX-master: "../../../main"
%%% End:
