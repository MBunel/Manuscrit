
\subsection{Principe de modélisation autonome}

Un autre principe que nous souhaitons développer est celui de la
séparation maximale des étapes du processus.

Comme nous l'expliquions dans le \autoref{chap:2} deux étapes sont
nécessaires pour transformer une position décrite en une \emph{zone de
  localisation probable,} la \emph{spatialisation} et la
\emph{fusion.} Mais ces deux étapes peuvent être combinées
différemment.

Une première solution consiste à \enquote{chainer} la
\emph{spatialisation} des différents \emph{indices de localisation,}
comme on peut le voir sur la figure \ref{fig:comp_approches_lin}. Avec
cette approche les différentes \emph{zones de localisation
  compatibles} sont créées les unes à la suite des autres, ainsi la
zone de recherche en entrée de chaque étape de \emph{spatialisation}
dépend des \emph{spatialisations} précédentes. Avec cette approche,
l'opération de \emph{fusion} est implicite, elle s'opère, en partie, à
chaque nouvelle \emph{spatialisation,} puisque les zones ne
correspondant pas aux \emph{indices de localisation} sont retirées peu
à peu. La \emph{zone de localisation probable,} en bleu sur la figure
\ref{fig:comp_approches}, est obtenue une fois que l'on a
\emph{spatialisé} tous les \emph{indices de localisation.} Cette
approche est la première que nous ayons envisagée. Elle a pour
avantage d'être simple à concevoir et à développer. Cependant cette
approche à un problème majeur. À cause de l'enchainement des étapes de
\emph{spatialisation} les erreurs se répercutent d'une étape à
l'autre. Ainsi, si un des \emph{indices de localisation} est mal
\emph{spatialisé}, que ce soit à cause de la méthode utilisée ou d'une
mauvaise description, cela se répercute sur toutes les
\emph{spatialisations} ultérieures. La moindre correction impose donc
de refaire l'ensemble des traitements.

Une seconde solution, plus robuste, consiste à effectuer l'ensemble
des opérations de \emph{spatialisation} en parallèle, comme le montre
la figure \ref{fig:comp_approches_sep}. Avec cette approche la
\emph{spatialisation} des différents \emph{indices de localisation}
est effectuée séparément et leur \emph{fusion} est effectuée dans un
second temps. Comme pour la démarche \enquote{chainée} une
\emph{spatialisation} erronée peut impacter la \emph{zone de
  localisation probable,} toutefois il est plus facile de corriger ces
erreurs. En effet, si les différentes \emph{zones de localisation
  compatibles} ont été conservées il est possible de faire une
nouvelle \emph{fusion} en retirant un, ou plusieurs \emph{indices de
  localisation.} Alors qu'avec une approche \enquote{chainée} il est
nécessaire de refaire toutes les opérations de \emph{spatialisations}
ultérieures à celle de \emph{l'indice de localisation} que l'on
souhaite modifier ou retirer et ce même si les résultats
intermédiaires ont été conservés.

\begin{figure}
  \centering
  \subfloat[Construction suivant une démarche linéaire]{
    \begin{tikzpicture}

  \begin{scope}
    \path[ffa] (0,0) rectangle (2,2);
    \path[ffc] (0,0) rectangle (2,2) ;
    \node[text width=2cm](1,1) {\footnotesize \itshape Zone initiale de recherche};
  \end{scope}
  
  \path[draw, ->] (2.5,1) --++ (3,0)  node[pos=.5, above] {\footnotesize \itshape spatialisation};

  \begin{scope}[xshift=6cm]
    \path[draw,dashed] (0,0) rectangle (2,2);
    \begin{scope}
      \begin{scope}
        \clip (0,0) rectangle (2,2);
        \fill[ffa] (1.25,.7) circle [radius=15pt];
        \path[ffc] (1.25,.7) circle [radius=15pt];
      \end{scope}
    \end{scope}
    \node[circle, inner sep=0pt,minimum size=4pt, fill] (c) at (1.25,.7)
    {};
  \end{scope}

  \path[draw, ->] (8.5,1) --++ (3,0)  node[pos=.5, above] {\footnotesize \itshape spatialisation};




  \begin{scope}[xshift=12cm]
    \path[draw,dashed] (0,0) rectangle (2,2);
    \path[draw,dashed] (1.25,.7) circle [radius=15pt];
    \path[draw,dashed](.8,1) circle [radius=20pt];
    \begin{scope}
      \begin{scope}
        \clip (1.25,.7) circle [radius=15pt];
        \fill[ffa2] (.8,1) circle [radius=20pt];
        \path[ffc2] (.8,1) circle [radius=20pt];
      \end{scope}
      \begin{scope}
        \clip (.8,1) circle [radius=20pt];
        \path[ffc2] (1.25,.7) circle [radius=15pt];
      \end{scope}
    \end{scope}
    \node[circle, inner sep=0pt,minimum size=4pt, fill] (c) at (1.25,.7)
    {};
    \node[circle, inner sep=0pt,minimum size=4pt, fill] (c) at (.8,1)
    {};
  \end{scope}


\end{tikzpicture}
    \label{fig:comp_approches_lin}
  }

  \subfloat[Construction suivant une démarche autonome]{
    \begin{tikzpicture}

  \begin{scope}
    \path[ffa] (0,0) rectangle (2,2);
    \path[ffc] (0,0) rectangle (2,2);
    \node[text width=3cm, align=center, anchor=center] at (1,-.5) {\footnotesize \itshape Zone initiale de recherche};
  \end{scope}

  \path[draw, ->] (2.5,1.25) --++ (3,1.25)  node[pos=.5, above, sloped] {\footnotesize \itshape spatialisation};
  \path[draw, ->] (2.5,.75) --++ (3,-1.25)  node[pos=.5, above, sloped] {\footnotesize \itshape spatialisation};

  \begin{scope}[xshift=6cm, yshift=-1.5cm]
    \path[draw, dashed] (0,0) rectangle (2,2);
    \begin{scope}
      \begin{scope}
        \clip (0,0) rectangle (2,2);
        \fill[ffa] (1.25,.7) circle [radius=15pt];
        \path[ffc] (1.25,.7) circle [radius=15pt];
      \end{scope}
    \end{scope}
    \node[circle, inner sep=0pt,minimum size=4pt, fill] (c) at
    (1.25,.7) {};
    \node[text width=3cm, align=center, anchor=center] at (1,-.5)
    {\footnotesize \itshape Zone de localisation compatible};
  \end{scope}

  \begin{scope}[xshift=6cm,yshift=1.5cm]
    \path[draw, dashed] (0,0) rectangle (2,2);
    \fill[ffa](.8,1) circle [radius=20pt];
    \path[ffc](.8,1) circle [radius=20pt];
    \node[circle, inner sep=0pt,minimum size=4pt, fill] (c) at (.8,1)
    {};
    \node[text width=3cm, align=center, anchor=center] at (1,2.5)
    {\footnotesize \itshape Zone de localisation compatible};
  \end{scope}

  \path[draw, -] (8.5,2.5) --++ (1.5,-1.5);
  \path[draw, -] (8.5,-.5) --++ (1.5,1.5);
  \path[draw, ->] (10,1) --++ (1.5,0)  node[pos=.5, above] {\footnotesize \itshape fusion};


  \begin{scope}[xshift=12cm]
    \path[draw,dashed] (0,0) rectangle (2,2);
    \path[draw,dashed] (1.25,.7) circle [radius=15pt];
    \path[draw,dashed](.8,1) circle [radius=20pt];
    \begin{scope}
      \begin{scope}
        \clip (1.25,.7) circle [radius=15pt];
        \fill[ffa2] (.8,1) circle [radius=20pt];
        \path[ffc2] (.8,1) circle [radius=20pt];
      \end{scope}
      \begin{scope}
        \clip (.8,1) circle [radius=20pt];
        \path[ffc2] (1.25,.7) circle [radius=15pt];
      \end{scope}
    \end{scope}
    \node[circle, inner sep=0pt,minimum size=4pt, fill] (c) at (1.25,.7)
    {};
    \node[circle, inner sep=0pt,minimum size=4pt, fill] (c) at (.8,1)
    {};
  \end{scope}


\end{tikzpicture}
    \label{fig:comp_approches_sep}
  }
  \caption{Comparaison du processus de construction de la \emph{zone
      de localisation probable} pour une alerte à deux \emph{indices
      de localisation}.}
  \label{fig:comp_approches}
\end{figure}

Les 

\subsection{Principe de décomposition}

\subsection{Principe de modélisation non bivalente}

\subsection{Principe de modélisation explicite des connaissances}

\subsection{Principe d'intégration dans le contexte métier}

\subsection{Principe de raisonnement en monde ouvert}




%%% Local Variables:
%%% mode: latex
%%% TeX-master: "../../../../main"
%%% End:
