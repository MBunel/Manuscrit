Dans ce dernier chapitre nous allons présenter l'application de la
méthode présentée au cours des chapitres précédents à des cas réels,
des alertes précédemment traitées par le \ac{pghm}.

Pour ce faire nous avons sélectionné plusieurs alertes




Dans la première partie de ce chapitre (\autoref{sec:9-1}) nous
présenterons la manière dont nous avons implémenté la méthode définie,
puis nous détaillerons la modélisation de trois alertes : le
\enquote{Grand Veymont} (\autoref{sec:9-2}), \enquote{Moucherotte}
(\autoref{sec:9-3}) et le \enquote{fil rouge} (\autoref{sec:9-4}).

%%% Local Variables:
%%% mode: latex
%%% TeX-master: "../../../main"
%%% End:
