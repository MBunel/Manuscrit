Lorsque George Perec relate sa visite d’une maison usonienne\footnote{
  Néologisme de Frank Lloyd Wright. Qualifie des petites maisons
  individuelles en harmonie avec leur environnement, une part
  importante de son œuvre construit.} du Michigan, il en décrit le
délicat cheminement vers l’intérieur:

\begin{displayquote}[Perec, 1974, pp. 52-53]
  \enquote{%
    On commençait par suivre un sentier\textelp{}. Peu à peu,
    \textelp{} sans qu’à aucun instant on ait été en droit d’affirmer
    avoir perçu quelque chose comme une transition \textelp{}, le
    sentier devenait \textelp{} une allée \textelp{}. Puis
    apparaissait \textelp{} une toiture \textelp{} pratiquement
    indissociable de la végétation \textelp{}. Mais en fait, il était
    déjà trop tard pour savoir si l’on était dehors ou dedans.%
  }
\end{displayquote}

Les jeux avec l’environnement et les matériaux, relevés par
l’écrivain, sont les éléments d’un parcours savamment orchestré par
l’architecte pour fondre la construction dans son environnement,
trompant le visiteur et l’empêchant d’identifier une délimitation
claire entre dehors et dedans ; deux espaces tacitement considérés
comme immiscibles. Ainsi, en lieu et place d’une rupture franche,
c’est une transition progressive qui les sépare, rendant la définition
d’une limite, autre qu’arbitraire, impossible.

Or, la délimitation d’espaces cohérents est centrale dans la réflexion
géographique et l’existence d’objets spatiaux difficilement
délimitables ne va pas sans soulever des questions techniques et
épistémologiques \autocite{Burrough1996b}, auxquelles de nombreux
travaux, comme ceux cherchant à délimiter des espaces à partir de
ressentis \autocite{Arabacioglu2010}, de descriptions de positions
\autocite{Jones2007, Wolter2018,Bunel2019} ou de cartes mentales
\autocite{Dutozia2014}, ont été confrontés. Ces différents travaux ont
pour point commun d’avoir nécessité l’emploi de modèles avancés,
permettant la représentation d’objets spatiaux aux frontières mal
délimitées. Mais le grand nombre de modèles de ce type et l’absence de
consensus pour l’un d’eux rend délicat leur recensement et leur
analyse en vue d’une application originale.

C’est ce travail de recensement et de classification que nous
souhaitons entreprendre ici, en espérant qu’il puisse servir de point
d’entrée à toute personne amenée à manipuler des objets spatiaux
difficilement délimitables.  Cet état de l’art présente avant tout une
catégorisation des différents modèles proposés dans la littérature,
mais également le cadre conceptuel auxquels ils se rattachent et
notamment la notion d’imprécision au centre de ces questions. Notre
objectif n’est cependant pas de proposer une typologie originale des
différents concepts, la question ayant déjà été largement traitée
\autocite{Bouchon-Meunier1995,Fisher2006,Devilliers2019}. Nous
présenterons également les différentes théories mathématiques
permettant la modélisation de l’imprécison, comme l’ont récemment
proposé \autocite{Batton-Hubert2019}, ainsi que les implémentation de
modèles proposées.

Nous commencerons par présenter plus abondamment ces notions et
définir les différents concepts, tels que le vague ou l’imprécision,
nécessaires à leur compréhension. Puis nous énumérerons les
différentes théories mathématiques permettant de modéliser
l’imprécision. Enfin nous présenterons les différentes modélisations
de l’imprécision spatiale proposées dans la littérature.

\section{Les concepts de vague et d’imprécision spatiale}

Pour commencer nous allons définir les notions d’imprécision et de
vague et présenter leurs spécificités lorsqu’elles sont appliquées au
contexte spatial. Nous commencerons par présenter les notions dans
leur ensemble, avant de nous concentrer sur les liens entre
imprécision et géographie. Puis nous définirons l’imprécision
spatiale, concept qui sera utilisé tout au long de cet état de l’art.

L’exercice de définition d’un concept équivoque, tel que le vague,
n’est pas une tâche aisée, car l’on est rapidement confronté à
l’imprécision sémantique du langage naturel. C’est, en partie, ce
constat qui conduisit les philosophes Gottlob Frege (1848–1925) et
Bertrand Russell (1872–1970) à travailler sur une formalisation
mathématique de la logique, à même d’affranchir le processus de
réflexion des ambiguïtés du langage naturel \autocite{Williamson1994}.

La réflexion contemporaine sur les notions de précision, de vague et
d’imprécision remonte, selon \textcite{Williamson1994}, aux travaux de
Russell et plus spécifiquement à la publication, en 1923, de son
article \emph{Vagueness} \autocite{Russell1923}. Pour Russell, le
vague est l’opposé de la précision. Les deux concepts ont pour domaine
tout système de signes et ne se limitent donc pas au langage
naturel. Ils concernent tout type de représentation (\eg cartes,
photographies, mots) et n’ont donc de sens que pour qualifier une
relation entre deux systèmes de signes, définie comme précise si
bijective, \ie pour un système de signes donné, quel que soit le signe
considéré, il ne partage sa signification qu’avec un et un seul signe
d’un second système de signes. À l’inverse, la relation entre deux
systèmes de signes est vague si une représentation à plus d’un (ou
aucun) équivalent dans le second système, laissant dès lors place à
l’interprétation. Pour illustrer le concept de vague, Russell prend
l’exemple du mot \enquote{rouge} décrivant une teinte tacitement
connue de tous, mais dont on ne peut qu’arbitrairement fixer les
limites. De la même manière, les concepts de lac et île ne sont pas
suffisamment précis pour que leur dénombrement soit trivial
\autocite{Sarjakoski1996}. On pourrait multiplier les exemples à
loisir, car, dans la conception russellienne, aucun domaine n’échappe
au vague; toute représentation l’est\footnote{Y compris le terme
  \enquote{vague}, lui-même \autocite{Russell1923}, on parle alors
  d’imprécision d’ordre supérieur ou \emph{hight-order vagueness}
  \autocite{Williamson1994, Varzi2003}.}, à des degrés divers, et la
précision n’est qu’un idéal, hors d’atteinte.

Dans la littérature, le terme imprécis est régulièrement utilisé comme
synonyme de vague, notamment par \textcite{Zadeh1965} et dans une
grande partie des travaux se rattachant à la logique floue. Par
métonymie, le terme flou\footnote{Ou \emph{fuzzy} dans les
  publications anglophones.} est également employé dans le sens de
vague, imprécis. C’est, par exemple, le cas lorsque
\textcite{Lagacherie1996} parlent de \emph{fuzziness} ou encore quand
\textcite[218]{Brunet1992} définissent le flou comme : \enquote{la
  partie d’un système ou d’un espace dont les contours et les limites
  sont, soit imparfaitement connus ou connaissables, soit instables,
  soit imprécis \textelp{}}. De nombreux autres termes sont
ponctuellement utilisés, rendant la terminologie confuse
(\ref{tab:1}). Pour rendre notre propos le plus clair possible, nous
n’emploierons le terme \emph{flou} que pour qualifier des
formalisations fondées sur la théorie des sous-ensembles flous de
\bsc{Zadeh.} De plus, pour rester le plus proche possible du
vocabulaire utilisé en géomatique nous préférerons le terme
\emph{imprécis} à celui de \emph{vague}.

\begin{table}
  \centering
  \begin{tabular}{lm{5cm}m{5cm}}
    \firsthline
    &\multicolumn{1}{c}{\textbf{Précis}} & \multicolumn{1}{c}{\textbf{Imprécis}}\\
    \cline{2-3}
    \textbf{Anglophone}&\emph{Crisp, Sharp, Well defined, Fiat boundaries} &\emph{Fuzzy, Indeterminate, Undefined, Uncertain, Ill-Defined, Unclear, Vagueness, Bona fide boundaries}\\
    \textbf{Francophone}&\emph{Net}&\emph{Flou, Incertain, Vague}\\
    \lasthline
  \end{tabular}
  \caption{sqdq}
  \label{tab:1}
\end{table}

La notion d’\emph{imprécision} peut être associée à d’autres concepts,
comme l’\emph{exactitude} chez \textcite{Russell1923}, que l’on
retrouve chez \textcite{Bouchon-Meunier1995,Bouchon-Meunier2007} sous
le nom d’\emph{incertitude}. Ici, l’\emph{incertitude} est entendue
comme le doute que l’on peut avoir sur la validité d’une connaissance
\autocite{Bouchon-Meunier1995}. L’imprécision et l’incertitude sont
foncièrement liées et varient généralement en sens inverse
\autocite{Russell1923}. Ainsi, si la proposition (1) : \enquote{la
  distance de la Terre à la Lune est de \SI{384397}{\km}} est plus
  précise que la proposition (2): \enquote{la distance de la Terre à
    la Lune est d’environ \SI{384000}{\km}}, mais la seconde
  proposition est plus certaine car, son cadre de validité (la plage
  de valeurs de la distance Terre-Lune) est plus large. On peut en
  effet, considérer que la proposition (2) reste vraie pour une
  distance réelle de \SI{384397} ou \SI{384000}{\km} alors que la
moindre variation de l’ordre d’un kilomètre suffit à invalider la
première proposition (1). Dans certains cas, et notamment lorsque ces
notions sont appliquées à des objets spatiaux, il peut être délicat de
distinguer ces deux concepts. Cependant, ils sont fondamentalement
différents : l’\emph{imprécision} est une caractéristique invariable,
alors que l’\emph{incertitude} est contextuelle. Pour reprendre
l’exemple précédent, la proposition (2) est \emph{imprécise} et le
restera quel que soit le contexte, alors que sa certitude dépend des
connaissances de l’observateur. La véracité des propositions (1) et
(2) est, toutes choses égales par ailleurs, invariable, mais la
certitude de cette véracité est contextuelle.

L’\emph{imprécision} et l’\emph{incertitude} sont également associées
à la notion d’\emph{incomplétude}, qui désigne une connaissance
partielle. Ceci est dû au fait qu’un manque de connaissances peut
entraîner des \emph{incertitudes}, mais également des
\emph{imprécisions}
\autocite{Bouchon-Meunier1995,Bouchon-Meunier2007}. Pour
\bsc{Bouchon-Meunier} et plus généralement pour la communauté de
chercheurs en intelligence artificielle, la composition de ces trois
concepts définit la notion d’\emph{imperfection}
\autocite{Bouchon-Meunier1995} Dans la suite de ce document, nous
travaillerons à partir de cette typologie, même si de nombreuses
autres typologies de concepts ont cependant été proposées, que ce soit
dans le domaine de l’intelligence artificielle ou de la géomatique
(Fisher et al.,, 2006).

\section{Imprécision et géographie}

Les objets et les concepts géographiques n’échappent évidemment pas à
l’imprécision. Ainsi, \textcite{Russell1923} mentionnait déjà
l’existence d’objets dont la délimitation spatiale est imprécise, tel
que le système solaire. De nombreux autres objets spatiaux imprécis
ont été identifiés, comme l’illustre l’exercice de définition du
\emph{Brownfield}, entrepris par \textcite{Alker2000} et relevée par
\textcite{Bennett2001}, qui y voit un bon exemple de la difficulté
d’identifier une délimitation satisfaisante d’espaces naturels. Le
\emph{Brownfield}, tout comme les \emph{forêts}
\autocite{Bennett2001,Dilo2006,Fisher2006}, les \emph{montagnes}
\autocite{Varzi2001,Fisher2006,Chaudhry2008}, les \emph{vallées}
\autocite{Schneider2003} ou même le \emph{Soleil}
\autocite{Simons1999}, appartiennent à cette catégorie d’objets
spatiaux dont on ne peut fixer une limite. Cette énumération pourrait
laisser penser que l’imprécision ne concerne pas les artefacts,
pourtant l’expérience de \bsc{Perec} nuance cette affirmation. Comme
l’indique \textcite{Campari1996}, l’identification des frontières d’un
artefact, n’est pas aisée puisque dépendante du contexte
d’observation. Ainsi, la limite d’une \emph{ville} ou d’un
\emph{village} est tout aussi \emph{vague} que celle d’une zone
\emph{frontalière} \autocite{Varzi2001,Fisher2006}, comme l’illustre
la grande variabilité des définitions du concept de ville.

Tous ces objets géographiques, généralement qualifiés de \emph{vagues}
\autocite{Erwig1997}, \emph{imprécis} \autocite{Winter2000},
\emph{flous} \autocite{Lagacherie1996} ou d’\emph{objets aux
  frontières indéterminées} \autocite{Burrough1996}, s’opposent aux
objets dits \emph{nets} \autocite{Schneider2001} ou \emph{précis.}
\textcite{Smith2000} font usage d’un vocabulaire très différent en
opposant les \emph{fiat boundaries,} \ie les frontières précises, aux
\emph{bona fide boundaries,} caractérisant les objets spatiaux dont la
délimitation est univoque (\ref{tab:1}). Pour
\textcite{Couclelis1996}, les objets géographiques nets sont plus
l’exception que la norme. Ce constat est corollaire de l’avis d’
\textcite{OddAmbrosetti1987} pour qui \enquote{\textelp{} il est
  problématique et généralement arbitraire de tracer des
  limites\textelp{}}\footnote{\foreignquote{italian}{La realta ci
    mostra quanto sia problematico e spesso arbitrario tracciare dei
    confini \textelp{}} \autocite[200]{OddAmbrosetti1987}, traduction
  de l’auteur.}, limites qui selon \textcite[106]{Brunet2001} sont
\enquote{\textelp{} indécises, fuyant sans cesse devant l’analyse, et
  même, localement indécidables}. \textcite{Dutozia1994} considèrent,
quant à eux, que \enquote{\textelp{} l’espace géographique est par
  essence flou \textelp{}}.

Ainsi, si les concepts présentés jusqu’ici nous semblent actuellement
peu utilisés en géographie, de nombreuses notions et objets entrant
dans le champ d’étude de la discipline y sont fondamentalement
liés. C’est notamment le cas des différents maillages administratifs,
comme les régions \autocite{Brennetot2014}, mais aussi des frontières
\autocite{Brunet1992}, des seuils \autocite{Brunet1992, Levy2013}, des
discontinuités \autocite{Brunet1992, Brunet1997}, des franges
\autocite{Brunet1992}, des confins \autocite{Brunet1997} ou encore des
fronts pionniers \autocite{Brunet1992}, qui, comme tous les concepts
dérivant de la notion de limite sont généralement définis comme
pouvant être graduels ou progressifs \autocite{Brunet1992, Levy2013},
\ie foncièrement imprécis.

La question de la formalisation des objets spatiaux imprécis a
cependant été abordée en géographie, avant même le développement des
systèmes d’information géographiques \autocite{Robinson2003}. Dans les
années 70, ou, à la suite de l’élaboration de la théorie des
sous-ensembles flous \autocite{Zadeh1965}, plusieurs géographes,
rattachés au courant \emph{béhavioriste,} tels que
\textcite{Gale1972,Gale1976}, \textcite{Pipkin1978} ou
\textcite{Leung1979,Leung1987} ont identité les problèmes que
l’existence d’objets géographiques aux limites imprécises pouvait
poser à la géographie. Parmi ces problèmes, \textcite{Gale1976} a
identifié la question de la régionalisation. Problématique qui sera
également abordée par \textcite{Rolland-May1996,Rolland-May1999} lors
de ses travaux sur la définition de \emph{territoires de cohérence.}
Les travaux \emph{béhavioristes} aboutiront à la formalisation du
concept \emph{d’objet géographique imprécis} à l’aide de la théorie
des sous-ensembles flous \autocite{Leung1987}. Ces travaux n’auront,
semble-t-il, pas suffi à inscrire durablement le concept
d’\emph{imprécision} dans le champ de la géographie, puisque,
régulièrement apparaîtront des publications ré-introduisant ce concept
en géographie, c’est notamment le cas de \textcite{Fisher1998},
\textcite{Collins2000, Varzi2001} qui se fonderont indépendamment sur
l’exemple de la définition d’une montagne pour introduire cette
notion.

Parallèlement, \textcite{Rolland-May1984,Rolland-May1987} se fondera
notamment sur les travaux de \bsc{Gale} et \bsc{Leung} pour développer
le \emph{concept d’espace géographique flou.} Cette dénomination
qualifie la formalisation, à l’aide de la théorie des sous-ensembles
flous, de l’espace tel que conceptualisé en géographie. Ce travail,
permettra à \bsc{Rolland-May} de proposer une définition formelle de
notions courant utilisées en géographie, telles que les \emph{franges}
\autocite{Rolland-May1987}, définies comme la limite floue d’un espace
géographique, ou les \emph{discontinuités}, dérites comme une
configuration particulière d’ensemble flou
\autocite{Rolland-May2003}. Les différents travaux de
\bsc{Rolland-May} autour de la question de l’imprécision en géographie
ont offert a différents chercheurs d’aborder différemment des
questions géographiques \autocite{Dutozia2014}, comme, par exemple,
\textcite{deRuffray2004}, qui feront usage des concepts développés par
\bsc{Rolland-May} pour quantifier la cohérence de territoires, ou
\textcite{Didelon2013} qui emploient la logique floue pour exploiter
des cartes mentales.











