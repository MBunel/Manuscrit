% Citation début de chapitre
\dictum[Loi de Hofstadter]{It always takes longer than you expect, even when
  you take into account Hofstadter's Law}%

\chaptertoc{}

\addsec{Introduction}

Introduction chapitre 8

\section{Présentation générale de la}

\subsection{Opérateurs}

\tdi{Si ne pas passer les caractéristiques des opérateurs ici.}

\tdi{Nécessité d'utiliser des opérateurs duals}

\tdi{Enlever les opérateurs drastiques car ils sont trop sévères et ne
permettent pas de conserver des positions dont le degré d'appartenance
n'est pas au moins un fois de 1. En plus si j'ai n indices il faut
qu'il y ait n-1 indices dont le dégré est de 1.}

\tdi{Loi du tier-exclu}
\tdi{Principe de non contradiction}

Concepts non excluant

On a pas eu de problèmes du fait que cette loi ne soit pas validée

On peut trouver des exemples réels de cas ou ce n'est pas vrai

Lister les opérateurs qui valident tout

Caractéristiques non excluantes

Discuter du fait que ce soit mieux que ces principes ne soit pas
validés.

\tdi{Opérateurs archimédiens}

Mettre une figure de l'intersection d'angle avec les opérateurs
archimédiens (ex. sud et sud-est).

Ce critère discalifie lukacevicz et probabiliste

\tdi{Raisonnement en termes d'interprétabilité}

\tdi{doublon avec le chapitre 4 partie 2 ??}

\section{La prise en compte de la confiance}

\addsec{Conclusion}

Conclusion chapitre 8
%%% Local Variables:
%%% mode: latex
%%% TeX-master: "../../../main"
%%% End:
