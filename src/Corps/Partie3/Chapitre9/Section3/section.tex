
\subsection{Présentation de l'alerte}
\label{subsec:9-3-1}


\begin{figure}
  \centering
  \input{./figures/ZIR_moucherotte.tex}
  \caption{Zone initiale de recherche de l'alerte
    \enquote{Moucherotte}}
  \label{fig:zir_moucherotte}
\end{figure}

\subsubsection{Identification des indices de localisation}
\label{subsec:9-3-1-1}

Section

\begin{dialogue}
  % 
  \Sec \nex{1-1} Donc vous étiez au sommet du Moucherotte à 17h30.
  %
  \Req Ouais.
  %
  \Sec \nex{2-1} Vous êtes redescendue par le chemin que
  vous avez pris à la montée ?
  %
  \Req \nex{3-1} Oui, ben j’ai essayé de le trouver, je l’ai pas
  trouvé, \nex{4-1} donc j’ai viré un peu à gauche, \nex{5-1} sur ce
  grand couloir enseigné. \nex{6-1, 6-2} Je crois qu’il y a des gens
  qui redescendent par Lens-en-Vercors, \nex{7-1, 7-2} mais je suis
  montée de Sain-Nizier, du haut du tremplin.
\end{dialogue}
% 


\begin{dialogue}
  % 
  \Sec \nex{8-1} Vous êtes gardée à Saint-Nizier ?
  %
  \Req \nex{9-1} Saint-Nizier ouais, le haut du tremplin
\end{dialogue}


\begin{dialogue}
  \Req \nex{10-1} Mais là je suis complètement paumée, \nex{10-2} dans
  les cailloux. \nex{11-1} Jusqu’à maintenant j’étais dans la forêt,
  \nex{11-2} un peu de neige. \nex{12-1} Mais là, j’ai failli me tuer
  en glissant \nex{12-2} parce qu’il y a plein de petits morceaux de
  cailloux.
\end{dialogue}


\subsection{Modélisation de l'alerte}
\label{subsec:9-3-2}

\subsubsection{Décomposition des indices de localisation}
\label{subsec:9-3-2-2}

\subsubsection{Spatialisation des indices de localisation}
\label{subsec:9-3-2-3}

\subsubsection{Fusion des zones de localisation compatibles}
\label{subsec:9-3-2-4}

\subsection{Critique de la modélisation}
\label{subsec:9-3-3}


