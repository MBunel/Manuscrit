Dans ce chapitre nous avons présenter le déroulé de la dernière phase
de notre méthode : \emph{la fusion.} La \emph{phase de fusion} a pour
objectif de regrouper toutes les \emph{zones de localisation}
construites lors de la phase de spatialisation, en vue d'obtenir la
zone de localisation probable de la victime.

Pour ce faire il est nécessaire de regrouper les \ac{zlc} qui ont été
construites à partir \emph{d'indices de localisation,} décomposés lors
de la première phase de la méthode. La phase de décomposition et la
phase de fusion se répondent, si bien qu'au trois étapes de
décomposition répondent trois étapes de fusion. Pour être mises en
place ces étapes nécessitent des opérateurs permettant de réaliser des
intersection et des unions inter-zones, les \emph{t-normes} et les
\emph{t-conormes.} Toutefois plusieurs couples de
\emph{t-normes}/\emph{t-conormes,} aux propriétés diverses, ont été
proposés. À près avoir analysé leurs caractéristiques spécifiques,
nous avons choisi d'utiliser les opérateurs initialement proposés par
\textcite{Zadeh1965}, qui sont apparus comme les plus adaptés à notre
cas d’application.

Nous avons également présenté la méthode retenue pour modéliser la
confiance des secouristes dans les indices qu'ils traitent. Pour ce
faire nous avons employé la \emph{théorie des possibilités} proposée
par \textcite{Zadeh1978} en complément de la théorie des
sous-ensembles flous et permettant de combiner modélisation de
l’imprécision et de l'incertitude.

Au terme de ce chapitre on a donc un regard complet sur l'ensemble de
la méthode proposée pour transformer des descriptions de position en
des positions exprimées dans un référentiel direct.

%%% Local Variables:
%%% mode: latex
%%% TeX-master: "../../../main"
%%% End:
