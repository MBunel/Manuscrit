Dans cette troisième et dernière partie nous allons détailler
l’application de la méthode précédemment présentée
(cf. \autoref{part:02}) à des cas réels. Cette partie présente et
commente la manière dont nous avons construit les \emph{zones de
  localisation probables} spatialisant deux alertes réelles,
auxquelles les secouristes du \ac{pghm} de Grenoble ont été
confrontés. Nous présenterons chaque alerte et détaillerons la manière
dont les indications données par les requérants ont été formalisés en
des indices de localisation. Puis nous détaillerons la manière dont
ces indices ont été décomposés, pour ensuite être spatialisés et
fusionnés. Cette approche nous permettra de mettre en évidence les
principales limites de notre travail et de présenter les solutions
permettant de les contourner. Cette partie ne se compose que d'un seul
chapitre, le \ref{chap:09}, au sein duquel nous présenterons la
spatialisation de deux alertes sélectionnées.

%%% Local Variables:
%%% mode: latex
%%% TeX-master: "../../main"
%%% End:
