
\tdi{Détailler les métriques définies
  figure \ref{fig:methode_rasterisation} + exemple}





\subsection{Métriques intrinsèques}

\tdi{Présentation des métriques qui ne dépendent pas de l'objet de
  référence -> elles sont calculables une fois. Ex : Pente, altitude
  absolue (mais déjà calculée vu que MNT).}

\tdi{Est-ce qu'on peut vraiment faire cette distinction ??}

\subsection{Métriques extrinsèques}

\tdi{Présentation des métriques locales, qui changent pour chaque
  \emph{objet de référence}, distance, orientation, visibilité}

\subsection{Métriques Paramétriques}

\subsection{Métriques non paramétriques}

\begin{table}
  \centering
  \includegraphics[width=\textwidth]{../../../../tmp/metriques.jpeg}
  \caption{Types de métriques}
  \label{fig:type_metriques}
\end{table}


%%% Local Variables:
%%% mode: latex
%%% TeX-master: "../../../../main"
%%% End:
