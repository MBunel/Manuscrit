Dans le \autoref{chap:2} nous expliquions que, dans notre cas, le
passage d'un référentiel spatial indirect à un référentiel direct
nécessitait deux étapes. La première d'entre elle, \emph{la
  spatialisation,} permettant de passer d'un \emph{indice de
  localisation} à une \emph{zone de localisation compatible} et la
seconde, la \emph{fusion} combinant les différentes \emph{zones de
  localisation compatibles} issues de la \emph{spatialisation} de tous
les \emph{indices de localisation} d'une alerte dans le but de créer
la \emph{zone de localisation probable,} \ie le résultat final de
notre processus.

Dans cette partie nous allons présenter une première formalisation de
notre méthodologie. L'objectif n'est cependant pas de présenter tous
les aspects des étapes de \emph{décomposition,} de
\emph{spatialisation} et de \emph{fusion} chaque étape de la méthode
étant détaillée dans un chapitre dédié (respectivement les chapitres
\ref{chap:05}, \ref{chap:07} et \ref{chap:08}). Nous ne parlerons donc
pas de certains aspects avancés de la modélisation, comme la prise en
compte de l'imprécision (cf. \autoref{chap:07}) ou de l'incertitude
(cf.). La méthode ici présentée est la plus générique possible  

La \autoref{fig:methodo_1} présente une synthèse visuelle de la
méthodologie que nous avons définie.

\subsection{Décomposition}

La décomposition est une composante de \emph{l'étape de spatialisation}

Ce principe sera détaillé dans le \autoref{chap:05}

\subsection{Spatialisation}

Faite pour chaque indice (chap2)
Nécessite des méthodes spécifiques (eda)
Plusieurs objets de référence

Ce principe sera détaillé dans le \autoref{chap:07}

Pour rappel, l'opération de \emph{spatialisation} consiste à
construire une zone (dite de \emph{localisation compatible})
correspondant à \emph{l'indice de localisation}
(\autoref{fig:obj_spa}). Ainsi on peut définir une \emph{zone de
localisation compatible} (pour un \emph{indice de localisation} donné)
comme l'ensemble des positions où \emph{l'indice de localisation} est
vrai, soit :


\begin{equation}
  \text{\textsf{ZLC}}_i = \{(x,y) \ |\ \text{\textsf{RL}}_i(x,y) \wedge (x,y) ∈ \mathbb{R}^2\}
\end{equation}

Avec \textsf{ZLC} la \emph{zone de localisation compatible,} \(i\) un
\emph{indice de localisation,} \((x,y)\) un couple de réels formant
une position et \(\text{\textsf{RL}}_i(x,y)\) une fonction vraie si la
position \((x,y)\) est dans la \emph{zone de localisation compatible}
correspondant à \emph{l'indice de localisation} \(i\).

\subsection{Fusion}

De toutes les étapes de notre méthodologie, la phase de \emph{fusion}
est, de loin, la plus simple. Son objectif est de combiner les
différentes \emph{zones de localisation compatibles} créées lors de la
phase de \emph{spatialisation} en une seule, la \emph{zone de
  localisation probable} (ZLP).

Ce principe sera détaillé dans le \autoref{chap:08}


\begin{equation}
  \text{\textsf{ZLP}}_I = ⋂_{i \in I} \text{\textsf{ZLC}}ᵢ
\end{equation}

Avec \textsf{ZLP} la \emph{zone de localisation probable,} \(I\),
l'ensemble des \emph{indices de localisation} pris en compte, \(i\) un
\emph{indice de localisation} et \textsf{ZLP} un \emph{zone de
  localisation compatible.}

\begin{landscape}
  \begin{figure}[H]
    \centering
    \begin{tikzpicture}
  \tikzset{
    acc/.style={decorate,decoration={brace,raise=0cm,amplitude=.1cm}},
    accm/.style={acc,decoration={mirror}},
    acc3/.style={acc,decoration={amplitude=.2cm}},
    acc3m/.style={acc3,decoration={mirror}}
  }
  
  %\draw[step=.5,black,thin] (0,0) grid (22,-14);

%% Matrices
\node[font=\bfseries, baseline] (I) at (0,0) {$I$};

\matrix [matrix of math nodes,row sep=0.1cm,column sep=0.1cm,
anchor=south west, nodes={anchor=base, baseline}] (is) at
([xshift=1.5cm]I.north east)
{i_1\\i_2\\\vdots{}\\i_n\\};

\matrix [matrix of math nodes,row sep=0.1cm,column sep=0.1cm,
anchor=south west, nodes={baseline}] (is_o) at
([xshift=1.5cm,yshift=.5cm]is-1-1.east) {o_1\\o_2\\\vdots{}\\o_n\\};

\matrix [matrix of math nodes,row sep=0.1cm,column sep=1.5cm,
anchor=south west, nodes={minimum height=5mm}] (is_o_r) at ([xshift=1.5cm,yshift=.5cm]is_o-1-1.east)
{
  \text{RSA}_1&\text{ZLC}_{rsa1}\\
  \text{RSA}_2&\text{ZLC}_{rsa_2}\\
  \vdots{}&\vdots{}\\
  \text{RSA}_n&\text{ZLC}_{rsa_n}\\};

\matrix [matrix of math nodes,row sep=0.1cm,column sep=0.1cm,
anchor=west, nodes={baseline}] (zlb) at ([xshift=1.5cm]is_o_r.east  |- is_o)
{ZLC_{o_1}\\ZLC_{o_2}\\\vdots{}\\ZLC_{o_n}\\};

\matrix [matrix of math nodes,row sep=0.1cm,column sep=0.1cm,
anchor=west, nodes={anchor=base, baseline}] (zlc) at ([xshift=1.5cm]zlb.east  |- is)
{ZLC_1\\ZLC_2\\\vdots{}\\ZLC_n\\};

\node [anchor=west] (zlp) at ([xshift=1.5cm]zlc.east |- I)
{ZLP};


%% Accolades 
\begin{scope}
  \draw[acc3m] (is.north west) -- (is.south west);
  \draw[acc3m] (is_o.north west) -- (is_o.south west);
  \draw[acc3m] (is_o_r.north west) -- (is_o_r.south west);

  \draw[acc] (is-1-1.north east) -- (is-1-1.south east);
  \draw[acc] (is_o-1-1.north east) -- (is_o-1-1.south east);
  \draw[acc] (is_o_r-1-1.north east) -- (is_o_r-1-1.south east);

  \draw[accm] (is_o_r-1-2.north west) -- (is_o_r-1-2.south west);
  \draw[accm] (zlb-1-1.north west) -- (zlb-1-1.south west);
  \draw[accm] (zlc-1-1.north west) -- (zlc-1-1.south west);

  \draw[acc3] (is_o_r.north east) -- (is_o_r.south east);
  \draw[acc3] (zlb.north east) -- (zlb.south east);
  \draw[acc3] (zlc.north east) -- (zlc.south east);
\end{scope}


\begin{scope}
  \draw[->, black] (I.east) -- (is.west) node[pos=0,below] {1} node[pos=1,below]{1..*};

  \draw[->, black] (is-1-1.east) -- (is_o.west) node[pos=0,below] {1} node[pos=1,below]{1..*}; 
  
  \draw[->, black] (is_o-1-1.east) --
  (is_o_r.west) node[pos=0,below]
  {1} node[pos=1,below]{1..*}; 
  
  \draw[->, black] (is_o_r-1-1.east) -- (is_o_r-1-2.west) node[pos=0,below]
  {1} node[pos=1,below]{1}; 
  
  \draw[->, black] (is_o_r.east) -- (zlb-1-1.west) node[pos=0,below] {1..*} node[pos=1,below]{1}; 
  
  \draw[->, black]
  (zlb.east) -- (zlc-1-1.west) node[pos=0,below] {1..*} node[pos=1,below]{1}; 
  
  \draw[->, black] (zlc.east) -- (zlp.west) node[pos=0,below] {1..*} node[pos=1,below]{1}; 
\end{scope}


\begin{scope}
    \draw[->, black, dotted] ([xshift=.5cm]I.east) -- ([xshift=-.5cm]zlp.west) node[pos=0,below]
  {1} node[pos=1,below]{1}; 

  \draw[->, black, dotted] ([xshift=.5cm]is.east) -- ([xshift=-.5cm]zlc.west) node[pos=0,below]
  {1} node[pos=1,below]{1}; 

  \draw[->, black, dotted] ([xshift=.5cm]is_o.east) -- ([xshift=-.5cm]zlb.west) node[pos=0,below]
  {1} node[pos=1,below]{1}; 
\end{scope}

%% XX

\begin{scope}[below of=I]
  \draw[acc3m] (0,0) --++ (8,0) node[pos=.5, yshift=-.5cm] {\emph{Décomposition}};
  \draw[acc3m] (8,0) --++ (3,0) node[pos=.5, yshift=-.5cm] {\emph{Spatialisation}};
  \draw[acc3m] (13.4,0) --++ (2,0) node[pos=.5, yshift=-.5cm] {\emph{Fusion}};
\end{scope}

\end{tikzpicture}
    \caption{Méthodologie}
    \label{fig:methodo_1}
  \end{figure}
\end{landscape}
