
\subsection{Présentation de l'alerte}
\label{subsec:9-3-1}


\begin{figure}
  \centering
  \input{./figures/ZIR_moucherotte.tex}
  \caption{Zone initiale de recherche de l'alerte
    \enquote{Moucherotte}}
  \label{fig:zir_moucherotte}
\end{figure}

\subsubsection{Identification des indices de localisation}
\label{subsec:9-3-1-1}

Section

\begin{dialogue}
  % 
  \Sec \nex{1} Donc vous étiez au sommet du Moucherotte à 17h30.
  %
  \Req \nex{2} Ouais.
  %
  \Sec \nex{3} Vous êtes redescendue par le chemin que
  vous avez pris à la montée ?
  %
  \Req \nex{4} Oui, ben j’ai essayé de le trouver, je l’ai pas trouvé,
  \nex{5} donc j’ai viré un peu à gauche, \nex{6} sur ce grand couloir
  enseigné. \nex{7} Je crois qu’il y a des gens qui redescendent par
  Lens-en-Vercors, \nex{8} mais je suis montée de Sain-Nizier, du haut
  du tremplin.
\end{dialogue}
% 
Les premiers extraits de cette alerte potent

\begin{dialogue}
  % 
  \Sec \nex{9} Vous êtes gardée à Saint-Nizier ?
  %
  \Req \nex{10} Saint-Nizier ouais, le haut du tremplin
\end{dialogue}
%
Ffdsfdsf


\begin{dialogue}
  \Req \nex{11} Mais là je suis complètement paumée, \nex{12} dans les
  cailloux. \nex{13} Jusqu’à maintenant j’étais dans la forêt,
  \nex{14} un peu de neige. \nex{15} Mais là, j’ai failli me tuer en
  glissant \nex{16} parce qu’il y a plein de petits morceaux de
  cailloux.
\end{dialogue}
%
Trucs à dire

\begin{dialogue}
  \Req \nex{17} En partant du sommet du Moucherotte, donc j’ai viré un
  peu à gauche et \nex{18} après j'ai coupé pour \nex{19} descendre,
  descendre, descendre, descendre et \nex{20} j’ai pas retrouvé mon
  sentier et \nex{21} je suis certainement descendue beaucoup trop bas
  et là \nex{22} j’ai plus dutout la force de remonter, je suis trop
  descendue.
\end{dialogue}
%
Trucs

\begin{dialogue}
  \Req \nex{23} Mon but c’était de descendre, descendre, descendre
  vers les maison qui sont en bas, \nex{24} mais je là j’ai une grosse
  barre rocheuse à droite, \nex{25} j’ai plein de petits cailloux très
  glissants, \nex{26} c’est très raide. \nex{27} Je suis assise là, au
  milieu de nulle part.
\end{dialogue}
%
dsq

\begin{dialogue}
  \Sec \nex{28} Dans l’idée, vous êtes repartie en direction du
  tremplin ? On est d’accord ? Vous avez emprunté le même chemin ?
  % 
  \Req \nex{29} Voila. Oui. \nex{30} Mais, pas par la droite où
  \nex{31} apparemment y’a deux sortes de chemins, pour le
  rejoindre. \nex{32} Parce-qu’à un moment donné y’a une espèce de
  cabane qu’on voit quand on est au Moucherotte. C’est pas
  là. \nex{33} Moi je suis allée plutôt sur la gauche. J’ai fait quoi,
  200 mètres à gauche et \nex{34} puis j’ai coupé à la perpendiculaire
  en redescendant.
  %
  \Sec \nex{35} Et vous avez jamais croisé un chemin ?
  %
  \Req \nex{36} J’étais arrivée par là et j’ai pas retrouvé de chemin.
\end{dialogue}
%
Ces extraits

\begin{dialogue}
  \Sec \nex{37} Vous êtes donc plutôt rentrée dans la forêt là ?
  %
  \Req \nex{38} Là je suis un peu à la sortie. \nex{39} J’ai fait un
  peu d’escalade là, \nex{40} parce que c’est très rocheux. \nex{41}
  Je suis dans une partie où il y a plein d’arbres tous secs et pas
  trop d’arbres. Je suis complétement à découvert
\end{dialogue}
%
Ces extraits


\begin{dialogue}
  \Req \nex{42} Et juste à ma droite y’a une grosse, grosse barre
  rocheuse. \nex{43} Ouais et devant moi j’ai la plaine et \nex{44} à
  ma gauche j’ai quand même des sapins.
\end{dialogue}
%
Ces extraits


\begin{dialogue}
  \Sec \nex{45} Et donc, dans la plaine qu’est-ce que vous voyez ?
  %
  \Req \nex{46} Eh ben je vois un peu d’eau.
  %
  \Sec \nex{47} Vous voyez des habitations ? Vous voyez une route ?
  %
  \Req \nex{48} Un petit peu, mais les habitations c’est très
  parses. \nex{49} Et juste devant moi y’a un groupe de 4, 5 maisons
  ou des fermes.
  %
  \Sec \nex{50} C’est-à-dire ? À combien de mètres ?
  %
  \Req \nex{51} Ah non, mais c’est très, très, très loin ça. Je les
  vois mais c’est pas accessible.
\end{dialogue}
%
Voila


\begin{dialogue}
  \Req \nex{52} Devant moi y’a des tas de barres rocheuses en fait, je
  vois pas comment je peux arriver jusque-là et \nex{53} en dessous
  des barres rocheuses ça repart en forêt. \nex{54} Je sais pas si je
  peux continuer à descendre, c’est vachement raide et glissant.
\end{dialogue}
%
Pour finir

\subsection{Modélisation de l'alerte}
\label{subsec:9-3-2}

\subsubsection{Décomposition des indices de localisation}
\label{subsec:9-3-2-2}

\subsubsection{Spatialisation des indices de localisation}
\label{subsec:9-3-2-3}

\subsubsection{Fusion des zones de localisation compatibles}
\label{subsec:9-3-2-4}

\subsection{Critique de la modélisation}
\label{subsec:9-3-3}



%%% Local Variables:
%%% mode: latex
%%% TeX-master: "../../../../main"
%%% End:
