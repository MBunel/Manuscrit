
Jusqu'à présent nous avons utilisé la notion \emph{d'indice de
  référence}

Nous allons à présent formaliser ce concept en vue de la définition de
la méthode de construction de la \emph{zone de localisation probable}
(\autoref{\label{sec:4-3}}).

Comme nous l'expliquions dans le \autoref{Chap:02}, les \emph{indices
  de localisation} sont l'élément de base des alertes.

Ils contiennent toutes les informations données par le requérant pour
décrire sa positon

Un \emph{indice de localisation} (\(i\)) est donc formalisable en un
triplet de la forme :
%
\begin{equation}
  i = (S, Rl, Or)
\end{equation}
%
Avec \(\) le \emph{sujet,} \(\) la \emph{relation de localisation} et
\(\) \emph{l'objet de référence.}

Cette définition est cependant assez restrictive, puisqu'elle ne
permet pas d'exprimer certaines des configurations que nous avons
décrites précédemment.                                                                                                                                             

Tout d'abord, il est trop restrictif de considérer qu'un \emph{indice
  de localisation} ne peut contenir qu'un seul \emph{objet de
  référence}

On peut donc remplacer \(\) par un ensemble \emph{d'objets de
  référence,} la définition de \emph{l'indice de localisation} est
alors :

\begin{equation}
  i = (S, Rl, {Or_1, Or_2, \ldots, Or_n})
\end{equation}

Cependant cette définition n'est pas non plus satisfaisante.

%%% Local Variables:
%%% mode: latex
%%% TeX-master: "../../../../main"
%%% End:
