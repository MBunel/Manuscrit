Dans ce chapitre nous allons présenter en détail la première phase de
notre méthode de construction de \emph{zones de localisation
  probables,} la \emph{phase de décomposition.} Comme nous
l'indiquions lors de sa présentation (cf. \autoref{chap:04}) cette
phase est décomposée en trois étapes distinctes, la
\emph{décomposition} de \emph{l'ensemble des indices de localisation,}
la \emph{décomposition} des \emph{objets de référence indéfinis} et la
\emph{décomposition} des \emph{relations de localisation.} Cependant
ces différentes étapes ne sont pas d'une difficulté comparable. Si la
\emph{décomposition} des \emph{relations de localisation} nécessite
une formalisation à priori (on ne demande pas au secouriste de saisir
les relations de localisation sous leur forme décomposée), ce n'est
pas le cas des deux premières étapes, où les différents éléments
(\emph{indices de localisation} et \emph{objets de référence,} même
\emph{indéfinis}) sont distingués dès la saisie. Ainsi, comme nous
l'indiquions dans le chapitre précédent, la dernière étape de cette
phase est la seule à présenter des difficultés scientifiques et
techniques. C'est pourquoi ce chapitre, destiné à détailler la phase
de \emph{décomposition,} traitera exclusivement de sa dernière étape,
la \emph{décomposition des relations de localisation.}

Dans la première partie de ce chapitre nous . Puis, dans une seconde
partie nous présenterons le processus de définition de
\emph{l'ontologie des relations de localisation} atomiques.

%%% Local Variables:
%%% mode: latex
%%% TeX-master: "../../../../../../Thèse/Manuscrit/src/main"
%%% End:
