Ce chapitre se destine à formaliser notre méthode de transformation
d'une \emph{description de position} en une zone de \emph{localisation
  probable.} La seconde partie de ce manuscrit étant construite
suivant une logique de complexification graduelle, nous ne
présenterons ici que les grands aspects de notre méthode, qui sera
complétée au fur et à mesure de cette partie.

Pour construire notre méthode nous avons pris le parti de définir un
certain nombre de \emph{principes de modélisation} que nous souhaitons
suivre. Ces derniers, couplés aux objectifs scientifiques de la thèse
(cf. \autoref{chap:02}) forment une ossature conceptuelle, permettant
de guider la définition de notre méthode.

La première partie de ce chapitre est dédiée à la présentation de ces
\emph{principes de modélisation,} puis une seconde partie sera
destinée à la définition de notre méthode.