Dans ce chapitre nous avons dressé un état de l'art sur les deux
objectifs scientifiques principaux de cette thèse, la question de la
spatialisation des relations de localisation et la modélisation de
l'imprécision spatiale. Ces deux travaux nous ont permis de mettre en
évidence le fait que de nombreuses solutions aux problèmes que nous
rencontrons avaient été proposées. Le choix d'une de ces solutions,
que ce soit pour la \emph{spatialisation} d'un type de \emph{relation
  de localisation} ou pour la modélisation de \emph{l'imprécision
  spatiale} devra donc être fait compte tenu des spécificités de notre
contexte applicatif. Ces choix seront détaillés dans la seconde partie
de ce document.


%%% Local Variables:
%%% mode: latex
%%% TeX-master: "../../../main"
%%% End:
