Les cinq objectifs scientifiques de cette thèse nous permettent de
cerner la problématique de ce travail de doctorat. Comme nous l'avons
déjà mentionné, notre objectif principal est de construire la
\emph{zone de localisation} correspondant à une description en langage
naturel. Toutefois notre travail ne porte pas directement sur
l'analyse du langage naturel, nous ne cherchons pas à identifier les
\emph{relations de localisation,} ou les \emph{objets de référence} en
direct lors de l'appel téléphonique du requérant aux secours ou à
l'aide d'une retranscription de celui-ci. Notre travail ne porte que
sur la \emph{spatialisation.} Cette étape nécessite de développer des
méthodes de spatialisation permettant construire les \emph{zones de
  localisation} correspondant à chaque \emph{relation de
  localisation,} c'est le propos de notre premier objectif :
scientifique \enquote{\nameref{subsec:2-1-1}}.

Cependant, la simple définition de méthodes de \emph{spatialisations}
ne peut suffire pour proposer un résultat satisfaisant. D'une part car
les \emph{relatons de localisation} sont des concepts \emph{imprécis.}
Ainsi, même si l'on est capable d'identifier des méthodes de
\emph{spatialisation} adéquates, il ne sera pas aisé de définir la
limite entre ce qui appartient à la \emph{zone de localisation} et ce
qui n'y appartient pas. Il est par conséquent nécessaire de prendre en
compte \emph{l'imprécision} lors de la définition de la méthode de
spatialisation, comme nous l'indiquions lors de la présentation de
notre second objectif scientifique :
\enquote{\nameref{subsec:2-1-2}}. Il est également nécessaire de
prendre en compte la fausseté potentielle des \emph{indices de
  localisation.} Comme nous l'expliquions lors de la présentation de
notre troisième objectif scientifique
\enquote{\nameref{subsec:2-1-3}}, les \emph{indices de localisation}
donnés par le requérant peuvent être faux. L'ajout de tels indices
dans le processus de \emph{spatialisation} peut l'impacter
considérablement, il est donc nécessaire de pouvoir gérer ces cas, en
prenant en compte \emph{l'incertitude} des \emph{indices de
  localisation,} \ie le doute que le secouriste a sur leur véracité.

Les méthodes de \emph{spatialisation} élaborées au sein de l'objectif
: \enquote{\nameref{subsec:2-1-1}}, sont destinées à spatialiser les
\emph{indices de localisation} séparément. Il est donc nécessaire de
définir une méthode de fusion, pour agréger les résultats
(cf. objectif \ref{subsec:2-1-4}). Enfin l'ensemble des résultats
produits doivent êtres évalués, comme décrit dans la présentation de
l'objectif \enquote{\nameref{subsec:2-1-5}}.

\begin{figure}[h]
  \centering
  \begin{tikzpicture}

\node[text width=2.5cm, align=center] (0,0) {\small\enquote{Je suis proche d'une maison}};

\path[draw, ->] (2,0) --++ (2.5,0)  node[pos=.5, above] {\footnotesize \itshape spatialisation};

\begin{scope}[xshift=6.5cm]
\path[ffa] (0,0) circle [radius=30pt];
\path[ffc] (0,0) circle [radius=30pt];

\node[circle, inner sep=0pt,minimum size=4pt, fill] (c) at (0,0) {};
\node[circle, inner sep=0pt,minimum size=8pt] (c2) at (0,0) {};
\node[anchor=west] (m) at  (1.75,1) {\footnotesize \itshape objet de référence};
\path[draw, ->] (m.west) -- (c2);

\node[anchor=west, text width=3cm] (m2) at  (1.75,0) {\footnotesize \itshape zone de localisation compatible};
\path[draw, ->] (m2.west) --++ (-.5,0);
\end{scope}
\end{tikzpicture}
  \caption{Illustration du processus de spatialisation d'un
    \emph{indice de localisation}}
  \label{fig:obj_spa}
\end{figure}


\begin{figure}[h]
  \centering
  \begin{tikzpicture}
  % ZLC 1
  \path[ffa] (0,0) circle [radius=30pt]; % Aire
  \path[ffc] (0,0) circle [radius=30pt]; % Contour
  % ZLC 2
  \path[ffa2, pattern color=RdBu-9-3] (.25,-.5) ellipse (40pt and 20pt); % Aire 
  \path[ffc2, draw=RdBu-9-2] (.25,-.5) ellipse (40pt and 20pt); % Contour
  % Objet de ref
  % 1
  \node[circle, inner sep=0pt,minimum size=4pt, fill] (c) at (0,0) {}; % visible
  \node[circle, inner sep=0pt,minimum size=8pt] (c2) at (0,0) {}; % fictif
  % 2
  \node[circle, inner sep=0pt,minimum size=4pt, fill] (c1) at (.25,-.5) {}; % visible
  \node[circle, inner sep=0pt,minimum size=8pt] (c12) at (.25,-.5) {}; % fictif
  % Légende
  \node[anchor=east] (m) at  (-1.5,.25) {\footnotesize \itshape objets de référence};
  \path[draw, ->] (m.east) -- (c2);
   \path[draw, ->] (m.east) -- (c12);
  \node[anchor=north west, text width=3cm] (m2) at (1.75,-.4) {\footnotesize
    \itshape zones de localisation compatibles};
  \path[draw, ->] (m2.west) --++ (-1.7,0);
  \path[draw, ->] (m2.west) -- (1.2,-.7);
  % Fléche + texte
  \path[draw, ->] (2.25,0) --++ (2.75,0)  node[pos=.5, above] {\footnotesize \itshape fusion};
  % Fusion
  \begin{scope}[xshift=6.75cm]
    \path[draw, dashed] (0,0) circle [radius=30pt]; % Contour
    \path[draw, dashed] (.25,-.5) ellipse (40pt and 20pt);
    \begin{scope}
      \clip (.25,-.5) ellipse (40pt and 20pt);
      \fill[ffa2] (0,0) circle [radius=30pt];
      \path[ffc2] (0,0) circle [radius=30pt];
    \end{scope}
    \begin{scope}
      \clip (0,0) circle [radius=30pt];
      \path[ffc2] (.25,-.5) ellipse (40pt and 20pt);
    \end{scope}
    \node[circle, inner sep=0pt,minimum size=4pt, fill] (c) at (0,0)
    {};
    \node[circle, inner sep=0pt,minimum size=4pt, fill] (c) at (.25,-.5) {};
    \node[anchor=west, text width=3cm] (m2) at  (1.25,.75) {\footnotesize \itshape zone de localisation probable};
    \path[draw, ->] (m2.south west) --++ (-.6,-.4);
  \end{scope}
\end{tikzpicture}
  \caption{Illustration du processus de spatialisation d'un
    \emph{indice de localisation}}
  \label{fig:obj_spa}
\end{figure}

%%% Local Variables:
%%% mode: latex
%%% TeX-master: "../../../../main"
%%% End:
