Pour formaliser les décompositions des \emph{relations de
  localisations} nous avons opté pour la construction d'une
\emph{ontologie} des \emph{relations de localisation atomiques}
contenant l'ensemble des \emph{relations de localisation} identifiées
comme pertinentes et leur(s) décomposition(s).

\tdi{Ajouter ontologie des objets de référence}

Pour identifier les \emph{relations de localisations} pertinentes pour
notre contexte applicatif nous avons entrepris de recenser celles qui
sont les plus utilisées lors des alertes. Pour ce faire nous avons
retranscrit et analysé différents enregistrements d'alertes passées.

Le travail de retranscription des alertes (et dans une moindre mesure
celui de leur analyse) a été réalisé collaborativement
\autocite{Bunel2019}, les résultats pouvant bénéficier à différents
objectifs du projet Choucas.

\subsection{Les corpus d'alertes}

Les retranscriptions utilisées pour construire \emph{l'ontologie des
  relations de localisation atomiques} ont été réalisées à partir d'un
ensemble de 52 enregistrements d'alertes transmis par le \ac{pghm} de
Grenoble et produits par les \ac{pghm} de Chamonix (5 alertes) et de
Grenoble (47 alertes).
%
Les fichiers qui nous ont été transmis ne contiennent cependant pas
l'entièreté des discussions entre requérants et secouristes. D'une
part car, ils ont été montés, de sorte à censurer les informations
personnelles (\eg identité de la victime, description des
blessures). De plus ces enregistrements sont réalisés au niveau de la
ligne téléphonique des \ac{pghm} et non des \ac{codis}, généralement
contactés par les requérants (cf. \autoref{chap:02}). Ainsi nous ne
disposons (généralement) que des discussions traitant de la
localisation des victimes.

Ces dernières nous ont été transmises en deux corpus, un premier de 20
alertes en 2017 et un second de 32 alertes en 2019.

dont la durée varie de quelques secondes à plusieurs dizaines de
minutes (\autoref{fig:dist_temps_alertes}).

\begin{figure}
  \centering
   \begin{tikzpicture}
   \begin{axis}
     \addplot [boxplot] table [y index=0] {./Corps/Partie2/Chapitre5/figures/duration.dat};
     
  %\addplot [thick] gnuplot [raw gnuplot] {plot './Corps/Partie2/Chapitre5/figures/duration.dat' smooth kdensity};
\end{axis}
\end{tikzpicture}

  \caption{Distribution de la durée des alertes}
  \label{fig:dist_temps_alertes}
\end{figure}

\subsection{Template de retranscription}

Pour exploiter les informations contenues dans les enregistrements il
a été nécessaire de les retranscrire afin de les analyser.
%
Notre parti-pris n'a pas été de travailler à partir d'une
retranscription libre de ces alertes, mais à l'aide d'un tableau
prédéfini, un \emph{template de retranscription}
(\autoref{tab:struct_temp}) dont la structure reprend celle des
\emph{indices de localisation} telle que nous l'avons définie
(cf. \autoref{chap:04}).

La version ici décrite ne 

Ce modèle de transcription se présente sous la forme d'un fichier
tabulaire (plus précisément une feuille de calcul) scindé en trois
onglets :
%
\begin{enumerate*}[label=(\alph*)]
\item \emph{métadonnées},
\item \emph{interprétation des expressions} et
\item \emph{objets de référence}
\end{enumerate*}
%
Le premier d'entre-eux est destiné à spécifier les métadonnées de
l'alerte retranscrite, comme la date et l'heure de l'appel, l'identité
du requérant (\eg victime, témoin, \emph{etc.}) ou l’activité
pratiquée lors de l'accident.

C'est dans le second onglet que sont saisies la plupart des
informations extraites des alertes et l’entièreté de celles utilisées
pour la construction de l'ontologie des \emph{relations de
  localisation.} L'onglet \emph{interprétation des expressions} est
composé de 15 colonnes (\autoref{tab:struct_temp}), que l'on peut
regrouper en trois catégories :
%
\begin{enumerate*}[label=(\alph*)]
\item \emph{extrait},
\item \emph{contexte} et
\item \emph{expression}
\end{enumerate*}
%
La première d'entre-elles regroupe les colonnes décrivant l'extrait
audio traité.

La seconde regroupe les colonnes permettant de décrire le contexte de
chaque expression. On y retrouve la colonne \emph{locuteur,}
renseignant sur la personne ayant prononcé l'extrait étudié (\eg
secouriste, requérant) et la colonne confiance, permettant au
transcripteur de préciser son jugement sur la plausibilité de
l'information donnée par l'extrait. Si nous distinguons ces deux
colonnes des \emph{métadonnées} de l'extrait (\ie le groupe de
colonnes précédent) c'est car ces dernières demandent une
interprétation de l'extrait, contrairement aux \emph{métadonnées.}

Enfin, la troisième catégorie, qui regroupe la majorité des colonnes,
permet l'interprétation et la saisie de chaque expression. On y
retrouve les colonnes \emph{verbe,} \emph{sujet,} \emph{relation de
  localisation} et \emph{objet de référence,} correspondant aux
éléments du même nom dans la formalisation des \emph{indices de
  localisation} et les \emph{modifieurs} qui y sont associés.
%
Les colonnes \enquote{\emph{objet de référence}} et
\enquote{\emph{modifieur de l'objet de référence}} ont la
particularité de pourvoir être doublées (voir plus si nécessaire) dans
le cas où la \emph{relation de localisation} le nécessiterait.


\begin{table}
  \centering
  \begin{tabular}{L{.3\textwidth}>{\footnotesize}p{.4\textwidth}}
  \toprule \multicolumn{1}{c}{\bfseries Colonne} &
  \multicolumn{1}{c}{\normalsize \bfseries Contenu} \\ \midrule
% Sémantique des relations spatiales
  \addlinespace
  Identifiant de l'extrait & dddd \\
  Identifiant de l'expression & dd \\
  Extrait & Verbatim de la phrase transcrite\\
  Confiance & Permet au transcripteur de saisir sa confiance en
              \emph{l'indice de localisation} (uniquement si la saisie
              est faite par un secouriste)\\
  Timestamp & Début de l'extrait dans le fichier audio source\\
  Locuteur & \\
  Verbe & Verbe utilisé dans \emph{l'indice de localisation}\\
  Modifieur du verbe & \\
  Sujet & Sujet de \emph{l'indice de localisation}\\
  Modifieur du sujet & \\
  \emph{Relation de localisation} & \\
  Modifieur de la relation de localisation & \\
  Objet de référence & Nom ou type de \emph{l'objet de référence}
                       (cette colonne peut être multipliée si la
                       \emph{relation de localisation} est bi ou n-aire)\\
  Modifieur de l'objet de référence & (cette colonne peut être multipliée si la
                                      \emph{relation de localisation} est bi ou n-aire)\\
  Commentaires & Champ permettant au transcripteur de commenter sa
                 saisie ou les indications données par le requérant
                 (erreurs potentielles, fautes de prononciation, \emph{etc.}) \\
  \bottomrule
\end{tabular}

  \caption{Structure du template de retranscription}
  \label{tab:struct_temp}
\end{table}

Une fois que le template a été défini

\subsection{Analyse des retranscriptions}

La retranscription des 52 alertes nous a permis d'identifier 374
expressions différentes.

%%% Local Variables:
%%% mode: latex
%%% TeX-master: "../../../../main"
%%% End:
