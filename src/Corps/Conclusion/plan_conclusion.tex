\section{Générale}

L'objectif de cette thèse était de développer une méthode permettant
la transformation d'une description de position en une zone, exprimée
dans un référentiel connu.


Comme nous l'avons montré dans les deux premiers chapitres de cette
thèse, de nombreuses



\section{Rappel des propositions}

\section{Perspectives}



\subsection{L'évaluation de la qualité des \ac{zlp}}

Comme l'étude du \emph{fil rouge} l'a montré (\autoref{chap:09}), la
\emph{zone de localisation probable} produite par notre méthode peut,
en fonction des indices de localisation, avoir des configurations très
différentes. Elle peut être d'un bloc, ou fragmentée. Dans cette
seconde situation,  e

\subsection{La sélection des seuils}


%%% Local Variables:
%%% mode: latex
%%% TeX-master: "../../main"
%%% End:
