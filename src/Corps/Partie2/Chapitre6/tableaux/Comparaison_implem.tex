\begin{tabular}{L{.3\textheight}L{.3\textheight}L{.3\textheight}} \toprule
\multicolumn{1}{c}{\bfseries Critère comparatif} &
\multicolumn{1}{c}{\bfseries \emph{Alpha-cuts}} & \multicolumn{1}{c}{\bfseries
Raster} \\ \midrule
% Sémantique des relations spatiales
  Précision de la modélisation & Identique à la moins précise des
                                 données. & Dépend de la résolution et
                                           de la moins précise des données.\\
  Difficulté de mise en place & Importante. Nécessite de nombreux
                                développements \emph{ad hoc} de bas
                                niveau & Peu importante. Il est assez
                                         rapide d'aboutir aux premières
                                         \emph{spatialisation.}\\
  %
  Difficulté d'extension & & \\
  %
  Temps de calcul & Assez faible pour l'exemple traité, risque
                    de varier fortement en fonction de la \emph{relation de
                    localisation} traitée. & Conséquent. Augmente quadratiquement avec la
                      résolution.\\
  %
  Volume des données & Assez faible. & Assez faible pour la comparaison, mais augmente rapidement avec le
                        nombre \emph{d'objets de référence} candidats.\\
  %
  Contraintes spécifiques &Nécessité de choisir le nombre et le degré
                            d'appartenance des \emph{alpha-cuts.} & Nécessité de définir une résolution de modélisation.\\
  \bottomrule
\end{tabular}
