\addsec{Résumé de la thèse}

L'objectif de cette thèse de doctorat est de définir et d'implémenter
une méthode permettant d'identifier la (ou les) position(s)
correspondant a une description de position, avec pour finalité
d'aider les secouristes en montagne à localiser des personnes perdues
ou blessées.

La première partie (\ref{part:01}) de cette thèse était dédiée à la
présentation du contexte applicatif, organisationnel et scientifique
de cette thèse. Nous y avons présenté le fonctionnement des secours en
montagne français, leur manière de procéder pour localiser des
personnes perdues en montagne et les limites de cette méthode
(\autoref{chap:1}). Nous avons également présenté de projet de
recherche CHOUCAS et les propositions qu'il porte pour aider les
\emph{unités de secours en montagne} dans leur tâche de
localisation. Puis, nous avons défini la problématique de notre thèse
et le contexte scientifique dans lequel elle s'inscrit, le
\emph{Geographic Information Retrieval} (\autoref{chap:02}). Enfin,
nous avons dressé un état de l'art sur les questions au centre de
notre travail de recherche, la modélisation de l'imprécision et des
relations de localisation (\autoref{chap:03}).

Dans la seconde partie (\ref{part:02}) de ce manuscrit, nous avons
défini une méthode permettant de transformer une description de
position en une zone de localisation compatible et de combiner ces
différentes zones afin d’identifier la position décrite par le
requérant lors d'une alerte (\autoref{chap:04}). Cette méthode est
basée sur un processus en trois phases : la décomposition des indices
de localisation (\autoref{chap:05}), la spatialisation de ces mêmes
indices (\autoref{chap:07}) en des zones de localisation floues
(\autoref{chap:06}) et la fusion de ces zones (\autoref{chap:08}).

Enfin, la dernière partie (\ref{part:03}) de cette thèse était
consacrée à l’application de notre méthode a des cas réels. À partir
d'enregistrements d'alertes passées, nous avons constitué un ensemble
d'indices de localisation, simulant l'action d'un secouriste durant la
phase d'alerte, puis nous les avons spatialisés et fusionnés, avant
d'en analyser les résultats (\autoref{chap:09}).

\addsec{Résumé des contributions}

La principale contribution de cette thèse est la définition et
l'implémentation d'une méthode de transformation des positions
exprimées dans un référentiel indirect en des positions exprimées dans
un référentiel direct. Le développement de cette méthode nous a
conduit à travailler à la résolution de plusieurs problèmes
scientifiques, tels que la prise en compte de l'imprécision des
\emph{indices de localisation} donnés par la victime, de la confiance
du secouriste en ces indices ou de leur fusion. Mais la question
centrale de ce travail est la spatialisation des descriptions de
positions. Pour y répondre, nous avons proposé une méthode que nous
avons voulue générique, tout en la calibrant spécifiquement pour notre
cas d’application, le secours en montagne.

Cette méthode a été structurée selon trois grandes phases, la
\emph{décomposition,} la \emph{spatialisation} et la \emph{fusion.}
Chacune d'entre elles répond à un objectif précis. La décomposition
traite les indices de localisation saisis par le secouriste de manière
à les rendre spatialisables, la spatialisation transforme ces indices
en des \emph{zones de localisation compatibles} et la fusion les
regroupe, jusqu'à aboutir à une seule zone, la \ac{zlp}, correspondant
au résultat de la spatialisation pour l'ensemble de l'alerte traitée.

Pour permettre la prise en compte de l'imprécision des indices de
localisation, nous avons construit notre méthode autour du cadre
théorique de la théorie des sous-ensembles flous
\autocite{Zadeh1965}. Ainsi, la phase de spatialisation crée des zones
de localisation floues, c'est-à-dire dotées de limites
progressives. On peut alors faire la distinction entre les positions
correspondant partiellement ou totalement à l'indice de localisation
spatialisé et donc modéliser leur imprécision. 

La confiance du secouriste traitant l'alerte en la véracité des
indices de localisation qu'il traite a quant à elle été modélisée à
l'aide de la théorie des possibilités \autocite{Zadeh1978}. Ce cadre
théorique nous permet d'attribuer un degré de confiance à chaque
indice de localisation et de le répercuter sur la zone de localisation
le spatialisant.

L'objectif du projet CHOUCAS n'étant pas de proposer des solutions se
substituant aux secouristes, nous avons conçu notre méthode de manière
a ce qu'elle s'intègre avec peu de contraintes dans le processus
métier de localisation des victimes. Note méthode est donc conçue pour
que les indices de localisation qu'elle spatialise soient saisis par
des humains et non interprétés à partir de l'appel téléphonique (ou
autre) à l'aide de techniques issues du domaine du traitement
automatisé des langues. Pour permettre la saisie des indices de
localisation et leur spatialisation, nous avons créé, à partir de
l'analyse d'un corpus d'alertes passées, une ontologie des
\emph{relations de localisation} \acp{orl} listant et définissant,
tous les concepts utilisés par les victimes et les secouristes pour
décrire leur position. Bien qu'elle s'inspire d'autres travaux,
notamment l'ontologie GUM-Space \autocite{Bateman2010}, l'ontologie
des relations de localisation présente la caractéristique d'être
spécifique a la description de positions en montagne, ce qui en fait
une production exploitable indépendamment de notre travail de thèse.

L'étude des relations de localisation préalable a la construction de
l'ontologie des relations de localisation, nous a permis de constater
l’existence de certaines récurrences sémantiques entre concepts. Nous
avons proposé d'exploiter ces récurrences, de manière à faciliter la
spatialisation des relations de localisations. Le principe de
décomposition consiste à subdiviser une relation de localisation en
plusieurs \emph{relations de localisation atomiques,} dont la
sémantique de la conjonction est équivalente à la sémantique de la
relation de localisation initiale. Les relations de localisation
atomiques décomposant une relation de localisation sont alors
spatialisables indépendamment, puis fusionnables, de manière à obtenir
la zone de localisation spatialisant la relation de localisation non
décomposée.

Le rôle de la phase de décomposition est, en partie, d'effectuer cette
décomposition des relations de localisation de manière à permettre
leur spatialisation et celui de la phase de fusion de les combiner, de
manière a obtenir la zone de localisation spatialisant la relation de
localisation non décomposée. La définition des relations de
localisation atomiques et les liens de décomposition les rattachant
aux relations de localisation sont définis dans l'ontologie des
relations de localisation atomiques. Pour pouvoir spatialiser les
relations de localisation atomiques nous avons chacun des concepts de
cette ontologie a un ensemble de concepts décrivant des méthodes qui
seront appliquées lors de la spatialisation.

La lecture et l’application de ces concepts est réalisée au cours de
la seconde phase, la spatialisation. Trois concepts différents,
correspondant à autant d'étapes de la spatialisation ont été
définis. Le \emph{rasteriser} transforme l'objet utilisé comme
référence en un raster. Le choix du \emph{rasteriser} permet de
sélectionner la partie de l'objet qui est utilisée comme référence
(centre, frontière, ensemble de l'objet). La seconde étape est celle
de la construction de la \emph{métrique}. Durant cette étape une
mesure permettant de quantifier la sémantique de la relation de
localisation spatialisée est calculée pour toutes les positions de la
zone étudiée à partir de l'objet de référence rasterisé. Cette
métrique est ensuite utilisé pour construire la zone de localisation
compatible lors de la \emph{fuzzyfication.} Un grand nombre de
\emph{rasterisers,} de \emph{métriques} et de \emph{fuzzyficateurs}
ont été définis. Certains sont très spécifiques et sont employés que
pour la spatialisation d'une relation de localisation particulière,
mais une grande partie sont génériques et partagés par plusieurs
relations de localisation. Ces derniers peuvent donc être réemployés
pour modéliser d'autre relations de localisation, voir pour en définir
des nouvelles, ce qui offre une grande flexibilité.

L'ensemble de la méthode a été implémenté dans un prototype, RUITOR,
qui nous a permis de confronter nos propositions théoriques a la
modélisations d'alertes réelles. Ce qui nous a permis de mettre en
évidence les qualités et les défauts de notre approche.

Les deux qualités majeures de notre méthode sont sa généricité et son
évolutivité. La généricité de la méthode est principalement permise
par le principe de décomposition et à la réalisation de la
spatialisation à l'aide de trois concepts spécifiques, le
\emph{rasterizer,} la \emph{métrique} et le \emph{fuzzyficateur.} La
possibilité de combiner ces trois éléments pour créer un grand nombre
de relations de localisation atomiques et la possibilité de combiner
ces mêmes relations de localisation atomiques permet la définition
d'un grand nombre de concepts. L'évolutivité de notre méthode est, en
partie, une conséquence de sa généricité. En effet, l'ajout d'un
nouveau concept, comme une \emph{métrique,} permet de définir de
nombreuses nouvelles relations de localisation atomiques, ces
dernières étant construites par la combinaison de
concepts. L'utilisation d'une  ontologie pour formaliser les liens de
décomposition permet également de faciliter l'évolution de notre
méthode. En effet, cette approche permet d'extraire la formalisation
de la spatialisation de l'implémentation, la rendant plus facilement
consultable et modifiable et facilite donc son évolution.

\addsec{Perspectives}

Comme l'ont montré les applications de note méthode a des cas réels
(\autoref{chap:09}), de nombreuses pistes d'évolutions sont
envisageables. On peut identifier trois grandes catégories
d'améliorations potentielles, celles concernant l'affinement de la
paramétrisation, celles permettant d'étendre notre méthode, lui
permettant de traiter plus finement des situations plus complexes et
celles qui ne pourront naitre que d'une confrontation de notre méthode
a l'expérience des secouristes.

\subsection*{Calibration de la méthode}

% Seuils de la confiance
Une première piste d'amélioration consiste à améliorer la
paramétrisation de notre méthode. En effet, dans la situation actuelle
certains paramètres, comme les seuils de confiance, ont été fixés
empiriquement.

C'est un problème qui se pose également pour le choix de la résolution
du calcul. Comme nous l'avons expliqué dans le \autoref{chap:06}, la
représentation raster des \ac{zlc}, implique de définir une résolution
de maille, qui impacte la précision du résultat, mais également le
temps de calcul et le volume du résultat. Pour les modélisations
présentées dans le \autoref{chap:09}, où nous n'étions pas contrains
par les temps de calcul, nous avons pris la décision de travailler à
l'aide de la résolution la plus fine possible. Toutefois cette
solution n'est pas perenne, notre méthode visant à être appliquée dans
des contextes où le temps peut être limité. Nous souhaitons donc
offrir aux secouristes la possibilité de paramétrer la résolution du
calcul. Pour ce faire deux solutions seraient envisageables, la
première serait de laisser le secouriste fixer manuellement la maille
du calcul, ce qui est déjà techniquement possible, mais non accessible
pour l'utilisateur. Une seconde solution serait d'adapter
automatiquement cette résolution en fonction de la zone étudiée, ce
qui libérerait les secouristes de ces considérations. Cependant la
mise en place de cette solution pose un autre problème, nous ne savons
pas quelle est la meilleure résolution de spatialisation. Il est en
effet possible de des résolutions plus faibles que celles que nous
avons utilisées dans nos essais (\eg 100, voire 250~m) soient
satisfaisantes pour déclencher une intervention, ce que nous ne
pouvons savoir à l'heure actuelle. Il est donc possible qu'une
spatialisation à une résolution fixe et peu élevée soit satisfaisante
pour la localisation d'appels. Il nous semble donc nécessaire de
confronter notre méthode a de nouveaux cas et a l'avis des secouristes
pour approfondir ces questions.

%% La sélection des seuils
Le problème majeur de la calibration de la méthode est la
paramétrisation des fuzzyficateurs. Comme nous l'avons montré dans le
\autoref{chap:09} cette paramétrisation doit être effectuée
manuellement, en fonction de considérations empiriques. Or, si
l'utilisation de fuzzyficateurs spécialisés comme
\onto[orla]{Sup\-Val\-0} ou \onto[orla]{Eq\-Val\-0} facilite
l'utilisation de certaines relations de localisation (\eg
\onto[orl]{Alt\-Inf} il est toujours nécessaire de fixer la valeur
d'un paramètre \(\delta\). Si cette tâche n'est que peu contraignante
pour l'exercice de modélisation effectué dans le \autoref{chap:09}, il
nous semble difficile de demander aux secouristes de paramétrer les
fuzzyficateurs durant la phase de secours.

Une première solution serait de définir un paramétrage par défaut des
fuzzyficateurs, mais cette solution n'est pas envisageable, un même
fuzzyficateur pouvant être couplé avec plusieurs métriques, qui ne
partagent donc pas leurs unités. On peut raffiner cette solution en
rattachant la paramétrisation à la relation de localisation et non
plus au fuzzyficateur. Pour donner un exemple, avec cette solution
toute utilisation de la relation de localisation
\onto[orl]{Sous\-Altitude} utilisera le même paramétrage implicite du
fuzzyficateur \onto[orla]{Inf\-Val\-0}, mais ce paramétrage pourra
être différent lorsque ce fuzzyficateur est appelé par une autre
relation de localisation, comme \onto[orla]{Dans\-Planimetrique}. Bien
qu'elle soit plus satisfaisante, cette seconde solution ne règle pas
tous les problèmes qui peuvent être posés par la question de la
paramétrisation. Peut-on, par exemple, considérer que la relation de
localisation \onto[orl]{Pres\-De} traduit le même ordre de grandeur
lorsqu'elle s'applique à une maison ou à une ville ? Des relations de
localisation comme \onto[orl]{Aux\-Alentours\-De} intègrent même cette
notion dans leur définition. Il nous semble donc indispensable de
prendre également en compte la taille, voir la saillance, de l'objet
de référence lors de la spatialisation. Les principes de décomposition
et de modélisation autonome (\autoref{chap:04}) permettent, en effet,
d'utiliser un paramétrage spécifique pour chaque objet de référence,
ce qui permettrait d'aboutir à une modélisation beaucoup plus fine.

La définition d'une méthode de paramétrisation des fuzzyficateurs
prenant en compte la nature et les caractéristiques (géométriques ou
non) des objets de référence pose toutefois de nombreux problèmes
scientifiques. Il est en effet nécessaire d'identifier les
caractéristiques des objets influent ces paramètres, d'identifier si
le type d'objet à une influence sur ces paramètres, \emph{etc.}

Une des pistes envisagées est de recourir a des méthodes
d'apprentissage, alimentées par des données recueillies lors
d'entretiens ou de sondages. On pourrait, par exemple, demander a un
panel de randonneurs comment ils décriraient une position donnée. Il a
même été envisagé d'organiser ce type d'études a l'aide d'un
environnement en réalité virtuelle, qui permettrait aux personnes
interrogées de décrire plusieurs positions. Toutes ces solutions ont
leurs propres limites, mais elles offrent des pistes de réflexion
intéressantes quant à la question de la paramétrisation du
fuzzyficateurs.

\subsection*{Extension de la méthode}

Différentes pistes d'amélioration de notre méthode ont été envisagées
au cours de notre travail. La plus avancée d'entre elles est la
décomposition des objets de référence.

Nous avons en effet remarqué que certains objets de référence utilisés
par les requérants étaient \enquote{composites}, c'est-à-dire qu'ils
sont une composition de plusieurs parties relativement
différentes. C'est par exemple le cas des lignes électriques, qui sont
composées de pylônes et d'un câble suspendu ou des lignes de
téléphérique, composées des mêmes éléments mais auxquels s'ajoutent
une station d'arrivée et de départ. Le problème que soulèvent ces
objets est que la sémantique ou le paramétrage des relations de
localisation qui s'y appliquent peut changer en fonction de la partie
de l'objet considérée. Par exemple pour une ligne de téléphérique il
parait raisonnable de considérer qu'une relation de proximité a un
rayon plus important si elle se rapporte aux bâtiments qu'aux pylônes
ou au câble. Dit autrement, on est plus vite loin d'un pylône que
d'une gare de téléphérique. Les parties de l'objet peuvent influer le
paramétrage des relations de localisation. Un exemple peut être celui
d'une route contenant des ponts et des tunnels. Si un requérant décrit
sa position en indiquant se trouver \enquote{sous une route} il est
pertinent d'adapter le concept à la composante de l'objet. On sera
\onto[orl]{Sous\-Proche\-De} une route mais
\onto[orl]{Sous\-Recouvert\-Par} un tunnel ou un pont. Les parties de
l'objet peuvent donc influencer le choix des concepts.

Cette piste, que nous avons
déjà explorée \autocite{Bunel2019a},

 %L'évaluation de la qualité des \ac{zlp}

Comme l'étude du \emph{fil rouge} l'a montré (\autoref{chap:09}), la
\emph{zone de localisation probable} produite par notre méthode peut,
en fonction des indices de localisation, avoir des configurations très
différentes. Elle peut être d'un bloc, ou fragmentée. Dans cette
seconde situation,  e


Différentes pistes ont longuement été explorées.

Une approche, fondée sur la théorie des fonctions de croyance
\autocite{Shafer1976} et inspirée des travaux de
\textcite{Olteanu2008},




%% Incomplétude de la base
Un autre point que nous souhaiterions aborder est la prise en compte
de l'incomplétude des données.

%% Objets de référence flous
Une dernière possibilité d'extension de ce travail, serait d'étendre
notre méthode de manière a ce qu'elle puisse traiter des objets de
référence imprécis.

Comme l'ont montré certains indices de localisation traités dans le
\autoref{chap:09}, la représentation géométrique de certains objets de
référence, notamment les formes du relief, peu être sommaire. Des
objets très entendus, comme les \emph{combes,} les \emph{pas} ou les
\emph{vallées} peuvent, par exemple, être représentés par des points.


La possibilité de travailler à partir d'objets de référence imprécis
offre également la possibilité d'utiliser les \ac{zlc} produites par
notre méthode comme objets de référence de nouveaux indices de
localisation.

\subsection*{Intégration contexte métier}

Une dernière perspective majeure pour ce travail est de travailler à
son intégration dans le processus métier de la localisation de victimes.

% Réduire / augmenter la précision

% Pertinance / Non pertinence de relations


% Ajout de relations
Nous pensons que cette confrontation impliquera également de nombreux
retours sur les relations de localisations définies dans l'ontologie
\ac{orl}.

Portée sonore

Passage ombre

antenne téléphonique

Une première perspective d'amélioration serait de proposer de
nouvelles relations de localisation adaptées aux besoins métier. Le
\autoref{chap:09} et plus particulièrement la modélisation du
\emph{fil rouge} nous a permis de constater que certains indices très
spécifiques comme \enquote{la victime entend des voitures} ou
\enquote{le téléphone de la victime est connectée à une antenne
  téléphonique donnée}, n'étaient pas correctement modélisés avec les
concepts que nous avions développés.
%
Nous avons donc proposé de compléter l'ontologie des relations de
localisation spécifiquement dédiées a la modélisation de ce type
d'indices
%
La spatialisation de telles relations de localisation est également
assez 


%%% Local Variables:
%%% mode: latex
%%% TeX-master: "../../main"
%%% End:
