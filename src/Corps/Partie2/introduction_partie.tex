La seconde partie de cette thèse de doctorat se destine à présenter et
à expliciter la méthode que nous avons définie pour passer d'une
description de position en une position exprimée dans un référentiel
direct. Pour ce faire, nous commencerons par présenter les choix
généraux qui nous ont guidés dans l'élaboration de notre méthode,
avant d'en décrire l'organisation globale. Puis, nous détaillerons
chaque aspect de la méthode. Dans le premier chapitre de cette partie,
nous présenterons nos choix de modélisation et l'organisation globale
de notre méthode (\autoref{chap:04}). Le \autoref{chap:05} est, quant
à lui, dédié à la présentation de la première phase de la méthode, la
\emph{décomposition.} Le \autoref{chap:06} présente la manière dont
nous représentons des objet géographiques \emph{imprécis.} Le
\autoref{chap:07} détaille la seconde phase de notre méthode, la
\emph{spatialisation.} Enfin, dans le \autoref{chap:08}, nous
présenterons la phase de \emph{fusion,} dernière étape de notre
méthode de transformation d'une description de position en une
position exprimée dans un référentiel direct.
