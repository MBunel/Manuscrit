% Citation début de chapitre
\dictum[Loi de Hofstadter]{It always takes longer than you expect, even when
  you take into account Hofstadter's Law}%

\chaptertoc{}

L'objectif de ce chapitre est de présenter l'implémentation des
méthodes présentées précédement (voir chapitres \ref{chap:04},
\ref{chap:05}, \ref{chap:06}, \ref{chap:07} et
\ref{chap:08}). Cependant contrairement aux précédents chapitres nous
allons adopter une démarche calquée sur le fonctionnement interne du
code et non sur des aspects plus thématiques.

\addsec{Introduction}

L'objectif de ce chapitre est de présenter l'implémentation des
méthodes présentées précédement (voir chapitres \ref{chap:04},
\ref{chap:05}, \ref{chap:06}, \ref{chap:07} et
\ref{chap:08}). Cependant contrairement aux précédents chapitres nous
allons adopter une démarche calquée sur le fonctionnement interne du
code et non sur des aspects plus thématiques.

\section{Analyse des indices}
\label{sec:9-1}

\subsection{Extraction des indices}
\label{subsec:9-1-1}

\subsection{Décomposition des indices}
\label{subsec:9-1-2}

\section{Spatialisation}
\label{sec:9-2}

\subsection{Rasterisation des géométries}
\label{subsec:9-2-1}

\subsection{Construction des métriques}
\label{subsec:9-2-2}

\subsection{Sélection floue}
\label{subsec:9-2-3}

\section{Agrégation}
\label{sec:9-3}

\subsection{Agrégation des RSA}
\label{subsec:9-3-1}

\subsection{Agrégation des objets}
\label{subsec:9-3-2}

\subsection{Agégation des indices}
\label{subsec:9-3-3}

\addsec{Conclusion}

Conclusion chapitre 9

%%% Local Variables:
%%% mode: latex
%%% TeX-master: "../../../main"
%%% End:
