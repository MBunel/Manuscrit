Section 1

\subsection{Généralités}

L'implémentation de la méthode proposée a été réalisée à l'aide du
langage de programmation python, dans sa version 3.5, de septembre
2015. Le choix de ce langage de programmation a été réalisé, d'une
part en fonction de nos compétences personnelles, mais également car
il s'agit d'un langage de programmation répandu dans le monde de la
géomatique et disposant de nombreux bibliothèques facilitant le
développement.

Plusieurs cadres étaient possibles pour mettre en place

Une première solution est d'employer le framework d'un logiciel SIG,
comme QGis, ArcGis, ou d'autres, ce qui permet d’accéder aux fonctions
internes au logiciel, ce qui, dans notre cas, peut faciliter le
développement des fonctions de spatialisation. Cette solution nous
rend cependant dépendant du logiciel de SIG utilisé, ce qui a deux
défauts majeurs. D'une part il est nécessaire de se familiariser avec
les fonctions spécifiques du logiciel, ce qui ajoute un coup de
développement et d'appropriation du code supplémentaire. De plus cela
ajoute une dépendance forte au code, ce qui est un frein
supplémentaire.

Nous avons donc choisi d'utiliser une solution totalement indépendante
des logiciels de SIG, uniquement fondée sur des bibliothèques de bas
niveau.



\subsection{Détail des trois phases}

Les trois phases de notre méthode traitant d'objets différents (\ie
des \emph{indices de localisation} dans la phase de décomposition et
des \ac{zlc} dans la phase de spatialisation et de fusion), il en
résulte que les bibliothèques qui 




%%% Local Variables:
%%% mode: latex
%%% TeX-master: "../../../../main"
%%% End:
