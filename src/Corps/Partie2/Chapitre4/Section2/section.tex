En vue de la définition de la méthode de transformation d'une
description de position en une \emph{zone de localisation probable}
(\autoref{sec:4-3}) nous allons formaliser le concept \emph{d'indice
  de localisation}. Comme nous l'expliquions dans le
\autoref{Chap:02}, les \emph{indices de localisation} sont l'élément
de base des alertes. Chacun d'entre-eux décrit une po


Une alerte 
Ils contiennent toutes les informations données par le requérant pour
décrire sa positon

Comme nous l'avons déjà indiqué la description orale d'une position
nécessite trois éléments distincts, un \emph{sujet} qui est l'objet ou
la personne dont la position est décrite, un \emph{objet de référence}
qui est l'objet géographique servant de point de référence et une
\emph{relation de localisation} décrivant la manière dont le
\emph{sujet} est localisé par rapport à \emph{l'objet de référence.}
Un \emph{indice de localisation} (\(i\)) est donc formalisable en un
triplet de la forme :
%
\begin{equation}
  i = (S, Rl, Or)
\end{equation}
%
Avec \(S\) le \emph{sujet,} \(Rl\) la \emph{relation de localisation}
et \(Or\) \emph{l'objet de référence.}

Cette définition est cependant assez restrictive, puisqu'elle ne
permet pas d'exprimer certaines des configurations, plus complexes,
notamment celles induites par le \emph{principe de décomposition.}
Comme nous l'avons expliqué lors de la présentation du \emph{principe
  de décomposition} derrière une \emph{relation de localisation}
manipulée par l'utilisateur peuvent se cacher plusieurs
\emph{relations de localisations atomiques,} chacune correspondant à
une part de la sémantique de la \emph{relation de localisation}
initiale. La \emph{relation} \(Rl\) n'est donc pas un objet
individuel, mais un ensemble (contenant au moins un élément) de
\emph{relations de localisations atomiques} (\(Rla\)). \(Rl\) est donc
définissable de la manière suivante :

\begin{equation}
  Rl = \{Rla_1, Rla_2, \ldots, Rla_n\}
\end{equation}

Avec \(Rl\) la \emph{relation de localisation} et \(Rla_i\) des
\emph{relations de localisation atomiques.}

Tout d'abord, il est trop restrictif de considérer qu'un \emph{indice
  de localisation} ne peut contenir qu'un seul \emph{objet de
  référence}

On peut donc remplacer \(\) par un ensemble \emph{d'objets de
  référence,} la définition de \emph{l'indice de localisation} est
alors :

\begin{equation}
  i = (S, Rl, \{Or_1, Or_2, \ldots, Or_n\})
\end{equation}

Cependant cette définition n'est pas non plus satisfaisante.

Classe d'équivalence


D'autres modifications sont cependant à apporter.

Le \emph{principe de décomposition des relations de localisation,}
implique que les \emph{relations de localisation} soient des
compositions de \emph{relations de localisation atomiques.}



Finalement, on peut définir un \emph{indice de localisation} de la
manière suivante :

\begin{equation}
  i =
\end{equation}

%%% Local Variables:
%%% mode: latex
%%% TeX-master: "../../../../main"
%%% End:
