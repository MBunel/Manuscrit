La première partie de cette thèse de doctorat se destine à en
présenter le cadre applicatif, scientifique et théorique et à en
définir la problématique. Pour ce faire nous commencerons par
présenter la mission régalienne du secours en montagne et leur
organisation. Puis nous détaillerons les difficultés auxquelles les
\emph{unités de secours en montagne} \acp{usem} sont confrontées,
notament lors de la phase de localisation des victimes en
montagne. Nous présenterons également le projet de recherche dans
lequel notre travail s'inscrit, son origine et ses objectifs
scientifiques (\autoref{chap:01}). Le second chapitre se destine à
définir précisément notre sujet de recherche. Nous formaliserons la
problématique de notre thèse, en définirons les concepts essentiels et
présenterons les objectifs scientifiques de notre thèse et comment ils
s'inscrivent dans le contexte scientifique et dans le projet de
recherche Choucas (\autoref{chap:02}).  Enfin, le troisième et dernier
chapitre de cette partie dressera un état de l'art sur les deux
domaines scientifiques au centre de notre travail de thèse, la
modélisation de l'imprécision et la spatialisation des \emph{relations
  de localisation} (\autoref{chap:03}).

%%% Local Variables:
%%% mode: latex
%%% TeX-master: "../../main"
%%% End:
