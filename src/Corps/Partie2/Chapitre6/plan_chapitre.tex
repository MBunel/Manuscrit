% Citation début de chapitre
\dictum[Pierre Dac]{\enquote{La synthèse du monde peut se résumer en ces deux mots : oui et non}}%

\chaptertoc{}

\addsec{Introduction}

Introduction chapitre 6

Modélisation effective des concepts proposés
- Impact :
-- Nécessité de rasteriser les objets de référence
-- Nécessité de définir une résolution


\section{Comparaison des modèles d'objets imprécis}

Choix du flou ?

\subsection{Le choix du flou}

\tdi{Les modèles discrets}

exacts, ensembles épais (autre ?)

Trop simples, pas de possibilité de modèliser des variations
importantes de degré d'appartenance. \textbf{Variation non linéaires
(importantes avec le relief})

\missingfigure{comparaison décroissance degré d'appartenace en
  fonction de la pente. Voir si possible de le faire avec une arte
  topo et un vrai relief. Illustrer l'hypothèse implicite d'évolution
  linéaire entre les frontières dans les modèles exacts.}

\tdi{Modèles non linéaires et non flous}

Pourquoi le flou ? En vrai parceque dans le titre de la thèse c'était
marqué ``modélisation floue'' et je l'ai pris au premier degré. Mais
il faut autre chose.


\subsection{Modélisation par Alpha-cuts}

\texttt{citer de Runz, zoglami}

\subsection{Modélisation par rasters}

\texttt{citer Bloch, Vanegass, takemura}

\subsection{Comparaison des approches et choix approche raster}

\paragraph{Comparaison des approches}

\tdi{Difficulté de mise en place}

Au moment de la sélection d'une méthode de modélisation nous avons
estimé que le développement d'une méthode utilisant des vecteurs
serait bien plus coutueuses. Sur ce point nous avons préféré les
rasters. De plus il a été bc plus difficile d'aboutir à une
modélisation vectorielle permettant la comparaison (citer RIG).

\tdi{Difficulté d'extension}

Il semble plus difficile de développer des méthodes de spatialisation
très différentes en vecteur qu'en raster.

Il est également plus compliqué de tester de nouveaux couples
d'opérateurs flous en vecteur.

\tdi{Temps de calcul}

% Critère moyennement pertinent
Le critère du temps de calcul n'est pas fondamental pour ce
travail. Toutefois il nous semble pertinent de préciser que les deux
méthodes différent sur ce point.

Avantage raster si le nombre de cellules n'est pas trop
important. Mais le temps de calcul augmente rapidement. L'approche
vecteur est moins rapide mais le code est moins (et plus
difficilement) optimisé. Le résultat est assez mitigé.

\tdi{Volume des données}

Avantage vecteur. Le raster tend à prendre beaucoup de place, encore
plus si l'on souhaite stoquer les objets de référence en raster ou
sauvegarder des résutlats intermédiaires.

\tdi{Qualité de la modélisation}

Comparable, l'article de la RIG n'a pas permis de mettre en évidence
des différences notables. Par construction l'approche vectorielle est
plus précise, mais sa précision dépend aussi des données (MNT pour
l'altitude). À l'échelle du pixel les résultats sont comparables.

\paragraph{Choix du raster}



\section{XX}


\subsection{Comment on modélise tout ce dont on a parlé ?}

\tdi{introduction visualisations}

\tdi{Présenter des cas théoriques de fusion avec des rasters flous}

\texttt{Hors sujet ??}

\missingfigure{Illustration du processus d'intersection et d'union des
  rasters}



\addsec{Conclusion}

Conclusion chapitre 6

%%% Local Variables:
%%% mode: latex
%%% TeX-master: "../../../main"
%%% End:
