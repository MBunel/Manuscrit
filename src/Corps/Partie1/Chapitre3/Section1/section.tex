
\subsection{Éléments généraux sur la localisation par référencement
  indirect et formalisation d'une relation de localisation}

% Mise au point vocabulaire
Comme nous l'expliquions lors de la présentation du concept
\emph{d'élément de localisation} (cf. \ref{}), ce dernier est
composé de trois éléments :
%
\begin{enumerate*}[label=(\alph*)]
\item \label{i:site} un \emph{sujet,}
\item \label{i:cible} un \emph{objet de référence} et
\item une \emph{relation de localisation,}
\end{enumerate*}
%
qui, combinés, permettent d'exprimer un \emph{référencement indirect.} 
Nous n'avons cependant pas pris le temps de détailler ces trois
concepts, dont la compréhension est nécessaire au développement de
méthodes de \emph{spatialisation.}

\subsubsection{Les concepts de \emph{sujet} et \emph{d'objet de référence}}

Les concepts de \emph{sujet} et \emph{d'objet de référence} sont
présents dans toutes les formalisations des \emph{référencements
  spatiaux indirects.} Tous les auteurs considèrent que ce type de
description nécessite deux objets spatiaux, dont la position de l'un
(le \emph{sujet}) est décrite par rapport à la position de l'autre
(\emph{l'objet de référence}), qui est supposée connue.

Toutefois, le vocabulaire utilisé pour décrire ces concepts ne fait
pas l'objet d'un consensus, comme le montre la synthèse proposée par
\textcite{RetzSchmidt1988}.  Les travaux francophones se rattachant au
domaine de la linguistique, comme ceux de
\textcite{Vandeloise1986,Borillo1998, Aurnague1997, Mathet2000},
parlent de \emph{site} et de \emph{cible}, pour désigner,
respectivement l'\emph{objet de référence} et le \emph{sujet.} La
littérature anglophone parle quant à elle de \emph{primary object} et
de \emph{reference object} ou de \emph{figure} et de \emph{ground}.

\subsubsection{\emph{Relations spatiales} ou \emph{relations de
    localisation ?}}

Le terme de \enquote{relation spatiale}, massivement utilisé dans la
littérature 

\textcite{Duchene2019}, fait la distinction entre \emph{relations de
  localisation} et \emph{relations spatiales.}

Toutefois, ces expressions n'utilisant pas explicitement de
\emph{prépositions spatiales} sont considérés comme des
\emph{relations spatiales} par les linguistes comme
\textcite{Vandeloise1986}, leur sémantique les ramenant à des notions
spatiales.

C'est par exemple le cas de la notion \emph{d'intervisibilité,} dont
l'énoncé, de la forme : \enquote{Je vois un lac}, ne fait pas appel à
une \emph{préposition spatiale.}

% Relations ternaires
Par ailleurs, contrairement à ce que pourraient laisser penser les
différents exemples \emph{d'indices de localisation} présentés
jusqu'ici, ces derniers peuvent contenir plus d'un \emph{objet de
  référence} \autocite{Clementini2013}. Ce cas, relativement peu
fréquent, se présente lorsque la \emph{relation de localisation}
utilisée n'est pas \emph{binaire,} c'est-à-dire qu'elle n'implique pas
qu'un \emph{sujet} et qu'un \emph{objet de référence.} Le cas le plus
fréquent est celui de la \emph{relation de localisation}
\enquote{entre} qui implique, au minimum, deux \emph{objets de
  référence}. Si l'on peut, en effet, dire que \enquote{La poste est
  entre le café et la banque}, il est incorrect d'utiliser cette
\emph{relation de localisation} avec un seul \emph{objet de
  référence.} La phrase \enquote{Je suis entre l'université}, par
exemple, est fautive et incompréhensible. Cependant, touts les
\emph{référencements indirects} liant trois objets, ne sont pas
nécessairement composés d'une \emph{relation de localisation
  ternaire.} \textcite{Duchene2019} identifient deux cas de ce
type. D'une part l'un des objets peut être le support des deux autres,
c'est par exemple lorsqu'une position est décrite sur un itinéraire,
comme dans la phrase \enquote{le parking avant le pont}. De même, des
phrases comme \enquote{Il est à l'angle de la rue Saint-Jacques et le
  la rue Pierre et Marie Curie}. Dans ces deux cas on peut recombiner
l'énoncé en un ensemble \emph{d'indices de localisation} utilisant des
\emph{relations de localisation binaires.} Par exemple,
\enquote{l'angle de la rue Saint-Jacques et de la rue Pierre et Marie
  Curie} peut être considéré comme un seul \emph{objet de référence,}
cette description de position peut s'interpréter comme deux indices de
localisation : \enquote{Je suis à $\phi$ et $\phi$ est à
  l'intersection de la rue Saint-Jacques et de la rue Pierre et Marie
  Curie}. Ces exemples n’intègrent donc pas de \emph{relations de
  localisation ternaires.} À l'inverse, des phrases ne contenant qu'un
seul \emph{objet de référence} peuvent utiliser des \emph{relations de
  localisation ternaires.} On pourra par exemple dire : \enquote{La
  chèvre est entre les arbres}, dans ce cas, seul un \emph{objet de
  référence,} \enquote{les arbres}, est présent, mais celui-ci est
composite, \textcite{Aurnague1993} parlent alors de
\enquote{\emph{collection}}. 


\subsubsection{Contexte d'interprétation et cadre de référence}

Certaines \emph{relations de localisation} ne peuvent être
interprétées (et donc spatialisées) qu'avec une connaissance
supplémentaire sur la configuration de la position décrite. Par
exemple des \emph{relations de localisations} comme \enquote{devant}
ou \enquote{à gauche} ne peuvent être interprétées sans la
connaissance d'un \emph{référentiel de directions,} ou \emph{cadre de
  référence} \autocite{Duchene2019}.

Le \emph{cadre de référence} n'est pas nécessairement explicité dans
un énoncé.

Par exemple, dans la phrase \enquote{il est à ma gauche}, le cadre de
\emph{référence} est explicité. L'article définit \enquote{ma} indique
que le locuteur utilise sa ligne de vue comme référence.

\subsubsection{Modélisation formelle d'une relation}

Différents travaux ont cherché à proposer une représentation formelle
des descriptions de positions.

% 1
\autocite{Bateman2010}

\autocite{Pustejovsky2017}

\subsection{Classification des \emph{relations de localisation}}

\subsubsection{Critères de classification}

\subsubsection{Classifications trouvées dans la littérature}

\subsection{Les taxonomies et ontologies de relations spatiales
  existantes}

\subsection{Travaux sur différentes familles de relations}

\subsubsection{\emph{Relations de localisation} du point de vue géométrique et
  computationel}

\subsubsection{\emph{Relations de localisation} du point de vue sémantique}

\paragraph{Proximité}

\paragraph{Durées de déplacement}

Pour modéliser des 

% Naismith
Le plus ancien d'entre eux est le modèle de
\textcite{Naismith1892}. Comme l'indique \textcite{Duchene2019}, ce
modèle ne consiste qu'en une phrase, concluant la présentation d'une
sortie dans un journal d'alpinisme. Cette règle énonce qu'une bonne
estimation du temps de marche consisterait à compter une heure pour 3
miles (\SI{5}{\kilo\meter}) en distance planaire et d'y ajouter une
heure de marche par tranche de \SI{2000}{ft} (\SI{600}{\meter}) de
dénivelé, positif ou négatif \autocite{Naismith1892}.

Ce modèle sera corrigé plusieurs fois, 

% Aitken

% Langmuir 

% Tobler
\textcite{Tobler1993} proposera un modèle différent où la vitesse de
marche est exprimée en fonction de la pente seule. Bien que présenté
sous une forme analytique, ce modèle se base sur des données
empiriques provenant de \textcite{Imhof1950} \autocite{Tobler1993}.

\begin{equation}
  \label{eq:marche_tobler}
  V = g × 6 × e^{-3,5 × \left| S + 0,05 \right|}
\end{equation}

Avec \(V\) la vitesse de déplacement (en \si{\kilo\meter\per\hour}),
\(S\) la pente, dont l'expression est:

\begin{equation}
 S = \frac{Δh}{Δx} = \tan θ
\end{equation}

Où \(θ\) est l'angle de la pente, \(Δh\) le dénivelé et \(Δx\) la
distance planimétrique. Quant à \(g\), il s'agit d'un coefficient de
pondération, permettant de faire varier l'estimation de la vitesse en
fonction d'éléments exogènes, comme le mode de
déplacement. \textcite{Tobler1993} en propose 3 valeurs :
%
\begin{enumerate*}[label=(\alph*)]
\item \(g = 1\), le cas standard, pour estimer un temps de marche sur
  sentier,
\item \(g = 0,6\), lorsque la marche est faite hors-sentier et
\item  \(g = 1.25\), pour un déplacement à cheval.
\end{enumerate*}
%

% Modèle de Ress

% Colin
\textcite{Kerouanton2020} propose quant à lui un modèle distinguant
les phases d’ascension et de descente.


% Sentiers ou terrain ?

\begin{figure}
  \centering
  \pgfmathdeclarefunction{naismith}{1}{%
  \pgfmathparse{1 / ((1/#1) + tan(x) / 0.6)}%
}

\pgfmathdeclarefunction{langmuir1}{1}{%
  \pgfmathparse{(1 / ((1/#1) + abs(tan(x)) * (0.16 /
    0.3)))}%
}

\pgfmathdeclarefunction{langmuir2}{1}{%
  \pgfmathparse{(1 / ((1/#1) - abs(tan(x)) * (0.16 /
    0.3)))}%
}

\pgfmathdeclarefunction{tobler}{1}{%
  \pgfmathparse{#1 * 6 * exp(-3.5 * abs(tan(x) + 0.05))}%
}

\pgfmathdeclarefunction{ress}{0}{%
  \pgfmathparse{6*exp(-3.5*abs(x*0.05))}%
}

\begin{tikzpicture}
  \begin{axis}[
    width=12cm,
    height=7cm,
    grid=major,
    xlabel=Pente,
    ylabel=Vitesse de marche (\si{\kilo\meter\per\hour}),
    every axis plot post/.append style={
      mark=none,domain=-60:60,samples=500,smooth},
    ymax=12,
    axis x line=bottom,
    axis y line*=left,
    enlargelimits=upper,
    legend style={
      at={(0.5,-0.25)},
      anchor=north,
      legend columns=3,
      font=\small,
      draw=none},
    ]

    %\addplot[restrict x to domain={0:60}, thick] {naismith(5)};
    %\addlegendentry{\textcite{Naismith1892}}
    
    \addplot[restrict x to domain={0:60}, thick] {naismith(4)};
    %\addlegendentry{\textcite{Naismith1892}}

    \addplot[restrict x to domain={-60:-12}, thick] {langmuir1(5)};
    \addplot[restrict x to domain={-12:-5}, thick] {langmuir2(5)};
    \addplot[restrict x to domain={-5:0}, thick] {5};
    \addplot[restrict x to domain={0:60}, thick] {naismith(5)};

    %\draw[black] (axis cs:-12,2) -- (axis cs:-12,12);
    %\draw[black] (axis cs:-5,7) -- (axis cs:-5,5);
    
    \addplot[thick] {tobler(1)};
    %\addlegendentry{\textcite{Tobler1993}, \(g=1\)}
    
    \addplot[thick] {tobler(0.6)};
    %\addlegendentry{\textcite{Tobler1993}, \(g=0,6\)}
    
    %\addplot {ress};

  \end{axis}
\end{tikzpicture}
  \caption{Vitesse de marche estimée en fonction de la pente par les
    différents modèles de marche proposés dans la littérature.}
  \label{fig:modeles_marche}
\end{figure}

\paragraph{Les relations de visibilité}



%%% Local Variables:
%%% mode: latex
%%% TeX-master: "../../../../main"
%%% End:
