Nous avons identifié cinq objectifs principaux pour ce travail de
recherche.

On trouvera une synthèse des objectifs scientifiques, ainsi que de
leurs verrous et des contributions attendues dans le
\autoref{tab:synthese_objectifs} page
\pageref{tab:synthese_objectifs}.



\subsection{Modélisation de localisations indirectes}
\label{subsec:2-1-1}


\subsubsection{Contexte scientifique}

Vasardini
Bloch

\subsubsection{Verrous}

\paragraph{Disparité des descriptions de localisation}

\paragraph{Peu de données}

présentés en détail dans le \autoref{chap:05}

\paragraph{Faible recoupement}

Le second problème, qui découle du précédent, est que les données
dont nous disposons ne se recoupent pas.

Ces deux problèmes rendent impossibles les méthodes type
apprentissage.

\subsection{Prise en compte de l'imprécision}
\label{subsec:2-1-2}

\subsubsection{Contexte scientifique}

Travail sur le flou
Russell
Varzi
Zadeh
Dilo ?


\subsubsection{Verrous}

\subsection{Prise en compte du jugement du secouriste}
\label{subsec:2-1-3}

\subsubsection{Contexte scientifique}

Travaux sur l'incertitude

Ana-Maria

\subsubsection{Verrous}

\subsubsection{Apports envisagés}

\subsection{Agrégation d'indices}
\label{subsec:2-1-4}

\subsubsection{Contexte scientifique}

\subsubsection{Verrous}

\paragraph{Contexte différent}

\paragraph{Conflits}

\subsection{Évaluation des résulats}
\label{subsec:2-1-5}

\subsubsection{Contexte scientifique}

\subsubsection{Verrous}

\paragraph{Pas d'évaluation supervisée}


% Tableau synthétique
\begin{table}
  \centering
  \begin{tabular}{L{5cm}L{4.5cm}L{4.5cm}} \toprule
\multicolumn{1}{c}{\bfseries Objectifs scientifiques} &
\multicolumn{1}{c}{\bfseries Verrous} & \multicolumn{1}{c}{\bfseries
Apports envisagés} \\ \midrule
  
  \nameref{subsec:2-1-1}
{\par\footnotesize\hspace{.25cm}$\longrightarrow$~Chapitre
\ref{chap:5}} & \begin{minipage}{4cm}
    \begin{itemize}
    \item This is item 1
    \item This is item 2
    \end{itemize}
  \end{minipage}& \begin{minipage}{4cm} \bigskip
    \begin{itemize}
    \item This is item 1
    \item This is item 2
    \item This is item 3
    \item This is item 3
    \item This is item 3
    \end{itemize} \bigskip
  \end{minipage} \\
  
  \nameref{subsec:2-1-2}
{\par\footnotesize\hspace{.25cm}$\longrightarrow$~Chapitre
\ref{chap:5}} & \begin{minipage}{4cm}
    \begin{itemize}
    \item This is item 1
    \item This is item 2
    \item This is item 3
    \end{itemize}
  \end{minipage} & \begin{minipage}{4cm}
    \begin{itemize}
    \item This is item 1
    \item This is item 2
    \item This is item 3
    \end{itemize}
  \end{minipage} \\
  
  \nameref{subsec:2-1-3}
{\par\footnotesize\hspace{.25cm}$\longrightarrow$~Chapitre
\ref{chap:5}} & \begin{minipage}{4cm}
    \begin{itemize}
    \item This is item 1
    \item This is item 2
    \end{itemize}
  \end{minipage}& \begin{minipage}{4cm}
    \begin{itemize}
    \item This is item 1
    \item This is item 2
    \item This is item 3
    \end{itemize}
  \end{minipage} \\
  
   \nameref{subsec:2-1-4}
{\par\footnotesize\hspace{.25cm}$\longrightarrow$~Chapitre
\ref{chap:9}} & \begin{minipage}{4cm}
    \begin{itemize}
    \item This is item 1
    \item This is item 2
    \end{itemize}
  \end{minipage}& \begin{minipage}{4cm}
    \begin{itemize}
    \item This is item 1
    \item This is item 2
    \item This is item 3
    \end{itemize}
  \end{minipage} \\
  
  \nameref{subsec:2-1-5}
{\par\footnotesize\hspace{.25cm}$\longrightarrow$~Chapitre
\ref{chap:11}} & \begin{minipage}{4cm}
    \begin{itemize}
    \item This is item 1
    \item This is item 2
    \end{itemize}
  \end{minipage}& \begin{minipage}{4cm}
    \begin{itemize}
    \item This is item 1
    \item This is item 2
    \item This is item 3
    \end{itemize}
  \end{minipage}\\
  
  \bottomrule
\end{tabular}

  \caption{Synthèse des verrous et des apports attendus pour chaque
    objectif scientifique de la thèse}
  \label{tab:synthese_objectifs}
\end{table}

%%% Local Variables:
%%% mode: latex
%%% TeX-master: "../../../../main"
%%% End:
