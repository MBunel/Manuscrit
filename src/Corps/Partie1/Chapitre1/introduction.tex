Comme nous l'expliquions en introduction (\autoref{prt:int}), cette
thèse s'inscrit dans le cadre d'un projet de recherche
pluridisciplinaire dont l'objectif principal est de développer des
méthodes et des outils facilitant la localisation de personnes
blessées ou perdues en montagne. Notre travail de recherche s'inscrit
donc, à la fois dans le contexte métier du secours en montagne mais
également dans un contexte organisationnel, celui du projet de
recherche Choucas.

L'objectif de ce premier chapitre est de présenter ces deux contextes
et leurs spécificités pour permettre au lecteur d'appréhender au mieux
ce travail doctoral. La présentation du secours en montagne proposée
ici, a pour but de présenter son fonctionnement, son organisation
actuelle et l'historique de cette activité, mais surtout de décrire la
démarche de localisation des victimes utilisée par les secouristes,
question au cœur de ce travail de recherche. La présentation du projet
de recherche Choucas, insistera, quant à elle sur ses différents
objectifs scientifiques et leurs interactions. Ce qui nous permettra
de distinguer les questions qui serons abordées dans notre thèse de
celles qui ne le seront pas, car rattachées à d'autres objectifs de
recherche.

La première partie de ce chapitre se concentrera donc sur les la
présentation du secours en montagne et son organisation. Quant à la
seconde partie de ce chapitre elle abordera la question de
l'inscription de cette thèse au sein du projet de recherche Choucas.

%%% Local Variables:
%%% mode: latex
%%% TeX-master: "../../../main"
%%% End:
