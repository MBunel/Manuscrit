Comme nous l'avons vu dans le chapitre précédent cette thèse s'inscrit
dans un projet de recherche a visée applicative, dont l'objectif
principal est d’élaborer des solutions destinées à aider les
\emph{unités de secours en montagne} \acp{usem} et plus spécifiquement
le \ac{pghm} de Grenoble, à localiser des personnes perdues en
montagne. Parmi toutes les problématiques abordées dans le projet de
recherche \bsc{Choucas,} notre travail s'inscrit dans celle visant à
développer des méthodes et des outils permettant de passer d'une
description de position à une zone, représentée par des coordonnées
géographiques.

La présentation qui a été faite de cette problématique a été,
volontairement, assez évasive. Nous n'en avons pas présenté, ni le
détail, ni l'intégration de cette thèse au sein de cet objectif
scientifique. Ce chapitre se destine à clarifier ces points, notamment
en détaillant la problématique de cette thèse (\autoref{sec:2-1}) et en
présentant les différents objectifs scientifiques de ce travail
(\autoref{sec:2-2}).

%%% Local Variables:
%%% mode: latex
%%% TeX-master: "../../../main"
%%% End:
