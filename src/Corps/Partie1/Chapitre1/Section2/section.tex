Une des solutions explorées par les secouristes pour améliorer la
gestion de ces cas critiques a été de participer à la mise en place
d'un projet de recherche destiné à développer des solution innovantes
visant à faciliter la localisation de personnes perdues en montagne.

\subsection{Origine du projet}
\label{subsec:1-2-1}


Ici on présente l'origine du projet choucas, comment il s'est
construit, suite au fil rouge. 

\subsection{Problématique du projet}
\label{subsec:1-2-2}

Quelle est la problématique du projet, à quoi répond t-elle, etc.

\subsection{Objectifs scientifiques du projet}
\label{subsec:1-2-3}

Comment se structure le projet, quels sont les différents axes et les
questions sur lesquelles je ne suis pas amené à travailler.

\subsubsection{Structuration des données issues de sources textuelles
  hétérogènes}

\subsubsection{Raisonnement spatial qualitatif flou}

\subsubsection{Géovisualisation de données multidimensionnelles et
  imparfaites pour la prise de décision}

\subsubsection{Intégration de sources hétérogènes spatialisables}

\texttt{infographie structure projet}
%%% Local Variables:
%%% mode: latex
%%% TeX-master: "../../../../main"
%%% End:
