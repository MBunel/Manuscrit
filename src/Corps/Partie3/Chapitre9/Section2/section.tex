La première des alertes que nous proposons de traiter est l'alerte
dite du \emph{Grand Veymont,} issue de l'enregistrement d'un appel au
secours effectué XXXXX par un accompagnateur en montagne, au sujet
d'un de ses clients.

\subsection{Présentation de l'alerte}
\label{subsec:9-2-1}

Cette alerte est la plus courte de celles que nous présenterons ici,
elle est composée de 12 extraits, pour un total de 16 \emph{indices de
  localisation.}



Pas d'objets multiples

les indications données par le requérant sont assez précises et
détaillées, il est donc possible de définir une \emph{zone initiale de
  recherche} de petite taille. Nous avons défini une \ac{zir} de
\SI{25}{\kilo\meter\squared} (\autoref{fig:zir_grand_veyont}).

\begin{figure}
  \centering
  \input{./figures/ZIR_grand_veymont.tex}
  \caption{Zone initiale de recherche pour l'alerte \enquote{Grand Veymont}}
  \label{fig:zir_grand_veyont}
\end{figure}


\subsubsection{Retranscription et identification des indices de localisation}
\label{subsec:9-2-1-1}


Le verbatim de l'enregistrement audio de l'alerte \emph{Grand Veymont}
est donné dans la \autoref{anx:retrans-gv-verb}.
%
Comme nous l’expliquions dans le \autoref{chap:05}, 


Une partie des indices retranscrits ne sont cependant pas pertinents
pour notre cas d'appliqaution.

% Entre Grand Veymont et pas de la ville
% Je ne suis pas sur le sommet
% Je suis sous le grand Veymont
% Côté Sud/Nord

% Au dela

% not foret

% 800


\subsection{Modélisation de l'alerte}
\label{subsec:9-2-2}


\subsubsection{Décomposition des indices des indices de localisation}
\label{subsec:9-2-2-2}

\subsubsection{Spatialisation des indices de localisation décomposés}
\label{subsec:9-2-2-3}

\subsubsection{Fusion des zones de localisation compatibles}
\label{subsec:9-2-2-4}

À la suite de la spatialisation des différents indices de localisation
on dispose de plusieurs \ac{zlc}, qu'il est nécessaire de fusionner.

Seul un indice de localisation 


\subsection{Critique de la modélisation}
\label{subsec:9-2-3}

\tdi{Impact de la position du toponyme "pas de la ville" sur la modélisation}

%%% Local Variables:
%%% mode: latex
%%% TeX-master: "../../../../main"
%%% End:
