\subfloat[Exemple de décomposition de la relation topologique \emph{overlap}]{
  \begin{tabular}{R{2.5cm}C{4cm}C{4cm}}
    \toprule
    & \scriptsize \(\begin{bmatrix}
      V & V & V \\
      V & V & V \\
      \mathbf{V} & F & V \\
    \end{bmatrix}\) & \scriptsize  \(\begin{bmatrix}
      V & V & V \\
      V & V & V \\
      \mathbf{F} & F & V \\
    \end{bmatrix}\) \\
    \midrule
    \scriptsize \(\text{overlap} ≡ \begin{bmatrix}
      V & V \\
      V & V \\
    \end{bmatrix}\)
    & \tikz{
      \path[ffa] (0,0) circle [radius=15pt];
      \path[ffc] (0,0) circle [radius=15pt] node[color=RdBu-9-1] {A};
      \path[ffc, draw=RdBu-9-9,*-*,shorten >=-3pt,shorten <=-3pt]
      (0,15pt).. controls (0pt, 35pt) and (30pt, 30pt) .. (10pt,0)
      node [pos=.5, above right,color=RdBu-9-9]{B};
      }
        & \tikz{
          \path[ffa] (0,0) circle [radius=15pt];
          \path[ffc] (0,0) circle [radius=15pt] node[color=RdBu-9-1] {A};
          \path[ffc, draw=RdBu-9-9,*-*,shorten >=-3pt,shorten <=-3pt]
          (0,15pt) arc (90:0:15pt) -- (7.5pt,0) node [pos=.5, above right,color=RdBu-9-9]{B};
          } \\
    \bottomrule    
  \end{tabular}
  \label{tab:RCC8_vs_RCC23_1}
}

\subfloat[Exemple de décomposition de la relation topologique \emph{meet}]{
  \begin{tabular}{R{2.5cm}C{4cm}C{4cm}}
    \toprule
    &  \scriptsize \(\begin{bmatrix}
      F & F & V \\
      F & V & V \\
      F & \mathbf{V} & V \\
    \end{bmatrix}\) & \scriptsize  \(\begin{bmatrix}
      F & F & V \\
      F & V & V \\
      F & \mathbf{F} & V \\
    \end{bmatrix}\) \\
    \midrule
    \scriptsize \(\text{meet} ≡ \begin{bmatrix}
      F & F \\
      F & V \\
    \end{bmatrix}\)
    & \tikz{
      \path[ffa] (0,0) circle [radius=15pt];
      \path[ffc] (0,0) circle [radius=15pt] node[color=RdBu-9-1] {A};
      \path[ffc, draw=RdBu-9-9,*-*,shorten >=-3pt,shorten <=-3pt]
      (0,15pt).. controls (0pt, 35pt) and (10pt, 35pt) .. (20pt,0)
      node [pos=.5, right,color=RdBu-9-9]{B};
      }
        & \tikz{
          \path[ffa] (0,0) circle [radius=15pt];
          \path[ffc] (0,0) circle [radius=15pt] node[color=RdBu-9-1] {A};
          \path[ffc, draw=RdBu-9-9,*-*,shorten >=-3pt,shorten <=-3pt]
          (0,15pt).. controls (0pt, 35pt) and (30pt, 30pt) .. (15pt,0)
          node [pos=.5, above, color=RdBu-9-9]{B};
          } \\
    \bottomrule    
  \end{tabular}
  \label{tab:RCC8_vs_RCC23_2}
}

