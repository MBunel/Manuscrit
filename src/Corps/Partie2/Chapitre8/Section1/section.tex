La \emph{fusion} est la dernière phase de notre méthodologie
(\autoref{fig:methodo_1}). C'est au cours des trois étapes de cette
phase que les \emph{zones de localisation compatibles,} construites
durant la \emph{phase de spatialisation} (\autoref{chap:07}) sont
\emph{fusionnées,} de manière à obtenir une \emph{zone de localisation
  probable,} représentant le résultat de la méthodologie pour
\emph{l'ensemble des indices de localisation} traités. Comme la
\emph{phase de décomposition,} la \emph{phase de fusion} se compose de
trois étapes successives, regroupant les \emph{zones de localisation}
\emph{spatialisées} à partir des \emph{indices de localisation}
\emph{décomposés} au cours de la \emph{première phase} de la méthode
(\autoref{chap:04}).

La \emph{phase de fusion} est fondée sur l'idée qu'il existe une
équivalence entre les \emph{indices de localisation} et les \ac{zlc}
qui les spatialisent et qu'il est par conséquent possible de
\emph{spatialiser} la \ac{zlp} en \emph{fusionnant} les \ac{zlc}
\emph{spatialisant} les \emph{indices de localisation}
\emph{décomposant} \emph{l'ensemble des indices de localisation}
(\autoref{fig:methodo_1}).




Le degré d'appartenance d'un pixel $p$ donné à la \emph{zone de
  localisation probable} peut donc être calculé en \emph{fusionnant}
le degré d'appartenance de ce même pixel pour l'ensemble des
\emph{zones de localisation compatibles} crée lors de la
\emph{spatialisation.} Si l'on reprend l'exemple de la
\autoref{fig:methodo_fusion}, où huit \ac{zlc} sont regroupées deux à
deux jusqu'à l'obtention d'une \ac{zlp}, le degré d'appartenance d'un
pixel à la \ac{zlp} est :

\begin{equation}
 p_{\mathsf{ZLP}} = \left((p_{\mathsf{ZLC}_1} \wedge
p_{\mathsf{ZLC}_2}) \vee (p_{\mathsf{ZLC}_3} \wedge
p_{\mathsf{ZLC}_4})\right) \wedge \left((p_{\mathsf{ZLC}_5} \wedge
p_{\mathsf{ZLC}_6}) \vee (p_{\mathsf{ZLC}_7} \wedge
p_{\mathsf{ZLC}_8})\right)
\end{equation}

Les différentes \emph{étapes} de la \emph{phase de fusion} permettent
d'effectuer ce calcul sur l'ensemble des pixels utilisés pour
représenter la \ac{zir}.




Au terme de la phase de \emph{spatialisation} on dispose généralement
d'un grand nombre de \emph{zones de localisation compatibles}
\footnote{Des configurations où seule une \ac{zlc} est construite sont
  possibles, mais cela correspond à des cas particuliers où l'ensemble
  des \emph{indices de localisation} (\autoref{chap:04}) ne contient
  qu'un \emph{indice} lui-même employant une \emph{relation de
    localisation atomique} exprimée par rapport à un \emph{objet de
    référence} défini. Dans ce cas les phases de \emph{décomposition}
  et de \emph{fusion} n'ont pas de rôle à jouer.}, chacune contenant
une part de l'information nécessaire à la construction de la
\ac{zlp}. Le rôle de la \emph{phase de fusion} consiste simplement à
combiner toutes ces informations de manière à obtenir une seule
\emph{zone de localisation,} correspondant à la description de
position donnée par l'ensemble des \emph{indices de localisation.}






\subsection{Présentation des \emph{étapes} de la \emph{phase de
    fusion}}

Comme nous l'avons expliqué dans le \autoref{chap:04}, la \emph{phase
  de fusion} est composée de trois étapes :
%
\begin{enumerate*}[label=(\arabic*)]
\item la \emph{fusion} des \emph{relations de localisation atomiques,}
\item la \emph{fusion} des objets de \emph{référence indéfinis,} et
\item la \emph{fusion} des \emph{indices de localisation,}
\end{enumerate*}
%
chacune destinée à fusionner des \emph{zones de localisation
  compatibles} issues de la \emph{spatialisation} \emph{d'indices de
  localisation} décomposés à une étape particulière de la \emph{phase
  de décomposition} (\autoref{fig:methodo_1}).

\begin{figure}
  \centering
  \begin{tikzpicture}
  \matrix [matrix of nodes,
  anchor=west,
  nodes={minimum size=.75cm}
  ] (zla0) at (0,0)
  {
    \draw[ffa,ffc] (0,0) circle (.3cm);&
    \draw[ffa,ffc] (0,0) circle (.3cm);\\
  };

  \matrix [matrix of nodes,
  anchor=west,
  nodes={minimum size=.75cm},
  right=.75cm of zla0
  ] (zla1)
  {
    \draw[ffa,ffc] (0,0) circle (.3cm);&
    \draw[ffa,ffc] (0,0) circle (.3cm);\\
  };


  \matrix [matrix of nodes,
  anchor=west,
  nodes={minimum size=.75cm},
  right=.75cm of zla1
  ] (zla2)
  {
    \draw[ffa,ffc] (0,0) circle (.3cm);&
    \draw[ffa,ffc] (0,0) circle (.3cm);\\
  };

  \matrix [matrix of nodes,
  anchor=west,
  nodes={minimum size=.75cm},
  right=.75cm of zla2
  ] (zla3)
  {
    \draw[ffa,ffc] (0,0) circle (.3cm);&
    \draw[ffa,ffc] (0,0) circle (.3cm);\\
  };


  % Ligne 2
  \matrix [matrix of nodes,
  anchor=north,
  nodes={minimum size=.75cm},
  below=2cm of zla0
  ] (zlb0)
  {
    \draw[ffa,ffc] (0,0) circle (.3cm);\\
  };

  \matrix [matrix of nodes,
  anchor=north,
  nodes={minimum size=.75cm},
  below=2cm of zla1
  ] (zlb1)
  {
    \draw[ffa,ffc] (0,0) circle (.3cm);\\
  };

  \matrix [matrix of nodes,
  anchor=north,
  nodes={minimum size=.75cm},
  below=2cm of zla2
  ] (zlb2)
  {
    \draw[ffa,ffc] (0,0) circle (.3cm);\\
  };
  \matrix [matrix of nodes,
  anchor=north,
  nodes={minimum size=.75cm},
  below=2cm of zla3
  ] (zlb3)
  {
    \draw[ffa,ffc] (0,0) circle (.3cm);\\
  };

  % Ligne 3
  \coordinate (zlb01) at ($(zlb0)!0.5!(zlb1)$);
  \coordinate (zlb23) at ($(zlb2)!0.5!(zlb3)$);

  
  \matrix [matrix of nodes,
  anchor=north,
  nodes={minimum size=.75cm},
  below=2cm of zlb01
  ] (zlc0)
  {
    \draw[ffa,ffc] (0,0) circle (.3cm);\\
  };

  \matrix [matrix of nodes,
  anchor=north,
  nodes={minimum size=.75cm},
  below=2cm of zlb23
  ] (zlc1)
  {
    \draw[ffa,ffc] (0,0) circle (.3cm);\\
  };

  % Ligne 4
    \coordinate (zlc01) at ($(zlc0)!0.5!(zlc1)$);
  
  \matrix [matrix of nodes,
  anchor=north,
  nodes={minimum size=.75cm},
  below=2cm of zlc01
  ] (zlc1)
  {
    \draw[ffa,ffc] (0,0) circle (.3cm);\\
  };


  % Accolades
  \foreach \m in {0,1,...,3} {
    \draw (zla\m.south west) |- ($(zla\m.south west)!0.5!(zla\m.south
    east) + (0,-.1)$) -| (zla\m.south east)  node[pos=0, yshift=.2]
    (zla\m-g) {};
    % 
    % \draw (zlb\m.north west) |- ($(zlb\m.north west)!0.5!(zlb\m.north
    % east) + (0,.1)$) -| (zlb\m.north east)  node[pos=0, yshift=.2]
    % (zlb\m-g) {};
    % 
    % \draw[->>, black,] ([yshift=-.2cm]zla\m.south) -- ([yshift=.2cm]zlb\m.north);
  }


  % \node[fill=white,align=center, font=\large\sffamily] at ($(zla0.south west)!0.5!(zlb5.north
  % east)$){\emph{Fusion} des \emph{relations de localisation atomiques}};
\end{tikzpicture}
  \caption{Méthode de \emph{fusion} d'un ensemble \emph{d'indices de localisation}}
  \label{fig:methodo_fusion}
\end{figure}

La première de ces étapes est la \emph{fusion} des \emph{relations de
  localisation atomiques.} Comme nous l'expliquions lors de la
présentation de la méthodologie (\autoref{chap:04}) et plus
particulièrement lors de la définition de la \emph{phase de
  décomposition} (\autoref{chap:05}), les \emph{relations de
  localisation} utilisée dans les \emph{indices de localisation} (\eg
\onto[orl]{Pres\-De}, \onto[orl]{Sous\-Al\-ti\-tu\-de}, \emph{etc.})
peuvent êtres décomposées en des \emph{relations de localisation
  atomiques} représentant chacune une \enquote{brique sémantique} de
l'ensemble de la \emph{relation de localisation.} Au terme de la
troisième étape de la \emph{phase de décomposition} on dispose donc
d'un ensemble \emph{d'indices de localisation} décomposés, puis
\emph{spatialisés} indépendamment. La construction des \emph{zones de
  localisation compatibles} correspondant à un \emph{indice de
  localisation} contenant une \emph{relation de localisation}
décomposée nécessite donc de fusionner ces \ac{zlc} de manière à en
obtenir une seule, correspondant à la \emph{spatialisation} de
\emph{l'indice de localisation} avant décomposition.

Comme nous l'avons déjà expliqué (\autoref{chap:04}), cette fusion est
effectuée à l'aide d'une intersection entre les \emph{zones de
  localisation compatibles} issues de la \emph{spatialisation} de
\emph{l'indice de localisation décomposé.} Cette construction par
intersection, équivaut à indiquer que, pour qu'une position
appartienne a le \ac{zlc} construite à partir d'une \emph{relation de
  localisation non atomique} il est nécessaire qu'elle appartienne à
toutes les \ac{zlc} construites à partir des \emph{relations de
  localisation atomiques} la décomposant. Par exemple, pour qu'une
position appartienne à la \ac{zlc} spatialisant la \emph{relation de
  localisation} \onto[orl]{Sous\-Pro\-che\-De} \footnote{Qui est une
  \emph{relation de localisation non atomique} (\autoref{chap:04}).},
il est nécessaire qu'elle appartienne aux deux \ac{zlc} \footnote{Et
  par conséquent à leur intersection.} \emph{spatialisant} les
\emph{relations de localisation atomiques} \onto[oral]{Sous\-Altitude}
et \onto[orla]{Proximal}, utilisée pour la décomposition de
\onto[orl]{Sous\-Pro\-che\-De}.
%
On dispose donc à la fin de cette première étape de \emph{fusion}
d'autant de \ac{zlc} qu'il y a \emph{d'indices de localisation} avant
la décomposition des \emph{relations de localisation atomiques}
(\autoref{fig:methodo_1}).

La seconde étape de la fusion est la fusion des \emph{objets de
  référence indéfinis.} L'objectif de cette étape est de regrouper les
\ac{zlc} issues de la fusion des \emph{relations de localisation
  atomiques.} On dispose donc d'un ensemble de \emph{zones de
  localisation compatibles} définies à partir de plusieurs instances
d'un même type \emph{d'objet de référence.}


La dernière étape de la phase de fusion est la \emph{fusion des
  indices de localisation.}

\begin{table}
  \centering
   \begin{tabular}{r>{\small}p{.35\textwidth}>{\small}p{.35\textwidth}}
  \toprule & \multicolumn{1}{c}{\ac{orl}} &
  \multicolumn{1}{c}{\ac{orla}} \\ \midrule
  \addlinespace
  Objectif & Recense et définit les \emph{relations de localisation} utilisées
  pour décrire une position dans le contexte de la localisation de
  personnes en montagne & Définit les \emph{relations de localisation
                          atomiques,} la décomposition des relations
  définies dans \ac{orl} et formalise le processus de
                          \emph{spatialisation} des \emph{relations
                          spatiales atomiques.}\\
  Contenu & Définition de 51 \emph{relations de localisation,}
            regroupées en 11 classes abstraites. & Définition de XX
                                                   \emph{relations de
                                                   localisation
                                                   atomiques,}
                                                   décomposant XX des
                                                   51 \emph{relations
                                                   de localisation}
                                                   définies dans \ac{orl}.\\
%  Hiérarchie & Blo & \\
  Modélisation & L'ensemble des concepts sont définis pour être facilement
       différentiables et proches de la perception humaine des
       localisations dans l'espace & Les \emph{relations de
                                     localisation atomiques} ne sont
                                     pas conçues pour être manipulées
                                     directement par les utilisateurs.
  Elles généralement plus abstraites que les \emph{relations de
                                     localisation} qu'elles
                                     décomposent.\\ 
  \bottomrule
\end{tabular}

   \caption{Synthèse des étapes de la \emph{phase de fusion}}
  \label{tab:comparaison_etapes_fusion}
\end{table}

Comme nous l'avons expliqué dans le \autoref{chap:03}, plusieurs
opérateurs peuvent être utilisés pour réaliser des intersections et
des unions entre \emph{sous-ensembles flous.} Chaque opérateur
présente des propriétés qui lui sont propres et qui impactent le
résultat des opérations inter-ensemblistes. Dans notre cas, le choix
d'un opérateur influence donc la valeur des degrés d'appartenance et
par conséquent la forme de la \ac{zlp} et plus généralement de toutes
les \emph{zones de localisation} construites au cours de la
\emph{phase de fusion.} Le choix des opérateurs de \emph{fusion} doit
donc être étudié, de manière à identifier les opérateurs plus adaptés
à notre cas d'utilisation.

\subsection{Le choix des opérateurs de \emph{fusion}}

Dans l'état de l'art (\autoref{sec:3-2}), nous avons présenté les
quatre couples d'opérateurs ---~t-normes et \emph{t-conormes}~--- les
plus fréquemment employés pour étendre les unions et les intersections
ensemblistes à la théorie des sous-ensembles flous (\ie les opérateurs
de \textcite{Zadeh1965}, de \bsc{Łukasiewicz}, probabilistes et
drastiques). Les opérateurs que nous avons utilisés jusqu'ici,
notamment lors de la définition de la représentation des \emph{zones
  de localisation} (\autoref{chap:06}), étaient les opérateurs dit de
\bsc{Zadeh} \footnote{Voir les équations \ref{eq:norm_zadeh} et
  \ref{eq:conorm_zadeh} et la \autoref{fig:zadeh_op}, dans le
  \autoref{chap:03}.}, proposés lors de la formulation originale de la
théorie des sous-ensembles flous \autocite{Zadeh1965}. Les opérateurs
de \bsc{Zadeh} sont tout à fait appropriés pour effectuer les
intersections et les unions nécessaires aux différentes étapes de la
\emph{phase de fusion,} mais les opérateurs alternatifs (\ie les
opérateurs de \bsc{Łukasiewicz}, probabilistes et drastiques), qui
proposent des intersections et des unions plus ou moins sévères
(\autoref{chap:03}), offrent la possibilité d'affiner, toutes choses
égales par ailleurs, les résultats de la \emph{fusion.} Cependant,
l'utilisation d'opérateurs alternatifs soulève plusieurs
questions. Chaque couple d'opérateur a, en effet, des propriétés
spécifiques, impactant significativement le résultat des intersections
et des unions. Le choix des opérateurs utilisés durant \emph{la phase
  de fusion} ne doit donc pas se fonder uniquement sur leur sévérité,
mais également sur leurs différentes propriétés et leur(s) impact(s)
sur la sémantique des différentes étapes de la phase de \emph{fusion.}
Nous allons donc détailler les différentes propriétés des opérateurs
de \emph{fusion} et étudier leur influence sur la \emph{fusion} des
\ac{zlc}, en vue de choisir le couple d'opérateur le plus adapté à
notre méthode.

\subsubsection{Propriétés partagées}

Si chaque couple d'opérateur à des propriétés spécifiques, tous ont
les mêmes caractéristiques de base. Comme nous l'avons expliqué lors
de la présentation des différents couples d'opérateurs
(\autoref{chap:03}), chacun d'entre eux est composé d'une
\emph{t-norme,} qui est l'opérateur utilisé pour les intersections (et
les conjonctions) et une \emph{t-conorme,} utilisée pour les unions
(et les disjonctions). Toutes les \emph{t-normes} et les
\emph{t-conormes} partagent certaines caractéristiques, à savoir la
\emph{commutativité,} \emph{l'associativité} et la \emph{monotonie.}
De plus, toutes les \emph{t-normes} ont 0 pour \emph{élément neutre}
et toutes les \emph{t-conormes} 1 pour \emph{élément neutre}
(\autoref{chap:03}). Ces propriétés garantissent un comportement
similaire des différents opérateurs et certaines sont indispensables
au bon déroulement de la \emph{phase de fusion,} telle que formulée
dans le \autoref{chap:04}.

La \emph{commutativité} et \emph{l'associativité} garantissent, en
effet, que l’application d'une même \emph{t-(co)norme} à un ensemble
de degrés d'appartenance donnera le même résultat quel que soit
l'ordre des degrés d’appartenance. Appliqué à la \emph{fusion} des
\ac{zlc}, ces deux propriétés garantissent que la \emph{fusion} de
plusieurs \ac{zlc} donnera, toutes choses égales par ailleurs, le même
résultat. Les trois étapes de la \emph{phase de fusion} ne sont donc
pas impactées par l'ordre des \ac{zlc} qu'elles traitent.

La \emph{monotonie}


Enfin, les \emph{éléments neutres} garantissent qu'une position dont
le degré d'appartenance est nul sera toujours XX lors des unions et
qu'une position dont le degré d’appartenance est de 1 sera toujours XX
lors des unions inter-\ac{zlc}.


Une autre caractéristique des ces quatre couples est qu'ils sont
\emph{duals} \autocite{Bouchon-Meunier2007}, ce qui signifie que pour
tous \(x\) et \(y\) dont la valeur est comprise entre 0 et 1, \emph{la
  négation d'une \emph{t-norme}} est égale à la \emph{t-conorme} de la
négation de \(x\) et de \(y\) et inversement, soit :

\begin{equation}
  n\left(⊤(x,y)\right) = ⊥\left(n(x), n(y)\right)
\end{equation}

\begin{equation}
  n\left(⊥(x,y)\right) = ⊤\left(n(x), n(y)\right)
\end{equation}

Avec \(n\) la négation. Cette caractéristique

\tdi{Nécessité d'utiliser des opérateurs duals}


\tdi{parler de la durté des opérateurs}

\subsubsection{La continuité des opérateurs}

Si l'on observe les figures \ref{fig:zadeh_op}, \ref{fig:luka_op},
\ref{fig:prob_op} et \ref{fig:drast_op}, représentant, pour chaque
couple d'opérateur, la valeurs des \emph{t-normes} et des
\emph{t-conormes} (\(\mu\)) pour une paire de valeurs données
(\(x,y\)), on peut remarquer que les \emph{opérateurs drastiques} ont
un comportement particulier, sans transition \footnote{À l'exception
  du cas où, soit \(x\), soit \(y\) est nul.} entre les degrés
d’appartenance minimaux et maximaux. Ce couple d'opérateurs à la
particularité d'être discontinu. La discontinuité des
\emph{t-(co)normes}


\tdi{Enlever les opérateurs drastiques car ils sont trop sévères et ne
permettent pas de conserver des positions dont le degré d'appartenance
n'est pas au moins un fois de 1. En plus si j'ai n indices il faut
qu'il y ait n-1 indices dont le dégré est de 1.}


\subsubsection{Les principes du \emph{tiers-exclu} \& de la
  \emph{non-contradiction}}

Un second critère de distinction entre ces quatre opérateurs est la
conservation des principes du \emph{tiers-exclu} et de la
\emph{non-contradiction.} Ces deux principes XXX

Le principe de \emph{non-contradiction} postule qu'une même proposition
logique ne peut pas être vraie et fausse en même temps.


Le principe du \emph{tiers-exclu} postule qu'une même proposition
logique est nécessairement vraie ou fausse \footnote{Ce principe ne
  doit pas être confondu avec le principe de \emph{non-bivalence}
  (dont nous avons déjà discuté) à la base de la \emph{logique floue.}
  Une logique peut être non-bivalente, comme la \emph{logique floue,}
  tout en respectant le principe du \emph{tiers-exclu} (\eg la logique
  ternaire de \bsc{Łukasiewicz}).}.






Ainsi, lorsqu'un couple d'opérateurs est conforme à la loi de
\emph{non-contradiction,} l’application de sa \emph{t-norme} à un
degré appartenance quelconque (\(x\)) et à sa négation donne toujours un
degré d'appartenance nul : 

\begin{equation}
  \forall x, ⊤(x,1-x) = 0
\end{equation}

De même, si un couple d'opérateurs est conforme au principe du
\emph{tiers-exclu} alors l’application de sa \emph{t-conorme} à un
degré appartenance quelconque (\(x\))  et à sa négation soit
toujours égal à 1 :

\begin{equation}
  \forall x, ⊥(x,1-x) = 1
\end{equation}

Par conséquent, si la \emph{t-norme} utilisée valide le principe de
\emph{non-contradiction,} alors l'intersection d'un sous-ensemble flou
\(A\) (de \(X\)) avec son complément \(A^C\) (\autoref{eq:comp}) avec
ces opérateurs est toujours égale à l'ensemble vide :

\begin{equation}
  A \cap A^C = \emptyset
\end{equation}

et l'union de ces mêmes sous-ensembles avec une \emph{t-conorme}
respectant le principe du \emph{tiers-exclu} est toujours égale à
l'ensemble net \(X\):

\begin{equation}
  A \cup A^C = X
\end{equation}

Parmi les opérateurs étudiés seuls les opérateurs de \bsc{Łukasiewicz}
(et les opérateurs drastiques, déjà disqualifiés) valident les
principes de \emph{non-contradiction} et du \emph{tiers-exclu}
\autocite{Bouchon-Meunier2007}, comme l'illustrent les figures
\ref{fig:tiers-exclu} et \ref{fig:non-contradiction}. La
\autoref{fig:tiers-exclu} représente le résultat de l'union d'un sous
ensemble-flou \textcolor{RdBu-9-1}{\textsf{A}} avec son complémentaire
\textcolor{RdBu-9-9}{\textsf{A\up{C}}}, en fonction de l'opérateur
utilisé. On peut remarquer deux comportements distincts. Les
opérateurs drastiques et de \bsc{Łukasiewicz} produisent un résultat
homogène, où l'ensemble des pixels possèdent un degré d'appartenance
de 1, correspondant à \emph{l'ensemble de référence} \(X\)
(\autoref{eq:def_ensemble_flou}), alors que les opérateurs de
\bsc{Zadeh} ou probabilistes aboutissent à une zone non uniforme et
contenant des degrés d'appartenance inférieurs à 1, ils ne valident
donc pas le principe de \emph{non-contradiction.} De manière similaire
la \autoref{fig:non-contradiction} représente le résultat de
l'intersection d'un sous ensemble-flou
\textcolor{RdBu-9-1}{\textsf{A}} avec son complémentaire
\textcolor{RdBu-9-9}{\textsf{A\up{C}}}, en fonction de l'opérateur
utilisé. Les opérateurs drastiques et de \bsc{Łukasiewicz} produisent
un résultat toujours homogène, où l'ensemble des pixels possèdent un
degré d'appartenance nul, ce qui équivaut à l'ensemble vide. À
l'inverse, les opérateurs de \bsc{Zadeh} ou probabilistes aboutissent
à une zone non uniforme et contenant des degrés d'appartenance non
nuls, ce qui contredit le principe du \emph{tiers-exclu.}

\begin{figure}
  \centering
  \begin{tikzpicture}
  \def\decalageX{-.2}
  \def\decalageY{-.2}

  \newcommand{\calcangle}[4]{%
    % #1 nom de la varaible renvoyée
    % #2 coordonnées de l'objet de référence
    % #3 coordonnées du point visé
    % #4 coordonées de la position testée
    \pgfmathanglebetweenlines{#2}{#3}{#2}{#4}%
    \global\let#1\pgfmathresult%
  }

  % Arrow
  \begin{scope}
    \path[draw, -, shorten >=5pt, shorten <=5pt] (-1,0) -- (-5,-2);
    %\path[draw, ->,shorten >=5pt, shorten <=5pt] (8.25,3) -- (10,3);
    %\path[draw, -,shorten >=12pt, shorten <=5pt] (1,-3) |- (3.65,-1);
    %\path[draw, ->,shorten >=5pt, shorten <=5pt] (8.25,-1) -- (10,-1);
  \end{scope}
  % Fuzzyfication 1 (Sud)
  \begin{scope}[xshift=-2cm, local bounding box=fuzz]
    \begin{scope}
      \foreach \x in {0,.25,...,1.75}{ \foreach \y in {0,.25,...,1.75}
        {
          % Calcul de l'angle
          \calcangle{\angle}{\pgfpoint{1cm}{1cm}}{\pgfpoint{1cm}{0cm}}{\pgfpoint{\x cm}{\y cm}}
          % Fuzzyfication de l'angle
          \pgfmathsetmacro\fuzzy{%
            ifthenelse(\angle < 90,(90-\angle)/90,
            ifthenelse(\angle > 270,(\angle-270)/90,0)
            )
          }
          % Calcul du rayon à partir de la fuzzyfication
          \pgfmathsetmacro\radius{\fuzzy*2.5}
          \fill[RdBu-9-1] (\x+.125,.125+\y) circle (\radius pt);
        }
      }
      \path[ffc] (0,0) rectangle (2,2);
    \end{scope}
    \node[text width=3cm, align=center, anchor=north, font=\footnotesize] at (1,-.25)
    {\itshape Zone de localisation compatible \normalfont \textcolor{RdBu-9-1}{\textsf{A}}};
  \end{scope} 
  % Fuzzyfication 2
  \begin{scope}[xshift=2cm,local bounding box=fuzz2]
    \begin{scope}
      \foreach \x in {0,.25,...,1.75}{
        \foreach \y in {0,.25,...,1.75}
        {
          % Calcul de l'angle
          \calcangle{\angle}{\pgfpoint{1cm}{1cm}}{\pgfpoint{1cm}{0cm}}{\pgfpoint{\x cm}{\y cm}}
          % Fuzzyfication de l'angle
          \pgfmathsetmacro\fuzzy{%
            ifthenelse(\angle < 90,(90-\angle)/90,
            ifthenelse(\angle > 270,(\angle-270)/90,0)
            )
          }
          % Calcul du rayon à partir de la fuzzyfication
          \pgfmathsetmacro\radius{(1-\fuzzy)*2.5}
          \fill[RdBu-9-9] (\x+.125,.125+\y) circle (\radius pt);
        }
      }
      \path[ffc2] (0,0) rectangle (2,2);
    \end{scope}
    \node[text width=3cm, align=center, anchor=north, font=\footnotesize] at (1,-.25)
    {\itshape Zone de localisation compatible \normalfont \textcolor{RdBu-9-9}{\textsf{A\up{C}}}};
  \end{scope}
  % Fusion 1
  \begin{scope}[xshift=-6cm, yshift=-4cm,local bounding box=fus1]
    \begin{scope}
      \foreach \x in {0,.25,...,1.75}{
        \foreach \y in {0,.25,...,1.75}
        {
          % Calcul des angles
          \calcangle{\angle}{\pgfpoint{1cm}{1cm}}{\pgfpoint{1cm}{0cm}}{\pgfpoint{\x
              cm}{\y cm}}
          % Fuzzyfication des angles
          \pgfmathsetmacro\fuzzy{%
            ifthenelse(\angle < 90,(90-\angle)/90,
            ifthenelse(\angle > 270,(\angle-270)/90,0)
            )
          }          
          % fusion (zadeh)
          \pgfmathsetmacro\fus{max(\fuzzy, 1-\fuzzy)}          
          % Calcul du rayon à partir de la fuzzyfication
          \pgfmathsetmacro\radius{\fus*2.5}
          \fill[RdBu-9-9] (\x+.125,.125+\y) circle (\radius pt);
        }
      }
      \path[ffc2] (0,0) rectangle (2,2);
    \end{scope}
    \node[text width=3cm, align=center, anchor=north, font=\footnotesize] at (1,-.25)
    {\itshape Zone de localisation compatible \normalfont \textcolor{RdBu-9-9}{\textsf{B}}};
  \end{scope}
  % Fusion 2
  \begin{scope}[xshift=-2cm, yshift=-4cm,local bounding box=fus2]
    \begin{scope}
      \foreach \x in {0,.25,...,1.75}{
        \foreach \y in {0,.25,...,1.75}
        {
          % Calcul des angles
          \calcangle{\angle}{\pgfpoint{1cm}{1cm}}{\pgfpoint{1cm}{0cm}}{\pgfpoint{\x
              cm}{\y cm}}
          % Fuzzyfication des angles
          \pgfmathsetmacro\fuzzy{%
            ifthenelse(\angle < 90,(90-\angle)/90,
            ifthenelse(\angle > 270,(\angle-270)/90,0)
            )
          }
          % fusion (luka)
          \pgfmathsetmacro\fus{min(\fuzzy + (1-\fuzzy), 0)}          
          % Calcul du rayon à partir de la fuzzyfication
          \pgfmathsetmacro\radius{\fus*2.5}
          \fill[RdBu-9-9] (\x+.125,.125+\y) circle (\radius pt);
        }
      }
      \path[ffc2] (0,0) rectangle (2,2);
    \end{scope}
    \node[text width=3cm, align=center, anchor=north, font=\footnotesize] at (1,-.25)
    {\itshape Zone de localisation compatible \normalfont \textcolor{RdBu-9-9}{\textsf{B}}};
  \end{scope}
  % Fusion 3
  \begin{scope}[xshift=2cm, yshift=-4cm,local bounding box=fus3]
    \begin{scope}
      \foreach \x in {0,.25,...,1.75}{
        \foreach \y in {0,.25,...,1.75}
        {
          % Calcul des angles
          \calcangle{\angle}{\pgfpoint{1cm}{1cm}}{\pgfpoint{1cm}{0cm}}{\pgfpoint{\x
              cm}{\y cm}}
          % Fuzzyfication des angles
          \pgfmathsetmacro\fuzzy{%
            ifthenelse(\angle < 90,(90-\angle)/90,
            ifthenelse(\angle > 270,(\angle-270)/90,0)
            )
          }
          % fusion (zadeh)
          \pgfmathsetmacro\fus{(\fuzzy + (1-\fuzzy)) - \fuzzy * (1-\fuzzy)}          
          % Calcul du rayon à partir de la fuzzyfication
          \pgfmathsetmacro\radius{\fus*2.5}
          \fill[RdBu-9-9] (\x+.125,.125+\y) circle (\radius pt);
        }
      }
      \path[ffc2] (0,0) rectangle (2,2);
    \end{scope}
    \node[text width=3cm, align=center, anchor=north, font=\footnotesize] at (1,-.25)
    {\itshape Zone de localisation compatible \normalfont \textcolor{RdBu-9-9}{\textsf{B}}};
  \end{scope}
  % Fusion 4
  \begin{scope}[xshift=6cm, yshift=-4cm,local bounding box=fus4]
    \begin{scope}
      \foreach \x in {0,.25,...,1.75}{
        \foreach \y in {0,.25,...,1.75}
        {
          % Calcul des angles
          \calcangle{\angle}{\pgfpoint{1cm}{1cm}}{\pgfpoint{1cm}{0cm}}{\pgfpoint{\x
              cm}{\y cm}}
          % Fuzzyfication des angles
          \pgfmathsetmacro\fuzzy{%
            ifthenelse(\angle < 90,(90-\angle)/90,
            ifthenelse(\angle > 270,(\angle-270)/90,0)
            )
          }
          % fusion (drast)
          \pgfmathsetmacro\fus{%
            ifthenelse(\fuzzy < 0.001, (1-\fuzzy),
            ifthenelse((1-\fuzzy) < 0.001, \fuzzy, 1)
            )
          }          
          % Calcul du rayon à partir de la fuzzyfication
          \pgfmathsetmacro\radius{\fus*2.5}
          \fill[RdBu-9-9] (\x+.125,.125+\y) circle (\radius pt);
        }
      }
      \path[ffc2] (0,0) rectangle (2,2);
    \end{scope}
    \node[text width=3cm, align=center, anchor=north, font=\footnotesize] at (1,-.25)
    {\itshape Zone de localisation compatible \normalfont \textcolor{RdBu-9-9}{\textsf{B}}};
  \end{scope}
\end{tikzpicture}
  \caption[Illustration du principe du
  \emph{tiers-exclu}]{Illustration du principe du \emph{tiers-exclu.}
    Une \emph{zone de localisation compatible}
    (\textcolor{RdBu-9-1}{\textsf{A}}) et son complémentaire
    (\textcolor{RdBu-9-9}{\textsf{A\up{C}}}) sont intersectées, si le
    résultat de cette intersection est égal à \emph{l'ensemble de
      référence} (\ie que toutes les positions ont un degré
    d'appartenance de 1), alors la \emph{t-conorme} utilisée valide ce
    principe.}
  \label{fig:tiers-exclu}
\end{figure}

\begin{figure}
  \centering
  \begin{tikzpicture}
  % Arrow
  \begin{scope}
    \path[draw, -, shorten >=5pt, shorten <=5pt] (-1,0) -- (-5,-2);
    % \path[draw, ->,shorten >=5pt, shorten <=5pt] (8.25,3) -- (10,3);
    % \path[draw, -,shorten >=12pt, shorten <=5pt] (1,-3) |- (3.65,-1);
    % \path[draw, ->,shorten >=5pt, shorten <=5pt] (8.25,-1) -- (10,-1);
  \end{scope}
  % Fuzzyfication 1 (Sud)
  \begin{scope}[xshift=-2cm, local bounding box=fuzz]
    \begin{scope}
      \foreach \x in {0,.25,...,1.75}{ \foreach \y in {0,.25,...,1.75}
        {
          % Calcul et représentation floue distance euclidienne
          \pgfmathsetmacro\dist{sqrt((\x-0.125)^2+(\y-1.375)^2)}
          % Fuzzyfication de la distance
          \pgfmathsetmacro\fuzzy{%
            ifthenelse(\dist < .25,1,%
            ifthenelse(\dist < 2,-0.57*\dist+1.14,0)%
            )
          }
          % Calcul du rayon à partir de la fuzzyfication
          \pgfmathsetmacro\radius{\fuzzy*2.5}
          \fill[RdBu-9-1] (\x+.125,.125+\y) circle (\radius pt);
        }
      }
      \path[ffc] (0,0) rectangle (2,2);
    \end{scope}
    \node[text width=3cm, align=center, anchor=north, font=\footnotesize] at (1,-.25)
    {\itshape Zone de localisation compatible \normalfont \textcolor{RdBu-9-1}{\textsf{A}}};
  \end{scope} 
  % Fuzzyfication 2
  \begin{scope}[xshift=2cm,local bounding box=fuzz2]
    \begin{scope}
      \foreach \x in {0,.25,...,1.75}{
        \foreach \y in {0,.25,...,1.75}
        {
          % Calcul et représentation floue distance euclidienne
          \pgfmathsetmacro\dist{sqrt((\x-0.125)^2+(\y-1.375)^2)}
          % Fuzzyfication de la distance
          \pgfmathsetmacro\fuzzy{%
            ifthenelse(\dist < .25,1,%
            ifthenelse(\dist < 2,-0.57*\dist+1.14,0)%
            )
          }
          % Calcul du rayon à partir de la fuzzyfication
          \pgfmathsetmacro\radius{(1-\fuzzy)*2.5}
          \fill[RdBu-9-9] (\x+.125,.125+\y) circle (\radius pt);
        }
      }
      \path[ffc2] (0,0) rectangle (2,2);
    \end{scope}
    \node[text width=3cm, align=center, anchor=north, font=\footnotesize] at (1,-.25)
    {\itshape Zone de localisation compatible \normalfont \textcolor{RdBu-9-9}{\textsf{A\up{C}}}};
  \end{scope}
  % Fusion 1
  \begin{scope}[xshift=-6cm, yshift=-4cm,local bounding box=fus1]
    \begin{scope}
      \foreach \x in {0,.25,...,1.75}{
        \foreach \y in {0,.25,...,1.75}
        {
          % Calcul et représentation floue distance euclidienne
          \pgfmathsetmacro\dist{sqrt((\x-0.125)^2+(\y-1.375)^2)}
          % Fuzzyfication de la distance
          \pgfmathsetmacro\fuzzy{%
            ifthenelse(\dist < .25,1,%
            ifthenelse(\dist < 2,-0.57*\dist+1.14,0)%
            )
          }
          \pgfmathsetmacro\fus{min(\fuzzy, 1-\fuzzy)}          
          % Calcul du rayon à partir de la fuzzyfication
          \pgfmathsetmacro\radius{\fus*2.5}
          \fill[black] (\x+.125,.125+\y) circle (\radius pt);
        }
      }
      \path[ffc, black] (0,0) rectangle (2,2);
    \end{scope}
    \node[text width=3cm, align=center, anchor=north, font=\footnotesize] at (1,-.25)
    {\itshape Intersection avec la \emph{t-norme} de \bsc{Zadeh}};
  \end{scope}
  % Fusion 2
  \begin{scope}[xshift=-2cm, yshift=-4cm,local bounding box=fus2]
    \begin{scope}
      \foreach \x in {0,.25,...,1.75}{
        \foreach \y in {0,.25,...,1.75}
        {
          % Calcul et représentation floue distance euclidienne
          \pgfmathsetmacro\dist{sqrt((\x-0.125)^2+(\y-1.375)^2)}
          % Fuzzyfication de la distance
          \pgfmathsetmacro\fuzzy{%
            ifthenelse(\dist < .25,1,%
            ifthenelse(\dist < 2,-0.57*\dist+1.14,0)%
            )
          }
          \pgfmathsetmacro\fus{max(\fuzzy + (1-\fuzzy) -1, 0)}          
          % Calcul du rayon à partir de la fuzzyfication
          \pgfmathsetmacro\radius{\fus*2.5}
          \fill[black] (\x+.125,.125+\y) circle (\radius pt);
        }
      }
      \path[ffc, black] (0,0) rectangle (2,2);
    \end{scope}
    \node[text width=3cm, align=center, anchor=north, font=\footnotesize] at (1,-.25)
    {\itshape Intersection avec la \emph{t-norme} de \bsc{Łukasiewicz}};
  \end{scope}
  % Fusion 3
  \begin{scope}[xshift=2cm, yshift=-4cm,local bounding box=fus3]
    \begin{scope}
      \foreach \x in {0,.25,...,1.75}{
        \foreach \y in {0,.25,...,1.75}
        {
          % Calcul et représentation floue distance euclidienne
          \pgfmathsetmacro\dist{sqrt((\x-0.125)^2+(\y-1.375)^2)}
          % Fuzzyfication de la distance
          \pgfmathsetmacro\fuzzy{%
            ifthenelse(\dist < .25,1,%
            ifthenelse(\dist < 2,-0.57*\dist+1.14,0)%
            )
          }
          \pgfmathsetmacro\fus{\fuzzy * (1-\fuzzy)}          
          % Calcul du rayon à partir de la fuzzyfication
          \pgfmathsetmacro\radius{\fus*2.5}
          \fill[black] (\x+.125,.125+\y) circle (\radius pt);
        }
      }
      \path[ffc, black] (0,0) rectangle (2,2);
    \end{scope}
    \node[text width=3cm, align=center, anchor=north, font=\footnotesize] at (1,-.25)
    {\itshape Intersection avec la \emph{t-norme} probabiliste};
  \end{scope}
  % Fusion 4
  \begin{scope}[xshift=6cm, yshift=-4cm,local bounding box=fus4]
    \begin{scope}
      \foreach \x in {0,.25,...,1.75}{
        \foreach \y in {0,.25,...,1.75}
        {
          % Calcul et représentation floue distance euclidienne
          \pgfmathsetmacro\dist{sqrt((\x-0.125)^2+(\y-1.375)^2)}
          % Fuzzyfication de la distance
          \pgfmathsetmacro\fuzzy{%
            ifthenelse(\dist < .25,1,%
            ifthenelse(\dist < 2,-0.57*\dist+1.14,0)%
            )
          }
          % fusion (drast)
          \pgfmathsetmacro\fus{%
            ifthenelse(\fuzzy > 0.999, (1-\fuzzy),
            ifthenelse((1-\fuzzy)  >0.999, \fuzzy, 0)
            )
          }          
          % Calcul du rayon à partir de la fuzzyfication
          \pgfmathsetmacro\radius{\fus*2.5}
          \fill[black] (\x+.125,.125+\y) circle (\radius pt);
        }
      }
      \path[ffc, black] (0,0) rectangle (2,2);
    \end{scope}
    \node[text width=3cm, align=center, anchor=north, font=\footnotesize] at (1,-.25)
    {\itshape Intersection avec la \emph{t-norme} drastique};
  \end{scope}
\end{tikzpicture}
  \caption[Illustration du principe de la
  \emph{non-contradiction}]{Illustration du principe de
    la\emph{non-contradiction.}  Une \emph{zone de localisation
      compatible} (\textcolor{RdBu-9-1}{\textsf{A}}) et son
    complémentaire (\textcolor{RdBu-9-9}{\textsf{A\up{C}}}) sont
    unies, si le résultat de cette union est égal à \emph{l'ensemble
      vide} (\ie que toutes les positions ont un degré d'appartenance
    nul), alors la \emph{t-norme} utilisée valide ce principe.}
  \label{fig:non-contradiction}
\end{figure}

La mise en évidence de ces principes et de leur (non) validation ne
permet cependant pas de juger de la pertinence de ces principes pour
notre cas d’application. Il est donc nécessaire que nous questionnons
sur l'impact de ces deux principes sur la \emph{fusion} des \ac{zlc}
et sur la \ac{zlp}.




La validation des principes du \emph{tiers-exclus} et de
\emph{non-contradiction} n'est donc pas nécessairement une condition
\emph{sine qua non} à l'utilisation d'opérateurs. Il est même
préférable que ces deux principes ne soient pas validés.

\subsubsection{L'archimédianité et l’idempotence}

Une autre propriété différenciant ces différents opérateurs est
l'archimédianté. Un couple d'opérateurs est dit archimédien lorsque sa
\emph{t-norme} et sa \emph{t-conorme} sont continues et que, pour
toute valeur de degré d’apparence (\(x\)) entre 0 et 1 (exclus), la
\emph{t-norme} de \(x\) avec lui-même est strictement inférieure à
\(x\) et la \emph{t-conorme} de \(x\) avec lui-même est strictement
supérieure à \(x\), soit : \(\forall x \in ]0,1[\) : \(⊤(x,x) < x\) et
\(⊥(x,x) > x\) \autocite{Bouchon-Meunier1995}. De plus, si les
opérateurs sont archimédiens et que, pour tous les degrés
d'appartenance \(x\), \(y\), \(v\) et \(w\) tels que \(x\) et
strictement inférieur à \(v\) et \(y\) est strictement inférieur à
\(w\), la \emph{t-norme} du couple \(x,y\) est strictement inférieure
à \emph{la t-norme} du couple \(v, w\) et la \emph{t-conorme} du
couple \(x,y\) est strictement supérieure à \emph{la t-conorme} du
couple \(v, w\), soit : pour tout \(x < v\) et \(y < w\) :
\(⊤(x,y) < ⊤(v,w)\) et \(⊥(x,y) > ⊥(v,w)\), alors les opérateurs sont
dit archimédiens stricts \autocite{Bouchon-Meunier1995}.

% Qui est quoi ?
Si, les \emph{t-normes} (et les \emph{t-conormes}) peuvent être
archimédiennes, archimédiennes strictes ou non archimédiennes,
indépendamment de la \emph{t-conorme} (ou de la \emph{t-norme}) à
laquelle elles sont associées, ce n'est pas le cas pour les quatre
couples étudiés ici, pour lesquels \emph{t-normes} et
\emph{t-conormes} partagent les mêmes propriétés. La moitié des
couples que nous détaillons ici sont archimédiens. Tout d'abord, les
opérateurs drastiques ne sont pas archimédiens, car discontinus. Les
opérateurs de \textcite{Zadeh1965} ne sont pas non plus archimédiens,
puisque l’application de la \emph{t-norme} (́\(⊤_Z\)) ou de la
\emph{t-conorme} de \bsc{Zadeh} (\(⊥_Z\)) à un couple du même degré
d'appartenance (\(x\)) est toujours égale a cette même valeur, soit :
\(∀ x ∈ [0,1]\): \(⊤_Z(x,x) = x\) et \(⊥_Z(x,x) = x\).
%
Les opérateurs de \bsc{Łukasiewicz} sont quant a eux archimédiens et
les opérateurs probabilistes archimédiens stricts.

L’idempotence est une propriété des opérateurs, liée à
l'acrhimédianité. Un opérateur est dit idempotent, lorsque le résultat
de son application n'est pas modifié par le nombre d’applications,
c'est-à-dire, que pour une \emph{t-norme} ou une \emph{t-conorme}
idempotante : \(⊤(⊤(x,y), ⊤(x,y)) = ⊤(x,y)\) et
\(⊥(⊥(x,y),⊥(x,y)) = ⊥(x,y)\). Parmi les quatre couples d'opérateurs
présentés ici, seuls les opérateurs de \bsc{Zadeh} sont idempotent.

% D'accord mais RAF non ?
Comme les principes du \emph{tiers-exclu} et de la
\emph{non-contradiction,} l'archimédianité et l'idempotence des
\emph{t-normes} et des \emph{t-conormes} apparaissent comme des
propriétés théoriques, assez éloignées des considérations principales
de ce travail. Cependant, en modifiant le comportement des
\emph{t-normes} et des \emph{t-conormes} ces propriétés influent le
résultat des intersections et des unions, c'est-à-dire le cœur de la
\emph{phase de fusion.} Il est donc nécessaire d'en étudier
attentivement les conséquences.
%
Comme nous l'avons expliqué ci-dessus,
lorsqu'une \emph{t-norme} est archimédienne, son application à la même
valeur de degré d’appartenance (\(x\), qui est strictement supérieure
à 0 et strictement inférieure à 1) renvoie un degré d’appartenance
strictement inférieur à la valeur de \(x\).
%
Si l'on utilise une telle \emph{t-norme} pour réaliser les
intersections entre \ac{zlc}, alors on tendra à pénaliser la


Cela
implique que l'intersection, avec une telle \emph{t-norme,} de deux
\ac{zlc} pénalisera les positions dont le degré d'appartenance est
compris entre zéro et un.
%
À l'inverse l'union de deux \ac{zlc} 
%
Ainsi, utiliser une \emph{t-conorme} archimédienne, revient a
considérer que l'assertion \enquote{De cette position on voit un lac}
est \emph{plus vraie} si l'on voit (partiellement) deux lacs, que si
on n'en voie qu'un.

Les figures \ref{fig:comparaison_operateurs_union} et
\ref{fig:comparaison_operateurs_intersection} représentent
respectivement, le résultat de l'union et de l'intersection des deux
\emph{zones de localisation compatibles,} à l'aide des quatre couples
d'opérateurs.
%
La \ac{zlc} \textcolor{RdBu-9-1}{\textsf{A}} représente
la \emph{spatialisation} de \emph{l'indice de localisation}
\enquote{au sud de} et la \ac{zlc} \textcolor{RdBu-9-9}{\textsf{B}} de
\emph{l'indice de localisation} \enquote{à l'est de}.
%
La \autoref{fig:comparaison_operateurs_union} représente une situation
qui n'a pas cours dans notre méthode (\autoref{chap:04})





\begin{figure}
  \centering
  \begin{tikzpicture}
  \def\decalageX{-.2}
  \def\decalageY{-.2}

  \newcommand{\calcangle}[4]{%
    % #1 nom de la varaible renvoyée
    % #2 coordonnées de l'objet de référence
    % #3 coordonnées du point visé
    % #4 coordonées de la position testée
    \pgfmathanglebetweenlines{#2}{#3}{#2}{#4}%
    \global\let#1\pgfmathresult%
  }

  % Arrow
  \begin{scope}
    \path[draw, -, shorten >=5pt, shorten <=5pt] (-1,0) -- (-5,-2);
    %\path[draw, ->,shorten >=5pt, shorten <=5pt] (8.25,3) -- (10,3);
    %\path[draw, -,shorten >=12pt, shorten <=5pt] (1,-3) |- (3.65,-1);
    %\path[draw, ->,shorten >=5pt, shorten <=5pt] (8.25,-1) -- (10,-1);
  \end{scope}
  % Fuzzyfication 1 (Sud)
  \begin{scope}[xshift=-2cm, local bounding box=fuzz]
    \begin{scope}
      \foreach \x in {0,.25,...,1.75}{ \foreach \y in {0,.25,...,1.75}
        {
          % Calcul de l'angle
          \calcangle{\angle}{\pgfpoint{1cm}{1cm}}{\pgfpoint{1cm}{0cm}}{\pgfpoint{\x cm}{\y cm}}
          % Fuzzyfication de l'angle
          \pgfmathsetmacro\fuzzy{%
            ifthenelse(\angle < 90,(90-\angle)/90,
            ifthenelse(\angle > 270,(\angle-270)/90,0)
            )
          }
          % Calcul du rayon à partir de la fuzzyfication
          \pgfmathsetmacro\radius{\fuzzy*2.5}
          \fill[RdBu-9-1] (\x+.125,.125+\y) circle (\radius pt);
        }
      }
      \path[ffc] (0,0) rectangle (2,2);
    \end{scope}
    \node[text width=3cm, align=center, anchor=north, font=\footnotesize] at (1,-.25)
    {\itshape Zone de localisation compatible \normalfont \textcolor{RdBu-9-1}{\textsf{A}}};
  \end{scope} 
  % Fuzzyfication 2
  \begin{scope}[xshift=2cm,local bounding box=fuzz2]
    \begin{scope}
      \foreach \x in {0,.25,...,1.75}{
        \foreach \y in {0,.25,...,1.75}
        {
          % Calcul de l'angle
          \calcangle{\angle}{\pgfpoint{1cm}{1cm}}{\pgfpoint{2cm}{1cm}}{\pgfpoint{\x cm}{\y cm}}
          % Fuzzyfication de l'angle
          \pgfmathsetmacro\fuzzy{%
            ifthenelse(\angle < 90,(90-\angle)/90,
            ifthenelse(\angle > 270,(\angle-270)/90,0)
            )
          }
          % Calcul du rayon à partir de la fuzzyfication
          \pgfmathsetmacro\radius{\fuzzy*2.5}
          \fill[RdBu-9-9] (\x+.125,.125+\y) circle (\radius pt);
        }
      }
      \path[ffc2] (0,0) rectangle (2,2);
    \end{scope}
    \node[text width=3cm, align=center, anchor=north, font=\footnotesize] at (1,-.25)
    {\itshape Zone de localisation compatible \normalfont \textcolor{RdBu-9-9}{\textsf{B}}};
  \end{scope}
  % Fusion 1
  \begin{scope}[xshift=-6cm, yshift=-4cm,local bounding box=fus1]
    \begin{scope}
      \foreach \x in {0,.25,...,1.75}{
        \foreach \y in {0,.25,...,1.75}
        {
          % Calcul des angles
          \calcangle{\angleA}{\pgfpoint{1cm}{1cm}}{\pgfpoint{1cm}{0cm}}{\pgfpoint{\x
              cm}{\y cm}}
          % Fuzzyfication des angles
          \pgfmathsetmacro\fuzzyA{%
            ifthenelse(\angleA < 90,(90-\angleA)/90,
            ifthenelse(\angleA > 270,(\angleA-270)/90,0)
            )
          }
          \calcangle{\angleB}{\pgfpoint{1cm}{1cm}}{\pgfpoint{2cm}{1cm}}{\pgfpoint{\x cm}{\y cm}}
          \pgfmathsetmacro\fuzzyB{%
            ifthenelse(\angleB < 90,(90-\angleB)/90,
            ifthenelse(\angleB > 270,(\angleB-270)/90,0)
            )
          }
          % fusion (zadeh)
          \pgfmathsetmacro\fus{max(\fuzzyA, \fuzzyB)}          
          % Calcul du rayon à partir de la fuzzyfication
          \pgfmathsetmacro\radius{\fus*2.5}
          \fill[RdBu-9-9] (\x+.125,.125+\y) circle (\radius pt);
        }
      }
      \path[ffc2] (0,0) rectangle (2,2);
    \end{scope}
    \node[text width=3cm, align=center, anchor=north, font=\footnotesize] at (1,-.25)
    {\itshape Zone de localisation compatible \normalfont \textcolor{RdBu-9-9}{\textsf{B}}};
  \end{scope}
  % Fusion 2
  \begin{scope}[xshift=-2cm, yshift=-4cm,local bounding box=fus2]
    \begin{scope}
      \foreach \x in {0,.25,...,1.75}{
        \foreach \y in {0,.25,...,1.75}
        {
          % Calcul des angles
          \calcangle{\angleA}{\pgfpoint{1cm}{1cm}}{\pgfpoint{1cm}{0cm}}{\pgfpoint{\x
              cm}{\y cm}}
          % Fuzzyfication des angles
          \pgfmathsetmacro\fuzzyA{%
            ifthenelse(\angleA < 90,(90-\angleA)/90,
            ifthenelse(\angleA > 270,(\angleA-270)/90,0)
            )
          }
          \calcangle{\angleB}{\pgfpoint{1cm}{1cm}}{\pgfpoint{2cm}{1cm}}{\pgfpoint{\x cm}{\y cm}}
          \pgfmathsetmacro\fuzzyB{%
            ifthenelse(\angleB < 90,(90-\angleB)/90,
            ifthenelse(\angleB > 270,(\angleB-270)/90,0)
            )
          }
          % fusion (luka)
          \pgfmathsetmacro\fus{min(\fuzzyA + \fuzzyB, 1)}          
          % Calcul du rayon à partir de la fuzzyfication
          \pgfmathsetmacro\radius{\fus*2.5}
          \fill[RdBu-9-9] (\x+.125,.125+\y) circle (\radius pt);
        }
      }
      \path[ffc2] (0,0) rectangle (2,2);
    \end{scope}
    \node[text width=3cm, align=center, anchor=north, font=\footnotesize] at (1,-.25)
    {\itshape Zone de localisation compatible \normalfont \textcolor{RdBu-9-9}{\textsf{B}}};
  \end{scope}
  % Fusion 3
  \begin{scope}[xshift=2cm, yshift=-4cm,local bounding box=fus3]
    \begin{scope}
      \foreach \x in {0,.25,...,1.75}{
        \foreach \y in {0,.25,...,1.75}
        {
          % Calcul des angles
          \calcangle{\angleA}{\pgfpoint{1cm}{1cm}}{\pgfpoint{1cm}{0cm}}{\pgfpoint{\x
              cm}{\y cm}}
          % Fuzzyfication des angles
          \pgfmathsetmacro\fuzzyA{%
            ifthenelse(\angleA < 90,(90-\angleA)/90,
            ifthenelse(\angleA > 270,(\angleA-270)/90,0)
            )
          }
          \calcangle{\angleB}{\pgfpoint{1cm}{1cm}}{\pgfpoint{2cm}{1cm}}{\pgfpoint{\x cm}{\y cm}}
          \pgfmathsetmacro\fuzzyB{%
            ifthenelse(\angleB < 90,(90-\angleB)/90,
            ifthenelse(\angleB > 270,(\angleB-270)/90,0)
            )
          }
          % fusion (zadeh)
          \pgfmathsetmacro\fus{(\fuzzyA + \fuzzyB) - \fuzzyA * \fuzzyB}          
          % Calcul du rayon à partir de la fuzzyfication
          \pgfmathsetmacro\radius{\fus*2.5}
          \fill[RdBu-9-9] (\x+.125,.125+\y) circle (\radius pt);
        }
      }
      \path[ffc2] (0,0) rectangle (2,2);
    \end{scope}
    \node[text width=3cm, align=center, anchor=north, font=\footnotesize] at (1,-.25)
    {\itshape Zone de localisation compatible \normalfont \textcolor{RdBu-9-9}{\textsf{B}}};
  \end{scope}
  % Fusion 4
  \begin{scope}[xshift=6cm, yshift=-4cm,local bounding box=fus4]
    \begin{scope}
      \foreach \x in {0,.25,...,1.75}{
        \foreach \y in {0,.25,...,1.75}
        {
          % Calcul des angles
          \calcangle{\angleA}{\pgfpoint{1cm}{1cm}}{\pgfpoint{1cm}{0cm}}{\pgfpoint{\x
              cm}{\y cm}}
          % Fuzzyfication des angles
          \pgfmathsetmacro\fuzzyA{%
            ifthenelse(\angleA < 90,(90-\angleA)/90,
            ifthenelse(\angleA > 270,(\angleA-270)/90,0)
            )
          }
          \calcangle{\angleB}{\pgfpoint{1cm}{1cm}}{\pgfpoint{2cm}{1cm}}{\pgfpoint{\x cm}{\y cm}}
          \pgfmathsetmacro\fuzzyB{%
            ifthenelse(\angleB < 90,(90-\angleB)/90,
            ifthenelse(\angleB > 270,(\angleB-270)/90,0)
            )
          }
          % fusion (drast)
          \pgfmathsetmacro\fus{%
            ifthenelse(\fuzzyA < 0.001, \fuzzyB,
            ifthenelse(\fuzzyB < 0.001, \fuzzyA, 1)
            )
          }          
          % Calcul du rayon à partir de la fuzzyfication
          \pgfmathsetmacro\radius{\fus*2.5}
          \fill[RdBu-9-9] (\x+.125,.125+\y) circle (\radius pt);
        }
      }
      \path[ffc2] (0,0) rectangle (2,2);
    \end{scope}
    \node[text width=3cm, align=center, anchor=north, font=\footnotesize] at (1,-.25)
    {\itshape Zone de localisation compatible \normalfont \textcolor{RdBu-9-9}{\textsf{B}}};
  \end{scope}
\end{tikzpicture}
  \caption{Comparaison de l'union de deux \ac{zlc} en fonction de la
    \emph{t-conorme} utilisée.}
  \label{fig:comparaison_operateurs_union}
\end{figure}

\begin{figure}
  \centering
  \begin{tikzpicture}
  \def\decalageX{-.2}
  \def\decalageY{-.2}

  \newcommand{\calcangle}[4]{%
    % #1 nom de la varaible renvoyée
    % #2 coordonnées de l'objet de référence
    % #3 coordonnées du point visé
    % #4 coordonées de la position testée
    \pgfmathanglebetweenlines{#2}{#3}{#2}{#4}%
    \global\let#1\pgfmathresult%
  }

  % Arrow
  % \begin{scope}
  %   \path[draw, -, shorten >=12pt, shorten <=5pt] (1,3) |- (3.65,3);
  %   \path[draw, ->,shorten >=5pt, shorten <=5pt] (8.25,3) -- (10,3);
  %   \path[draw, -,shorten >=12pt, shorten <=5pt] (1,-3) |- (3.65,-1);
  %   \path[draw, ->,shorten >=5pt, shorten <=5pt] (8.25,-1) -- (10,-1);
  % \end{scope}
  % Fuzzyfication 1 (Sud)
  \begin{scope}[xshift=-2cm, local bounding box=fuzz]
    \begin{scope}
      \foreach \x in {0,.25,...,1.75}{ \foreach \y in {0,.25,...,1.75}
        {
          % Calcul de l'angle
          \calcangle{\angle}{\pgfpoint{1cm}{1cm}}{\pgfpoint{1cm}{0cm}}{\pgfpoint{\x cm}{\y cm}}
          % Fuzzyfication de l'angle
          \pgfmathsetmacro\fuzzy{%
            ifthenelse(\angle < 90,(90-\angle)/90,
            ifthenelse(\angle > 270,(\angle-270)/90,0)
            )
          }
          % Calcul du rayon à partir de la fuzzyfication
          \pgfmathsetmacro\radius{\fuzzy*2.5}
          \fill[RdBu-9-1] (\x+.125,.125+\y) circle (\radius pt);
        }
      }
      \path[ffc] (0,0) rectangle (2,2);
    \end{scope}
    \node[text width=3cm, align=center, anchor=north, font=\footnotesize] at (1,-.25)
    {\itshape Zone de localisation compatible \normalfont \textcolor{RdBu-9-1}{\textsf{A}}};
  \end{scope} 
  % Fuzzyfication 2
  \begin{scope}[xshift=2cm,local bounding box=fuzz2]
    \begin{scope}
      \foreach \x in {0,.25,...,1.75}{
        \foreach \y in {0,.25,...,1.75}
        {
          % Calcul de l'angle
          \calcangle{\angle}{\pgfpoint{1cm}{1cm}}{\pgfpoint{2cm}{1cm}}{\pgfpoint{\x cm}{\y cm}}
          % Fuzzyfication de l'angle
          \pgfmathsetmacro\fuzzy{%
            ifthenelse(\angle < 90,(90-\angle)/90,
            ifthenelse(\angle > 270,(\angle-270)/90,0)
            )
          }
          % Calcul du rayon à partir de la fuzzyfication
          \pgfmathsetmacro\radius{\fuzzy*2.5}
          \fill[RdBu-9-9] (\x+.125,.125+\y) circle (\radius pt);
        }
      }
      \path[ffc2] (0,0) rectangle (2,2);
    \end{scope}
    \node[text width=3cm, align=center, anchor=north, font=\footnotesize] at (1,-.25)
    {\itshape Zone de localisation compatible \normalfont \textcolor{RdBu-9-9}{\textsf{B}}};
  \end{scope}
  % Fusion 1
  \begin{scope}[xshift=-6cm, yshift=-4cm,local bounding box=fus1]
    \begin{scope}
      \foreach \x in {0,.25,...,1.75}{
        \foreach \y in {0,.25,...,1.75}
        {
          % Calcul des angles
          \calcangle{\angleA}{\pgfpoint{1cm}{1cm}}{\pgfpoint{1cm}{0cm}}{\pgfpoint{\x
              cm}{\y cm}}
          % Fuzzyfication des angles
          \pgfmathsetmacro\fuzzyA{%
            ifthenelse(\angleA < 90,(90-\angleA)/90,
            ifthenelse(\angleA > 270,(\angleA-270)/90,0)
            )
          }
          \calcangle{\angleB}{\pgfpoint{1cm}{1cm}}{\pgfpoint{2cm}{1cm}}{\pgfpoint{\x cm}{\y cm}}
          \pgfmathsetmacro\fuzzyB{%
            ifthenelse(\angleB < 90,(90-\angleB)/90,
            ifthenelse(\angleB > 270,(\angleB-270)/90,0)
            )
          }
          % fusion (zadeh)
          \pgfmathsetmacro\fus{min(\fuzzyA, \fuzzyB)}          
          % Calcul du rayon à partir de la fuzzyfication
          \pgfmathsetmacro\radius{\fus*2.5}
          \fill[RdBu-9-9] (\x+.125,.125+\y) circle (\radius pt);
        }
      }
      \path[ffc2] (0,0) rectangle (2,2);
    \end{scope}
    \node[text width=3cm, align=center, anchor=north, font=\footnotesize] at (1,-.25)
    {\itshape Zone de localisation compatible \normalfont \textcolor{RdBu-9-9}{\textsf{B}}};
  \end{scope}
  % Fusion 2
  \begin{scope}[xshift=-2cm, yshift=-4cm,local bounding box=fus2]
    \begin{scope}
      \foreach \x in {0,.25,...,1.75}{
        \foreach \y in {0,.25,...,1.75}
        {
          % Calcul des angles
          \calcangle{\angleA}{\pgfpoint{1cm}{1cm}}{\pgfpoint{1cm}{0cm}}{\pgfpoint{\x
              cm}{\y cm}}
          % Fuzzyfication des angles
          \pgfmathsetmacro\fuzzyA{%
            ifthenelse(\angleA < 90,(90-\angleA)/90,
            ifthenelse(\angleA > 270,(\angleA-270)/90,0)
            )
          }
          \calcangle{\angleB}{\pgfpoint{1cm}{1cm}}{\pgfpoint{2cm}{1cm}}{\pgfpoint{\x cm}{\y cm}}
          \pgfmathsetmacro\fuzzyB{%
            ifthenelse(\angleB < 90,(90-\angleB)/90,
            ifthenelse(\angleB > 270,(\angleB-270)/90,0)
            )
          }
          % fusion (luka)
          \pgfmathsetmacro\fus{max(\fuzzyA + \fuzzyB - 1, 0)}          
          % Calcul du rayon à partir de la fuzzyfication
          \pgfmathsetmacro\radius{\fus*2.5}
          \fill[RdBu-9-9] (\x+.125,.125+\y) circle (\radius pt);
        }
      }
      \path[ffc2] (0,0) rectangle (2,2);
    \end{scope}
    \node[text width=3cm, align=center, anchor=north, font=\footnotesize] at (1,-.25)
    {\itshape Zone de localisation compatible \normalfont \textcolor{RdBu-9-9}{\textsf{B}}};
  \end{scope}
  % Fusion 3
  \begin{scope}[xshift=2cm, yshift=-4cm,local bounding box=fus3]
    \begin{scope}
      \foreach \x in {0,.25,...,1.75}{
        \foreach \y in {0,.25,...,1.75}
        {
          % Calcul des angles
          \calcangle{\angleA}{\pgfpoint{1cm}{1cm}}{\pgfpoint{1cm}{0cm}}{\pgfpoint{\x
              cm}{\y cm}}
          % Fuzzyfication des angles
          \pgfmathsetmacro\fuzzyA{%
            ifthenelse(\angleA < 90,(90-\angleA)/90,
            ifthenelse(\angleA > 270,(\angleA-270)/90,0)
            )
          }
          \calcangle{\angleB}{\pgfpoint{1cm}{1cm}}{\pgfpoint{2cm}{1cm}}{\pgfpoint{\x cm}{\y cm}}
          \pgfmathsetmacro\fuzzyB{%
            ifthenelse(\angleB < 90,(90-\angleB)/90,
            ifthenelse(\angleB > 270,(\angleB-270)/90,0)
            )
          }
          % fusion (zadeh)
          \pgfmathsetmacro\fus{\fuzzyA * \fuzzyB}          
          % Calcul du rayon à partir de la fuzzyfication
          \pgfmathsetmacro\radius{\fus*2.5}
          \fill[RdBu-9-9] (\x+.125,.125+\y) circle (\radius pt);
        }
      }
      \path[ffc2] (0,0) rectangle (2,2);
    \end{scope}
    \node[text width=3cm, align=center, anchor=north, font=\footnotesize] at (1,-.25)
    {\itshape Zone de localisation compatible \normalfont \textcolor{RdBu-9-9}{\textsf{B}}};
  \end{scope}
  % Fusion 4
  \begin{scope}[xshift=6cm, yshift=-4cm,local bounding box=fus4]
    \begin{scope}
      \foreach \x in {0,.25,...,1.75}{
        \foreach \y in {0,.25,...,1.75}
        {
          % Calcul des angles
          \calcangle{\angleA}{\pgfpoint{1cm}{1cm}}{\pgfpoint{1cm}{0cm}}{\pgfpoint{\x
              cm}{\y cm}}
          % Fuzzyfication des angles
          \pgfmathsetmacro\fuzzyA{%
            ifthenelse(\angleA < 90,(90-\angleA)/90,
            ifthenelse(\angleA > 270,(\angleA-270)/90,0)
            )
          }
          \calcangle{\angleB}{\pgfpoint{1cm}{1cm}}{\pgfpoint{2cm}{1cm}}{\pgfpoint{\x cm}{\y cm}}
          \pgfmathsetmacro\fuzzyB{%
            ifthenelse(\angleB < 90,(90-\angleB)/90,
            ifthenelse(\angleB > 270,(\angleB-270)/90,0)
            )
          }
          % fusion (drast)
          \pgfmathsetmacro\fus{%
            ifthenelse(\fuzzyA > 0.999, \fuzzyB,
            ifthenelse(\fuzzyB > 0.999, \fuzzyA, 0)
            )
          }          
          % Calcul du rayon à partir de la fuzzyfication
          \pgfmathsetmacro\radius{\fus*2.5}
          \fill[RdBu-9-9] (\x+.125,.125+\y) circle (\radius pt);
        }
      }
      \path[ffc2] (0,0) rectangle (2,2);
    \end{scope}
    \node[text width=3cm, align=center, anchor=north, font=\footnotesize] at (1,-.25)
    {\itshape Zone de localisation compatible \normalfont \textcolor{RdBu-9-9}{\textsf{B}}};
  \end{scope}
\end{tikzpicture}
  \caption{Comparaison de l'intersection de deux \ac{zlc} en fonction
    de la \emph{t-norme} utilisée.}
  \label{fig:comparaison_operateurs_intersection}
\end{figure}

Les conséquences de l'archimédianité nous conduisent donc à rejeter
les opérateurs de \bsc{XX} et probabilistes. 


L'étude des différentes caractéristiques
(\autoref{tab:synthese_operateurs}) des couples d'opérateurs les plus
utilisés dans la littérature nous a conduit à en rejeter la plupart
d'entre-eux. Les opérateurs drastiques ont été rejetés car
discontinus, trop sévères et donc inadaptés à notre cas
d'utilisation. L'étude de l'impact du rejet des principes de
\emph{non-contradiction} et du \emph{tiers-exclu} ne nous a pas permis
de disqualifier des opérateurs, cependant nous avons mis en évidence
que l'archimédianité (à fortiori stricte) des opérateurs n'était pas
souhaitable. C'est pourquoi nous avons retenu les opérateurs de
\textcite{Zadeh1965} pour effectuer les opérations d'union et
d'intersection inter-\ac{zlc}.

\begin{table}
  \centering
  \begin{tabular}{r>{\small}C{.15\textwidth}>{\small}C{.15\textwidth}>{\small}C{.15\textwidth}>{\small}C{.15\textwidth}}
  \toprule
  & \bfseries \textcite{Zadeh1965}&\bfseries \bsc{Łukasiewicz}&\bfseries
                                                   Probabilistes &\bfseries Drastiques\\
  \midrule
  %\addlinespace
  \bfseries Continu & \bfseries Oui & Oui & Oui & Non \\
  \bfseries Non-contradiction & \bfseries Non & Oui & Non & Oui \\
  \bfseries Tiers-exclu & \bfseries Non & Oui & Non & Oui \\
  \bfseries Archimédien & \bfseries Non & Oui & Strict & Non \\
  \bottomrule
\end{tabular}

  \caption{Synthèse des caractéristiques des opérateurs}
  \label{tab:synthese_operateurs}
\end{table}

\subsection{La prise en compte de la confiance}
\tdi{voi si c,'est pas mieux de le mettre en section}


La théorie des \emph{sous-ensembles flous} ne permet pas, du moins
dans sa formulation initiale, de modéliser l'incertitude d'une
connaissance, seulement son \emph{imprécision.}

\textcite{Zadeh1978}

\begin{figure}
  \centering
  %\begin{tikzpicture}
  \begin{scope}[xshift=-2cm]
    \begin{scope}
      \foreach \x in {0,.25,...,1.75}{ \foreach \y in {0,.25,...,1.75}
        {
          % Calcul et représentation floue distance euclidienne
          \pgfmathsetmacro\dist{sqrt((\x-0.125)^2+(\y-1.375)^2)}
          % Fuzzyfication de la distance
          \pgfmathsetmacro\fuzzy{%
            ifthenelse(\dist < .25,1,%
            ifthenelse(\dist < 2,-0.57*\dist+1.14,0)%
            )
          }
          % Calcul du rayon à partir de la fuzzyfication
          \pgfmathsetmacro\radius{max(\fuzzy, 0.6)*2.5}
          \fill[black] (\x+.125,.125+\y) circle (\radius pt);
        }
      }
      \path[ffc, black] (0,0) rectangle (2,2);
    \end{scope}
  \end{scope} 
  
  \begin{scope}[xshift=1.2cm]
    \begin{scope}
      \foreach \x in {0,.25,...,1.75}{ \foreach \y in {0,.25,...,1.75}
        {
          % Calcul et représentation floue distance euclidienne
          \pgfmathsetmacro\dist{sqrt((\x-0.125)^2+(\y-1.375)^2)}
          % Fuzzyfication de la distance
          \pgfmathsetmacro\fuzzy{%
            ifthenelse(\dist < .25,1,%
            ifthenelse(\dist < 2,-0.57*\dist+1.14,0)%
            )
          }
          % Calcul du rayon à partir de la fuzzyfication
          \pgfmathsetmacro\radius{max(\fuzzy, 0.2)*2.5}
          \fill[RdBu-9-1] (\x+.125,.125+\y) circle (\radius pt);
        }
      }
      \path[ffc] (0,0) rectangle (2,2);
    \end{scope}
  \end{scope}

    \begin{scope}[xshift=6.8cm]
    \begin{scope}
      \foreach \x in {0,.25,...,1.75}{ \foreach \y in {0,.25,...,1.75}
        {
          % Calcul et représentation floue distance euclidienne
          \pgfmathsetmacro\dist{sqrt((\x-0.125)^2+(\y-1.375)^2)}
          % Fuzzyfication de la distance
          \pgfmathsetmacro\fuzzy{%
            ifthenelse(\dist < .25,1,%
            ifthenelse(\dist < 2,-0.57*\dist+1.14,0)%
            )
          }
          % Calcul du rayon à partir de la fuzzyfication
          \pgfmathsetmacro\radius{max(\fuzzy, 0.6)*2.5}
          \fill[RdBu-9-9] (\x+.125,.125+\y) circle (\radius pt);
        }
      }
      \path[ffc2] (0,0) rectangle (2,2);
    \end{scope}
  \end{scope}

  \node[scale=5] at (-3.2,1) {\(\top(\)};
  \node[scale=5] at (.45,0) {\(,\)};
  \node[scale=5] at (4.7,1) {\() = \)};
  
\end{tikzpicture}
  \caption{Influence de l'incertitude d'un \emph{indice de
      localisation} sur la \emph{zone de localisation} résultant de la
    fusion de deux \ac{zlc}.}
  \label{fig:intersection}
\end{figure}


Une fonction d'appartenance normalisée est donc une distribution de
possibilité.

La confiance peut se définir en amont de la spatialisation.

Diminuer la confiance c'est augmenter le degré d'appartenance minimal

Proposer une version améliorée de la fig finale du chapitre 4 avec la
confiance.

Dire que les fonctions présentées reviennent a considérer que la
confiance est absolue

Parler des seuils de la confiance (3) et des valeurs associées ou
contraintes.


\begin{figure}
  \centering
  \begin{tikzpicture}[scale=.7]
  \def\decalageX{-.2}
  \def\decalageY{-.2}
  % Courbe
  \begin{scope}[transparency group]
    % fond
    \begin{scope}
      \path[ffa]  (1,.4) -- (3.3,.4) -- (4.5, 2) -- (5.7,.4) -- (8,.4)
      -- (8,0) -- (1,0) -- cycle;
      \path[ffa_fade_m] (0,.4) -- (1,.4) -- (1,0) -- (0,0) -- cycle ;
      \path[ffa_fade] (8,.4) -- (9,.4) -- (9,0) -- (8,0) -- cycle ;
    \end{scope}
    % bords
    \begin{scope}
      \path[ffc] (1,.4) -- (3.3,.4) -- (4.5, 2) -- (5.7,.4) -- (8,.4);
      \path[ffc, dotted] (3.3,.4) -- (3,0);
      \path[ffc, dotted] (5.7,.4) -- (6,0);
      \path[ffc_fade_m] (0,.4) -- (1,.4) ;
      \path[ffc_fade] (8,.4) -- (9,.4) ;
    \end{scope}
  \end{scope}
  % Axes X, Y
  \begin{scope}
    % Axe X
    \begin{scope}
      % Axe
      \draw[<->] (0, \decalageX) --++ (9, 0) coordinate (x axis);
      % Graduations
      \foreach \n/\t in {1/{},2/{},3/{},4/{},5/{},6/{},7/{},8/{}}
      {
        \draw[-] (\n, \decalageX - .05) --++ (0, .1);
        \node[below, font=\footnotesize] at (\n, \decalageX - .05) {\t};
      }
      % label
      \node[below left, yshift=-.1cm, font=\small] at (x axis) {\itshape Métrique};
    \end{scope}
    % Axe Y
    \begin{scope}
      % Axe
      \draw[-] (\decalageY ,0) --++ (0, 2) coordinate (y axis);
      % Graduations
      \foreach \n/\t in {0/{0},2/{1}}
      {
        \draw[-] (\decalageY -.05, \n) --++ (.1, 0);
        \node[left, font=\footnotesize] at (\decalageY -.05, \n) {\t};
      }
      % Label
      \node[above] at (y axis) {$\mu$};
    \end{scope}
  \end{scope}
  \begin{scope}
    % Seuil 1
    \draw[ffc,line width=.5] (3.3,\decalageY) -- (3.3,.4);
    \draw[fill, RdBu-9-1] (3.3,\decalageY) circle (2pt);
    \draw[fill, RdBu-9-1] (3.3,.4) circle (2pt);
    % Seuil 2
    \draw[ffc,line width=.5] (4.5,\decalageY) -- (4.5,2);
    \draw[fill, RdBu-9-1] (4.5,\decalageY) circle (2pt);
    \draw[fill, RdBu-9-1] (4.5,2) circle (2pt);
    \node[above] at (4.5,2) {\(v\)};
    % Seuil 3
    \draw[ffc,line width=.5] (5.7,\decalageY) -- (5.7,.4);
    \draw[fill, RdBu-9-1] (5.7,\decalageY) circle (2pt);
    \draw[fill, RdBu-9-1] (5.7,.4) circle (2pt);
    
    \draw[|-|] (3,-.7cm) --++(3,0) node[pos=.5, fill=white, inner
    sep=1pt, font=\small] {$\delta$};

    \draw[<->, shorten >=2pt,shorten <=2pt] (2,.4cm) --++ (0,1.6cm) node[pos=.5, fill=white, inner
    sep=1pt, font=\small] {$i$};
  \end{scope}
\end{tikzpicture}

  \caption{sq}
  \label{fig:qs}
\end{figure}




\begin{figure}
  \centering
  \missingfigure{Comparaison fusion indices avec et sans incertitude}
\end{figure}




%%% Local Variables:
%%% mode: latex
%%% TeX-master: "../../../../main"
%%% End:
