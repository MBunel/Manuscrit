Comme nous l'illustrions dans l'état de l'art de cette thèse de
nombreuses implémentations des \emph{sous-ensembles flous spatialisés}
ont été proposées. On y retrouve aussi bien des approches vectorielles
\autocite{Kanjilal2010,Dilo2007,Zoghlami2016} ou raster
\autocite{Bloch1996}, que des approches fondées sur la sélection
d'objets préexistants \autocite{Duraciova2017}.

Chacune de ces solutions a des avantages et des inconvénients qu'il
est nécessaire d'étudier. De plus, toutes ses solutions ne permettent
pas nécessairement de conserver toutes les qualités du modèle
théorique. Par exemple, les approches vectorielles proposant une
discrétisation du sous-ensemble flou, comme celles proposées par
\textcite{Kanjilal2010,Zoghlami2016}, ont une précision de
représentation qui dépend du nombre de régions utilisées pour
discrétiser le \emph{sous-ensemble flou.} Si elles sont au nombre de
deux, ces implémentations se comportent comme les modèles
\enquote{exacts} et ne sont pas plus utilisables. L'augmentation du
nombre de vecteurs permet de régler ce problème et de représenter tout
type de \emph{zones de localisation.}


\tdi{OK : On veut un modèle qui permet de prendre en compte des relations
  de localisation traduisant des limites assez nettes (en dessous de)
  et des relations avec des limites très continues (proches [voir
  concept similaire dans l'ontologie]).}

\tdi{OK : Possibilité de construire des zones de localisation issues de
  relations de localisation non linéaire (en dessous de).}

Chacune de ces deux implémentations permet la construction de
\emph{zones de localisation compatibles} à partir de \emph{relations
  de localisation} non linéaires. En effet, bien qu'elles imposent
toutes deux un échantillonnage des sous-ensembles flous, elles
permettent de rendre des variations non linéaires.

De la même manière, ces deux approches permettent de construire des
\emph{zones de localisation} dont les limites peuvent être
\emph{imprécises} ou assez précises.


Nous proposons de confronter les deux implémentations qui nous
semblent être les plus prometteuses, l'utilisation \emph{d'alpha-cuts}
vectorielles et l'utilisation de rasters.

\subsection{Définition d'un exemple comparatif}

Les implémentations de la \emph{théorie des sous-ensembles flous} à
l'aide d'une approche par \emph{alpha-cuts} ou par raster nous
semblent toutes les deux pertinentes pour notre cas
d'utilisation. Nous proposons donc de les appliquer à une même exemple
pour trancher en faveur de l'une ou de l'autre
\autocite{Bunel2019a}. Pour ce faire nous avons sélectionné l'un des
\emph{indices de localisation} du \emph{fil rouge} : \enquote{La
  victime est située sous une ligne électrique 3 brins}. Cet
\emph{indice} présente le double avantage d'être simple à modéliser
---~et donc à comparer~---, tout en utilisant une \emph{relation de
  localisation,} décomposable, ce qui permet de tester également les
opérations entre \emph{zones de localisation.}

% Définition de l'aire
Pour faciliter la comparaison de ces deux implémentations nous nous
sommes focalisés sur une petite partie de la zone du \emph{fil rouge,}
ce qui nous permet de ne travailler qu'avec un seul \emph{objet de
  référence} candidat (\autoref{fig:ZIR_HT}). Ainsi, pour cet exemple
nous ne nous focalisons que sur une partie de la méthodologie
présentée dans le \autoref{chap:04}. Conne il n'y a qu'un \emph{indice
  de localisation} à \emph{spatialiser,} il n'est pas nécessaire de
\emph{décomposer} \emph{l'ensemble des indices de localisation} et de
fusionner les zlc\todo{ac} correspondantes \footnote{Ce qui
  correspond, respectivement à la première étape de la \emph{phase de
    décomposition} et à la dernière étape de la \emph{phase de fusion}
  (\autoref{fig:methodo_1}).}. De même, comme nous ne travaillons que
sur un \emph{objet de référence} il n'est pas nécessaire de procéder à
la \emph{décomposition} d'un \emph{objet de référence} non nommé et à
la \emph{fusion} des zlc\todo{ac} issues de leur spatialisation
\footnote{Ce qui correspond, respectivement aux secondes étapes des
  \emph{phases de décomposition} et de \emph{fusion}
  (\autoref{fig:methodo_1}).}. L'absence de ces quatre étapes ne pose
pas de problèmes pour la comparaison des implémentations. D'une part
car le choix de la \emph{modélisation} des \emph{zones de
  localisation} n'a pas impact sur la \emph{phase de décomposition,}
aucun \emph{objet géographique} n'étant encore construit. De l'autre,
car les étapes de la \emph{phase de fusion} emploient les mêmes outils
théoriques \footnote{C'est-à-dire des unions et intersections
  ensemblistes, faisant appel aux \emph{t-normes} et \emph{t-conormes}
  de la \emph{théorie des sous-ensembles flous.}}, ce qui rend leur
comportement similaire. Il n'est donc pas nécessaire de tester chaque
étape de la \emph{phase de fusion} pour estimer l'adéquation d'une
implémentation.

\begin{carte}
  \centering \input{../figures/Raster_ZIR_HT.tex}
  \caption{Zone utilisée pour la comparaison des implémentations.}
  \label{fig:ZIR_HT}
\end{carte}

% Décomposition de l'indice
La \emph{relation de localisation} que nous avons identifiée comme la
plus adaptée à la formalisation de cet \emph{indice de localisation}
est \onto[orl]{Sous\-Pro\-che\-De}, traduisant une relation sur l'axe
vertical (l'altitude du \emph{sujet} est plus basse que celle de
\emph{l'objet de référence}), contrainte par une distance
planimétrique (\autoref{anx:orl_dic}). Cette \emph{relation} se
décompose en deux \emph{relations de localisation atomiques} :
\onto[orla]{Al\-ti\-tu\-de\-Stric\-te\-ment\-Inf\-érieu\-re} et
\onto[orla]{Dis\-tan\-ce\-Quan\-ti\-ta\-ti\-ve\-Pla\-ni\-mé\-tri\-que},
traduisant respectivement, la notion de position sur l'axe vertical et
la contrainte posée sur la distance planimétrique. Conformément au
\emph{principe de décomposition} et à notre méthodologie (Chapitres
\ref{chap:02} et \ref{chap:04}), la \emph{zone de localisation
  compatible} correspondant à la \emph{relation de localisation}
\onto[orl]{Sous\-Pro\-che\-De} peut être créée en intersectant les
deux \emph{zones de localisation compatibles} résultant de la
\emph{spatialisation} de ces deux \emph{relations de localisation
  atomiques.}

% Définition fonctions appartenance
La construction des \emph{zones de localisation compatibles}
correspondant à chacune des \emph{relations de localisation atomiques}
nécessite de définir des \emph{fonctions d'appartenance} ($f_A$), qui,
associent à chaque élément de l'ensemble traité, un degré ---~compris
entre 0 et 1~--- définissant l'appartenance de l'élément au
\emph{sous-ensemble flou} (Section \ref{subsec:theorie_flou}). Dans
notre contexte, où les \emph{sous-ensembles fous} représentent des
\emph{objets géographiques} et chaque élément une position, ces
fonctions définissent l'appartenance de chacune d'entre elles à la
\emph{zone de localisation compatible} construite. La définition des
\emph{fonctions d'appartenance} est donc une étape extrêmement
importante, puisqu'elle définit les règles de construction des
\emph{zones de localisation} et donc la sémantique des \emph{relations
  de localisation atomiques.} Leur définition se fait en deux étapes,
tout d'abord il faut identifier une \emph{métrique,} c'est-à-dire une
valeur quantitative ou qualitative, calculable en chaque position de
la \emph{zone initiale de recherche} et qui peut servir de mesure à la
sémantique de la \emph{relation de localisation atomique} (\eg une
distance, un angle). Puis on définit la \emph{fonction
  d'appartenance,} transformant les valeurs de la \emph{métrique} en
un \emph{degré d'appartenance.} Dans notre exemple, les
\emph{métriques} à utiliser sont assez instinctives et simples à
calculer. Par exemple, la \emph{relation de localisation atomique}
\onto[orla]{Dis\-tan\-ce\-Quan\-ti\-ta\-ti\-ve\-Pla\-ni\-mé\-tri\-que}
traduit le fait que la distance planimétrique entre le \emph{sujet} et
\emph{l'objet de référence} est faible. La métrique adaptée est donc
la distance euclidienne entre chaque position et le point le plus
proche de \emph{l'objet de référence} (\autoref{fig:ISO_DIST_HT}).

\begin{carte}
  \centering
  \begin{tikzpicture}
  \tikzset{et/.style={above,font=\footnotesize\vphantom{Ag}}}
  % 
  \node[inner sep=0pt, anchor=south west] (image) at (0,0){\includegraphics{./figures/Raster_ISO_DIST_HT.png}};
  % 
  \begin{scope}
    \node (P2) at ([yshift=-.5cm]image.south east) {};
    \node (P1) at ([yshift=-.5cm]image.south west) {};

    \node (rect) [anchor=north west, minimum width=1cm,minimum
    height=.25cm] at ([yshift=-.25cm]P1) {}; \path[draw=RdBu-9-1, line
    width=.4mm](rect.west) --([xshift=-1ex]rect.south) -- ([xshift=1ex]rect.north)
    -- (rect.east);
    \node[anchor=west, font=\tiny\vphantom{Ag}, text width = 4cm] at
    ([xshift=1ex]rect.east) {Ligne électrique utilisée comme
      \emph{objet de référence}};

    \node (rect2) [anchor=north west, minimum width=1cm,minimum
    height=.25cm] at ([xshift=5.5cm,yshift=-.25cm]P1) {};
    \path[draw=RdBu-9-9, line width=.25mm](rect2.west)
    --([xshift=-1ex]rect2.north) -- ([xshift=1ex]rect2.south) --
    (rect2.east); \node[anchor=west, font=\tiny\vphantom{Ag}, text
    width = 4cm] at ([xshift=1ex]rect2.east) {Isolignes d'éloignement
      à \emph{l'objet de référence} (équidistance \SI{250}{\meter})};

    % Échelle
    \draw[-] (P2 |- -1cm,-1cm) --++ (-1,0) node[et,pos=.5] {\SI{500}{\meter}};
    % Légende détaillée
    \path (P1) -- (P2) node[pos=.5, yshift=-1cm] {\tiny Pour la légende détaillée du fond topographique voir \autoref{anx:topo_leg}. Sources: BD TOPO 2018, BD ALTI 2018.}; 
  \end{scope}
\end{tikzpicture}
  \caption{\emph{Métrique} pour la \emph{relation de localisation
      atomique}    \protect\onto[orla]{Dis\-tan\-ce\-Quan\-ti\-ta\-ti\-ve\-Pla\-ni\-mé\-tri\-que}
    : La distance planaire à la ligne électrique trois brins.}
  \label{fig:ISO_DIST_HT}
\end{carte}

La \emph{relation de localisation atomique}
\onto[orla]{Al\-ti\-tu\-de\-Stric\-te\-ment\-Inf\-érieu\-re} traduit,
quant à elle, un positionnement relatif sur l'axe vertical. La
\emph{métrique} la plus adaptée est donc la différence entre
l'altitude du point le plus proche de \emph{l'objet de référence} et
l'altitude de chaque position (\autoref{fig:ISO_DALT_HT}).

\begin{carte}
  \centering
  \begin{tikzpicture}
  \tikzset{et/.style={above,font=\footnotesize\vphantom{Ag}}}
  % 
  \node[inner sep=0pt, anchor=south west] (image) at (0,0){\includegraphics{./figures/Raster_ISO_DALT_HT.png}};
  % 
  \begin{scope}
    \node (P2) at ([yshift=-.5cm]image.south east) {};
    \node (P1) at ([yshift=-.5cm]image.south west) {};

    \node (rect) [anchor=north west, minimum width=1cm,minimum
    height=.25cm] at ([yshift=-.25cm]P1) {}; \path[draw=RdBu-9-1, line
    width=.4mm](rect.west) --([xshift=-1ex]rect.south) -- ([xshift=1ex]rect.north)
    -- (rect.east);
    \node[anchor=west, font=\tiny\vphantom{Ag}, text width = 4cm] at
    ([xshift=1ex]rect.east) {Ligne électrique utilisée comme
      \emph{objet de référence}};
    %
    \node[anchor=west, font=\footnotesize\vphantom{Ag}, text width=8cm] at
    (P1 |- 0cm,-1.85cm) {Différence d'altitude avec la ligne électrique:};
    %
    \begin{scope}
      \foreach \x [evaluate=\xshift using 2.5+\x/10, evaluate=\rad using (\x * -.0008) + .05] in {0,...,50}
      {
        \draw[fill=RdBu-9-9,draw=none, below] ([xshift=\xshift cm, yshift=-2.5cm]P1) circle [radius=\rad cm];
      }
      \foreach \x [evaluate=\xshift using 7.5+\x/10, evaluate=\rad using (\x * .0008) + .01] in {0,...,50}
      {
        \draw[fill=RdBu-9-1,draw=none, below] ([xshift=\xshift cm, yshift=-2.5cm]P1) circle [radius=\rad cm];
      }
      % 
      \path(2.5,-3) --++ (10,0)
      node[et,pos=0] {$<$ \SI{-500}{\meter}}
      node[et,pos=.5] {\SI{0}{\meter}}
      node[et,pos=1] {$>$ \SI{500}{\meter}};
    \end{scope}
    
    % \node (rect2) [anchor=north west, minimum width=1cm,minimum
    % height=.25cm] at ([xshift=5.5cm,yshift=-.25cm]P1) {};
    % \path[draw=RdBu-9-9, line width=.25mm](rect2.west)
    % --([xshift=-1ex]rect2.north) -- ([xshift=1ex]rect2.south) --
    % (rect2.east); \node[anchor=west, font=\tiny\vphantom{Ag}, text
    % width = 4cm] at ([xshift=1ex]rect2.east) {Isolignes d'éloignement
    % à \emph{l'objet de référence} (équidistance \SI{250}{\meter})};

    % Échelle
    \draw[-] (P2 |- -1cm,-1cm) --++ (-1,0) node[et,pos=.5] {\SI{500}{\meter}};
    % Légende détaillée
    \path (P1) -- (P2) node[pos=.5, yshift=-3cm] {\tiny Pour la
      légende détaillée du fond topographique voir
      \autoref{anx:topo_leg}. Sources: BD TOPO 2018, BD ALTI 2018.};
  \end{scope}
\end{tikzpicture}
  \caption{Différence entre l'altitude locale et l'altitude de la
    ligne électrique la plus proche.}
  \label{fig:ISO_DALT_HT}
\end{carte}

La transformation des valeurs de ces deux métriques en degré
d'appartenance se fait à l'aide d'une fonction d'appartenance qu'il
convient de définir. Pour ce faire nous procédons généralement en deux
étapes. La première consiste à identifier la forme de \emph{la
  fonction d'appartenance} qui nous semble être la plus pertinente
pour retranscrire la sémantique de la \emph{relation de localisation
  atomique.} Puis nous identifions les seuils de cette
fonction. Ainsi, la première étape permet \footnote{Conjointement à la
  sélection d'une \emph{métrique.}} de fixer la sémantique de la
\emph{relation,} alors que la seconde étape fait plus office de
paramétrisation fine. La sélection de la forme de la fonction ne pose
pas de problèmes particulier, contrairement à la définition des seuils
qui, comme nous allons le voir, est généralement assez
arbitraire. Cependant, pour cet exemple particulier les seuils définis
ont été validés par un secouriste du \ac{pghm}.

La \emph{relation de localisation atomique}
\onto[orla]{Dis\-tan\-ce\-Quan\-ti\-ta\-ti\-ve\-Pla\-ni\-mé\-tri\-que}
est utilisée pour figurer le fait que le \emph{sujet} n'est pas très
éloigné de \emph{l'objet de référence.} Ainsi, une position
\enquote{proche} de \emph{l'objet de référence} doit avoir un
\emph{degré d'appartenance} à la \emph{zone de localisation
  compatible} correspondant à cette \emph{relation de localisation
  atomique} plus élevé qu'une position qui en est éloigné. Le degré
d'appartenance doit donc diminuer lorsque la distance à \emph{l'objet
  de référence} (\ie la valeur de la métrique) augmente. La forme de
la fonction d'appartenance étant fixée, nous pouvons à présent en
définir les seuils. Deux valeurs sont à fixer, la première est la
distance à partir de laquelle le degré d'appartenance décroit. Si elle
est nulle le degré d’appartenance décroit dès que la distance
augmente, sinon il y aura une zone, dont la largeur dépend de la
valeur choisie, où le degré d'appartenance sera maximal. Définir ce
premier seuil revient à se demander quelle est la distance à partir de
laquelle on ne peut plus considérer que l'on est \enquote{proche}
d'une ligne électrique. Pour cet exemple nous avons fixé empiriquement
cette valeur à \SI{50}{\meter}. Le second seuil à fixer est la
distance à partir de laquelle le degré d'appartenance est nul. De
manière tout aussi empirique, nous avons fixé cette valeur à
\SI{100}{\meter}. Ainsi, entre 0 et 50 mètres, le degré d'appartenance
est de 1, puis il décroit jusqu’à être de 0 à 100 mètres et reste nul
au-delà, comme le montre la figure \ref{fig:fnc_app_Dist}.

La définition de la forme de la fonction d'appartenance pour la
\emph{relation de localisation atomique}
\onto[orla]{Al\-ti\-tu\-de\-Stric\-te\-ment\-Inf\-érieu\-re} ne pose
pas non plus de problèmes particuliers. Comme cette \emph{relation}
traduit une position relative sur l'axe vertical, le degré
d'appartenance doit être à 1 lorsque l'altitude de la position est
inférieure à celle de \emph{l'objet de référence} (\ie que la métrique
est négative) et nul dans le cas contraire (\ie que la métrique est
positive). Il faut alors définir les valeurs de la métrique à partir
desquelles le degré d'appartenance est de 1 ou de 0. Étant donné que
la différence d'altitude est calculée à partir du sommet de la ligne
électrique nous avons considéré que toute position dont l'altitude
était inférieure au point le plus proche de la ligne électrique (\ie
dès que la valeur de la métrique est négative) avait un degré
d'appartenance de 1. Nous avons fixé le second seuil à \SI{5}{\meter},
ce qui permet d'ajouter une petite marge d'erreur. Ainsi, le degré
d'appartenance est de 1 jusqu’à ce que la différence d'altitude soit
nulle, puis il décroit rapidement jusqu'à atteindre la valeur de 0
lorsque la différence est de \SI{5}{\meter} (\ie que la position est
\SI{5}{\meter} au-dessus du sommet de la ligne électrique), la valeur
reste nulle au-delà (Figure \ref{fig:fnc_app_AltInf}).

\begin{figure}
  \centering
  \subfloat{%
    \begin{tikzpicture}[scale=.7]
  \def\decalageX{-.2}
  \def\decalageY{-.2}
  % Courbe
  \begin{scope}[transparency group]
    % fond
    \begin{scope}
      \path[ffa] (0,2) -- (2.5,2) -- (5, 0)  -- (0,0) -- cycle;
    \end{scope}
    % bords
    \begin{scope}
      \path[ffc] (0,2) -- (2.5,2) -- (5, 0) -++ (3,0) ;
      \path[ffc_fade] (8,0) -- (9,0) ;
    \end{scope}
  \end{scope}
  % Axes X, Y
  \begin{scope}
    % Axe X
    \begin{scope}
      % Axe
      \draw[->] (0, \decalageX) --++ (9, 0) coordinate (x axis);
      % Graduations
      \foreach \n/\t in {0/{0},1/{20},2/{40},3/{60},4/{80},5/{100},6/{120},7/{140},8/{160}}
      {
        \draw[-] (\n, \decalageX - .05) --++ (0, .1);
        \node[below, font=\footnotesize] at (\n, \decalageX - .05) {\t};
      }
      % label
      \node[above left, yshift=.1cm, font=\small] at (x axis) {$Distance\ (m)$};
    \end{scope}
    % Axe Y
    \begin{scope}
      % Axe
      \draw[-] (\decalageY ,0) --++ (0, 2) coordinate (y axis);
      % Graduations
      \foreach \n/\t in {0/{0},2/{1}}
      {
        \draw[-] (\decalageY -.05, \n) --++ (.1, 0);
        \node[left, font=\footnotesize] at (\decalageY -.05, \n) {\t};
      }
      % Label
      \node[above] at (y axis) {$\mu$};
    \end{scope}
  \end{scope}
\end{tikzpicture}

    \label{fig:fnc_app_Dist}
  }
  \hfill%
  \subfloat{%
    \begin{tikzpicture}[scale=.7]
  \def\decalageX{-.2}
  \def\decalageY{-.2}
  % Courbe
  \begin{scope}[transparency group]
    % fond
    \begin{scope}
      \draw[ffa] (1,2) -- (4.5, 2) -- (5.5,0) -- (1,0)-- cycle;
      \draw[ffa_fade_m] (0,2) -- (1, 2) -- (1,0) -- (0,0)-- cycle;
    \end{scope}
    % bords
    \begin{scope}
      \path[ffc] (1, 2) --(4.5, 2) -- (5.5,0) -- (8,0);
      \path[ffc_fade_m] (0,2) -- (1,2);
      \path[ffc_fade] (8,0) -- (9,0) ;
    \end{scope}
  \end{scope}
  % Axes
  \begin{scope}
    % Axe X
    \begin{scope}
      % Axe
      \draw[<->] (0, \decalageX) --++ (9, 0) coordinate (x axis);
      % Graduations
      \foreach \n/\t in {.5/{-20},1.5/{-15},2.5/{-10},3.5/{-5},4.5/{0},5.5/{5},6.5/{10},7.5/{15},8.5/{20}}
      {
        \draw[-] (\n, \decalageX - .05) --++ (0, .1);
        \node[below, font=\footnotesize] at (\n, \decalageX - .05) {\t};
      }
      % label
      \node[above left, yshift=.1cm, font=\small] at (x axis) {$\Delta\,Altitude\,(m)$};
    \end{scope}
    % Axe Y
    \begin{scope}
      % Axe
      \draw[-] (\decalageY ,0) --++ (0, 2) coordinate (y axis);
      % Graduations
      \foreach \n/\t in {0/{0},2/{1}}
      {
        \draw[-] (\decalageY -.05, \n) --++ (.1, 0);
        \node[left, font=\footnotesize] at (\decalageY -.05, \n) {\t};
      }
      % Label
      \node[above] at (y axis) {$\mu$};
    \end{scope}
  \end{scope}
  % Analyse
  \begin{scope}
    \draw (4.5,0) -- (4.5,2);
    \draw[fill] (4.5,0) circle (1pt);
    \draw[fill] (4.5,2) circle (1pt);
    \node[above] at (4.5,2) {P};
  \end{scope}
\end{tikzpicture}

    \label{fig:fnc_app_AltInf}
  }
  \caption{Fonctions d'appartenance pour les \emph{relations de
      localisation}
    \protect\onto[orla]{Al\-ti\-tu\-de\-Stric\-te\-ment\-Inf\-érieu\-re}
    et
    \protect\onto[orla]{Dis\-tan\-ce\-Quan\-ti\-ta\-ti\-ve\-Pla\-ni\-mé\-tri\-que},
    utilisées pour construire la \emph{zone de localisation probable}
    de \emph{l'indice de localisation} \enquote{je suis sous une ligne
      électrique trois brins}.}
  \label{fig:fnc_app_sousProche}
\end{figure}

La construction de la \emph{zone de localisation probable} nécessite
de \emph{fusionner} ---~à l'aide d'une intersection~--- les deux
\emph{zones de localisations compatibles} correspondant aux
\emph{relations de localisation atomiques}
\onto[orla]{Al\-ti\-tu\-de\-Stric\-te\-ment\-Inf\-érieu\-re} et
\onto[orla]{Dis\-tan\-ce\-Quan\-ti\-ta\-ti\-ve\-Pla\-ni\-mé\-tri\-que}. Si
son appliquation est dépendante de l'implémentation, la méthode
utilisée est fixée par le modèle théorique. La \emph{théorie des
  sous-ensembles flous} nécessite d'employer une \emph{t-norme.} Étant
donné que cette comparaison ne porte que sur les implémenations et non
sur les opérateurs d'unions et d'intersection, nous avons choisi
d'utiliser les opérateurs les plus courants, ceux proposés par
\textcite{Zadeh1965}, où la fonction \emph{minimum} fait office de
\emph{t-norme.}

\subsection{Implémentation raster}

L'implémentation a l'avantage d'être extrêmement facile a mettre en
place. Une fois que les \emph{métriques} et les \emph{fonctions
  d’appartenance} ont été définies, elle se résume à calculer, pour
chaque pixel d'une grille échantillonnant l'espace, la valeur de la
\emph{métrique,} puis du degré d'appartenance à la \emph{zone de
  localisation compatible} \emph{spatialisée.} Cette opération est
répétée pour chaque \emph{relation de localisation atomique,} puis les
\emph{zones de localisation compatibles} résultantes sont
\emph{fusionnées} à l'aide des opérateurs d'intersection et de fusion
sélectionnés. Le seul choix imposé par l'utilisation de cette
implémentation et celui de la résolution du raster. Ce critère fixe la
taille et le nombre de pixels du raster et influe, par conséquent, sur
la précision et la durée du calcul. Cette résolution doit être
suffisamment grande pour permettre le calcul des fonctions
\emph{d'appartenance définies.} Par exemple, la \emph{fonction
  d'appartenance} définie pour la \emph{relation de localisation
  atomique}
\onto[orla]{Dis\-tan\-ce\-Quan\-ti\-ta\-ti\-ve\-Pla\-ni\-mé\-tri\-que}
(figure \ref{fig:fnc_app_Dist}) a deux seuils, \num{50} et
\SI{100}{\meter}.

L’application de la fonction d'appartenance (figure
\ref{fig:fnc_app_Dist}) à la métrique précédemment calculée (carte
\ref{fig:ISO_DIST_HT}) permet d'obtenir la carte
\ref{fig:ZLC_DIST_HT}, représentant la \emph{zone de localisation
  compatible} de la \emph{relation de localisation atomique}
\onto[orla]{Dis\-tan\-ce\-Quan\-ti\-ta\-ti\-ve\-Pla\-ni\-mé\-tri\-que}.
Le degré d'appartenance de chaque pixel à la \emph{zone de
  localisation compatible} est représenté par un cercle, centré sur le
pixel et dont le diamètre augmente avec celui-ci. Lorsque ce degré est
maximal, le diamètre du cercle est égal à la maille du raster (50
mètres dans le cas présent). Le diamètre est minimal lorsque le degré
d'appartenance est supérieur à zéro et les valeurs nulles sont
filtrées pour faciliter la lisibilité.

\begin{carte}
  \centering
  \input{../figures/Raster_ZLC_DIST_HT.tex}
  \caption{Mon beau raster}
  \label{fig:ZLC_DIST_HT}
\end{carte}

Une méthode identique est utilisée pour construire la \emph{zone de
  localisation compatible} issue de la \emph{spatialisation} de la
\emph{relation de localisation atomique}
\onto[orla]{Al\-ti\-tu\-de\-Stric\-te\-ment\-Inf\-érieu\-re}. La
\emph{métrique} identifiée (carte \ref{fig:ISO_DALT_HT}) est calculé
pour chaque pixel, puis ces valeurs sont transformées en \emph{degrés
  d'appartenance} à l'aide de fonction d'appartenance définie
précédemment (Figure \ref{fig:fnc_app_AltInf}). La carte
\ref{fig:ZLC_ALTINF_HT}, qui résulte de ce processus, représente alors
le degré d'appartenance de chaque pixel à la \emph{zone de
  localisation compatible.}

\begin{carte}
  \centering
  \input{../figures/Raster_ZLC_ALTINF_HT.tex}
  \caption{Mon très beau raster}
  \label{fig:ZLC_ALTINF_HT}
\end{carte}

Les rasters représentant les deux \emph{zones de localisation
  compatibles} à intersecter possédant les mêmes résolutions et
étendues, il y a bijection entre deux ensembles de pixels. On peut
alors, sans effectuer de ré-échantillonnage, appliquer un opérateur
d'intersection à chaque paire de pixels représentant la même
position. Après l'intersection des \emph{zones de localisation
  compatibles} figurées sur les cartes \ref{fig:ZLC_DIST_HT} et
\ref{fig:ZLC_ALTINF_HT} on obtient la \emph{zone de localisation
  probable} (carte \ref{fig:ZLP_SOUS_HT}) correspondant à l'ensemble
des positions situées \emph{à proximité} et à une \emph{altitude
  inférieure} de la ligne à haute-tension. Comme on peut le voir, le
résultat obtenu est semblable à la carte \ref{fig:ZLC_DIST_HT} et
assez différent de la carte \ref{fig:ZLC_ALTINF_HT}. L'étendue de la
\emph{zone de localisation probable} est donc plus contraint par
l'éloignement à la ligne électrique que par la différence
d'altitude. Pour le dire autrement, il y a peu de positions
\enquote{proches} et \enquote{au-dessus} d'une ligne électrique.

\begin{carte}
  \centering
  \input{../figures/Raster_ZLP_SOUS_HT.tex}
  \caption{Mon très très beau raster}
  \label{fig:ZLP_SOUS_HT}
\end{carte}


\subsection{Implémentation par \emph{Alpha-cuts}}

L'implémentation par \emph{alpha-cuts} s'est révelée beaucoup plus
difficile à mettre en place que l'implémentation par rasters.

L'approche par \emph{alpha-cuts} impose donc de calculer une limite



D'une part pour des soucis techniques. Aucun \ac{sig} ne permettant de
permettant de réaliser les traitements que nous envisagions, nous
avons été contrains de nous orienter sur des solutions de plus bas
niveau, comme la bibliothèque de calcul géométrique CGAL
\autocite{CGAL2019}.

\tdi{Revenir un peu sur les alpha-cuts et le lac}


\tdi{acuts et intersecions}

\missingfigure{Illustration modélisation concrète par alpha-cuts}

Pour cette comparaison nous avons choisi de construire trois
\emph{alpha-cuts,} celle dont le degré est égal à 1, délimitant le
\emph{noyau} du \emph{sous-ensemble flou,} celle dont le degré
d'appartenance est supérieur à 0, délimitant le \emph{support} et
enfin \emph{l'alpha-cut} de degré 0,5.

Pour chaque segment d'une ligne électrique, nous construisons un
polyèdre, représentant la limite tridimensionnelle de l'aire
appartenant à \emph{l'alpha-cut} considérée. Ce dernier est composé de
6 faces.

\missingfigure{Polyhère}

Cette manière de procéder n'impose évite de procéder à une phase de
fusion des \emph{zones de localisation compatibles.}

\subsection{Comparaison des approches et choix approche raster}

\paragraph{Comparaison des approches}

Les deux implémentations testées


Les résultats obtenus sont similaires

\missingfigure{Différence zones}



\tdi{problème des jointures}


\tdi{Difficulté de mise en place}

Au moment de la sélection d'une méthode de modélisation nous avons
estimé que le développement d'une méthode utilisant des vecteurs
serait bien plus coutueuses. Sur ce point nous avons préféré les
rasters. De plus il a été bc plus difficile d'aboutir à une
modélisation vectorielle permettant la comparaison (citer RIG).

\tdi{Difficulté d'extension}

Il semble plus difficile de développer des méthodes de spatialisation
très différentes en vecteur qu'en raster.

Il est également plus compliqué de tester de nouveaux couples
d'opérateurs flous en vecteur.

\tdi{Tendance à la fragmentation des alpha-cuts}

\tdi{Temps de calcul}

% Critère moyennement pertinent
Le critère du temps de calcul n'est pas fondamental pour ce
travail. Toutefois il nous semble pertinent de préciser que les deux
méthodes différent sur ce point.

Avantage raster si le nombre de cellules n'est pas trop
important. Mais le temps de calcul augmente rapidement. L'approche
vecteur est moins rapide mais le code est moins (et plus
difficilement) optimisé. Le résultat est assez mitigé.

\tdi{Volume des données}

Avantage vecteur. Le raster tend à prendre beaucoup de place, encore
plus si l'on souhaite stoquer les objets de référence en raster ou
sauvegarder des résutlats intermédiaires.

\tdi{Qualité de la modélisation}

Comparable, l'article de la RIG n'a pas permis de mettre en évidence
des différences notables. Par construction l'approche vectorielle est
plus précise, mais sa précision dépend aussi des données (MNT pour
l'altitude). À l'échelle du pixel les résultats sont comparables.

\tdi{OK pour les deux: Stabilité du modèle ? (par exemple si l'on fait
  des intersection avec le modèle egg-yolk on doit choisir ce qu'est
  le jaune. C'est l'intersection des jaunes ? ou il suffit qu'il y a
  ait un jaune?  est-ce qu'il faut introduire des valeurs
  intermédiaires (gris, lorsqu'on intersecte jaune et blanc) -> mais
  dans ce cas on sort du egg yolk, le modèle n'est pas stable
  alors). Le résultat doit pouvoir s'exprimer avec la même
  modélisation sans perte de sémantique et de finesse de
  modélisation.}


\paragraph{Choix du raster}


La comparaison de ces deux implémentations de la \emph{théorie des
  sous-ensembles flous} nous a conduit à adopter une démarche fondée sur
l'utilisation de rasters.

\tdi{Plus facile à développer}

\tdi{Plus facile à étendre}

\tdi{Plus rapide d'arriver à spatialiser -> possibilité de travailler
  sur plus d'alertes}

Le premier avantage de cette implémentation est d'ordre technique. Si
cette approche a tendance à être moins performante et plus
consommatrice de ressources informatiques, elle offre en retour
l'avantage d'être plus facile à mettre en place et à étendre. Ainsi,
le choix d'une approche raster nous permet de développer plus
rapidement une méthode fonctionnelle et donc de pouvoir nous
concentrer sur la spatialisation de plus d'alertes et non sur des
aspects plus techniques.

\tdi{Plus facile à utiliser (pas de fragmentation)}

Mais l'approche raster se distingue également par ces caractéristiques
XXXX. En réduisant toutes les opérations inter-\emph{zones de
  localisation} à des opérations arithmétiques composantes à
composantes, l'approche raster permet de conserver une certaines
stabilité du résultat, contrairement à l'approche par
\emph{alpha-cuts.}


Le choix d'utiliser une modélisation basée sur la \emph{théorie des
  sous-ensembles flous} et implémentée à l'aide de raster à des
conséquences théoriques et pratiques.

\tdi{Nécessité de choix des opérateurs flous}

Les différentes opérations inter-\emph{zones de localisation} peuvent
être réalisée avec différents opérateurs flous.

\tdi{Nécessité de définir une résolution}

De plus, l'utilisation de rasters pose un autre questionnement, celui
de la définition d'une résolution, dont dépendra la finesse, le poids
et la rapidité de construction des \emph{zones de localisation.}

%%% Local Variables:
%%% mode: latex
%%% TeX-master: "../../../../main"
%%% End:
