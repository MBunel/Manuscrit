Les \emph{zones de localisations} apparaissent dans la phase de
\emph{spatialisation} de notre méthode et sont traitées jusqu'à la fin
de la phase de \emph{fusion.} Ces deux phases, de la méthode ont des
contraintes spécifiques, nécessitant des caractéristiques
particulières de la part de la part de la \emph{modélisation} des
\emph{zones de localisation.} Nous allons ici présenter ces différents
critères, en les regroupant par phases de la méthode.

\subsection{Critères portant sur la \emph{spatialisation}}

Chacune des nombreuses \emph{relations de localisation} utilisées par
les requérants pour décrire leur position ont une sémantique qui leur
est propre et qui conduit à la définition de \emph{zones de
  localisation} très différentes. La modélisation utilisée doit donc
permettre la représentation de tous les différents cas possibles.

Un premier point à considérer est celui de \emph{l'imprécision} des
\emph{zones de localisations.} Si nous avons détaillé cette notion et
justifié sa prise en compte lors de la \emph{spatialisation,} nous
n'avons pas présenté ses possibles variations. Or, de la même manière
que la forme d'une \emph{zone de localisation} varie d'une
\emph{relation de localisation} à une autre, leur \emph{imprécision}
peut être plus ou moins importante. Par exemple, la \emph{relation de
  localisation} \enquote{proche} \footnote{Formalisée dans l'ontologie
  des r\emph{elations de localisation} par le concept
  \onto[orl]{Prés\-De}.}, qui décrit une relation de proximité
géographique entre \emph{le sujet} et \emph{l'objet de référence,} est
fortement \emph{imprécise} \footnote{Comme nous l'illustrions avec le
  \emph{paradoxe sorites} dans le second chapitre (voir la section
  \ref{subsec:2-1-2}).}. Ainsi, les limites de la \emph{zone de
  localisation} y correspondant sont continues, sans réelle rupture
entre son intérieur et son extérieur. À l'inverse, la \emph{zone de
  localisation} construite à partir de la \emph{relation de
  localisation} \enquote{en dessous} \footnote{Formalisée avec le
  concept \onto[orl]{Sous\-Al\-ti\-tu\-de}.} a des limites plus
\emph{précises.} On peut en effet imaginer une situation où l'altitude
du \emph{sujet,} alors clairement situé \enquote{en dessous} de
\emph{l'objet de référence}, augmente peu à peu, jusqu'à dépasser
celle de \emph{l'objet de référence,} passant alors de la \emph{zone
  de localisation} à son extérieur. Cependant, contrairement à la
\emph{zone de localisation} construite à partir de la \emph{relation
  de localisation} \enquote{proche}, cette transition est, sinon
immédiate\footnote{Ce qui impliquerait que la \emph{relation de
    localisation} est \emph{précise,} ce qui, comme nous l'expliquions
  dans les chapitres \ref{chap:02} et \ref{chap:03}, n'est pas
  possible.}, rapide. Ainsi, s'il est abusif de considérer que
l'appartenance à cette \emph{zone de localisation} est bivalente, on
peut remarquer que la longue transition entre l'intérieur et
l'extérieur de la \emph{zone de localisation} construite d'après la
\emph{relation de localisation} \onto[orl]{Prés\-De} est ici fortement
réduite. Entre ces deux cas archétypaux on trouve une multitude
d'autres situations, où les limites de la \emph{zone de localisation}
sont plus ou moins marquées. Il est nécessaire que notre modélisation
des \emph{zones de localisation} propose une prise en compte
satisfaisante de ces différentes situations et soit donc à même de
modéliser des \emph{zones de localisation} aux limites presque
\emph{nettes} ou fortement \emph{imprécises.}

Un problème similaire se pose lorsque l'on traite, non plus de la
vitesse de transition ente l'intérieur d'une \emph{zone de
  localisation} et son extérieur, mais de sa forme. Une \emph{relation
  de localisation} comme \enquote{proche} voit, toutes choses égales
par ailleurs, son \textbf{degré de validité} diminuer linéairement
avec l'éloignement à \emph{l'objet de référence.} On ne s'éloigne pas
plus (ou moins) vite à 5 mètres qu'a 50 mètres. Cette observation est
également vraie pour la \emph{relation de localisation} \enquote{en
  dessous}. La zone \emph{d'imprécision} est certes plus réduite, mais
la relation entre la différence d'altitude est l'appartenance à la
\emph{zone de localisation} est linéaire, il n'y a ni cassures, ni
variations. Cependant, la linéarité de cette relation n'implique qu'il
en soit de même pour les frontières. En effet, contrairement à la
\emph{relation de localisation} \enquote{proche}, qui ne dépend que
d'une distance planaire dans un espace que l'on considère généralement
comme isotrope, la \emph{zone de localisation} issue de la
\emph{spatialisation} de la \emph{relation} \enquote{en dessous}
dépend des variations du relief, tout sauf linéaires. On pourrait
augurer que cette considération n'a un intérêt que conceptuel et que
l'on pourrait se contenter de faire l'hypothèse que la transition
entre l'intérieur et l'extérieur de la \emph{zone de localisation} est
toujours linéaire. Toutefois c'est oublier que l'appartenance à la
\emph{zone de localisation} n'a pas à être strictement décroissante à
mesure que l'on s'en éloigne. Ainsi, une \emph{zone de localisation}
construite à partir de la \emph{relation de localisation} \enquote{en
  dessous} pourrait posséder une \emph{zone de transition} avec des
variations importantes, émaillée de parties presque situées en dehors
(ou dans) de la \emph{zone de localisation} (\ie dont l'altitude est
proche, voir légèrement supérieure, à celle de \emph{l'objet de
  référence}). Si ces variations sont suffisamment importantes pour
que cette région soit clairement située à l'extérieur (ou à
l'intérieur) de la \emph{zone de localisation,} le résultat sera alors
fragmenté, mais sans que cela ne pose de de problèmes
particuliers. Par contre si ces variations restent suffisamment
faibles pour rester dans la \emph{zone de transition,} elles ne sont
pas prises en compte par une modélisation simplifiant à outrance la
zone de transition. Ainsi, en fonction de la morphologie de la région
étudiée, les \emph{zones de localisation} pourront être fragmentées ou
compactes et avoir des zones frontières plus ou moins larges (en
fonction de la pente) et plus ou moins linéaires (en fonction de la
forme du relief). Ainsi, nous ne pouvons nous contenter d'une
modélisation ne permettant que la manipulation de \emph{zones de
  localisations} aux frontières linéaires, de nombreuses autres
configurations étant possibles.

% \begin{figure}
%   \centering
%   \begin{tikzpicture}

  \def\decalageX{-.2}
  \def\decalageY{-.2}

  \begin{scope}[scale=1, xshift=6, yshift=1cm]
    % Courbe
    \begin{scope}[transparency group]
      % fond
      \begin{scope}
        \path[ffa] (0,0)--++ (0,2) -- (4.5, 0) -- cycle;
      \end{scope}
      % bords
      \begin{scope}
        \path[ffc] (0, 2) -- (4.5, 0);
        \path[ffc_fade] (4.5,0) -- (5,0);
      \end{scope}
    \end{scope}
    % Axes X, Y
    \begin{scope}
      % Axe X
      \begin{scope}
        % Axe
        \draw[->] (0, \decalageX) --++ (5, 0) coordinate (x axis);
        % Graduations
        \foreach \n/\t in
        {0/{}}
        { \draw[-] (\n, \decalageX - .05) --++ (0, .1); \node[below,
          font=\footnotesize] at (\n, \decalageX - .05) {\t}; }
        % label
        \node[below left] at (x axis) {$Truc$};
      \end{scope}
      % Axe Y
      \begin{scope}
        % Axe
        \draw[-] (\decalageY ,0) --++ (0, 2) coordinate (y axis);
        % Graduations
        \foreach \n/\t in {0/{0},2/{1}}
        {
          \draw[-] (\decalageY -.05, \n) --++ (.1, 0);
          \node[left, font=\footnotesize] at (\decalageY -.05, \n) {\t};
        }
        % Label
        \node[above] at (y axis) {$\mu$};
      \end{scope}
    \end{scope}
  \end{scope}

  % Axes
  \draw[<->] (0,0) -- (12,0) node[pos=0] {Peu précis} node[pos=1]
  {Très imprécis};

  \draw[<->] (6,2.5) -- (6,-2.5) node[pos=0] {Peu précis} node[pos=1]
  {Très imprécis};
  
\end{tikzpicture}
%   \caption{dsqdsq}
%   \label{fig:temp}
% \end{figure}

\subsection{Critères portant sur la \emph{fusion}}

La méthodologie de construction des \emph{zones de localisation}
présentée dans le \autoref{chap:04} et plus précisément la \emph{phase
  de fusion,} impose d'effectuer de combiner des \emph{zones de
  localisation} par des unions et des intersections. La modélisation
retenue doit donc permettre de réaliser ces opérations, mais également
garantir leur commutativité. En effet, l'utilisation d'un opérateur
d'union ou d'intersection non commutatif aurait pour conséquence de
modifier le résultat de la \emph{phase de fusion} en fonction de
l'ordre de traitement des \emph{zones de localisation
  compatibles}. Par exemple, si l'on souhaite construire la \emph{zone
  de localisation probable} ($\text{\textsl{zlp}}$) correspondant à
l'ensemble d'indices de localisation : \enquote{Je suis à côté d'une
  maison et dans une forêt}. Pour ce faire il est nécessaire
d'identifier les positions situées dans les \emph{zone de localisation
  compatibles} correspondant au premier ($\text{\textsf{zlc}}_{i1}$)
et au second ($\text{\textsf{zlc}}_{i2}$) indice, à l'aide d'un
opérateur d'intersection. Si cet opérateur n'est pas commutatif, alors
l'intersection de $\text{\textsf{zlc}}_{i1}$ avec
$\text{\textsf{zlc}}_{i2}$ pourra donner un résultat différent de
l'intersection de $\text{\textsf{zlc}}_{i2}$ avec
$\text{\textsf{zlc}}_{i1}$. Un tel comportement est incompatible avec
plusieurs des principes que nous avons définis, notamment l'autonomie
de la spatialisation ou le \emph{principe de décomposition,} qui se
fondent sur l'hypothèse ---~implicite~--- d'une commutativité des
opérateurs utilisés durant la phase de fusion.

\tdi{Nécessité que les résultats soient facilement
  interprétables. Sémantique des résultats et pas trop de zones (par
  exemple beaucoup d'alpha cuts).}

De plus, il est nécessaire que les \emph{zones de localisation}
produites durant la \emph{phase de fusion} restent interprétables.


Il faut que le résultat des ces fusions soit interprétable est puisse
être exprimé avec la modélisation choisie.

%%% Local Variables:
%%% mode: latex
%%% TeX-master: "../../../../main"
%%% End:
