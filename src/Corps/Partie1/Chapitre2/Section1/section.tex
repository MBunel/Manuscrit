Nous avons identifié cinq objectifs principaux pour ce travail de
recherche.

On trouvera une synthèse des objectifs scientifiques, ainsi que de
leurs verrous et des contributions attendues dans le
\autoref{tab:synthese_objectifs} page
\pageref{tab:synthese_objectifs}.

\subsection{Modélisation de localisations indirectes}
\label{subsec:2-1-1}

Un permier objectif de cette thèse est 

\subsubsection{Contexte scientifique}

Vasardini
Bloch
Brézil icc

\subsubsection{Verrous}

\paragraph{Peu de données et Faible recoupement}

présentés en détail dans le \autoref{chap:05}

Un second problème, intimement lié au manque de données, est leur
faible recoupement. En effet, les enregistrements d'alerte dont nous
disposons ne concerent jamais les mêmes zones. Et les mêmes points


\paragraph{Disparité des descriptions de localisation}

Deux personnes ne décrivent pas une position de la même façon

\paragraph{Descriptions non exhaustives}

Pas de présuppositions sur ce qui est dit.

le requérant ne dit pas tout

-> complexifie le raisonnement


\subsection{Prise en compte de l'imprécision}
\label{subsec:2-1-2}

La question de la spatialisation nécessiter cependant d'en s'en poser
une seconde, celle de la prise en compte de l'imprécision. En effet,
les positions que nous cherchons à spatialiser possèdent la
particularité d'être exprimées en langage naturel. Orn bien que nous
ne travaillons pas directement à l'interprétation du langage naturel,
\ie que nous ene faisons pas de TAL, nous sommes tout de même
confrontés à certaines de ces limites.

La principale d'entre-elle est la question de l'imprécison du langage
naturel. En effet, le langage naturel n'est pas un outil idéal pour
exprimer clairement des propositions, celle-ci sont toujours
soumisses à l'interpréatation et donc imprécises. Il en va
nécessairement de même pour les propositions décrivants dune ou des
positions? Ces dernières s'expriment en langage natuurel et son donc
soumisses à l'iumprécisions.

Dans les faits ,cela implique qu'une même phrase, un même mot,
peutvent -être utilisés pour décrire des localisations dans l'espace
différentes, voir même totalement incompatibles.

Cette nature du langage naturel peut poser plusieurs soucis. D'une
part il est impossible d'utiliser la langue comme un outil de
raisonement précis. C'est d'ailleurs ce constat qui a conduit les
philosophes analytiques, tels que Fredge ou Russel à travailler à la
formalisation de la logique, outil à même d'affranchir la réflexion
des limites du langage naturel.



\subsubsection{Contexte scientifique}

Travail sur le flou
Russell
Varzi
Zadeh
Dilo ?


\subsubsection{Verrous}

\paragraph{Quantification de l'imprécision}

\subsection{Prise en compte du jugement du secouriste}
\label{subsec:2-1-3}

\subsubsection{Contexte scientifique}

Travaux sur l'incertitude

Ana-Maria

\subsubsection{Verrous}

\paragraph{Quantification de l'incertitude}


\subsubsection{Apports envisagés}

\subsection{Agrégation d'indices}
\label{subsec:2-1-4}

\subsubsection{Contexte scientifique}

\subsubsection{Verrous}

\paragraph{Contexte différent}

\paragraph{Conflits}

\subsection{Évaluation des résulats}
\label{subsec:2-1-5}

Enfin, le dernier objectif scientifique de cette thèse consiste à
définir une méthode d'évaluation des résultats produits.

L'objectif de cette évaluation est de permettre une adaptation des
résultats en fonction 

\subsubsection{Contexte scientifique}

\subsubsection{Verrous}

\paragraph{Pas d'évaluation supervisée}


% Tableau synthétique
\begin{table}[h]
  \centering
  \begin{tabular}{p{.2\textheight}>{\small}L{.35\textheight}>{\small}L{.35\textheight}} \toprule
\multicolumn{1}{c}{\bfseries Objectif scientifique} &
\multicolumn{1}{c}{\normalsize\bfseries Verrous} & \multicolumn{1}{c}{\normalsize\bfseries
Apports envisagés} \\ \midrule
% Sémantique des relations spatialesss
  \addlinespace
  \nameref{subsec:2-1-1}
{\par\footnotesize\hspace{.25cm}$\longrightarrow$~Chapitre
\ref{chap:07}} & \begin{minipage}[t]{.35\textheight}
    \begin{itemize}
    \item Variation sémantique des \emph{relations de localisation}
    \item Faible redondance des \emph{indices de localisation}
    \end{itemize}
  \end{minipage} & \begin{minipage}[t]{.35\textheight}
    \begin{itemize}
    \item Recensement :
      \begin{itemize}
      \item des \emph{relations de localisation}
      \item des \emph{objets de référence}
      \end{itemize}
    \item Identification de la sémantique des \emph{relations de
        localisation}
    \item Définition d'une méthode de \emph{spatialisation}
    \end{itemize}
  \end{minipage} \\
  %\addlinespace[.5cm]
  %
  % Prise en compte imprécision
  \nameref{subsec:2-1-2}
{\par\footnotesize\hspace{.25cm}$\longrightarrow$~Chapitre
\ref{chap:07}} & \begin{minipage}[t]{.35\textheight}
    \begin{itemize}
    \item Intégration de \emph{l'imprécision} à la \emph{spatialisation}
    \item Quantification de \emph{l'imprécision} des \emph{relations
        de localisation}
    \end{itemize}
  \end{minipage} & \begin{minipage}[t]{.35\textheight}
    \begin{itemize}
    \item Définition d'une méthodes :
      \begin{itemize}
      \item de prise en compte de \emph{l’imprécision} des
        \emph{relations de localisation}
      \item de quantification de \emph{l'imprécision} des
        \emph{relations de localisation}
      \end{itemize}
    \end{itemize}
  \end{minipage} \\
  %\addlinespace[.5cm]
  %
  % Incertitude
  \nameref{subsec:2-1-3}
{\par\footnotesize\hspace{.25cm}$\longrightarrow$~Chapitre
\ref{chap:08}} & \begin{minipage}[t]{.35\textheight}
    \begin{itemize}
    \item Intégration de \emph{l'incertitude} à la
      \emph{spatialisation}
    \item Quantification de \emph{l'incertitude} des \emph{indices de
        localisation}
    \end{itemize}
  \end{minipage}& \begin{minipage}[t]{.35\textheight}
    \begin{itemize}
    \item Définition de méthodes :
      \begin{itemize}
      \item de prise en compte de \emph{l’incertitude} des
        \emph{indices de localisation}
      \item de quantification de \emph{l'incertitude} des
        \emph{indices de localisation}
      \end{itemize}
    \end{itemize}
  \end{minipage} \\
  %\addlinespace[.5cm]
  %
   \nameref{subsec:2-1-4}
{\par\footnotesize\hspace{.25cm}$\longrightarrow$~Chapitre
\ref{chap:08}} & \begin{minipage}[t]{.35\textheight}
    \begin{itemize}
    \item Compatibilité avec la spatialisation
    \item Gestion des conflits
    \end{itemize}
  \end{minipage}& \begin{minipage}[t]{.35\textheight}
    \begin{itemize}
    \item Définition d'une méthode de fusion
    \end{itemize}
  \end{minipage} \\
  %\addlinespace[.5cm]
  %
  \nameref{subsec:2-1-5}
{\par\footnotesize\hspace{.25cm}$\longrightarrow$~Chapitre
\ref{chap:11}} & \begin{minipage}[t]{.35\textheight} \small
    \begin{itemize}
    \item Risque d'évaluer les \emph{indices}.
    \item Intégration dans le contexte métier.
    \end{itemize}
  \end{minipage}& \begin{minipage}[t]{.35\textheight}
    \begin{itemize}
    \item Développement d'une méthode d'évaluation des \emph{zones de
        localisation}
    \end{itemize}
  \end{minipage}\\
  \bottomrule
\end{tabular}

  \caption{Synthèse des verrous et des apports attendus pour chaque
    objectif scientifique de la thèse}
  \label{tab:synthese_objectifs}
\end{table}

%%% Local Variables:
%%% mode: latex
%%% TeX-master: "../../../../main"
%%% End:
