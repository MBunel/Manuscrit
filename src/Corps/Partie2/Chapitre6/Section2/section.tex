Tous les modèles théoriques que nous avons présenté dans l'état de
l'art de cette thèse ne sont pas nécessairement adaptés aux exigences
de modélisation que nous avons formulé.

\tdi{Les modèles discrets (exacts, ensembles épais, autres ?)}

Par exemple, les modèles \enquote{exacts}, comme les modèles
\emph{egg-yolk} \autocite{Cohn1996} ou \emph{min-max}
\autocite{Clementini1996}, imposent une modélisation discrète de
\emph{l'imprécision} spatiale. Or, si ces solutions ont pour avantage
leur grande simplicité, ils ne permettent pas de représenter
efficacement les différents cas que nous avons présenté. Si les
modèles \enquote{exacts} permettent de modéliser toute forme de
\emph{zone de localisation} \footnote{À l'exception de zones
  \emph{nettes,} qui ne sont modélisables qu'avec le modèle de
  \textcite{Bejaoui2009}. Cette caractéristique n'est cependant pas
  limitante dans notre cas, aucune \emph{relations de localisation} ne
  pouvant définir une zone précise.} comme nous le souhaitons, il leur
est impossible de rendre compte de la variation de
\emph{l'imprécision,} la zone frontière n'étant définie que par ses
bornes supérieures et inférieures. Nous devons donc rejeter ces
modèles théoriques.

\missingfigure{comparaison décroissance degré d'appartenace en
  fonction de la pente. Voir si possible de le faire avec une arte
  topo et un vrai relief. Illustrer l'hypothèse implicite d'évolution
  linéaire entre les frontières dans les modèles exacts.}

\tdi{Modèles non linéaires et non flous}

\tdi{Modèle flous plus éprovés dans la littérature}

Les modèles flous ont également l'avantage d'avoir été fortement
éprouvés dans la littérature

\tdi{Dire ques les modèles flous sont assez au point sur les opération
  de fusion avec les opérateurs flous qu'il définissent.}

%%% Local Variables:
%%% mode: latex
%%% TeX-master: "../../../../main"
%%% End:
