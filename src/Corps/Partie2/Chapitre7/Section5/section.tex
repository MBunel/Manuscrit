Dans le \autoref{chap:05}, nous avions abordé la question de la
représentation des règles de décomposition des \emph{relations de
  localisation} et avions décidé de les représenter dans une ontologie
\emph{ad hoc,} \ac{orla}. Cette solution présentait le double avantage
de proposer une formalisation des règles qui soit lisible par un être
humain, tout en les centralisant, ce qui, à terme favorise leur
compréhension, leur maintient et leur évolution. Une question
similaire se pose à présent, celle de la représentation des règles de
\emph{spatialisation} des \emph{relations de localisation atomiques.}
Il est à nouveau possible de représenter ces règles dans une
ontologie, nous avons donc décidé de les intégrer à \ac{orla}. Pour
permettre une bonne représentation des règles de
\emph{spatialisation,} il est nécessaire de spécifier quelques
contraintes, comme nous l'avions fait dans la
\autoref{chap:05-sec1}. Tout d'abord, toute \emph{relation de
  localisation atomique} doit, pour être spatialisée, être liée à un
\emph{rasteriser,} une \emph{métrique,} un \emph{fuzzyficateur} et,
optionnellement à un \emph{modifieur.} Comme pour les \emph{relations
  de localisation} nous ne permettons pas l’existence de règles
concurrentes, une \emph{relation de localisation atomique} donnée
ne correspond qu'à une \emph{méthode de spatialisation} et inversement
\footnote{Conformément à la règle d'unicité des \emph{relations de
  localisations atomiques} formulée dans le \autoref{chap:05}.}

Pour représenter les règles de spatialisation dans \emph{l'ontologie
  des relations de localisation atomiques} nous avons défini quatre
nouvelles relations \onto[orla]{has\-Rasteriser},
\onto[orla]{has\-Metrique}, \onto[orla]{has\-Fuzzyfier} et
\onto[orla]{has\-Modifieur}. Chacune de ces nouvelles relations lie
une \emph{relation de localisation atomique} à une classe
représentant, respectivement, une méthode de \emph{rastérisation,} une
méthode de calcul d'une \emph{métrique,} une méthode de
\emph{fuzzyfication} et une méthode appliquant un \emph{modifieur}
(\autoref{fig:structure_spatialisation_ontologie}). Ainsi, les
\emph{relations de localisation atomiques} ne sont pas seulement
définissables comme des \emph{relations de localisation} desquelles ne
proviennent aucune relation
\onto[orla]{has\-Relation\-Spatiale\-Atomique} \footnote{Comme nous
  l'indiquions dans le \autoref{chap:05}.}, mais également comme des
\emph{relations de localisation} auxquelles sont associés un
\emph{rastériser,} une \emph{métrique} et un \emph{fuzzyfier} (\ie à
l'origine d'une relation \onto[orla]{has\-Rasteriser},
\onto[orla]{has\-Metrique} et \onto[orla]{has\-Fuzzyfier}).

\begin{figure}
  \centering
  \tdi{Ajout liens modifieurs}
  \input{../figures/structure_spatialisation_ontologie.tex}
  \caption{xxx}
  \label{fig:structure_spatialisation_ontologie}
\end{figure}

La \emph{spatialisation} d'une \emph{relation de localisation
  atomique} peut alors être simplement effectuée en calculant,
successivement, la \emph{rastérisation,} la \emph{métrique,} sa
\emph{fuzzyfication} et, le cas échant, le \emph{modifieur,} définis
dans \ac{orla} pour cette \emph{relation.} Par exemple, la
\emph{relation de localisation atomique}
\onto[orla]{Sous\-Al\-ti\-tu\-de} ---~notamment utilisée pour définir
la \emph{relation} \onto[orla]{Sous\-Pro\-che\-De} (\autoref{chap:05})
\todo{Changer par une référence à la figure de décomposition lorsque
  définie}~--- est spatialisée à l'aide du \emph{rasteriser}
\onto[orla]{Geometrie}, de la \emph{métrique}
\onto[orla]{Delta\-Nearest\-Val} et du \emph{fuzzyficateur}
\onto[orla]{Inf\-Zero}
(\autoref{fig:ex_parties_statialisation_sousalt}).

\begin{figure}
  \centering
  \input{../figures/parties_spatialisation_ontologie.tex}
  \caption{Exemple de la XXX de la \emph{relation de localisation
      atomique} \protect\onto[orla]{Sous\-Al\-ti\-tu\-de}.}
  \label{fig:ex_parties_statialisation_sousalt}
\end{figure}

La \emph{relation de localisation atomique} \onto[orla]{Dans\-Pla\-ni\-mé\-tri\-que} est quant à elle \emph{spatialisée} avec le même \emph{rasteziser} (\onto[orla]{Geometrie}), la \emph{métrique} \onto[orla]{XX} et le \emph{fuzzyficateur} \onto[orla]{XX} (\autoref{fig:ex_parties_statialisation_dansplani}).

\begin{figure}
  \centering
  \input{../figures/exemple_parties_spatialisation_dansPlanimetrique.tex}
  \caption{Exemple de la XXX de la \emph{relation de localisation
      atomique} \protect\onto[orla]{Dans\-Pla\-ni\-mé\-tri\-que}.}
  \label{fig:ex_parties_statialisation_dansplani}
\end{figure}

La \emph{relation de localisation atomique} \onto[orla]{Not\-A\-La\-Fron\-ti\-ere\-De} est quant à elle définie avec un \emph{modifieur,} \onto[orla]{Not}, qui permet de la distinguer de son opposée, \onto[orla]{A\-La\-Fron\-ti\-ere\-De}, utilisant le même \emph{rasteriser}, \onto[orla]{Geometrie}, la même \emph{métrique,} \onto[orla]{XX} et le même \emph{fuzzyficateur,} \onto[orla]{XX} (\autoref{fig:ex_parties_statialisation_notalafrontierede}).

\begin{figure}
  \centering
  \begin{tikzpicture} 
  \node[anchor=east] (x1) at (-3.5,0) {\onto[orla]{NotALaFrontiereDe}};

  \node[anchor=west] (y1) at (3.5,1.2) {\onto[orla]{Frontiere}};
  \node[anchor=west] (y2) at (3.5,0.4) {\onto[orla]{Distance}};
  \node[anchor=west] (y3) at (3.5,-0.4) {\onto[orla]{Eq\-Val\-0}};
  \node[anchor=west] (y4) at (3.5,-1.2) {\onto[orla]{Not}};
  
  \fill[RdBu-9-9] (x1.east) circle (2pt);
  
  \fill[RdBu-9-1] (y1.west) circle (2pt);
  \fill[RdBu-9-1] (y2.west) circle (2pt);
  \fill[RdBu-9-1] (y3.west) circle (2pt);
  \fill[RdBu-9-1] (y4.west) circle (2pt);

  \draw[->, shorten >=5pt, shorten <=5pt] (x1.east) |- (y1.west)
  node[pos=0.75, sloped, anchor=center, fill=white] {\tiny\emph{A pour rasteriser}};
  
  \draw[->, shorten >=5pt, shorten <=5pt] (x1.east) |- (y2.west)
  node[pos=0.75, sloped, anchor=center, fill=white] {\tiny\emph{A pour métrique}};

  \draw[->, shorten >=5pt, shorten <=5pt] (x1.east) |- (y3.west)
  node[pos=0.75, sloped, anchor=center, fill=white] {\tiny\emph{A pour
      fuzzyficateur}};
  \draw[->, shorten >=5pt, shorten <=5pt] (x1.east) |- (y4.west)
  node[pos=0.75, sloped, anchor=center, fill=white] {\tiny\emph{A pour Modifieur}};
\end{tikzpicture}
  \caption{Exemple de la XXX de la \emph{relation de localisation
      atomique} \protect\onto[orla]{Not\-A\-La\-Fron\-ti\-ere\-De}.}
  \label{fig:ex_parties_statialisation_notalafrontierede}
\end{figure}

%%% Local Variables:
%%% mode: latex
%%% TeX-master: "../../../../main"
%%% End:
