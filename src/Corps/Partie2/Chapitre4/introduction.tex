Ce chapitre se destine à présenter la structure générale de notre
méthode, conçue pour transformer une \emph{description de position} en
une zone de \emph{localisation probable.} La seconde partie de ce
manuscrit étant construite suivant une logique de complexification
graduelle, nous ne présenterons ici que les grands aspects de cette
méthode, qui sera complétée au fur et à mesure de cette partie.

Pour construire notre méthode, nous avons adopté une démarche par
contraintes. La première partie de ce chapitre détaille quatre grands
principes, qui nous semblent nécessaires au développement d'une
méthode robuste, intelligible et adaptée à une utilisation
professionnelle. Ces quatre principes, couplés aux objectifs
scientifiques de la thèse (\autoref{chap:02}), forment une ossature
conceptuelle, permettant de guider la définition de notre méthode.

La première partie de ce chapitre est dédiée à la présentation de ces
principes de modélisation. Une seconde partie sera destinée à la
formalisation du concept \emph{d'indice de localisation} au centre de
notre méthode. Enfin, la troisième partie de ce chapitre est destinée
à la définition de cette dernière.
%%% Local Variables:
%%% mode: latex
%%% TeX-master: "../../../main"
%%% End:
