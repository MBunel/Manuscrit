Ce chapitre nous a permis de définir les grandes lignes de notre
méthode de construction d'une \emph{zone de localisation probable} à
partir d'une description de position.

Cette méthode est organisée autour de trois phases distinctes et
successives (\emph{décomposition,} \emph{spatialisation} et
\emph{fusion}). C'est au cours de la première d'entre-elles que les
\emph{indices de localisation} saisis et transmis par le secouriste
sont traités en vue d'obtenir un ensemble d'indices décomposés et
indépendants qui seront ensuite \emph{spatialisés.} Cette
décomposition, qui s'applique aux \emph{relations de localisation non
  atomiques} et aux \emph{objets de référence indéfinis} conduit à la
multiplication des \emph{indices de localisation} à \emph{spatialiser}
mais offre en revanche la possibilité de les traiter de manière
totalement indépendante, les résultats d'un \emph{spatialisation}
n'ayant d'impact sur les autres \emph{spatialisation} que lors de la
dernière phase de la méthode, la \emph{fusion.}  Cette dernière phase
permet de regrouper les différentes \emph{zones de localisation
  compatibles,} jusqu'à aboutir à une seule zone, la \emph{zone de
  localisation probable.}

Telle que présentée, cette méthode ne permet pas de prendre en compte
\emph{l'incertitude} et \emph{l'imprécision.} Cependant ces différents
éléments serons ajoutés à la méthode au fur et à mesure des chapitres
de cette seconde partie.
