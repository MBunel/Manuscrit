Ces nouvelles connaissances peuvent être représentées dans l'ontologie

Pour représenter ces connaissances dans \emph{l'ontologie des relations de localisation atomiques} nous avons défini quatre nouvelles relations ---~\onto[orla]{has\-Rasteriser}, \onto[orla]{has\-Metrique}, \onto[orla]{has\-Fuzzyfier} et \onto[orla]{has\-Modifieur}~---, qui s'ajoutent à la relation \onto[orla]{has\-Re\-la\-tion\-Spatiale\-Atomique} précédemment présentée (\autoref{chap:05}). Chacune de ces nouvelles relations lie une \emph{relation de localisation atomique} à une classe représentant, respectivement, une méthode de \emph{rastérisation,} une méthode de calcul d'une \emph{métrique,} une méthode de \emph{fuzzyfication} et une méthode appliquant un \emph{modifieur} (\autoref{fig:structure_spatialisation_ontologie}). La spatialisation d'une \emph{relation de localisation atomique} peut alors être effectuée en calculant, successivement, la \emph{rastérisation,} la \emph{métrique,} sa \emph{fuzzyfication} et, le cas échant, le \emph{modifieur,} définis dans \ac{orla} pour cette \emph{relation.}

Ainsi, les \emph{relations de localisation atomiques} ne sont pas seulement définissables comme des \emph{relations de localisation} desquelles ne proviennent aucune relation \onto[orla]{has\-Relation\-Spatiale\-Atomique} \footnote{Comme nous l'indiquions dans le \autoref{chap:05}.}, mais également comme des \emph{relations de localisation} auxquelles sont associés un \emph{rastériser,} une \emph{métrique} et un \emph{fuzzyfier} (\ie à l'origine d'une relation \onto[orla]{has\-Rasteriser}, \onto[orla]{has\-Metrique} et \onto[orla]{has\-Fuzzyfier}).

\begin{figure}
  \centering
  \tdi{Ajout liens modifieurs}
  \begin{tikzpicture}
  \usetikzlibrary{shapes.geometric,fit}

  \begin{scope}[xshift=-3.5cm]
    \node (x1) at (0,0.8) {$rla_1$};
    \node (x2) at (0,0) {$rla_2$};
    \node (x3) at (0,-0.8) {$rla_3$};
  \end{scope}
  
    \begin{scope}
      \node (y1) at (3.5,.8) {$rla_1$};
      \node (y2) at (3.5,0) {$rla_2$};
      \node (y3) at (3.5,-0.8) {$rla_3$};
    \end{scope}
  
    \fill[RdBu-9-9] (x1.east) circle (2pt);
    \fill[RdBu-9-9] (x2.east) circle (2pt);
    \fill[RdBu-9-9] (x3.east) circle (2pt);
    
    \fill[RdBu-9-1] (y1.west) circle (2pt);
    \fill[RdBu-9-1] (y2.west) circle (2pt);
    \fill[RdBu-9-1] (y3.west) circle (2pt);

     \node[fit=(x1.east) (x3.west), ellipse, draw, minimum width=2cm,
    ffc2] (c1) {};
    \node[fit=(y1.east) (y3.west), ellipse, draw, minimum width=2cm,
    ffc] (c2) {};

    % draw the arrows
    \draw[->, shorten >=5pt, shorten <=5pt] (x1.east) -- (y1.west)
    node[pos=0.75, sloped, anchor=center, fill=white] {\tiny\emph{se
        décompose en}};
    
    \draw[->, shorten >=5pt, shorten <=5pt] (x1.east) -- (y2.west)
    node[pos=0.5, sloped, anchor=center, fill=white] {\tiny\emph{se
        décompose en}};
    
    \draw[->, shorten >=5pt, shorten <=5pt] (x2.east) -- (y3.west)
    node[pos=0.5, sloped, anchor=center, fill=white] {\tiny\emph{se
        décompose en}};

    \draw[->, shorten >=5pt, shorten <=5pt] (x3.east) -- (y3.west)
    node[pos=0.5, sloped, anchor=center, fill=white] {\tiny\emph{se
        décompose en}};

    \draw[->, shorten >=5pt, shorten <=5pt] (x2.east) to[out=30, in=-30, looseness=1.5] (x1.east);
    

    \node[text width=3.5cm, anchor=center,align=center] at
    ([yshift=-.5cm]c1.south) {\small Ensemble des \emph{relations de
        localisation}}; \node[text width=4cm,
    anchor=center,align=center] at ([yshift=-.5cm]c2.south) {\small
      Ensemble des \emph{Relations de localisation atomiques}};
\end{tikzpicture}
  \caption{xxx}
  \label{fig:structure_spatialisation_ontologie}
\end{figure}

Par exemple, la \emph{relation de localisation atomique} \onto[orla]{Sous\-Al\-ti\-tu\-de} ---~notamment utilisée pour définir la \emph{relation} \onto[orla]{Sous\-Pro\-che\-De} (\autoref{chap:05}) \todo{Changer par une référence à la figure de décomposition lorsque définie}~--- est spatialisée à l'aide du \emph{rasteriser} \onto[orla]{Geometrie}, de la \emph{métrique} \onto[orla]{Delta\-Nearest\-Val} et du \emph{fuzzyficateur} \onto[orla]{Inf\-Zero} (\autoref{fig:ex_parties_statialisation_sousalt}).

\begin{figure}
  \centering
  \begin{tikzpicture} 
  \node[anchor=east] (x1) at (-3.5,0) {\onto[orla]{Sous\-Altitude}};

  \node[anchor=west] (y1) at (3.5,.8) {\onto[orla]{Geometrie}};
  \node[anchor=west] (y2) at (3.5,0) {\onto[orla]{DeltaNearestVal}};
  \node[anchor=west] (y3) at (3.5,-0.8) {\onto[orla]{Inf\-Val\-0}};
  
  \fill[RdBu-9-9] (x1.east) circle (2pt);
  
  \fill[RdBu-9-1] (y1.west) circle (2pt);
  \fill[RdBu-9-1] (y2.west) circle (2pt);
  \fill[RdBu-9-1] (y3.west) circle (2pt);

  \draw[->, shorten >=5pt, shorten <=5pt] (x1.east) |- (y1.west)
  node[pos=0.75, sloped, anchor=center, fill=white] {\tiny\emph{A pour rasteriser}};
  
  \draw[->, shorten >=5pt, shorten <=5pt] (x1.east) -- (y2.west)
  node[pos=0.5, sloped, anchor=center, fill=white] {\tiny\emph{A pour métrique}};

    \draw[->, shorten >=5pt, shorten <=5pt] (x1.east) |- (y3.west)
  node[pos=0.75, sloped, anchor=center, fill=white] {\tiny\emph{A pour fuzzyficateur}};
\end{tikzpicture}
  \caption{Exemple de la XXX de la \emph{relation de localisation
      atomique} \protect\onto[orla]{Sous\-Al\-ti\-tu\-de}.}
  \label{fig:ex_parties_statialisation_sousalt}
\end{figure}

La \emph{relation de localisation atomique} \onto[orla]{Dans\-Pla\-ni\-mé\-tri\-que} est quant à elle \emph{spatialisée} avec le même \emph{rasteziser} (\onto[orla]{Geometrie}), la \emph{métrique} \onto[orla]{XX} et le \emph{fuzzyficateur} \onto[orla]{XX} (\autoref{fig:ex_parties_statialisation_dansplani}).

\begin{figure}
  \centering
  \begin{tikzpicture} 
  \node[anchor=east] (x1) at (-3.5,0) {\onto[orla]{DansPlanimétrique}};

  \node[anchor=west] (y1) at (3.5,.8) {\onto[orla]{Geometrie}};
  \node[anchor=west] (y2) at (3.5,0) {\onto[orla]{Distance}};
  \node[anchor=west] (y3) at (3.5,-0.8) {\onto[orla]{Inf\-0}};
  
  \fill[RdBu-9-9] (x1.east) circle (2pt);
  
  \fill[RdBu-9-1] (y1.west) circle (2pt);
  \fill[RdBu-9-1] (y2.west) circle (2pt);
  \fill[RdBu-9-1] (y3.west) circle (2pt);

  \draw[->, shorten >=5pt, shorten <=5pt] (x1.east) |- (y1.west)
  node[pos=0.75, sloped, anchor=center, fill=white] {\tiny\emph{A pour rasteriser}};
  
  \draw[->, shorten >=5pt, shorten <=5pt] (x1.east) -- (y2.west)
  node[pos=0.5, sloped, anchor=center, fill=white] {\tiny\emph{A pour métrique}};

    \draw[->, shorten >=5pt, shorten <=5pt] (x1.east) |- (y3.west)
  node[pos=0.75, sloped, anchor=center, fill=white] {\tiny\emph{A pour fuzzyficateur}};
\end{tikzpicture}
  \caption{Exemple de la XXX de la \emph{relation de localisation
      atomique} \protect\onto[orla]{Dans\-Pla\-ni\-mé\-tri\-que}.}
  \label{fig:ex_parties_statialisation_dansplani}
\end{figure}

La \emph{relation de localisation atomique} \onto[orla]{Not\-A\-La\-Fron\-ti\-ere\-De} est quant à elle définie avec un \emph{modifieur,} \onto[orla]{Not}, qui permet de la distinguer de son opposée, \onto[orla]{A\-La\-Fron\-ti\-ere\-De}, utilisant le même \emph{rasteriser}, \onto[orla]{Geometrie}, la même \emph{métrique,} \onto[orla]{XX} et le même \emph{fuzzyficateur,} \onto[orla]{XX} (\autoref{fig:ex_parties_statialisation_notalafrontierede}).

\begin{figure}
  \centering
  \begin{tikzpicture} 
  \node[anchor=east] (x1) at (-3.5,0) {\onto[orla]{NotALaFrontiereDe}};

  \node[anchor=west] (y1) at (3.5,1.2) {\onto[orla]{Frontiere}};
  \node[anchor=west] (y2) at (3.5,0.4) {\onto[orla]{Distance}};
  \node[anchor=west] (y3) at (3.5,-0.4) {\onto[orla]{Eq\-Val\-0}};
  \node[anchor=west] (y4) at (3.5,-1.2) {\onto[orla]{Not}};
  
  \fill[RdBu-9-9] (x1.east) circle (2pt);
  
  \fill[RdBu-9-1] (y1.west) circle (2pt);
  \fill[RdBu-9-1] (y2.west) circle (2pt);
  \fill[RdBu-9-1] (y3.west) circle (2pt);
  \fill[RdBu-9-1] (y4.west) circle (2pt);

  \draw[->, shorten >=5pt, shorten <=5pt] (x1.east) |- (y1.west)
  node[pos=0.75, sloped, anchor=center, fill=white] {\tiny\emph{À pour rasteriser}};
  
  \draw[->, shorten >=5pt, shorten <=5pt] (x1.east) |- (y2.west)
  node[pos=0.75, sloped, anchor=center, fill=white] {\tiny\emph{À pour métrique}};

  \draw[->, shorten >=5pt, shorten <=5pt] (x1.east) |- (y3.west)
  node[pos=0.75, sloped, anchor=center, fill=white] {\tiny\emph{À pour
      fuzzyficateur}};
  \draw[->, shorten >=5pt, shorten <=5pt] (x1.east) |- (y4.west)
  node[pos=0.75, sloped, anchor=center, fill=white] {\tiny\emph{À pour Modifieur}};
\end{tikzpicture}
  \caption{Exemple de la XXX de la \emph{relation de localisation
      atomique} \protect\onto[orla]{Not\-A\-La\-Fron\-ti\-ere\-De}.}
  \label{fig:ex_parties_statialisation_notalafrontierede}
\end{figure}

%%% Local Variables:
%%% mode: latex
%%% TeX-master: "../../../../main"
%%% End:
