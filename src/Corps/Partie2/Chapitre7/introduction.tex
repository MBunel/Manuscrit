Ce chapitre est dédié à la phase de \emph{spatialisation,} étape
primordiale et centrale de notre méthode de transformation d'un
référencement spatial indirect en référencement spatial direct,
présentée dans le \autoref{chap:04} (\autoref{fig:methodo_1}).
%
% Rappeler la définition de la spatialisation
L'objectif de la spatialisation est de créer la \emph{zone de
  localisation compatible} correspondant à un \emph{indice de
  localisation décomposé.} Il s'agit de la seule étape de notre
méthodologie aboutissant à la création d'objets géographiques, les
étapes composant les phases de \emph{décomposition}
(\autoref{chap:05}) et de \emph{fusion} (\autoref{chap:08}) ne faisant
que transformer des objets, respectivement des \emph{indices de
  localisation} et des \emph{zones de localisation compatibles,}
préexistants.

Notre méthode de \emph{spatialisation} est composée de trois parties :
la \emph{rasterisation,} le calcul de la \emph{métrique} et la
\emph{fuzzyfication,} que nous avons rapidement évoqués dans le
chapitre précédent. Nous allons ici les présenter en détail et exposer
la manière dont elles se combinent pour aboutir à la construction
d'une \ac{zlc}.

Dans la première partie de ce chapitre (\autoref{chap:07-sec1}) nous
présenterons l'organisation générale de la \emph{méthode de
  spatialisation,} puis nous en détaillerons les différentes parties :
la \emph{rasterisation} (\autoref{chap:07-sec2}), le \emph{calcul de
  la métrique} (\autoref{chap:07-sec3}) et la \emph{fuzzyfication}
(\autoref{chap:07-sec4}). Enfin, nous aborderons la représentation des
connaissances relatives à la \emph{méthode de spatialisation}
(\autoref{chap:07-sec5}).

%%% Local Variables:
%%% mode: latex
%%% TeX-master: "../../../main"
%%% End:
