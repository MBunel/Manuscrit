Pour guider la définition de notre méthode nous avons identifié
plusieurs principes que nous voulons voir respectés.

\subsection{Intégration dans le contexte métier}

Le premier des principes que nous voulons voir respecté par notre
méthodologie est celui de sa bonne intégration dans le contexte
métier.
%
C'est un point que nous avons déjà évoqué a plusieurs reprises. Nous
souhaitons que notre méthode soit conçue dans l'optique de son
application finale, \ie que nous avons à cœur de réfléchir à son
appropriation et son développement par les secouristes. Pour autant
nous ne souhaitons pas sacrifier la généricité de notre méthode, c'est
pourquoi son développement relève d'un équilibre entre la généricité
de la méthode et la spécificité de certains paramétrages.

\subsubsection{\texttt{Le secouriste comme super joker + ZIR}}

Parmi les choix visant à faciliter l'intégration de notre méthode il y
en a un dont nous avons déjà parlé, le rôle du secouriste. En effet,
comme nous l'indiquions dans la première partie, nous ne souhaitons
pas que notre méthode se substitue au secouriste, mais au contraire
qu'elle l’assiste. Ce choix (tout du moins tel que présenté dans le
\autoref{chap:01}) est hérité d'une décision faite au sein du projet
CHOUCAS dans son ensemble, celle de ne pas travailler sur
l'interprétation automatisée du requérant, nécessitant dès lors un
intermédiaire, le secouriste, entre le combiné téléphonique et
l'interface.

\begin{figure}
  \centering
  % \begin{tikzpicture}
  \begin{umlseqdiag}
    \umlactor{A}\umlactor{B}\umlobject{C}
    \begin{umlcall}{A}{C}
      \begin{umlcall}{C}{A}
      \end{umlcall}
      \begin{umlcall}[dt=10 ]{B}{C}
        \begin{umlcall}{C}{B}\end{umlcall}
        \begin{umlcall}[
          return=1]{B}{C}
        \end{umlcall}\end{umlcall}
      \begin{umlcall}[
        dt=20,return=1]{A}{C}
      \end{umlcall}
    \end{umlcall}
  \end{umlseqdiag}
\end{tikzpicture}
  \caption{dcsqd}
  \label{fig:diag_acti_secours}
\end{figure}

\subsubsection{La modélisation explicite des connaissances}

Pour faciliter l'intégration de notre méthode dans le contexte métier
du secours en montagne nous avons décidé d'adopter une démarche basée
sur les connaissances.

\subsection{Principes de modélisation}

\subsubsection{Décomposition des \emph{relations de localisation}}

\subsubsection{Modélisation autonome}

Un autre principe que nous souhaitons développer est celui de la
séparation maximale des étapes du processus.

Comme nous l'expliquions dans le \autoref{chap:2} deux étapes sont
nécessaires pour transformer une position décrite en une \emph{zone de
  localisation probable,} la \emph{spatialisation} et la
\emph{fusion.} Mais ces deux étapes peuvent être combinées
différemment.

Une première solution consiste à \enquote{chainer} la
\emph{spatialisation} des différents \emph{indices de localisation,}
comme on peut le voir sur la figure \ref{fig:comp_approches_lin}. Avec
cette approche les différentes \emph{zones de localisation
  compatibles} sont créées les unes à la suite des autres, ainsi la
zone de recherche en entrée de chaque étape de \emph{spatialisation}
dépend des \emph{spatialisations} précédentes. Avec cette approche,
l'opération de \emph{fusion} est implicite, elle s'opère, en partie, à
chaque nouvelle \emph{spatialisation,} puisque les zones ne
correspondant pas aux \emph{indices de localisation} sont retirées peu
à peu. La \emph{zone de localisation probable} (en bleu sur la figure
\ref{fig:comp_approches}) est obtenue une fois que l'on a
\emph{spatialisé} tous les \emph{indices de localisation.} Cette
approche est la première que nous ayons envisagée. Elle a pour
avantage d'être simple à concevoir et à développer. Cependant cette
approche à un problème majeur. À cause de l'enchainement des étapes de
\emph{spatialisation} les erreurs se répercutent d'une étape à
l'autre. Ainsi, si un des \emph{indices de localisation} est mal
\emph{spatialisé}, que ce soit à cause de la méthode utilisée ou d'une
mauvaise description, cela se répercute sur toutes les
\emph{spatialisations} ultérieures. La moindre correction impose donc
de refaire l'ensemble des traitements.

Une seconde solution, plus robuste, consiste à effectuer l'ensemble
des opérations de \emph{spatialisation} en parallèle, comme le montre
la figure \ref{fig:comp_approches_sep}. Avec cette approche la
\emph{spatialisation} des différents \emph{indices de localisation}
est effectuée séparément et leur \emph{fusion} est effectuée dans un
second temps. Comme pour la démarche \enquote{chainée} une
\emph{spatialisation} erronée peut impacter la \emph{zone de
  localisation probable,} toutefois il est plus facile de corriger ces
erreurs. En effet, si les différentes \emph{zones de localisation
  compatibles} ont été conservées il est possible de faire une
nouvelle \emph{fusion} en retirant un, ou plusieurs \emph{indices de
  localisation.} Alors qu'avec une approche \enquote{chainée} il est
nécessaire de refaire toutes les opérations de \emph{spatialisations}
ultérieures à celle de \emph{l'indice de localisation} que l'on
souhaite modifier ou retirer et ce même si les résultats
intermédiaires ont été conservés.

\begin{figure}
  \centering
  \subfloat[Construction suivant une démarche linéaire]{
    \begin{tikzpicture}

  \begin{scope}
    \path[ffa] (0,0) rectangle (2,2);
    \path[ffc] (0,0) rectangle (2,2) ;
    \node[text width=2cm](1,1) {\footnotesize \itshape Zone initiale de recherche};
  \end{scope}
  
  \path[draw, ->] (2.5,1) --++ (3,0)  node[pos=.5, above] {\footnotesize \itshape spatialisation};

  \begin{scope}[xshift=6cm]
    \path[draw,dashed] (0,0) rectangle (2,2);
    \begin{scope}
      \begin{scope}
        \clip (0,0) rectangle (2,2);
        \fill[ffa] (1.25,.7) circle [radius=15pt];
        \path[ffc] (1.25,.7) circle [radius=15pt];
      \end{scope}
    \end{scope}
    \node[circle, inner sep=0pt,minimum size=4pt, fill] (c) at (1.25,.7)
    {};
  \end{scope}

  \path[draw, ->] (8.5,1) --++ (3,0)  node[pos=.5, above] {\footnotesize \itshape spatialisation};




  \begin{scope}[xshift=12cm]
    \path[draw,dashed] (0,0) rectangle (2,2);
    \path[draw,dashed] (1.25,.7) circle [radius=15pt];
    \path[draw,dashed](.8,1) circle [radius=20pt];
    \begin{scope}
      \begin{scope}
        \clip (1.25,.7) circle [radius=15pt];
        \fill[ffa2] (.8,1) circle [radius=20pt];
        \path[ffc2] (.8,1) circle [radius=20pt];
      \end{scope}
      \begin{scope}
        \clip (.8,1) circle [radius=20pt];
        \path[ffc2] (1.25,.7) circle [radius=15pt];
      \end{scope}
    \end{scope}
    \node[circle, inner sep=0pt,minimum size=4pt, fill] (c) at (1.25,.7)
    {};
    \node[circle, inner sep=0pt,minimum size=4pt, fill] (c) at (.8,1)
    {};
  \end{scope}


\end{tikzpicture}
    \label{fig:comp_approches_lin}
  }

  \subfloat[Construction suivant une démarche autonome]{
    \begin{tikzpicture}

  \begin{scope}
    \path[ffa] (0,0) rectangle (2,2);
    \path[ffc] (0,0) rectangle (2,2);
    \node[text width=3cm, align=center, anchor=center] at (1,-.5) {\footnotesize \itshape Zone initiale de recherche};
  \end{scope}

  \path[draw, ->] (2.5,1.25) --++ (3,1.25)  node[pos=.5, above, sloped] {\footnotesize \itshape spatialisation};
  \path[draw, ->] (2.5,.75) --++ (3,-1.25)  node[pos=.5, above, sloped] {\footnotesize \itshape spatialisation};

  \begin{scope}[xshift=6cm, yshift=-1.5cm]
    \path[draw, dashed] (0,0) rectangle (2,2);
    \begin{scope}
      \begin{scope}
        \clip (0,0) rectangle (2,2);
        \fill[ffa] (1.25,.7) circle [radius=15pt];
        \path[ffc] (1.25,.7) circle [radius=15pt];
      \end{scope}
    \end{scope}
    \node[circle, inner sep=0pt,minimum size=4pt, fill] (c) at
    (1.25,.7) {};
    \node[text width=3cm, align=center, anchor=center] at (1,-.5)
    {\footnotesize \itshape Zone de localisation compatible};
  \end{scope}

  \begin{scope}[xshift=6cm,yshift=1.5cm]
    \path[draw, dashed] (0,0) rectangle (2,2);
    \fill[ffa](.8,1) circle [radius=20pt];
    \path[ffc](.8,1) circle [radius=20pt];
    \node[circle, inner sep=0pt,minimum size=4pt, fill] (c) at (.8,1)
    {};
    \node[text width=3cm, align=center, anchor=center] at (1,2.5)
    {\footnotesize \itshape Zone de localisation compatible};
  \end{scope}

  \path[draw, -] (8.5,2.5) --++ (1.5,-1.5);
  \path[draw, -] (8.5,-.5) --++ (1.5,1.5);
  \path[draw, ->] (10,1) --++ (1.5,0)  node[pos=.5, above] {\footnotesize \itshape fusion};


  \begin{scope}[xshift=12cm]
    \path[draw,dashed] (0,0) rectangle (2,2);
    \path[draw,dashed] (1.25,.7) circle [radius=15pt];
    \path[draw,dashed](.8,1) circle [radius=20pt];
    \begin{scope}
      \begin{scope}
        \clip (1.25,.7) circle [radius=15pt];
        \fill[ffa2] (.8,1) circle [radius=20pt];
        \path[ffc2] (.8,1) circle [radius=20pt];
      \end{scope}
      \begin{scope}
        \clip (.8,1) circle [radius=20pt];
        \path[ffc2] (1.25,.7) circle [radius=15pt];
      \end{scope}
    \end{scope}
    \node[circle, inner sep=0pt,minimum size=4pt, fill] (c) at (1.25,.7)
    {};
    \node[circle, inner sep=0pt,minimum size=4pt, fill] (c) at (.8,1)
    {};
  \end{scope}


\end{tikzpicture}
    \label{fig:comp_approches_sep}
  }
  \caption{Comparaison du processus de construction de la \emph{zone
      de localisation probable} pour une alerte à deux \emph{indices
      de localisation}.}
  \label{fig:comp_approches}
\end{figure}

Le principe de modélisation autonome peut, de plus, tirer parti du
principe de décomposition précédemment présenté.

% Justifier choix

\subsection{Principes sémantiques}

\subsubsection{Modélisation non bivalente}
\texttt{Ref images EDA}

\subsubsection{Le raisonnement en monde ouvert}


Comme le montre la figure \ref{fig:md_ferme}, dans \emph{l'hypothèse
  du monde clos} toute règle inconnue est considérée comme
fausse. Alors que dans \emph{l'hypothèse du monde ouvert} (Figure
\ref{fig:md_ouvert}) les règles inconnues sont considérées comme
telles, \ie que l'on estime qu'elles peuvent être vraies, comme
fausses.

Pour illustrer la différence entre ces deux hypothèses on peut prendre
l'exemple suivant. Imaginons que je décrive le contenu de ma
bibliothèque de la suivante : \enquote{Dans ma bibliothèque on trouve
  les ouvrages : \emph{méthodes de logique,} de Willard \bsc{Quine} et
  \emph{le projet \emph{Cybersyn},} d'Eden \bsc{Medina}.} Cette phrase
peut être décomposée en deux assertions logiques, \enquote{ma
  bibliothèque contient l'ouvrage \emph{méthodes de logique}} et
\enquote{ma bibliothèque contient l'ouvrage \emph{le projet
    \emph{Cybersyn}}.} Avec \emph{l'hypothèse du monde clos} toute
autre proposition logique est considérée comme fause, ainsi à la
question \enquote{Est-ce que tu as tel livre ?} (à l'exception deux
précédemment mentionnés) la réponse sera toujours \enquote{non}, dans
ce contexte cela revient à considérer que j'ai donné une description
exhaustive du contenu de ma bibliothèque. Avec \emph{l'hypothèse du
  monde ouvert} on considère que les règles qui nous sont inconnues
peuvent être vraies ou fausses, ainsi, dans ce cadre on ne peut que
répondre \enquote{Je ne sais pas} à la question précédente. Dans
\emph{l'hypothèse d'un monde ouvert,} l’absence d'une règle n'implique
pas sa fausseté.

\begin{figure}
  \centering
  \subfloat[Illustration de l'hypothèse du monde clos]{
    \begin{tikzpicture}
  \begin{scope}
    \fill[ffa] (1,.75) arc(90:270:.75) -- cycle;% [radius=.75cm];
    \path[ffc] (1,.75) arc(90:270:.75) -- cycle;
    \node[color=RdBu-9-1] at (.625,0) {V};
  \end{scope}
  \begin{scope}
    \fill[ffa2] (1,1) arc(270:90:-1) -- cycle;% [radius=.75cm];
    \path[ffc2] (1,1) arc(270:90:-1) -- cycle;
    \node[color=RdBu-9-9] at (1.375,0) {F};
  \end{scope}
  \begin{scope}
    \path[ffc, draw=black] (1,0) circle [radius=.75cm];
  \end{scope}
  \begin{scope}
    \node (rect) [anchor=north, minimum width=.5cm,minimum
    height=.25cm,ffc, draw=black] at (0,-1.25) {};
    \node[anchor=west, font=\tiny\vphantom{Ag}, text width = 4cm] at
    ([xshift=1ex]rect.east) {Connues};
    
    \node (rect2) [anchor=north, minimum width=.5cm,minimum
    height=.25cm, ffa, ffc] at ([yshift=-.25cm]rect) {};
    \node[anchor=west, font=\tiny\vphantom{Ag}, text width = 4cm] at
    ([xshift=1ex]rect2.east) {Vraies};
    
    \node (rect3) [anchor=north, minimum width=.5cm,minimum
    height=.25cm, ffa2, ffc2] at ([yshift=-.25cm]rect2) {};
    \node[anchor=west, font=\tiny\vphantom{Ag}, text width = 4cm] at
    ([xshift=1ex]rect3.east) {Fausses};
    
    \draw[decorate,decoration={brace}] ([xshift=8ex]rect.north
    east) -- ([xshift=8ex]rect3.south east);
    
    \node[anchor=west, font=\tiny\vphantom{Ag}, text width = 2cm] at
    ([xshift=8.25ex]rect2.east) {Ensemble des règles};   
  \end{scope}
\end{tikzpicture}
    \label{fig:md_ferme}
  }\hspace{3cm}
  \subfloat[Illustration de l'hypothèse du monde ouvert]{
    \begin{tikzpicture}
  \begin{scope}
    \fill[ffa] (1,1) arc(90:270:1) -- cycle;% [radius=.75cm];
    \path[ffc] (1,1) arc(90:270:1) -- cycle;
    \node[color=RdBu-9-1] at (.625,0) {V};
  \end{scope}
  \begin{scope}
    \fill[ffa2] (1,1) arc(270:90:-1) -- cycle;% [radius=.75cm];
    \path[ffc2] (1,1) arc(270:90:-1) -- cycle;
    \node[color=RdBu-9-9] at (1.375,0) {F};
  \end{scope}
  \begin{scope}
    \path[ffc, draw=black] (1,0) circle [radius=.75cm];
    \node[text width=3cm] (leg) at (4,1)
    {\footnotesize \itshape Ensemble des règles connues};
    \path[draw, ->] (leg.west) --++ (25:-.9);
  \end{scope}
\end{tikzpicture}
    \label{fig:md_ouvert}
  }
  \caption{Illustration des hypothèses du \emph{monde clos} et du
    \emph{monde ouvert}}
  \label{fig:comp_md}
\end{figure}

Appliqué à notre cas d'étude le choix d'une de ces deux hypothèses
revient à se demander si la description d'une position donnée par les
requérants est systématiquement exhaustive. Si la réponse est
\enquote{oui} on peut alors faire \emph{l'hypothèse d'un monde clos}
et considérer que toute information qui n'est pas donnée par le
requérant est fausse. Ainsi, s'il décrit sa position en indiquant
qu'il \enquote{est sur une route} on pourra en conclure qu'il est pas
en forêt, puisque cette information ne nous a pas été donnée. Par
conséquent on pourra \emph{spatialiser} cette alerte à l'aide de deux
\emph{indices de localisation} l'un explicite (\enquote{je suis sur
  une route}) et l'autre inféré (\enquote{je ne suis pas en
  forêt}). Or cet exemple illustre bien que cette approche n'est pas
satisfaisante, des routes peuvent traverser des forêts, ou non et rien
dans cette alerte ne permet de rejeter cette hypothèse, le
raisonnement en monde ouvert est donc plus approprié.

%%% Local Variables:
%%% mode: latex
%%% TeX-master: "../../../../main"
%%% End:
