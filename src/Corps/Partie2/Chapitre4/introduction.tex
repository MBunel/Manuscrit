Ce chapitre se destine à présenter la structure générale de la méthode
conçue pour transformer une \emph{description de position} en une zone
de \emph{localisation probable.} La seconde partie de ce manuscrit
étant construite suivant une logique de complexification graduelle,
nous ne présenterons ici que les grands aspects de notre méthode, qui
sera complétée au fur et à mesure de cette partie.

Pour construire notre méthode nous avons adopté une démarche par
contraintes. La première partie de ce chapitre détaille six grands
principes, qui nous semblent nécessaires au développement d'une
méthode robuste, intelligible et adaptée à une utilisation
professionnelle. Ces six principes, couplés aux objectifs
scientifiques de la thèse (\autoref{chap:02}) forment une ossature
conceptuelle, permettant de guider la définition de notre méthode.

La première partie de ce chapitre est dédiée à la présentation de ces
\emph{principes de modélisation,} puis une seconde partie sera
destinée à la définition de notre méthode.
%%% Local Variables:
%%% mode: latex
%%% TeX-master: "../../../main"
%%% End:
