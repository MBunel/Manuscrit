Comme nous l'avons vu dans le chapitre précédent, cette thèse
s'inscrit dans le projet de recherche Choucas, dont l'objectif
principal est d’élaborer des solutions destinées à aider les
\emph{unités de secours en montagne} \acp{usem} et plus spécifiquement
le \ac{pghm} de Grenoble, à localiser des personnes perdues en
montagne. Parmi toutes les problématiques abordées dans ce projet
(\autoref{chap:1)}, notre travail s'inscrit dans celle visant à
développer des méthodes et des outils permettant de transformer une
description de position (\eg \enquote{Je suis en face de la Meije}) en
une zone, dont les coordonnées sont connues et donc moins ambiguës que
la description y correspondant.

La présentation qui a été faite de cette problématique a été,
volontairement, assez évasive. Nous n'en avons pas présenté, ni le
détail, ni la manière dont cette thèse s'y intègre. Ce chapitre se
destine à clarifier ces points, notamment en détaillant la
problématique de cette thèse (\autoref{sec:2-1}) et en présentant les
différents objectifs scientifiques de ce travail (\autoref{sec:2-2}).

%%% Local Variables:
%%% mode: latex
%%% TeX-master: "../../../main"
%%% End:
