Dans ce dernier chapitre nous avons présenté l'implémentation de notre
méthode dans un prototype, Ruitor et détaillé la manière dont nous
avons procédé pour modéliser des alertes réelles. Pour ce faire nous
avons sélectionnée deux alertes : \emph{Grand Veymont} et le \emph{fil
  rouge} et avons cherché à les spatialiser avec les \emph{relations
  de localisation} définies dans l'ontologie \ac{orl}. Nous avons
analysé ces deux alertes de manière à identifier la manière la plus
adaptée de les modéliser, puis avons construit les \emph{zones de
  localisation compatibles} correspondant à chacune d'entre-elles.

La spatialisation de ces deux alertes nous a permis de mettre en
évidence plusieurs défauts à nos modélisation. Certains sont
directement liés à la méthode que nous avons définie, comme le fait
qu'il n'existe pas de solution autre qu'empirique pour fixer les
seuils des \emph{fuzzificateurs,} et d'autres sont liés au données que
nous utilisons. Ainsi, de nombreuses pistes d'amélioration de notre
méthode ont été identifiée.

%%% Local Variables:
%%% mode: latex
%%% TeX-master: "../../../main"
%%% End:
