Tous les modèles théoriques que nous avons présentés dans l'état de
l'art de cette thèse ne sont pas nécessairement adaptés aux exigences
de modélisation que nous avons formulé.

% Les modèles discrets (exacts, ensembles épais, autres ?)
Ainsi, les modèles \enquote{exacts} \footnote{Comme les modèles
  \emph{egg-yolk} \autocite{Cohn1996}, \emph{min-max}
  \autocite{Clementini1996}, \emph{etc.}} ou basés sur la
\emph{théorie des ensembles approximatifs} \autocite{Pawlak1982},
imposent une modélisation discrète de \emph{l'imprécision}
spatiale. Si cette approche a pour avantage sa grande simplicité, elle
ne remplit pas toutes les conditions que nous jugeons
nécessaires. Tout d'abord, bien que ces modèles permettent de
modéliser toute forme de \emph{zone de localisation} \footnote{À
  l'exception de zones \emph{nettes,} qui ne sont modélisables qu'avec
  le modèle de \textcite{Bejaoui2009}. Cette caractéristique n'est
  cependant pas limitante dans notre cas, aucune \emph{relation de
    localisation} ne pouvant définir une zone précise.}, il leur est
impossible de rendre compte de la variation de \emph{l'imprécision} au
sein de la zone frontière, cette dernière n'étant définie que par ses
bornes supérieure et inférieure. On peut illustrer ce problème par la
\autoref{fig:Comp_mod}, où deux frontières imprécises, très
différentes, sont représentées. Dans les deux situations, la zone
\emph{imprécise} débute et se termine au même endroit. Dans une telle
situation, ces deux frontières seraient \enquote{lissées} par
l'utilisation d'un modèle \enquote{exact}. Il serait pourtant
intéressant de pouvoir prendre en compte ces différences, d'autant
plus que la zone \emph{imprécise} est étendue.

\begin{figure}[hb]
  \centering \begin{tikzpicture}[scale=.7]
  \def\decalageX{-.2}
  \def\decalageY{-.2}
  % Courbe
  \begin{scope}[local bounding box=crb1]
    \begin{scope}[transparency group]
      % fond
      \begin{scope}
        \path[ffa] (0,2) -- (2.5,2) -- (5, 0)  -- (0,0) -- cycle;
      \end{scope}
      % bords
      \begin{scope}
        \path[ffc] (0,2) -- (2.5,2) -- (5, 0) -++ (3,0) ;
        \path[ffc_fade] (8,0) -- (9,0) ;
      \end{scope}
    \end{scope}
    % Axes X, Y
    \begin{scope}
      % Axe X
      \begin{scope}
        % Axe
        \draw[->] (0, \decalageX) --++ (9, 0) coordinate (x axis);
        % Graduations
        \foreach \n/\t in {0/{},1/{},2/{},3/{},4/{},5/{},6/{},7/{},8/{}}
        {
          \draw[-] (\n, \decalageX - .05) --++ (0, .1);
          \node[below, font=\footnotesize] at (\n, \decalageX - .05) {\t};
        }
      \end{scope}
      % Axe Y
      \begin{scope}
        % Axe
        \draw[-] (\decalageY ,0) --++ (0, 2) coordinate (y axis);
        % Graduations
        \foreach \n/\t in {0/{$-$},2/{$+$}}
        {x3
          \draw[-] (\decalageY -.05, \n) --++ (.1, 0);
          \node[left, font=\footnotesize] at (\decalageY -.05, \n) {\t};
        }
        % Label
        \node[above, anchor=south west] at (y axis) {Précision};
      \end{scope}
    \end{scope}
  \end{scope}
  % Courbe 2
  \begin{scope}[yshift=-4cm, local bounding box=crb2]
    \begin{scope}[transparency group]
      % fond
      \begin{scope}
        \path[ffa] (0,2) -- (2.5,2) ..controls(4,1.5).. (5, 0)  -- (0,0) -- cycle;
      \end{scope}
      % bords
      \begin{scope}
        \path[ffc] (0,2) -- (2.5,2) ..controls(4,1.5).. (5, 0) -++ (3,0) ;
        \path[ffc_fade] (8,0) -- (9,0) ;
      \end{scope}
    \end{scope}
    % Axes X, Y
    \begin{scope}
      % Axe X
      \begin{scope}
        % Axe
        \draw[->] (0, \decalageX) --++ (9, 0) coordinate (x axis);
        % Graduations
        \foreach \n/\t in {0/{},1/{},2/{},3/{},4/{},5/{},6/{},7/{},8/{}}
        {
          \draw[-] (\n, \decalageX - .05) --++ (0, .1);
          \node[below, font=\footnotesize] at (\n, \decalageX - .05) {\t};
        }
      \end{scope}
      % Axe Y
      \begin{scope}
        % Axe
        \draw[-] (\decalageY ,0) --++ (0, 2) coordinate (y axis);
        % Graduations
        \foreach \n/\t in {0/{$-$},2/{$+$}}
        {
          \draw[-] (\decalageY -.05, \n) --++ (.1, 0);
          \node[left, font=\footnotesize] at (\decalageY -.05, \n) {\t};
        }
        % Label
        \node[above, anchor=south west] at (y axis) {Précision};
      \end{scope}
    \end{scope}
  \end{scope}
  % Courbe 3
  \begin{scope}[shift={($(crb1.south east)!0.5!(crb2.north east) + (3cm,-1cm)$)}]
    \begin{scope}[transparency group]
      % fond
      \begin{scope}
        \path[ffa] (0,2) -- (2.5,2) -- (2.5, 0)  -- (0,0) -- cycle;
        \path[ffa2] (2.5,0) -- (2.5,1) -- (5, 1)  -- (5,0) -- cycle;
      \end{scope}
      % bords
      \begin{scope}
        %\path[ffc] (0,2) -- (2.5,2) -- (5, 0) -++ (3,0) ;
        %\path[ffc] (0,2) -- (2.5,2) -- (5, 0) -++ (3,0) ;
        %\path[ffc_fade] (8,0) -- (9,0) ;
      \end{scope}
    \end{scope}
    % Axes X, Y
    \begin{scope}
      % Axe X
      \begin{scope}
        % Axe
        \draw[->] (0, \decalageX) --++ (9, 0) coordinate (x axis);
        % Graduations
        \foreach \n/\t in {0/{},1/{},2/{},3/{},4/{},5/{},6/{},7/{},8/{}}
        {
          \draw[-] (\n, \decalageX - .05) --++ (0, .1);
          \node[below, font=\footnotesize] at (\n, \decalageX - .05) {\t};
        }
      \end{scope}
      % Axe Y
      \begin{scope}
        % Axe
        \draw[-] (\decalageY ,0) --++ (0, 2) coordinate (y axis);
        % Graduations
        \foreach \n/\t in {0/{},1/{},2/{}}
        {
          \draw[-] (\decalageY -.05, \n) --++ (.1, 0);
          \node[left, font=\footnotesize] at (\decalageY -.05, \n) {\t};
        }
        % Label
        \node[above, anchor=south west] at (y axis) {Précision};
      \end{scope}
    \end{scope}
  \end{scope}

\end{tikzpicture}

  \caption{Illustration du passage de différents types de frontières
    \emph{imprécises} à une modélisation \enquote{exacte}.}
  \label{fig:Comp_mod}
\end{figure}

De plus, si les \emph{zones de localisation} construites par la
\emph{spatialisation} d'une \emph{relation de localisation} seraient
relativement simples \footnote{Elles ne sont constituées que de deux
  polygones marquant les limites de la zone frontière
  (cf. \autoref{chap:03}).} et donc faciles à interpréter, elles se
complexifieraient rapidement au fil des unions et des
intersections. En effet, contrairement aux intersections et unions
d'objets \enquote{classiques}, il est ici nécessaire de manipuler,
pour chaque objet, deux géométries, la frontière supérieure et la
frontière inférieure. Ainsi, si les \emph{zones de localisation
  compatibles} modélisées avec un modèle \enquote{exact} pourraient
être faciles à interpréter, il n'en sera pas de même avec la
\emph{zone de localisation probable,} issue de la fusion de toutes ces
zones. Par conséquent, les modèles \enquote{exacts} ou basés sur la
\emph{théorie des ensembles approximatifs} n'ont pas les
caractéristiques attendues et ne permettent donc pas de modéliser les
\emph{zones de localisation compatibles} et \emph{probables} comme
nous le souhaitons.

% Modèles non linéaires et non flous
En permettant de définir, pour chaque position, un \emph{degré
  d'appartenance} compris entre 1 (appartenance totale à la \emph{zone
  de localisation}) et 0 (non-appartenance), la \emph{théorie des
  sous-ensembles flous} permet de contourner le premier problème posé
par l'utilisation de \emph{modèles exacts.} Cette approche permet, en
effet, de distinguer des zones \emph{imprécises} qui auraient été
considérées comme équivalentes dans un \emph{modèle exact}
(\autoref{fig:Comp_mod}). De plus, les opérations d'union et
d'intersection des \emph{zones de localisation compatibles}
nécessaires à la mise en place de notre méthodologie peuvent être
facilement effectuées à l'aide des opérateurs \emph{t-norme} et
\emph{t-conorme} (\autoref{chap:03}) qui prennent et renvoient un
\emph{degré d'appartenance.} Ainsi, la sémantique des sous-ensembles
flous n'est pas altérée par les unions et les intersections, ce qui
garantit leur interprétabilité.

Les \emph{sous-ensembles flous} offrent donc un cadre théorique
permettant de répondre à tous les critères de modélisation que nous
avons définis. De plus, cette méthode a pour avantage d'avoir été
régulièrement utilisée dans la littérature (cf. \autoref{chap:03}), ce
qui est un atout supplémentaire. Ce modèle apparaît comme la solution
la plus pertinente pour modéliser des \emph{zones de localisation} et
plus généralement des \emph{objets géographiques imprécis.} Cependant,
de nombreuses implémentations de ce modèle théorique ont été proposées
(cf. \autoref{chap:03}). Il est donc nécessaire de se pencher sur ces
différentes solutions afin d'identifier celle qui est la plus
pertinente pour notre cas d'utilisation.

% \begin{landscape}
%   \begin{table}[H]
%     \centering
%     %\input{../tableaux/Comparaison_implems.tex}
%     \caption{Synthèse de la comparaison des modèles théoriques.}
%     \label{tab:comp_mod_theo}
%   \end{table}
% \end{landscape}

%%% Local Variables:
%%% mode: latex
%%% TeX-master: "../../../../main"
%%% End:
