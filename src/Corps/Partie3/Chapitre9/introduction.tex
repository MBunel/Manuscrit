Dans ce neuvième et dernier chapitre, nous allons présenter
l'application de la méthode présentée au cours des chapitres
précédents à des cas réels, des alertes précédemment traitées par le
\ac{pghm}. Pour ce faire, nous avons sélectionné deux alertes qui,
combinées, permettent de soulever de nombreuses questions sur la
spatialisation et mettent en évidence plusieurs problèmes. Avant de
détailler ces deux alertes et la manière dont elles ont été traitées,
nous présenterons le détail de l'implémentation de notre méthode et
les différents choix techniques qui ont été faits dans cette optique.

Dans la première partie de ce chapitre (\autoref{sec:9-1}), nous
présenterons la manière dont nous avons implémenté la méthode définie,
puis nous détaillerons la modélisation de deux alertes : le
\enquote{Grand Veymont} (\autoref{sec:9-2}) et le \enquote{fil rouge}
(\autoref{sec:9-4}).

%%% Local Variables:
%%% mode: latex
%%% TeX-master: "../../../main"
%%% End:
