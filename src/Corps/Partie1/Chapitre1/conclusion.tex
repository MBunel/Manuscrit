Comme nous l'avons vu au cours de ce chapitre, notre thèse s'inscrit à
la fois dans le contexte métier du secours en montagne et dans
le contexte organisationnel du projet de recherche Choucas.

Parmi tous les points critiques auxquels les secouristes sont
confrontés dans l'exercice de leur métier, nous nous intéressons à la
question de la localisation de la victime par les secouristes. En
effet, malgré les développement récent de la solution de
géolocalisation \emph{Gend'Loc,} il est quelquefois difficile pour les
secouristes de localiser une personne perdue ou bléssée en
montagne. Or la rapidité d'un secours est un facteur déterminant de la
survie de la victime. Le projet Choucas a donc été construit pour
aider les secouristes, en premier lieu ceux du \ac{pghm} de Grenoble,
a traiter ce type de cas. L'objectif de ce projet est, plus
précisément, de développer des méthodes et des outils destinés à
assister les secouristes durant la phase de localisation de la
victime. Notamment en proposant des outils d'aide à la décision, de
géovisualisation ou en enrichissant les bases de données utilisées par
le secouristes avec des données métier utiles, telles que certaines
formes spécifiques du relief (\eg vires, barres rocheuses, etc.).

Au sein de ce projet, notre rôle est de travailler sur le
développement d'une solution d'aide à la décision, permettant de
transformer une description de position, \eg \enquote{je suis sous une
  cascade}, en une zone y correspondant, nous parlonsalors de
\enquote{spatialisation}. Les détails de cet objectif et de son
contexte scientifique serons présentés dans le chapitre suivant.

%%% Local Variables:
%%% mode: latex
%%% TeX-master: "../../../main"
%%% End:
