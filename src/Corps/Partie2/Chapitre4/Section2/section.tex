En vue de la définition de la méthode de transformation d'une
description de position en une \emph{zone de localisation probable}
(\autoref{sec:4-3}) nous allons formaliser le concept \emph{d'indice
  de localisation}. Comme nous l'expliquions dans le
\autoref{Chap:02}, les \emph{indices de localisation} sont l'élément
de base des alertes. Chacun d'entre-eux décrit une po


Une alerte 
Ils contiennent toutes les informations données par le requérant pour
décrire sa positon

Comme nous l'avons déjà indiqué la description orale d'une position
nécessite trois éléments distincts, un \emph{sujet} qui est l'objet ou
la personne dont la position est décrite, un \emph{objet de référence}
qui est l'objet géographique servant de point de référence et une
\emph{relation de localisation} décrivant la manière dont le
\emph{sujet} est localisé par rapport à \emph{l'objet de référence.}
Un \emph{indice de localisation} (\(i\)) est donc formalisable en un
triplet de la forme :
%
\begin{equation}
  i = (S, Rl, Or)
\end{equation}
%
Avec \(S\) le \emph{sujet,} \(Rl\) la \emph{relation de localisation}
et \(Or\) \emph{l'objet de référence.}

Cette définition est cependant assez restrictive, puisqu'elle ne
permet pas d'exprimer certaines des configurations, plus complexes,
notamment celles induites par le \emph{principe de décomposition.}
Comme nous l'avons expliqué lors de la présentation du \emph{principe
  de décomposition} derrière une \emph{relation de localisation}
manipulée par l'utilisateur peuvent se cacher plusieurs
\emph{relations de localisations atomiques,} chacune correspondant à
une part de la sémantique de la \emph{relation de localisation}
initiale. La \emph{relation} \(Rl\) n'est donc pas un objet
individuel, mais un ensemble (contenant au moins un élément) de
\emph{relations de localisations atomiques} (\(Rla\)). \(Rl\) est donc
définissable de la manière suivante :

\begin{equation}
  Rl = \{Rla_1, Rla_2, \ldots, Rla_n\}
\end{equation}

Avec \(Rl\) la \emph{relation de localisation} et \(Rla_i\) des
\emph{relations de localisation atomiques.}

Si cette amélioration permet la prise en compte des \emph{relations de
  localisation atomiques,} elle ignore le second problème, celui de la
prise en compte des \emph{objets de référence} multiples. Il est en
effet trop restrictif de considérer qu'un \emph{indice de
  localisation} ne peut contenir qu'un seul \emph{objet de référence.}
Il y a tout d'abord le cas, déjà présenté, des \emph{relations de
  localisation} \emph{bi} ou \emph{n-aire} (\eg \enquote{Je suis entre
  Lyon et Grenoble} ou \enquote{Je suis de l'autre côté de la vallée
  par rapport à la forêt}) qui décrivent une position à partir de deux
ou de plusieurs \emph{objets de référence}. Il est donc nécessaire que
\emph{l'objet de référence} \(Or\) puisse contenir plusieurs
\emph{objets de référence.} Cette possible pluralité des \emph{objets
  de référence} est cependant bien différente de celle des
\emph{relations de localisation} précédemment introduite. Il n'est en
effet pas nécessaire que les \emph{relations de localisation
  atomiques} soient ordonnées. Pour reprendre l'exemple utilisé dans
la \autoref{subsec:4-1-2-1}, définir la \emph{relation de
  localisation} \enquote{sous} comme la combinaison des
\emph{relations} \enquote{proche de} et \enquote{à une altitude
  inférieure a} est identique à la définir comme la combinaison des
\emph{relations} \enquote{à une altitude inférieure a} et
\enquote{proche de}. Au contraire, si certaines relations, comme
\enquote{entre} sont commutatives ce n'est pas toujours le cas, comme
pour la relation \enquote{de l'autre côté de, par rapport à}. Il est
donc nécessaire que les \emph{objets de référence} multiples soient
représentés par un n-tuplet, par définition ordonné.

Il existe cependant une autre situation, orthogonale à la précédente,
où plusieurs \emph{objets de référence} peuvent être employés. On
peut, en effet, décrire une position par un \emph{objet de référence}
de deux manières, soit en se référant à un objet nommément (\eg
\enquote{Grenoble}, \enquote{le dôme des Écrins}), soit en le
décrivant par sa nature (\eg \enquote{une ville}, \enquote{un
  sommet}). Ces deux manières de désigner \emph{l'objet de référence}
ont des implications différentes. Lorsqu'un \emph{objet} est désigné
par son nom on se réfère à une instance donnée, le nom fait alors
office d'identifiant unique \footnote{Des cas d’homonymie restent
  cependant possibles, même si c'est rarement le cas aux grandes
  échelles spatiales.} et seul un seul \emph{objet de référence} est
contenu dans \emph{l'indice de localisation}. À l'inverse, lorsqu'un
\emph{objet de référence} est uniquement désigné par son type on ne
peut que rarement identifier une instance précise. Dire, toutes choses
égales par ailleurs, \enquote{je suis proche d'une maison} est une
description valide pour toutes les maisons présentes dans la zone
\emph{initiale de recherche.} La \emph{spatialisation} d'une telle
description de position nécessite donc de prendre en considération
toutes les \emph{instances} de type \enquote{maison} présentes dans la
\ac{zir}. Ainsi, comme pour les \emph{relations de localisation,}
l'objet \(Or\) n'est pas un objet individuel, mais un ensemble (non
ordonné) \emph{d'objets} (ou de n-tuple \emph{d'objets}) \emph{de
  référence.} \(Or\) est donc définissable de la manière suivante :

\begin{equation}
  Or = \{(or_1, \ldots,  or_n) \mid \forall x \in Or, |x| ≥ 1 \wedge
  \forall x,y \in Or^2, |x| = |y|\}
\end{equation}

Avec \(Or\) l'ensemble des \emph{objets de référence} et \(or_i\) un
\emph{objet de référence} donné.


%%% Local Variables:
%%% mode: latex
%%% TeX-master: "../../../../main"
%%% End:
