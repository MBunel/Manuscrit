
La dernière alerte modélisée est le cas dit du \enquote{fil rouge} que
nous avions présenté dans le premier chapitre de cette thèse. Comme
nous l'avions indiqué lors de sa première présentation, le \emph{fil
  rouge} est une alerte particulière, contenant de nombreux indices
très imprécis et incertains. Par conséquent la \ac{zir} est
extrêmement étendue.


\subsection{Présentation de l'alerte}
\label{subsec:9-4-1}

Le fil rouge est une alerte composée de XX \emph{indices de
  localisation.}


Faute d'enregistrement audio, le fil rouge n'a pas été retranscrit,
les indices dont nous disposons (et qui avaient été présentés dans le
\autoref{chap:01}) ont donc été directement saisis par le secouriste.
%
de ce fait, le \emph{fil rouge} est une alerte assez différente des
précédentes, puisque XXX n'est pas
%
Le traitement du \emph{fil rouge} est donc plus proche du traitement
opérationnel qui sera réellement effectué lors du traitement d'une
alerte.


Pour le cas du \emph{fil rouge} nous ne disposons pas de
l'enregistrement audio et donc de son verbatim, les indices ayant déjà 
été synthétisés par les secouristes. Les les avions déjà présentés
dans le \autoref{chap:01}.
%
Pour rappel voici la synthèse de l'alerte qui nous a été donnée par
les secouristes :
%
\begin{enumerate}
\item La victime est partie de \emph{Bourg-d'Oisans,} à pied, sur
  chemin, en direction d'une station de ski.
\item La victime a marché plusieurs heures.
\item La victime a chuté de plusieurs mètres.
\item La victime voit une partie de plan d'eau.
\item La victime est sous une route et entend des véhicules.
\item La victime est sous une ligne électrique 3 brins.
\item La victime vient de passer du soleil à l'ombre.
\end{enumerate}

Comme nous l'avions déjà signalé lors de la présentation du \emph{fil
  rouge} (\autoref{chap:01}), ces différentes informations sont
fortement imprécises, surtout si on les compare aux extraits des
alertes \emph{Grand Veymont} (\autoref{anx:retrans_grand_veymont}) et
\emph{Moucherotte} (\autoref{anx:retrans_moucherotte}). Les
descriptions sont ici peu informatives et discriminantes. La
construction d'indices de localisations est ici plus délicate.

La première de ces sept descriptions donne beaucoup d'informations sur
le trajet suivit par la victime jusqu'à sa position actuelle. On
connait sont point de départ, clairement identifié, la destination
visée, une station de ski inconnue et son mode de déplacement, à pied
sur un chemin. Cependant la plupart des informations contenues dans
cette description ne sont pas directement utilisables pour construire
un indice de localisation et devrons être combinées à d'autres pour
avoir suffisamment d'informations pour construire un indice de
localisation. On pourrait par exemple augurer que l'information
\enquote{La victime est partie de \emph{Bourg-d'Oisans}} est
suffisante pour utiliser la relation de localisation
\onto[orl]{Aux\-Alentours\-De} ou \onto[orl]{Proximal}. Cependant,
rien dans cette première description ne nous permet d'affirmer que la
victime est toujours à proximité de \emph{Bourg-d'Oisans,} en fonction
de la durée de sa sortie il peut avoir parcouru suffisamment de
kilomètres pour que ces indices ne soient plus valides, c'est pourquoi
il nous semble préférable de ne pas créer, pour le moment, ces indices
de localisation. De manière similaire nous pourrions envisager
d'utiliser la relation de localisation
\onto[orla]{Si\-tue\-Sur\-Itineraire\-Ou\-Reseau\-Support} pour
prendre en compte le fait que la victime ait indiqué avoir utilisé un
chemin. Cependant, il est une nouvelle fois impossible d'affirmer que
la victime se situe toujours sur un chemin, c'est pourquoi cette
seconde possibilité doit également attendre d'être confirmée par de
nouvelles informations.

Il est cependant possible de construire un indice de localisation à
partir de cette première description. En effet, la victime indique
qu'elle est partie de \emph{Bourg-d'Oisans} dans la direction
\enquote{globale} d'une station de ski. Ainsi, 
%
On peut représenter cette information en utilisant la relation de
localisaiton \onto[orl]{Dans\-La\-Direction\-De\-X\-Depuis\-Y}.



Le second élément de cette alerte, \enquote{La victime a marché
  plusieurs heures}, nous donne une indication de la durée du
déplacement de la victime. Cette nouvelle information complète
considérablement la description précédente.
%
Grâce a ces deux informations nous savons désormais quel est le point
de départ de la victime et nous avons une estimation, certes
imprécise, du temps pendant lequel elle s'est déplacée.
%
La modélisation de ce type d'information est réalisable avec la
relation de localisation \onto[orl]{A\-Temps\-De\-Marche\-De}.




Le troisième élément de l'alerte nous indique que \enquote{la victime
  a chuté de plusieurs mètres}. Prise sous cette forme, cette
description n'est pas interprétable comme un indice de localisaiton et
aucune des relations que nous avons définies ne permet de
l'exprimer. Il est cependant possible de combiner cette information
avec les descriptions précédentes, de manière à construire un nouvel
indice de localisation. Ces descriptions nous renseignent, en effet,
sur le potentiel contexte de cette chute. Comme la victime a indiqué
qu'elle se déplaçait à pied, sur un chemin, nous pouvons supposer que
c'était également le cas lors de sa chute. Nous pouvons alors en
déduire que, suite à sa chute, la victime se situe sous un
chemin. Cette information extrapolée peut ainsi être modélisée avec la
relation de localisation \onto[orl]{Sous\-Altitude} ou l'une des
relations qui en dérivent. Compte-tenu du contexte de la chute, la
relation de localisation plus appropriée nous semble être
\onto[orl]{Sous\-Proche\-De}, puisqu’elle ajoute une contrainte de
proximité planimétrique. Ainsi s'il nous est impossible de construire
directement une \ac{zlc} grâce à l'information \enquote{J'ai chuté de
  plusieurs mètres}, il nous est en revanche possible de dériver de
cette information un indice de localisation exprimant la position
actuelle de la victime, par rapport au chemin dont elle vient de
chuter.




L'interprétation de la quatrième description de l'alerte, \enquote{La
  victime voit une partie d'un plan d'eau} est plus évidente que celle
des descriptions précédentes. En effet, deux relations de localisation
spécifiques ont été introduites pour traiter des relations de
visibilité \onto[orl]{Site\-Voit\-Cible} et
\onto[orl]{Cible\-Voit\-Site}. Ces différentes relations permettent de
distinguer les visibilités actives et passives. Dans le cas présent,
la victime décrit ce qu'elle voit, on est donc dans une situation
modélisable à l'aide de la relation de localisation
\onto[orl]{Cible\-Voit\-Site}.
%
Pour retranscrire le fait que la victime indique ne voir qu'une partie
du site il est nécessaire d'employer un modifieur, ce que nous
détaillerons lors de la modélisation.


De manière similaire, la cinquième description est assez explicite. La
victime indique qu'elle est située sous une route
%
Or, comme nous l’indiquions précédemment, cela implique qu'elle soit
%
On peut donc utiliser une nouvelle fois la relation de localisation
\onto[orl]{Sous\-Proche\-De} pour construire un nouvel indice de
localisation.
%
La seconde information qui nous est donnée, \enquote{la victime entend
  des véhicules}, n'est pas directement interprétable avec une
relation de localisation, pourtant cette information peut être
exploitée de manière à raffiner la modélisation de \emph{l'indice de
  localisation.}
%
On peut tirer deux informations différentes de cette indication. Tout
d'abord une indication de proximité.
%
Il serait même possible de définir une nouvelle relation de
localisation \onto{A\-Porte\-De\-Son} pour retranscrire cette
indication.
%
Une telle relation permettrait sans aucun doute d'affiner cette
approximation, particulièrement si elle permet la prise en compte des
effets d’échos, particulièrement présents en montagne, pouvant
conduire à une perception faussée de la distance d'un son.
%
La seconde information que l'on peut tirer de cet élément est la
nature de la route. On peut en effet considérer que, pour que la
victime note entendre une route, il est nécessaire que celle-ci soit
suffisamment fréquentée pour que le trafic soit notable. Un tel
critère permet d'affiner la sélection des objets de référence, qui,
compte-tenu de l’imprécision de la description, seront relativement
nombreux. Il est cependant délicat de procéder à une telle sélection
sans risquer de supprimer des objets de référence pertinents, et donc
de créer des faux négatifs. En effet, l'hypothèse que nous proposons
ici est assez contraignante et il nous semble inenvisageable de
l'appliquer sans la confirmation de la victime.


Dans la sixième description, la victime indique qu'elle se situe sous
une ligne électrique trois brins. Nous avons déjà présenté ce cas,
notamment dans le \autoref{chap:06} où nous comparions les différentes
implémentations de la théorie des sous-ensembles flous pour
représenter des zones de localisation. Si nous n'avions pas encore
présenté les différentes relations de localisation, définies dans
l'ontologie des relations de localisation, nous avions déjà expliqué
que cette description nécessitait que la relation de verticalité entre
le \emph{site} et la \emph{cible} soit contrainte selon l'axe
horizontal. Dit autrement, il est ici nécessaire d'être proche du
\emph{site} et à une altitude inférieure du \emph{site} pour que l'on
puisse affirmer qu'une position soit \enquote{sous} le \emph{site.}
Une relation de localisation, \onto{Sous\-Proche\-De} a spécifiquement
été définie pour modéliser ce type de relation.

Au terme de l'analyse de l'alerte on dispose donc de 6 indices de
localisation à spatialiser:
% 
\begin{enumerate}
\item \label{ind:fr1} La victime est
  \onto[orl]{Dans\-La\-Direction\-De\-X\-Depuis\-Y}
  \emph{Bourg-d'Oisans} (\texttt{X}) et une station de ski
  (\texttt{Y})
  % 
\item \label{ind:fr2} La victime est
  \onto[orl]{A\-Temps\-De\-Marche\-De} \emph{Bourg-d'Oisans} 
  % 
\item \label{ind:fr3} La victime est \onto[orl]{Sous\-Proche\-De} un
  chemin
  % 
\item \label{ind:fr4} La victime voit
  % 
\item \label{ind:fr5} La victime est \onto[orl]{Sous\-Proche\-De} une
  route
  % 
\item \label{ind:fr6} La victime est \onto[orl]{Sous\-Proche\-De} une
  ligne électrique trois brins
\end{enumerate}

\subsection{Modélisation de l'alerte}
\label{subsec:9-4-2}

La \ac{zir} définie pour cette alerte est une zone de
\SI{625}{\kilo\meter\squared}

\begin{figure}
  \centering
  \begin{tikzpicture}
  \tikzset{et/.style={above,font=\footnotesize\vphantom{Ag}}}
  % 
  \node[inner sep=0pt, anchor=south west] (image) at (0,0){\includegraphics{./figures/ZIR_fil_rouge.png}};
  % 
  \begin{scope}
    \node (P2) at ([yshift=-.5cm]image.south east) {};
    \node (P1) at ([yshift=-.5cm]image.south west) {};
    % 
    \node (rect) [anchor=north west, minimum width=1cm,minimum
    height=.25cm] at ([yshift=-.25cm]P1) {}; \path[draw=RdBu-9-1, line
    width=1mm](rect.west) --([xshift=-1ex]rect.south) -- ([xshift=1ex]rect.north)
    -- (rect.east);
    % 
    \node[anchor=west, font=\tiny\vphantom{Ag}, text width = 4cm] at
    ([xshift=1ex]rect.east) {Limite de la \ac{zir}};
    % Échelle
    % Échelle
    \draw[-] (P2 |- -1cm,-1cm) --++ (-1,0) node[et,pos=.5] {\SI{2,5}{\kilo\meter}};
    % Légende détaillée
    \path (P1) -- (P2) node[pos=.5, yshift=-1cm] {\tiny Pour la légende détaillée du fond topographique voir \autoref{anx:topo_leg}. Sources: BD TOPO 2018, BD ALTI 2018.}; 
  \end{scope}
\end{tikzpicture}
  \caption{Zone initiale de recherche pour le \emph{fil rouge}}
  \label{fig:zir_fil_rouge}
\end{figure}

\subsubsection{La victime est \protect\onto[orl]{Sous\-Proche\-De} un chemin}

L'indice de localisation suivant (\ref{ind:fr3}) utilise la relation
de localisation \onto[orl]{Sous\-Proche\-De} que nous avions déjà
présenté, sans la nommer, dans le \autoref{chap:06}. Cette relation de
localisation n'est pas atomique, elle est définie comme la conjonction
de deux relations de localisation atomiques,
\onto[orla]{Sous\-Altitude} et \onto[orla]{Proximal}.

La première de ces deux relations de localisation atomiques
\onto[orla]{Sous\-Altitude} a déjà été employée dans les exemples
précédents, comme pour la spatialisation de l'indice de localisaiton :
\enquote{Le requérant est \onto[orl]{Sous\-Altitude} du sommet du
  \emph{Grand Veymont}} (\ref{ind:gv2}) pour l'alerte :
\enquote{\emph{Grand Veymont}}. Pour rappel, cette relation de
localisation atomique emploie \onto[orla]{Geometrie} comme rasteriser,
\onto[orla]{Delta\-Nearest\-Val} comme métrique et
\onto[orla]{Inf\-Val\-0} comme fuzzyficateur
(\autoref{fig:ex_parties_statialisation_sousalt}).

Le seconde relation de localisaiton atomique, \onto[orla]{Proximal},


\subsubsection{La victime est \protect\onto[orl]{Sous\-Proche\-De} une ligne
  électrique trois brins}

Comme pour l'indice précédent (\ref{ind:fr5}, ce sixième indice de
localisation emploie la relation de localisation
\onto[orla]{Sous\-Proche\-De}.
%
La spatialisation de cet indice de localisaiton nécessite donc de
construire la \ac{zlc} valisant les relations de localisation
atomiques \onto[orla]{Sous\-Altitude} et \onto[orla]{Proximal}.

Objets de ref

\subsection{Critique de la modélisation}
\label{subsec:9-4-3}

%%% Local Variables:
%%% mode: latex
%%% TeX-master: "../../../../main"
%%% End:
