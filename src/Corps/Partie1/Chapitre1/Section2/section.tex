Une des solutions explorées par les secouristes pour améliorer la
gestion de ces cas critiques a été de participer à la mise en place
d'un projet de recherche destiné à développer des solution innovantes
visant à faciliter la localisation de personnes perdues en montagne.


\subsection{Origine du projet}
\label{subsec:1-2-1}

Ici on présente l'origine du projet choucas, comment il s'est
construit, suite au fil rouge. 

Le cas du \emph{fil rouge} a mis en évidence certaines limites dans la
méthode de localisaiton des victimes par les \ac{usem}, a savoir que,
dans le cas où la solution \emph{Gend'Loc} n'est pas utilisable, il
peut s'avérer extrèmement difficile de localiser une personne perdue
ou bléssée en montagne. Une réflexion a donc été amorcée au sein du
\ac{pghm} du Grenoble pour trouver solutions permettant d'améliorer la
gestion de ces cas, qui bien que minoritaires, sont extrèmement
problématiques. Une des voies osuivie a été d'organiser des
collaborations avec des laboratoires de recherche pour mettre en place
une réflecion autour de ces questions.

Une de ces collaborations a abouti à création du projet de recherche
Choucas \footnote{\url{http://choucas.ign.fr/}}, dont la note
d'intention est d'appliquer l'expérience des chercheurs de l'IGN dans
la manipulation et le traitement de \emph{données géographiques
  vagues} au secours en montagne.


\subsection{Problématique du projet}
\label{subsec:1-2-2}

La résolution du cas \emph{fil rouge} s'est heurtée à deux problèmes
principaux, le premier est celui de la disponibilité des données

Le second est celui du raisonnement

\subsection{Objectifs scientifiques du projet}
\label{subsec:1-2-3}

Pour répondre à cette problématique, le projet s'organise autour de la
réponse à 2 objectifs principaux.

\subsubsection{Structuration des données issues de sources textuelles
  hétérogènes}

\subsubsection{Intégration de sources hétérogènes spatialisables}

\subsubsection{Raisonnement spatial qualitatif flou}


\subsubsection{Géovisualisation de données multidimensionnelles et
  imparfaites pour la prise de décision}

Vient enfin la question de la visualisation des résultats. L'objectif
de cet axe de recherche est de 


\texttt{infographie structure projet}
%%% Local Variables:
%%% mode: latex
%%% TeX-master: "../../../../main"
%%% End:
