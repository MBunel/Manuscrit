La dernière partie de la \emph{phase de spatialisation} est la
\emph{fuzzyfication de la métrique.} C'est à ce moment qu'est créée la
\emph{zone de localisation compatible.}

\subsection{XXX}

\begin{figure}
  \centering
  \includegraphics[width=\textwidth]{../../../../tmp/selecteur.jpeg}
  \caption{Méthodes de sélection}
  \label{fig:methode_selecteur}
\end{figure}

\tdi{Présentation des différents sélecteurs}

\tdi{Sélecteur = forme}

\tdi{La sémantique est dans la forme}%
Comme nous le mentions précédemment (\autoref{chap:05}), le choix de la forme de la fonction d'appartenance est essentiel à la \emph{spatialisation,} puisqu'il
%
Ainsi, la sémantique est dans la forme de la fonction d'appartenance.

\tdi{Expliquer en quoi la définition des seuils est un problème
  d'implémentation
}%
La question de la définition des seuils de la fonction d'appartenance utilisée est également importante. Toutefois, il s'agit 


\begin{figure}
  \centering
  \begin{tikzpicture}[scale=.7]
  \def\decalageX{-.2}
  \def\decalageY{-.2}
  % Courbe
  \begin{scope}[transparency group]
    % fond
    \begin{scope}
      \path[ffa]  (3,0) -- (4.5, 2) -- (6,0) -- cycle;
    \end{scope}
    % bords
    \begin{scope}
      \path[ffc] (1,0) -- (3,0) -- (4.5, 2) -- (6,0) -- (8,0) ;
      \path[ffc_fade_m] (0,0) -- (1,0) ;
      \path[ffc_fade] (8,0) -- (9,0) ;
    \end{scope}
  \end{scope}
  % Axes X, Y
  \begin{scope}
    % Axe X
    \begin{scope}
      % Axe
      \draw[<->] (0, \decalageX) --++ (9, 0) coordinate (x axis);
      % Graduations
      \foreach \n/\t in {1/{},2/{},3/{},4/{},5/{},6/{},7/{},8/{}}
      {
        \draw[-] (\n, \decalageX - .05) --++ (0, .1);
        \node[below, font=\footnotesize] at (\n, \decalageX - .05) {\t};
      }
      % label
      \node[below left, yshift=-.1cm, font=\small] at (x axis) {\itshape Métrique};
    \end{scope}
    % Axe Y
    \begin{scope}
      % Axe
      \draw[-] (\decalageY ,0) --++ (0, 2) coordinate (y axis);
      % Graduations
      \foreach \n/\t in {0/{0},2/{1}}
      {
        \draw[-] (\decalageY -.05, \n) --++ (.1, 0);
        \node[left, font=\footnotesize] at (\decalageY -.05, \n) {\t};
      }
      % Label
      \node[above] at (y axis) {$\mu$};
    \end{scope}
  \end{scope}
  \begin{scope}
    % Seuil 1
    \draw[ffc,line width=.5] (3,\decalageY) -- (3,0);
    \draw[fill, RdBu-9-1] (3,\decalageY) circle (2pt);
    \draw[fill, RdBu-9-1] (3,0) circle (2pt);
    % Seuil 2
    \draw[ffc,line width=.5] (4.5,\decalageY) -- (4.5,2);
    \draw[fill, RdBu-9-1] (4.5,\decalageY) circle (2pt);
    \draw[fill, RdBu-9-1] (4.5,2) circle (2pt);
    \node[above] at (4.5,2) {\(v\)};
    % Seuil 3
    \draw[ffc,line width=.5] (6,\decalageY) -- (6,0);
    \draw[fill, RdBu-9-1] (6,\decalageY) circle (2pt);
    \draw[fill, RdBu-9-1] (6,0) circle (2pt);
    \draw[|-|] (3,-.7cm) --++(3,0) node[pos=.5, fill=white, inner
    sep=1pt, font=\small] {$\delta$};
  \end{scope}
\end{tikzpicture}

  \caption{eqval}
  \label{fig:select_eqval}
\end{figure}


\begin{figure}
  \centering
  \begin{tikzpicture}[scale=.7]
  \def\decalageX{-.2}
  \def\decalageY{-.2}
  % Courbe
  \begin{scope}[transparency group]
    % fond
    \begin{scope}
      \path[ffa]  (3,0) -- (6, 2)  -- (8,2) -- (8,0) -- cycle;
      \path[ffa_fade]  (8,2) -- (9, 2)  -- (9,0) -- (8,0) -- cycle;
    \end{scope}
    % bords
    \begin{scope}
      \path[ffc_fade_m] (0,0) -- (2,0) ;
      \path[ffc] (2,0) -- (3,0) -- (6, 2) -++ (3,0) ;
      \path[ffc_fade] (8,2) -- (9,2) ;
    \end{scope}
  \end{scope}
  % Axes X, Y
  \begin{scope}
    % Axe X
    \begin{scope}
      % Axe
      \draw[<->] (0, \decalageX) --++ (9, 0) coordinate (x axis);
      % Graduations
      \foreach \n/\t in {1/{},2/{},3/{},4/{},5/{},6/{},7/{},8/{}}
      {
        \draw[-] (\n, \decalageX - .05) --++ (0, .1);
        \node[below, font=\footnotesize] at (\n, \decalageX - .05) {\t};
      }
      % label
      \node[below left, yshift=-.1cm, font=\small] at (x axis) {\itshape Métrique};
    \end{scope}
    % Axe Y
    \begin{scope}
      % Axe
      \draw[-] (\decalageY ,0) --++ (0, 2) coordinate (y axis);
      % Graduations
      \foreach \n/\t in {0/{0},2/{1}}
      {
        \draw[-] (\decalageY -.05, \n) --++ (.1, 0);
        \node[left, font=\footnotesize] at (\decalageY -.05, \n) {\t};
      }
      % Label
      \node[above] at (y axis) {$\mu$};
    \end{scope}
  \end{scope}
  \begin{scope}
    \draw (6,\decalageY) -- (6,2);
    \draw[fill] (6,\decalageY) circle (1pt);
    \draw[fill] (6,2) circle (1pt);
    \draw[fill] (3,0) circle (1pt);
    \node[above] at (6,2) {P};
    %
    \draw[|-|] (3,-.7cm) --++(3,0) node[pos=.5, fill=white, inner
    sep=1pt, font=\small] {$\delta$};
  \end{scope}
\end{tikzpicture}

  \caption{supval}
  \label{fig:select_supval}
\end{figure}


\begin{figure}
  \centering
  \begin{tikzpicture}[scale=.7]
  \def\decalageX{-.2}
  \def\decalageY{-.2}
  % Courbe
  \begin{scope}[transparency group]
    % fond
    \begin{scope}
      \path[ffa_fade_m]  (0,2) -- (1, 2)  -- (1,0) -- (0,0) -- cycle;
      \path[ffa]  (1,2) -- (3, 2)  -- (6,0) -- (1,0) -- cycle;
    \end{scope}
    % bords
    \begin{scope}
      \path[ffc_fade_m] (0,2) -- (2,2) ;
      \path[ffc] (2,2) -- (3,2) -- (6,0) -++ (3,0) ;
      \path[ffc_fade] (8,0) -- (9,0) ;
    \end{scope}
  \end{scope}
  % Axes X, Y
  \begin{scope}
    % Axe X
    \begin{scope}
      % Axe
      \draw[<->] (0, \decalageX) --++ (9, 0) coordinate (x axis);
      % Graduations
      \foreach \n/\t in {1/{},2/{},3/{},4/{},5/{},6/{},7/{},8/{}}
      {
        \draw[-] (\n, \decalageX - .05) --++ (0, .1);
        \node[below, font=\footnotesize] at (\n, \decalageX - .05) {\t};
      }
      % label
      \node[below left, yshift=-.1cm, font=\small] at (x axis)
      {\itshape Métrique};
    \end{scope}
    % Axe Y
    \begin{scope}
      % Axe
      \draw[-] (\decalageY ,0) --++ (0, 2) coordinate (y axis);
      % Graduations
      \foreach \n/\t in {0/{0},2/{1}}
      {
        \draw[-] (\decalageY -.05, \n) --++ (.1, 0);
        \node[left, font=\footnotesize] at (\decalageY -.05, \n) {\t};
      }
      % Label
      \node[above] at (y axis) {$\mu$};
    \end{scope}
  \end{scope}
  \begin{scope}
    \draw (3,\decalageY) -- (3,2);
    \draw[fill] (3,\decalageY) circle (1pt);
    \draw[fill] (3,2) circle (1pt);
    \draw[fill] (6,0) circle (1pt);
    \node[above] at (3,2) {P};
    % 
    \draw[|-|] (3,-.7cm) --++(3,0) node[pos=.5, fill=white, inner
    sep=1pt, font=\small] {$\delta$};
  \end{scope}
\end{tikzpicture}

  \caption{infval}
  \label{fig:select_infval}
\end{figure}

\subsection{Les modifieurs}

\tdi{Modifeur not}

\tdi{resampling}

\begin{figure}
  \centering
  \includegraphics[width=\textwidth]{../../../../tmp/modifieur.jpeg}
  \caption{Modifieur}
  \label{fig:methode_modifieur}
\end{figure}



%%% Local Variables:
%%% mode: latex
%%% TeX-master: "../../../../main"
%%% End:
