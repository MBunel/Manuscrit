Ici il faudra mettre le texte d'introduction de la première partie.

Common Lisp est un dialecte de Lisp standardisé par l'ANSI X3.226-1994. Développé pour standardiser les variantes divergentes de Lisp qui l'ont précédé, ce n'est pas une implémentation mais une spécification à laquelle les implémentations Lisp essayent de se conformer. Il est fréquemment abrégé en CL.

Common Lisp est un langage de programmation à usage général, a contrario de dialectes de Lisp comme Emacs Lisp et AutoLisp, qui sont des langages d'extension embarqués dans des produits particuliers. Contrairement à de nombreux Lisp plus anciens, mais comme Scheme, Common Lisp utilise la portée lexicale par défaut pour les variables.

Common Lisp est un langage de programmation multi-paradigmes qui :

Accepte des techniques de programmation impérative, fonctionnelle et orientée objet (CLOS).
Est typé dynamiquement, mais avec des déclarations de type optionnelles qui peuvent améliorer l'efficacité et la sûreté,
Dispose d'un système de gestion d'exceptions puissant, nommé Condition System (système de gestion de conditions),
Est syntaxiquement extensible à travers des fonctionnalités comme les macros et les macros de lecture.