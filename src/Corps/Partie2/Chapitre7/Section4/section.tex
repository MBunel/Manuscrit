La dernière partie de la \emph{phase de spatialisation} est la
\emph{fuzzyfication de la métrique.} C'est à ce moment qu'est créée la
\emph{zone de localisation compatible.}

\subsection{XXX}

\begin{figure}
  \centering
  \includegraphics[width=\textwidth]{../../../../tmp/selecteur.jpeg}
  \caption{Méthodes de sélection}
  \label{fig:methode_selecteur}
\end{figure}

\tdi{Présentation des différents sélecteurs}

\tdi{Sélecteur = forme}

\tdi{La sémantique est dans la forme}%
Comme nous le mentions précédemment (\autoref{chap:05}), le choix de la forme de la fonction d'appartenance est essentiel à la \emph{spatialisation,} puisqu'il
%
Ainsi, la sémantique est dans la forme de la fonction d'appartenance.

\tdi{Expliquer en quoi la définition des seuils est un problème
  d'implémentation
}%
La question de la définition des seuils de la fonction d'appartenance utilisée est également importante. Toutefois, il s'agit 


\begin{figure}
  \centering
  \input{../figures/fnc_eq_val.tex}
  \caption{eqval}
  \label{fig:select_eqval}
\end{figure}


\begin{figure}
  \centering
  \input{../figures/fnc_sup_val.tex}
  \caption{supval}
  \label{fig:select_supval}
\end{figure}


\begin{figure}
  \centering
  \input{../figures/fnc_inf_val.tex}
  \caption{infval}
  \label{fig:select_infval}
\end{figure}

\subsection{Les modifieurs}

\tdi{Modifeur not}

\tdi{resampling}

\begin{figure}
  \centering
  \includegraphics[width=\textwidth]{../../../../tmp/modifieur.jpeg}
  \caption{Modifieur}
  \label{fig:methode_modifieur}
\end{figure}



%%% Local Variables:
%%% mode: latex
%%% TeX-master: "../../../../main"
%%% End:
