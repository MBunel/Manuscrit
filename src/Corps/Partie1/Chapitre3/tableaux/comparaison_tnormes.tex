\subfloat[Degré d'appartenance résultant en fonction de la
\emph{t-norme} utilisé]{
  \begin{tabular}{rl}
    \toprule
    \emph{t-norme} & \(a ∧ b\) \\
    \midrule
    \bsc{Zadeh} & \(⊤_Z(0,8, 0,5) = 0,5 \) \\
    \bsc{Łukasiewicz} & \(⊤_L(0,8, 0,5) = 0,3\) \\
    Probabiliste & \(⊤_P(0,8,0,5) = 0,4 \) \\
    Drastique & \(⊤_D(0,8, 0,5) = 0,0 \) \\
    \bottomrule
  \end{tabular}
  \label{ss}
}
\hspace{1cm}
\subfloat[Degré d'appartenance résultant en fonction de la
\emph{t-norme} utilisé]{
  \begin{tabular}{rl}
    \toprule
    \emph{t-conorme} & \(a ∨ b\) \\
    \midrule
    \bsc{Zadeh} & \(⊥_Z(0,8, 0,5) = 0,8 \) \\
    \bsc{Łukasiewicz} & \(⊥_L(0,8, 0,5) = 1,0\) \\
    Probabiliste & \(⊥_P(0,8,0,5) = 0,9 \) \\
    Drastique & \(⊥_D(0,8, 0,5) = 1,0 \) \\
    \bottomrule
  \end{tabular}
  \label{ss2}
}