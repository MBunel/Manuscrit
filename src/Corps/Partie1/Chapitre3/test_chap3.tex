\chaptertoc{}

\section{Bou}

\blindtext

\begin{table}
  \centering
  \begin{tabular}{
      % Définition de la première colonne
      % Barre verticale, puis texte en gras et bleu,
      % puis valeur centrée, puis l'unitée
    | >{\color{blue}\bfseries} c <{\degres C}
    % La | est remplacée par une flèche 
    !{$\rightarrow$} 
    % Colonne 2, la valeur est précédée
    % du mot eau, puis texte à gauche en italique
    >{l'eau \itshape}l|
    } 
    \hline 
    -1 & gel\\ 
    \hline 
    90 & bout\\ 
    \hline 
    22 & est bonne\\ 
    \hline 
  \end{tabular}
  \caption{test array}
\end{table}

\blindtext

\begin{equation}
  FoldList \equiv \lambda f\, d\, xs\,.\, xs\ d\ \big(\lambda\ h\ t \,.\, f\ h\ (FoldList\ f\ d\ t)\big)
\end{equation}


\begin{equation}
 FoldList ≡ λ f d xs . xs d (λ h t . f h (FoldList f d t))
\end{equation}

\begin{equation}
  \begin{aligned}
    &\to_\beta FoldList \left[\,f := zou,\ d := 0,\ xs := (1,2,3,4)\right]\\
    & \equiv (1,2,3,4)\ 0\ \big(\lambda\,h\,t \,.\, zou\, h(FoldList \, zou \, 0 \, t)\big)\\
    &\to_\beta \Big(\lambda\ h\ t \,.\, zou\, h[h := 1]\ \big(FoldList \ zou \ 0 \ t[t:=(2,3,4)]\big)\Big)\\
    & \equiv zou\ 1\ \big(FoldList \ zou \ 0 \ (2,3,4)\big) \\
    & \equiv zou\ 1\ \Big( zou\ 2\ \big(FoldList \ zou \ 0 \ (3,4)\big)\Big) \\
    & \equiv zou\ 1\ \big( zou\ 2\ (zou\ 3\ (FoldList \ zou \, 0 \, (4))))) \\
    & \equiv zou\ 1\ \big( zou\ 2\ (zou\ 3\ (zou\ 4 \ (FoldList \, zou \, 0 \, ()))))) \\
    & \equiv zou\ 1\ \Big( zou\, 2\ \big(zou\, 3\ (zou\, 4\, 0) \big) \Big) \\
    &\to_\beta zou\ 1\ ( zou\, 2\ (zou\, 3\ (zou\, 4\, 0)))))[zou := \lambda\,x\,y\,.\, x+y] \\
    & \equiv 1 + \Big( 2 + \big( 3 + (4 + 0)\big)\Big) \\
    & \equiv 1 + 2 + 3 + 4 + 0 \\
  \end{aligned}
\end{equation}
\blindtext

Voire le \autoref{lst:fuzzyfier} et le \autoref{lst:ff}

\begin{code}
  \begin{minted}{python}
    class FuzzyfierMoreSpeeeeed(Fuzzyfier):
        def __init__(self, context):
            super().__init__(context)
   \end{minted}
   \caption{oputxff}
      \label{lst:fuzzyfier}
\end{code}

\blindtext

\begin{code}
  \begin{minted}{python}
    def def_fuzzyfy_function(self, *args, **kwargs):
    """
    fonction chargée de la génération de la fonction "fuzzy_fun".
    "fuzzy_fun" crée une copie fuzzyfiée du raster donné en entrée.
    """

    # Tri des paramètres
    sorted(args, key=lambda x: x[0])
    largs = len(args)
    # Définition de la fonction
    # Liste de patterns en fonction du nombre de paramètres
    fun_rules = {
      # Cas crisp
      1: (self._inf_vals,),
      # Cas où la fonction est une droite
      2: (self._inf_vals, self._fst_slp, self._sup_vals),
      # la fonction est de forme triangulaire
      3: (self._inf_vals, self._fst_slp, self._lst_slp, self._sup_vals),
      # forme trapézoidale
      4: (
      self._inf_vals, self._fst_slp,
      self._cnt_flt, self._lst_slp, self._sup_vals,
      ),
    }

    # Caneva de la fonction "fuzzy fun"
    def fuzzy_fun(raster):
    raster_copy = np.zeros_like(raster)
    try:
    # On récupère la liste d'instruction correspondant
    au nombre d'arguments
    for ins in fun_rules[largs]:
    # On applique les fonctions, dans l'ordre du tuple
    # "fun_rules[largs]"
    ins(raster, raster_copy, *args)
    return raster_copy
    except KeyError:
    raise ValueError("2, 3 or 4 parameters needed")

    # Renvoi de la fonction générée
    return fuzzy_fun
  \end{minted}
  \caption{oputx}
  \label{lst:ff}
\end{code}

\blindtext

%%% Local Variables:
%%% mode: lualatex
%%% TeX-master: "../main"
%%% End:
