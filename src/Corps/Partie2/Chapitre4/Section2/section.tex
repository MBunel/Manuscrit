Dans le \autoref{chap:2} nous expliquions que, dans notre cas, le
passage d'un référentiel spatial indirect à un référentiel direct
nécessitait deux étapes. La première d'entre elle, \emph{la
  spatialisation,} permettant de passer d'un \emph{indice de
  localisation} à une \emph{zone de localisation compatible} et la
seconde, la \emph{fusion} combinant les différentes \emph{zones de
  localisation compatibles} issues de la \emph{spatialisation} de tous
les \emph{indices de localisation} d'une alerte dans le but de créer
la \emph{zone de localisation probable,} \ie le résultat final de
notre processus.

Dans cette partie nous allons détailler la présentation de notre
méthodologie, en présentant les étapes composant notre méthodologie.

\subsection{Décomposition}

\subsection{Spatialisation}

Faite pour chaque indice (chap2)
Nécessite des méthodes spécifiques (eda)
Plusieurs objets de référence


\subsection{Fusion}

\missingfigure{Schéma synthèse méthode}