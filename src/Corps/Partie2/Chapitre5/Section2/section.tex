Pour formaliser les décompositions des \emph{relations de
  localisations} nous avons opté pour la construction d'une
\emph{ontologie} des \emph{relations de localisation atomiques}
contenant l'ensemble des \emph{relations de localisation} identifiées
comme pertinentes et leur(s) décomposition(s).

\tdi{Ajouter ontologie des objets de référence}

Pour identifier les \emph{relations de localisations} pertinentes pour
notre contexte applicatif nous avons entrepris de recenser celles qui
sont les plus utilisées lors des alertes. Pour ce faire nous avons
retranscrit et analysé différents enregistrements d'alertes passées.

\subsection{Retranscriptions des alertes}

Le travail de retranscription des alertes (et dans une moindre mesure
celui de leur analyse) a été réalisé collaborativement
\autocite{Bunel2019}, les résultats pouvant bénéficier à différents
objectifs du projet Chouacs.

\subsubsection{Le corpus des alertes}

Les retranscriptions que nous avons effectuées pour définir
\emph{l'ontologie des relations de localisation atomiques} ont été
réalisées à partir d'un ensemble de 52 enregistrements audio qui nous
ont été transmis par le \ac{pghm} de Grenoble. Ces fichiers ont été
produits par les \ac{pghm} de Grenoble (5 alertes) et de Chamonix (47
alertes) à l'aide d'un enregistreur numérique. Comme les requérants
contactent généralement les \ac{codis} (cf. \autoref{chap:02}) les
enregistrements ne contient pas nécessairement toute la discussion
entre le requérant et les services de secours, les appels ne sont
enregistrés que lorsque un secouriste du \ac{pghm} intègre la
discussion. De plus les enregistrements qui nous ont été fournis ont
été modifiés, de sorte à censurer les informations soumises au secret
médical.

Ces dernières nous ont été transmises en deux corpus, un premier de 20
alertes en 2017 et un second de 32 alertes en 2019.

dont la durée varie de quelques secondes à plusieurs dizaines de
minutes (\autoref{fig:dist_temps_alertes}).

\begin{figure}
  \centering
   \begin{tikzpicture}
   \begin{axis}
     \addplot [boxplot] table [y index=0] {./Corps/Partie2/Chapitre5/figures/duration.dat};
     
  %\addplot [thick] gnuplot [raw gnuplot] {plot './Corps/Partie2/Chapitre5/figures/duration.dat' smooth kdensity};
\end{axis}
\end{tikzpicture}

  \caption{Distribution de la durée des alertes}
  \label{fig:dist_temps_alertes}
\end{figure}

\subsubsection{Template de retranscription}

Les alertes étant sous la forme d'enregistrements audio il a été
nécessaire de les retranscrire afin de les analyser.
%
Notre parti-pris n'a pas été de travailler à partir d'une
retranscription libre de ces alertes, mais à l'aide d'un tableau
prédéfini, un \emph{template de retranscription,} reprenant la
structure des \emph{indices de localisation} (cf. \autoref{chap:04}).


\begin{table}
  \centering
  \begin{tabular}{L{.3\textwidth}>{\footnotesize}p{.4\textwidth}}
  \toprule \multicolumn{1}{c}{\bfseries Colonne} &
  \multicolumn{1}{c}{\normalsize \bfseries Contenu} \\ \midrule
% Sémantique des relations spatiales
  \addlinespace
  Identifiant de l'extrait & Permet d'identifier une expression\\
  Identifiant de l'expression & Permet d'identifier une expression au
                                sein d'un extrait en comprenant plusieurs\\
  Extrait & Verbatim de la phrase transcrite\\
  Confiance & Permet au transcripteur de saisir sa confiance en
              \emph{l'indice de localisation} (uniquement si la saisie
              est faite par un secouriste)\\
  Timestamp & Début de l'extrait dans le fichier audio source\\
  Locuteur & Permet d'identifier le locuteur (\eg secouriste,
             requérant, témoin, \emph{etc.})\\
  Verbe & Verbe utilisé dans \emph{l'indice de localisation}\\
  Modifieur du verbe & (\eg marcher \emph{vite})\\
  Sujet & Sujet de \emph{l'indice de localisation}\\
  Modifieur du sujet & (\eg marcher \emph{vite})\\
  \emph{Relation de localisation} & \emph{Relation de localisation}
                                    utilisée dans l'extrait\\
  Modifieur de la relation de localisation & (\eg \emph{très} loin)\\
  \emph{Objet de référence} & Nom ou type de \emph{l'objet de référence}
                              (cette colonne peut être multipliée si la
                              \emph{relation de localisation} est bi ou n-aire)\\
  Modifieur de \emph{l'objet de référence} & (cette colonne peut être multipliée si la
                                             \emph{relation de localisation} est bi ou n-aire)\\
  Commentaires & Champ permettant au transcripteur de commenter sa
                 saisie ou les indications données par le requérant
                 (erreurs potentielles, fautes de prononciation, \emph{etc.}) \\
  \bottomrule
\end{tabular}

  \caption{Structure du template de retranscription}
  \label{tab:struct_temp}
\end{table}

\subsubsection{Analyse des retranscriptions}

\subsection{Construction des ontologies}

%%% Local Variables:
%%% mode: latex
%%% TeX-master: "../../../../main"
%%% End:
