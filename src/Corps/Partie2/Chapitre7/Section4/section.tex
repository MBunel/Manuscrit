La dernière partie de la \emph{phase de spatialisation} est la
\emph{fuzzyfication de la métrique.} C'est à ce moment qu'est créée la
\emph{zone de localisation compatible.}

\subsection{XXX}

\begin{figure}
  \centering
  \begin{tikzpicture}
  \def\decalageX{-.2}
  \def\decalageY{-.2}
  % Arrow
  \begin{scope}
    \path[draw, -, shorten >=12pt, shorten <=5pt] (1,2) |- (3.65,3);
    \path[draw, ->,shorten >=5pt, shorten <=5pt] (8.25,3) -- (10,3);
    \path[draw, -,shorten >=12pt, shorten <=5pt] (1,0) |- (3.65,-1);
    \path[draw, ->,shorten >=5pt, shorten <=5pt] (8.25,-1) -- (10,-1);
  \end{scope}
  % Représentation métrique
  \begin{scope}[local bounding box=met]
    % \path[ffa] (1,.25) rectangle (1.25, .5);
    \begin{scope}
      \foreach \x in {0,.25,...,1.75}{
        \foreach \y in {0,.25,...,1.75}
        {
          % Traçage pixels 
          \path[draw, line width=.01mm] (\x,\y) rectangle (\x +.25, \y +
          .25);
          % Calcul et représentation distance euclidienne
          \pgfmathsetmacro\radius{sqrt((\x-1.125)^2+(\y-.375)^2)*1.25}
          \fill[black] (\x+.125,.125+\y) circle (\radius pt);
        }
      }
      \path[ffc] (0,0) rectangle (2,2);
    \end{scope}
    \node[text width=3cm, align=center, anchor=north, font=\footnotesize,fill=white] at (1,-.25)
    {\itshape Métrique};
  \end{scope}
  % Fuzzyfication 1
  \begin{scope}[xshift=10cm, yshift=2cm,local bounding box=fuzz]
    \begin{scope}
      \foreach \x in {0,.25,...,1.75}{
        \foreach \y in {0,.25,...,1.75}
        {
          % Traçage pixels 
          \path[draw, line width=.01mm] (\x,\y) rectangle (\x +.25, \y +
          .25);
          % Calcul et représentation floue distance euclidienne
          \pgfmathsetmacro\dist{sqrt((\x-1.125)^2+(\y-.375)^2)}
          % Fuzzyfication de la distance
          \pgfmathsetmacro\fuzzy{%
            ifthenelse(\dist < .25,1,%
            ifthenelse(\dist < 2,-0.57*\dist+1.14,0)%
            )
          }
          % Calcul du rayon à partir de la fuzzyfication
          \pgfmathsetmacro\radius{\fuzzy*2.5}
          \fill[black] (\x+.125,.125+\y) circle (\radius pt);
        }
      }
      \path[ffc] (0,0) rectangle (2,2);
    \end{scope}
    \node[text width=3cm, align=center, anchor=north, font=\footnotesize] at (1,-.25)
    {\itshape Zone de localisation compatible \textrm{A}};
  \end{scope} 
  % Fuzzyfication 2
  \begin{scope}[xshift=10cm, yshift=-2cm,local bounding box=fuzz2]
    \begin{scope}
      \foreach \x in {0,.25,...,1.75}{
        \foreach \y in {0,.25,...,1.75}
        {
          % Traçage pixels 
          \path[draw, line width=.01mm] (\x,\y) rectangle (\x +.25, \y +
          .25);
          % Calcul et représentation floue distance euclidienne
          \pgfmathsetmacro\dist{sqrt((\x-1.125)^2+(\y-.375)^2)}
          % Fuzzyfication de la distance
          \pgfmathsetmacro\fuzzy{%
            ifthenelse(\dist < .25,1,%
            ifthenelse(\dist < 2,-0.57*\dist+1.14,0)%
            )
          }
          % Calcul du rayon à partir de la fuzzyfication
          \pgfmathsetmacro\radius{\fuzzy*2.5}
          \fill[black] (\x+.125,.125+\y) circle (\radius pt);
        }
      }
      \path[ffc] (0,0) rectangle (2,2);
    \end{scope}
    \node[text width=3cm, align=center, anchor=north, font=\footnotesize] at (1,-.25)
    {\itshape Zone de localisation compatible \textrm{B}};
  \end{scope}

  \begin{scope}[xshift=3.75cm,yshift=2.5cm, scale=.5]
    % Courbe
    \begin{scope}[transparency group]
      % fond
      \begin{scope}
        \path[ffa]  (3,0) -- (6, 2)  -- (8,2) -- (8,0) -- cycle;
        \path[ffa_fade]  (8,2) -- (9, 2)  -- (9,0) -- (8,0) -- cycle;
      \end{scope}
      % bords
      \begin{scope}
        \path[ffc_fade_m] (0,0) -- (2,0) ;
        \path[ffc] (2,0) -- (3,0) -- (6, 2) -++ (3,0) ;
        \path[ffc_fade] (8,2) -- (9,2) ;
      \end{scope}
    \end{scope}
    % Axes X, Y
    \begin{scope}
      % Axe X
      \begin{scope}
        % Axe
        \draw[->] (0, \decalageX) --++ (9, 0) coordinate (x axis);
        % Graduations
        \foreach \n/\t in {0/,{0},1/{},2/{},3/{},4/{},5/{},6/{},7/{},8/{}}
        {
          \draw[-] (\n, \decalageX - .05) --++ (0, .1);
          \node[below, font=\tiny] at (\n, \decalageX - .05) {\t};
        }
        % label
        \node[below left, font=\tiny] at (x axis) {\itshape Métrique};
      \end{scope}
      % Axe Y
      \begin{scope}
        % Axe
        \draw[-] (\decalageY ,0) --++ (0, 2) coordinate (y axis);
        % Graduations
        \foreach \n/\t in {0/{0},2/{1}}
        {
          \draw[-] (\decalageY -.05, \n) --++ (.1, 0);
          \node[left, font=\tiny] at (\decalageY -.05, \n) {\t};
        }
        % Label
        \node[above, font=\tiny] at (y axis) {$\mu$};
      \end{scope}
    \end{scope}
  \end{scope}

  \begin{scope}[xshift=3.75cm,yshift=-1.5cm, scale=.5]
    % Courbe
    \begin{scope}[transparency group]
      % fond
      \begin{scope}
        \path[ffa]  (3,0) -- (6, 2)  -- (8,2) -- (8,0) -- cycle;
        \path[ffa_fade]  (8,2) -- (9, 2)  -- (9,0) -- (8,0) -- cycle;
      \end{scope}
      % bords
      \begin{scope}
        \path[ffc_fade_m] (0,0) -- (2,0) ;
        \path[ffc] (2,0) -- (3,0) -- (6, 2) -++ (3,0) ;
        \path[ffc_fade] (8,2) -- (9,2) ;
      \end{scope}
    \end{scope}
    % Axes X, Y
    \begin{scope}
      % Axe X
      \begin{scope}
        % Axe
        \draw[->] (0, \decalageX) --++ (9, 0) coordinate (x axis);
        % Graduations
        \foreach \n/\t in {0/,{0},1/{},2/{},3/{},4/{},5/{},6/{},7/{},8/{}}
        {
          \draw[-] (\n, \decalageX - .05) --++ (0, .1);
          \node[below, font=\tiny] at (\n, \decalageX - .05) {\t};
        }
        % label
        \node[below left, font=\tiny] at (x axis) {\itshape Métrique};
      \end{scope}
      % Axe Y
      \begin{scope}
        % Axe
        \draw[-] (\decalageY ,0) --++ (0, 2) coordinate (y axis);
        % Graduations
        \foreach \n/\t in {0/{0},2/{1}}
        {
          \draw[-] (\decalageY -.05, \n) --++ (.1, 0);
          \node[left, font=\tiny] at (\decalageY -.05, \n) {\t};
        }
        % Label
        \node[above, font=\tiny] at (y axis) {$\mu$};
      \end{scope}
    \end{scope}
  \end{scope}
\end{tikzpicture}
  \caption{Méthodes de sélection}
  \label{fig:importance_fuzzyfication}
\end{figure}

\tdi{Présentation des différents sélecteurs}

\tdi{Sélecteur = forme}

\tdi{La sémantique est dans la forme}%
Comme nous le mentions précédemment (\autoref{chap:05}), le choix de la forme de la fonction d'appartenance est essentiel à la \emph{spatialisation,} puisqu'il
%
Ainsi, la sémantique est dans la forme de la fonction d'appartenance.

\tdi{Expliquer en quoi la définition des seuils est un problème
  d'implémentation
}%
La question de la définition des seuils de la fonction d'appartenance utilisée est également importante. Toutefois, il s'agit 

\tdi{Liste des fuzzyficateurs}

Nous proposons la définition de trois fonctions de \emph{fuzzyfication} principales, déclinées en plusieurs variantes, portant le total des \emph{fuzzyficateurs} à sept.

Le premier d'entre-eux est le \emph{fuzzyficateur} \onto{Eq\-Val}, défini par une fonction d'appartenance triangulaire (Figure \ref{fig:select_eqval_b}).

\begin{equation}
  \label{eq:eq_val}
  x = \left\{
    \begin{array}{ll}
      XX & \text{si}\ θ > -5° \\
      xx & \text{si}\ -12° ≤ θ ≤ -5° \\
      0 & \text{si}\ θ < -12°\ \text{ou}\ θ < -12°\\
    \end{array}
  \right.
\end{equation}


\begin{figure}
  \centering
  \subfloat[eqval]{
    \input{../figures/fnc_eq_val.tex}
    \label{fig:select_eqval_b}
  }
  
  \subfloat[eqval0]{
    \input{../figures/fnc_eq_val.tex}
    \label{fig:select_eqval_0}
  }\hfill  
  \subfloat[eqvalangle]{
    \input{../figures/fnc_eq_val.tex}
    \label{fig:select_eqval_ang}
  }  
  \caption{eqval}
  \label{fig:select_eqval}
\end{figure}

Le second \emph{fuzzyfieur,} \onto{Sup\-Val}, est défini par une fonction d'appartenance linéaire (Figure \ref{fig:select_supval_b}).

\begin{equation}
  \label{eq:sup_val}
  x = \left\{
    \begin{array}{ll}
      1 & \text{si}\ θ > -5° \\
      xx & \text{si}\ -12° ≤ θ ≤ -5° \\
      0 & \text{si}\ θ < -12°\\
    \end{array}
  \right.
\end{equation}

\begin{figure}
  \centering
  \subfloat[supval]{
    \input{../figures/fnc_sup_val.tex}
    \label{fig:select_supval_b}
  }\hfill
  \subfloat[supval0]{
    \input{../figures/fnc_sup_val.tex}
    \label{fig:select_supval_0}
  }
  \caption{supval}
  \label{fig:select_supval}
\end{figure}

Enfin, le dernier fuzzyfieur

\begin{equation}
  \label{eq:inf_val}
  x = \left\{
    \begin{array}{ll}
      1 & \text{si}\ θ > -5° \\
      xx & \text{si}\ -12° ≤ θ ≤ -5° \\
      0 & \text{si}\ θ < -12°\\
    \end{array}
  \right.
\end{equation}

\begin{figure}
  \centering
  \subfloat[infval]{
    \input{../figures/fnc_inf_val.tex}
  }\hfill
  \subfloat[infval0]{
    \input{../figures/fnc_inf_val.tex}
  }

  \caption{infval}
  \label{fig:select_infval}
\end{figure}

\subsection{Les modifieurs}

\tdi{Modifeur not}

\tdi{resampling}

\begin{figure}
  \centering
  \includegraphics[width=\textwidth]{../../../../tmp/modifieur.jpeg}
  \caption{Modifieur}
  \label{fig:methode_modifieur}
\end{figure}



%%% Local Variables:
%%% mode: latex
%%% TeX-master: "../../../../main"
%%% End:
