
\subsection{Présentation de l'alerte}
\label{subsec:9-3-1}




\begin{figure}
  \centering
  \input{./figures/ZIR_moucherotte.tex}
  \caption{Zone initiale de recherche de l'alerte
    \enquote{Moucherotte}}
  \label{fig:zir_moucherotte}
\end{figure}

\subsubsection{Identification des indices de localisation}
\label{subsec:9-3-1-1}

De la même manière que pour le traitement de l'alerte \emph{Grand
  Veymont,} il est nécessaire d'identifier tous les indices de
localisation contenus dans la retranscription de l'alerte. Comme
précédemment, les extraits figurant ici ont été modifiés par rapport
au verbatim de l'alerte, que l'on peut trouver dans l'annexe
\ref{anx:retrans_moucherotte}.



Dès les premiers extraits on peut remarquer une particularité de
l'alerte \emph{Moucherotte,} de nombreux extraits décrivent une
position passée. Par exemple, dans les deux premiers extraits, la
victime donne sa position à 17h30 :
% 
\begin{dialogue*}
  % 
  \Sec \lnex{1}{mc:1} Donc vous étiez au sommet du Moucherotte à 17h30.
  % 
  \Req \lnex{2}{mc:2} Ouais.
\end{dialogue*}

Ces deux premiers extrais décrivent une positon,
%
ce qui peut être
représenté par une relation de localisaiton \onto[orl]{Proximal}.

Cette relation de localisation n'est cependant pas directement
exploitable, puisqu'elle décrit la position de la victime dans le
passé.
%
Il serait cependant possible de la prendre en compte si nous
disposions d'autres informations, qui ne sont pas présentes dans la
retranscription de l'alerte mais qui auraient été connu durant
l'alerte.
%
Par exemple si nous connaissions l'heure de l'appel nous pourrions
définir un indice de localisation utilisant la relation de
localisation \onto[orl]{A\-Temps\-De\-Marche}, ce qui nous aurait
permis d'estimer la zone de présence de la victime.


\begin{dialogue*}
  \Sec \lnex{3}{mc:3} Vous êtes redescendue par le chemin que
  vous avez pris à la montée ?
  % 
  \Req \lnex{4}{mc:4} Oui, ben j’ai essayé de le trouver, je l’ai pas
  trouvé, \lnex{5}{mc:5} donc j’ai viré un peu à gauche,
  \lnex{6}{mc:6} sur ce grand couloir enseigné. \lnex{7}{mc:7} Je
  crois qu’il y a des gens qui redescendent par Lens-en-Vercors,
  \lnex{8}{mc:8} mais je suis montée de Sain-Nizier, du haut du
  tremplin.
\end{dialogue*}
% 
Les premiers extraits de cette alerte potent

\begin{dialogue*}
  % 
  \Sec \lnex{9}{mc:9} Vous êtes gardée à Saint-Nizier ?
  % 
  \Req \lnex{10}{mc:10} Saint-Nizier ouais, le haut du tremplin
\end{dialogue*}
% 
Comme les extraits \ref{mc:1} et \ref{mc:2} ces deux extraits donnent
une indication assez précise d'une position qui
%
On peut cependant interpréter les extraits \ref{mc:1} à \ref{mc:10}
pour construire un nouvel indice de localisation. La victime a, en
effet, décrit son point de départ, \enquote{le haut du tremplin}
(\ref{mc:10}) et d'arrivée, sommet du \emph{Moucherotte} (extraits
\ref{mc:1} et \ref{mc:2}).


C'est à partir de l'extrait \ref{mc:11} qu'apparaissent les premières
informations sur la position actuelle de la victime :
%
\begin{dialogue*}
  \Req \lnex{11}{mc:11} Mais là je suis complètement paumée,
  \lnex{12}{mc:12} dans les cailloux. \lnex{13}{mc:13} Jusqu’à
  maintenant j’étais dans la forêt, \lnex{14}{mc:14} un peu de
  neige. \lnex{15}{mc:15} Mais là, j’ai failli me tuer en glissant
  \lnex{16}{mc:16} parce qu’il y a plein de petits morceaux de
  cailloux.
\end{dialogue*}
% 
Cependant l'ensemble de ces informations ne sont pas directement
exploitables.

L'extrait \ref{mc:16} est, quant à lui, porteur de deux autres
informations. Tout d'abord on apprend que la victime a glissé dans la
zone où elle se situe, puis, que cette même zone est
caillouteuse.
%
Comme pour l'extrait \ref{mc:12} on peut modéliser ces
informations à l'aide de la relation de localisation
\onto[orl]{Dans\-Planimetrique}. Ce 


La victime revient ensuite sur son précédent trajet, et sur la manière
dont elle s'est perdue :
%
\begin{dialogue*}
  \Req \lnex{17}{mc:17} En partant du sommet du Moucherotte, donc j’ai
  viré un peu à gauche et \lnex{18}{mc:18} après j'ai coupé pour
  \lnex{19}{mc:19} descendre, descendre, descendre, descendre et
  \lnex{20}{mc:20} j’ai pas retrouvé mon sentier et \lnex{21}{mc:21}
  je suis certainement descendue beaucoup trop bas et là
  \lnex{22}{mc:22} j’ai plus dutout la force de remonter, je suis trop
  descendue.
\end{dialogue*}
% 
Trucs

\begin{dialogue*}
  \Req \lnex{23}{mc:23} Mon but c’était de descendre, descendre,
  descendre vers les maison qui sont en bas, \lnex{24}{mc:24} mais je
  là j’ai une grosse barre rocheuse à droite, \lnex{25}{mc:25} j’ai
  plein de petits cailloux très glissants, \lnex{26}{mc:26} c’est très
  raide. \lnex{27}{mc:27} Je suis assise là, au milieu de nulle part.
\end{dialogue*}
% 
dsq

\begin{dialogue*}
  \Sec \lnex{28}{mc:27} Dans l’idée, vous êtes repartie en direction
  du tremplin ? On est d’accord ? Vous avez emprunté le même chemin ?
  % 
  \Req \lnex{29}{mc:29} Voila. Oui. \lnex{30}{mc:30} Mais, pas par la
  droite où \lnex{31}{mc:31} apparemment y’a deux sortes de chemins,
  pour le rejoindre. \lnex{32}{mc:32} Parce-qu’à un moment donné y’a
  une espèce de cabane qu’on voit quand on est au Moucherotte. C’est
  pas là. \lnex{33}{mc:33} Moi je suis allée plutôt sur la
  gauche. J’ai fait quoi, 200 mètres à gauche et \lnex{34}{mc:34} puis
  j’ai coupé à la perpendiculaire en redescendant.
  % 
  \Sec \lnex{35}{mc:35} Et vous avez jamais croisé un chemin ?
  % 
  \Req \lnex{36}{mc:36} J’étais arrivée par là et j’ai pas retrouvé de
  chemin.
\end{dialogue*}
% 
Ces extraits

\begin{dialogue*}
  \Sec \lnex{37}{mc:37} Vous êtes donc plutôt rentrée dans la forêt là ?
  % 
  \Req \lnex{38}{mc:38} Là je suis un peu à la
  sortie. \lnex{39}{mc:39} J’ai fait un peu d’escalade là,
  \lnex{40}{mc:40} parce que c’est très rocheux. \lnex{41}{mc:41} Je
  suis dans une partie où il y a plein d’arbres tous secs et pas trop
  d’arbres. Je suis complétement à découvert
\end{dialogue*}
% 
Ces extraits \ref{mc:32}


\begin{dialogue*}
  \Req \lnex{42}{mc:42} Et juste à ma droite y’a une grosse, grosse
  barre rocheuse. \lnex{43}{mc:43} Ouais et devant moi j’ai la plaine
  et \lnex{44}{mc:44} à ma gauche j’ai quand même des sapins.
\end{dialogue*}
% 
Ces extraits


\begin{dialogue*}
  \Sec \lnex{45}{mc:45} Et donc, dans la plaine qu’est-ce que vous
  voyez ?
  % 
  \Req \lnex{46}{mc:46} Eh ben je vois un peu d’eau.
  % 
  \Sec \lnex{47}{mc:47} Vous voyez des habitations ? Vous voyez une
  route ?
  % 
  \Req \lnex{48}{mc:48} Un petit peu, mais les habitations c’est très
  parses. \lnex{49}{mc:49} Et juste devant moi y’a un groupe de 4, 5
  maisons ou des fermes.
  % 
  \Sec \lnex{50}{mc:50} C’est-à-dire ? À combien de mètres ?
  % 
  \Req \lnex{51}{mc:51} Ah non, mais c’est très, très, très loin
  ça. Je les vois mais c’est pas accessible.
\end{dialogue*}
% 
Voila


\begin{dialogue*}
  \Req \lnex{52}{mc:52} Devant moi y’a des tas de barres rocheuses en
  fait, je vois pas comment je peux arriver jusque-là et
  \lnex{53}{mc:53} en dessous des barres rocheuses ça repart en
  forêt. \lnex{54}{mc:54} Je sais pas si je peux continuer à
  descendre, c’est vachement raide et glissant.
\end{dialogue*}
% 
Pour finir

\subsection{Modélisation de l'alerte}
\label{subsec:9-3-2}


\subsection{Critique de la modélisation}
\label{subsec:9-3-3}



%%% Local Variables:
%%% mode: latex
%%% TeX-master: "../../../../main"
%%% End:
