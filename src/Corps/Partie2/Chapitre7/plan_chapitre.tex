\chaptertoc{}

\addsec{Introduction}

\tdi{Rappeler la définition de la spatialisation}

\tdi{La spatialisation en plusieurs étapes}

\section{Les étapes de la \emph{spatialisation}}

\tdi{Importance de la zir}

\tdi{Importance du raster de base}


\tdi{méthode de spatialisation = rasterisation + calcul métrique +
  sélection floue}

\begin{figure}
  \centering
  \includegraphics[width=\textwidth]{../../../../tmp/methodo_spatialisation.jpeg}
  \caption{Méthode de \emph{spatialisation}}
  \label{fig:methodo_spatialisation}
\end{figure}



\section{La rasterisation}

\tdi{Détailler les différentes méthodes de rasterisation définies
  figure \ref{fig:methode_rasterisation} + exemple}


\subsection{Les méthode de rasterisation}

\begin{figure}
  \centering
  \includegraphics[width=\textwidth]{../../../../tmp/rasterisation.jpeg}
  \caption{Méthodes de rasterisation}
  \label{fig:methode_rasterisation}
\end{figure}

\section{Le calcul des métriques}

\tdi{Détailler les métriques définies
  figure \ref{fig:methode_rasterisation} + exemple}

\subsection{Métriques globale}

\tdi{Présentation des métriques qui ne dépendent pas de l'objet de
  référence -> elles sont calculables une fois. Ex : Pente, altitude
  absolue (mais déjà calculée vu que MNT).}

\tdi{Est-ce qu'on peut vraiment faire cette distinction ??}

\subsection{Métriques locales}

\tdi{Présentation des métriques locales, qui changent pour chaque
  \emph{objet de référence}, distance, orientation, visibilité}

\subsection{Métriques Paramétriques}

\subsection{Métriques non paramétriques}

\begin{table}
  \centering
  \includegraphics[width=\textwidth]{../../../../tmp/metriques.jpeg}
  \caption{Types de métriques}
  \label{fig:type_metriques}
\end{table}


\section{La sélection floue}

\tdi{Présentation des différents sélecteurs}

\begin{figure}
  \centering
  \includegraphics[width=\textwidth]{../../../../tmp/selecteur.jpeg}
  \caption{Méthodes de sélection}
  \label{fig:methode_selecteur}
\end{figure}

\subsection{La définition des fonctions d'appartenance}

\subsection{Le problème de la définition des seuils}

\tdi{Expliquer en quoi la définition des seuils est un problème}



\section{Vers la prise en compte de modifieurs}


\tdi{Parler du fait que le modifieur utilisé dépend de ce qui est
  considéré. Not s'applique à tout}

\tdi{Différence paramètre/modifieur}

\begin{figure}
  \centering
  \includegraphics[width=\textwidth]{../../../../tmp/dec.jpeg}
  \caption{Exemple de la XXX de la \emph{relation de localisation
      atomique} \protect\onto[orla]{Sous\-Al\-ti\-tu\-de}.}
  \label{fig:methode_selecteur}
\end{figure}


\addsec{Conclusion}

Conclusion chapitre 7
%%% Local Variables:
%%% mode: latex
%%% TeX-master: "../../../main"
%%% End:
