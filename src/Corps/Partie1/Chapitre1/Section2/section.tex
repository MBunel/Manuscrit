\subsection{Origine du projet}
\label{subsec:1-2-1}

L’alerte présentée ci-dessus a mis en évidence les limites de la
méthode de localisation des victimes utilisée par le \ac{pghm} et le
risque qu'elles font courir aux victimes. Dans le cas où la solution
\emph{Gend'Loc} n'est pas utilisable, les secouristes doivent opter
pour une localisation manuelle qui, si elle fonctionne la plupart du
temps, peu être inefficace et conduire a la perte d'un temps précieux
---~comme c'est le cas ici. Une réflexion a donc été amorcée au sein
du \ac{pghm} du Grenoble pour trouver des solutions permettant
d'améliorer la gestion de ces cas, qui bien que minoritaires, sont
problématiques. Une des voies suivie a été d'organiser des
collaborations avec des laboratoires de recherche pour mettre en place
une réflexion autour de ces questions.

Une de ces collaborations a abouti à création du projet de recherche
pluridisciplinaire Choucas \footnote{\url{http://choucas.ign.fr/}}, du
sein du quel notre travail de thèse s'inscrit. Ce projet de recherche
implique quatre institutions partenaires :
%
\begin{enumerate*}[label=(\arabic*)]
\item Le \emph{laboratoire en sciences et technologies de
    l'information géographique} (LASTIG)
  \footnote{\url{https://www.umr-lastig.fr/}} de l'institut national
  de l'information géographique et forestière (IGN), qui regroupe des
  chercheurs spécialisés dans la modélisation, la qualification,
  l'enrichissement et la visualisation des données géographiques;
\item l'équipe \emph{probabilités et statistiques} du
  \emph{laboratoire de mathématiques et de leurs applications}
  \footnote{\url{https://lma-umr5142.univ-pau.fr/}} (LMAP), dont les
  membres sont spécialisés dans la recherche autour des modèles
  probabilistes et de l'inférence statistique;
\item L'équipe STeamer \footnote{\url{http://steamer.imag.fr/}} du
  \emph{laboratoire d'informatique de Grenoble} (LIG), travaillant à
  la conception d'outils et de méthodes destinés à la construction de
  systèmes d'information spatio-temporelles; et
\item du \ac{pghm} de Grenoble qui est à la fois un contributeur et
  l'utilisateur final des résultats du projet Choucas
  \autocite{Choucas2020}.
\end{enumerate*}

\subsection{Problématique du projet}
\label{subsec:1-2-2}

L'objectif premier du projet Choucas et de développer des méthodes et
des outils permettant aux secouristes de localiser plus facilement,
rapidement et efficacement des victimes en montagne et ainsi éviter de
reproduire la situation de l'alerte présentée ci-dessus, où la victime
n'a été retrouvée que par chance. De part son statut de
\enquote{déclencheur} du projet de recherche Choucas, mais surtout en
raisons de ses spécificités qui en font un cas d'école
(\autoref{subsec:1-1-2-3}), cette alerte aucune une place centrale
dans le projet Choucas, au point d'être baptisée \enquote{\emph{fil
    rouge}} \autocite{OlteanuRaimond2017}.

À la suite de l'alerte \emph{fil rouge} un processus de réflexion,
visant à identifier les lacunes du processus de localisation des
victimes, a été mis en place. Ce dernier a permis d'identifier
plusieurs problèmes. Le premier d'entre eux est qu'il est difficile,
voir impossible, de traiter une partie des informations données par le
requérant. Les raisons peuvent être multiples. Par exemple le
requérant peut donner une information qui pourrait se révéler
discriminante, mais qu'il est très difficile d'exploiter
manuellement. C'est par exemple le cas lorsque, dans le \emph{fil
  rouge,} la victime indique \enquote{voir une partie de plan
  d'eau}. Si relativement peu de positions correspondent à cet énoncé,
il n'est pas exploitable lors d'une localisation manuelle, puisqu'il
serait nécessaire d'arriver à construire manuellement des zones de
visibilité. Un autre problème se pose lorsque le requérant décrit sa
position à l'aide d'un objet non nommé, mais défini par son type ou sa
nature. C'est par exemple le cas lorsque la victime indique être
\enquote{sous une ligne électrique} ou être \enquote{partie \textelp{}
  en direction d'une station de ski}. Dans ce cas de très nombreuses
positions peuvent correspondre à une même description et il est
difficile pour un opérateur humain de toutes les identifier et les
vérifier. Un autre problème est que les requérants peuvent décrire une
position en utilisant des termes très vagues, par exemple lorsque,
dans le \emph{fil rouge,} la victime indique avoir marché
\enquote{plusieurs heures}. Les secouristes ne peuvent qu'approximer
la distance qu'a pu parcourir la victime, s'agit-il de quelques
kilomètres ? Auquel cas la victime aurait marché deux ou trois heures,
en fonction du terrain ou beaucoup plus ? Bien entendu ces différents
problèmes peuvent se combiner, c'est par exemple le cas lorsque la
victime indique être \enquote{sous une route et entend des
  voitures}. Dans ce cas le \enquote{sous} est imprécis (\ie on ne
sait pas s'il s'agit de quelques mètres ou de beaucoup plus), l'objet
est inconnu (\ie il peut s'agir de n'importe quelle route) et il est
impossible de traiter une partie de l'information manuellement (\ie
\enquote{entendre des véhicules}).

Un autre point problématique, bien que moins critique que le
précédent, est qu'il est difficile pour les secouristes de traiter
différentes hypothèses de localisation de la victime en parallèle,
surtout lorsque ces dernières sont complexes. Les secouristes ont donc
tendance à se focaliser sur une même hypothèse, celle qui semble la
plus probable. Toutefois cela peut conduire à se focaliser sur une
hypothèse fausse, d’antan plus que, comme l'indique
\textcite{Bachelard1934} : \enquote{À l'usage, les idées se valorisent
  indûment}. Il serait donc préférable que les secouristes puissent
disposer d'un outil leur permettant de traiter plus facilement et
confortablement plusieurs hypothèses en parallèle.

Une solution à cet ensemble de problèmes \footnote{C'est-à-dire :%
  \begin{enumerate*}[label=(\arabic*)]
  \item Impossibilité d'identifier les positions correspondant à
    certaines descriptions,
  \item difficulté à considérer l'ensemble des cas possibles lorsque
    le requérant se réfère par rapport à un objet sans le nommer,
  \item difficulté à prendre en compte l'imprécision et
  \item difficulté de traiter des hypothèses en parallèle.
  \end{enumerate*}} peut être apportée par le développement de
méthodes et d'outils informatiques \emph{ad hoc,} destinés à assister
le secouriste procédant à la localisation. Les différents objets
correspondant à la description donnée par le requérant peuvent être
identifiés automatiquement à l'aide d'un outil d’interrogation des
données. Les informations inexploitables manuellement (\eg les notions
de visibilité) peuvent alors être à l'aide d'un outil destiné à
transformer une description de position (\eg \enquote{la victime voit
  une partie de plant d'eau}) en une zone, dont les coordonnées sont
connues et permettant de prendre en compte l'imprécision des
descriptions et les situations où plusieurs objets peuvent
correspondre à la description. Enfin, ces deux outils peuvent être
combinés dans un outil d'aide à la décision permettant de définir
plusieurs hypothèses et de les traiter indépendamment. Le
développement d'une telle solution pourrait apporter un réel support
aux secouristes durant la phase de localisation. Cependant, il nous
semble nécessaire que cette solution ne soit pas conçue pour se
substituer aux secouristes (\eg en interprétant directement la
discussion téléphonique et en cherchant à modéliser la description
donnée par le requérant), mais qu'elle s'intègre dans leur cadre de
travail.

Une seconde catégorie de problèmes, liés aux premiers, concerne
spécifiquement l'exploitation des données. Pour localiser les
requérants les secouristes ont à leur disposition de nombreuses
données, comme des bases de données géographiques, des cartes et des
plans, des \emph{topo-guides,} \emph{etc.} Une grande partie de ces
données n'existent que sous format papier, ce qui rend leur
utilisation difficile, notamment lorsque les secouristes cherchent une
information sans connaitre le document dans lequel elle se trouve. Les
secouristes sont donc confrontés à un autre problème : l'interrogation
des données. Une solution pour faciliter l'exploitation de ces données
consiste à travailler à leur intégration et à proposer des outils
d'interrogation des données. Toutefois, même en considérant toutes les
données à la disposition des secouristes, quel que soit leur format,
il reste des descriptions qu'il est impossible d'interpréter… faute de
données. Ce peut être le cas lorsqu'il se réfère à un objet qui n'est
pas saisi dans les bases de données géographiques, soit parce que
l'objet n'entre pas dans les spécifications de la base de donnée
utilisée, soit car il est trop récent, ou autre. Il est également
possible qu'un objet existe dans la base de données géographique
utilisée, mais que le requérant s'y réfère d'une manière telle que
l'objet n'est pas identifiable ; par exemple en utilisant certaines
caractéristiques qui ne sont pas présentes dans la base de
données. Prenons pour exemple le cas où une victime indique être sur
un \enquote{pentu}, \enquote{en lacets} ou encore \enquote{en
  balcon}. Il est fort probable que le chemin auquel elle se réfère
soit présent dans la base de données géographique utilisée, mais il ne
sera pas qualifié de \enquote{pentu}, \enquote{en lacets} ou
\enquote{en balcon}, il n'est donc pas possible d'identifier ce chemin
à l'aide de ces qualificatifs, à moins d'être en mesure de les
construire. Une solution consiste à \emph{enrichir} les bases de
données géographiques en y ajoutant toutes les informations jugées
nécessaires, comme les qualificatifs \enquote{pentu} ou \enquote{en
  lacet} s'il s'agit d'un chemin.

\subsection{Objectifs scientifiques du projet}
\label{subsec:1-2-3}

Pour répondre à ces problèmes, quatre objectifs principaux ont été
définis et répartis entre les partenaires du projet Choucas :%
%
\begin{enumerate*}[label=(\arabic*)]
\item la structuration des données issues de sources textuelles
  hétérogènes, principalement traitée au sein du laboratoire LMAP ;
\item le raisonnement spatial qualitatif flou, réalisé par les membres
  du LASTIG ;
\item la géovisualisation de données multidimensionnelles et
  imparfaites pour la prise de décision, traitée par le LIG et
\item l'intégration de sources hétérogènes spatialisables, impliquant
  des membres du LMAP et du LASTIG.
\end{enumerate*}
%
Une synthèse des objectifs du projet Choucas est présentée par la
\autoref{tab:synthese_objectifs_choucas}, à la fin du chapitre.

\subsubsection{Structuration des données issues de sources textuelles
  hétérogènes}
\label{subsec:1-2-3-1}

Comme nous l’expliquions précédemment les bases de données
géographiques ne contiennent pas l'ensemble des objets pouvant être
utilisés pour décrire une position en montagne. Si on y trouve la
plupart des sentiers et des sommets, il est moins fréquent de trouver
des représentations de \emph{barres rocheuses,} ou de \emph{vires,}
alors qu'il s'agit de points de repères saillants et donc
régulièrement utilisés pour décrire des positions dans notre
contexte. L'ajout de ce type d'objet dans la base de données utilisées
par les secours permettrait d’affiner ou de faciliter la localisation
des victimes. Le premier objectif scientifique du projet : \enquote{la
  structuration des données issues de sources textuelles hétérogènes};
consiste à enrichir les bases de données à la disposition du \ac{pghm}
par l'analyse de sources textuelles et plus spécifiquement de
descriptions textuelles d'itinéraires de randonnée. En effet, la
majorité des informations disponibles sur les itinéraires de randonnée
est présentée (et diffusée) sous forme de texte, notamment par le
biais de \emph{topoguides,} de sites collaboratifs \footnote{Comme le
  site web \emph{camp2camp.}}, ou encore de blogs. Les itinéraires y
sont généralement présentés sous la forme d'un texte décrivant
principalement les points de bifurcation et les objets permettant de
les repérer (\eg \enquote{prendre à droite à la bifurcation située au
  niveau d'un gros rocher}) et son parfois (notamment sur les sites
tels que \emph{camp2camp}), complétés par une trace GPS. La présence
de ces deux informations complémentaires permet d'envisager
d'identifier les positions correspondant à certaines descriptions, et
inversement. Or, comme certains des objets décrits et utilisés comme
points de repère dans le récit peuvent être absents des bases de
données géographiques, il devient possible de les enrichir
\autocite{Moncla2019,Medad2018}.

\subsubsection{Raisonnement spatial qualitatif flou}
\label{subsec:1-2-3-2}

Le second objectif scientifique du projet Choucas est l’élaboration de
méthodes permettant d'identifier les zones correspondant à des
descriptions de positions (\eg \enquote{la victime est sous un
  chemin}), dans le but de faciliter l'interprétation de descriptions
difficiles, voire impossibles à traiter manuellement (\eg \enquote{la
  victime voir une partie de plant d'eau}) ou nécessitant d'étudier de
nombreux cas, notamment lorsque l'objet utilisé comme point de
référence n'est pas nommé et donc directement identifiable (\eg
\enquote{la victime est sous une route}). Le principal verrou
scientifique de cet objectif et qu'il impose d'être en mesure de
prendre en compte l'imprécision inhérente a ce type de
description. Comme nous l'avons déjà énoncé il peut être difficile
d'interpréter une description telle que : \enquote{la victime est sous
  une route et entend des véhicules}, sans savoir a quel point la
victime est \enquote{sous} la route (s'agit-il de quelques mètres ou
de beaucoup plus) ou si le son fait par les véhicules est fort ou
faible. À cela s'ajoute le fait qu'une même description peut être
utilisée dans des contextes différents ou pour décrire des positions
qui le sont tout autant. Il est donc nécessaire de prendre cet aspect
en considération et d'adapter la méthodologie pour qu'elle puisse
prendre en compte l'imprécision. Un second verrou est que les
descriptions données au sein d'une même alerte peuvent être très
différentes. Par exemple dans le \emph{fil rouge} la victime se repère
en utilisant un indice visuel, mais aussi en décrivant son trajet. La
modélisation d'une telle alerte nécessite donc d'être en mesure de
traiter des informations de nature très différente, tout en prenant en
considération leur imprécision et la possibilité que l'objet utilisé
pour décrite la position ne soit pas nommé (\eg \enquote{une station
  de ski}, \enquote{une route}, \emph{etc.}). C'est dans cet axe que
s'inscrit notre travail doctoral et nous détaillerons donc cet
objectif dans les chapitres suivants.

\subsubsection{Géovisualisation de données multidimensionnelles et
  imparfaites pour la prise de décision}
\label{subsec:1-2-3-3}

Un autre point abordé par ce projet de recherche est la question de la
visualisation des données géographiques et du développement
d'interfaces homme-machine. Pour utiliser les outils d'aide au
raisonnement proposés dans note thèse (et présentés ci-dessus), le
secouriste doit disposer d'une interface adaptée, permettant de
décrire la localisation de la victime et de visualiser les résultats
de l'outil d'aide à la localisation. 

Cette interface de géovisualisation est également destinée à afficher
les résultats produits par l'outil d'aide à la localisation. Cet outil
devant prendre en compte la nature imprécise des descriptions en
langage naturel, il est nécessaire que l'interface puisse représenter
ces résultats sans omettre leur nature imprécise.  

Également confrontée à la question de l'imprécision du langage
naturel, le développement de cette interface nécessite un
questionnement autour de la représentation de données de cette nature
qui seront produites par l'outil. Cet objectif nécessite une réflexion
autour de la question de la représentation de données imprécises,
comme celles qui seront produites lors de notre travail de thèse.

\autocite{Viry2019a}

\subsubsection{Intégration de sources hétérogènes spatialisables}
\label{subsec:1-2-3-4}

Enfin le dernier objectif du projet Choucas consiste à travailler à la
structuration de l'ensemble des données dont disposent les
secouristes. Comme nous l'expliquions lors de notre description du
processus de localisation utilisé par les secouristes
(\ref{subsec:1-1-2-1}), ces derniers disposent d'un important corpus
de données. Mais ces dernières sont de types divers et ne sont pas
organisées. Il est, par conséquent difficile de trouver une
information utilise au processus de localisation sans savoir où
chercher. La structuration de ces données offrirait la possibilité de
les interroger conjointement et ainsi de disposer d'un maximum
d'informations, ce qui améliora le processus de localisation.

Le principal verrou scientifique de cette tâche est la prise en compte
de la contradiction entre sources. En effet, deux sources différentes
peuvent placer un même sommet à des endroits différents, il est donc
nécessaire d'être capable de traiter ces cas.

\autocite{VanDamme2019,Halilali2018}

% Tableau synthétique
\begin{landscape}
\begin{table}[H]
  \centering
  \begin{tabular}{p{.15\textheight}>{\small}L{.35\textheight}>{\small}L{.4\textheight}}
  \toprule
  \multicolumn{1}{c}{\normalsize\bfseries Objectif scientifique} &
  \multicolumn{1}{c}{\normalsize\bfseries Verrous scientifiques} &
  \multicolumn{1}{c}{\normalsize\bfseries Apports envisagés} \\ \midrule
  % Lot 1
  \nameref{subsec:1-2-3-1}
& \begin{minipage}[t]{.35\textheight}
    \begin{itemize}
    \item Annotation, catégorisation et désambiguïsation des entités nommées
    \item Extraction et interprétation des relations sémantiques
    \end{itemize}
  \end{minipage} & \begin{minipage}[t]{.4\textheight}
    \begin{itemize}
    \item Définition d'une méthode d'annotation des entités nommées
    \item Définition d'une méthode permettant l'interprétation des
      relations sémantiques
    \end{itemize}
  \end{minipage} \\
  \addlinespace[.5cm]
  %
  \nameref{subsec:1-2-3-2}
  {\par\footnotesize\hspace{.25cm}$\longrightarrow$~Chapitre \ref{chap:02}}
& \begin{minipage}[t]{.35\textheight}
    \begin{itemize}
    \item Spatialisation des indices de localisation
    \item Modélisation des objets géographiques imprécis
    \end{itemize}
  \end{minipage}& \begin{minipage}[t]{.4\textheight}
    \begin{itemize}
    \item Définition d'une méthode de spatialisation
    \item Définition d'une méthode de prise en compte de l'imprécision
      des objets géographiques
    \end{itemize}
  \end{minipage} \\
  \addlinespace[.5cm]
  % Lot 
  \nameref{subsec:1-2-3-3}
& \begin{minipage}[t]{.35\textheight}
    \begin{itemize}
    \item Définir des méthodes permettant d'assister le secouriste
      dans sa prise de décision
    \item Visualisation de données imprécises
    \end{itemize}
  \end{minipage}& \begin{minipage}[t]{.4\textheight}
    \begin{itemize}
    \item Construction d'une interface d'aide à la décision
    \item Définition de méthodes permettant de prendre en compte
      l'imprécision lors de la visualisation
    \end{itemize}
  \end{minipage} \\
  \addlinespace[.5cm]
  %
  \nameref{subsec:1-2-3-4}
& \begin{minipage}[t]{.35\textheight}
    \begin{itemize}
    \item Structuration d'un référentiel de données métier
    \item Appareillement de données hétérogènes
    \end{itemize}
  \end{minipage}& \begin{minipage}[t]{.4\textheight}
    \begin{itemize}
    \item Définition de méthodes d'appareillement
    \item Construction d'un référentiel de données métier
    \end{itemize}
  \end{minipage}\\
  \bottomrule
\end{tabular}

  \caption{Synthèse des verrous et des apports attendus pour chaque
    objectif scientifique du projet Choucas}
  \label{tab:synthese_objectifs_choucas}
\end{table}
\end{landscape}

%%% Local Variables:
%%% mode: latex
%%% TeX-master: "../../../../main"
%%% End:
