Une fois que la structure globale de l'ontologie des règles de
décomposition a été fixée, il est nécessaire de définir son contenu en
détail (\ie les \emph{relations de localisation,} les \emph{relations
  de décomposition} et les \emph{relations de localisation
  atomiques}). Pour ce faire, on peut se fonder sur les différents
travaux formalisant des \emph{relations de localisation}
\autocite{Vandeloise1986,Aurnague1997,Borillo1998}, pour identifier
des \emph{relations de localisation} et en proposer une
décomposition. Cependant, bien que cette approche soit pertinente,
elle ne nous semble pas être pleinement adaptée à notre cas
d’application. En effet ces travaux traitent des \emph{relations de
  localisation} dans leur acception générale, alors que nous
travaillons sur un contexte particulier, avec un relief et des
\emph{objets de référence spécifiques} (\eg montagne, col,
\emph{etc.}). On peut donc augurer que le contexte influence
l’utilisation des \emph{relations de localisation} et donc impacte
leur sémantique, comme l'illustre l'aphorisme de \bsc{Wittgenstein} :
\enquote{la signification c'est l'usage}. Or nous souhaitons
développer une ontologie orientée métier et donc prenant en compte les
spécificités de localisation en milieu montagneux. Il semble alors
essentiel d'étudier les descriptions de positions dans le contexte du
secours en montagne afin de construire une ontologie la plus
pertinente possible.

Pour ce faire nous avons entrepris de recenser et d'étudier les
\emph{relations de localisation} les plus utilisées lors des
alertes. Les \ac{pghm} disposant d'un enregistreur d'appels, nous
pouvons obtenir des enregistrements de discussions où les requérants
décrivent leur position. En analysant le contenu de ces
enregistrements il est alors possible d'identifier les \emph{relations
  de localisation} les plus utilisées, de savoir si elles le sont dans
un contexte particulier ou avec une signification plus spécifique que
leur sens général.

\subsection{La retranscription et l'analyse des alertes}

L'exploitation des enregistrements audio d'alertes nécessite au
préalable leur retranscription textuelle. Nous avons donc développé
une méthode et un template (ou canevas) de retranscription a cet
effet. Le travail de définition du template et de retranscription a
été effectué collaborativement \autocite{Bunel2019}, les résultats
pouvant être exploités par différents membres du projet Choucas et
notamment pour l'étude du processus d'alerte (Lot 4).

\subsubsection{Les corpus d'alertes}

Les retranscriptions ont été réalisées à partir d'un ensemble de 52
enregistrements d'appels de demande d'assistance auprès des \ac{pghm}
de Chamonix (5 alertes) et de Grenoble (47 alertes). Ces fichiers,
sélectionnés et compilés par le \ac{pghm} de Grenoble, forment deux
corpus d'alertes distincts, dont le premier nous a été transmis en
2017 et le second en 2019. Ces deux corpus sont relativement
similaires, que ce soit par le nombre d'alertes, leur durée ---~qui
varie de quelques secondes à plusieurs dizaines de minutes
(\autoref{fig:dist_temps_alertes})~--- ou leur
contenu. Malheureusement nous n'avons que peu, voir pas,
d'informations exogènes sur ces différentes alertes. Pour certaines
d'entre elles les secouristes nous ont transmis les coordonnées de la
victime ou la date et l'heure de l'intervention, mais dans la plupart
des cas ces informations nous sont inconnues, à moins qu'elles soient
explicitement données dans l'alerte (\eg énoncé oral de coordonnées
GPS, précision de l'heure, \emph{etc.}). De plus, nous ne savons pas
comment les alertes qui nous ont été transmises furent sélectionnées,
bien qu'elles nous aient été décrites comme représentatives des phases
de localisation des victimes.

Une autre particularité importante de ces enregistrements est qu'ils
sont réalisés au niveau de la ligne téléphonique des \ac{pghm} et non
des \ac{codis}, généralement contactés par les requérants
(cf. \autoref{chap:02}). Par conséquent nous n'avons généralement pas
connaissance du début ou de la fin de la communication. De plus, les
fichiers qui nous ont été transmis ne correspondent pas aux
enregistrements initiaux. Ces derniers ont été montés, initialement de
sorte à censurer les informations personnelles (\eg identité de la
victime, description des blessures), mais dans les faits ils ne
continent généralement que la phase de description de position.

\begin{figure}
  \centering
   \begin{tikzpicture}
   \begin{axis}
     \addplot [boxplot] table [y index=0] {./Corps/Partie2/Chapitre5/figures/duration.dat};
     
  %\addplot [thick] gnuplot [raw gnuplot] {plot './Corps/Partie2/Chapitre5/figures/duration.dat' smooth kdensity};
\end{axis}
\end{tikzpicture}

  \caption{Distribution de la durée des alertes}
  \label{fig:dist_temps_alertes}
\end{figure}

\subsubsection{Template de retranscription}

Notre parti-pris n'a pas été de travailler à partir d'une
retranscription libre des alertes, mais à l'aide d'un tableau
prédéfini, un \emph{template de retranscription}
(\autoref{tab:struct_temp}) dont la structure reprend celle des
\emph{indices de localisation} telle que nous l'avons définie
(cf. \autoref{chap:04}). Ce choix présente deux avantages. Tout
d'abord il permet de faciliter l'analyse des alertes en imposant, dès
la saisie, la définition de certains critères. Pour faciliter les
saisies et leur répartition entre transcripteurs, un fichier est créé
pour chaque alerte.

Dans sa forme finale, ce template de transcription se présente sous la
forme d'un fichier tabulaire structuré, auto-documenté et exemplifié,
composé de trois onglets principaux
%
---~\begin{enumerate*}[label=(\alph*)]
\item \emph{métadonnées},
\item \emph{interprétation des expressions} et
\item \emph{toponymie}
\end{enumerate*}~---,
%
permettant de transcrire différents aspects des alertes. L'ensemble du
fichier est conçu pour faciliter le processus de retranscription et
limiter au maximum l'influence du transcripteur sur la saisie. Chaque
onglet et champ à saisir est documenté, des instructions de saisie
présentent l'ensemble des tâches à effectuer sont données et un
exemple de saisie complet et complexe est détaillé. De plus, la
structure même du document est conçue pour faciliter la saisie. Les
onglets sont placés dans leur ordre de consultation ou de saisie (\eg
on commence par lire les instructions dans l'onglet
%
\begin{enumerate*}[label=(\arabic*)]
\item \emph{Instructions \& Consignes,} puis on saisit les
  informations générales sur l'alerte dans l'onglet
\item \emph{métadonnées,} avant de saisir l'alerte en détail dans les
  onglets
\item \emph{interprétation des expressions} et
\item \emph{toponymie,} \emph{etc.}
\end{enumerate*}).

L'onglet \emph{métadonnées,} permet de renseigner toutes les
informations d'ordre général sur une alerte, telles que l'identité du
requérant (victime, témoin, \emph{etc.}) ou l’activité pratiquée lors
de l'accident. C'est également ici que sont saisies toutes les
informations exogènes, comme la date et l'heure de l'appel ou la
position réelle de la victime.

L'onglet \emph{interprétation
  des expressions} se compose, quant à lui, de 15 colonnes
(\autoref{tab:struct_temp}), que l'on peut regrouper en trois
catégories :
%
\begin{enumerate*}[label=(\alph*)]
\item \emph{extrait},
\item \emph{contexte} et
\item \emph{expression.}
\end{enumerate*}

La première d'entre elles regroupe les colonnes décrivant l'extrait
audio traité, comme les deux colonnes \enquote{\emph{identifiant}}, la
colonne \enquote{\emph{extrait}} et la colonne
\enquote{\emph{timestamp}}. Les colonnes \enquote{\emph{identifiant}}
permettent de donner une référence unique à chaque élément
transcrit. La présence de deux colonnes permet de faire la distinction
entre les \emph{extraits,} qui correspondent à découpage rythmique de
la conversation (changement de locuteur, nouvelle phrase,
\emph{etc.}), et les \emph{expressions,} contenues dans les
\emph{extraits,} correspondant au découpage en \emph{indices de
  localisation.} Par exemple, si la conversation entre le secouriste
(\bsc{s.}) et le requérant (\bsc{r.}) est :
%
\begin{quote}
  \begin{dialogue}
    \Sec Vous êtes au sommet ?
    \Req Non, je suis bien en dessous.
  \end{dialogue}
\end{quote}
%
On identifiera deux \emph{extraits :}
%
\begin{enumerate*}[label=(\alph*)]
\item \enquote{Vous êtes au sommet ?} et
\item \enquote{Non, je suis bien en dessous},
\end{enumerate*}
%
délimités par le changement de locuteur. Si le premier ne contient
qu'une seule \emph{expression,} le second en contient deux, la
négation de la question, dont la sémantique correspond à
\emph{l'expression} \enquote{je ne suis pas au sommet} et
\emph{l'expression} \enquote{je suis bien en dessous}. Ainsi, cet
exemple est composé de deux \emph{extraits} et de trois
\emph{expressions.} Pour faciliter l'analyse et la vérification des
transcriptions, nous avons défini une colonne
\enquote{\emph{extrait}}, permettant de saisir le verbatim de
l'extrait analysé, et une colonne \enquote{\emph{timestamp}} indiquant
la position de l'extrait dans le fichier audio correspondant.

La seconde catégorie regroupe les colonnes permettant de décrire le
contexte de chaque expression. On y retrouve la colonne
\enquote{\emph{locuteur}}, renseignant sur la personne ayant prononcé
l'extrait étudié (\eg secouriste, requérant) et la colonne
\enquote{\emph{confiance}}, permettant au transcripteur (dans le cas
où il s'agit d'un secouriste) de juger de la plausibilité de
l'information donnée par l'extrait. Si nous distinguons ces deux
colonnes des \emph{métadonnées} de l'extrait (\ie le groupe de
colonnes précédent) c'est car ces dernières demandent une
interprétation de l'extrait, contrairement aux \emph{métadonnées.}

Enfin, la troisième catégorie, qui regroupe la majorité des colonnes,
permet l'interprétation et la saisie de chaque expression. On y
retrouve les colonnes \emph{verbe,} \emph{sujet,} \emph{relation de
  localisation} et \emph{objet de référence,} correspondant aux
éléments du même nom dans la formalisation des \emph{indices de
  localisation} et les \emph{modifieurs} qui y sont associés. Les
colonnes \enquote{\emph{objet de référence}} et
\enquote{\emph{modifieur de l'objet de référence}} ont la
particularité de pourvoir être doublées (voir plus si nécessaire) dans
le cas où la \emph{relation de localisation} le nécessiterait.

% Exemple
Si l'on reprend le second extrait de l'exemple précédent (\ie
\enquote{Non, je suis bien en dessous}), deux extraits sont à
saisir. Le premier est la négation de la question du secouriste (\ie
\enquote{Vous êtes au sommet}), que l'on peut interpréter comme
\emph{l'indice de localisation} \enquote{je ne suis pas au
  sommet}. Ici le verbe de l'expression est \enquote{être}, le sujet
est le requérant, la \emph{relation de localisation} est la
préposition \enquote{à} et l'objet de référence est
---~implicitement~--- \enquote{le sommet} \footnote{Le \enquote{à le}
  est contracté en \enquote{au} dans l'extrait}. La forme négative de
la phrase est transcrite avec un \emph{modifieur du verbe} et les
autres modifieurs ne sont pas renseignés. Pour la seconde expression
(\ie \enquote{je suis bien en dessous}) le \emph{sujet} et sont
modifieur et \emph{l'objet de référence} et sont modifieur sont
inchangés, il s'agit toujours du \emph{requérant} qui se situe par
rapport au \emph{sommet.} Le \emph{verbe} est toujours \enquote{être},
mais son modifieur disparait, l'expression n'étant pas une
négation. Pour finir la \emph{relation de localisation} devient
\enquote{\emph{en dessous}} et son modifieur est
\enquote{\emph{bien}}.

Enfin, le dernier onglet du template de transcription est destiné à la
saisie détaillée des \emph{objets de référence utilisés.} Comme pour
l'onglet des \emph{extraits,} chaque ligne correspond à une
\emph{expression} issue de l'alerte. Par ailleurs toutes les lignes de
ce troisième onglet doivent être saisies, il y a donc une bijection
entre les enregistrements de l'onglet \enquote{expressions} et ceux de
l'onglet \enquote{objets de référence}. Le but de cet onglet est de
permettre la saisie d'informations complémentaires sur les
\emph{objets géographiques} utilisés dans les expressions, comme les
\emph{objets de référence} ou le \emph{sujet,} notamment dans le cas
où l'expression décrit la position d'un autre objet qu'une personne
(\eg \enquote{le chalet à côté de la cascade}). En effet, lorsqu'on
saisit un \emph{objet géographique} (\eg \emph{l'objet de référence})
dans l'onglet \emph{expression} on utilise le terme employé par le
locuteur dans l'expression (\eg \enquote{La Bérarde}, \enquote{un
  sommet}, \emph{etc.}) sans modification. Ainsi les objets définis
sont désignés par leur nom (ou tout du moins le nom que le locuteur
leur donne) et les objets indéfinis par leur type. L'onglet
\emph{toponymie} permet alors de préciser le type de tous les objets
géographiques, en vue de leur analyse ultérieure. Si l'on reprend
l'exemple précédent (\ie \enquote{Non, je suis bien en dessous}) nous
ne sommes pas en mesure d'ajouter des informations sur \emph{l'objet
  de référence} \enquote{sommet}, son nom étant inconnu.

\begin{table}
  \centering
  \begin{tabular}{L{.3\textwidth}>{\footnotesize}p{.4\textwidth}}
  \toprule \multicolumn{1}{c}{\bfseries Colonne} &
  \multicolumn{1}{c}{\normalsize \bfseries Contenu} \\ \midrule
% Sémantique des relations spatiales
  \addlinespace
  Identifiant de l'extrait & Permet d'identifier une expression\\
  Identifiant de l'expression & Permet d'identifier une expression au
                                sein d'un extrait en comprenant plusieurs\\
  Extrait & Verbatim de la phrase transcrite\\
  Confiance & Permet au transcripteur de saisir sa confiance en
              \emph{l'indice de localisation} (uniquement si la saisie
              est faite par un secouriste)\\
  Timestamp & Début de l'extrait dans le fichier audio source\\
  Locuteur & Permet d'identifier le locuteur (\eg secouriste,
             requérant, témoin, \emph{etc.})\\
  Verbe & Verbe utilisé dans \emph{l'indice de localisation}\\
  Modifieur du verbe & (\eg marcher \emph{vite})\\
  Sujet & Sujet de \emph{l'indice de localisation}\\
  Modifieur du sujet & (\eg marcher \emph{vite})\\
  \emph{Relation de localisation} & \emph{Relation de localisation}
                                    utilisée dans l'extrait\\
  Modifieur de la relation de localisation & (\eg \emph{très} loin)\\
  \emph{Objet de référence} & Nom ou type de \emph{l'objet de référence}
                              (cette colonne peut être multipliée si la
                              \emph{relation de localisation} est bi ou n-aire)\\
  Modifieur de \emph{l'objet de référence} & (cette colonne peut être multipliée si la
                                             \emph{relation de localisation} est bi ou n-aire)\\
  Commentaires & Champ permettant au transcripteur de commenter sa
                 saisie ou les indications données par le requérant
                 (erreurs potentielles, fautes de prononciation, \emph{etc.}) \\
  \bottomrule
\end{tabular}

  \caption{Structure de l'onglet \enquote{\emph{expressions}} du
    template de retranscription.}
  \label{tab:struct_temp}
\end{table}

\subsubsection{Analyse des retranscriptions}

Les deux corpus d'alertes ont été transcrits séparément. Le premier
d'entre-eux a été transcrit au cours de l'année 2018 par un secouriste
du \ac{pghm} de Grenoble. Ce premier essai a mis en évidence de
nombreuses lacunes dans le processus de retranscription. La version
originale du template n'était pas assez documentée et ne contenait pas
d'exemples. Par conséquent le processus de saisie était assez délicat
et les résultats étaient très dépendants du transcripteur. Nous avons
donc clarifié le template pour faciliter la transcription
---~aboutissant à la version présentée ci-dessus~--- et avons corrigé
collégialement les transcriptions pour aboutir à une version qui fasse
l'objet d'un consensus. Compte tenu des problèmes que nous avions
rencontrés lors de la saisie du premier corpus nous avons adopté une
démarche de transcription plus avancée. La trentaine d'alertes à
transcrire a été répartie entre trois personnes
\autocite{Bunel2019}. Après une première transcription, les fichiers
produits ont été amendés et commentés par un second transcripteur, en
vue d'aboutir, en collaboration avec le transcripteur initial, à une
seconde version. Une fois l'ensemble du second corpus retranscrit nous
avons procédé a une harmonisation collégiale des saisies, de manière à
obtenir un ensemble de retranscriptions le plus cohérent possible.

Au final, la retranscription des 52 alertes nous a permis d'identifier
374 expressions différentes, que nous avons ensuite analysé.

% configuration vs position
L'étude détaillée des \emph{relations de localisation} utilisées dans
les \emph{indices} nous a permis de constater que tous les indices ne
permettent pas de construire \emph{directement} une \emph{zone de
  localisation compatible.} Par exemple, \emph{l'indice de
  localisation :} \enquote{Je marche vers le nord} ne peut pas être
\emph{spatialisé,} à moins de la combiner avec d'autres \emph{indices}
(\eg \enquote{Je marche depuis une heure, vers le nord}) ou de les
approximer avec d'autres relations de localisation (\eg \enquote{Je
  suis en direction du nord}). Ces \emph{indices de localisation,} qui
dénotent une \emph{configuration spatiale,} décrivent généralement une
trajectoire, suivie par le \emph{sujet} et partant de \emph{l'objet de
  référence.} Par conséquent cette distinction recoupe celle qui est
généralement faire en linguistique entre les \emph{relations de
  localisation statiques,} décrivant une position fixe et les
\emph{relations de localisation dynamiques,} décrivant un mouvement
\autocite{Borillo1998}. Comme l'ontologie que nous développons est
avant destinée à la \emph{spatialisation} des \emph{indices de
  localisation} nous avons pris la décision de ne pas y intégrer les
\emph{relations de localisation} décrivant une \emph{configuration
  spatiale,} la \emph{spatialisation} de trajectoires ne faisant pas
partie de nos objectifs.

\subsection{Construction des ontologies}

\subsubsection{Cadre des ontologies}

Il a été choisi de développer deux ontologies complémentaires pour
formaliser les \emph{relations de localisation.} La première d'entre
elles est \emph{l'ontologie des relations de localisation}
\acp{orl}. Son objectif est de constituer un thésaurus des
\emph{relations de localisation} utilisées dans le contexte de la
description d'une position en montagne. Chacune des notions est
définie et une grande partie d'entre elles sont illustrées par des
extraits des retranscriptions. La seconde ontologie définie est celle
des \emph{relations de localisation atomiques} \acp{orla}, dont
l'objectif est de définir les \emph{relations de localisation
  atomiques,} les décompositions dont elles résultent et le processus
de \emph{spatialisation} \footnote{Ce point spécifique sera détaillé
  dans le \autoref{chap:06}.} Ces deux composantes, fortement liées,
auraient pu être combinées dans une même ontologie. Cependant, alors
que la décomposition est fortement liée à notre travail, le thésaurus
des relations de localisation peut être exploité pour d'autres travaux
au sein du projet Choucas ou à l'extérieur. C'est pourquoi nous avons
décidé de séparer les deux ontologies et de ne diffuser que
l'ontologie des \emph{relations de localisation.}

\begin{table}
  \centering
  \begin{tabular}{r>{\small}p{.35\textwidth}>{\small}p{.35\textwidth}}
  \toprule & \multicolumn{1}{c}{\ac{orl}} &
  \multicolumn{1}{c}{\ac{orla}} \\ \midrule
  \addlinespace
  Objectif & Recense et définit les \emph{relations de localisation} utilisées
  pour décrire une position dans le contexte de la localisation de
  personnes en montagne & Définit les \emph{relations de localisation
                          atomiques,} la décomposition des relations
  définies dans \ac{orl} et formalise le processus de
                          \emph{spatialisation} des \emph{relations
                          spatiales atomiques.}\\
  Contenu & Définition de 51 \emph{relations de localisation,}
            regroupées en 11 classes abstraites. & Définition de XX
                                                   \emph{relations de
                                                   localisation
                                                   atomiques,}
                                                   décomposant XX des
                                                   51 \emph{relations
                                                   de localisation}
                                                   définies dans \ac{orl}.\\
  Hiérarchie & Blo & \\
  XX & L'ensemble des concepts sont définis pour être facilement
       différentiables et proches de la perception humaine des
       localisations dans l'espace & Les \emph{relations de
                                     localisation atomiques} ne sont
                                     pas conçues pour être manipulées
                                     directement par les utilisateurs.
  Elles généralement plus abstraites que les \emph{relations de
                                     localisation} qu'elles
                                     décomposent.\\ 
  \bottomrule
\end{tabular}

  \caption{Éléments de comparaison des ontologies \ac{orl} et
    \ac{orla}.}
  \label{tab:orl_vs_orla}
\end{table}

\subsubsection{L'ontologie des \emph{relations de localisation}}

La construction de l'ontologie des relations de localisation s'est
faite en deux étapes. Dans un premier temps nous avons défini les
concepts à partir des \emph{relations de localisation} utilisées dans
les alertes, puis nous avons hiérarchisé ces concepts.

% Défintion des concepts
Pour chaque expression nous avons cherché à définir un concept
permettant d'exprimer la sémantique de la \emph{relation de
  localisation} utilisée.
%
Pour ce faire nous avons cherché à identifier, pour chaque expression,
le concept de l'ontologie de \textcite{Bateman2010} le plus proche de
la sémantique de la relation de localisation utilisée par le
requérant.
%
À la suite de cette étape toutes les \emph{relations de localisations}
sont associées à un concept tiré de l'ontologie de
\textcite{Bateman2010}. Cependant, dans la majorité des cas, le
concept issu de GUM-Space ne nous a pas donné entière satisfaction,
soit parce-que le concept retenu ne fonctionnait qu'au pris d'une
interprétation très \enquote{flexible} de la définition de
\textcite{Bateman2010}, soit parce-que la description donnée par le
requérant n'était que grossièrement retranscrite par les concepts de
GUM-Space.

On peut prendre pour exemple la \emph{préposition spatiale}
\enquote{entre} utilisée dans l'extrait suivant :
%
\begin{quote}
  \begin{dialogue}
    \Sec Vous êtes entre Grand Veymont et Pas de la Ville ?  \Req Je
    suis entre le Grand Veymont, sous le Grand Veymont et Pas de la
    Ville, tout à fait. Je suis côté… heu… côté sud.
  \end{dialogue}
\end{quote}
%
L'ontologie GUM-Space ne propose pas de concepts permettant une
retranscription satisfaisante de la sémantique de cette \emph{relation
  spatiale.} On peut faire appel aux concepts
\onto[gum]{Dis\-tri\-bu\-tion} ou \onto[gum]{Surround\-ing}, mais le
premier s'applique qu'au cas où \emph{l'objet de référence} est une
collection (\eg \enquote{entre les arbres}, \enquote{parmi la foule}),
et le second correspond plus à la \emph{relation de localisation}
\enquote{entouré de}, qui, bien que proche, n'est pas une
approximation satisfaisante de la \emph{relation de localisation}
utilisée par le requérant. De la même manière, dans l'extrait :
%
\begin{quote}
  \begin{dialogue}
    \Req On est à 10 minutes du sommet de la Bastille.
  \end{dialogue}
\end{quote}
%
Le requérant décrit sa position à l'aide d'une durée de marche à
partir de \emph{l'objet de référence,} \ie à l'aide d'une
\emph{distance-temps quantitative.} Cependant, l'ontologie GUM-Space
ne possède qu'un seul concept permettant d'exprimer une distance
quantitative, \onto[gum]{Quanti\-tative\-Dist\-ance}. Au vu de la
description qui est faite de ce concept dans \textcite{Bateman2010} on
peut légitimement penser que ce concept a essentiellement été prévu
pour conceptualiser des distances métriques, uniquement des distances
métriques. On pourrait se contenter de cette situation, considérant
que ces distances sont équivalentes. Cependant, comme nous
l'expliquions dans l'état de l'art (\autoref{chap:03}) les
distances-temps sont beacoup plus XXX que les distances métriques et
leur \emph{spatialisation} implique dès lors des méthodes plus
avancées. Par conséquent il nous semble nécessaire de développer un
nouveau concept, \onto{A\-Dist\-ance\-Temps}, plus précis et adapté.

Au final, 51 \emph{relations de localisation} ont été définies dont
seules 8 sont des concepts originaux de GUM-Space. Les 43 concepts
restants sont soit des précisions de concepts existants (\eg
\onto{A\-Dist\-ance\-Temps}) soit des concepts totalement nouveaux
(\eg \onto{En\-tre\-X\-\&\-Y}).

\tdi{- Structure orienté utilisateur (concepts fortement liés à
  l'acceptation des utilisateurs)}

% Hiérarchisation des concepts
Une fois que l'ensemble des \emph{relations de localisation}
nécessaires a la retranscription des alertes a été défini nous avons
procédé à leur hiérarchisation. Pour ce faire nous avons défini de
nouvelles classes, permettant de regrouper les relations de
localisation par similarité sémantique. Ces nouvelles classes sont
abstraites, \ie qu'elles ne sont pas des \emph{relations de
  localisation,} elles ne servent qu'à leur regroupement.

Pour définir cette hiérarchie nous nous sommes appuyés sur différentes
propositions précédentes et principalement de l'ontologie GUM-Space
\autocite{Bateman2010}.

Au premier niveau de l'ontologie on trouve trois classes principales,
\onto[orl]{Re\-la\-tion\-Spa\-ti\-ale\-De\-Prox\-imi\-té},
\onto[orl]{Re\-la\-tion\-Spa\-ti\-ale\-Fonc\-tio\-nelle} et
\onto[orl]{Re\-la\-tion\-Spa\-ti\-ale\-Po\-si\-ti\-on\-Re\-la\-ti\-ve},
regroupant l'ensemble des \emph{relations de localisation} et deux
\emph{relations de localisation} isolées et complémentaires,
\onto[orl]{Ci\-ble\-Voit\-Si\-te} et \onto[orl]{Si\-te\-Voit\-Ci\-ble}
qui n'entrent dans aucune des ces classes. Les trois classes
principales sont similaires aux trois classes du premier niveau
hiérarchique de l'ontologie GUM-Space. La classe
\onto[orl]{Re\-la\-tion\-Spa\-ti\-ale\-De\-Prox\-imi\-té}, qui
correspond à la classe
\onto[gum]{Spa\-ti\-al\-Dist\-an\-ce\-Mo\-da\-li\-ty} de GUM-Space,
regroupe les \emph{relations de localisation} dont la sémantique
traduit une proximité spatiale. On y retrouve, par exemple, la
relation \onto[orl]{Près\-De} ou la relation
\onto[orl]{A\-Dist\-ance\-Temps}, qui modélise une distance exprimée
en temps de déplacement. La seconde classe,
\onto[orl]{Re\-la\-tion\-Spa\-ti\-ale\-Fonc\-tio\-nelle} est le
pendant de la classe
\onto[gum]{Func\-ti\-on\-al\-Spa\-ti\-al\-Mo\-da\-li\-ty} de
l'ontologie GUM-Space. Elle regroupe les \emph{relations spatiales}
qui décrivent une situation où le sujet et \emph{l'objet de référence}
ont un lien, non seulement spatial, mais aussi fonctionnel. C'est, par
exemple, le cas de la \emph{relation de localisation}
\onto[orl]{Prox\-imal} qui implique une proximité entre le
\emph{sujet} et \emph{l'objet de référence}, mais qui, contrairement à
la relation \onto[orl]{Près\-De}, peut être fonctionnelle et pas
seulement spatiale. C'est également le cas de la \emph{relation de
  localisation}
\onto[orl]{Si\-tué\-Sur\-Iti\-né\-rai\-re\-Ou\-Ré\-seau\-Sup\-port},
la seule autre \emph{relation fonctionnelle} que nous avons défini.
Enfin, la classe
\onto[orl]{Re\-la\-tion\-Spa\-ti\-ale\-Po\-si\-ti\-on\-Re\-la\-ti\-ve},
similaire à la classe
\onto[gum]{Re\-la\-ti\-ve\-Spa\-ti\-al\-Mo\-da\-li\-ty}, regroupe
toutes les \emph{relations} qui décrivent la position du \emph{sujet}
en fonction de celle du \emph{site} mais sans notion de
distance. C'est cette classe qui regroupe le plus de \emph{relations
  de localisation.} On y retrouve, par exemple, les relations
traduisant un contact\footnote{Correspondant à la classe
  \onto[orl]{Re\-la\-tion\-Spa\-ti\-ale\-De\-Con\-tact}.} (\eg
\onto[orl]{Dans\-Pla\-ni\-mé\-tri\-que},
\onto[orl]{A\-La\-Fr\-on\-ti\-ère\-De}), les relations de direction
\footnote{Correspondant à la classe
  \onto[orl]{Re\-la\-tion\-Spa\-ti\-ale\-De\-Di\-rec\-tion}.} (\eg
\onto[orl]{Au\-Nord\-De}, \onto[orl]{Dans\-La\-Dir\-ec\-tion\-De}) ou
les relations verticales \footnote{Correspondant à la classe
  \onto[orl]{Re\-la\-tion\-Spa\-ti\-ale\-De\-Ver\-ti\-ca\-le}.} (\eg
\onto[orl]{So\-us\-Pro\-che\-De}, \onto[orl]{Au\-Des\-sus\-De}). Au
total, l'ontologie des \emph{relations de localisation} contient 11
classes servant à définition de la hiérarchie. Ces classes, qui ne
correspondant à aucune \emph{relation de localisation,} ne sont ni
décomposables, ni spatialisables. C'est pourquoi nous les qualifions
de \emph{classes abstraites} et nous les signalons comme telles dans
l'ontologie, pour éviter qu'elles soient manipulées comme des
\emph{relations de localisation} 

On peut remarquer que certaines des \emph{relations de localisation}
que nous avons présentées pourraient appartenir à plusieurs
\emph{classes abstraites}. C'est par exemple le cas de la
\emph{relation} \onto[orl]{Prox\-imal}, que nous avons décrit comme
appartenant à la \emph{classe abstraite}
\onto[orl]{Re\-la\-tion\-Spa\-ti\-ale\-Fonc\-tio\-nelle}, mais qui
traduit également une notion de proximité spatiale. La \emph{relation}
\onto[orl]{Prox\-imal} pourrait donc également appartenir à la
\emph{classe abstraite}
\onto[orl]{Re\-la\-tion\-Spa\-ti\-ale\-De\-Prox\-imi\-té}. De fait les
regroupements que nous avons définis n'ont pas été conçus pour être
mutuellement disjoints \footnote{C'est également le cas de la
  hiérarchie proposée dans GUM-Space \autocite{Bateman2010}.}, ainsi
une même \emph{relation} peut, comme \onto[orl]{Prox\-imal},
appartenir à plusieurs classes. Ce découpage ne pose cependant pas de
problèmes particuliers, quelle que soit le cas d'utilisation de
l'ontologie. Nous pensons même qu'il peut être utile pour certaines
applications, comme la recherche de \emph{relations de localisation}
par les secouristes utilisant l'interface.

Les concepts définis dans \ac{orl} étant fortement inspirés, voir
repris, de l'ontologie GUM-Space, nous avons complété l'ontologie en
liant, pour chaque concept le concept équivalent (ou similaire) de
l'ontologie GUM-Space. Plus spécifiquement deux types de liens sont
définis, la relation \emph{seeAlso,} utilisée 24 fois et la relation
\emph{equivalentClass,} utilisée 8 fois. Cette dernière traduit une
équivalence parfaite entre concepts, alors que la relation
\emph{seeAlso} est avant tout informative et traduit une certaine
proximité sémantique entre deux concepts sans qu'ils soient
équivalents ou même similaires. Ainsi, seules 8 des 62 \emph{relations
  de localisation} définies dans \ac{orl} sont identiques à celles
définies par \textcite{Bateman2010}, les autres concepts sont définis
de manière significativement différente.

\begin{figure}
  \centering
  \dict{\onto{ADistanceQuantitativeRelativeDe}}{La \emph{cible} est située
  à une certaine distance du \emph{site.} Cette distance est
  quantitative et est exprimée par une métrique dépendant du
  \emph{site} (\eg \enquote{À un pâté de maisons}). Il n’y a pas de
  contrainte sur la dimension de la distance, elle peut être
  \emph{planimétrique,} \emph{altimétrique} ou les
  deux.}{\enquote{\textins{Il} est à une longueur en dessous
    \textins{du dos d'âne}, donc à une petite longueur du haut du
    couloir de la Meije}}%

\dict{\onto{ADistanceTempsDe}}{La \emph{cible} est à une certaine
  distance du \emph{site}. Cette distance est exprimée en temps de
  parcours. Il est par conséquent nécessaire de connaître le mode de
  déplacement ou la vitesse de déplacement.}{}%

\dict{\onto{ADroiteDe}}{La \emph{cible} est à droite du \emph{site.}
  Ce concept nécessite de connaître l’orientation de la
  \emph{cible.}}{\enquote{\textelp{} sur la droite quand on regarde
    vers le haut de la station.}}%

\dict{\onto{ALEstDe}}{La \emph{cible} se situe globalement à l’est du
  \emph{site,} sans préciser si elle est dans la partie est du
  \emph{site} ou disjointe et à l'est du \emph{site.}}{}%

\dict{\onto{ALEstDeExterne}}{La \emph{cible} est disjointe
  ($\Rightarrow$ \onto{Hors\-De\-Pla\-ni\-me\-tri\-que}) et se situe
  globalement à l’est du \emph{site} ($\Rightarrow$
  \onto{A\-Est\-De}).}{}%

\dict{\onto{A\-L\-Extremite\-De}}{Le \emph{site} est de forme allongée
  (il a alors deux extrémités), ou possède une ou plusieurs parties
  saillantes allongées (il peut alors avoir plusieurs extrémités). La
  \emph{cible} est située au niveau ($\Rightarrow$ \onto{Proche\-De})
  l'une de ces extrémités. Elle peut être située à l'intérieur du
  \emph{site,} ou non.}{\enquote{Il est à l'extrémité ouest d'une
    espèce de terrasse en béton.}}%

\dict{\onto{AGaucheDe}}{La \emph{cible} est à gauche du \emph{site.}
  Ce concept nécessite de connaître l’orientation de la
  \emph{cible.}}{\enquote{Juste sur la gauche, il y a une autre dent.
    Je ne sais pas comment elle s’appelle.}}%

\dict{\onto{A\-La\-Frontiere\-De}}{Le \emph{site} a une emprise
  spatiale linéaire ou surfacique. La \emph{cible} est
  \enquote{proche} de ses frontières. La \emph{cible} peut
  indifféremment se situer à l’intérieur ($\Rightarrow$
  \onto{Dans\-Pla\-ni\-mé\-tri\-que}) ou à l'extérieur du \emph{site}
  ($\Rightarrow$ \onto{Hors\-De\-Pla\-ni\-mé\-tri\-que}). Si elle est
  à l’intérieur, elle est alors dans le complémentaire de la zone de
  l’espace qui vérifie la relation \onto{Au\-Milieu\-De}. La distance
  à considérer dépend très probablement de la nature ou de la taille
  du \emph{site.}}{\enquote{Je suis un peu à la sortie de la forêt.}}%

\dict{\onto{ALaMemeAltitudeQue}}{La \emph{cible} est située à la même
  altitude que le \emph{site.} Le \emph{site} peut être une altitude
  absolue ou un objet dont l’altitude sert de
  référence.}{\enquote{C’est à 2300 mètres d’altitude.}}%

\dict{\onto{ALOuestDe}}{La \emph{cible} se situe globalement à l’ouest
  du \emph{site,} sans préciser si elle en est à l'intérieur ou à
  l'extérieur.}{\enquote{\textelp{} et là on est à l'Ouest.}}%

\dict{\onto{ALOuestDeExterne}}{La \emph{cible} se situe globalement à
  l’ouest ($\Rightarrow$ \onto{AOuestDe}) et est disjointe du site
  ($\Rightarrow$ \onto{Hors\-De\-Pla\-ni\-me\-tri\-que}).}{\enquote{On
    est versant ouest, côté Vercors intérieur.}}%

\dict{\onto{ApresJalonSurItineraire}}{La \emph{cible} est située sur
  un itinéraire ($\Rightarrow$
  \onto{Si\-tue\-Sur\-Iti\-ne\-rai\-re\-Ou\-Reseau\-Support}), et
  après le \emph{site,} qui est un jalon de cet itinéraire. Cela
  suppose d’avoir défini un référentiel des directions associé à
  l’itinéraire, généralement lié au sens de progression (l’avant est
  vers la destination, l’arrière vers l’origine).}{\enquote{Après le
    parking de la Villette.}}%

\dict{\onto{ATempsDeMarche}}{La \emph{cible} est à tel temps de marche
  du \emph{site,} en marchant depuis le \emph{site} vers la
  \emph{cible.}}{\enquote{On est à 10 minutes du sommet de la
    Bastille.}}%

\dict{\onto{AuDessusALAplombDe}}{}{}%

\dict{\onto{AuDessusAltitude}}{La \emph{cible} a une altitude
  supérieure au \emph{site.} La distance entre le \emph{site} est la
  \emph{cible} n’est pas contraignante.}{\enquote{Au-dessus de
    Percolin, y’a la cabane du berger de Bellefont et on est 50 mètres
    au-dessus.}}%

\dict{\onto{AuDessusProche}}{La \emph{cible} est proche et a une
  altitude supérieure au site.}{\enquote{Juste au-dessus de Bernin.}}%

\dict{\onto{AuDessusJalonSurItineraire}}{La \emph{cible} est au-dessus
  du \emph{site} et sur un chemin menant au site (implique la relation
  \onto{Situe\-Sur\-Itineraire\-Ou\-Re\-seau\-Support}).}{\enquote{\textelp{}
    Sur le sentier qui mène à la cascade de l’Oursière, haut dessus de
    la cascade de l’Oursière.}}%

\dict{\onto{AuMilieuDe}}{}{}%

\dict{\onto{AuNordDe}}{La \emph{cible} se situe globalement au nord du
  \emph{site,} sans préciser si elle en est à l'intérieur ou à
  l'extérieur.}{}%

\dict{\onto{AuNordDeExterne}}{La \emph{cible} se situe globalement au
  nord et est disjointe du\emph{ site.}}{\enquote{\textelp{} non, côté
    Nord (du Pas de la Ville), pardon.}}%

\dict{\onto{AuSudDe}}{La \emph{cible} se situe globalement au sud du
  site, sans préciser si elle en est à l'intérieur ou à
  l'extérieur.}{}%

\dict{\onto{AuSudDeExterne}}{La \emph{cible} se situe globalement au
  sud et est disjointe du \emph{site.}}{\enquote{Je suis entre le
    grand Veymont et Pas de la Ville, tout à fait, coté sud.}}%

\dict{\onto{AuxAlentoursDe}}{La \emph{cible} est suffisamment proche
  du site \emph{pour} qu’on puisse considérer que le \emph{site} est
  un point de repère qui a un sens. La proximité qui pourrait s’en
  déduire est généralement dépendante d’un \enquote{rayonnement} qui
  pourrait êtrre affecté au \emph{site} (lié à sa renommée, à sa
  taille typiquemnet pour un lieu habité, à sa saillance,
  \emph{etc.}). Il peut y avoir connexion topologique entre la
  \emph{cible} et le \emph{site,} ou non.}{\enquote{Dans le massif de
    la Chartreuse}}%

\dict{\onto{AvantJalonSurItineraire}}{La \emph{cible} est située sur
  un itinéraire (implique la relation
  \onto{Situe\-Sur\-Itineraire\-Ou\-Reseau\-Support}) et avant le
  \emph{site} qui en est un jalon. Cela suppose d’avoir défini un
  référentiel des directions associé à l’itinéraire, généralement lié
  au sens de progression sur l’itinéraire.}{}%

\dict{\onto{AvoirASaDroite}}{La \emph{cible} est munie d’un
  référentiel des directions intrinsèque : elle a une gauche et une
  droite (et un avant, un arrière). Dans ce référentiel des
  directions, et plus précisément sur l’axe gauche-droite de ce
  référentiel, le \emph{site} est situé à droite de la \emph{cible}
  (donc la \emph{cible} a le site à sa \emph{droite,} au sens commun
  de l’expression).}{}%

\dict{\onto{AvoirASaGauche}}{La cible est munie d’un référentiel
  des directions intrinsèque : elle a une gauche et une droite (et un
  avant, un arrière), par exemple parce que c’est un être animé muni
  d’un visage. Dans ce référentiel des directions, et plus précisément
  sur l’axe gauche-droite de ce référentiel, le site est situé à
  gauche de la cible (donc la cible a le site à sa gauche, au sens
  commun de l’expression). Le référentiel des directions doit être
  modélisé explicitement (à préciser : comment)...}{}%

\dict{\onto{AvoirDerriereSoit}}{La cible est munie d’un
  référentiel des directions intrinsèque : elle a un avant et un
  arrière, (et une gauche, une droite), par exemple parce que c’est un
  être animé muni d’un visage. Dans ce référentiel des directions, et
  plus précisément sur l’axe avant-arrière de ce référentiel, le site
  est situé derrière la cible (donc la cible a le site derrière elle,
  au sens commun de l’expression). Le référentiel des directions doit
  être modélisé explicitement (à préciser : comment)...}{}%

\dict{\onto{AvoirDevantSoit}}{La \emph{cible} est munie d’un
  référentiel des directions intrinsèque : elle a un avant et un
  arrière, (et une gauche, une droite), par exemple parce que c’est un
  être animé muni d’un visage. Dans ce référentiel des directions, et
  plus précisément sur l’axe avant-arrière de ce référentiel, le
  \emph{site} est situé devant la \emph{cible} (donc la \emph{cible} a
  le \emph{site} devant elle, au sens commun de
  l’expression).}{\enquote{Devant moi j’ai la plaine, et à gauche j’ai
    quand même des sapins.}}%

\dict{\onto{CibleVoitSite}}{Le \emph{site} est visible depuis la
  \emph{cible}.}{\enquote{Je suis vraiment en montagne, je vois les
    plaines, je vois un grand découvert devant moi.}}%

\dict{\onto{DansLaDirectionDe}}{}{}%

\dict{\onto{DansLaPartieBasseDe}}{La \emph{cible} est située dans
  la partie basse du \emph{site}.}{}%

\dict{\onto{DansLaPartieDeXLaPlusProcheDeY}}{Le \emph{site} 1
  ($X$) est un surfacique (en 2,5D) le \emph{site} 2 ($Y$) est situé
  en dehors du \emph{site} 1 et permet de délimiter une portion du
  \emph{site} 1 qui est \enquote{la partie la plus proche} du
  \emph{site} 2. La \emph{cible} est alors située dans le \emph{site}
  1 (\onto{Dans\-Planimétrique}), dans cette \enquote{partie la
    plus proche} du \emph{site} 2.}{\enquote{Dans la Combe de la
    Glière côté télésiège.}}%

\dict{\onto{DansLaPartieEstDe}}{La \emph{cible} se situe dans le
  \emph{site} et dans sa partie est.}{}%

\dict{\onto{DansLaPartieHauteDe}}{La \emph{cible} est située dans
  la partie haute du \emph{site}. Cela suppose de pouvoir définir une
  \enquote{partie haute} du \emph{site}.}{}%

\dict{\onto{DansLaPartieNordDe}}{La \emph{cible} se situe dans le
  \emph{site} et dans sa partie nord.}{}%

\dict{\onto{DansLaPartieOuestDe}}{La \emph{cible} se situe dans
  le \emph{site} et dans sa partie ouest.}{\enquote{Il est à
    l'extrémité ouest d'une espèce de terrasse en béton.}}%

\dict{\onto{DansLaPartieSudDe}}{La \emph{cible} se situe dans le
  \emph{site} et dans se partie sud.}{}%

\dict{\onto{Dans\-Planimétrique}}{}{\enquote{Non, on est en
    forêt.}}%

\dict{\onto{DeAutreCoteDeParRapportA}}{La \emph{cible} est de l’autre
  côté du \emph{site} 1 par rapport au \emph{site} 2. Autrement dit,
  la \emph{cible} et le \emph{site} 2 sont de part et d’autre du
  \emph{site} 1. C’est une situation opposée à
  \onto{Du\-Meme\-Cote\-Que\-Par\-Rapport\-A}, où la \emph{cible} et
  le \emph{site} 1 sont situés du même côté du \emph{site} 2. Comme
  pour la relation \onto{Du\-Meme\-Cote\-Que\-Par\-Rapport\-A}, cela
  suppose qu’on peut partitionner l’espace en deux zones appelées les
  deux \enquote{côtés} du \emph{site} 1. Selon la forme et la nature
  du \emph{site} 1 (et son contexte spatial) :
  %
  \begin{enumerate*}[label=(\arabic*)]
  \item soit le \emph{site} 1 a intrinsèquement deux côtés (un col,
    une crête, une rivière, une ville située sur une rivière,
    \emph{etc.}),
  \item soit, pour deux \emph{sites} assimilables à des ponctuels et
    sans contexte spatial particulier, c’est le couple (\emph{site} 1,
    \emph{site} 2) qui permet de définir deux côtés au \emph{site} 1 :
    un côté qui contient le \emph{site} 1, et un côté qui ne le
    contient pas.
  \end{enumerate*}
  Typiquement la limite est alors la perpendiculaire au segment
  joignant le \emph{site} 1 au \emph{site} 2, passant par le
  \emph{site} 1.}{\enquote{Les rochers de l’homme, vous descendiez de
    l’autre côté plutôt ?}}%

\dict{\onto{DistanceQuantitativePlanimetrique}}{La \emph{cible} est à
  telle distance du \emph{site}, exprimée dans une unité de longueur
  (et non un temps de marche ou d'accès). La distance considérée est
  planimétrique.}{\enquote{J'estime \textins{que je suis} à 800 m
    \textins{du Pas de la Ville}, je crois, à peu près, à vol
    d'oiseau}}%

\dict{\onto{DuMemeCoteQueParRapportA}}{La \emph{cible} située du même
  côté que le \emph{site} 1, par rapport au \emph{site} 2. Cela
  suppose qu’on peut partitionner l’espace en deux zones appelées les
  deux \enquote{côtés} du \emph{site} 2. Selon la forme et la nature
  du \emph{site} 2 (et son contexte spatial) :
  %
  \begin{enumerate*}[label=(\arabic*)]
  \item soit le \emph{site} 2 a intrinsèquement deux côtés (un col,
    une crète, une rivère, une ville située sur une rivière,
    \emph{etc.}),
  \item soit, pour deux \emph{sites} assimilables à des ponctuels et
    sans contexte spatial particulier, c’est le couple (\emph{site} 1, \emph{site}
    2) qui permet de définir deux côtés au \emph{site} 2 : un côté qui
    contient le \emph{site} 1, et un côté qui ne le contient pas.
  \end{enumerate*}
  Typiquement la limite est alors la perpendiculaire au segment
  joignant le \emph{site} 1 au \emph{site} 2 et passant par le
  \emph{site} 2. Cette relation est l’exact opposée de
  \onto{De\-L\-Autre\-Cote\-De\-Par\-Rap\-port\-A.}}{\enquote{C'est
    côté lac Robert.}}%

\dict{\onto{EntreXetY}}{Relation ternaire, nécessitant deux
  \emph{sites.} La \emph{cible} est située entre les deux
  \emph{sites}. Le cadre de référence est particulièrement important
  ici. Il faudra le modéliser pour pouvoir spatialiser cette
  relation. Par exemple, si les deux \emph{sites} peuvent être
  considérés comme de faible extension spatiale (donc assimilables à
  des ponctuels), le cadre de référence peut inclure un ou plusieurs
  tracés à une dimension qui relient les deux \emph{sites} : tracé
  d’un itinéraire, \emph{etc.} (ou une ligne droite en l’absence
  d’information). Si l’extension spatiale est plus importante, on peut
  définir une zone \enquote{entre les deux} comme dans le modèle 5IM
  (Clementini et Billen 2006). Mais attention aux cas déictiques comme
  \enquote{je vois Z entre X et Y} ou \enquote{nagez entre les deux
    poteaux} de \autocite{Bateman2010}.}{\enquote{Là on est dans le
    secteur entre \textelp{} Cordéac et Pellafol.}}%

\dict{\onto{HorsDePlanimetrique}}{}{}%

\dict{\onto{PresDe}}{}{\enquote{Il était près du sommet et dans le
    brouillard.}}%

\dict{\onto{Proximal}}{La \emph{cible} est dans le \emph{site} ou à sa
  proximité immédiate, qui peut être une proximité
  fonctionnelle.}{\enquote{Elle est à 50m, je suis sur un promontoire
    car le réseau ne passait pas.}}%

\dict{\onto{SiteVoitCible}}{Le \emph{site} voit la \emph{cible}
  (vision active). La position de la \emph{cible} est visible depuis
  une autre position connue (\eg un refuge depuis lequel un témoin
  contacte les secours).}{}%

\dict{\onto{Si\-tue\-Sur\-Iti\-ne\-rai\-re\-Ou\-Re\-seau\-Sup\-port}}{La
  \emph{cible} se situe sur un réseau ou un itinéraire. Le
  \enquote{sur} a ici un sens fonctionnel. Le \emph{site} peut être un
  élément de réseau au sens large : réseau parcourable à pied, réseau
  de pistes de ski, voies d’escalade ou d’alpinisme, réseau
  hydrographique (\eg en kayak ou canyoning), itinéraire de ski de
  randonnée, \emph{etc.} Aucun apriori n’est considéré sur la
  modélisation géométrique du réseau dans les données. Il n’y a pas
  nécessairement de connexion topologique entre la \emph{cible} et le
  \emph{site} : la \emph{cible} peut se trouver de fait légèrement
  éloignée de l’élément de réseau (\eg cas d’une aire de pique-nique
  accesible depuis le réseau de randonnée pédestre).}{\enquote{On est
    sur le GR 54.}}%

\dict{\onto{SousALAplombDe}}{La \emph{cible} est à une altitude
  supérieure à celle du \emph{site} et ce dernier est plus ou moins
  situé sur la ligne de plus grande pente qui passe par la
  \emph{cible,} ou inversement. Typiquement, le \emph{site} et la
  \emph{cible} appartiennent à une même vallée et sont situés à peu
  près au même niveau longitudinalement.}{}%

\dict{\onto{SousAltitude}}{La \emph{cible} a une altitude inférieure à
  celle du \emph{site.} Sa distance au site n’est pas
  contrainte.}{\enquote{Je pense pas qu'ils soient passés par les
    vires en-dessous.}}%

\dict{\onto{SousJalonSurItineraire}}{La \emph{cible} est située sur un
  chemin menant au \emph{site} et sous ce dernier}{\enquote{On est
    sous le déversoir du lac de Belledonne.}}%

\dict{\onto{SousProcheDe}}{La \emph{cible} est proche et a une
  altitude inférieure au \emph{site}.}{\enquote{On est à 100 m ou 150,
    ou 200 m du col, vraiment juste en-dessous.}}%

\dict{\onto{SousRecouvertPar}}{La \emph{cible} est sous le \emph{site}
  et elle est recouverte par ce dernier.}{\enquote{\textelp{} un mec
    \textins{est} coincé sous un arbre.}}%
%%% Local Variables:
%%% mode: latex
%%% TeX-master: "../../main"
%%% End:

  \caption{Onto}
  \label{fig:ontho}
\end{figure}

\subsubsection{L'ontologie des \emph{relations de localisation
    atomiques}}

Le processus de construction de \emph{l'ontologie des relations de
  localisation atomiques} \acp{orla} est assez différent de celui
suivit pour construire \emph{l'ontologie des relations de
  localisation.} Là où \ac{orl} est construite pour être manipulable
par un humain, \ac{orla} n'est destinée qu'a l'automatisation des
décompostions.

Ainsi s'il était nécessaire de converger vers une première version
d'\ac{orl} qui soit utilisable par les différents lot du projet
Choucas concernés, nous avons la maitrise totale d'\ac{orla}. Par
conséquent il n'est pas nécessaire de converger rapidement vers une
organisation stable.

Pour permettre la décomposition, \emph{l'ontologie des relations de
  localisations atomiques} ne nécessite que deux types d'éléments, un
ensemble de relations de localisation atomiques et un ensemble de
relations de décomposition, liant les \emph{relations de localisation}
définies dans \ac{orl} aux relations atomiques.


Pour la construire nous nous sommes basés sur

- décrire relations atomiques définies

DansLaDirectionDe
ADistanceTemps
Cardinalité (verifier)
même alt
Direction DE
Dans plani
Distance Quanti
Not a la frontiere de
a la frontiere de
altEq
AltSup
Contact
dedans
int
ext
sous altitude
sous proche de 

%%% Local Variables:
%%% mode: latex
%%% TeX-master: "../../../../main"
%%% End:
