Dans ce chapitre nous allons présenter l'approche retenue pour
modéliser les \emph{zones de localisation compatibles} (\ie qui
vérifient un indice de localisation donné, éventuellement après
décomposition) et \emph{probable} (qui vérifie tous les indices d'une
alerte donnée). Cette question est centrale dans notre travail de
recherche, car nous souhaitons prendre en compte \emph{l'imprécision}
des \emph{relations de localisation} et des \emph{zones de
  localisation} en résultant, ce qu'aucune modélisation courante des
objets géographiques ne permet.

% Modélisation = Modèle théorique + implémentation
Il existe cependant \footnote{Voir l'état de l'art
  (\autoref{chap:03}).} de nombreux modèles alternatifs, fondés sur
diverses théories et implémentés de différentes manières, permettant
de modéliser \emph{cette imprécision.} Le choix d'une
\emph{modélisation} des objets spatiaux \emph{imprécis} nécessite donc
d'identifier un modèle théorique et une de ses implémentations. Pour
ce faire, nous allons identifier les différentes caractéristiques que
nous voulons voir validées par notre \emph{modélisation} et permettant
d'appliquer tous les principes méthodologiques précédemment
définis. Nous sélectionnerons ensuite un \emph{modèle théorique} et
confronterons ses implémentations afin d'identifier la modélisation la
plus adaptée à notre cas d’application.

La \emph{modélisation} ici identifiée ne se destine qu'à représenter
les \emph{zones de localisation compatibles} et \emph{probables.}
Notre objectif n'est pas de représenter \emph{l'imprécision} des
\emph{objets de référence} utilisés en entrée de notre processus de
\emph{spatialisation.} Cette approche, notamment utilisée par
\textcite{Bloch1996} n'a pas été retenue pour notre travail, bien
qu'elle en soit une extension possible.

Nous commencerons donc par lister les différents critères de sélection
que nous avons retenus, puis nous sélectionnerons un modèle
théorique. Enfin, nous confronterons les implémentations du modèle
théoriques et sélectionnerons l'une d'entre-elle.

%%% Local Variables:
%%% mode: latex
%%% TeX-master: "../../../main"
%%% End:
