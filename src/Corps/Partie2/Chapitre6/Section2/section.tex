\subsection{Le choix du flou}

\tdi{Les modèles discrets (exacts, ensembles épais, autres ?)}

-Trop simples, pas de possibilité de modèliser des variations
importantes de degré d'appartenance. \textbf{Variation non linéaires
  (importantes avec le relief})

\missingfigure{comparaison décroissance degré d'appartenace en
  fonction de la pente. Voir si possible de le faire avec une arte
  topo et un vrai relief. Illustrer l'hypothèse implicite d'évolution
  linéaire entre les frontières dans les modèles exacts.}

\tdi{Modèles non linéaires et non flous}

\tdi{Modèle flous plus éprovés dans la littérature}

\tdi{Dire ques les modèles flous sont assez au point sur les opération
  de fusion avec les opérateurs flous qu'il définissent.}

%%% Local Variables:
%%% mode: latex
%%% TeX-master: "../../../../main"
%%% End:
