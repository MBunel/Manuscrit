La première partie de cette thèse de doctorat se destine à en
présenter les cadres applicatifs, scientifiques et théoriques.

Comme nous l'avons annoncé dans l'introduction générale le 

% 

% Annonce plan
Le premier chapitre sera dédié à la présentation du contexte
applicatif de la thèse. Nous présenterons la mission régalienne du
secours en montagne et son organisation, ainsi que les difficultés
auxquelles les \emph{unités de secours en montagne} \acp{usem} sont
confrontées. Nous aborderons en particulier la question de la
localisation des victimes en montagne, qui est au cœur de ce travail
de recherche. Nous présenterons ensuite, le projet de recherche dans
lequel notre travail s'inscrit.

Nous présenterons également le projet de recherche dans lequel nous
nous inscrivons, son origine et ses objectifs scientifiques.

Le second chapitre se destine à présenter les objectifs de notre
thèse. Comment elle s'organise, quels sont

Enfin, le troisième et dernier chapitre de cette partie dressera un
état de l'art sur deux aspects théoriques de cette thèse, à savoir la
modélisation de l'imprécision et la spatialisation des \emph{relations
  de localisation.}

%%% Local Variables:
%%% mode: latex
%%% TeX-master: "../../main"
%%% End:
