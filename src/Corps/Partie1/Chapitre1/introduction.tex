Comme nous l'expliquions précédement (cf. \nameref{part:int}) cette
thèse est à l'interface de plusieurs enjeux. L'un d'eux est
scientifique, bien évidement, et un second est plus appliquatif,
circonscrit à des problématiques métier, celles du secours en
montagne.

Or, bien que le secours en montagne soit un service public familier,
son histoire, son fonctionnement, ses acteurs et ses problématiques
sont mal connues.

Cette thése, et plus généralement le projet de recherche au sein
duquel elle s'inscrit ont étés construits pour répondre à une
problématique spécifique du secours en montagne actuel et par
conséquent il nous semble indispendable de commencer la présentation
de notre travail par une mise en contexte axée sur le secours en
montagne et ses problématiques.

Une meilleur compréhension de ces questions nous permettra de
présenter la méthodologie de traitement d'une alerte suivie par les
secouristes des \ac{pghm}, puis d'en montrer les limites.