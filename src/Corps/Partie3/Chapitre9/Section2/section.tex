La première des alertes que nous proposons de traiter est l'alerte
dite du \emph{Grand Veymont,} issue de l'enregistrement d'un appel au
secours effectué par un accompagnateur en montagne, au sujet d'un de
ses clients.
%
Cette alerte est la plus simple est la plus courte de celles que nous
traiterons ici.

\subsection{Présentation de l'alerte}
\label{subsec:9-2-1}

L'alerte du \emph{Grand Veymont} est l'extrait, d'une durée de 1
minute 06, d'un appel effectué par un accompagnateur en montagne au
sujet d'un de ses clients \autoref{anx:retrans-gv-verb}.


Cette alerte est la plus courte de celles que nous présenterons ici,
elle est composée de 12 extraits, pour un total de 16 expressions.



Pas d'objets multiples

les indications données par le requérant sont assez précises et
détaillées, il est donc possible de définir une \emph{zone initiale de
  recherche} de petite taille. Nous avons défini une \ac{zir} de
\SI{25}{\kilo\meter\squared} (\autoref{fig:zir_grand_veyont}).

\begin{figure}
  \centering
  \input{./figures/ZIR_grand_veymont.tex}
  \caption{Zone initiale de recherche pour l'alerte \enquote{Grand Veymont}}
  \label{fig:zir_grand_veyont}
\end{figure}


\subsubsection{Retranscription et identification des indices de localisation}
\label{subsec:9-2-1-1}


Le verbatim de l'enregistrement audio de l'alerte \emph{Grand Veymont}
est donné dans la \autoref{anx:retrans-gv-verb}.
%
Comme nous l’expliquions dans le \autoref{chap:05}, 


L'alerte commence par deux extraits (\nex{1} et \nex{2}) décrivant
le contexte de la descente :
%
\begin{dialogue}
  % 
  \Req \nex{1-1} J'ai eu du mal à descendre entre le sommet du grand
veymont et là où je suis. \nex{2-1} La descente était très très lente.
\end{dialogue}
%
Ces deux premiers extraits ne donnent pas
%
On peut cependant extrapoler certaines informations de ces deux
extraits. Par exemple, on peut légitimement considérer que le
requérant n'est plus sur le sommet du \emph{Grand Veymont,} ce que
l'on peut alors traduire par la relation de localisation
\onto[orl]{Hors\-De\-Planimétrique}, indiquant que la cible est à
l'extérieur du site. Dans le contexte de l'alerte, le site est le
sommet du \emph{Grand Veymont,} sont étendue est donc particulièrement
réduite. Par conséquent la \ac{zlc} de cet indice sera de grande
taille, ce qui rend cet indice de localisation peu discriminant.
%
Compte-tenu du contexte on peut également considérer que la position
de la cible est située à une altitude inférieure à celle de du sommet
du \emph{Grand Veymont}.
%
On peut donc ajouter un second indice de localisation, utilisant la
relation de localisation \onto[orl]{Sous\-Altitude}. L'ajout de ce
second indice de localisation est cependant plus contraignant que pour
l'indice de localisation précédent. En effet, là où la relation de
localisation \onto[orl]{Hors\-De\-Planimétrique} ne nous conduira pas
à intégrer des positions inattendues à la \ac{zlc} (\eg des positions
très éloignées du \emph{Grand Veymont}), il n'en va pas de même pour
la relation de localisation \onto[orl]{Sous\-Altitude}. En effet,
cette dernière n'impose pas de contraintes sur la distance au site,
toute position ayant une altitude inférieure au sommet du grand
Veymont




Ce second indice de localisation n'est donc pas plus discriminant que
le précédent.


Les premiers indices donnant une information réellement discriminante
sont ceux que l'on peut construire à partir des extraits \nex{3} et
\nex{4}, plus riches :
%
\begin{dialogue}
  %
  \Sec \nex{3-1} Vous êtes entre \emph{Grand Veymont} et \emph{Pas de
la Ville} ?

  \Req \nex{4-1} Je suis entre le \emph{Grand Veymont} et \emph{Pas de la
Ville,} tout à fait, \nex{4-2} coté sud.
\end{dialogue}
%
Les extraits \nex{3-1} et \nex{4-1}
%
Dans l'extrait \nex{4-2} le requérant complète cette information en
précisant sa position par une relation de localisation cardinale,
relative au site.
%
Cette description est cependant difficile à interpréter, d'une part
car le contexte ne permet pas de savoir si c'est le sommet du
\emph{Grand Veymont} ou le \emph{Pas de la Ville} qui fait office de
site, ce qui est relevé par le secouriste :
%
\begin{dialogue}
  %
  \Sec \nex{5-1} Côté Sud du \emph{Pas de la Ville} ?
%
  \Req \nex{6-1} Non, côté Nord.
\end{dialogue}
%
Ces deux extraits contredisent clairement l'expression \nex{4-2}

On peut déduire de ces extraits (\nex{4-2, 5-1, 6-1}) que le requérant
est situé au
%
On peut donc envisager d'utiliser la relation de localisation
\onto[orl]{Au\-Nord\-De} pour spatialiser cet indice de
localisation. Cependant, cette relation correspond au regroupement de
deux autres relations \onto[orl]{Dans\-La\-Partie\-Nord\-De} et
\onto[orl]{Au\-Nord\-De\-Externe}, dont l'usage peut être ici plus
pertinent.
%
La relation de localisation \onto[orl]{Dans\-La\-Partie\-Nord\-De}
décrit une situation où la cible est située à \emph{l'intérieur} du
site et dans sa partie nord.
%
Rien dans la situation décrite par le requérant n'indique qu'il serait
situé dans le site
%
La relation de localisation \onto[orl]{Au\-Nord\-De\-Externe} décrit
quant à elle une situation où la cible est au nord du site, sans être
ni contact, ni au sein de celui-ci.
%
Dans le contexte de cette alerte, l'utilisation de cette relation
indiquerait que le requérantxxx





\begin{dialogue}
  % 
  \Sec \nex{7-1} Vous êtes au-delà du \emph{Pas de la Ville} ?
  \nex{7-2} Entre \emph{Pas de la ville} et \emph{Pierre Blanche ?}
  % 
  \Req \nex{8-1} Oui, je suis au-delà du \emph{Pas de la Ville.}
\end{dialogue}
%



Viennent ensuite les extraits \nex{9-1} et \nex{10-1} :
%
\begin{dialogue}
  %
  \Req \nex{9-1} Sur une zone à peu près plate et
  caillouteuse. \nex{10-1} Sur une petite prairie.
\end{dialogue}
%
Ces deux extraits décrivent la nature du terrain où se situe
actuellement le requérant. Dans ce cas la relation de localisation la
plus adaptée est \onto[orl]{Dans\-Planimetrique}, qui est conçue pour
décrire des situations d'inclusions.


La dernière partie de l'alerte est composée des extraits \nex{11-1} et
\nex{12-1}, portant sur l'éloignement de la position du requérant au
\emph{Pas de la ville :}
%
\begin{dialogue}
  \Sec \nex{11-1} Vous êtes à combien du Pas de la ville ?
  % 
  \Req \nex{12-1} \textins{À} 800 mètres, je crois, à vol d'oiseau.
\end{dialogue}
%
Dans cet extrait, le requérant précise clairement la nature de la
distance qu'il décrit, il s'agit d'une distance à vol d'oiseau,
visuellement approximée.

Ce type de configuration est représenté par la relation de
localisation
\onto[orl]{Distance\-Quanti\-ta\-ti\-ve\-Planimetrique}. Comme nous
l'indiquions dans le \autoref{chap:07}, cette relation de localisation
est atomique, elle n'admet donc pas de décomposition.

%
Au terme de cette phase de retranscription et d'identification des
indices de localisation on a donc un ensemble de XX indices de
localisation.
%
\begin{enumerate}
\item ss
\item XX
\item R, \onto[orl]{Distance\-Quantitative}(800m), \emph{Pas de la Ville}
\end{enumerate}


\subsection{Modélisation de l'alerte}
\label{subsec:9-2-2}




\subsubsection{Au nord du Pas de la Ville}

La relation de localisation \onto[orl]{Au\-Nord\-De\-Externe},
utilisée pour retranscrire la sémantique de l'extrait \nex{6-1} est
une relation de localisation décomposable.
%
Cette relation se décompose en deux relation de localisation
atomiques, \onto[orla]{Au\-Nord\-De} et
\onto[orla]{Hors\-De\-Planimétrique}. La relation de localisation
atomique \onto[orla]{Au\-Nord\-De} \footnote{Que la relation de
  localisation \protect\onto[orla]{Au\-Nord\-De\-Externe} partage avec
  sa relation sœur \protect\onto[orla]{Dans\-La\-Partie\-Nord\-De}.}


\begin{figure}
  \centering
  \input{./figures/EcartNord_PasVille.tex}
  \caption{Métrique \protect\onto[orla]{Ecart\-Angulaire}, calculée
    pour la spatialisation de la relation de localisation
    \protect\onto[orl]{AuNordDe}. La résolution du raster a
    été réduite de 5 à 50 mètres pour la représentation.}
  \label{fig:veyont_EcartNord}
\end{figure}


% \begin{figure}
%   \centering
%   \begin{tikzpicture}[scale=.7]
  \def\decalageX{-.2}
  \def\decalageY{-.2}
  % Courbe
  \begin{scope}[transparency group]
    % fond
    \begin{scope}
      \path[ffa_fade_m] (0,.8) -- (1,.8) -- (1,0) -- (0,0) -- cycle ;
      \path[ffa_fade] (8,.8) -- (9,.8) -- (9,0) -- (8,0) -- cycle;
      \path[ffa]  (1,.8) -- (3.6,.8) -- (4.5, 2) -- (5.4,.8) --(8,.8)
      --(8,0) --(1,0) -- cycle;
    \end{scope}
    % bords
    \begin{scope}
      \path[ffc, dotted] (3.6,.8) -- (3,0);
      \path[ffc, dotted] (5.4,.8) -- (6,0);
      % 
      \path[ffc] (1,.8) -- (3.6,.8) -- (4.5, 2) -- (5.4,.8) -- (8,.8) ;
      \path[ffc_fade_m] (0,.8) -- (1,.8) ;
      \path[ffc_fade] (8,.8) -- (9,.8) ;
    \end{scope}
  \end{scope}
  % Axes X, Y
  \begin{scope}
    % Axe X
    \begin{scope}
      % Axe
      \draw[<->] (0, \decalageX) --++ (9, 0) coordinate (x axis);
      % Graduations
      \foreach \n/\t in {0.5/{},1.5/{},2.5/{400},3.5/{},4.5/{800},5.5/{},6.5/{1200},7.5/{},8.5/{}}
      {
        \draw[-] (\n, \decalageX - .05) --++ (0, .1);
        \node[below, font=\footnotesize] at (\n, \decalageX - .05) {\t};
      }
      % label
      \node[below, yshift=-.1cm, font=\small] at (x axis)
      {\itshape Distance \normalfont (m)};
    \end{scope}
    % Axe Y
    \begin{scope}
      % Axe
      \draw[-] (\decalageY ,0) --++ (0, 2) coordinate (y axis);
      % Graduations
      \foreach \n/\t in {0/{0},2/{1}}
      {
        \draw[-] (\decalageY -.05, \n) --++ (.1, 0);
        \node[left, font=\footnotesize] at (\decalageY -.05, \n) {\t};
      }
      % Label
      \node[above] at (y axis) {$\mu$};
    \end{scope}
  \end{scope}
  \begin{scope}
    % Seuil 1
    \draw[fill, RdBu-9-1] (3,\decalageY) circle (2pt);
    % Seuil 2
    \draw[ffc,line width=.5] (4.5,\decalageY) -- (4.5,2);
    \draw[fill, RdBu-9-1] (4.5,\decalageY) circle (2pt);
    \draw[fill, RdBu-9-1] (4.5,2) circle (2pt);
    % Seuil 3
    \draw[fill, RdBu-9-1] (6,\decalageY) circle (2pt);
  \end{scope}
\end{tikzpicture}

%   \caption{XXXX \enquote{Grand Veymont}}
%   \label{fig:fuzzy_veyont_distance}
% \end{figure}



\subsubsection{Au-delà du Pas de la Ville}

La spatialisation de l'indice de localisation \enquote{je suis au-delà
  du Pas de la Ville}, pose certaines difficultés.

La relation de localisation de cet indice est représentée par le
concept \onto[orl]{Après\-Jalon\-Sut\-Itineraire}
(\autoref{anx:orl_dic})

\subsubsection{À 800 mètres du Pas de la Ville}

Ce dernier indice de localisation ne présente pas de difficultés
spécifiques pour être mis en place.


Cette relation est spatialisée à l'aide du \emph{rasteriser}
\onto[orla]{Geometrie}, de la \emph{métrique}
\onto[orla]{Dis\-tan\-ce} et du \emph{fuzzyficateur}
\onto[orla]{Eq\-Val}.

Dans le cas présent, \emph{l'objet de référence} mentionné est le
\emph{Pas de la Ville,} représenté par un ponctuel dans la composante
oronymie de la BDTOPO.


La rasterisation de ce



La \autoref{fig:veyont_distance} représente le résultat du calcul de
la métrique \onto[orla]{Distance} ---~calculée à partir du ponctuel
(rasterisé) représentant le \emph{Pas de la Ville}~--- pour l'ensemble
des positions de la \ac{zir}. Si cette métrique ne présente pas de
caractéristiques particulièrement surprenantes, on peut quand même
noter l'approximation qui est ici faite. Le \emph{Pas de la Ville} est
en effet résumé par un point, alors que le toponyme désigne un
passage, une zone de transition, qui serait sans doute mieux
représentée par un polygone, voire une polyligne. De plus, le point
utilisé n'est pas placé au niveau du point central du Pas de la Ville,
mais à plusieurs centaines de mètres de ce dernier.


\begin{figure}
  \centering
  \begin{tikzpicture}
  \tikzset{et/.style={above,font=\footnotesize\vphantom{Ag}}}
  % 
  \node[inner sep=0pt, anchor=south west] (image) at (0,0){\includegraphics{./figures/Distance_PasVille.png}};
  % 
  \begin{scope}
    \node (P2) at ([yshift=-.5cm]image.south east) {};
    \node (P1) at ([yshift=-.5cm]image.south west) {};
    % 
    \node (rect) [anchor=north west, minimum width=1cm,minimum
    height=.25cm] at ([yshift=-.25cm]P1) {}; \path[draw=RdBu-9-1, line
    width=1mm](rect.west) --([xshift=-1ex]rect.south) -- ([xshift=1ex]rect.north)
    -- (rect.east);
    % 
    \node[anchor=west, font=\tiny\vphantom{Ag}, text width = 4cm] at
    ([xshift=1ex]rect.east) {Limite de la \ac{zir}};
    %
    \node[anchor=west, font=\tiny\vphantom{Ag}, text width = 4cm] at
    (P1 |- 0cm,-1.5cm) {Distance au \emph{Pas de la Ville :}};
    %
    \foreach \x [evaluate=\xshift using \x/10, evaluate=\rad using (\x * .0004) + .01] in {0,...,100}
    {
      \draw[fill=black,draw=none, below] ([xshift=\xshift cm, yshift=-1.75cm]P1) circle [radius=\rad cm];
    }
    % 
    \path(P1 |- 0cm,-2.25cm) --++ (10,0)
    node[et,pos=0] {0}
    node[et,pos=.5] {\SI{2500}{\meter}}
    node[et,pos=1] {\SI{5000}{\meter}};

    % Échelle
    \draw[-] (P2 |- -1cm,-1cm) --++ (-1,0) node[et,pos=.5] {\SI{500}{\meter}};
    % Légende détaillée
    \path (P1) -- (P2) node[pos=.5, yshift=-2.5cm] {\tiny Pour la légende détaillée du fond topographique voir \autoref{anx:topo_leg}. Sources: BD TOPO 2018, BD ALTI 2018.}; 
  \end{scope}
\end{tikzpicture}
  \caption{Métrique \protect\onto[orla]{Distance}, calculée pour la
    spatialisation de la relation de localisation atomique
    \protect\onto[orla]{Distance\-Pla\-ni\-mé\-trique}. La résolution
    du raster a été réduite de 5 à 50 mètres pour la représentation.}
  \label{fig:veyont_distance}
\end{figure}

Pour construire la \ac{zlc} spatialisant cet indice de localisation il
est ensuite nécessaire de \emph{fuzzyfier} la métrique
(\autoref{fig:veyont_distance}) à l'aide du fuzzyficateur
\onto[orla]{Eq\-Val}.

\begin{figure}
  \centering
  \begin{tikzpicture}[scale=.7]
  \def\decalageX{-.2}
  \def\decalageY{-.2}
  % Courbe
  \begin{scope}[transparency group]
    % fond
    \begin{scope}
      \path[ffa_fade_m] (0,.8) -- (1,.8) -- (1,0) -- (0,0) -- cycle ;
      \path[ffa_fade] (8,.8) -- (9,.8) -- (9,0) -- (8,0) -- cycle;
      \path[ffa]  (1,.8) -- (3.6,.8) -- (4.5, 2) -- (5.4,.8) --(8,.8)
      --(8,0) --(1,0) -- cycle;
    \end{scope}
    % bords
    \begin{scope}
      \path[ffc, dotted] (3.6,.8) -- (3,0);
      \path[ffc, dotted] (5.4,.8) -- (6,0);
      % 
      \path[ffc] (1,.8) -- (3.6,.8) -- (4.5, 2) -- (5.4,.8) -- (8,.8) ;
      \path[ffc_fade_m] (0,.8) -- (1,.8) ;
      \path[ffc_fade] (8,.8) -- (9,.8) ;
    \end{scope}
  \end{scope}
  % Axes X, Y
  \begin{scope}
    % Axe X
    \begin{scope}
      % Axe
      \draw[<->] (0, \decalageX) --++ (9, 0) coordinate (x axis);
      % Graduations
      \foreach \n/\t in {0.5/{},1.5/{},2.5/{400},3.5/{},4.5/{800},5.5/{},6.5/{1200},7.5/{},8.5/{}}
      {
        \draw[-] (\n, \decalageX - .05) --++ (0, .1);
        \node[below, font=\footnotesize] at (\n, \decalageX - .05) {\t};
      }
      % label
      \node[below, yshift=-.1cm, font=\small] at (x axis)
      {\itshape Distance \normalfont (m)};
    \end{scope}
    % Axe Y
    \begin{scope}
      % Axe
      \draw[-] (\decalageY ,0) --++ (0, 2) coordinate (y axis);
      % Graduations
      \foreach \n/\t in {0/{0},2/{1}}
      {
        \draw[-] (\decalageY -.05, \n) --++ (.1, 0);
        \node[left, font=\footnotesize] at (\decalageY -.05, \n) {\t};
      }
      % Label
      \node[above] at (y axis) {$\mu$};
    \end{scope}
  \end{scope}
  \begin{scope}
    % Seuil 1
    \draw[fill, RdBu-9-1] (3,\decalageY) circle (2pt);
    % Seuil 2
    \draw[ffc,line width=.5] (4.5,\decalageY) -- (4.5,2);
    \draw[fill, RdBu-9-1] (4.5,\decalageY) circle (2pt);
    \draw[fill, RdBu-9-1] (4.5,2) circle (2pt);
    % Seuil 3
    \draw[fill, RdBu-9-1] (6,\decalageY) circle (2pt);
  \end{scope}
\end{tikzpicture}

  \caption{XXXX \enquote{Grand Veymont}}
  \label{fig:fuzzy_veyont_distance}
\end{figure}



\begin{figure}
  \centering
  \input{./figures/Distance_GrandVeymont.tex}
  \caption{XXXX \enquote{Grand Veymont}}
  \label{fig:Distance_GrandVeymont}
\end{figure}




\subsubsection{Fusion des zones de localisation compatibles}

À la suite de la spatialisation des différents indices de localisation
et de la composition des 

\subsection{Critique de la modélisation}
\label{subsec:9-2-3}

\tdi{Impact de la position du toponyme "pas de la ville" sur la
  modélisation}

Un autre problème notable est l'impact que peut avoir la géométrie des
objets de référence sur le résultat de la spatialisation. Par exemple,
l'indice de localisation : \enquote{À \SI{800}{\meter} du Pas de la
  Ville} a été spatialisé en calculant la métrique
\onto[orla]{Distance} à partir du ponctuel plaçant l'oronyme
\enquote{Pas de la Ville} dans la base de données. Or, la
représentation d'un géotype étendu, comme un pas ou une crête, par un
point est une approximation conséquente, impactant fortement la
spatialisation.


%%% Local Variables:
%%% mode: latex
%%% TeX-master: "../../../../main"
%%% End:
