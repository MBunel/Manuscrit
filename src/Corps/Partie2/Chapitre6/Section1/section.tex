La mise en place de la méthodologie que nous avons définie nécessite
la sélection d'une modélisation qui permette d'en traiter tous les
aspects.

Nous avons identifié deux catégories principales de critères de
sélection, ceux concernant la phase de spatialisation et ceux
concernant la phase de fusion.

\subsection{Critères portant sur la \emph{spatialisation}}

Comme nous l'avons vu précédemment, de nombreuses et diverses
\emph{relations de localisation} sont utilisées par les requérants
pour décrire leur position. Chacune de ces relations à une sémantique
qui lui est propre et qui conduit à la définition de \emph{zones de
  localisation} pouvant être très différentes. La modélisation
utilisée doit donc permettre la représentation de tous les différents
cas possibles.

\tdi{On veut un modèle qui permet de prendre en compte des relations
  de localisation traduisant des limites assez nettes (en dessous de)
  et des relations avec des limites très continues (proches [voir
  concept similaire dans l'ontologie]).}

Un premier point à considérer est celui de \emph{l'imprécision} des
\emph{zones de localisations.} Si nous avons détaillé cette notion et
justifié sa prise en compte lors de la \emph{spatialisation,} nous
n'avons pas présenté ses possibles variations. Or, de la même manière
que la forme d'une \emph{zone de localisation} varie d'une
\emph{relation de localisation} à une autre, leur \emph{imprécision}
peut être plus ou moins importante. Par exemple, la \emph{relation de
  localisation} \enquote{proche} \footnote{Formalisée dans l'ontologie
  des r\emph{elations de localisation} par le concept
  \onto[orl]{Prés\-De}.}, qui décrit une relation de proximité
géographique entre \emph{le sujet} et \emph{l'objet de référence,} est
fortement \emph{imprécise} \footnote{Comme nous l'illustrions avec le
  \emph{paradoxe sorites} dans le second chapitre (voir la section
  \ref{subsec:2-1-2}).}. Ainsi, les limites de la \emph{zone de
  localisation} y correspondant sont continues, sans réelle rupture
entre son intérieur et son extérieur. À l'inverse, la \emph{zone de
  localisation} construite à partir de la \emph{relation de
  localisation} \enquote{en dessous} \footnote{Formalisée avec le
  concept \onto[orl]{Sous\-Al\-ti\-tu\-de}.} a des limites plus
\emph{précises.} On peut en effet imaginer une situation où l'altitude
du \emph{sujet,} alors clairement situé \enquote{en dessous} de
\emph{l'objet de référence}, augmente peu à peu, jusqu'à dépasser
celle de \emph{l'objet de référence,} passant alors de la \emph{zone
  de localisation} à son extérieur. Cependant, contrairement à la
\emph{zone de localisation} construite à partir de la \emph{relation
  de localisation} \enquote{proche}, cette transition est, sinon
immédiate\footnote{Ce qui impliquerait que la \emph{relation de
    localisation} est \emph{précise,} ce qui, comme nous l'expliquions
  dans les chapitres \ref{chap:02} et \ref{chap:03}, n'est pas
  possible.}, rapide. Ainsi, s'il est abusif de considérer que
l'appartenance à cette \emph{zone de localisation} est bivalente, on
peut remarquer que la longue transition entre l'intérieur et
l'extérieur de la \emph{zone de localisation} construite d'après la
\emph{relation de localisation} \onto[orl]{Prés\-De} est ici fortement
réduite. Entre ces deux cas archétypaux on trouve une multitude
d'autres situations, où les limites de la \emph{zone de localisation}
sont plus ou moins marquées. Il est nécessaire que notre modélisation
des \emph{zones de localisation} propose une prise en compte
satisfaisante de ces différentes situations et soit donc à même de
modéliser des \emph{zones de localisation} aux limites presque
\emph{nettes} ou fortement \emph{imprécises.}

\tdi{Possibilité de construire des zones de localisation issues de
  relations de localisation non linéaire (en dessous de).}

De manière analogue, la progression de \emph{l'imprécision} peu être
assez différente en fonction de la \emph{relation de localisation}
modélisée. Une \emph{relation} comme \onto[orl]{Près\-De} voit, toutes
choses égales par ailleurs, son \textsl{degré de validité} diminuer
linéairement avec l'éloignement à \emph{l'objet de référence.} Alors
qu'une relation comme \onto[orl]{Au\-Des\-sus\-Al\-ti\-tu\-de}, dont
la validité dépend du relief, pourra évoluer de manière différente et
non linéaire.
%
La méthode développée devra donc arriver à modéliser aussi bien ces
deux cas.


\begin{figure}
  \centering
  \begin{tikzpicture}

  \def\decalageX{-.2}
  \def\decalageY{-.2}

  \begin{scope}[scale=1, xshift=6, yshift=1cm]
    % Courbe
    \begin{scope}[transparency group]
      % fond
      \begin{scope}
        \path[ffa] (0,0)--++ (0,2) -- (4.5, 0) -- cycle;
      \end{scope}
      % bords
      \begin{scope}
        \path[ffc] (0, 2) -- (4.5, 0);
        \path[ffc_fade] (4.5,0) -- (5,0);
      \end{scope}
    \end{scope}
    % Axes X, Y
    \begin{scope}
      % Axe X
      \begin{scope}
        % Axe
        \draw[->] (0, \decalageX) --++ (5, 0) coordinate (x axis);
        % Graduations
        \foreach \n/\t in
        {0/{}}
        { \draw[-] (\n, \decalageX - .05) --++ (0, .1); \node[below,
          font=\footnotesize] at (\n, \decalageX - .05) {\t}; }
        % label
        \node[below left] at (x axis) {$Truc$};
      \end{scope}
      % Axe Y
      \begin{scope}
        % Axe
        \draw[-] (\decalageY ,0) --++ (0, 2) coordinate (y axis);
        % Graduations
        \foreach \n/\t in {0/{0},2/{1}}
        {
          \draw[-] (\decalageY -.05, \n) --++ (.1, 0);
          \node[left, font=\footnotesize] at (\decalageY -.05, \n) {\t};
        }
        % Label
        \node[above] at (y axis) {$\mu$};
      \end{scope}
    \end{scope}
  \end{scope}

  % Axes
  \draw[<->] (0,0) -- (12,0) node[pos=0] {Peu précis} node[pos=1]
  {Très imprécis};

  \draw[<->] (6,2.5) -- (6,-2.5) node[pos=0] {Peu précis} node[pos=1]
  {Très imprécis};
  
\end{tikzpicture}
  \caption{dsqdsq}
  \label{fig:temp}
\end{figure}


\subsection{Critères portant sur la \emph{fusion}}


\tdi{Possibilité de fusionner (intersection et union). Nécessité que
l'union et l'intersection soient commutatives.}

La méthodologie de construction des \emph{zones de localisation}
présentée dans le \autoref{chap:04} et plus précisément la \emph{phase
  de fusion,} impose d'effectuer de combiner des \emph{zones de
  localisation} par des unions et des intersections. La modélisation
retenue doit donc permettre de réaliser ces opérations, mais également
garantir leur commutativité. En effet, l'utilisation d'un opérateur
d'union ou d'intersection non commutatif aurait pour conséquence de
modifier le résultat de la \emph{phase de fusion} en fonction de
l'ordre de traitement des \emph{zones de localisation
  compatibles}. Par exemple, si l'on souhaite construire la \emph{zone
  de localisation probable} ($\text{\textsl{zlp}}$) correspondant à
l'ensemble d'indices de localisation : \enquote{Je suis à côté d'une
  maison et dans une forêt}. Pour ce faire il est nécessaire
d'identifier les positions situées dans les \emph{zone de localisation
  compatibles} correspondant au premier ($\text{\textsf{zlc}}_{i1}$)
et au second ($\text{\textsf{zlc}}_{i2}$) indice, à l'aide d'un
opérateur d'intersection. Si cet opérateur n'est pas commutatif, alors
l'intersection de $\text{\textsf{zlc}}_{i1}$ avec
$\text{\textsf{zlc}}_{i2}$ pourra donner un résultat différent de
l'intersection de $\text{\textsf{zlc}}_{i2}$ avec
$\text{\textsf{zlc}}_{i1}$. Un tel comportement est incompatible avec
plusieurs des principes que nous avons définis, notamment l'autonomie
de la spatialisation ou le \emph{principe de décomposition,} qui se
fondent sur l'hypothèse ---~implicite~--- d'une commutativité des
opérateurs utilisés durant la phase de fusion.

\tdi{Nécessité que les résultats soient facilement
  interprétables. Sémantique des résultats et pas trop de zones (par
  exemple beaucoup d'alpha cuts).}

Nous avons déjà présenté l'importance que nous accordons à
l'interprétabilité des résultats (\emph{principe d'intégration dans le
  contexte métier}).
%
Par conséquent nous souhaitons que la méthodologie retenue produise
des résultats qui soient interprétables.

\tdi{Stabilité du modèle ? (par exemple si l'on fait des intersection
  avec le modèle egg-yolk on doit choisir ce qu'est le jaune. C'est
  l'intersection des jaunes ? ou il suffit qu'il y a ait un jaune?
  est-ce qu'il faut introduire des valeurs intermédiaires (gris,
  lorsqu'on intersecte jaune et blanc) -> mais dans ce cas on sort du
  egg yolk, le modèle n'est pas stable alors). Le résultat doit
  pouvoir s'exprimer avec la même modélisation sans perte de
  sémantique et de finesse de modélisation.}

Ce point est corroboré par le fait que la construction des \emph{zones
  de localisation} passe par une phase de fusion.
%
Il faut que le résultat des ces fusions soit interprétable est puisse
être exprimé avec la modélisation choisie.

\missingfigure{Figure explicitant le point 5}

%%% Local Variables:
%%% mode: latex
%%% TeX-master: "../../../../main"
%%% End:
